%% LyX 2.3.2 created this file.  For more info, see http://www.lyx.org/.
%% Do not edit unless you really know what you are doing.
\documentclass[oneside,english]{amsbook}
\usepackage[T1]{fontenc}
\usepackage[latin9]{inputenc}
\usepackage{geometry}
\geometry{verbose,tmargin=3cm,bmargin=3cm,lmargin=3cm,rmargin=3cm}
\setcounter{tocdepth}{1}
\usepackage{color}
\usepackage{babel}
\usepackage{refstyle}
\usepackage{float}
\usepackage{mathrsfs}
\usepackage{amstext}
\usepackage{amsthm}
\usepackage{amssymb}
\usepackage[unicode=true,
 bookmarks=true,bookmarksnumbered=false,bookmarksopen=false,
 breaklinks=false,pdfborder={0 0 0},pdfborderstyle={},backref=false,colorlinks=true]
 {hyperref}
\hypersetup{pdftitle={Analysis & Probability},
 pdfauthor={Yashaswi Mohanty},
 pdfsubject={Mathematics}}

\makeatletter

%%%%%%%%%%%%%%%%%%%%%%%%%%%%%% LyX specific LaTeX commands.

\AtBeginDocument{\providecommand\propref[1]{\ref{prop:#1}}}
\AtBeginDocument{\providecommand\thmref[1]{\ref{thm:#1}}}
\AtBeginDocument{\providecommand\defref[1]{\ref{def:#1}}}
%% Because html converters don't know tabularnewline
\providecommand{\tabularnewline}{\\}
\RS@ifundefined{subsecref}
  {\newref{subsec}{name = \RSsectxt}}
  {}
\RS@ifundefined{thmref}
  {\def\RSthmtxt{theorem~}\newref{thm}{name = \RSthmtxt}}
  {}
\RS@ifundefined{lemref}
  {\def\RSlemtxt{lemma~}\newref{lem}{name = \RSlemtxt}}
  {}


%%%%%%%%%%%%%%%%%%%%%%%%%%%%%% Textclass specific LaTeX commands.
\numberwithin{section}{chapter}
\theoremstyle{plain}
\newtheorem{thm}{\protect\theoremname}[section]
\theoremstyle{definition}
\newtheorem{example}[thm]{\protect\examplename}
\theoremstyle{plain}
\newtheorem{prop}[thm]{\protect\propositionname}
\theoremstyle{definition}
\newtheorem{defn}[thm]{\protect\definitionname}
\theoremstyle{remark}
\newtheorem*{rem*}{\protect\remarkname}
\theoremstyle{plain}
\newtheorem{cor}[thm]{\protect\corollaryname}
\theoremstyle{plain}
\newtheorem{lem}[thm]{\protect\lemmaname}
\theoremstyle{remark}
\newtheorem{rem}[thm]{\protect\remarkname}
\theoremstyle{plain}
\newtheorem{fact}[thm]{\protect\factname}
\theoremstyle{definition}
\newtheorem{xca}[thm]{\protect\exercisename}
\theoremstyle{definition}
\newtheorem*{sol*}{\protect\solutionname}

%%%%%%%%%%%%%%%%%%%%%%%%%%%%%% User specified LaTeX commands.
\usepackage{dsfont}
\usepackage{enumitem,lipsum}
\usepackage{soul}

% Set chapter paths
\makeatletter
\def\input@path{{chapters/}}
\makeatother


\DeclareMathOperator{\Ran}{\mathbf{Ran}}

\makeatother

\providecommand{\corollaryname}{Corollary}
\providecommand{\definitionname}{Definition}
\providecommand{\examplename}{Example}
\providecommand{\exercisename}{Exercise}
\providecommand{\factname}{Fact}
\providecommand{\lemmaname}{Lemma}
\providecommand{\propositionname}{Proposition}
\providecommand{\remarkname}{Remark}
\providecommand{\solutionname}{Solution}
\providecommand{\theoremname}{Theorem}

\begin{document}
\global\long\def\ring{\mathrm{ring}}%

\global\long\def\borel{\mathscr{B}}%

\global\long\def\mesh{\mathrm{mesh}}%

\global\long\def\spans{\mathrm{span}}%

\global\long\def\R{\mathds{R}}%

\global\long\def\N{\mathds{N}}%

\global\long\def\X{\mathcal{X}}%

\global\long\def\F{\mathcal{F}}%

\global\long\def\measurablespace{\left(\X,\F\right)}%

\global\long\def\measurespace{\left(\X,\F,\mu\right)}%

\global\long\def\measurableFunctions{\mathcal{M}\left(\X,\F\right)}%

\global\long\def\nonnegMeasurableFunctions{\mathcal{M}^{+}\left(\X,\F\right)}%

\global\long\def\lebInt#1#2{\bar{#1}\left(#2\right)}%

\global\long\def\indicate{\mathds{1}}%

\global\long\def\innerproduct#1#2{\langle#1,#2\rangle}%

\global\long\def\Lp#1#2{\mathcal{L}^{#1}\left(#2\right)}%

\global\long\def\pnorm#1#2{\lVert#1\rVert_{#2}}%

\frontmatter
\title{Basic Mathematics for Statistics and Economics}
\author{Yashaswi Mohanty}

\maketitle
\tableofcontents{}

\chapter*{Preface}

This document consists of notes on basic mathematical material that
is expected as background knowledge for graduate students working
in mathematical statistics, econometrics, or economic theory. The
treatment includes an overview of rudimentary measure and integration
theory, basic functional analysis, and introductory probability theory,
along with a smattering of topics from optimization theory, statistical
inference, game theory, and microeconomic theory. Anyone with a facility
with the analysis on Euclidean spaces and linear algebra can use these
notes. This background is contained in the various appendices. Since
this document hasn't been proofread by others, it almost surely contains
typos and other mathematical (and non-mathematical) errors. Please
raise a GitHub issue if you find such errors.

\mainmatter

\part{Analysis}


\chapter{Measures}

\section{Why is measurement hard?}

On the real line $\mathds{R}$, we may want our measure to satisfy
some properties that are consistent with our intuitive notion of ``length''.
Formally, we want a function
\[
\lambda:2^{\mathds{R}}\longrightarrow\left[0,\infty\right]
\]
that satisfies
\begin{enumerate}
\item $\lambda\left(\emptyset\right)=0$
\item $\lambda\left(\left[a,b\right]\right)=b-a$ for $a\leq b\in\mathds{R}$
\item Countable additivity: For a countable collection of pairwise-disjoint
sets $\left\{ A_{i}\right\} _{i\in\mathbb{N}}\subseteq\mathds{R}$
\begin{equation}
\lambda\left(\bigcup_{i\in\mathbb{N}}A_{i}\right)=\sum_{i\in\mathbb{N}}\lambda\left(A_{i}\right)\label{eq:countableAdditivity}
\end{equation}
\item Translation invariance: $\lambda\left(A+a\right)=\lambda\left(A\right)$
for any $a\in\mathds{R}$ where $A+a:=\left\{ \alpha+a\mid\alpha\in A\right\} $.
\end{enumerate}
Quite counterintuitively, it turns out that no such function exists!
To prove this assertion, we need to construct some special kinds of
sets that only exist if we assume the Axiom of Choice.
\begin{example}
\label{exa:vitaliSet} Define an equivalence relation $\sim$ on $\left[0,1\right]$
such that
\[
x\sim y\Leftrightarrow x-y\in\mathbb{Q}.
\]
Note that there are uncountably many classes in such a construction
as the equivalence class for any given irrational number can contain
at most countably many other irrational numbers. For example,
\[
\left[\frac{\pi}{4}\right]=\left\{ \frac{\pi}{4}+q\mod1\mid q\in\mathbb{Q}\right\} .
\]
Thus, using the Axiom of Choice, we can construct a set $E\subseteq\left[0,1\right]$
such that $E$ consists of exactly one ``representative'' from each
equivalence class. Next, we can define
\[
E_{q}:=\left\{ x+q\mod1\mid x\in E\right\} 
\]
so that $\left\{ E_{q}\right\} _{q\in\mathbb{Q}}$ is a partition
of $\left[0,1\right]$. To see that the sets are disjoint, suppose
for contradiction that for any distinct $q,\tilde{q}\in\mathbb{Q}\cap\left[0,1\right]$,
$E_{q}\cap E_{\tilde{q}}\neq\emptyset$. If $x\in E_{q}\cap E_{\tilde{q}}$,
then $x-q\in E$ and $x-\tilde{q}\in E$. But they clearly belong
to the same equivalence class and this is a contradiction given our
construction of $E.$ To see that the union of these sets is $\left[0,1\right]$,
consider an arbitrary $y\in\left[0,1\right]$ and observe that since
our equivalence relation $\sim$ partitions $\left[0,1\right]$, $y\in\left[x\right]$
for some $x\in E$. Then $q^{*}=y-x\in\mathbb{Q}$ and so $y=x+q^{*}\mod1\in E_{q^{*}}$.
Thus we have that
\[
\left[0,1\right]\subseteq\bigcup_{q\in\mathbb{Q}}E_{q}.
\]
Since the reverse inclusion follows by the definition of $E_{q}$,
we have that $\left\{ E_{q}\right\} _{q\in\mathbb{Q}}$ is a partition
of $\left[0,1\right]$.
\end{example}

\begin{prop}
\label{prop:vitalitSetNotMeasurable}There exists no function $\lambda:2^{\mathds{R}}\longrightarrow\left[0,\infty\right]$
that satisfies properties (1)-(4) described above
\end{prop}

\begin{proof}
Suppose, for contradiction, that such a function $\lambda$ exists.
We can define the collection of sets $\left\{ E\right\} _{q\in\mathbb{Q}}$
as in Example \ref{exa:vitaliSet} and observe that
\begin{align*}
1=\lambda\left(\left[0,1\right]\right) & =\lambda\left[\bigcup_{q\in\mathbb{Q}}E_{q}\right]\\
 & =\sum_{q\in\mathbb{Q}}\lambda\left[E_{q}\right]\\
 & =\sum_{q\in\mathbb{Q}}c
\end{align*}
where the first equality follows from property (2), the second equality
follows from the fact that $\left\{ E\right\} _{q\in\mathbb{Q}}$
is a partition of $\left[0,1\right]$, the third equality is due to
property (3). The last equality follows as a consequence of translation
invariance (property (4)). Since $c\in\left[0,1\right]$
\[
\sum_{q\in\mathbb{Q}}c=0\text{ or }\infty\neq1
\]
which is a contradiction. Thus no such function $\lambda$ exists.
\end{proof}
This particular example of a \emph{non-measurable }set is called a
\emph{Vitali set. }While we used the interval $\left[0,1\right]$
to construct such a set, it turns out that this contruction can be
extended to any set of positive length in the Lebesgue sense.

\section{Constructing measures on $\sigma-$algebras}

The key issue with our previous definition of a measure on $\mathds{R}$
is that one cannot have a set-valued function that both has our four
desired properties \emph{and }is defined on all subsets of the real
line. As a convention, the canonical construction of a measure retains
the desired properties in exchange for restricting the class of subsets
on which the measure is defined. These subsets are called $measurable$
and the standard construction of the Lebesgue measure leads to the
class of measurable subsets on the real line to have a special structure
of a \emph{$\sigma-$algebra. }Before we define this structure it
might be worthwhile looking at various types of structures a class
of sets could have

\subsection{Structures of sets}

In the rest of this chapter, we assume that $\left(\mathcal{X},\tau\right)$
is an abstract topological space.
\begin{defn}
\label{def:ring}Let $\mathcal{F}\subseteq2^{\mathcal{X}}$. We call
$\mathcal{F}$ a \emph{ring }if

\begin{enumerate}[label=(\roman*),leftmargin=.1\linewidth,rightmargin=.4\linewidth]
	\item $\emptyset \in \mathcal{F}$
	\item $A,B \in \mathcal{F} \Rightarrow A\cup B \in \mathcal{F}$
	\item $A,B \in \mathcal{F} \Rightarrow A\setminus B \in \mathcal{F}$.
\end{enumerate}
\end{defn}

\noindent Note that the above definition implies that $A\cap B=A\setminus\left(A\setminus B\right)\in\mathcal{F}$.
\begin{defn}
\label{def:algebra}Let $\mathcal{F}\subseteq2^{\mathcal{X}}$. We
call $\mathcal{F}$ an \emph{algebra }if

\begin{enumerate}[label=(\roman*),leftmargin=.1\linewidth,rightmargin=.4\linewidth]
	\item $\mathcal{F}$ is a ring
	\item $\mathcal{X} \in \mathcal{F}$.
\end{enumerate}
\end{defn}

\noindent For example, if we let $\mathcal{X}$ be an arbitrary infinite
set, the collection of all finite subsets of $\mathcal{X}$ forms
a ring but not an algebra.
\begin{defn}
\label{def:sigmaRing}Let $\mathcal{F}\subseteq2^{\mathcal{X}}$.
We call $\mathcal{F}$ a $\sigma$-ring if

\begin{enumerate}[label=(\roman*),leftmargin=.1\linewidth,rightmargin=.4\linewidth]
	\item $\mathcal{F}$ is a ring
	\item $\mathcal{F}$ is closed under countable unions.
\end{enumerate}
\end{defn}

\begin{defn}
\label{def:sigmaAlgebra}Let $\mathcal{F}\subseteq2^{\mathcal{X}}$.
We call $\mathcal{F}$ a $\sigma$-algebra if

\begin{enumerate}[label=(\roman*),leftmargin=.1\linewidth,rightmargin=.4\linewidth]
	\item $\mathcal{F}$ is an algebra.
	\item $\mathcal{F}$ is closed under countable unions.
\end{enumerate}
\end{defn}

\noindent Naturally, the power set $2^{\mathcal{X}}$ is a ring,
algebra, $\sigma$-ring, and $\sigma$-algebra all rolled into one.
\begin{rem*}
Algebras are sometimes referred to as \emph{fields} in the probability
literature.
\end{rem*}
As we said earlier, the notion of a $\sigma$-algebra is important
because the standard Lebesgue measurable sets form a $\sigma$-algebra
of subsets of $\mathds{R}$. However, the other structures we have
defined are also important; as we ``extend'' the notion of the length
of an interval on the real line to more complicated sets, we shall
first expand our class of measurable sets to a ring of sets.

\subsection{Lengths of intervals}

The mosts intuitive notion of a measure on $\mathds{R}$ arises from
the length of an interval. Thus, in our construction of the Lebesgue
measure, we start with the simplest class of sets which consists of
intervals in $\mathds{R}.$ Define $\mathcal{L}=\left\{ \left(a,b\right]\mid-\infty<a\leq b<\infty\right\} $
and let $\lambda_{1}:\mathcal{L}\longrightarrow\left[0,\infty\right]$
be given by $\lambda_{1}\left(\left(a,b\right]\right)=b-a$. It turns
out that our collection of half-open intervals in $\mathds{R}$ has
the structure of a \emph{semi-ring.}
\begin{defn}
\label{def:semiRing} Let $\mathcal{F}\subseteq2^{\mathcal{X}}$.
We call $\mathcal{F}$ a \emph{semi-ring }if

\begin{enumerate}[label=(\roman*),leftmargin=.1\linewidth,rightmargin=.4\linewidth]
	\item $\emptyset \in \mathcal{F}$.
	\item $A,B \in \mathcal{F} \Rightarrow A\cap B \in \mathcal{F}$.
	\item $A,B \in \mathcal{F} \Rightarrow \exists \left\{A_i\right\}_{i=1}^{n} \in \mathcal{F}$ such that $ A_i \cap A_j = \emptyset$ for $i \neq j$ and
	\[
		A \setminus B = \bigcup_{i=1}^{n}A_i 
	\]
\end{enumerate}
\end{defn}

\begin{prop}
\label{prop:intervalSigmaRing}$\mathcal{L}$ is a semi-ring.
\end{prop}

\begin{proof}
To see (i), note that $\emptyset=\left(a,a\right]\in\mathcal{L}$.
For (ii), note that for any intervals $A=\left(a_{1},b_{1}\right]$
and $B=\left(a_{2},b_{2}\right]$, \footnote{If $\max\left(a_{1},a_{2}\right)>\min\left(b_{1},b_{2}\right)$, then
$\left(\max\left(a_{1},a_{2}\right),\min\left(b_{1},b_{2}\right)\right]=\emptyset\in\mathcal{L}$}
\[
\left(a_{1},b_{1}\right]\cap\left(a_{2},b_{2}\right]=\left(\max\left(a_{1},a_{2}\right),\min\left(b_{1},b_{2}\right)\right]\in\mathcal{L}.
\]
To see (iii), we have to consider two possible cases. First, if $A,B$
are disjoint then $A\setminus B=A\in\mathcal{L}.$ If $A,B$ have
a non-trivial intersection, then
\begin{align*}
A\setminus B=A\cap B^{C} & =\left(a_{1},b_{1}\right]\cap\left\{ \left(-\infty,a_{2}\right]\cup\left(b_{2},\infty\right)\right\} \\
 & =\left(a_{1},b_{1}\right]\cap\left(-\infty,a_{2}\right]\bigcup\left(a_{1},b_{1}\right]\cap\left(b_{2},\infty\right)\\
 & =\left(a_{1},\min\left(b_{1},a_{2}\right)\right]\bigcup\left(\max\left(a_{1},b_{2}\right),b_{1}\right]
\end{align*}
where the components of the union expressed in the last equality are
in $\mathcal{L}$. This completes the proof.
\end{proof}
The fact that $\mathcal{L}$ is a semi-ring is important because there's
a relatively straightforward way to ``expand'' a semi-ring into
a ring.
\begin{thm}
\label{thm:expandSemiRing}Let $\mathcal{F}$ be a semi-ring and let
$\mathcal{B}$ be the set of all finite disjoint unions of sets in
$\mathcal{F}$. Then $\mathcal{B}$ is a ring.
\end{thm}

\begin{proof}
Property (i) in Definition \ref{def:ring} is trivially satisfied
thus we need to prove properties (ii) and (iii). Let $A,B\in\mathcal{B}$.
To prove property (iii), we first establish the weaker claim that
$A\cap B\in\mathcal{B}$. Observe that
\begin{align*}
A & =\bigcup_{i=1}^{n_{A}}A_{i},\ A_{i}\in\mathcal{F},A_{i}\cap A_{j}=\emptyset\mathrm{\ for\ }i\neq j,\\
B & =\bigcup_{i=1}^{n_{B}}B_{i},\ B_{i}\in\mathcal{F},B_{i}\cap B_{j}=\emptyset\mathrm{\ for\ }i\neq j,
\end{align*}
by the definition of $\mathcal{B}.$ Then
\begin{align*}
A\cap B & =\left(\bigcup_{i=1}^{n_{A}}A_{i}\right)\bigcap\left(\bigcup_{j=1}^{n_{B}}B_{j}\right)\\
 & =\bigcup_{i=1}^{n_{A}}\bigcup_{j=1}^{n_{B}}\left(A_{i}\cap B_{j}\right)
\end{align*}
where $\forall i,j:\ A_{i}\cap B_{j}\in\mathcal{F}$ as $\mathcal{F}$
is a semi-ring. Clearly, $A_{i}\cap B_{j}$ is disjoint from $A_{i^{\prime}}\cap B_{j^{\prime}}$,
thus proving the claim. Next, we establish property (iii) by noting
that
\begin{align*}
A\setminus B & =\left(\bigcup_{i=1}^{n_{A}}A_{i}\right)\setminus B\\
 & =\left(\bigcup_{i=1}^{n_{A}}A_{i}\right)\bigcap B^{C}\\
 & =\bigcup_{i=1}^{n_{A}}\left(A_{i}\cap B^{C}\right)\\
 & =\bigcup_{i=1}^{n_{A}}\left(A_{i}\cap\left(\bigcap_{j=1}^{n_{B}}B_{j}^{C}\right)\right)\\
 & =\bigcup_{i=1}^{n_{A}}\bigcap_{j=1}^{n_{B}}\left(A_{i}\cap B_{j}^{C}\right)\\
 & =\bigcup_{i=1}^{n_{A}}\bigcap_{j=1}^{n_{B}}A_{i}\setminus B_{j}
\end{align*}
where the $A_{i}\setminus B_{j}\in\mathcal{B}$ since $A_{i},B_{j}\in\mathcal{F}$.
By the closure under finite intersections property established earlier,
$E_{i}=\bigcap_{j=1}^{n_{B}}A_{i}\setminus B_{j}\in\mathcal{B}$ for
any $1\leq i\leq n_{A}$. Thus we can rewrite the chain of equalities
above as
\[
A\setminus B=\bigcup_{i=1}^{n_{A}}E_{i}
\]
where $E_{i}\cap E_{i^{\prime}}=\emptyset$ because $A_{i}\cap A_{i^{\prime}}=\emptyset.$
Since the finite disjoint union of elements of $\mathcal{B}$ is also
a finite disjoint union of elements of $\mathcal{F}$, our claim follows.
Finally, to establish property (ii), observe that
\[
A\cup B=\left(A\setminus B\right)\cup\left(A\cap B\right)\cup\left(B/A\right)
\]
which is a disjoint union of elements in $\mathcal{B}$ and so is
also in $\mathcal{B}$ by the same argument as earlier.
\end{proof}
\begin{cor}
Let $\mathcal{J}$ be the set of all finite disjoint unions of sets
in $\mathcal{L}$. Then $\mathcal{J}$ is a ring.
\end{cor}

\begin{proof}
By Proposition \propref{intervalSigmaRing}, $\mathcal{L}$ is a semi-ring.
The claim then follows by an application of Theorem \thmref{expandSemiRing}.
\end{proof}
Now we can extend our proto-measure $\lambda_{1}$ to a new proto-measure
$\lambda_{2}:\mathcal{J}\longrightarrow\left[0,\infty\right]$ as
follows:
\[
\lambda_{2}\left(A\right):=\begin{cases}
\lambda_{1}\left(A\right), & A\in\mathcal{L}\\
\sum_{i=1}^{n}\lambda_{1}\left(B_{i}\right), & A=\bigcup_{i=1}^{n}B_{i},\left\{ B_{i}\right\} _{i=1}^{n}\text{ are disjoint in }\mathcal{L}
\end{cases}
\]

\begin{prop}
\label{prop:ringMeasureFinitelyAdditive}$\lambda_{2}$ is fiinitely
additive on $\mathcal{J}$. That is, for any finite disjoint collection
of sets $\left\{ A_{i}\right\} _{i=1}^{n}\in\mathcal{J}$
\[
\lambda_{2}\left(\bigcup_{i=1}^{n}A_{i}\right)=\sum_{i=1}^{n}\lambda_{2}\left(A_{i}\right).
\]
\end{prop}

\begin{proof}
For clarity, we will prove finite additivity for two sets , since
the general case follows by induction. Let $A,B\in\mathcal{J}$ such
that $A\cap B=\emptyset$. By definition,
\begin{align*}
A & =\bigcup_{i=1}^{n_{A}}A_{i},\left\{ A_{i}\right\} _{i=1}^{n_{A}}\text{ are disjoint in }\mathcal{L}\\
B & =\bigcup_{i=1}^{n_{B}}B_{i},\left\{ B_{i}\right\} _{i=1}^{n_{B}}\text{ are disjoint in }\mathcal{L}
\end{align*}
and so we have that
\begin{align*}
\lambda_{2}\left(A\cup B\right) & =\lambda_{2}\left(\left(\bigcup_{i=1}^{n_{A}}A_{i}\right)\cup\left(\bigcup_{i=1}^{n_{B}}B_{i}\right)\right)\\
 & =\sum_{i=1}^{n_{A}}\lambda_{2}\left(A_{i}\right)+\sum_{i=1}^{n_{B}}\lambda_{2}\left(B_{i}\right)\\
 & =\lambda_{2}\left(A\right)+\lambda_{2}\left(B\right)
\end{align*}
where the second equality follows from associativity of addition along
with the fact that $\left(\bigcup_{i=1}^{n_{A}}A_{i}\right)\cup\left(\bigcup_{i=1}^{n_{B}}B_{i}\right)$
is a disjoint union of sets in $\mathcal{L}$.
\end{proof}

\subsection{Structures generated by a class of sets}

A key way to ``expand'' a particular class of sets into a larger
structure is to look at the structure \emph{generated }by the class
of sets. This idea can be formalized in the following definition,
which serves as particular example of this general concept of generation.
\begin{defn}
\label{def:ringGeneratedByClass}For any $\mathcal{A}\subseteq2^{\mathcal{X}},$
we refer to the intersection of all rings that contain $\mathcal{A}$
as the ring \emph{generated }by $\mathcal{A}$. Formally, we write
\[
\ring\left(\mathcal{A}\right)=\bigcap\left\{ \mathcal{R\subseteq}2^{\mathcal{X}}\text{ is a ring }\mid\mathcal{A\subseteq\mathcal{R}}\right\} .
\]
\end{defn}

\begin{prop}
\label{prop:ringGeneratedByClassIsRing}For any $\mathcal{A}\subseteq2^{\mathcal{X}},$
$\ring\left(\mathcal{A}\right)$ is a ring.
\end{prop}

\begin{proof}
First note that $\ring\left(\mathcal{A}\right)$ exists since $2^{\mathcal{X}}$
is a ring and so $\left\{ \mathcal{R\subseteq}2^{\mathcal{X}}\text{ is a ring }\mid\mathcal{A\subseteq\mathcal{R}}\right\} $
is non-empty. Next observe that $\emptyset\in\ring\left(\mathcal{A}\right)$
vacuously, so property (i) in Definition \defref{ring} is easily
satisfied. For property (ii), let $A,B\in\ring\left(\mathcal{A}\right)$
and observe that $A,B\in\mathcal{R}$ for every $\mathcal{R}\in\left\{ \mathcal{R\subseteq}2^{\mathcal{X}}\text{ is a ring }\mid\mathcal{A\subseteq\mathcal{R}}\right\} $.
Since $\mathcal{R}$ is a ring, $A\cup B\in\mathcal{R}$ for every
$\mathcal{R}$ and thus $A\cup B$ is in the intersection i.e. $\ring\left(\mathcal{A}\right)$.
A similar argument establishes property (iii) and thus we can conclude
that $\ring\left(\mathcal{A}\right)$ is a ring (as it should, given
its name).
\end{proof}
In our construction of the Lebesgue measure on $\mathds{R}$, we discovered
that $\mathcal{J}$, which is the set of all disjoint unions of half-open
intervals in $\mathds{R}$, is a ring. It turns out that we can make
a stronger statement using the language of generators developed here.
\begin{prop}
\label{prop:JisRingGeneratedByL}$\mathcal{J}=\ring\left(\mathcal{L}\right)$
\end{prop}

\begin{proof}
Let $A\in\mathcal{J}$ be arbitrary. Then we can write 
\[
A=\bigcup_{i=1}^{n_{A}}A_{i}
\]
where $A_{i}\in\mathcal{L}$ are pairwise disjoint. Let $\mathcal{R}$
be an arbitrary ring that contains $\mathcal{L}$ and observe that
since rings are closed under finite unions, $A\in\mathcal{R}.$ Since
$\mathcal{R}$ was arbitrary, $A$ is contained by every ring that
contains $\mathcal{L}$ and is thus contained in the intersection
of all such rings i.e. $\ring\left(\mathcal{L}\right).$ This proves
that $\mathcal{J}\subseteq\ring\left(\mathcal{L}\right).$

To see reverse inclusion, recall that $\mathcal{J}$ is a ring that
contains $\mathcal{L}$, and so the intersection of all rings that
contain $\mathcal{L}$ is certaintly contained in $\mathcal{J}$.
This completes the proof.
\end{proof}
In measure theory, the most important structure on sets is the $\sigma$-algebra,
and the $\sigma$-algebra generated by a class of sets $\mathcal{A}$,
defined analagously to Definition \ref{def:ringGeneratedByClass}
about rings and notated as $\sigma\left(\mathcal{A}\right)$, plays
in an important role in this theory. Using a similar argument as the
one shown earlier, one can conclude that $\sigma\left(\mathcal{A}\right)$
is indeed a $\sigma$-algebra. In analysis and probability theory,
mathematicians are interested in $\sigma$-algebras generated by a
special class of sets.
\begin{defn}
\label{def:borelSigma}The $\sigma$-algebra generated by the topology
$\tau$ on set $\mathcal{X}$ is called the \emph{Borel $\sigma$-algebra
}on $\mathcal{X}$ and is denoted $\mathscr{B}\left(\mathcal{X}\right)$.
\end{defn}

The Borel $\sigma$-algebra is interesting because it turns that it
is the $\sigma$-algebra generated by $\mathcal{L}$ is indeed $\mathscr{B}\left(\mathds{R}\right)$,
where $\mathds{R}$ has the usual topology. To prove this fact, we
need a little lemma from an introductory course on analysis and topology.
\begin{lem}
\label{lem:openSetDisjointUnionInterval} Any open set in the usual
topology of $\R$ can be written as a countable disjoint union of
open intervals in $\R$.
\end{lem}

\begin{proof}
Let $O$ be an open set in $\R$ and let $x\in O$ be arbitrary. Define
$I_{x}\subseteq O$ to be the largest open interval that contains
$x$ (that is, $I_{x}$ is the union of all open intervals in $O$
that contain $x$). Note that at least one such interval exists because
$O$ is open and so there exists some $\varepsilon>0$ such that $\left(x-\varepsilon,x+\varepsilon\right)\subseteq O.$
Now for any distinct $x,y\in O$, $I_{x}$ and $I_{y}$ are either
disjoint or equal since if they were neither, $I_{x}\cup I_{y}\subseteq O$
would be a larger interval that contains both $x$ and $y$. Let $\mathcal{I}$
denote the collection of all disjoint such intervals (that is, we
get $\mathcal{I}$ by discarding all the ``redundant'' intervals
in $\left\{ I_{x}\right\} _{x\in O}$). We can do this without invoking
the Axiom of Choice since there are only countably many intervals
in $\mathcal{I}$: every interval $I\in\mathcal{I}$ contains at least
one rational number because the rationals are a countably dense subset
of $\R$. Thus, since the intervals are disjoint, $\mathcal{I}$ can
have at most countably many intervals. Of course
\[
O=\bigcup_{I\in\mathcal{I}}I
\]
and so our claim follows.
\end{proof}
\begin{prop}
\label{prop:sigmaAlgebraGeneratedbyLisBorel}$\sigma\left(\mathcal{L}\right)=\borel\left(\R\right)$
\end{prop}

\begin{proof}
Let $O$ be an open set in $\R$. Then, by Lemma \ref{lem:openSetDisjointUnionInterval}
\begin{align*}
O & =\bigcup_{i=1}^{\infty}\left(a_{i},b_{i}\right)\\
 & =\bigcup_{i=1}^{\infty}\bigcup_{n=1}^{\infty}\left(a_{i},b_{i}-\frac{1}{n}\right]
\end{align*}
which is in $\sigma\left(\mathcal{L}\right)$ by closure under countable
unions (property (ii) in Definition \ref{def:sigmaAlgebra}). Therefore
the topology of $\R$ is in $\sigma\left(\mathcal{L}\right)$ which
implies that $\borel\left(\R\right)\subseteq\sigma\left(\mathcal{L}\right)$.
The

To see the reverse inclusion, observe that for any $\left(a,b\right]\in\mathcal{L}$,
we can write
\[
\left(a,b\right]=\left(a,b\right)\cup\left\{ b\right\} \in\borel\left(\R\right)
\]
since $\left\{ b\right\} $ is closed in $\R$ and closed sets are
the complements of open sets and thus contained in $\borel\left(\R\right)$.\footnote{$\sigma$-algebras on $\mathcal{X}$ are closed under complements
since they are closed under set-differences and contain $\mathcal{X}$.} Therefore $\mathcal{L\subseteq\borel\left(\R\right)}$ and so $\sigma\left(\mathcal{L}\right)\subseteq\borel\left(\R\right)$,
completing the proof.
\end{proof}
Now we are ready to prove that our proto-measure $\lambda_{2}$ is
actually a countably-additive pre-measure on $\ring\left(\mathcal{L}\right)$.
But first, we need a lemma about double sums!
\begin{lem}[Tonelli for series]
\label{lem:TonelliForSeries}Let $\left\{ x_{ij}\right\} _{i,j\in\N\times\N}$
be a sequence of non-negative (extended) real numbers. Then

\[
\sum_{i,j\in\N^{2}}x_{ij}=\sum_{i=1}^{\infty}\sum_{j=1}^{\infty}x_{ij}=\sum_{j=1}^{\infty}\sum_{i=1}^{\infty}x_{ij}.
\]
\end{lem}

\begin{proof}
We will prove the first equality since the second then follows by
symmetry. Let $F\subset\N^{2}$ be arbitrary and finite. Then, there
exists some $N\in\N$ such that $F\subseteq\left\{ 1,2\ldots,N\right\} ^{2}$
and so, by the non-negativity of $x_{ij}$
\[
\sum_{i,j\in F}x_{ij}\leq\sum_{i,j\in\left\{ 1,2\ldots,N\right\} ^{2}}x_{ij}=\sum_{i=1}^{N}\sum_{j=1}^{N}x_{ij}\leq\sum_{i=1}^{\infty}\sum_{j=1}^{\infty}x_{ij}.
\]
This inequality holds for any finite $F\subset\N^{2}$ and so it holds
for the supremum of all such finite sums. That is to say,

\[
\sup_{F\subset\N^{2}\mid F\text{ is finite}}\sum_{i,j\in F}x_{ij}\leq\sum_{i=1}^{\infty}\sum_{j=1}^{\infty}x_{ij}.
\]
But recall that for any $\left\{ a_{i}\right\} _{i\in\mathcal{I}}\in\left[0,\infty\right]$
where $\mathcal{I}$ is any index set
\[
\sum_{i\in\mathcal{I}}a_{i}:=\sup_{I\subset\mathcal{I}\mid I\text{ is finite}}\sum_{i\in I}a_{i},
\]
and so we have that
\[
\sum_{i,j\in\N^{2}}x_{ij}\leq\sum_{i=1}^{\infty}\sum_{j=1}^{\infty}x_{ij}.
\]

To derive the other inequality, observe that it is sufficient to prove
that
\[
\sum_{i,j\in\N^{2}}x_{ij}\geq\sum_{i=1}^{I}\sum_{j=1}^{\infty}x_{ij}
\]
for every $I\in\N$. Fix $I=I_{0}$ and note that
\[
\sum_{i=1}^{I_{0}}\sum_{j=1}^{\infty}x_{ij}=\sum_{i=1}^{I_{0}}\lim_{J\to\infty}\sum_{j=1}^{J}x_{ij}=\lim_{J\to\infty}\sum_{i=1}^{I_{0}}\sum_{j=1}^{J}x_{ij}.
\]
Thus to prove $\sum_{i,j\in\N^{2}}x_{ij}\geq\sum_{i=1}^{I_{0}}\sum_{j=1}^{\infty}x_{ij}$
we need to prove that 
\[
\sum_{i,j\in\N^{2}}x_{ij}\geq\sum_{i=1}^{I_{0}}\sum_{j=1}^{J}x_{ij}
\]
for every $J\in\N$. Fix $J=J_{0}$ and then observe that
\[
\sum_{i=1}^{I_{0}}\sum_{j=1}^{J_{0}}x_{ij}=\sum_{i,j\in\left\{ 1,2,\ldots,I_{0}\right\} \times\left\{ 1,2,\ldots,J_{0}\right\} }x_{ij}\leq\sum_{i,j\in\N^{2}}x_{ij}
\]
where the inequality follows due to non-negativity of $x_{ij}$. This
concludes the proof.
\end{proof}
\begin{rem*}
This lemma is a special case of Tonelli's theorem, a fundamental theorem
that allows us to construct measures on Cartesian products of measure
spaces from the measures on those spaces themselves. This theorem
will be motivated and proved in Chapter 5.
\end{rem*}
\begin{prop}
\label{prop:ringMeasureCountablyAdditive} $\lambda_{2}$ is a countably
additive pre-measure on $\ring\left(\mathcal{L}\right)$, that is
to say,

\begin{enumerate}[label=(\roman*),leftmargin=.1\linewidth,rightmargin=.4\linewidth]
	\item $\lambda_2\left(\emptyset\right) = 0$ 
	\item For disjoint $\left\{A_i\right\}_{i=1}^{\infty}\in \mathcal{J}$ such that $\bigcup_{i=1}^{\infty}A_i \in \mathcal{J}$
	\[
			\lambda_2\left(\bigcup_{i=1}^{\infty}A_i\right) = \sum_{i=1}^{\infty}\lambda_2\left(A_i\right).
	\]
\end{enumerate}
\end{prop}

\begin{proof}
Property (i) is inherited from $\lambda_{1}$. To see property (ii),
let $\left\{ A_{i}\right\} _{i=1}^{\infty}\in\mathcal{J}$ be disjoint
and write $A:=\bigcup_{i=1}^{\infty}A_{i}$ where $A\in\mathcal{J}$
by assumption. First, note that if $\lambda_{2}\left(A_{i}\right)=\infty$
for any $i\in\N$, then $\infty=\lambda_{2}\left(A_{i}\right)\leq\lambda_{2}\left(\bigcup_{i=1}^{\infty}A_{i}\right)=\infty$
where the inequality is due to the monotonicity\footnote{For any $A,B\in\ring\left(\mathcal{L}\right)\text{ such that }A\subseteq B,\lambda_{2}\left(B\right)=\lambda_{2}\left(A\right)+\lambda_{2}\left(B\setminus A\right)\geq\lambda_{2}\left(A\right)$}
of $\lambda_{2}$. Thus, in this case, the claim follows vacuously.
So, without loss of generality, we can assume that $\lambda_{2}\left(A_{i}\right)<\infty$
for every $i\in\N$. First, note that for any $n\in\N$,$\bigcup_{i=1}^{n}A_{i}\subseteq A$
and so, by the monotonictity and finite additivity of $\lambda_{2}$,
we have that
\[
\lambda_{2}\left(A\right)\geq\lambda_{2}\left(\bigcup_{i=1}^{n}A_{i}\right)=\sum_{i=1}^{n}\lambda_{2}\left(A_{i}\right)
\]
for every $n\in\N$. Taking limits, we have countable superadditivity:
\[
\lambda_{2}\left(A\right)\geq\sum_{i=1}^{\infty}\lambda_{2}\left(A_{i}\right).
\]
In order to deduce the reverse inequality, first suppose that both
$A$ and $\left\{ A_{i}\right\} $ are in $\mathcal{L}.$ Then, we
can write 
\[
A:=\left(a,b\right]
\]
and
\[
A_{i}=\left(a_{i},b_{i}\right]
\]
for each $i\in\N.$ Pick an arbitrary $0<\epsilon<b-a$ and observe
that 
\[
\left[a+\epsilon,b\right]\subseteq\bigcup_{i=1}^{\infty}\left(a_{i},b_{i}+\frac{\epsilon}{2^{i}}\right)
\]
and so by the Heine-Borel theorem, there exists some finite $K$ such
that 
\[
\left[a+\epsilon,b\right]\subseteq\bigcup_{k=1}^{K}\left(a_{i_{k}},b_{i_{k}}+\frac{\epsilon}{2^{i_{k}}}\right).
\]
By the finite additivity established in Proposition \ref{prop:ringMeasureFinitelyAdditive}
and monotonicity, we have that
\[
\underbrace{b-a}_{\lambda_{2}\left(A\right)}-\epsilon\leq\sum_{k=1}^{K}b_{i_{k}}+\frac{\epsilon}{2^{i_{k}}}-a_{i_{k}}\leq\underbrace{\sum_{i=1}^{\infty}\left(b_{i}-a_{i}\right)}_{\sum_{i\in\N}\lambda_{2}\left(A_{i}\right)}+\epsilon
\]
and since $\epsilon$ can be arbitrary small the claim follows.

Deducing the general case from the special one outlined above is straightforward.
If $A,\left\{ A_{i}\right\} \in\mathcal{J}$ then 
\[
A=\bigcup_{j=1}^{J}B_{j}
\]
where $\left\{ B_{j}\right\} \in\mathcal{L}$ are pairwise disjoint.
Similarly, 
\[
A_{i}=\bigcup_{k=1}^{n_{i}}C_{ik}
\]
where $\left\{ C_{ij}\right\} _{i\in\N,j\in\N}\in\mathcal{L}$ are
pairwise disjoint and $n_{i}\in\N$. Note that then
\[
\lambda_{2}\left(A\right)=\sum_{j=1}^{J}\lambda_{2}\left(B_{j}\right)\leq\sum_{i=1}^{\infty}\sum_{k=1}^{n_{i}}\lambda_{2}\left(C_{ik}\right)=\sum_{i=1}^{\infty}\lambda_{2}\left(A_{i}\right)
\]
where the first equality follow from the finite additivity of $\lambda_{2}$
on $\mathcal{J},$ the inequality by the fact that for any $j\in\left\{ 1,2,\ldots,J\right\} ,$
there exists a partition of the collection $\left\{ C_{ik}\right\} $
into subcollections $\left\{ C_{ik}^{j}\right\} _{1\leq j\leq J}$such
that 
\[
B_{j}=\bigcup_{i,k}C_{ik}^{j}
\]
and so the special case of our result on $\mathcal{L}$ applies (along
with an application of Lemma \ref{lem:TonelliForSeries}). The final
equality again follows by finite additivity. 
\end{proof}

\subsection{Outer measures}
\begin{defn}
\label{def:outerMeasure}A set valued function 
\[
\mu^{*}:2^{\mathcal{X}}\longrightarrow\left[0,\infty\right]
\]
is called an outer measure on $\mathcal{X}$ if

\begin{enumerate}[label=(\roman*),leftmargin=.1\linewidth,rightmargin=.4\linewidth]
	\item $ \mu^*\left(\emptyset\right) = 0$ 
	\item $A\subseteq B \in 2^\mathcal{X} \Longrightarrow \mu^*\left(A\right) \leq \mu^*\left(B\right) $
	\item For $\left\{A_i\right\}_{i=1}^{\infty}\in 2^\mathcal{X}$ 
	\[
			\mu^*\left(\bigcup_{i=1}^{\infty}A_i\right) \leq \sum_{i=1}^{\infty}\mu^*\left(A_i\right).
	\]
\end{enumerate}
\end{defn}

\begin{example}
\label{exa:canonicalOuterMeasure}Given a non-negative extended-real
valued function $\mu$ on a collection $\mathcal{A\subseteq}2^{\mathcal{X}}$
such that $\mu\left(\emptyset\right)=0$, define for any $E\subseteq\mathcal{X}$
\[
\mu^{*}\left(E\right):=\inf\left\{ \sum_{i=1}^{\infty}\mu\left(A_{i}\right)\mid A_{i}\in\mathcal{A},E\subseteq\bigcup_{i=1}^{\infty}A_{i}\right\} 
\]
\end{example}

Note that this function is defined on $2^{\mathcal{X}}$ since every
bounded below subset of the (extended) real numbers has an infimum.
Now we prove that the set-function descibed above is indeed an outer
measure.
\begin{prop}
\label{prop:canonicalOuterMeasureIsOuterMeasure}The function $\mu^{*}:2^{\mathcal{X}}\longrightarrow\left[0,\infty\right]$
defined in Example \ref{exa:canonicalOuterMeasure} is an outer measure
\end{prop}

\begin{proof}
For (i), observe that $\emptyset\in\mathcal{A}$ and so $\mu^{*}\left(\emptyset\right)=\mu\left(\emptyset\right)=0.$
Next, let $A\subseteq B\subseteq\mathcal{X}$ and observe that 
\[
\left\{ \sum_{i=1}^{\infty}\mu\left(A_{i}\right)\mid A_{i}\in\mathcal{A},B\subseteq\bigcup_{i=1}^{\infty}A_{i}\right\} \subseteq\left\{ \sum_{i=1}^{\infty}\mu\left(A_{i}\right)\mid A_{i}\in\mathcal{A},A\subseteq\bigcup_{i=1}^{\infty}A_{i}\right\} 
\]
and so 
\[
\mu^{*}\left(B\right)=\inf\left\{ \sum_{i=1}^{\infty}\mu\left(A_{i}\right)\mid A_{i}\in\mathcal{A},B\subseteq\bigcup_{i=1}^{\infty}A_{i}\right\} \geq\inf\left\{ \sum_{i=1}^{\infty}\mu\left(A_{i}\right)\mid A_{i}\in\mathcal{A},A\subseteq\bigcup_{i=1}^{\infty}A_{i}\right\} =\mu^{*}\left(A\right)
\]
which gives us (ii). For (iii), let $\left\{ E_{i}\right\} _{i=1}^{\infty}\in2^{\mathcal{X}}$
be disjoint and assume that $\sum_{i=1}^{\infty}\mu^{*}\left(E_{i}\right)<\infty$
since otherwise the claim is trivial. Fix $\epsilon>0$ and choose
$A_{ij}\in\mathcal{A}$ such that $E_{i}\subseteq\bigcup_{j=1}^{\infty}A_{ij}$
and

\[
\mu^{*}\left(E_{i}\right)\leq\sum_{j=1}^{\infty}\mu(A_{ij})<\mu^{*}\left(E_{i}\right)+\frac{\epsilon}{2^{i}}
\]
for every $i\in\N$\footnote{This is possible due to the assumption that $\mu^{*}\left(E_{i}\right)<\infty$,
which implies that the set $\left\{ \sum_{j=1}^{\infty}\mu\left(A_{ij}\right)\mid A_{ij}\in\mathcal{A},E_{i}\subseteq\bigcup_{j=1}^{\infty}A_{ij}\right\} $
is non-empty. The definition of an infimum then implies that such
a cover $\left\{ A_{ij}\right\} $ exists.}. Observe that
\begin{align*}
E & :=\bigcup_{i=1}^{\infty}E_{i}\subseteq\bigcup_{i=1}^{\infty}\bigcup_{j=1}^{\infty}A_{ij}
\end{align*}
and so 
\[
\mu^{*}\left(E\right)\leq\sum_{i,j\in\N^{2}}\mu\left(A_{ij}\right)=\sum_{i=1}^{\infty}\sum_{j=1}^{\infty}\mu\left(A_{ij}\right)\leq\sum_{i=1}^{\infty}\mu^{*}(E_{i})+\epsilon
\]
where the equality follows by Lemma \ref{lem:TonelliForSeries} and
the second inequality is due to properties of the geometric series.
Since $\epsilon$ was arbitrary, the claim folllows.
\end{proof}
\begin{rem}
The outer measure described above is called the \emph{canonical }outer-measure
as it as by far the most useful type of outer measure in measure theory.
Given a space $\X$, a collection of subsets $\mathcal{A}\subseteq2^{\X}$,
and a countably additive pre-measure $\mu$ on $\mathcal{A}$, we
can call
\[
\mu^{*}\left(E\right):=\inf\left\{ \sum_{i=1}^{\infty}\mu\left(A_{i}\right)\mid A_{i}\in\mathcal{A},E\subseteq\bigcup_{i=1}^{\infty}A_{i}\right\} 
\]
the canonical outer measure generated by $\left(\mu,\mathcal{A}\right)$.
\end{rem}

\begin{prop}
\label{prop:restrictionOfOuterMeasure}Let $\mathcal{A}$, $\mu,$
and $\mu^{*}$ be defined as in Example \ref{exa:canonicalOuterMeasure}.
Then, for any $A\in\mathcal{A}$
\[
\mu^{*}\left(A\right)=\mu\left(A\right).
\]
\end{prop}

\begin{proof}
First, observe that $A$ is a cover for itself and that $\emptyset\in\mathcal{A}$
and so 
\[
\mu^{*}\left(A\right)=\inf\left\{ \sum_{i=1}^{\infty}\mu(A_{i})\mid A_{i}\in\mathcal{A},A\subseteq\bigcup_{i=1}^{\infty}A_{i}\right\} \leq\sum_{i=1}^{\infty}\mu(A_{i})
\]
where $A_{1}=A$ and $A_{i}=\emptyset$ for $i\neq1.$ Therefore,
\[
\mu^{*}\left(A\right)\leq\mu\left(A\right).
\]

To see the reverse inequality, let $\left\{ A_{i}\right\} _{i\in\N}\in\mathcal{A}$
be an arbitrary cover of $A.$ Define,
\[
B_{i}:=A\cap\left(A_{i}\setminus\bigcup_{j=1}^{i-1}A_{j}\right)
\]
and notice that the $\left\{ B_{i}\right\} $ is a pairwise disjoint
collections whose union is $A$ such that $B_{i}\subseteq A_{i}$
for every $i\in\N$. By countable additivity and monotonicity,

\[
\mu\left(A\right)=\sum_{i=1}^{\infty}\mu\left(B_{i}\right)\leq\sum_{i=1}^{\infty}\mu\left(A_{i}\right).
\]
Since $\left\{ A_{i}\right\} \subseteq\mathcal{A}$ is an arbitrary
cover of $A$ , we have that
\[
\mu\left(A\right)\leq\inf\left\{ \sum_{i=1}^{\infty}\mu\left(A_{i}\right)\mid A_{i}\in\mathcal{A},A\subseteq\bigcup_{i=1}^{\infty}A_{i}\right\} =\mu^{*}\left(A\right)
\]
which completes the proof.
\end{proof}
Now we are (finally!!) ready to extend our pre-measure to a bona-fide
measure on a $\sigma$-algebra, using the following theorem.
\begin{thm}[Caratheodory's Extension Theorem]
\label{thm:caratheodoryExtn}Let $\X$ be a set. Given a countably-additive
pre-measure $\mu$ on ring $\mathcal{A\subseteq}2^{\mathcal{X}}$
with canonical outer measure $\mu^{*}$ generated by $\left(\mu,\mathcal{A}\right)$,
define the collection 
\[
\mathcal{C}\left(\mu^{*}\right):=\left\{ A\subseteq\mathcal{X}\mathrm{\ such\ that\ }\mu^{*}\left(E\right)=\mu^{*}\left(A\cap E\right)+\mu^{*}\left(A^{C}\cap E\right)\forall E\in2^{\mathcal{X}}\right\} .
\]
Then

\begin{enumerate}[label=(\roman*),leftmargin=.1\linewidth,rightmargin=.4\linewidth]
	\item $ \mathcal{A}\subseteq \mathcal{C}$.
	\item $ \mathcal{C}\left(\mu^*\right) $ is a $\sigma$-algebra.
	\item $\left.\mu^*\right|_{\mathcal{C}}$ is a countably additive measure on $\mathcal{C}$. 
\end{enumerate}
\end{thm}

\begin{proof}
First we will show (i). Let $A\in\mathcal{A}$ be arbitrary. By the
countable subadditivity of $\mu^{*}$, we know that 
\[
\mu^{*}\left(E\right)=\mu^{*}\left(\left(A\cap E\right)\bigcup\left(A^{C}\cap E\right)\right)\leq\mu^{*}\left(A\cap E\right)+\mu^{*}\left(A^{C}\cap E\right)
\]
for every $E\subseteq\mathcal{X}$. To deduce the reverse inequality,
fix $E$ such that $\mu^{*}\left(E\right)<\infty$ because otherwise
the claim follows trivially. Pick an $\epsilon>0$ and find a cover
$\left\{ A_{i}\right\} _{i=1}^{\infty}\in\mathcal{A}$ of $E$ such
that
\[
\mu^{*}\left(E\right)\leq\sum_{i=1}^{\infty}\mu\left(A_{i}\right)<\mu^{*}\left(E\right)+\epsilon
\]
As in the proof of Proposition \ref{prop:canonicalOuterMeasureIsOuterMeasure},
this is possible because $\mu^{*}\left(E\right)<\infty$ and the definition
of an infimum. Next, observe that
\begin{align*}
E\cap A & \subseteq\bigcup_{i=1}^{\infty}(A_{i}\cap A),\\
E\cap A^{C} & \subseteq\bigcup_{i=1}^{\infty}(A_{i}\cap A^{C})
\end{align*}
and so 
\begin{align*}
\mu^{*}\left(E\cap A\right) & \leq\mu^{*}\left(\bigcup_{i=1}^{\infty}(A_{i}\cap A)\right)\leq\sum_{i=1}^{\infty}\mu^{*}\left(A_{i}\cap A\right)\\
\mu^{*}\left(E\cap A^{C}\right) & \leq\mu^{*}\left(\bigcup_{i=1}^{\infty}(A_{i}\cap A^{C})\right)\leq\sum_{i=1}^{\infty}\mu^{*}\left(A_{i}\cap A^{C}\right)
\end{align*}
where the first inequality follows due to monotonicity and the second
due to subadditivity. Together, these inequalities imply that
\begin{align*}
\mu^{*}\left(A\cap E\right)+\mu^{*}\left(A^{C}\cap E\right) & \leq\sum_{i=1}^{\infty}\mu^{*}\left(A_{i}\cap A\right)+\mu^{*}\left(A_{i}\cap A^{C}\right)\\
 & =\sum_{i=1}^{\infty}\mu\left(A_{i}\cap A\right)+\mu\left(A_{i}\cap A^{C}\right)\\
 & =\sum_{i=1}^{\infty}\mu\left(A_{i}\right)\\
 & <\mu^{*}\left(E\right)+\epsilon
\end{align*}
where the first equality is due to the fact that rings are closed
under intersections and set-differences along with Proposition \ref{prop:restrictionOfOuterMeasure}
and the second equality is due to the countable additivity of $\mu.$
Since $\epsilon$ and $E$ are arbitrary, we have that 
\[
\mu^{*}\left(A\cap E\right)+\mu^{*}\left(A^{C}\cap E\right)\leq\mu^{*}\left(E\right)
\]
for every $E\subseteq\mathcal{X},$ establishing that $\mathcal{A}\subseteq\mathcal{C}$.

Next we show (ii); that is, we prove $\mathcal{C}$ is a $\sigma-$algebra.
Recall Definition \ref{def:sigmaAlgebra} and notice that it is sufficient
to prove that (1) $\emptyset,\mathcal{X}\in\mathcal{C}$; (2) if $A\in\mathcal{C}$
then $A^{C}\in\mathcal{C}$; (3) if $\left\{ A_{i}\right\} _{i=1}^{\infty}\in\mathcal{C}$
then $\bigcup_{i=1}^{\infty}A_{i}\in\mathcal{C}.$ Note that $\emptyset,\mathcal{X}\in\mathcal{C}$
because, trivially,
\[
\mu^{*}\left(E\cap\mathcal{X}\right)+\mu^{*}\left(E\cap\emptyset\right)=\mu^{*}\left(E\right).
\]
Symmetry between $A$ and $A^{C}$ in the definition of $\mathcal{C}$
establishes (2). For (3), we first establish closure under finite
unions and bootstrap this weaker result to yield the stronger claim.
Let $A,B\in\mathcal{C}$ and let $E\subseteq\mathcal{X}$ be arbitrary.
Then

\begin{align*}
\mu^{*}\left(E\right) & =\mu^{*}\left(E\cap A\right)+\mu^{*}\left(E\cap A^{C}\right)\\
 & =\mu^{*}\left(\left(E\cap A\right)\cap B\right)+\mu^{*}\left(\left(E\cap A\right)\cap B^{C}\right)+\mu^{*}\left(E\cap A^{C}\right)\\
 & =\mu^{*}\left(E\cap A\cap B\right)+\mu^{*}\left(E\cap\left(A\cap B\right)^{C}\cap A\right)+\mu^{*}\left(E\cap\left(A\cap B\right)^{C}\cap A^{C}\right)\\
 & =\mu^{*}\left(E\cap A\cap B\right)+\mu^{*}\left(E\cap\left(A\cap B\right)^{C}\right)
\end{align*}
where the second equality is due to the definition of $\mathcal{C}$
and the fact that $B\in\mathcal{\mathcal{C}}$, the third equality
is due to the identities
\begin{align*}
\left(A\cap B\right)^{C}\cap A & =\left(A^{C}\cup B^{C}\right)\cap A=A\cap B^{C}\\
\left(A\cap B\right)^{C}\cap A^{C} & =\left(A^{C}\cup B^{C}\right)\cap A^{C}=A^{C},
\end{align*}
and the fourth equality follows from the definition of $\mathcal{C}$
and that $A\in\mathcal{C}$. This proves that for any $A,B\in\mathcal{C}$,
$A\cap B\in\mathcal{C}.$ Property (2) then implies that $A\cup B\in\mathcal{C}.$

To establish closure under countable unions, fix $E\subseteq\mathcal{X}$
and let $\left\{ A_{i}\right\} _{i=1}^{\infty}\in\mathcal{C}$ be
arbitrary with $B=\bigcup_{i\in\N}A_{i}$ and define
\[
B_{n}:=\bigcup_{i=1}^{n}A_{i}
\]
where $B_{n}\in\mathcal{C}$ by our result on closure under finite
unions. Without loss of generality, we can assume that the $\left\{ A_{i}\right\} $
are pairwise disjoint (since we could otherwise replace $A_{i}$ with
$C_{i}:=A_{i}\setminus\bigcup_{j=1}^{i-1}A_{j}$ which are disjoint
such that $\bigcup_{i=1}^{\infty}A_{i}=\bigcup_{i=1}^{\infty}C_{i}$).
Then, we have that
\begin{align*}
\mu^{*}\left(E\right) & =\mu^{*}\left(E\cap B_{n}^{C}\right)+\mu^{*}\left(E\cap B_{n}\right)\\
 & =\mu^{*}\left(E\cap B_{n}^{C}\right)+\mu^{*}\left(E\cap B_{n}\cap A_{n}\right)+\mu^{*}\left(E\cap B_{n}\cap A_{n}^{C}\right)\\
 & =\mu^{*}\left(E\cap B_{n}^{C}\right)+\mu^{*}\left(E\cap A_{n}\right)+\mu^{*}\left(E\cap B_{n-1}\right)
\end{align*}
where we used the fact that $A_{n}\in\mathcal{C}$ for the second
equality and the disjointness of $A_{i}$ for the third equality.
Observe that the equality $\mu^{*}\left(E\cap B_{n}\right)=\mu^{*}\left(E\cap A_{n}\right)+\mu^{*}\left(E\cap B_{n-1}\right)$
is a recurrence relation that can be expanded as
\[
\mu^{*}\left(E\cap B_{n}\right)=\sum_{i=1}^{n}\mu^{*}\left(E\cap A_{i}\right)
\]
and so 
\begin{align*}
\mu^{*}\left(E\right) & =\mu^{*}\left(E\cap B_{n}^{C}\right)+\sum_{i=1}^{n}\mu^{*}\left(E\cap A_{i}\right)\\
 & \geq\mu^{*}\left(E\cap B^{C}\right)+\sum_{i=1}^{n}\mu^{*}\left(E\cap A_{i}\right)
\end{align*}
for every $n\in\N$ where the inequality is due to the the fact that
$B^{C}\subseteq B_{n}^{C}$ and the monotonicity of outer measures.
After taking limits, we have that
\begin{align*}
\mu^{*}\left(E\right) & \geq\mu^{*}\left(E\cap B^{C}\right)+\sum_{i=1}^{\infty}\mu^{*}\left(E\cap A_{i}\right)\\
 & \geq\mu^{*}\left(E\cap B^{C}\right)+\mu^{*}\left(\bigcup_{i\in\N}\left(E\cap A_{i}\right)\right)\\
 & =\mu^{*}\left(E\cap B^{C}\right)+\mu^{*}\left(E\cap B\right)
\end{align*}
where the second inequality follows by countable subadditivity. Another
application of countable subadditivity yields
\[
\mu^{*}\left(E\right)\leq\mu^{*}\left(E\cap B^{C}\right)+\mu^{*}\left(E\cap B\right)
\]
and together the two inequalities establish that $B\in\mathcal{C},$
finishing the proof of (ii).

Finally, in order to show that $\left.\mu^{*}\right|_{\mathcal{C}}$
is indeed a countably additive measure on $\mathcal{C}$, let $\left\{ A\right\} _{i=1}^{\infty}\in\mathcal{C}$
be pairwise disjoint, and observe that for $B:=\bigcup_{i\in\N}A_{i}\in\mathcal{C}$
and any $E\subseteq\mathcal{X}$
\[
\mu^{*}\left(E\right)\geq\mu^{*}\left(E\cap B^{C}\right)+\sum_{i=1}^{\infty}\mu^{*}\left(E\cap A_{i}\right)
\]
due to our previous work. Letting $E=B$, we have 
\begin{align*}
\mu^{*}\left(B\right) & \geq\sum_{i=1}^{\infty}\mu^{*}\left(B\cap A_{i}\right)\\
 & =\sum_{i=1}^{\infty}\mu^{*}\left(A_{i}\right).
\end{align*}
Since the reverse inequality follows by the subadditivity of the outer
measure, our proof is complete.
\end{proof}
\begin{rem}
\label{rem:noRingReqd}Note that the proof of the facts that $\mathcal{C}\left(\mu^{*}\right)$
is a $\sigma$-algebra and $\mu^{*}|_{\mathcal{C}}$ is countably
additive do not depend on the fact $\mathcal{A}$ is a ring; the proofs
would hold if $\mathcal{A}$ was any collection of sets and $\mu^{*}$
was any outer measure (as opposed to a \emph{canonical }outer measure).
\end{rem}

Note that in general such an extension may not be unique and we provide
sufficient conditions for uniqueness in Theorem \ref{thm:uniquenessMeasures}.

\section{The Steiltjes measure on $\protect\R$}

We now have enough machinery to construct the Lebesgue measure on
$\borel\left(\R\right)$; in fact, the Caratheodory measurabality
critierion discussed in the proof of the extension theorem \ref{thm:caratheodoryExtn}
is strictly larger than the Borel sets, a fact that we shall be able
to establish soon. To show the Lebesgue measure exists, we observe
that $\lambda_{2}$ is a countably-additive pre-measure on $\mathcal{J}$
which is a ring. The canonical outer measure $\lambda^{*}$ generated
from $\left(\lambda_{2},\mathcal{J}\right)$ then can be restricted
to the $\sigma$-algebra of measurable sets $\mathcal{C}\left(\lambda^{*}\right)$
as a measure via Caratheodory's extension theorem. That the Borel
sets $\borel\left(\R\right)\subseteq\mathcal{C}\left(\lambda^{*}\right)$
is clear from the fact that $\mathcal{L}\subseteq\mathcal{\mathcal{J}\subseteq C}\left(\lambda^{*}\right)$
and $\sigma\left(\mathcal{L}\right)=\borel\left(\R\right)$ (see Proposition\ref{prop:sigmaAlgebraGeneratedbyLisBorel}).
The fact that this inclusion is strict is, of course, not obvious;
we will return to this point later.

While this strategy to build the Lebesgue meaure works, we can in
fact do something more general, which will incidentally also help
us establish the properties of the Lebesgue measure. This involves
the notion of what is called a \emph{Steiljes }measure, a concept
which is particularly useful in probability theory. To start, we prove
a stronger version of Caratheodory's theorem.
\begin{thm}[Extension from semi-rings]
\label{thm:semiRingCaratheodoryExtn}Let $\X$ be a set and $\mathcal{A}\subseteq2^{\X}$
be a semi-ring. Suppose $\mu:\mathcal{A}\longrightarrow\left[0,\infty\right]$
is a set function such that

\begin{enumerate}[label=(\roman*),leftmargin=.1\linewidth,rightmargin=.4\linewidth]
	\item $\mu\left(\emptyset\right) = 0 $
	\item  For any disjoint $A,B \in \mathcal{A}$ such that $A\cup B \in \mathcal{A}$
	\[
			\mu\left(A\cup B\right) = \mu\left(A\right) + \mu\left(B\right)
	\]
	\item For any collection $A_i \in \mathcal{A}$ such that $\bigcup_{i\in \mathds{N}} A_i \in \mathcal{A}$
	\[
		\mu\left(\bigcup_{i\in\mathds{N}} A_i \right) \leq \sum_{i \in \mathds{N}} \mu \left(A_i\right)
	\]
\end{enumerate}then the restriction of the canonical outer measure $\mu^{*}$ generated
by $\left(\mu,\mathcal{A}\right)$ to $\mathcal{C}\left(\mu^{*}\right)$
is a measure.
\end{thm}

\begin{proof}
Note that Remark \ref{rem:noRingReqd} tells us that $\mathcal{C}\left(\mu^{*}\right)$
is a $\sigma-$algebra and $\mu^{*}|_{\mathcal{C}}$ is a measure.
Thus our two tasks are to show \emph{(i)} that $\mu^{*}$ and $\mu$
agree on\emph{ $\mathcal{A}$ }and\emph{ (ii)} that $\mathcal{A}\subseteq\mathcal{C}\left(\mu^{*}\right)$.
The first result is mostly straightforward; to see that $\mu^{*}\left(A\right)\leq\mu\left(A\right)$
for $A\in\mathcal{A}$ we can simply observe that $A$ is a cover
for itself. For the reverse inequality, first note that finite additivity
and the fact that $\mathcal{A}$ is a semi-ring implies that for any
$A,B\in\mathcal{A}$ such that $A\subseteq B,\mu\left(A\right)\leq\mu\left(B\right).$
Indeed, there exist disjoint $\left\{ C_{i}\right\} _{1\leq i\leq n}\in\mathcal{A}$
where $n\in\N$ such that $B\setminus A=\cup_{1\leq i\leq n}C_{i}$
and so $B=A\cup\cup_{1\leq i\leq n}C_{i}$ and $\mu\left(B\right)=\mu\left(A\right)+\sum_{1\leq i\leq n}\mu\left(C_{i}\right)\geq\mu\left(A\right).$
Then for any cover $\left\{ A_{i}\right\} _{i\in\N}\in\mathcal{A}$
of $A$, we have that 
\[
\mu\left(A\right)=\sum_{i=1}^{\infty}\mu\left(A\cap A_{i}\right)\leq\sum_{i=1}^{\infty}\mu\left(A_{i}\right)
\]
where the equality follows from the fact that $A\cap A_{i}\in\mathcal{A}$
since semi-rings are closed under intersection along with the fact
that $\left\{ A\cap A_{i}\right\} _{i\in\N}$ forms a partition of
$A$ and the inequality follows from monotonicty. Since the cover
was arbitrary, we have $\mu^{*}\left(A\right)\geq\mu\left(A\right).$

To prove \emph{(ii), }note that for any $A\in\mathcal{A}$, countable
subadditivity of the outer measure implies that 
\[
\mu^{*}\left(E\right)\leq\mu^{*}\left(A\cap E\right)+\mu^{*}\left(A^{C}\cap E\right)
\]
for any $E\in2^{\X}.$ To deduce the other inequality, we follow almost
exactly the same steps as we did in the proof of Theorem \ref{thm:caratheodoryExtn}.
First, we pick an $E\subseteq2^{\X}$ such that $\mu^{*}\left(E\right)<\infty$since
otherwise the claim follows trivially. Then we use this fact (since
only empty subsets of the reals have inifinite infima) to deduce that
for any $\epsilon>0$, there exists a cover $\left\{ A_{i}\right\} _{i\in\N}\in\mathcal{A}$of
$E$ such that
\[
\mu^{*}\left(E\right)\leq\sum_{i=1}^{\infty}\mu\left(A_{i}\right)<\mu^{*}\left(E\right)+\epsilon.
\]
Again, we observe that 
\begin{align*}
E\cap A & \subseteq\bigcup_{i=1}^{\infty}A_{i}\cap A\\
E\cap A^{C} & \subseteq\bigcup_{i=1}^{\infty}A_{i}\cap A^{C}
\end{align*}
Note that $A_{i}\cap A^{C}=A_{i}\setminus A$ and so by the properties
of semi-rings there exists, for each $i,$a disjoint collection of
sets $\left\{ C_{j}^{i}\right\} _{1\leq j\leq n_{i}}\in\mathcal{A}$
such that $A_{i}\cap A^{C}=\bigcup_{1\leq j\leq n_{i}}C_{j}^{i}.$
Using the monotonicty and countable-subadditivity of outer measures
as before, we have that
\begin{align*}
\mu^{*}\left(E\cap A\right)+\mu^{*}\left(E\cap A^{C}\right) & \leq\sum_{i=1}^{\infty}\left(\mu^{*}\left(A_{i}\cap A\right)+\mu^{*}\left(\bigcup_{1\leq j\leq n_{i}}C_{j}^{i}\right)\right)\\
 & \leq\sum_{i=1}^{\infty}\left(\mu^{*}\left(A_{i}\cap A\right)+\sum_{j=1}^{n_{i}}\mu^{*}\left(C_{j}^{i}\right)\right)\\
 & =\sum_{i=1}^{\infty}\left(\mu\left(A_{i}\cap A\right)+\sum_{j=1}^{n_{i}}\mu\left(C_{j}^{i}\right)\right)\\
 & =\sum_{i=1}^{\infty}\left(\mu\left(A_{i}\right)\right)\\
 & <\mu^{*}\left(E\right)+\epsilon
\end{align*}
where the first equality follows from part \emph{(i) }and the fact
that $A_{i}\cap A,\left\{ C_{j}^{i}\right\} _{1\leq j\leq n_{i}}\in\mathcal{A}$
whereas the second equality follows from finite additivity. Since
$\epsilon$can be as small as one wants, our result follows.
\end{proof}
\hl{ADD STIELJES MEASURE CONSTRUCTION FROM ASH PROBABILITY}
\begin{thm}[Existence of the Lebesgue measure]
\label{thm:existenceLebesgueR}There exists a $\sigma-$algebra $\F$
which contains all the open sets in $\R$ and a set function $\lambda:\F\longrightarrow\left[0,\infty\right]$
such that

\begin{enumerate}[label=(\roman*),leftmargin=.1\linewidth,rightmargin=.4\linewidth]
	\item $\lambda\left((a,b]\right) = b - a $ for any $a \leq b \in \mathds{R}$\footnote{The intervals could be open, closed or neither.}
	\item For any disjoint $A_i \in \mathcal{F}$
	\[
			\lambda\left(\bigcup_{i\in\mathds{N}}\right) = \sum_{i \in \mathds{N}} \lambda\left(A_i\right)
	\]
	\item For disjoint $\{A_i\}_{i\in \N} \in \mathcal{F}$ 
	\[
			\mu\left(\bigcup_{i=1}^{\infty}A_i\right) = \sum_{i=1}^{\infty}\mu\left(A_i\right).
	\]
	\item
\end{enumerate}
\end{thm}


\section{Abstract measure spaces}
\begin{defn}
\label{def:measurableSpace}A pair $\left(\mathcal{X},\mathcal{F}\right)$,
where $\mathcal{X}$ is an arbitrary set and $\mathcal{F}$ is a $\sigma-$algebra
on $\mathcal{X}$, is called a \emph{measurable space.}
\end{defn}

Although we had implicitly defined a measure in the previous section,
it's appopriate to write down a formal definition in this section.
\begin{defn}
\label{def:measureSpace}Let $\left(\mathcal{X},\mathcal{F}\right)$
be a measurable space. A function $\mu:\mathcal{F}\longrightarrow\left[0,\infty\right]$
is a \emph{measure }on $\mathcal{X}$ if

\begin{enumerate}[label=(\roman*),leftmargin=.1\linewidth,rightmargin=.4\linewidth]
	\item $\mu\left(\emptyset\right)= 0$
	\item For disjoint $\{A_i\}_{i\in \N} \in \mathcal{F}$ 
	\[
			\mu\left(\bigcup_{i=1}^{\infty}A_i\right) = \sum_{i=1}^{\infty}\mu\left(A_i\right).
	\]
\end{enumerate}The triple $\left(\mathcal{X},\mathcal{F},\mu\right)$ is called a
\emph{measure space. }If $\mu\left(\mathcal{X}\right)=1$ then $\mu$
is called a \emph{probability measure }and $\left(\mathcal{X},\mathcal{F},\mu\right)$
is called a \emph{probability space.}
\end{defn}

\begin{defn}
\label{def:measurableSet}Given a measurable space $\left(\mathcal{X},\mathcal{F}\right)$,
any set $A\in\mathcal{F}$ is called a \emph{measurable }set. Conversely,
any set $A\subset\mathcal{X}$ such that $A\notin\mathcal{F}$ is
referred to as a \emph{non-measurable }set.
\end{defn}

While the definition of a measure is simple, it turns out to have
some remarkable properties that are useful in the theory of integration
and probability that is built on top of measure theory (or, as we
shall later see, is equivalent to it).
\begin{prop}
\label{prop:measureProperties}Let $\left(\mathcal{X},\mathcal{F}\right)$
be a measurabe space and let 
\[
\mu:\mathcal{F}\longrightarrow\left[0,\infty\right]
\]
be a function. Then $\mu$ is a measure if and only if

\begin{enumerate}[label=(\roman*),leftmargin=.1\linewidth,rightmargin=.4\linewidth]
	\item $\mu\left(\emptyset\right)= 0$
	\item For disjoint $A,B \in \mathcal{F}$ 
	\[
			\mu\left(A \cup B\right) = \mu\left(A\right) + \mu\left(B\right) .
	\]
	\item For any increasing sequence of sets $ A_1 \subseteq A_2 \ldots $ in $\mathcal{F}$ such that $\bigcup_{i\in\N} A_i = A $
	\[
			\mu\left(A\right) = \lim_{i \to \infty}\mu\left(A_i\right)
	\]
\end{enumerate}
\end{prop}

\begin{proof}
First we shall establish that Definition \ref{def:measureSpace} implies
properties (i)-(iii) above. Property (i) is inherited straight from
the definition; to see (ii), we can let $A_{1}=A,A_{2}=B$ and $A_{j}=\emptyset$
for all $j\geq3$. Then
\[
\mu\left(A\cup B\right)=\mu\left(\bigcup_{j\in\N}A_{j}\right)=\sum_{j=1}^{\infty}\mu\left(A_{j}\right)=\mu\left(A\right)+\mu\left(B\right)
\]
where the second equality is due countably additiivity and the third
equality is due to property (i). To see property (iii), let $\left\{ A_{i}\right\} _{i\in\N}$
be an increasing sequence of sets such that $A_{i}\subseteq A_{i+1}$
for every $i\in\N$ and let $A:=\bigcup_{i\in N}A_{i}$. Define 
\[
B_{i}:=A_{i}\setminus\bigcup_{j=1}^{i-1}A_{j}
\]
which is the standard ``disjointification'' of $\left\{ A_{i}\right\} _{i\in\N}$
as we have seen earlier. By countable additivity
\begin{align*}
\mu\left(A\right) & =\sum_{i=1}^{\infty}\mu\left(B_{i}\right)\\
 & =\lim_{n\to\infty}\sum_{i=1}^{n}\mu\left(B_{i}\right)\\
 & =\lim_{n\to\infty}\mu\left(\bigcup_{i=1}^{n}B_{i}\right)\\
 & =\lim_{n\to\infty}\mu\left(\bigcup_{i=1}^{n}A_{i}\right)\\
 & =\lim_{n\to\infty}\mu\left(A_{n}\right)
\end{align*}
where the third equality is due to property (ii). The fourth equality
follows from the disjointification and the last equality is due to
the increasing nature of the sequence of sets.

Next, we shall establish countable additivity while assuming properties
(i)-(iii) in order to complete the equivalence. Let $\left\{ A_{i}\right\} _{i\in\N}$
be pairwise disjoint in $\mathcal{F}$. Then, letting $A:=\bigcup_{i\in\N}A_{i}$
we can define
\[
B_{n}:=\bigcup_{i=1}^{n}A_{i}
\]
and observe that $\bigcup_{n\in\N}B_{n}=A$ and $B_{n}\subseteq B_{n+1}.$
Then, by property (iii), 
\begin{align*}
\mu\left(A\right) & =\lim_{n\to\infty}\mu\left(B_{n}\right)\\
 & =\lim_{n\to\infty}\sum_{i=1}^{n}\mu\left(A_{i}\right)\\
 & =\sum_{i=1}^{\infty}\mu\left(A_{i}\right)
\end{align*}
where the second equality is due to finite additivity (property (ii)).
This completes the proof.
\end{proof}
\begin{rem*}
Property (iii) resembles a continuity condition, and is indeed called
\emph{continuity from below }of measures. There is an analagous definition
for \emph{continuity from above} which is implied by \emph{continuity
from above }for finitely additive measures and pre-measures. If the
measures are finite, these two notions of continuity are in fact equivalent.
\end{rem*}
\begin{cor}
\label{cor:countableSubadditivity}Every measure $\mu$ on an arbitrary
measurable space $\left(\mathcal{X},\mathcal{F}\right)$ is countably
subadditive i.e. for any collection $\left\{ A_{i}\right\} _{i\in\N}\in\mathcal{F}$
\[
\mu\left(\bigcup_{i=1}^{\infty}A_{i}\right)\leq\sum_{i=1}^{\infty}\mu\left(A_{i}\right).
\]
\end{cor}

\begin{proof}
We shall first establish \emph{finite }subadditivity and bootstrap
this result to countable subadditivity. To see finite subadditivity,
let $A,B\in\mathcal{F}$ be arbitrary, and observe that
\[
A\cup B=\left(A\setminus B\right)\cup B.
\]
The two sets on the right hand side are disjoint and so by finite
additivity
\begin{align*}
\mu\left(A\cup B\right) & =\mu\left(A\setminus B\right)+\mu\left(B\right).
\end{align*}
Adding $\mu\left(A\cap B\right)$ and applying finite additivity again,
we deduce that
\[
\mu\left(A\cup B\right)+\mu\left(A\cap B\right)=\mu\left(A\right)+\mu\left(B\right)
\]
which establishes finite subadditivity. To prove the countable analogue,
let
\[
B_{n}:=\bigcup_{i=1}^{n}A_{i}
\]
and observe that by finite subadditivity
\[
\mu\left(B_{n}\right)\leq\sum_{i=1}^{n}\mu\left(A_{i}\right)\leq\sum_{i=1}^{\infty}\mu\left(A_{i}\right)
\]
where the last inequality follows by the non-negativity of $\mu.$
Note that since $B_{n}$ is an increasing sequence, we can apply Proposition
\ref{prop:measureProperties} (iii) to infer that
\[
\mu\left(\bigcup_{i=1}^{\infty}A_{i}\right)=\lim_{n\to\infty}\mu\left(B_{n}\right)\leq\sum_{i=1}^{\infty}\mu\left(A_{i}\right).
\]
\end{proof}
\begin{prop}
\label{prop:equivalenceContinuityMeasures}For a finitely additive
measure $\mu:\mathcal{F}\longrightarrow\left[0,\infty\right),$ the
following statements are equivalent:

\begin{enumerate}[label=(\roman*),leftmargin=.1\linewidth,rightmargin=.4\linewidth]
	\item For any increasing sequence of sets $\left\{ A_{i}\right\} _{i\in\N}$ such that $A_i \subseteq A_{i+1}$ for all $i\in \N$
	\[
					\mu\left(\bigcup_{i=1}^{\infty} A_i\right) = \lim_{i\to\infty}\mu\left(A_i\right).
	\]
	\item For any decreasing sequence of sets $\left\{ A_{i}\right\} _{i\in\N}$ such that $ A_{i+1}\subseteq A_i$ for all $i\in \N$ 
	\[
					\mu\left(\bigcap_{i=1}^{\infty} A_i\right) = \lim_{i\to\infty}\mu\left(A_i\right).
	\]
\end{enumerate}
\end{prop}

\begin{proof}
Assuming (i), let $\left\{ A_{i}\right\} _{i\in\N}$ be a decreasing
sequence of sets and let $A:=\bigcap_{i\in\N}A_{i}$. Then define
$B_{i}=A_{1}\setminus A_{i}$ which is an increasing sequence of sets
such that $A_{1}\setminus A=\bigcup_{i\in\N}B_{i}$. By (i), 
\[
\mu\left(A_{1}\right)-\mu\left(A\right)=\mu\left(A_{1}\setminus A\right)=\lim_{i\to\infty}\mu\left(B_{i}\right)=\mu\left(A_{1}\right)-\lim_{i\to\infty}\mu\left(A_{i}\right)
\]
where the first and last equality are due to finite additivity, the
finiteness of $\mu.$ We can subtract $\mu\left(A_{1}\right)$ from
both sides to yield the result.

To establish the converse, assume (ii) and let $\left\{ A_{i}\right\} _{i\in\N}$
be an increasing sequence of sets and define $A:=\bigcup_{i\in\N}A_{i}$.
Let $B_{i}:=A\setminus A_{i}$ which is a decreasing sequence of sets
such that $\bigcap_{i\in\N}B_{i}=\emptyset$. By (ii), we have that
\[
0=\mu\left(\emptyset\right)=\lim_{i\to\infty}\mu\left(B_{i}\right)=\lim_{i\to\infty}\mu\left(A\setminus A_{i}\right)=\mu\left(A\right)-\lim_{i\to\infty}\mu\left(A_{i}\right)
\]
where the last equality is again due to finite additivity and the
finitenesss of $\mu.$ Rearrangement yields the proof.
\end{proof}
Observe how the two results apply without modification to pre-measures
as well and so we can establish the countable additivity of $\lambda_{2}$
(see the previous section) using a continuity argument instead of
the Heine-Borel argument we previously used (Exercise!).
\begin{prop}
\label{prop:sumOfCountableMeasures}Let $\left(\mathcal{X},\mathcal{F}\right)$
be a measurable space and let $\left\{ \mu_{i}\right\} _{i\in\mathcal{I}}$
be a collection of measures on $\mathcal{F}$ where $\mathcal{I}$
is at most countable. Then
\[
\mu:=\sum_{i\in\mathcal{I}}\mu_{i}
\]
is a measure on $\mathcal{F}$.
\end{prop}

\begin{proof}
First observe that 
\[
\mu\left(\emptyset\right)=\sum_{i\in\mathcal{I}}\mu_{i}\left(\emptyset\right)=0.
\]
Next, let $\left\{ A_{j}\right\} _{j\in\N}\in\mathcal{F}$ be disjoint.
Then
\begin{align*}
\mu\left(\bigcup_{j\in\N}A_{j}\right) & =\sum_{i\in\mathcal{I}}\mu_{i}\left(\bigcup_{j\in\N}A_{j}\right)\\
 & =\sum_{i\in\mathcal{I}}\sum_{j\in\N}\mu_{i}\left(A_{j}\right)\\
 & =\sum_{j\in\N}\sum_{i\in\mathcal{I}}\mu_{i}\left(A_{j}\right)\\
 & =\sum_{j\in\N}\mu\left(A_{j}\right)
\end{align*}
where the second equality follows from the countable additivity of
$\mu_{i}$ and the third equality follows from the non-negativity
of measures and Lemma \ref{lem:TonelliForSeries}. This completes
the proof.
\end{proof}

\subsection{$\sigma-$finite measure spaces}
\begin{defn}
\label{def:sigmaFinite}Let $\left(\X,\F,\mu\right)$ be a measure
space. The measure $\mu$ is said to be $\sigma-$\emph{finite }if
there exists some increasing sequence of sets $\left\{ E_{i}\right\} _{i\in\N}\in\F$
such that $\mu\left(E_{i}\right)<\infty$ and
\[
\bigcup_{i\in\N}E_{i}=\X.
\]

\hl{Describe the Lebesgue measure on real line as a canonical example of a sigma-finite measure}
\end{defn}

\begin{prop}
\label{prop:equivSigmaFinite}Let $\left(\X,\F,\mu\right)$ be a measure
space. The measure $\mu$ being $\sigma-$finite is equivalent to
any of the following conditions

\begin{enumerate}[label=(\roman*),leftmargin=.1\linewidth,rightmargin=0.15\linewidth]
	\item There exists some \textbf{pairwise disjoint} countable collection of sets $\{A_i\}_{i\in\N} \in \F$ such that $ \mu\left(A_i\right) < \infty $ for all $ i \in \N $ and
	\[
					\bigcup_{i\in\N}A_i = \X
	\]
	\item There exists some countable collection of sets $\{B_i\}_{i\in\N} \in \F$ such that $ \mu\left(B_i\right) < \infty $ for all $ i \in \N $ and
	\[
					\bigcup_{i\in\N}B_i = \X
	\]
\end{enumerate}
\end{prop}

\begin{proof}
First assume that the measure $\mu$is $\sigma-$finite and so there
exists some increasing sequence $\left\{ E_{i}\right\} _{i\in\N}\in\F$
such that $E_{i}\subseteq E_{I+1}$, $\mu\left(E_{i}\right)<\infty$
and 
\[
\bigcup_{i\in\N}E_{i}=\X.
\]
Recall the disjointification
\[
A_{i}:=E_{i}\setminus\bigcup_{j=1}^{i-1}E_{j}
\]
and notice that
\begin{align*}
\mu\left(A_{i}\right) & =\mu\left(E_{i}\right)-\mu\left(\bigcup_{j=1}^{i-1}E_{j}\right)\\
 & =\mu\left(E_{i}\right)-\mu\left(E_{i-1}\right)\\
 & <\infty
\end{align*}
where the first equality follows from the fact that $\mu\left(E_{i}\right)<\infty$
and $\bigcup_{j=1}^{i-1}E_{j}=E_{i-1}\subseteq E_{i}$ along with
(finite) additivity. Further,
\[
\bigcup_{i\in\N}A_{i}=\X
\]
 and so Definition \ref{def:sigmaFinite} implies (i).

Next notice that (i) trivially implies (ii) and so all we just need
to verify (ii) $\implies$Definition \ref{def:sigmaFinite}. To this
end, observe that if $\left\{ B_{i}\right\} _{i\in\N}\in\F$ is an
arbitrary collection that satisfies (ii), then
\[
E_{n}:=\text{\ensuremath{\bigcup_{i=1}^{n}B_{i}}}
\]
is an increasing sequence of sets $E_{n}\subseteq E_{n+1}$ such that
\[
\mu\left(E_{n}\right)\leq\sum_{i=1}^{n}\mu\left(B_{i}\right)<\infty
\]
and
\[
\bigcup_{n\in\N}E_{n}=\X.
\]
\end{proof}
\begin{prop}
\label{prop:sumSigmaFiniteMeasures}Let $\left(\X,\F\right)$ be a
measurable space and let $\left\{ \mu_{i}\right\} _{i=1}^{N}$be a
finite collection of $\sigma-$finite measure on $\F.$ Then the total
measure
\[
\mu:\F\longrightarrow\R
\]
 given by
\[
\mu\left(A\right):=\sum_{i=1}^{N}\mu_{i}\left(A\right)
\]
is also $\sigma-$finite.
\end{prop}

\begin{proof}
A weaker variant of Proposition \ref{prop:sumOfCountableMeasures}
shows that $\mu$is at least a measure on $\F$. We show $\sigma-$finiteness
for $N=2$; the general case follows by induction. Note that if $\mu_{1}$
and $\mu_{2}$ are both $\sigma-$finite then by Proposition \ref{prop:equivSigmaFinite}
there exist $\left\{ E_{1,i}\right\} _{i\in\N},\left\{ E_{2,i}\right\} _{i\in\N}\in\F$
such that $\mu_{1}\left(E_{i,1}\right)<\infty$ and $\mu_{2}\left(E_{2,i}\right)<\infty$
for all $i\in\N$. Further,
\[
\bigcup_{i\in\N}E_{1,i}=\bigcup_{i\in\N}E_{2,i}=\X.
\]
Then, define
\[
C_{i,j}:=E_{1,i}\bigcap E_{2,j}
\]
and observe that $C_{i,j}\in\F$ and that 
\begin{align*}
\mu\left(C_{i,j}\right) & =\mu_{1}\left(E_{1,i}\cap E_{2,j}\right)+\mu_{2}\left(E_{1,i}\cap E_{2,j}\right)\\
 & \leq\mu_{1}\left(E_{1,i}\right)+\mu_{2}\left(E_{2,j}\right)\\
 & <\infty
\end{align*}
for all $\left(i,j\right)\in\N^{2}$. Finally,
\[
\bigcup_{i\in\N}\bigcup_{j\in\N}C_{i,j}=\X
\]
which by Proposition \ref{prop:equivSigmaFinite} establishes the
result.
\end{proof}




\chapter{Measurable functions\label{chap:measurableFunctions}}

\section{Limits of sets and their indicator functions}

Before we embark on a general description of measurable functions,
it's useful to look at a special kind of function: the indicator function
of a set. These functions are special, because they are essentially
the building blocks of all important functions in measure theory.
In fact, indicator functions of sets are the key to linking abstract
measure theory on one hand, to the theory of integration on the other.
As we shall later see, this link is actually an equivalence: measures
and integrals are equivalent objects, and so, in the context of this
theory, sets and their indicator functions are also in some sense
equivalent. While the full scope of this equivalence will only become
salient when we discuss integration, this section will shed some light
on why we perhaps should expect this \emph{ex-ante}.
\begin{defn}
\label{def:indicatorFunction}Let $\mathcal{X}$ be a set and let
$A\subseteq\mathcal{X}$ be an arbitrary subset. The function 
\[
\mathds{1}_{A}:\mathcal{X}\longrightarrow\left\{ 0,1\right\} 
\]
defined by
\[
\mathds{1}_{A}\left(x\right)=\begin{cases}
1 & x\in A\\
0 & x\notin A
\end{cases}
\]
is called the \emph{indicator function }of set $A$.
\end{defn}

The algebra of sets implies a corresponding Boolean algebra for indicator
functions.
\begin{fact}
\label{fact:indicatorFunctionsFiniteOperations}Let $A,B\subseteq\mathcal{X}$
be arbitrary and let $\mathds{1}_{A},\mathds{1}_{B}$ be their respective
indicator functions. Then the indicator function of the set $C:=A\cup B$
is given by
\[
\mathds{1}_{C}=\max\left\{ \mathds{1}_{A},\mathds{1}_{B}\right\} =\mathds{1}_{A}+\mathds{1}_{B}-\mathds{1}_{A}\mathds{1}_{B}
\]
where the maximum is taken pointwise. Similarly, the indicator function
for set $D:=A\cap B$ is given by
\[
\mathds{1}_{D}=\min\left\{ \mathds{1}_{A},\mathds{1}_{B}\right\} =\mathds{1}_{A}\mathds{1}_{B}.
\]
The indicator function for $A^{C}$ is given by
\[
\mathds{1}_{A^{C}}=1-\mathds{1}_{A}.
\]
\end{fact}

Note that if $A,B$ are disjoint, then the indicator function of their
union is simply the sum of their individual indicators, i.e.
\[
\mathds{1}_{A\cup B}=\mathds{1}_{A}+\mathds{1}_{B}.
\]
We can extend these facts to describe indicator functions of arbitrary
unions and intersections of sets in the obvious way
\begin{prop}
\label{prop:indicatorFunctionsArbitraryOperations}Let $\mathcal{I}$
be an arbitrary index set and let $\left\{ A_{i}\right\} _{i\in\mathcal{I}}\subseteq\mathcal{X}$
be subsets with indicator functions $\left\{ \mathds{1}_{A_{i}}\right\} _{i\in\mathcal{I}}$.
Then, the indicator function for $B:=\bigcup_{i\in\mathcal{I}}A_{i}$
is given by
\[
\mathds{1}_{B}=\sup_{i\in\mathcal{I}}\mathds{1}_{A_{i}}
\]
where the supremum is taken pointwise. Similarly, the indicator function
for $C:=\bigcap_{i\in\mathcal{I}}A_{i}$ is given by
\[
\mathds{1}_{C}=\inf_{i\in\mathcal{I}}\mathds{1}_{A_{i}}.
\]
\end{prop}

\begin{proof}
We provide the argument for $B$; the argument for $C$ is analagous.
Observe that
\begin{align*}
\mathds{1}_{B}\left(x\right)=1 & \Longleftrightarrow x\in\bigcup_{i\in\mathcal{I}}A_{i}\\
 & \Longleftrightarrow x\in A_{i_{0}}\textrm{ for some }i_{0}\in\mathcal{I}\\
 & \Longleftrightarrow\mathds{1}_{A_{i_{0}}}=1\text{ for some }i_{0}\in\mathcal{I}\\
 & \Longleftrightarrow\sup_{i\in\mathcal{I}}\mathds{1}_{A_{i}}=1
\end{align*}
which completes the argument.
\end{proof}
These arguments appear to be rather pedantic, but they are key to
defining limiting operations on sets. With a background in undergraduate
calculus, it can be quite cumbersome to think of a sequence of sets
converging to another set. However, it is quite straightforward to
imagine the pointwise convergence of a sequence of \emph{indicator
functions }of sets. For example, we have what appears to be a fairly
daunting definition for the limit of a sequence of sets.
\begin{defn}
\label{def:limSupInfSets} Let $\left\{ A_{i}\right\} _{i\in\N}$
be a sequence of sets in $2^{\mathcal{X}}$. Then the limit superior
of the sequence is given by
\[
\limsup_{n\to\infty}A_{n}:=\bigcap_{n=1}^{\infty}\bigcup_{i=n}^{\infty}A_{i}.
\]
Similarly, the limit inferior of the sequence is given by
\[
\liminf_{n\to\infty}A_{n}:=\bigcup_{n=1}^{\infty}\bigcap_{i=n}^{\infty}A_{i}.
\]
If $\limsup_{n\to\infty}A_{n}=\liminf_{n\to\infty}A_{n}$ then the
limit of the sequence is defined and is equal to the limit superior
and inferior i.e.
\[
\lim_{n\to\infty}A_{n}:=\limsup_{n\to\infty}A_{n}=\liminf_{n\to\infty}A_{n}.
\]

While these definitions appear arbitrary, they demarcate important
concepts in both analysis and probability. To unpack the intuition,
let's try to understand what it means for an element $x\in\mathcal{X}$
to be in $\limsup_{n\to\infty}A_{n}$. If $x\in\bigcap_{n=1}^{\infty}\bigcup_{i=n}^{\infty}A_{i}$,
then $x\in\bigcup_{i=n}^{\infty}A_{i}$ for every $n\in\N$. That
it is to say, for any $n\in\N$, there exists an $i\geq n$ such that
$x\in A_{i}$. This essentially says that $x$ is in infinitely many
of the sets $\left\{ A_{i}\right\} _{i\in\N}$. In the language of
probability, the event $\limsup_{n\to\infty}A_{n}$ is the event of
outcomes that occur infinitely often in the collection of events $\left\{ A_{i}\right\} _{i\in\N}.$

On the other hand, if $x\in\liminf_{n\to\infty}A_{n},$then there
exists some $n_{0}\in\N$ such that $x\in A_{i}$ for every $i\geq n_{0}$.
Clearly then, $\liminf_{n\to\infty}A_{n}\subseteq\limsup_{n\to\infty}A_{n}$
which mirrors the domination condition for limit superiors and inferiors
of sequences of real numbers or real functions. So when does equality
hold? Note that if $\left\{ A_{i}\right\} _{i\in\N}$ is an increasing
sequence of sets
\begin{align*}
\limsup_{n\to\infty}A_{n} & =\bigcap_{n=1}^{\infty}\bigcup_{i=n}^{\infty}A_{i}\\
 & =\bigcap_{n=1}^{\infty}\bigcup_{i=1}^{\infty}A_{i}\\
 & =\bigcup_{i=1}^{\infty}A_{i}\\
 & =\bigcup_{i=1}^{\infty}\bigcap_{n=i}^{\infty}A_{n}\\
 & =\liminf_{i\to\infty}A_{i}
\end{align*}
where the second and fourth equalitiies follow from the increasing
nature of the sets $A_{i}$. This shows that the continuity from below
condition described in Proposition \ref{prop:measureProperties} is
in fact bona-fide continuity. After developing the theory of integration,
we will (seemingly) generalize this continuity result to measurable
functions in the form of the famous \hyperref[thm:monotoneConvergenceLebInt]{\emph{monotone convergence theorem}}
. Of course, once we know that measures and integrals are essentially
the same objects, it will be clear that continuity from below and
monotone convergence are two sides of the same coin. While this result
is better known in it's integral formulation, there's another result
that is perhaps better known in its measure-theoretic formulation:
the Borel-Cantelli lemma.
\end{defn}

\begin{thm}[First Borel-Cantelli lemma]
\label{thm:borelCantelli}Let $\left(\mathcal{X},\mathcal{F},\mu\right)$
be an arbitrary measure space and let $\left\{ A_{i}\right\} _{i\in\N}\in\mathcal{F}$
be a sequence of sets. If
\[
\sum_{i=1}^{\infty}\mu\left(A_{i}\right)<\infty
\]
then
\[
\mu\left(\limsup_{i\to\infty}A_{i}\right)=0.
\]
\end{thm}

\begin{proof}
Define by $B_{n}:=\bigcup_{i=n}^{\infty}A_{i}$. It's clear that $\left\{ B_{n}\right\} _{n\in\N}$
is a decreasing sequence of sets. More over $\mu\left(B_{1}\right)=\mu\left(\bigcup_{i\in\N}A_{i}\right)\leq\sum_{i=1}^{\infty}\mu\left(A_{i}\right)<\infty$
by \hyperref[cor:countableSubadditivity]{subadditivity} and so, since
$\mu$ is finite on $\left\{ B_{n}\right\} _{n\in\N}$, we can apply
Propositions \ref{prop:measureProperties} and \ref{prop:equivalenceContinuityMeasures}
to establish
\begin{align*}
\mu\left(\bigcap_{n=1}^{\infty}B_{n}\right) & =\lim_{n\to\infty}\mu\left(B_{n}\right)\\
 & =\lim_{n\to\infty}\text{\ensuremath{\mu\left(\bigcup_{i=n}^{\infty}A_{i}\right)}}\\
 & \leq\lim_{n\to\infty}\sum_{i=n}^{\infty}\mu\left(A_{i}\right)\\
 & =\lim_{n\to\infty}\left[\sum_{i=1}^{\infty}\mu\left(A_{i}\right)-\sum_{i=1}^{n-1}\mu\left(A_{i}\right)\right]\\
 & =0
\end{align*}
where the inequality follows from subadditivity and the last equality
is due to the the assumption that $\sum_{i\in\N}\mu\left(A_{i}\right)<\infty$
along with the fact that a sequence and its tail have the same limit.
\end{proof}
\begin{rem*}
This version of the Borel-Cantelli lemma is sometimes called the \emph{first
}Borel-Cantelli lemma since its converse, which is true under certain
conditions, is also called the Borel-Cantelli lemma in the literature.
To prevent ambiguity, we refer to the conversee result as the \emph{second
}Borel-Cantelli lemma. The second Borel-Cantelli lemma uses the probabilistic
concept of independence and is covered in Theorem\ref{thm:secondBorelCantelli}
in the chapter on independence and as such, we will relegate the discussion
of the second Borel-Cantelli lemma to when we formally delve into
probabiility theory in the second part of these notes.
\end{rem*}
By now you should be sufficiently convinced that our definitions of
the limiting behavior sets indeed make sense. However, if you any
doubts, our treatment of indicator functions should help resolve them
completely
\begin{prop}
\label{prop:limSupInfIndicator}Let $\left\{ A_{i}\right\} _{i\in\N}$
be a collection of subsets of $\mathcal{X}$ and let $\left\{ \indicate_{A_{i}}\right\} _{i\in\N}$
be their corresponding indicator functions. Then
\[
\limsup_{n\to\infty}\indicate_{A_{n}}=\indicate_{\limsup_{n\to\infty}A_{n}}
\]
and
\[
\liminf_{n\to\infty}\indicate_{A_{n}}=\indicate_{\liminf_{n\to\infty}A_{n}}.
\]
\end{prop}

\begin{proof}
We prove the first assertion as the second one follows by the same
argument. Note that
\begin{align*}
\limsup_{n\to\infty}\indicate_{A_{n}} & =\inf_{n\in\N}\left\{ \sup_{i\geq n}\indicate_{A_{i}}\right\} \\
 & =\inf_{n\in\N}\left\{ \indicate_{\bigcup_{i\geq n}A_{i}}\right\} \\
 & =\indicate_{\bigcap_{n\in\N}\bigcup_{i\geq n}A_{i}}\\
 & =\indicate_{\limsup_{n\to\infty}A_{n}}
\end{align*}
where the second and third equalities follow by Proposition \ref{prop:indicatorFunctionsArbitraryOperations}.
\end{proof}
By now, you should have begun to appreciate that indicator functions
are essentially functional equivalents of the sets they indicate:
sets and their indicator functioins are just two representations of
the same object. This is not particularly suprising, given that sets
are defined by their membership and indicator functions describe membership.
In this context, it should then not be suprising that indicator functions
of \emph{measurable sets }are \emph{measurable functions},\emph{ }even
though we have not yet described the latter concept yet. This is indeed
true, and moreover, any non-negative measurable function can be built
by taking a limit of a linear combination of indicators of sets. But
before we can show this, we should first define what a measurable
function is!
\begin{defn}
\label{def:measurableFunction}Let $\left(\X,\mathcal{F}\right)$
and $\left(\mathcal{Y},\mathcal{G}\right)$ be two measurable spaces.
A function 
\[
f:\mathcal{X}\longrightarrow\mathcal{Y}
\]
is called \emph{$\mathcal{F}/\mathcal{G}-$measurable }if for any
$G\in\mathcal{G}$
\[
f^{-1}\left[G\right]\in\mathcal{F}.
\]
\end{defn}

\begin{rem*}
This definition also resembles a continuity condition; indeed, if
$\mathcal{F}$ and $\mathcal{G}$ were topologies rather than $\sigma-$algebras,
this would be the definition of a continuous function. It turns out
that if $\mathcal{F}$ and $\mathcal{G}$ are Borel $\sigma-$algebras,
then continuity implies measurability: this is a fact that we establish
in the next section. Moreover, all measurable functions are in some
sense \emph{almost }continuous: we make this notion precise when we
discuss the deep connections between topology and measure theory in
Chapter \ref{chap:measureAndTopology}.
\end{rem*}
Later in these notes, we will stop writing $\mathcal{F}/\mathcal{G}$
explicitly and let the reader infer the $\sigma-$algebras in play
from the context.

\section{Properties of measurable functions}

Armed with our definition of measurable functions, we are ready to
discuss interesting examples of such functions along with their properties.
First, we establish that the measurability of a sets and its indicator
function is indeed equivalent, as we had guessed earlier
\begin{prop}
\label{prop:measurableSetsFunctions}Let $\left(\X,\mathcal{F}\right)$
be a measurable space. Then for any $A\subseteq\mathcal{X}$, $A$
is measurable (i.e. $A\in\mathcal{F}$) if and only if 
\[
\indicate_{A}:\mathcal{X}\longrightarrow\left\{ 0,1\right\} 
\]
is $\mathcal{F}/2^{\left\{ 0,1\right\} }-$measurable.
\end{prop}

\begin{proof}
First assume that $A\in\mathcal{F}$ and observe that if $B=\left\{ 0,1\right\} $
then $\indicate_{A}^{-1}\left[B\right]=\X\in\mathcal{F}$, if $B=\{1\}$
then $\indicate_{A}^{-1}\left[B\right]=A\in\mathcal{F}$, if $B=\left\{ 0\right\} $
then $\indicate_{A}^{-1}\left[B\right]=A^{C}\in\mathcal{F}$, and
if $B=\emptyset$ then $\indicate_{A}^{-1}\left[B\right]=\emptyset\in\mathcal{F}$.

Conversely, assume that $\indicate_{A}$ is measurable and notice
how $A=f^{-1}\left[\left\{ 1\right\} \right]\in\mathcal{F}$ which
completes the proof.
\end{proof}
\begin{cor}
\label{cor:borelMeasurableSetsFunctions}Let $\left(\X,\F\right)$
be a measurable space. For any $A\subseteq\X$, $A\in\F$ if and only
if
\[
\indicate_{A}:\X\longrightarrow\R
\]
is $\F/\borel\left(\R\right)-$measurable.
\end{cor}

\begin{proof}
First assume that $A\in\F$ and observe that for any set $B\in\borel\left(\R\right)$
\begin{align*}
\indicate_{A}^{-1}\left[B\right] & =\indicate_{A}^{-1}\left[\left(B\setminus\left\{ 0,1\right\} \right)\bigcup\left(B\cap\left\{ 0,1\right\} \right)\right]\\
 & =\indicate_{A}^{-1}\left[B\setminus\left\{ 0,1\right\} \right]\bigcup\indicate_{A}^{-1}\left[B\cap\left\{ 0,1\right\} \right]\\
 & =\indicate_{A}^{-1}\left[B\cap\left\{ 0,1\right\} \right]\in\F
\end{align*}
where the second equality follows by the property of preimages and
the last equality follows by the fact that $\indicate_{A}^{-1}\left[B\setminus\left\{ 0,1\right\} \right]=\emptyset$.
The inclusion on the last line then is a consequence of Proposition
\ref{prop:measurableSetsFunctions}.

Conversely, assume that $\indicate_{A}$ is measurable and observe
that since $\left\{ 1\right\} \in\borel\left(\R\right)$, the results
follows trivially.
\end{proof}
In this case, measurability of our function was easy to establish
because the $\sigma-$algebra $2^{\left\{ 0,1\right\} }$ could be
explicitly enumerated. Generally, this is not possible as $\sigma-$algebras
can be extremely large. Nevertheless, it is possible establish measurability
using a smaller class of sets in the target $\sigma-$algebra; this
is the crux of generating class arguments which we discusss more abstractly
in the next section. Even the simplest of such arguments can be quite
powerful, as we shall see with the following result.
\begin{thm}[Generic generating class argument]
\label{thm:genericGeneratingClassArgument}Let $\left(\X,\mathcal{F}\right)$
and $\left(\mathcal{Y},\mathcal{G}\right)$ be measurable spaces and
let $\mathcal{E}\subseteq\mathcal{G}$ be a collection of sets such
that $\sigma\left(\mathcal{E}\right)=\mathcal{G}.$ A function
\[
f:\X\longrightarrow\mathcal{Y}
\]
is $\mathcal{F}/\mathcal{G}-$measurable if and only if 
\[
f^{-1}\left[E\right]\in\mathcal{F}
\]
for every $E\in\mathcal{E}.$
\end{thm}

\begin{proof}
If $f$ is measurable, then by definition, $f^{-1}\left[E\right]\in\mathcal{F}$
for every $E\in\mathcal{E}$ since $\mathcal{E}\subseteq\mathcal{G}$.
Conversely, suppose that $f^{-1}\left[E\right]\in\mathcal{F}$ for
every $E\in\mathcal{E}$ and define 
\[
\mathcal{D}=\left\{ G\in\mathcal{G}\mid f^{-1}\left[G\right]\in\mathcal{F}\right\} .
\]
By assumption, $\mathcal{E}\subseteq\mathcal{D}$ and with a little
effort we can show that $\mathcal{D}$ is in fact a $\sigma-$algebra,
which then shows that $\mathcal{G}=\sigma\left(\mathcal{E}\right)\subseteq\sigma\left(\mathcal{D}\right)=\mathcal{D}.$
First, it is clear that $\emptyset\in\mathcal{D}$ as $f^{-1}\left[\emptyset\right]=\emptyset\in\mathcal{F}$.
Next, for any $A\in\mathcal{D}$, observe that $f^{-1}\left[A^{c}\right]=\left(f^{-1}\left[A\right]\right)^{C}\in\mathcal{F}$
since $\mathcal{F}$ is a $\sigma-$algebra. Finally, for any collection
$\left\{ A_{i}\right\} _{i\in\N}\in\mathcal{D},$ $f^{-1}\left[\bigcup_{i\in\N}A_{i}\right]=\bigcup_{i\in\N}f^{-1}\left[A_{i}\right]\in\mathcal{F}$
again because $\mathcal{F}$ is a $\sigma-$algebra. This completes
the proof.
\end{proof}
Now we can show that continuous functions between two spaces equipped
with Borel $\sigma-$algebras are indeed measurable.
\begin{cor}
\label{cor:continuousFunctionsAreMeasurable}Let $\left(\X,\borel\left(\X\right)\right)$
and $\left(\mathcal{Y},\borel\left(\mathcal{Y}\right)\right)$ be
measurable spaces and let 
\[
f:\mathcal{X}\longrightarrow\mathcal{Y}
\]
be continuous. Then $f$ is $\borel\left(\X\right)/\borel\left(\mathcal{Y}\right)-$measurable.
\end{cor}

\begin{proof}
Let $\mathcal{O}_{\mathcal{Y}}$ be the topology on $\mathcal{Y}$.
By definition
\[
\sigma\left(\mathcal{O}_{\mathcal{Y}}\right)=\borel\left(\mathcal{Y}\right)
\]
and by continuity, for any open set $O\in\mathcal{O}_{\mathcal{Y}}$
\[
f^{-1}\left[O\right]\in\mathcal{O}_{\mathcal{X}}\subseteq\borel\left(\X\right)
\]
where $\mathcal{O}_{\mathcal{X}}$ is the topology on $\X$. Thus
by Theorem \ref{thm:genericGeneratingClassArgument}, $f$ is measurable.
\end{proof}
Next we can use the generic generating class argument to establish
routine properties of measurable functions. First, we prove the following
useful lemma, which is an important property on its own.
\begin{lem}
\label{lem:compositionMeasurableFunctions}Let $\left(\X,\mathcal{F}\right),\left(\mathcal{Y},\mathcal{G}\right),$
and $\left(\mathcal{Z},\mathcal{H}\right)$ be measurable spaces.
If the functions
\begin{align*}
f & :\X\longrightarrow\mathcal{Y}\\
g & :\mathcal{Y}\longrightarrow\mathcal{Z}
\end{align*}
are $\mathcal{F}/\mathcal{G}$ and $\mathcal{G}/\mathcal{H}-$measurable
respectively, then the composition function $\phi:=f\circ g$ is $\mathcal{F}/\mathcal{H}-$measurable.
\end{lem}

\begin{proof}
Let $H\in\mathcal{H}$ be arbitrary and note that
\begin{align*}
\phi^{-1}\left[H\right] & =f^{-1}\left[g^{-1}\left[H\right]\right]\\
 & =f^{-1}\left[G\right]\\
 & =F\in\mathcal{F}
\end{align*}
where $G:=g^{-1}\left[H\right]\in\mathcal{G}$ since $g$ is measurable.
\end{proof}
\begin{prop}
\label{prop:binaryOperationsMeasurableFunctions} Let $\left(\X,\mathcal{F}\right)$
be a measurable space and let $f,g$ be real-valued Borel-measurable
functions on $\X$; that is $f,g:\X\longrightarrow\R$ and are $\mathcal{F}/\borel\left(\R\right)-$measurable.
Define $T:\X\longrightarrow\R^{2}$ as
\[
T\left(x\right):=\left(\begin{array}{c}
f\left(x\right)\\
g\left(x\right)
\end{array}\right)
\]
where $\R^{2}$ is equipped with its Borel $\sigma-$algebra $\borel\left(\R^{2}\right).$
Finally, let the function $\psi:\R^{2}\longrightarrow\R$ be continuous
with respect to the standard topologies. Then the function 
\begin{align*}
h & :=\psi\circ T:\X\longrightarrow\R
\end{align*}
is Borel measurable.
\end{prop}

\begin{proof}
First, note that by Corollary \ref{cor:continuousFunctionsAreMeasurable}
and Lemma \ref{lem:compositionMeasurableFunctions}, we only need
to prove the measurability of $T$ in order to deduce the measurability
of $h$. Let $R$ be an open rectangle in $\R^{2}$ i.e. $R=I_{1}\times I_{2}$
where
\[
I_{j}=\left(a_{j},b_{j}\right)
\]
for $a_{j}>b_{j}\in\R$. Then consider 
\begin{align*}
T^{-1}\left[R\right] & =\left\{ x\in\X\mid T\left(x\right)\in R\right\} \\
 & =\left\{ x\in\X\mid\left(f\left(x\right),g\left(x\right)\right)\in I_{1}\times I_{2}\right\} \\
 & =\left\{ x\in\X\mid f\left(x\right)\in I_{1}\text{ and }g\left(x\right)\in I_{2}\right\} \\
 & =\left\{ x\in\X\mid f\left(x\right)\in I_{1}\right\} \bigcap\left\{ x\in\X\mid g\left(x\right)\in I_{2}\right\} \\
 & =f^{-1}\left[I_{1}\right]\bigcap g^{-1}\left[I_{2}\right]\in\mathcal{F}
\end{align*}
due to the measurability of $f,g$ and the fact that $\sigma-$algebras
are closed under intersection.

Let $\mathcal{R}$ denote the collection of all open rectanges in
$\R^{2}$. Since $\mathcal{R}$ is a subset of all open sets in $\R^{2}$,
we know that $\sigma\left(\mathcal{R}\right)\subseteq\borel\left(\R^{2}\right).$
To deduce the converse inclusion, recall that a small modification
of Lemma \ref{lem:openSetDisjointUnionInterval} will show that any
open set in $\R^{2}$ can be written as a countable union of disjoint
sets in $\mathcal{R}$ i.e. for any open set $O\subseteq\R^{2}$
\[
O=\bigcup_{i\in\N}R_{i}
\]
where $R_{i}\in\mathcal{R}.$ This implies that $O\in\sigma\left(\mathcal{R}\right)$
and so $\borel\left(\R^{2}\right)\subseteq\sigma\left(\mathcal{R}\right).$
Applying a \hyperref[thm:genericGeneratingClassArgument]{generating class argument}
with $\mathcal{R}$ then shows that $T$ is measurable.
\end{proof}
\begin{cor}
\label{cor:examplesBinaryOpsMeasFunc}For any real-valued Borel-measurable
functions $f,g$ on $\left(\X,\mathcal{F}\right)$, the following
functions are also measurable

\begin{enumerate}[label=(\roman*),leftmargin=.1\linewidth,rightmargin=.4\linewidth]
	\item $ h(x) := f(x) + g(x) $
	\item $ h(x) := f(x)g(x) $
	\item $ h(x) := f(x)/g(x)\ \mathrm{where} \ g(x)\neq 0 $
\end{enumerate}
\end{cor}

\begin{proof}
For (i), let $\psi\left(x,y\right):=x+y$ (which is a continuous function)
and apply Proposition \ref{prop:binaryOperationsMeasurableFunctions}.
The other cases are simiilar.
\end{proof}
So we know that measurabiltiy is preserved under addition and multiplication,
but the principal concept in analysis is the limit, and we would like
measurability of functions to be preserved under limiting operations.
The following results help us establish that measurability of real-valued
functions is indeed preserved under pointwise limits, when one exists.
\begin{lem}
\label{lem:collectionIntervalsMeasurable}The following sets are generating
classes for $\borel\left(\R\right):$
\[
\left\{ \left(a,\infty\right)\mid\forall a\in\R\right\} ,\ \left\{ \left[a,\infty\right)\mid\forall a\in\R\right\} ,\ \left\{ \left(-\infty,b\right)\mid\forall b\in\R\right\} ,\ \left\{ \left(-\infty,b\right]\mid\forall a\in\R\right\} .
\]
\end{lem}

\begin{proof}
\hyperref[prop:sigmaAlgebraGeneratedbyLisBorel]{Recall} that the
collection of half-open intervals $\mathcal{L}=\left\{ \left(a,b\right]\mid-\infty<a\leq b<\infty\right\} $
is a generating class for $\borel\left(\R\right).$ Observe that for
any $a\leq b\in\R$,
\[
\left(a,b\right]=\bigcap_{n\in\N}\left(\left(a,\infty\right)\cap\left(b+\frac{1}{n},\infty\right)\right)
\]
and so $\mathcal{L}\subseteq\sigma\left(\left\{ \left(a,\infty\right)\mid\forall a\in\R\right\} \right)$
which implies that $\borel\left(\R\right)\subseteq\sigma\left(\left\{ \left(a,\infty\right)\mid\forall a\in\R\right\} \right).$
To see the reverse inclusion, note that for any $a\in\R$
\[
\left(a,\infty\right)=\bigcup_{b\in\N}\left(a,b\right]
\]
which shows that $\sigma\left(\left\{ \left(a,\infty\right)\mid\forall a\in\R\right\} \right)\subseteq\borel\left(\R\right)$.
The other sets can be shown to be generators in a similar fashion.
\end{proof}
\begin{prop}
\label{prop:supInfMeasurable}For a sequence of real-valued Borel-measurable
functions $\left\{ f_{n}\right\} _{n\in\N}$ on $\left(\X,\mathcal{F}\right)$,
the functions
\[
g:=\sup_{n\in\N}f_{n}
\]
and 
\[
h:=\inf_{n\in\N}f_{n}
\]
are $\mathcal{F}/\borel\left(\overline{\R}\right)$ measurable where
$\overline{\R}$ represents the extended real line.
\end{prop}

\begin{proof}
Let $a\in\R$ be arbitrary. Note that the set
\[
g^{-1}\left[\left(a,\infty\right)\right]=\left\{ x\in\X\mid g\left(x\right)>a\right\} =\bigcup_{n\in\N}\left\{ x\in\X\mid f_{n}\left(x\right)>a\right\} =\bigcup_{n\in\N}f_{n}^{-1}\left[\left(a,\infty\right)\right]
\]
since if $g\left(x\right)>a$ then there's at least one $n\in\N$
such that $f_{n}\left(x\right)>a$. Note that by the measurability
of $f_{n}$, $f_{n}^{-1}\left[\left(a,\infty\right)\right]\in\mathcal{F}$
for every $n\in\N.$ Since $\mathcal{F}$ is a $\sigma-$algebra,
$g^{-1}\left[\left(a,\infty\right)\right]\in\mathcal{F}$ by closure
under countable unions. This establishes the measurability of $g$
by Lemma \ref{lem:collectionIntervalsMeasurable}. A similar argument
establishes the measurability of $h$.
\end{proof}
\begin{cor}
\label{cor:limSupLimInfMeasurable}For a sequence of real-valued Borel-measurable
functions $\left\{ f_{n}\right\} _{n\in\N}$ on $\left(\X,\mathcal{F}\right)$,
the functions
\[
g:=\limsup_{n\to\infty}f_{n}
\]
and 
\[
h:=\liminf_{n\to\infty}f_{n}
\]
are measurable (provided they exist) and if $h=g$ then 
\[
\lim_{n\to\infty}f_{n}=h=g
\]
is measurable.
\end{cor}

\begin{proof}
Recall that 
\[
g=\limsup_{n\to\infty}f_{n}=\inf_{n\in\N}\left\{ \sup_{k\geq n}f_{k}\right\} 
\]
and apply Proposition \ref{prop:supInfMeasurable}. The other results
follow in the same fashion.
\end{proof}
Of course, we are often more interested in establishing the measurability
of vector-valued function i.e functions whose range is some subset
of the Euclidean space $\R^{n}$. This limit theorems for real-valued
functions proved here extend naturally to arbitrary finite dimensional
spaces.
\begin{prop}
Let $\left(\X,\mathcal{F}\right)$ be a measurable space and let the
sequence of functions $\left\{ f_{m}\right\} _{m\in\N}$
\[
f_{m}:\X\longrightarrow\R^{n}
\]
be $\mathcal{F}/\borel\left(\R^{n}\right)-$measurable. The pointwise
limit function $f:=\lim_{m\to\infty}f_{m}$ (if it exists) is $\mathcal{F}/\borel\left(\R^{n}\right)-$measurable
if and only if the projection functions $f_{1},\ldots,f_{n}$ are
$\mathcal{F}/\borel\left(\R\right)-$measurable.
\end{prop}

\begin{proof}
First assume that the projections $f_{1},\ldots,f_{n}$ are measurable.
Then, applying the same proof we used to prove the measurability of
the function $T$ in Proposition \ref{prop:binaryOperationsMeasurableFunctions},
we have that
\[
f=\left(\begin{array}{c}
f_{1}\\
\vdots\\
f_{n}
\end{array}\right)
\]
is measurable.

Conversely, suppose that $f$ is measurable, then recall that the
projection functions $\pi_{k}:\R^{n}\longrightarrow\R$ given by
\[
\pi_{k}\left(\begin{array}{c}
x_{1}\\
\vdots\\
x_{k}\\
\vdots\\
x_{n}
\end{array}\right)=x_{k}
\]
is a continuous function and hence measurable for any $1\leq k\leq n$.
Applying Lemma \ref{lem:compositionMeasurableFunctions}, we have
that $f_{k}=\pi_{k}\circ f$ is measurable for any $1\leq k\leq n.$
\end{proof}
In general, it appears that a measurability is preserved under a wide
range of operations on functions. Additionally, the richness of commonly
seen $\sigma-$algebras makes it the case that almost all functions
we encounter in everyday mathematics are measurable. This heuristic
is a fine one, but it begs the question: precisely how rich does the
$\sigma-$algebra of the domain have to be for a given function to
be measurable? This has a relatively straightforward answer.
\begin{prop}
\label{prop:sigmaAlgebraGeneratedByFunction}Let $\X$ be a set and
let $\left(\mathcal{Y},\mathcal{G}\right)$ be a measurable space.
Let
\[
T:\X\longrightarrow\mathcal{Y}
\]
be a function. Then the ``smallest'' $\sigma-$algebra on $\mathcal{X}$
that makes $T$ measurable is 
\[
\sigma\left(T\right):=\left\{ T^{-1}\left[G\right]\mid G\in\mathcal{G}\right\} .
\]
That is, $\sigma\left(T\right)$ is the intersecion of all $\sigma-$algebras
on $\mathcal{X}$ that makes T measurable.
\end{prop}

\begin{proof}
Let $\mathcal{R}$ be the collection of all $\sigma-$algebras on
$\X$ that makes $T$ measurable. Clearly, $\sigma\left(T\right)\subseteq R$
for every $R\in\mathcal{\mathcal{R}}$ by the definition of measurability.
Thus 
\[
\sigma\left(T\right)\subseteq\bigcap_{R\in\mathcal{R}}R
\]
and so all we have to do is show that $\sigma\left(T\right)\in\mathcal{\mathcal{R}}$
to show the reverse inclusion. To do this, we simply show that $\sigma\left(T\right)$
is a $\sigma-$algebra.

Note that $\emptyset=T^{-1}\left[\emptyset\right]\in\sigma\left(T\right).$
Next, let $A\in\sigma\left(T\right)$ be arbitrary and observe that
$A=T^{-1}\left[G\right]$ for some $G\in\mathcal{G}$. Then $A^{C}=\left(T^{-1}\left[G\right]\right)^{C}=T^{-1}\left[G^{C}\right]\in\sigma\left(T\right)$
as $G^{C}\in\mathcal{G}$ since $\mathcal{G}$ is a $\sigma-$algebra.
Finally, suppose $\left\{ A_{i}\right\} _{i\in\N}\in\sigma\left(T\right)$
and note that $A_{i}=T^{-1}\left[G_{i}\right]$ for $G_{i}\in\mathcal{G}$
and so
\[
\bigcup_{i\in\N}A_{i}=\bigcup_{i\in\N}T^{-1}\left[G_{i}\right]=T^{-1}\left[\bigcup_{i\in\N}G_{i}\right]\in\sigma\left(T\right)
\]
 since $\mathcal{G}$ is closed under countable unions. This completes
the proof.
\end{proof}
\begin{cor}
\label{cor:generatorPreimage}Let $\X$ be a set and let $\left(\mathcal{Y},\mathcal{G}\right)$
be a measurable space. Let $\mathcal{H}\subset\mathcal{G}$ be a generating
class such that $\sigma\left(\mathcal{H}\right)=\mathcal{G}$. Then
for a function $T:\mathcal{X}\to\mathcal{Y}$
\[
\sigma\left(T^{-1}\left(\mathcal{H}\right)\right)=\sigma\left(T\right).
\]
\end{cor}

\begin{proof}
Note that $T^{-1}\left(\mathcal{H}\right)\subseteq\sigma\left(T\right)$
and so $\sigma\left(T^{-1}\left(\mathcal{H}\right)\right)\subseteq\sigma\left(T\right)$.
Conversely, note that $T$ is $\sigma\left(T^{-1}\left(\mathcal{H}\right)\right)/\mathcal{G}$
measurable by the generating class argument in Theorem \ref{thm:genericGeneratingClassArgument}.
But since by Proposition \ref{prop:sigmaAlgebraGeneratedByFunction},$\sigma\left(T\right)$
is the smallest $\sigma-$algebra on which $T$ is measurable, we
have that $\sigma\left(T\right)\subseteq\sigma\left(T^{-1}\left(\mathcal{H}\right)\right)$.
\end{proof}
\begin{prop}
\label{prop:sigmaFunctionOfMeasurableFunctionSubset}Let $\left(\X,\mathcal{F}\right)$
, $\left(\mathcal{Y},\mathcal{G}\right),$ and $\left(\mathcal{Z},\mathcal{H}\right)$
be measurable spaces and let $f:\X\to\mathcal{Y}$ be $\mathcal{F}/\mathcal{G}$
measurable. Then for any $\mathcal{G}/\mathcal{H}$ measurable function
$g:\mathcal{Y}\to\R$
\[
\sigma\left(g\circ f\right)\subseteq\sigma\left(f\right).
\]
\end{prop}

\begin{proof}
Note that for any $H\in\mathcal{H}$, $\left(g\circ f\right)^{-1}\left[H\right]=f^{-1}\left[g^{-1}\left[H\right]\right]\in\sigma\left(f\right)$
which implies the claim.
\end{proof}
Note that $\sigma-$algebras generated by measurable functions constitute
an important subclass of all $\sigma-$algebras.
\begin{prop}
\label{prop:countablyGeneratedSigmaAlgebrasMeasFunc}Let $\X$ be
a set and let $\{B_{i}\}_{i\in\N}\subseteq\X$ be a countable collection
of sets. Then there exists a Borel-measurable function $f:\X\to\R$
such that 
\[
\sigma\left(\{B_{i}\}_{i\in\N}\right)=\sigma\left(f\right).
\]
\end{prop}


\section{Constructing measurable functions}

\subsection{Simple functions}

In the beginning of this chapter, we foreshadowed how the indicator
functions of measurable sets were in fact the building blocks of all
real-valued non-negative measurable functions; now we have developed
just enough of the theory to show that this is true. First, we should
introduce some useful notation to save space in the future. We denote
by $\mathcal{M}\left(\X,\mathcal{F}\right)$ the set of (extended)
real-valued Borel-measurable functions on $\left(\X,\mathcal{F}\right)$.
We write $\mathcal{M}^{+}\left(\X,\mathcal{F}\right)\subset\mathcal{M}\left(\X,\mathcal{F}\right)$
to denote the set of all non-negative (extended) real-valued Borel-measurable
functions on $\left(\X,\mathcal{F}\right)$. Note that the properties
we described for the set of real-valued measurable functions on $\left(\X,\F\right)$
in the previous section carry over to the set of extended real valued
measurable functions, as long as no weird $\infty-\infty$ or $\infty\cdot0$
situations arise.
\begin{defn}
\label{def:simpleFunction} A function $s:\mathcal{X}\longrightarrow\left[0,\infty\right)$
is called a \emph{simple function }if its range is a finite subset
of $\left[0,\infty\right)$ It can be represented as an aggregation
of indicator functions as
\[
s=\sum_{i=1}^{k}\alpha_{i}\indicate_{A_{i}}
\]
where $\left\{ \alpha_{i}\right\} _{i=1}^{k}\in\left[0,\infty\right)$
is the range of $s$ and $A_{i}:=\left\{ x\in\X\mid s\left(x\right)=\alpha_{i}\right\} $
partitions the domain $\X$ into preimages of the singletons in the
range. Such a representation is called the \emph{standard representation
}of $s$
\end{defn}

\begin{rem*}
Note that in general, a simple function could be written as a finite
linear combination of indicator functions in more than one way. For
example, the function $s=3\indicate_{A}+7\indicate_{B}$ can also
be expressed as $3\indicate_{A\setminus B}+10\indicate_{A\cap B}+7\indicate_{B\setminus A}$,
where the latter is the \emph{standard representation }since the sets
$A\cap B,A\setminus B,B\setminus A$ ( and the implicitly included
$\X\setminus\left(A\cup B\right)$) form a partition of the domain
given by preimages of the singletons in the range (namely $\{0,3,7,10\}$).
Standard representations are, of course, unique.
\end{rem*}
It should be clear that a non-negative simple function $s\in\mathcal{M^{+}}\left(\mathcal{X},\mathcal{F}\right)$
if and only if $\left\{ A_{i}\right\} _{i=1}^{k}\in\mathcal{F}$;
the ``if'' part follows directly from Corollary \ref{cor:examplesBinaryOpsMeasFunc}.
To see the ``only if'' part, note that if $s$ is measurable then
$s^{-1}\left[\left\{ \alpha_{i}\right\} \right]=A_{i}\in\mathcal{F}$
since singletons in $\R$ are Borel sets (they are closed). Denote
the collection of these measurable simple functions on $\left(\X,\mathcal{F}\right)$
as $M_{\textnormal{sim}}\left(\X,\mathcal{F}\right)$. Then, we have
shown that $M_{\textnormal{sim}}\left(\X,\mathcal{F}\right)\subset\mathcal{M^{+}}\left(\mathcal{X},\mathcal{F}\right)$.
It turns out that we can make a stronger claim: the measurable simple
functions are in some sense ``dense'' in the space of non-negative
measurable functions.
\begin{prop}
\label{prop:simpleFunctionMonotoneConvergence}Let $f\in\mathcal{M^{+}}\left(\mathcal{X},\mathcal{F}\right)$
be arbitrary. Then there exists a sequence of simple measurable functions
$\left\{ f_{n}\right\} _{n\in\N}\in M_{\textnormal{sim}}\left(\X,\mathcal{F}\right)$
such that
\[
f_{n}\leq f_{n+1}
\]
pointwise for every $n\in\N$ and
\[
\lim_{n\to\infty}f_{n}=f
\]
where the limit is taken pointwise.
\end{prop}

\begin{proof}
It turns out that we can establish the existence of such functions
$f_{n}$ constructively. Define
\[
f_{n}\left(x\right):=\sum_{k=0}^{4^{n}-1}\frac{k}{2^{n}}\indicate_{\left\{ \frac{k}{2^{n}}\leq f\left(x\right)<\frac{k+1}{2^{n}}\right\} }+2^{n}\indicate_{\left\{ f\left(x\right)\geq2^{n}\right\} }
\]
and observe that $f_{n}$ are Borel-measurable simple functions (given
the measurability of $f$) and that $f_{n}\leq f$ pointwise by definition.
Next, note that $\left\{ \frac{k}{2^{n}}\right\} _{k=0}^{4^{n}}\subset\left\{ \frac{k}{2^{n+1}}\right\} _{k=0}^{4^{n+1}}$
and fix $x_{0}\in\X$ to be arbitrary. If $\frac{k_{0}}{2^{n}}\leq f\left(x_{0}\right)<\frac{k_{0}+1}{2^{n}}$
for some $k_{0}\in\left\{ 0,1,\ldots,4^{n}-1\right\} $ then either
$\frac{k_{0}}{2^{n}}=\frac{2k_{0}}{2^{n+1}}\leq f\left(x_{0}\right)<\frac{2k_{0}+1}{2^{n+1}}$
or $\frac{2k_{0}+1}{2^{n+1}}\leq f\left(x_{0}\right)<\frac{2k_{0}+2}{2^{n+1}}=\frac{k_{0}+1}{2^{n}}$.
In either case,
\[
f_{n}\left(x_{0}\right)=\frac{k_{0}}{2^{n}}\leq f_{n+1}\left(x_{0}\right)\leq\frac{2k_{0}+1}{2^{n+1}}
\]
by the definitions of our simple functions $f_{n}.$ Conversely, if
$f\left(x_{0}\right)\geq2^{n}$ then either $f\left(x_{0}\right)\geq2^{n+1}$
or there exists some $k_{0}\in\left\{ 2^{2n+1},\ldots,4^{n+1}-1\right\} $
such that
\[
\frac{k_{0}}{2^{n+1}}\leq f\left(x_{0}\right)<\frac{k_{0}+1}{2^{n+1}}.
\]
Again, in either case,
\[
f_{n}\left(x_{0}\right)=2^{n}\leq\frac{k_{0}}{2^{n+1}}\leq f_{n+1}\left(x_{0}\right)\leq2^{n+1}
\]
which shows that $f_{n}\leq f_{n+1}$ pointwise.

To show convergence, again pick an arbitrary $x_{0}\in\X$ and observe
that if $f\left(x_{0}\right)=\infty$ then $f_{n}\left(x_{0}\right)=2^{n}\nearrow\infty=f\left(x_{0}\right).$
Conversely, suppose that $f\left(x_{0}\right)<\infty$. Then, since
the natural numbers are not bounded above in $\R$, we know that there
exists some $n_{x_{0}}\in\N$ such that for every $n\geq n_{x_{0}}$:
$f\left(x_{0}\right)<2^{n}$. Thus for each such $n,$ there exists
some $k_{n}\in\left\{ 0,1,\ldots,4^{n}-1\right\} $ such that 
\begin{equation}
\frac{k_{n}}{2^{n}}\leq f\left(x_{0}\right)<\frac{k_{n}+1}{2^{n}}\label{eq:pointwiseInequalitySimpleFunc}
\end{equation}
which would imply that $f_{n}\left(x_{0}\right)=\frac{k_{n}}{2^{n}}$
for all $n\geq n_{x_{0}}.$ Then
\[
0\leq f\left(x_{0}\right)-f_{n}\left(x_{0}\right)=f\left(x_{0}\right)-\frac{k_{n}}{2^{n}}\leq\frac{k_{n}+1}{2^{n}}-\frac{k_{n}}{2_{n}}=\frac{1}{2^{n}}
\]
where the first inequality is due to the fact that $f_{n}\leq f$
pointwise and the second inequality is due to (\ref{eq:pointwiseInequalitySimpleFunc}).
Taking limits then yields the result.
\end{proof}
\begin{prop}
\label{prop:minMaxMeasurable}For any function $f\in\mathcal{M}\left(\X,\mathcal{F}\right)$,
the derived functions\footnote{Again, maxima are taken pointwise.}
\begin{align*}
f^{+} & :=\max\left\{ f,0\right\} \\
f^{-} & :=\max\left\{ -f,0\right\} 
\end{align*}
are contained in $\mathcal{M}^{+}\left(\X,\mathcal{F}\right)$
\end{prop}

\begin{proof}
Define for any $a\in\R$ , $F_{a}:=\left\{ x\in\X\mid f^{+}\left(x\right)\in\left(a,\infty\right)\right\} $
and notice that if $a>0$, then
\[
F_{a}=f^{-1}\left[\left(a,\infty\right)\right]\in\mathcal{F}
\]
by the measurability of $f.$ Conversely, if $a\leq0$ then
\[
F_{a}=\X\in\mathcal{F}
\]
and so, by Lemma \ref{lem:collectionIntervalsMeasurable} and a generating
class argument, $f^{+}\in\mathcal{M}^{+}\left(\X,\mathcal{F}\right)$.

To see that $f^{-1}\in\mathcal{M}^{+}\left(\X,\mathcal{F}\right)$,
notice that by Corollary \ref{cor:examplesBinaryOpsMeasFunc}, $-f\in\mathcal{M}\left(\X,\mathcal{F}\right)$
and apply the same argument.
\end{proof}
\begin{rem*}
Another way to prove the above proposition would be to observe that
\[
\max\left\{ x,y\right\} =\frac{x+y+\left|x-y\right|}{2}
\]
and apply Proposition \ref{prop:binaryOperationsMeasurableFunctions}.
\end{rem*}
Proposition \ref{prop:minMaxMeasurable} is important because any
function $f\in\measurableFunctions$ can be decomposed as
\[
f=f^{+}-f^{-}
\]
and since both $f^{+},f^{-}\in\nonnegMeasurableFunctions$, by Proposition
\ref{prop:simpleFunctionMonotoneConvergence}, there are non-negative
simple functions $\left\{ s_{n}^{+}\right\} _{n\in\N},\left\{ s_{n}^{-}\right\} _{n\in\N}$
such that $s_{n}^{+}\nearrow f^{+}$and $s_{n}^{-}\nearrow f^{-}$.
By the linearity of limits, the sequence of functions $h_{n}:=s_{n}^{+}-s_{n}^{-}$
converges (although not monotonically) to $f$. This fact will prove
important in the chapter on integration.
\begin{prop}
\label{prop:simpleFunctionsAddMultiply}Let $s,t\in\mathcal{M}_{\textnormal{sim}}\measurablespace$.
Then
\[
h:=s+t\in\mathcal{M}_{\textnormal{sim}}\measurablespace
\]
and 
\[
g:=st\in\mathcal{M}_{\textnormal{sim}}\measurablespace.
\]
Moreover, if $s$ and $t$ be given by the standard representations
\begin{align*}
s & =\sum_{i=1}^{I}\alpha_{i}\indicate_{A_{i}}\\
t & =\sum_{j=1}^{J}\beta_{j}\mathds{1}_{B_{j}}
\end{align*}
then
\begin{align*}
h & =\sum_{i=1}^{I}\sum_{j=1}^{J}\left(\alpha_{i}+\beta_{j}\right)\indicate_{A_{i}\cap B_{j}}\\
g & =\sum_{i=1}^{I}\sum_{j=1}^{J}\left(\alpha_{i}\beta_{j}\right)\indicate_{A_{i}\cap B_{j}}
\end{align*}
\end{prop}

\begin{proof}
First we prove that 
\[
\sum_{i=1}^{I}\alpha_{i}\indicate_{A_{i}}+\sum_{j=1}^{J}\beta_{j}\mathds{1}_{B_{j}}=\sum_{i=1}^{I}\sum_{j=1}^{J}\left(\alpha_{i}+\beta_{j}\right)\indicate_{A_{i}\cap B_{j}}.
\]
To see this, let $x_{0}\in\X$ be arbitrary and observe that since
$\left\{ A_{i}\right\} _{i=1}^{I}$ and $\left\{ B_{j}\right\} _{j=1}^{J}$
are partitions of $\X$, there exists exactly one $1\leq i_{0}\leq I$
and one $1\leq j_{0}\leq J$ such that $x_{0}\in A_{i_{0}}$ and $x\in B_{j_{0}}$.
Then, $x_{0}\in A_{i_{0}}\cap B_{j_{0}}$ and $x\notin A_{i}\cap B_{j}$
if either $i\neq i_{0}$ or $j\neq j_{0}$.\footnote{In other words, $\left\{ A_{i}\cap B_{j}\right\} _{\left(i,j\right)\in\left\{ 1,\ldots,I\right\} \times\left\{ 1,\ldots,J\right\} }$forms
a partition of $\X$.}Then 
\begin{align*}
\left(\sum_{i=1}^{I}\alpha_{i}\indicate_{A_{i}}+\sum_{j=1}^{J}\beta_{j}\mathds{1}_{B_{j}}\right)\left(x_{0}\right) & =\sum_{i=1}^{I}\alpha_{i}\indicate_{A_{i}}\left(x_{0}\right)+\sum_{j=1}^{J}\beta_{j}\mathds{1}_{B_{j}}\left(x_{0}\right)\\
 & =\alpha_{i_{0}}+\beta_{j_{0}}\\
 & =\sum_{i=1}^{I}\sum_{j=1}^{J}\left(\alpha_{i}+\beta_{j}\right)\indicate_{A_{i}\cap B_{j}}\left(x_{0}\right)
\end{align*}
which establishes our claim. Note that the representation above need
not be standard as the partition $\left\{ A_{i}\cap B_{j}\right\} _{\left(i,j\right)\in\left\{ 1,\ldots,I\right\} \times\left\{ 1,\ldots,J\right\} }$
need not be the collection of preimages of singletons in $\Ran\left(h\right)$;
in fact, it is a \emph{refinement }of the collection of preimages,
and so the preimage of every singleton in $\Ran\left(h\right)$ can
be written as a union of sets in $\left\{ A_{i}\cap B_{j}\right\} _{\left(i,j\right)\in\left\{ 1,\ldots,I\right\} \times\left\{ 1,\ldots,J\right\} }$.

Next, for closure under multiplication, observe that
\begin{align*}
st & =\left(\sum_{i=1}^{I}\alpha_{i}\indicate_{A_{i}}\right)\left(\sum_{j=1}^{J}\beta_{j}\mathds{1}_{B_{j}}\right)\\
 & =\sum_{i=1}^{I}\sum_{j=1}^{J}\left(\alpha_{i}\beta_{j}\right)\indicate_{A_{i}}\indicate_{B_{j}}\\
 & =\sum_{i=1}^{I}\sum_{j=1}^{J}\left(\alpha_{i}\beta_{j}\right)\indicate_{A_{i}\cap B_{j}}
\end{align*}
where the last equality follows from Fact \ref{fact:indicatorFunctionsFiniteOperations}.
For the same reason as before, this representation need not be standard.
\end{proof}
\begin{prop}
\label{prop:doobDynkin}Let $\X$ be an arbitrary set and let $f:\X\to\mathcal{\R}$
be a function, then for any $\sigma\left(f\right)/\mathcal{\borel}\left(\R\right)-$measurable
function $g:\X\longrightarrow\R$, there exists a Borel-measurable
map $h:\R\to\R$ such that
\[
g=h\circ f.
\]
\end{prop}

\begin{proof}
First suppose that $g=\indicate_{A}$ for some $A\in\sigma\left(f\right).$
Then
\[
g=\indicate_{A}\left(x\right)=\indicate_{f^{-1}\left[B\right]}\left(x\right)=\indicate_{B}\left(f(x)\right)
\]
for some $B\in\mathcal{G}.$ Thus in this case, $h=\indicate_{B}$.

Now consider the case of a simple function $g=\sum_{i=1}^{n}\alpha_{i}\indicate_{A_{i}}$
for $\alpha_{i}\in\R,A_{i}\in\sigma\left(F\right).$ Note that by
the above example for indicator functions, we can find $h_{i}:\R\to\R$
such that 
\[
g=\sum_{i=1}^{n}\alpha_{i}h_{i}\left(f(x)\right)
\]
and so the function $h=\sum_{i=1}^{n}\alpha_{i}h_{i}$ does the trick.

Next, for $g\in\nonnegMeasurableFunctions,$ by Proposition \ref{prop:simpleFunctionMonotoneConvergence}we
can find a sequence of simple functions $s_{n}$ such that $s_{n}\nearrow g$
pointwise. Note from our work above that we can write $s_{n}=h_{n}\circ f$
where $h_{n}$ is Borel-measurable as above. Let $h:=\lim_{n\to\infty}h_{n}$
which exists pointwise since it is a monotone limit ($s_{n}\leq s_{n+1}\implies h_{n+1}\leq h_{n}$)
. In particular, $h$ is Borel-measurable by Corollary \ref{cor:limSupLimInfMeasurable}
and so for any fixed $x\in\X$
\[
\lim_{n\to\infty}s_{n}\left(x\right)=\lim_{n\to\infty}h_{n}\left(f\left(x\right)\right)=h\left(f\left(x\right)\right)=g\left(x\right).
\]

Finally, for general measurable $g\in\measurableFunctions$, we can
write $g=g^{+}-g^{-}$ and by the previous results, we know there
exist Borel-measurable real-valued maps $h_{1},h_{2}$ such that $g^{+}=h_{1}\circ f$,
$g^{-}=h_{2}\circ f.$ Then
\begin{align*}
g & =h_{1}\circ f-h_{2}\circ f\\
 & =(h_{1}-h_{2})\circ f
\end{align*}
and $h:=h_{1}-h_{2}$ is Borel-measurable by Corollary \ref{cor:examplesBinaryOpsMeasFunc}.
This completes the proof.
\end{proof}

\subsection{Images of measurable sets under measurable functions}

We have the shown (or rather, defined) that for measurable functions,
preimages of measurable sets are measurable. Does this hold for images
under measurable functions as well? That is, for any pair of measurable
spaces $\left(\X,\F\right)$ and $\left(\mathcal{Y},\mathcal{G}\right)$
and a measurable function $f:\X\to\mathcal{G}$, is it true that $f\left[A\right]\in\mathcal{G}$
for any $A\in\F$? The answer to this question is ``no'' and to
show this, we continue on a thread we began in Subsection \ref{subsec:cantorSets}
(read that section again). We shall first construct a strictly increasing
analogue of the Cantor function $\psi_{C}$ from that section.
\begin{prop}
\label{prop:strictlyIncreasingCantorFunction}Let $\phi\left(x\right):=\psi_{C}\left(x\right)+x$.
Then $\phi$ is a strictly increasing, continuous function that maps
$\left[0,1\right]$ onto $\left[0,2\right]$. Moreover, $\phi\left[C\right]\in\borel\left(\R\right)$
is a set of positive Lebesgue measure.
\end{prop}

\begin{proof}
Note that the sum of a non-decreasing and strictly increasing function
is strictly increasing and the sum of two continuous functions is
continuous. That gives us the first two properties. Now since $\phi\left(0\right)=0$
and $\phi\left(1\right)=2$, by the intermediate value theorem $\phi\left[\left[0,1\right]\right]=\left[0,2\right].$
Next, note that $\phi\left[C\right]=\left(\phi^{-1}\right)^{-1}\left[C\right]$
which is Borel measurable since $C$ is closed (and so Borel measurable)
and $\phi^{-1}$ is a continuous (and hence Borel measurable) function
by Proposition \ref{prop:strictlyIncreasingContinuousFunctionInverse}.
Finally, consider the image $\phi\left[\left[0,1\right]\setminus C\right]$.
We know from Proposition \ref{prop:cantorFunctionConstantOutside}that
there exists a countable collection of disjoint open intervals $\left\{ O_{j}\right\} _{j\in\N}\subset\left[0,1\right]$
such that $\bigcup_{j\in\N}O_{j}=\left[0,1\right]\setminus C$ such
that $\psi_{C}\left(x\right)=c_{j}$ for any $x\in O_{j}$. Then,
\begin{align*}
\phi\left[\left[0,1\right]\setminus C\right] & =\phi\left[\bigcup_{j\in\N}O_{j}\right]\\
 & =\bigcup_{j\in\N}\phi\left[O_{j}\right]\\
 & =\bigcup_{j\in\N}O_{j}+c_{j}
\end{align*}
where each $O_{j}+c_{j}$ is disjoint since $\phi$ is a bijection
and so its image preserves intersections. Then,
\begin{align*}
\lambda\left(\phi\left[\left[0,1\right]\setminus C\right]\right) & =\lambda\left(\bigcup_{j\in\N}O_{j}+c_{j}\right)\\
 & =\sum_{i=1}^{\infty}\lambda\left(O_{j}\right)\\
 & =\lambda\left(\bigcup_{j\in\N}O_{j}\right)\\
 & =\lambda\left(\left[0,1\right]\setminus C\right)\\
 & =1
\end{align*}
where in the inequality we used countable additivity and translation
invariance. Then, by additivity, finiteness of $\lambda$ on bounded
intervasls, and the fact that the image of a bijective function preserves
set differences-
\begin{align*}
1=\lambda\left(\phi\left[\left[0,1\right]\setminus C\right]\right) & =\lambda\left(\phi\left[\left[0,1\right]\right]\setminus\phi\left[C\right]\right)\\
 & =\lambda\left(\phi\left[\left[0,1\right]\right]\right)-\lambda\left(\phi\left[C\right]\right)\\
 & =\lambda\left(\left[0,2\right]\right)-\lambda\left(\phi\left[C\right]\right)\\
 & =2-\lambda\left(\phi\left[C\right]\right).
\end{align*}
Therefore 
\[
\lambda\left(\phi\left[C\right]\right)=1
\]
which completes the proof.
\end{proof}
\begin{thm}
\label{thm:continuousImageLebesgueMeasurable}There exists a measurable
set $A\in\mathcal{C}\left(\lambda^{*}\right)$ such that $\phi\left[A\right]\notin\mathcal{C}\left(\lambda^{*}\right).$
\end{thm}

\begin{proof}
Note that by Theorem \ref{thm:positiveMeasureNonMeasurable} and Proposition
\ref{prop:strictlyIncreasingCantorFunction}, $\phi\left[C\right]$
contains a set $E$ such that $E\notin\mathcal{C}\left(\lambda^{*}\right).$
Let $A=C\cap\phi^{-1}\left[E\right]$ and notice that $A$ is measurable
as it is a subset of a measure-zero set and since $\mathcal{C}\left(\lambda^{*}\right)$
is complete. Then, $\phi\left[A\right]=E$ which is not measurable.
\end{proof}
\begin{cor}
\label{cor:lebesgueNotBorel}There exists a set $A\in\mathcal{C}\left(\lambda^{*}\right)$
such that $A\notin\borel\left(\R\right)$.
\end{cor}

\begin{proof}
Let $A=C\cap\phi^{-1}\left[E\right]$ as in Theorem \ref{thm:continuousImageLebesgueMeasurable}.
Then $A\in\mathcal{C}\left(\lambda^{*}\right)$ but $\phi\left[A\right]\notin\mathcal{C}\left(\lambda^{*}\right).$
But recall that $\phi^{-1}$ is measurable since it is continuous
and $\phi\left[A\right]=\left(\phi^{-1}\right)^{-1}\left[A\right]$
and so if $A\in\borel\left(\R\right)$ then $\phi\left[A\right]\in\borel\left(\R\right)$
which would be a contradiction.
\end{proof}

\subsection{An iterative construction of the Cantor function}

\section{Types of generating class arguments\label{sec:genClassArgs}}

So far we have seen how a generating class argument, such as the one
formalized in Theorem \ref{thm:genericGeneratingClassArgument}, can
help establish the measurability of functions. In general, measurability
can be replaced by any property, opening up other avenues to use such
arguments. The abstract analogue of Theorem \ref{thm:genericGeneratingClassArgument}
goes as follows: Let $\mathcal{F}$ be a $\sigma-$algebra on $\X$.
We want to show that all sets in $\mathcal{F}$ enjoy some property
$\left(*\right)$. In order to prove that this is indeed the case
we
\begin{enumerate}
\item Find a subclass of sets $\mathcal{E}\subseteq\mathcal{F}$ that enjoys
property $\left(*\right)$ such that $\sigma\left(\mathcal{E}\right)=\mathcal{F}.$
\item Define $\mathcal{D}=\left\{ F\in\mathcal{F}\mid F\ \text{enjoys property }\text{\ensuremath{\left(*\right)}}\right\} $.
\item Observe that if $\mathcal{D}$ is a $\sigma-$algebra then
\[
\mathcal{E}\subseteq\mathcal{D}\subseteq\mathcal{F}\Longrightarrow\mathcal{F}=\sigma\left(\mathcal{E}\right)\subseteq\sigma\left(\mathcal{D}\right)=\mathcal{D}\subseteq\mathcal{F}\Longrightarrow\mathcal{F}=\mathcal{D}.
\]
\end{enumerate}
In this argument, we do not assume any structure on the special class
$\mathcal{E}.$ However, imposing structure can add power to generating
class arguments, since then we can drop the requirement that subclass
$\mathcal{D}$ of ``desirable'' sets be a $\sigma-$algebra. The
fact that we can do this is not \emph{a-priori }obvious, and the point
of this section is to develop the theory to show that this can be
done with different types of structural assumptions. The results of
this section form the backbone of standard techniques in measure theory,
and have important applications, one of which we show here.

\subsection{Dynkin's $\pi-\lambda$ Theorem}
\begin{defn}
\label{def:dynkinSystem}A collection of sets $\mathcal{D}\subseteq2^{\X}$
is called a \emph{Dynkin system }or a $\lambda-$system if

\begin{enumerate}[label=(\roman*),leftmargin=.1\linewidth,rightmargin=.4\linewidth]
	\item $ \X \in \mathcal{D} $
	\item $ A_1, A_2 \in \mathcal{D} \text{ s.t. } A_2 \subseteq A_1 \Longrightarrow A_2\setminus A_1 \in \mathcal{D} $
	\item $\{A_i\}_{i\in\N} \in \mathcal{D} \text{ s.t. } A_i \subseteq A_{i+1} \Longrightarrow \bigcup_{i\in\N} A_i \in \mathcal{D} $
\end{enumerate}
\end{defn}

\begin{prop}
\label{prop:dynkinSystemEquivDefn}A collection of sets $\mathcal{D\subseteq}2^{\mathcal{X}}$
is a $\lambda-$system if and only if

\begin{enumerate}[label=(\roman*'),leftmargin=.1\linewidth,rightmargin=.4\linewidth]
	\item $ \X \in \mathcal{D} $
	\item $ A \in \mathcal{D} \Longrightarrow A^C \in \mathcal{D} $
	\item $\{A_i\}_{i\in\N} \in \mathcal{D}$ s.t $A_i \cap A_j = \emptyset \Longrightarrow \bigcup_{i\in\N} A_i \in \mathcal{D}$
\end{enumerate}
\end{prop}

\begin{proof}
First suppose that $\mathcal{D}$ is a $\lambda-$system. Then, property
$(i')$ above is automatically satisfied. For property $(ii')$, observe
that $\text{(i),(ii})$ together imply $(ii')$ since $A^{C}=\X\setminus A.$
Next, let $\left\{ A_{i}\right\} _{i\in\N}\in\mathcal{D}$ be pairwise
disjoint and define 
\[
B_{n}:=\bigcup_{i=1}^{n}A_{i},
\]
observing that $B_{n}\subseteq B_{n+1}$ and that $\bigcup_{n\in\N}B_{n}=\bigcup_{i\in\N}A_{i}.$
Then $(iii')$ follows from (iii).

Conversely, assume that $\mathcal{D}$ satisfies $(i')-(iii')$ above.
Again, (i) follows from $(i')$. Next, pick $A_{1},A_{2}\in\mathcal{D}$
such that $A_{2}\subseteq A_{1}$. By $(ii')$ , $A_{1}^{C}\in\mathcal{D}$
and we know that$A_{1}^{C}\cap A_{2}=\emptyset$ and so by $\left(ii'\right),\left(iii'\right)$,
we have that
\[
A_{1}\setminus A_{2}=\left(A_{1}^{C}\cup A_{2}\right)^{C}\in\mathcal{D}
\]
which proves (ii). Finally, let $\left\{ A_{i}\right\} _{i\in\N}\in\mathcal{D}$
be an increasing sequence of sets (that is, $A_{i}\subseteq A_{i+1}$
for every $i\in\N$). By (ii), 
\[
B_{i}:=A_{i}\setminus A_{i-1}\in\mathcal{D}
\]
 and are pairwise disjoint. Applying $(iii')$ we have that
\[
\bigcup_{i\in\N}A_{i}=\bigcup_{i\in\N}B_{i}\in\mathcal{D}
\]
completing the proof.
\end{proof}
It is easy to see that a $\sigma-$algebra is a $\lambda-$system,
and arguments akin to those in Proposition \ref{prop:ringGeneratedByClassIsRing}
will prove that
\[
\lambda\left(\mathcal{E}\right):=\bigcap\left\{ \mathcal{A}\subseteq2^{\X}\text{ is a }\lambda-\text{system}\mid\mathcal{E}\subseteq\mathcal{A}\right\} 
\]
is itself a $\lambda-$system, and these two facts together show that
\[
\mathcal{E}\subseteq\lambda\left(\mathcal{E}\right)\subseteq\sigma\left(\mathcal{E}\right).
\]

\begin{defn}
\label{def:piSystem}A collection of sets $\mathcal{E}\subseteq2^{\X}$
is called a $\pi-$system if it is closed under finite intersections
i.e. if $A,B\in\mathcal{E}$ then $A\cap B\in\mathcal{E}$.
\end{defn}

Properties of $\pi-$systems and $\lambda-$systems can be combined
to yield a $\sigma-$algebra.
\begin{lem}
\label{lem:piLambdaIsSigma}A collection of sets $\mathcal{F}\subseteq2^{\X}$
is a $\sigma-$algebra if and only if it is both a $\pi-$system and
a $\lambda-$system.
\end{lem}

\begin{proof}
Clearly, if $\mathcal{F}$ is a $\sigma-$algebra, then it is both
a $\pi-$system and a $\lambda-$system. To see the converse, note
that Proposition \ref{prop:dynkinSystemEquivDefn} gives us (i) containing
$\X$ and (ii) closure under complements for free. Thus we need to
prove that for any countable (not necessarily disjoint) collection
of sets $\left\{ E_{i}\right\} _{i\in\N}\in\mathcal{E}$
\[
\bigcup_{i\in\N}E_{i}\in\mathcal{F}.
\]
As we usually do, we shall look at the ``disjointification''
\begin{align*}
F_{i} & :=E_{i}\setminus\bigcup_{j=1}^{i-1}E_{j}\\
 & =E_{i}\cap\left(\bigcap_{j=1}^{i-1}E_{j}^{C}\right)
\end{align*}
and observe that closure under finite intersections (from being a
$\pi-$system) and complements implies that $F_{i}\in\mathcal{F}$.
Moreover, the $F_{i}$ are pairwise disjoint, and so by closure under
countable \emph{disjoint }unions, we have that
\[
\bigcup_{i\in\N}E_{i}=\bigcup_{i\in\N}F_{i}\in\mathcal{F}
\]
which finishes the proof.
\end{proof}
\begin{thm}[Dynkin's $\pi-\lambda$ Theorem]
\label{thm:piLambdaThm}Let $\mathcal{E}\subseteq2^{\X}$ be a $\pi-$system.
Then
\[
\lambda\left(\mathcal{E}\right)=\sigma\left(\mathcal{E}\right).
\]
\end{thm}

\begin{proof}
We had shown earlier that $\lambda\left(\mathcal{E}\right)\subseteq\sigma\left(\mathcal{E}\right)$
and so we now we use the additional hypothesis of closure under finite
intersections to show the reverse inclusion. By Lemma \ref{lem:piLambdaIsSigma},
we only need to show that $\lambda\left(\mathcal{E}\right)$ inherits
closure under finite intersections from $\mathcal{E}$, since then
$\lambda\left(\mathcal{E}\right)$ would be a $\sigma-$algebra that
contains $\mathcal{E}$ and so $\sigma\left(\mathcal{E}\right)\subseteq\lambda\left(\mathcal{E}\right)$.

In order to show that $\lambda\left(\mathcal{E}\right)$ is indeed
closed under finite intersections, define for every $B\in\lambda\left(\mathcal{E}\right)$
\[
\mathcal{D}_{B}:=\left\{ A\subseteq\X\mid A\cap B\in\lambda\left(\mathcal{E}\right)\right\} .
\]
It turns out that $\mathcal{D}_{B}$ is a $\lambda-$system for every
$B\in\lambda\left(\mathcal{E}\right)$. To see this, note that $\X\in\mathcal{D}_{B}$
since $\X\cap B=B\in\lambda\left(\mathcal{E}\right).$ Moreover, for
any sets $A_{1},A_{2}\in\mathcal{D}_{B}$ such that $A_{2}\subseteq A_{1}$
\[
\left(A_{1}\setminus A_{2}\right)\cap B=A_{1}\cap A_{2}^{C}\cap B=A_{1}\cap B\setminus A_{2}=A_{1}\cap B\setminus A_{2}\cap B\in\lambda\left(\mathcal{E}\right)
\]
 since $A_{1}\cap B,A_{2}\cap B\in\lambda\left(\mathcal{E}\right)$
and $A_{2}\cap B\subseteq A_{1}\cap B$ (see property (ii) in Definition
\ref{def:dynkinSystem}). This shows that $A_{1}\setminus A_{2}\in\mathcal{D}_{B}.$
Finally, let $\left\{ A_{i}\right\} _{i\in\N}\in\mathcal{D}_{B}$
be an increasing sequence of sets, then
\[
A_{i}\cap B\subseteq A_{i+1}\cap B
\]
and so using property (iii) of $\lambda-$systems 
\[
\left(\bigcup_{i\in\N}A_{i}\right)\cap B=\bigcup_{i\in\N}\left(A_{i}\cap B\right)\in\lambda\left(\mathcal{E}\right)
\]
which shows that $\bigcup_{i\in\N}A_{i}\in\mathcal{D}_{B}$, proving
our claim that $\mathcal{D}_{B}$ is a $\lambda-$system for any $B\in\lambda\left(\mathcal{E}\right).$
In particular, for any $E\in\mathcal{E}$, $\mathcal{D}_{E}$ is a
$\lambda-$system such that $\mathcal{E}\subseteq\mathcal{D}_{E}$
because of $\mathcal{E}$ is closed under finite intersections. Since
$\mathcal{D}_{E}$ is a $\lambda-$system,
\[
\lambda\left(\mathcal{E}\right)\subseteq\mathcal{D}_{E}
\]
for every $E\in\mathcal{E}$. Now we have proved that for any $B\in\lambda\left(\mathcal{E}\right)$
and any $E\in\mathcal{E}$, their intersection $A\cap E\in\lambda\left(\mathcal{E}\right)$,
which seems like it is just a little bit short of what we need. But
notice that this means $\mathcal{E}\subseteq\mathcal{D}_{B}$ for
every $B\in\lambda\left(\mathcal{E}\right)$, which implies that 
\[
\lambda\left(\mathcal{E}\right)\subseteq\mathcal{D}_{B}
\]
for every $B\in\lambda\left(\mathcal{E}\right).$ Thus we have proved
that for any $A,B\in\mathcal{\lambda\left(E\right)},A\cap B\in\mathcal{\lambda\left(E\right)}$
which completes the proof.
\end{proof}
\begin{cor}
\label{cor:piLambdaGeneratingClassArg}Let $\mathcal{D}\subseteq2^{\X}$
be a $\lambda-$system and let $\mathcal{E}\subseteq\mathcal{D}$
be a $\pi-$system. Then
\[
\sigma\left(\mathcal{E}\right)\subseteq\mathcal{D}.
\]
\end{cor}

\begin{proof}
By the $\pi-\lambda$ theorem
\[
\sigma\left(\mathcal{E}\right)=\lambda\left(\mathcal{E}\right)\subseteq\mathcal{D}
\]
since $\mathcal{D}$ is a $\lambda-$system that contains $\mathcal{E}$.
\end{proof}
This corollary is useful because it adds structural assumptions on
the generating class (by requiring that it be a $\pi-$system), which
allows us to loosen the restriction that $\mathcal{D}$ be a $\sigma-$algebra
like we used to assume in standard generating class arguments. This
technique is powerful, and has important applications such as finding
sufficient conditions to show that two measures on a $\sigma-$algebra
are equal without having to explicitly check that equality holds on
the entire $\sigma-$algebra.
\begin{thm}[Uniqueness of Measures]
 \label{thm:uniquenessMeasures}Let $\left(\X,\mathcal{F}\right)$
be a measurable space and let $\mu,\nu$ be measures on $\mathcal{F}$.
If there exists a $\pi-$system $\mathcal{E\subseteq}2^{\X}$ such
that $\sigma\left(\mathcal{E}\right)=\mathcal{F}$ and 
\[
\mu\left(E\right)=\nu\left(E\right)
\]
for every $E\in\mathcal{E}$, and there exists increasing sequence
$\left\{ E_{i}\right\} _{i\in\N}\in\mathcal{E}$ such that $\X=\bigcup_{i\in\N}E_{i}$
and
\[
\mu\left(E_{i}\right)=\nu\left(E_{i}\right)<\infty
\]
for all $i\in\N$ then
\[
\mu\left(F\right)=\nu\left(F\right)
\]
for every $F\in\mathcal{F}$.
\end{thm}

\begin{proof}
Define
\[
\mathcal{D}_{i}:=\left\{ F\in\mathcal{F}\mid\mu\left(F\cap E_{i}\right)=\nu\left(F\cap E_{i}\right)\right\} 
\]
and note that $\X\in\mathcal{D}_{i}$ for every $i\in\N$ since $\mu\left(E_{i}\right)=\nu\left(E_{i}\right)$.
Next, let $F\in\mathcal{D}_{i}$ and observe that 
\begin{align*}
\mu\left(F^{C}\cap E_{i}\right) & =\mu\left(E_{i}\setminus\left(E_{i}\cap F\right)\right)\\
 & =\mu\left(E_{i}\right)-\mu\left(E_{i}\cap F\right)\\
 & =\nu\left(E_{i}\right)-\nu\left(E_{i}\cap F\right)\\
 & =\nu\left(E_{i}\setminus\left(E_{i}\cap F\right)\right)\\
 & =\nu\left(F^{C}\cap E_{i}\right)
\end{align*}
where the second and fourth equalities follow from (finite) additivity
of measures and the fact that $\mu\left(E_{i}\cap F\right)\leq\mu\left(E_{i}\right)<\infty$.
This proves that for any $F\in\mathcal{D}_{i}$, $F^{C}\in\mathcal{D}_{i}$
for every $i\in\N$. Finally, let $\left\{ F_{j}\right\} _{j\in\N}$be
a pairwise disjoint collection of sets in $\mathcal{D}_{i}$. Then
\begin{align*}
\mu\left(\left(\bigcup_{k\in\N}F_{j}\right)\cap E_{i}\right) & =\mu\left(\bigcup_{j\in\N}\left(F_{j}\cap E_{i}\right)\right)\\
 & =\sum_{j=1}^{\infty}\mu\left(F_{j}\cap E_{i}\right)\\
 & =\sum_{j=1}^{\infty}\nu\left(F_{j}\cap E_{i}\right)\\
 & =\nu\left(\bigcup_{j\in\N}\left(F_{j}\cap E_{i}\right)\right)\\
 & =\nu\left(\left(\bigcup_{k\in\N}F_{j}\right)\cap E_{i}\right)
\end{align*}
where the second and fourth equality follow from countable additivity
and the third equality follows from the uniqueness of limits. This
proves that $\bigcup_{j\in\N}F_{j}\in\mathcal{D}_{i}$ for every $i\in\N,$
which shows (through Proposition \ref{prop:dynkinSystemEquivDefn})
that every $\mathcal{D}_{i}$ is a $\lambda-$system.

Now note that since $\mathcal{E}$ is a $\pi-$system (and so closed
under finite intersections), we have that 
\[
\mathcal{E}\subseteq\mathcal{D}_{i}
\]
for every $i\in\N$. By Corollary \ref{cor:piLambdaGeneratingClassArg},
we have that 
\[
\mathcal{F}=\sigma\left(\mathcal{E}\right)\subseteq\mathcal{D}_{i}
\]
for every $i\in\N$. This means that for any $F\in\mathcal{F}$ and
any $i\in\N$
\[
\mu\left(F\cap E_{i}\right)=\nu\left(F\cap E_{i}\right).
\]
Let $F\in\mathcal{F}$ be arbitrary and observe that since $\left\{ E_{i}\right\} _{i\in\N}\nearrow\X$,
$\left\{ E_{i}\cap F\right\} _{i\in\N}\nearrow\X\cap F=F.$ Then,
by the \hyperref[prop:measureProperties]{continuity of measures},
\[
\mu\left(F\right)=\lim_{i\to\infty}\mu\left(F\cap E_{i}\right)=\lim_{i\to\infty}\nu\left(F\cap E_{i}\right)=\nu\left(F\right),
\]
again by the uniqueness of limits.
\end{proof}

\subsection{Monotone class theorem}
\begin{defn}
\label{def:monotoneClass} A collection of sets $\mathcal{M\subseteq}2^{\X}$
is called a \emph{monotone class }if

\begin{enumerate}[label=(\roman*),leftmargin=.1\linewidth,rightmargin=.4\linewidth]
	\item For an increasing sequence of sets $ \{A_i\}_{i\in\N} \in \mathcal{M}: \bigcup_{i\in\N}A_i \in \mathcal{M} $
	\item For a decreasing sequence of sets $ \{A_i\}_{i\in\N} \in \mathcal{M}: \bigcap_{i\in\N}A_i \in \mathcal{M} $
\end{enumerate}
\end{defn}

Note that every $\sigma-$algebra is already a monotone class. However,
we can make an even stronger claim: every $\lambda-$system is monotone
class.
\begin{prop}
\label{prop:dynkinSystemIsMonotoneClass} Let $\mathcal{D}\subseteq2^{\X}$
be a $\lambda-$system. Then $\mathcal{D}$ is a monotone class
\end{prop}

\begin{proof}
Property (i) above follows directly from Definition \ref{def:dynkinSystem}.
To see property (ii), let $\left\{ A_{i}\right\} _{i\in\N}\in\mathcal{D}$
be a decreasing sequence of sets and observe that 
\[
B_{i}=\X\setminus A_{i}\in\mathcal{D}
\]
is an inccreasing sequence of sets such that
\[
\bigcup_{i\in\N}B_{i}\in\mathcal{D}.
\]
However, 
\begin{align*}
\bigcap_{i\in\N}A_{i} & =\left(\bigcup_{i\in\N}A_{i}^{C}\right)^{C}\\
 & =\left(\bigcup_{i\in\N}B_{i}\right)^{C}\\
 & =\X\setminus\left(\bigcup_{i\in\N}B_{i}\right)
\end{align*}
which is in $\mathcal{D}$. This completes the proof.
\end{proof}
As usual, we define the monotone class generated by class of sets
$\mathcal{A}$ as 
\[
\mathscr{M}\left(\mathcal{A}\right):=\bigcap\left\{ \mathcal{M}\subseteq2^{\X}\text{ is a monotone class}\mid\mathcal{A}\subseteq\mathcal{M}\right\} 
\]
which is itself a monotone class by the usual arguments. Again, since
$\sigma-$algebras and $\lambda-$systems are monotone classes, we
have that
\[
\mathcal{A}\subseteq\mathscr{M}\left(\mathcal{A}\right)\subseteq\lambda\left(\mathcal{A}\right)\subseteq\sigma\left(\mathcal{A}\right)
\]
where we also use the fact that a $\sigma-$algebra is a $\lambda-$system.
Finally, note that an \hyperref[def:algebra]{\emph{algebra} of sets}
$\mathcal{A}$ is also a $\pi-$system and so, by the $\pi-\lambda$
theorem, 
\[
\sigma\left(\mathcal{A}\right)=\lambda\left(\mathcal{A}\right).
\]

\begin{thm}[Monotone class theorem]
\label{thm:monotoneClassThm} Let $\mathcal{A}\subseteq2^{\X}$ be
an algebra of sets. Then
\[
\mathscr{M}\left(\mathcal{A}\right)=\sigma\left(\mathcal{A}\right).
\]
\end{thm}

\begin{proof}
We have shown that $\mathscr{M}\left(\mathcal{A}\right)\subseteq\sigma\left(\mathcal{A}\right)$
and so we need to show the reverse inclusion. Since $\mathcal{A}$
is also a $\pi-$system, this is equivalent to showing that $\lambda\left(\mathcal{A}\right)\subseteq\mathscr{M}\left(\mathcal{A}\right)$.
To this end, it is sufficient to show that $\mathscr{M}\left(\mathcal{A}\right)$
is a $\lambda-$system. Note that $\X\in\mathscr{M}\left(\mathcal{A}\right)$
since $\X\in\mathcal{A}$. Further, $\mathscr{M}\left(\mathcal{A}\right)$
is closed under limits of increasing sequences of sets since it's
a monotone class. Thus we need to prove that for any $A_{1},A_{2}\in\mathscr{M}\left(\mathcal{A}\right)$
such that $A_{2}\subseteq A_{1}$ we have that $A_{1}\setminus A_{2}\in\mathscr{M}\mathcal{\left(A\right)}.$

For any set $E\in\mathscr{M}\left(\mathcal{A}\right)$, define
\[
M_{E}:=\left\{ F\subseteq\mathscr{M}\left(\mathcal{A}\right)\mid E\setminus F,F\setminus E\in\mathscr{M}\left(\mathcal{A}\right)\right\} 
\]
and note that for an increasing collection of sets $\left\{ F_{i}\right\} _{i\in\N}\in M_{E}$
\begin{align*}
E\setminus\bigcup_{i\in\N}F_{i} & =E\cap\left(\bigcap_{i\in\N}F_{i}^{C}\right)\\
 & =\bigcap_{i\in\N}\left(E\cap F_{i}^{C}\right)\\
 & =\bigcap_{i\in\N}E\setminus F_{i}\in M_{E}
\end{align*}
since $E\setminus F_{i}\supseteq E\setminus F_{i+1}$ and $E\setminus F_{i}\in\mathscr{M}\left(\mathcal{A}\right)$
for all $i\in\N$. A similar aragument shows that $\bigcup_{i\in\N}F_{i}\setminus E\in\mathscr{M}\left(\mathcal{A}\right)$
which shows that $\bigcup_{i\in\N}F_{i}\in M_{E}.$ We can apply the
same argument to show that for a decreasing sequence of sets $\left\{ F_{i}\right\} _{i\in\N}\in M_{E}$,
the intersection $\bigcap_{i\in\N}F_{i}\in M_{E}$. This proves that
$M_{E}$ is in fact a monotone class, and since $\mathcal{A}$ is
an algebra (and so it's closed under set differences), $\mathcal{A}\subseteq M_{E}$
which implies that $\mathscr{M}\left(\mathcal{A}\right)\subseteq M_{E}$
for any $E\in\mathscr{M}\left(\mathcal{A}\right).$ In other words,
this proves that for any $E,F\in\mathscr{M}\left(\mathcal{A}\right):$
\[
E\setminus F,F\setminus E\in\mathscr{M}\left(\mathcal{A}\right)
\]
 which completes the proof.
\end{proof}
Note how we used our previous work with the $\pi-\lambda$ theorem
to establish the monotone class theorem; it turns out that the monotone
class also implies theorem implies the $\pi-\lambda$ theorem, and
so these two theorems are in fact equivalent. This should not be particularly
surprising to you at this juncture since they are both generating
class arguments which make slightly different structural assumptions
to yield the same result. We can present these assumptions in a concise
and organized fashion, using the notation defined in the abstract
generating class argument described at the beginning of this section.
In Table \ref{tab:typesGenClassArg} $\mathcal{E}$ is the generating
class which satisfies the desired property $\left(*\right)$ and $\mathcal{D}$
is all the sets in the $\sigma-$algebra $\mathcal{F}$ which satisfies
property $\left(*\right)$. In the generic argument, we put very strong
structural assumptions on $\mathcal{D}$ and none on $\mathcal{E}$.
In the $\pi-\lambda$ case, we weaken the assumptions on $\mathcal{D}$
in exchange for imposing (weak) assumptions on $\mathcal{E}$. Finally,
in the monotone class approach, we impose a strong structural assumption
on $\mathcal{E}$, whereas $\mathcal{D}$ has the minimal structure
of a monotone class. The type of argument one uses in practice depends
on context, and in many situations one can pick to use either Dynkin's
theorem or the monotone class theorem.

\begin{table}[H]
\caption{\label{tab:typesGenClassArg}Types of generating class arguments}

\centering{}%
\begin{tabular}{ccc}
\hline 
Name & Structure of $\mathcal{E}$ & Structure of $\mathcal{D}$\tabularnewline
\hline 
\hline 
Generic & No structure & $\sigma-$algebra\tabularnewline
Dynkin's theorem & $\pi-$system & $\lambda-$system\tabularnewline
Monotone class theorem & Algebra  & Monotone class\tabularnewline
\hline 
\end{tabular}
\end{table}


\subsection{Generating classes of functions}
\begin{defn}
\label{def:lambdaSpaceOfFunctions}A set $\mathcal{H}$ of bounded
real functions on $\X$ is called a $\lambda-$\emph{space of functions
}if

\begin{enumerate}[label=(\roman*),leftmargin=.1\linewidth,rightmargin=.4\linewidth]
	\item The constant function $\indicate_{\X} \in \mathcal{H} $
	\item $\mathcal{H}$ is a vector space over $\R$
	\item If $h_n \in \mathcal{H} $ such that $h_n \leq h_{n+1} $ pointwise and
		\[
					\lim_{n\to\infty} h_n = h
		\]
		 is well defined and bounded then $h \in \mathcal{H}$
\end{enumerate}
\end{defn}

$\lambda-$spaces of functions are named so because they generalize
$\lambda-$systems of sets, as we show in the following result.
\begin{prop}
\label{prop:lambdaFuncsGeneralizeLambdaSystems} If $\mathcal{H}$
is a $\lambda-$space of functions on $\X$ then
\[
\mathcal{D:=}\left\{ A\subseteq\X\mid\indicate_{A}\in\mathcal{H}\right\} 
\]
is a $\lambda-$system.
\end{prop}

\begin{proof}
Note that since $\indicate_{\X}\in\mathcal{H}$ we have that $\X\in\mathcal{D}.$
Next, assume that $A_{1},A_{2}\in\mathcal{D}$ such that $A_{2}\subseteq A_{1}$
and observe that
\begin{align*}
\indicate_{A_{1}\setminus A_{2}} & =\indicate_{A_{1}\cap A_{2}^{C}}\\
 & =\indicate_{A_{1}}\left(1-\indicate_{A_{2}}\right)\\
 & =\indicate_{A_{1}}-\indicate_{A_{2}}\in\mathcal{H}
\end{align*}
where the second and third equalities follows from Fact \ref{fact:indicatorFunctionsFiniteOperations},
and the inclusion follows from the fact that $\mathcal{H}$ is closed
under linear combinations. This prove that $A_{1}\setminus A_{2}\in\mathcal{H}$.
Fiinally, let $\left\{ A_{i}\right\} _{i\in\N}$ be an increasing
sequence of sets in $\mathcal{D}$ and observe that
\begin{align*}
\indicate_{\bigcup_{i\in\N}A_{i}} & =\sup_{i\in\N}\indicate_{A_{i}}\\
 & =\lim_{n\to\infty}\indicate_{A_{i}}\in\mathcal{H}
\end{align*}
where the first equality follows from \ref{prop:indicatorFunctionsArbitraryOperations}
and the second equality is due to the fact that $A\subseteq B\Longrightarrow\indicate_{A}\leq\indicate_{B}$,
and so $\indicate_{A_{i}}$ converges pointwise to its supremum. This
proves that $\bigcup_{i\in\N}A_{i}\in\mathcal{D}$ which implies that
$\mathcal{D}$ is a $\lambda-$system.
\end{proof}
In the next theorem and elsewhere, let $\mathcal{M}_{\text{bdd}}\left(\X,\mathcal{F}\right)$
and $\mathcal{M}_{\text{bdd}}^{+}\left(\X,\mathcal{F}\right)$ denote
the set of all bounded real-valued Borel-measurable functions on $\left(\mathcal{X},\mathcal{F}\right),$and
the set of all non-negative and bounded real-valued Borel-measurable
functions on $\left(\mathcal{X},\mathcal{F}\right)$ respectively.
\begin{thm}[$\pi-\lambda$ theorem for functions]
\label{thm:piLambdaThmFunctions} Let $\mathcal{G}$ be a $\pi-$system
of sets and let $\mathcal{H}$ be a $\lambda-$space of functions
on $\X$. If
\[
\left\{ \indicate_{A}\mid A\in\mathcal{G}\right\} \subseteq\mathcal{H}
\]
then $\mathcal{M}_{\textnormal{bdd}}\left(\X,\sigma\left(\mathcal{G}\right)\right)\subseteq\mathcal{H}.$
\end{thm}

\begin{proof}
Define $\mathcal{D:=}\left\{ D\subseteq\X\mid\indicate_{D}\in\mathcal{H}\right\} $.
Then, by Proposition \ref{prop:lambdaFuncsGeneralizeLambdaSystems},
$\mathcal{D}$ is a $\lambda-$system such that $\mathcal{G}\subseteq\mathcal{D}.$
By the \hyperref[cor:piLambdaGeneratingClassArg]{$\pi-\lambda$ theorem for sets},
$\sigma\left(\mathcal{G}\right)\subseteq\mathcal{D}$. This means
that the indicator functions of sets in $\sigma\left(\mathcal{G}\right)$
is contained in $\mathcal{H}$.

Next, let $f\in\mathcal{M}_{\text{bdd}}^{+}\left(\X,\sigma\left(\mathcal{G}\right)\right)$
be arbitrary. Since $f\in\mathcal{M}^{+}\left(\X,\sigma\left(\mathcal{G}\right)\right),$
by Proposition \ref{prop:simpleFunctionMonotoneConvergence} there
exists an increasing sequence of simple functions $\left\{ s_{n}\right\} _{n\in\N}\in\mathcal{M}^{+}\left(\X,\sigma\left(\mathcal{G}\right)\right)$
such that 
\[
f=\lim_{n\to\infty}s_{n}.
\]
Since, $f$ is bounded, $s_{n}\leq f$ are all bounded. Moreover,
since every $s_{n}$ is a measurable simple function, it's a finite
linear combination of indicator functions of sets in $\sigma\left(\mathcal{G}\right)$.
Therefore, $\left\{ s_{n}\right\} _{n\in\N}\in\mathcal{H}$ and since
$f$ is the monotone limit of $s_{n}$, $f\in\mathcal{H}$.

Finally, assume that $f\in\mathcal{M}_{\textnormal{bdd}}\left(\X,\sigma\left(\mathcal{G}\right)\right)$
and note that $f^{+},f^{-}\in\mathcal{H}$ by Proposition \ref{prop:minMaxMeasurable}
and the last step, and so $f=f^{+}-f^{-}\in\mathcal{H}$.
\end{proof}




\chapter{Integration\label{chap:integration}}

\section{Constructing the Lebesgue integral}
\begin{thm}
\label{thm:existenceUniquenessLebesgueIntegral}Let $\left(\X,\mathcal{F}\right)$
be measurable space. For any measure $\mu$ on $\mathcal{F}$, there
exists a unique linear functional
\[
\bar{\mu}:\mathcal{\mathcal{M}^{+}}\left(\X,\mathcal{F}\right)\longrightarrow\left[0,\infty\right]
\]
that satisfies the following properties

\begin{enumerate}[label=(\roman*),leftmargin=.1\linewidth,rightmargin=.4\linewidth]
	\item $ \bar{\mu}\left(\indicate_A\right) = \mu\left(A\right) $ for any $ A \in \mathcal{F} $
	\item \textnormal{(Linearity)}  For any $f,g\in \mathcal{\mathcal{M}^{+}}\left(\X,\mathcal{F}\right)$ and any $\alpha,\beta \geq 0 $ 
	\[
				\bar{\mu}\left(\alpha f + \beta g\right) = \alpha \bar{\mu}\left(f\right) + \beta \bar{\mu}\left(g\right)                
	\]
	\item \textnormal{(Monotone convergence)} For a sequence of increasing functions $ \left\{f_n\right\}_{n\in\N} \in \mathcal{\mathcal{M}^{+}}\left(\X,\mathcal{F}\right)$ 
	\[
				\bar{\mu}\left(\lim_{n\to\infty}f_n\right) = \lim_{n\to\infty}\bar{\mu}\left(f_n\right)
	\]
\end{enumerate}
\end{thm}

We can prove Theorem \ref{thm:existenceUniquenessLebesgueIntegral}
constructively, in a similar fashion to how we proved the existence
of the Lebesgue measure in Chapter \ref{chap:measures}. In this spirit,
we shall define a functional on the non-negative measurable simple
functions and extend the domain of this functional to more complicated
function spaces.
\begin{defn}
\label{def:simpleFuncIntegral}For any measurable simple function
$s\in\mathcal{M}_{\textnormal{sim}}\left(\X,\F\right)$ with the standard
representation
\[
s=\sum_{i=1}^{I}\alpha_{i}\indicate_{A_{i}}
\]
define, for any measure $\mu$ on $\F$, the functional $\bar{\mu}_{0}$
as
\[
\bar{\mu}_{0}\left(s\right):=\sum_{i=1}^{I}\alpha_{i}\mu\left(A_{i}\right).
\]
\end{defn}

Immediately, we can see that our proto-integral $\bar{\mu}_{0}$ behaves
quite nicely: it is always non-negative and since indicator functions
are special cases of simple functions, we have that 
\[
\bar{\mu}_{0}\left(\indicate_{A}\right)=\mu\left(A\right)
\]
for any $A\in\F$. Moreover, our functional satisfies the linearity
property on the space of simple measurable functions. To see this
we shall need the following lemma which solves a minor technical issue
that arises due to our definition of $\bar{\mu}$ relying on the standard
representation of simple functions.
\begin{lem}
\label{lem:lebIntDiffRep}For any non-negative measurable simple function
$s\in\mathcal{M}_{\textnormal{sim}}\left(\X,\F\right)$ with a standard
representation
\[
s=\sum_{i=1}^{I}\alpha_{i}\indicate_{A_{i}}
\]
and another representation
\[
s=\sum_{j=1}^{J}\beta_{i}\indicate_{B_{j}}
\]
where $\left\{ B_{j}\right\} _{j=1}^{J}$ is a partition of $\X$
we have 
\[
\bar{\mu}_{0}\left(s\right):=\sum_{i=1}^{I}\alpha_{i}\mu\left(A_{i}\right)=\sum_{j=1}^{J}\beta_{j}\mu\left(B_{j}\right).
\]
where $\bar{\mu}_{0}$ is the functional derived from a measure $\mu$
as in Definition \ref{def:simpleFuncIntegral}.
\end{lem}

\begin{proof}
Note that both $A_{i}$ and $B_{j}$ partition $\X$ and so observe
that
\begin{align*}
\sum_{i=1}^{I}\alpha_{i}\mu\left(A_{i}\right) & =\sum_{i=1}^{I}\alpha_{i}\mu\left(\bigcup_{j=1}^{J}\left(A_{i}\cap B_{j}\right)\right)\\
 & =\sum_{i=1}^{I}\sum_{j=1}^{J}\alpha_{i}\mu\left(A_{i}\cap B_{j}\right)
\end{align*}
where the last equality follows by finite addivity. Similarly,

\[
\sum_{j=1}^{J}\beta_{j}\mu\left(B_{j}\right)=\sum_{i=1}^{I}\sum_{j=1}^{J}\beta_{j}\mu\left(A_{i}\cap B_{j}\right).
\]
Now observe that since $\sum_{i=1}^{I}\alpha_{i}\indicate_{A_{i}}=\sum_{j=1}^{J}\beta_{i}\indicate_{B_{j}}$
we know that $\alpha_{i}=\beta_{j}$ if $A_{i}\cap B_{j}\neq\emptyset$;
conversely, if $A_{i}\cap B_{j}=\emptyset$ then $\mu\left(A_{i}\cap B_{j}\right)=0$
and so we have that 
\[
\sum_{i=1}^{I}\alpha_{i}\mu\left(A_{i}\right)=\sum_{i=1}^{I}\sum_{j=1}^{J}\alpha_{i}\mu\left(A_{i}\cap B_{j}\right)=\sum_{i=1}^{I}\sum_{j=1}^{J}\beta_{j}\mu\left(A_{i}\cap B_{j}\right)=\sum_{j=1}^{J}\beta_{j}\mu\left(B_{j}\right).
\]
\end{proof}
\begin{prop}
\label{prop:lebIntLinearitySimpleFunc}Let $s,t\in\mathcal{M}_{\textnormal{sim}}\left(\X,\F\right)$
and let $\mu$be a measure on $\mathcal{F}$. Then, for any $\alpha,\beta\geq0$,
we have that
\[
\bar{\mu}_{0}\left(\alpha s+\beta t\right)=\alpha\bar{\mu}_{0}\left(s\right)+\beta\bar{\mu}_{0}\left(t\right).
\]
\end{prop}

\begin{proof}
It is sufficient to prove that $\bar{\mu}_{0}\left(\alpha s\right)=\alpha\bar{\mu}_{0}\left(s\right)$
and $\bar{\mu}_{0}\left(s+t\right)=\bar{\mu}_{0}\left(s\right)+\bar{\mu}_{0}\left(t\right).$
To show the first equality, simply notice that if $s$ is given by
the standard representation 
\[
s=\sum_{i=1}^{I}a_{i}\indicate_{A_{i}}
\]
then the standard representation of $\alpha s$ when $\alpha>0$ is
simply
\[
\alpha s=\sum_{i=1}^{I}\alpha a_{i}\indicate_{A_{i}}
\]
and so 
\begin{align*}
\bar{\mu}_{0}\left(\alpha s\right) & =\sum_{i=1}^{I}\alpha a_{i}\mu\left(A_{i}\right)\\
 & =\alpha\sum_{i=1}^{I}a_{i}\mu\left(A_{i}\right)\\
 & =\alpha\bar{\mu}_{0}\left(s\right).
\end{align*}
When $\alpha=0$, $\alpha s=\indicate_{\emptyset}$ and so $\bar{\mu}_{0}\left(\alpha s\right)=\mu\left(\emptyset\right)=0=\alpha\bar{\mu}_{0}\left(s\right).$

In order to prove the second equality, observe that if the standard
representations of $s$ and $t$ are given by
\begin{align*}
s & =\sum_{i=1}^{I}a_{i}\indicate_{A_{i}}\\
t & =\sum_{j=1}^{J}b_{j}\indicate_{B_{j}}
\end{align*}
then
\begin{align*}
\bar{\mu}_{0}\left(s+t\right) & =\bar{\mu}_{0}\left(\sum_{i=1}^{I}\sum_{j=1}^{J}\left(a_{i}+b_{j}\right)\indicate_{A_{i}\cap B_{j}}\right)\\
 & =\sum_{i=1}^{I}\sum_{j=1}^{J}\left(a_{i}+b_{j}\right)\mu\left(A_{i}\cap B_{j}\right)\\
 & =\sum_{i=1}^{I}a_{i}\sum_{j=1}^{J}\mu\left(A_{i}\cap B_{j}\right)+\sum_{j=1}^{J}b_{j}\sum_{i=1}^{I}\mu\left(A_{i}\cap B_{j}\right)\\
 & =\sum_{i=1}^{I}a_{i}\mu\left(A_{i}\right)+\sum_{j=1}^{J}b_{j}\mu\left(B_{j}\right)\\
 & =\bar{\mu}_{0}\left(s\right)+\bar{\mu}_{0}\left(t\right)
\end{align*}
where the first equality follows from Proposition \ref{prop:simpleFunctionsAddMultiply},
the second equality due to Lemma \ref{lem:lebIntDiffRep}, and the
fourth equality due to finite additivity of $\mu$. This completes
the proof.
\end{proof}
Linearity and non-negativity of $\bar{\mu}_{0}$ tells us that for
any simple measurable functions $f\leq g$, we have that
\[
\bar{\mu}_{0}\left(f\right)\leq\bar{\mu}_{0}\left(g\right).
\]
Indeed, we can decompose $g=\left(g-f\right)+f$ where $g-f$ and
$f$ are both non-negative simple functions and so
\[
\bar{\mu}_{0}\left(g\right)=\bar{\mu}_{0}\left(g-f\right)+\bar{\mu}_{0}\left(f\right)
\]
which by non-negativity proves our claim. In other words, our functional
$\bar{\mu}_{0}$ is an \emph{increasing }or \emph{monotone }functional.
With this final fact, we are now ready to extend $\bar{\mu}_{0}$
to the space of non-negative measurable functions.
\begin{proof}[Proof of Theorem \ref{thm:existenceUniquenessLebesgueIntegral}]
For any $f\in\nonnegMeasurableFunctions$, define 
\[
\lebInt{\mu}f:=\sup\left\{ \bar{\mu}_{0}\left(s\right)\mid s\in\mathcal{M}_{\textnormal{sim}}\left(\X,\F\right)\textnormal{ such that }s\leq f\right\} .
\]
We claim that $\bar{\mu}$ is the unique functional (for a given measure
$\mu$) that satisfies the properties described in the statement of
the theorem. To see this, we first have to show that the functional
defined above indeed satisfes the three requisite properties and then
show that it is the only such functional. Before we do this, note
that the set $\left\{ \bar{\mu}_{0}\left(s\right)\mid s\in\mathcal{M}_{\textnormal{sim}}\left(\X,\F\right)\text{s.t\ }s\leq f\right\} $
is always non-empty, thanks to Proposition \ref{prop:simpleFunctionMonotoneConvergence}.
Therefore the supremum is at least 0 since $\bar{\mu}_{0}\left(s\right)\geq0.$
Moreover, since for any non-negative measurable functions $f\leq g$,
$\left\{ \bar{\mu}_{0}\left(s\right)\mid s\in\mathcal{M}_{\textnormal{sim}}\left(\X,\F\right)\textnormal{ such that }s\leq f\right\} \subseteq\left\{ \bar{\mu}_{0}\left(s\right)\mid s\in\mathcal{M}_{\textnormal{sim}}\left(\X,\F\right)\textnormal{ such that }s\leq g\right\} ,$we
can conclude that $\lebInt{\mu}f\leq\lebInt{\mu}g.$ Thus we already
know that our functional $\bar{\mu}$ is both non-negative and monotone.
Next, to show property (i), observe that for any simple function $t\in\mathcal{M}_{\textnormal{sim}}\left(\X,\F\right)$
\begin{align*}
\lebInt{\mu}t & =\sup\left\{ \bar{\mu}_{0}\left(s\right)\mid s\in\mathcal{M}_{\textnormal{sim}}\left(\X,\F\right)\textnormal{ such that }s\leq t\right\} \\
 & =\bar{\mu}_{0}\left(t\right)
\end{align*}
where the second equality follows from the monotonicty of $\bar{\mu}_{0}$.
Property (i) then follows from letting $t=\indicate_{A}$ for any
set $A\in\F.$ Next, we prove property (iii) which is monotone convergence.

Let $\left\{ f_{n}\right\} _{n\in\N}\in\nonnegMeasurableFunctions$
an increasing sequence of measurable functions. Since we are working
with the extended non-negative real numbers, we know that 
\[
f:=\lim_{n\to\infty}f_{n}=\sup_{n\in\N}f_{n}\in\nonnegMeasurableFunctions
\]
by Proposition \ref{prop:supInfMeasurable}. Note that by the monotonicity
of $\bar{\mu},$ $\bar{\mu}\left(f_{n}\right)\leq\bar{\mu}\left(f\right)$
for all $n\in\N$ and so 
\[
\lim_{n\to\infty}\lebInt{\mu}{f_{n}}\leq\lebInt{\mu}f.
\]
To deduce the reverse inequality, let $s\leq f$ be a non-negative
measurable simple function, fix some $t\in\left(0,1\right)$ and define
$A_{n}=\left\{ x\in\X\mid f_{n}\left(x\right)\geq ts\left(x\right)\right\} $.
Since $f_{n}\leq f_{n+1}$, we have that $A_{n}\subseteq A_{n+1}$.
Moreover, for any $x\in\X$, if $f\left(x\right)>0$ then there is
some $\epsilon_{x}>0$ such that $f\left(x\right)-ts\left(x\right)=2\epsilon_{x}$
and so, by pointwise convergence, there exists some $n_{\epsilon_{x}}\in\N$
such that 
\begin{align*}
f_{n}\left(x\right)-ts\left(x\right) & =\underbrace{\left(f\left(x\right)-ts\left(x\right)\right)}_{2\epsilon_{x}}-\underbrace{\left(f\left(x\right)-f_{n}\left(x\right)\right)}_{\leq\epsilon_{x}}\geq\epsilon_{x}>0
\end{align*}
for all $n\geq n_{\epsilon_{x}},$ proving that $x\in A_{n}$ for
such $n.$ Conversely, if $f\left(x\right)=0$ then $f\left(x\right)=f_{n}\left(x\right)=ts\left(x\right)=0$
and so $x\in A_{n}$ for every $n\in\N$. Together, these two cases
show that $\bigcup_{n\in\N}A_{n}=\X$. Now, by the montonicity of
$\bar{\mu}$, observe that
\begin{align}
\lebInt{\mu}{f_{n}} & \geq\lebInt{\mu}{f_{n}\indicate_{A_{n}}}\nonumber \\
 & \geq\lebInt{\mu}{ts\indicate_{A_{n}}}\nonumber \\
 & =\bar{\mu}_{0}\left(ts\indicate_{A_{n}}\right)\nonumber \\
 & =t\bar{\mu}_{0}\left(s\indicate_{A_{n}}\right)\nonumber \\
 & =t\lebInt{\mu}{s\indicate_{A_{n}}}\label{eq:monotoneConvPointwiseEq}
\end{align}
where the third and last equalities follow from the result above that
$\bar{\mu}$ extends $\bar{\mu}_{0}$ and the fourth equality is due
to the \hyperref[prop:lebIntLinearitySimpleFunc]{linearity of $\bar{\mu}_0$}.
Let $s=\sum_{i=1}^{J}b_{i}\indicate_{B_{i}}$then by Lemma \ref{lem:lebIntDiffRep}
\[
\bar{\mu}\left(s\indicate_{A_{n}}\right)=\sum_{i=1}^{I}b_{i}\mu\left(B_{i}\cap A_{n}\right).
\]
Since $A_{n}\subseteq A_{n+1}\Longrightarrow B_{i}\cap A_{n}\subseteq B_{i}\cap A_{n+1}$,
by \hyperref[prop:measureProperties]{continuity from below} of measures,
we have that 
\begin{align}
\lim_{n\to\infty}\lebInt{\mu}{s\indicate_{A_{n}}} & =\lim_{n\to\infty}\sum_{i=1}^{I}b_{i}\mu\left(B_{i}\cap A_{n}\right)\nonumber \\
 & =\sum_{i=1}^{I}b_{i}\mu\left(\bigcup_{n\in\N}\left(B_{i}\cap A_{n}\right)\right)\nonumber \\
 & =\sum_{i=1}^{I}b_{i}\mu\left(B_{i}\right)\nonumber \\
 & =\mu\left(s\right).\label{eq:monotoneConvSimpleLimit}
\end{align}
where the third equality follows from the fact that $\bigcup_{n\in\N}A_{n}=\X$.
Together, (\ref{eq:monotoneConvPointwiseEq}) and (\ref{eq:monotoneConvSimpleLimit})
imply that 
\[
\lim_{n\to\infty}\lebInt{\mu}{f_{n}}\geq t\lebInt{\mu}s.
\]
Since this is true for any $t\in\left(0,1\right)$, we have that 
\[
\lim_{n\to\infty}\lebInt{\mu}{f_{n}}\geq\lebInt{\mu}s.
\]
Finally, since $s\leq f$ was an arbitrary simple function, we have
that 
\[
\lim_{n\to\infty}\lebInt{\mu}{f_{n}}\geq\sup\left\{ \bar{\mu}_{0}\left(s\right)\mid s\in\mathcal{M}_{\textnormal{sim}}\left(\X,\F\right)\textnormal{ such that }s\leq f\right\} =\lebInt{\mu}f.
\]

Next, we show that $\bar{\mu}$ is a linear functional. As before,
it is sufficient to show that for any $f,g\in\nonnegMeasurableFunctions$
and any $\alpha\geq0$,
\begin{align*}
\lebInt{\mu}{\alpha f} & =\alpha\lebInt{\mu}f\\
\lebInt{\mu}{f+g} & =\lebInt{\mu}f+\lebInt{\mu}g.
\end{align*}
First we shall show the homogenous scaling property. If $\alpha=0$
then the proof is trivial; if $\alpha>0$ then by the monotone convergence
property 
\begin{align*}
\lebInt{\mu}{\alpha f} & =\lim_{n\to\infty}\lebInt{\mu}{s_{n}}
\end{align*}
where $\left\{ s_{n}\right\} _{n\in\N}$ is an increasing sequence
of simple functions such that $s_{n}\nearrow\alpha f$. Then, $h_{n}=\frac{s_{n}}{\alpha}$
is an increasing sequence of simple functions such that $h_{n}\nearrow f$
and so 
\begin{align*}
\lebInt{\mu}f & =\lim_{n\to\infty}\lebInt{\mu}{h_{n}}\\
 & =\lim_{n\to\infty}\frac{1}{\alpha}\bar{\mu}_{0}\left(s_{n}\right)\\
 & =\frac{1}{\alpha}\lebInt{\mu}{\alpha f}.
\end{align*}
Rearranging yields the proof. Now let $\left\{ s_{n}\right\} _{n\in\N},\left\{ t_{n}\right\} _{n\in\N}$
be increasing sequences of measurable simple functions such that $s_{n}\nearrow f,t_{n}\nearrow g$.
By the linearity of limits, we have that $s_{n}+t_{n}\nearrow f+g$
and so
\begin{align*}
\lebInt{\mu}{f+g} & =\lim_{n\to\infty}\lebInt{\mu}{s_{n}+t_{n}}\\
 & =\lim_{n\to\infty}\bar{\mu}_{0}\left(s_{n}+t_{n}\right)\\
 & =\lim_{n\to\infty}\bar{\mu}_{0}\left(s_{n}\right)+\lim_{n\to\infty}\bar{\mu}_{0}\left(t_{n}\right)\\
 & =\lebInt{\mu}f+\lebInt{\mu}g
\end{align*}
where the first equality follows from the monotone convergence property,
the second equality follows from the fact that measurable simple functions
are \hyperref[prop:simpleFunctionsAddMultiply]{closed under addition}
and the fact that $\bar{\mu}$ extends $\bar{\mu}_{0}$, the third
equality due to the linearity of $\bar{\mu}_{0}$, and finally the
fourth equality due to a second application of monotone convergence.

To show that our functional $\bar{\mu}$ is unique, suppose that there
were two functionals $\bar{\mu}_{1},\bar{\mu}_{2}$ that satisfied
properties (i)-(iii) with respect to some measure $\mu.$ Then, by
property (i), $\bar{\mu}_{1}\left(\indicate_{A}\right)=\mu\left(A\right)=\bar{\mu}_{2}\left(\indicate_{A}\right)$
for any measurable set $A\in\F$. Next, by the linearity property
along with equality on indicator functions, $\bar{\mu}_{1}\left(s\right)=\bar{\mu}_{2}\left(s\right)$
for any $s\in\mathcal{M}_{\textnormal{sim}}\left(\X,\F\right).$ Finally,
for any arbitrary $f\in\nonnegMeasurableFunctions$ and $s_{n}\in\mathcal{M}_{\textnormal{sim}}\left(\X,\F\right)$
such that $s_{n}\nearrow f$, observe that
\begin{align*}
\bar{\mu}_{1}\left(f\right) & =\lim_{n\to\infty}\bar{\mu}_{1}\left(s_{n}\right)\\
 & =\lim_{n\to\infty}\bar{\mu}_{2}\left(s_{n}\right)\\
 & =\bar{\mu}_{2}\left(f\right)
\end{align*}
where the first equality uses the monotone convergence propert of
$\bar{\mu}_{1}$, the second equality uses the fact that our two functionals
are equal on simple functions, and the last equality uses the monotone
convergence property of $\bar{\mu}_{2}$. This completes the proof.
\end{proof}
\begin{defn}
\label{def:integrable}A function $f\in\mathcal{M}\left(\X,\mathcal{F}\right)$
is called \emph{integrable }with respect to a measure $\mu$(or $\mu-$\emph{integrable
}in short) if 
\[
\bar{\mu}\left(\left|f\right|\right)<\infty.
\]
The collection of all such functions is denoted $\Lp 1{\X,\F,\mu}$.
\end{defn}

When the underlying space and $\sigma-$algebra are clear, we shall
simply write $\Lp 1{\mu}.$ The significance of the exponent $1$
will become clear in the next chapter.
\begin{prop}
\label{prop:L1conditions}A function $f\in\measurableFunctions$ is
in $\Lp 1{\mu}$ if and only if 
\[
\lebInt{\mu}{f^{+}},\lebInt{\mu}{f^{-}}<\infty.
\]
\end{prop}

\begin{proof}
Note that if $\lebInt{\mu}{\left|f\right|}<\infty$ then
\begin{align*}
\lebInt{\mu}{\left|f\right|} & =\lebInt{\mu}{f^{+}+f^{-}}\\
 & =\lebInt{\mu}{f^{+}}+\lebInt{\mu}{f^{-}}\\
 & <\infty
\end{align*}
which would imply that $\lebInt{\mu}{f^{+}},\lebInt{\mu}{f^{-}}<\infty.$
The converse follows similarly.
\end{proof}
\begin{prop}
\label{prop:L1VectorSpace}For any measure space $\left(\X,\F,\mu\right)$,
the space of $\mu-$integrable functions $\Lp 1{\mu}$ is a vector
space over $\R$.
\end{prop}

\begin{proof}
First observe that the zero function $\indicate_{\emptyset}\in\Lp 1{\mu}$since
$\lebInt{\mu}{\left|\indicate_{\emptyset}\right|}=\mu\left(\emptyset\right)=0<\infty$
and so we have an additive identity. Next, notice that $f\in\Lp 1{\mu}\Longleftrightarrow-f\in\Lp 1{\mu}$
and so we have additive inverses for each function. Commutativity
and associativity of addition follow from the definition of addition
on spaces of functions, as does distributivity of scalar multiplication
over addition. Finally, notice that for any $\alpha\in\R$ and any
functions $f,g\in\Lp 1{\mu}$
\begin{align*}
\lebInt{\mu}{\left|\alpha f+g\right|} & \leq\lebInt{\mu}{\left|\alpha f\right|+\left|g\right|}\\
 & =\left|\alpha\right|\lebInt{\mu}{\left|f\right|}+\lebInt{\mu}{\left|g\right|}\\
 & <\infty
\end{align*}
where the first inequality follows from the triangle inequality of
$|\cdot|$ and monotonicty of $\bar{\mu}$, and the equality follows
from the linearity of $\bar{\mu}$. This proves that $\Lp 1{\mu}$
is closed under finite linear combinations and so is a vector space.
\end{proof}
\begin{defn}
\label{def:integral}For any function $f\in\measurableFunctions$
and a measure $\mu$ on $\F$, we can define the \emph{Lebesgue integral}
\[
\tilde{\mu}\left(f\right):=\lebInt{\mu}{f^{+}}-\lebInt{\mu}{f^{-}}
\]
wherever the difference is defined (i.e. at least one of $\lebInt{\mu}{f^{+}}$
and $\lebInt{\mu}{f^{-}}$ is finite).
\end{defn}

\begin{rem*}
Note that for any non-negative measurable function $f$, $\tilde{\mu}\left(f\right)=\lebInt{\mu}f$
and so $\tilde{\mu}$ extends $\bar{\mu}$ just like $\bar{\mu}$
extended $\bar{\mu}_{0}$
\end{rem*}

\section{Properties of the integral}
\begin{prop}
\label{prop:linearityLebIntL1}For any functions $f,g\in\Lp 1{\X,\F,\mu}$
and scalar $\alpha\in\R$, we have that 
\[
\tilde{\mu}\left(\alpha f+g\right)=\alpha\tilde{\mu}\left(f\right)+\tilde{\mu}\left(g\right).
\]
\end{prop}

\begin{proof}
First we shall prove that $\tilde{\mu}\left(\alpha f\right)=\alpha\tilde{\mu}\left(f\right).$
Note that if $\alpha=0$ then the equality follows trivially. If $\alpha>0$
then 
\begin{align*}
\tilde{\mu}\left(\alpha f\right) & =\lebInt{\mu}{\left(\alpha f\right)^{+}}-\lebInt{\mu}{\left(\alpha f\right)^{-}}\\
 & =\lebInt{\mu}{\alpha f\indicate_{\left\{ \alpha f>0\right\} }}-\lebInt{\mu}{-\alpha f\indicate_{\left\{ \alpha f<0\right\} }}\\
 & =\lebInt{\mu}{\alpha f\indicate_{\left\{ f>0\right\} }}-\lebInt{\mu}{-\alpha f\indicate_{\left\{ f<0\right\} }}\\
 & =\alpha\lebInt{\mu}{f^{+}}-\alpha\lebInt{\mu}{f^{-}}\\
 & =\alpha\tilde{\mu}\left(f\right)
\end{align*}
where the third equality follows from the fact that $\alpha>0$ and
the fourth equality due to the linearity of $\bar{\mu}.$ Finally,
if $\alpha<0$ then
\begin{align*}
\tilde{\mu}\left(\alpha f\right) & =\lebInt{\mu}{\left(\alpha f\right)^{+}}-\lebInt{\mu}{\left(\alpha f\right)^{-}}\\
 & =\lebInt{\mu}{\alpha f\indicate_{\left\{ \alpha f>0\right\} }}-\lebInt{\mu}{-\alpha f\indicate_{\left\{ \alpha f<0\right\} }}\\
 & =\lebInt{\mu}{\alpha f\indicate_{\left\{ f<0\right\} }}-\lebInt{\mu}{-\alpha f\indicate_{\left\{ f>0\right\} }}\\
 & =\lebInt{\mu}{-\alpha\times-f\indicate_{\left\{ f<0\right\} }}-\lebInt{\mu}{-\alpha f\indicate_{\left\{ f>0\right\} }}\\
 & =\alpha\bar{\mu}\left(f^{+}\right)-\alpha\lebInt{\mu}{f^{-}}\\
 & =\alpha\tilde{\mu}\left(f\right).
\end{align*}

Next, let $h=f+g$ and observe that
\begin{align*}
h & =h^{+}-h^{-}\\
 & =f+g\\
 & =\left(f^{+}-f^{-}\right)+\left(g^{+}-g^{-}\right).
\end{align*}
Rearranging, we have that 
\[
h^{+}+f^{-}+g^{-}=h^{-}+f^{+}+g^{-}
\]
where the functions on each side are non-negative measurable functions
and so
\[
\lebInt{\mu}{h^{+}+f^{-}+g^{-}}=\lebInt{\mu}{h^{+}}+\lebInt{\mu}{f^{-}}+\lebInt{\mu}{g^{-}}=\lebInt{\mu}{h^{-}}+\lebInt{\mu}{f^{+}}+\lebInt{\mu}{g^{+}}=\lebInt{\mu}{h^{-}+f^{+}+g^{-}}
\]
by linearity of $\bar{\mu}.$ As $h,g,f\in\Lp 1{\mu}$, the integrals
of the individual components are not infinite and so we can rearrange
the second equality above as
\[
\underbrace{\lebInt{\mu}{h^{+}}-\lebInt{\mu}{h^{-}}}_{\tilde{\mu}\left(h\right)}=\underbrace{\lebInt{\mu}{f^{+}}-\lebInt{\mu}{f^{-}}}_{\tilde{\mu}\left(f\right)}+\underbrace{\lebInt{\mu}{g^{+}}-\lebInt{\mu}{g^{-}}}_{\tilde{\mu}\left(g\right)}
\]
which completes the proof.
\end{proof}
\begin{cor}
\label{cor:monotonicityLebIntL1}For any functions $f,g\in\Lp 1{\X,\F,\mu}$
such that $f\leq g$ pointwise,
\[
\tilde{\mu}\left(f\right)\leq\tilde{\mu}\left(g\right).
\]
\end{cor}

\begin{proof}
Note that $h=g-f\geq0$ and so 
\[
\tilde{\mu}\left(g-f\right)=\lebInt{\mu}{g-f}\geq0.
\]
Then, by Proposition \ref{prop:linearityLebIntL1}, 
\[
\tilde{\mu}\left(g-f\right)=\tilde{\mu}\left(g\right)-\tilde{\mu}\left(f\right)\geq0
\]
which completes the proof.
\end{proof}
\begin{cor}
\label{cor:triangleIneqLebIntL1}For any function $f\in\Lp 1{\X,\F,\mu}$
\[
\left|\tilde{\mu}\left(f\right)\right|\leq\tilde{\mu}\left(\left|f\right|\right).
\]
\end{cor}

\begin{proof}
By Corollary \ref{cor:monotonicityLebIntL1} and the triangle inequality
for $\left|\cdot\right|$
\[
\tilde{\mu}\left(f\right)\leq\tilde{\mu}\left(\left|f\right|\right).
\]
Similarly, we have that
\[
-f\leq\left|f\right|\Longrightarrow\tilde{\mu}\left(-f\right)\leq\tilde{\mu}\left(\left|f\right|\right)\Longrightarrow\tilde{\mu}\left(f\right)\geq-\tilde{\mu}\left(\left|f\right|\right)
\]
where the second implication follows due to linearity. Together, the
two inequalities imply that
\[
\left|\tilde{\mu}\left(f\right)\right|\leq\tilde{\mu}\left(\left|f\right|\right)
\]
which is the result.
\end{proof}

\subsection{Interchanging limits and integrals}

At this point we have defined the functionals $\bar{\mu}_{0},\bar{\mu,}$and
$\tilde{\mu}$ to operate on simple functions, non-negative measurable
functions, and all measurable functions respectively. Since $\bar{\mu}$
extends $\bar{\mu}_{0}$ and $\tilde{\mu}$ extends $\bar{\mu}$,
we can dispense with the unnecessary amounts of new notation and simply
denote the integral as $\bar{\mu}\left(f\right)$ for any function
$f\in\measurableFunctions$, provided that the integral is defined.
As discussed earlier, this is analagous to our construction of the
Lebesgue measure in Chapter \ref{chap:measures}, where we ``extended''
the measure from simple sets to more complicated sets over the course
of the chapter.

Next we show the power of this integration theory by establishing
two results which allow us to interchange pointwise limits of functions
and with their integrals. We have already proved the following result
in Theorem \ref{thm:existenceUniquenessLebesgueIntegral}.
\begin{thm}[Monotone convergence theorem]
\label{thm:monotoneConvergenceLebInt}Let $\left\{ f_{n}\right\} _{n\in\N}\in\nonnegMeasurableFunctions$
be such that $f_{n}\leq f_{n+1}$ pointwise. Then, for any meaasure
$\mu$on $\F$,
\[
\lim_{n\to\infty}\lebInt{\mu}{f_{n}}=\lebInt{\mu}{\lim_{n\to\infty}f_{n}}.
\]
\end{thm}

The other theorem drops the requirement that the functions $f_{n}$
be non-negative or that the sequence be monotone, in exchange for
asking the sequence $f_{n}$ to be uniformly bounded by an integrable
function.
\begin{thm}[Dominated convergence theorem]
\label{thm:dominatedConvergenceLebInt}Let $\left\{ f_{n}\right\} _{n\in\N}$
be a sequence of functions in $\Lp 1{\X,\F,\mu}$ and suppose there
exists some $g\in\Lp 1{\mu}$ such that
\[
\left|f_{n}\right|\leq g
\]
pointwise for every $n\in\N$. Then, if $f:=\lim_{n\to\infty}f_{n}$
is defined, we have that
\[
\lim_{n\to\infty}\lebInt{\mu}{\left|f_{n}-f\right|}=0
\]
which also shows that
\[
\lim_{n\to\infty}\bar{\mu}\left(f_{n}\right)=\lebInt{\mu}f.
\]
\end{thm}

\begin{proof}
Note that by the continuity of the absolute value function, $\left|f_{n}\right|\leq g\Longrightarrow\left|f\right|\leq g$
pointwise which shows $f\in\Lp 1{\mu}$. Then, by the triangle inequality
\[
\left|f_{n}\left(x\right)-f\left(x\right)\right|\leq\left|f_{n}\left(x\right)\right|+\left|f\left(x\right)\right|\leq2g\left(x\right)
\]
for every $n\in\N$ and every $x\in\X.$ Note that
\[
\phi_{i}\left(x\right):=\sup_{n\geq i}\left|f_{n}\left(x\right)-f\left(x\right)\right|\leq2g\left(x\right)
\]
for every $i\in\N$ and every $x\in\X,$ which proves that $\phi_{i}\in\Lp 1{\mu}$
for every $i\in\N$. Next, observe that 
\begin{align*}
\lim_{i\to\infty}\phi_{i} & =\limsup_{n\to\infty}\left|f_{n}-f\right|\\
 & =\lim_{n\to\infty}\left|f_{n}-f\right|\\
 & =0
\end{align*}
where the second and third equalities follows from the assumption
that $f_{n}\longrightarrow f$ pointwise. Further, define
\[
\psi_{i}\left(x\right):=2g\left(x\right)-\phi_{i}\left(x\right)\geq0
\]
and observe that since $\phi_{i}\geq\phi_{i+1}$ pointwise for every
$i\in\N$, $\psi_{i}\leq\psi_{i+1}$ and $\lim_{i\to\infty}\psi_{i}=2g$.
Then, by the monotone convergence theorem
\begin{align*}
\lim_{i\to\infty}\lebInt{\mu}{\psi_{i}} & =\lebInt{\mu}{2g}.
\end{align*}
However, note that
\begin{align*}
\lebInt{\mu}{\psi_{i}} & =\lebInt{\mu}{2g-\phi_{i}}\\
 & =\lebInt{\mu}{2g}-\lebInt{\mu}{\phi_{i}}
\end{align*}
and so, since $\lebInt{\mu}{\psi_{i}},\lebInt{\mu}{2g}<\infty$ ,
\[
\lim_{i\to\infty}\lebInt{\mu}{\phi_{i}}=\lebInt{\mu}{2g}-\lim_{i\to\infty}\lebInt{\mu}{\psi_{i}}=0.
\]
Finally, note that $0\leq\lebInt{\mu}{\left|f_{n}-f\right|}\leq\lebInt{\mu}{\phi_{n}}$
for every $n\in\N$ by the monotonicty of $\bar{\mu}$ and so 
\[
\lim_{n\to\infty}\lebInt{\mu}{\left|f_{n}-f\right|}=0.
\]

By Corollary \ref{cor:triangleIneqLebIntL1},
\[
0\leq\lim_{n\to\infty}\left|\lebInt{\mu}{f_{n}-f}\right|\leq\lim_{n\to\infty}\lebInt{\mu}{\left|f_{n}-f\right|}=0
\]
which shows that 
\[
\lim_{n\to\infty}\lebInt{\mu}{f_{n}}=\lebInt{\mu}f.
\]
\end{proof}
We have shown results that establish interchanging limits and integrals
when such limits exist. The next result makes a more general statement
about interchanging limit inferiors of non-negative measurable functions
(which always exist in the extended real numbers) with their integrals,
although we cannot get strict equality.
\begin{thm}[Fatou's lemma]
\label{thm:fatouLemmaLebInt}For any functions $f_{n}\in\nonnegMeasurableFunctions$
\[
\lebInt{\mu}{\liminf_{n\to\infty}f_{n}}\leq\liminf_{n\to\infty}\lebInt{\mu}{f_{n}}.
\]
\end{thm}

\begin{proof}
Define $g_{n}\left(x\right)=\inf_{i\geq n}f_{i}\left(x\right).$ Clearly,
$g_{n}\leq f_{n}$ pointwise and
\[
\liminf_{n\to\infty}f_{n}=\lim_{n\to\infty}g_{n}\in\nonnegMeasurableFunctions
\]
by Corollary \ref{cor:limSupLimInfMeasurable}. Further, $g_{n}\leq g_{n+1}$
pointwise and so by the monotone convergence theorem
\[
\lebInt{\mu}{\liminf_{n\to\infty}f_{n}}=\lebInt{\mu}{\lim_{n\to\infty}g_{n}}=\lim_{n\to\infty}\lebInt{\mu}{g_{n}}=\liminf_{n\to\infty}\lebInt{\mu}{g_{n}}\leq\liminf_{n\to\infty}\lebInt{\mu}{f_{n}}
\]
where the fourth equality follows from the fact the when the limit
exists, limit superiors and inferiors are both equal to the limit.
\end{proof}
\begin{cor}
\label{cor:reverseFatouLemma}For any functions $f_{n}\in\nonnegMeasurableFunctions$
such that $f_{n}\leq g$ for some $g\in\nonnegMeasurableFunctions$
such that $\lebInt{\mu}g<\infty$
\[
\lebInt{\mu}{\limsup_{n\to\infty}f_{n}}\geq\limsup_{n\to\infty}\lebInt{\mu}{f_{n}}.
\]
\end{cor}

\begin{proof}
Define $h_{n}:=g-f_{n}\in\nonnegMeasurableFunctions$ and observe
that by Fatou's lemma
\[
\lebInt{\mu}{\liminf_{n\to\infty}h_{n}}\leq\liminf_{n\to\infty}\lebInt{\mu}{h_{n}}.
\]
Notice that we can rewrite the left hand side as 
\begin{align*}
\lebInt{\mu}{\liminf_{n\to\infty}h_{n}} & =\lebInt{\mu}{g+\liminf_{n\to\infty}-f_{n}}\\
 & =\lebInt{\mu}{g-\limsup_{n\to\infty}f_{n}}\\
 & =\lebInt{\mu}g-\lebInt{\mu}{\limsup_{n\to\infty}f_{n}}.
\end{align*}
Similarly, we can rewrite the right hand side as 
\[
\liminf_{n\to\infty}\lebInt{\mu}{h_{n}}=\lebInt{\mu}g-\limsup_{n\to\infty}\lebInt{\mu}{f_{n}}.
\]
Since $\lebInt{\mu}g<\infty$, we can subtract it from both sides
to yield the result.
\end{proof}
Note that Fatou's lemma is essentially a simple corollarly of the
monotone convergence theorem; it turns out that we can deduce the
monotone convergence theorem from Fatou's lemma as well, meaning that
the two theorems are in fact equivalent. Thus we could characterisze
Lebesgue integrals as linear functionals that satisfy Fatou's lemma
instead of montone convergence.
\begin{prop}
\label{prop:monotoneFromFatou}Let $\measurespace$ be a measure space
and let $\mu^{\prime}:\nonnegMeasurableFunctions\to\left[0,\infty\right]$
be a linear functional satisfying $\mu\left(A\right)=\mu^{\prime}\left(\indicate_{A}\right)$,
linearity and Fatou's lemma. Then for any increasing sequence$f_{n}\in\nonnegMeasurableFunctions$
\[
\mu^{\prime}\left(\lim_{n\to\infty}f_{n}\right)=\lim_{n\to\infty}\mu^{\prime}\left(f_{n}\right).
\]
\end{prop}

One important application of the dominated convergence theorem is
differentiating under the integral sign.
\begin{thm}[Differentiating under the integral sign]
\label{thm:diffIntSign}For any $\theta\in\left[-\delta,\delta\right],$
define $g\left(x,\theta\right)\in\Lp 1{\X,\F,\mu}$ to be differentiable
with respect to $\theta$, with derivative
\[
g_{\theta}\left(x,\theta\right)=\lim_{\epsilon\to0}\frac{g\left(x,\theta+\epsilon\right)-g\left(x,\theta\right)}{\epsilon}.
\]
Define the function
\[
m\left(\theta\right):=\lebInt{{\mu}_{x}}{g\left(x,\theta\right)}
\]
where the subscript $x$ clarifies the variable of integration. If
there exists a function $G\in\Lp 1{\mu}$ such that $\left|g_{\theta}\right|\leq G$
for every $\theta\in\left(-\delta,\delta\right)$ then
\[
\frac{dm\left(\theta\right)}{d\theta}=\frac{d\lebInt{{\mu}_{x}}{g\left(x,\theta\right)}}{d\theta}=\lebInt{{\mu}_{x}}{g_{\theta}\left(x,\theta\right)}.
\]
\end{thm}

\begin{proof}
Define the function 
\[
f_{n}\left(x,\theta\right):=\frac{g\left(x,\theta+\frac{1}{n}\right)-g\left(x,\theta\right)}{\frac{1}{n}}
\]
and observe by the linearity of the Lebesgue integral that 
\[
\frac{m\left(\theta+\frac{1}{n}\right)-m\left(\theta\right)}{\frac{1}{n}}=\lebInt{\mu_{x}}{f_{n}\left(x,\theta\right)}
\]
and that 
\[
\lim_{n\to\infty}f_{n}\left(x,\theta\right)=g_{\theta}\left(x,\theta\right).
\]
Next, observe that by the mean value theorem
\[
f_{n}\left(x,\theta\right)=g_{\theta}\left(x,\tilde{\theta}_{n}\right)
\]
for some $\theta<\tilde{\theta}_{n}<\theta+\frac{1}{n}$ for every
$n\in\N$, which implies that for large enough $n$
\[
\left|f_{n}\left(x,\theta\right)\right|=\left|g_{\theta}\left(x,\tilde{\theta}_{n}\right)\right|\leq G\left(x\right).
\]
Applying the dominated convergence theorem, we have that
\begin{align*}
\lebInt{{\mu}_{x}}{g_{\theta}\left(x,\theta\right)} & =\lebInt{{\mu}_{x}}{\lim_{n\to\infty}f_{n}\left(x,\theta\right)}\\
 & =\lim_{n\to\infty}\lebInt{{\mu}_{x}}{f_{n}\left(x,\theta\right)}\\
 & =\lim_{n\to\infty}\frac{m\left(\theta+\frac{1}{n}\right)-m\left(\theta\right)}{\frac{1}{n}}\\
 & =\frac{dm\left(\theta\right)}{d\theta}
\end{align*}
which completes the proof.
\end{proof}

\subsection{New measures from old and their integrals}
\begin{prop}
\label{prop:densities}Let $\left(\X,\F,\mu\right)$ be a measure
space. For any function $f\in\nonnegMeasurableFunctions$, the set
valued function $\nu:\F\longrightarrow\left[0,\infty\right]$ given
by
\[
\nu\left(A\right):=\lebInt{\mu}{f\indicate_{A}}
\]
is a measure on $\F$. Moreover, for any $A\in\F$, if $\mu\left(A\right)=0$
then $\nu\left(A\right)=0$.
\end{prop}

\begin{proof}
Note that $\nu\left(\emptyset\right)=$$\lebInt{\mu}{f\indicate_{\emptyset}}=\lebInt{\mu}{\indicate_{\emptyset}}=\mu\left(\emptyset\right)=0.$
Next, let $\left\{ A_{i}\right\} _{i\in\N}\in\F$ be disjoint and
define
\[
B_{n}=\bigcup_{i=1}^{n}A_{i}
\]
which is an increasing sequence of sets such that $\bigcup_{n\in\N}B_{n}=\bigcup_{i\in\N}A_{i}$.
Then
\begin{align*}
\nu\left(\bigcup_{i\in\N}A_{i}\right) & =\lebInt{\mu}{f\indicate_{\bigcup_{i\in\N}A_{i}}}\\
 & =\lebInt{\mu}{f\indicate_{\lim_{n\to\infty}B_{n}}}\\
 & =\lebInt{\mu}{f\lim_{n\to\infty}\indicate_{B_{n}}}\\
 & =\lebInt{\mu}{f\lim_{n\to\infty}\sum_{i=1}^{n}\indicate_{A_{i}}}\\
 & =\lim_{n\to\infty}\lebInt{\mu}{\sum_{i=1}^{n}f\indicate_{A_{i}}}\\
 & =\lim_{n\to\infty}\sum_{i=1}^{n}\lebInt{\mu}{f\indicate_{A_{i}}}\\
 & =\sum_{i=1}^{\infty}\lebInt{\mu}{f\indicate_{A_{i}}}\\
 & =\sum_{i=1}^{\infty}\nu\left(A_{i}\right)
\end{align*}
where the second equality follows from the discussion on convergence
of sets in Section 2.1, the third equality due to Proposition \ref{prop:limSupInfIndicator},
the fourth equality from induction on Fact \ref{fact:indicatorFunctionsFiniteOperations},
the fifth equality from the monotone convergence theorem, and the
sixth equality due to the linearity of $\bar{\mu}$. Finally, let
$A\in\F$ be a $\mu-$measure zero set and define $\left\{ s_{n}\right\} _{n\in\N}\in\mathcal{M}_{\textnormal{sim}}\left(\X,\F\right)$
to be an \hyperref[prop:simpleFunctionMonotoneConvergence]{increasing sequence of simple functions}
which converges to $f$ with standard representation 
\[
s_{n}=\sum_{i=1}^{I_{n}}a_{i.n}\indicate_{A_{i,n}}.
\]
Then the $s_{n}\indicate_{A}\nearrow f\indicate_{A}$and 
\begin{align*}
\lebInt{\mu}{s_{n}\indicate_{A}} & =\lebInt{\mu}{\sum_{i=1}^{I_{n}}a_{i,n}\indicate_{A_{i,n}\cap A}}\\
 & =\sum_{i=1}^{I_{n}}a_{i,n}\mu\left(A_{i,n}\cap A\right)\\
 & =0
\end{align*}
where the last equality follows from the monotonicty of measures.
Applying the monotone convergence theorem, we have that 
\begin{align*}
\nu\left(A\right) & =\lebInt{\mu}{f\indicate_{A}}\\
 & =\lim_{n\to\infty}\lebInt{\mu}{s_{n}\indicate_{A}}\\
 & =0
\end{align*}
which completes the proof.
\end{proof}
\begin{rem}
\label{rem:absoluteContinuity}A measure $\nu$ on $\F$ with the
relation $\mu\left(A\right)=0\Longrightarrow\nu\left(A\right)=0$
for every $A\in\F$ and a refence measure $\mu$ is called \emph{absolutely
continuous }with respect to $\mu$. This relation is denoted as $\nu<<\mu$
symbolically. We discuss absolute continuity in Chapter \ref{chap:Differentiation},
where we prove the converse of this theorem under a minor restriction.
The non-negative measurable function $f$ which generates the new
measure $\nu$ is called the \emph{density }of $\nu$ with respect
to $\mu.$ If $\nu$ is a probability measure then $f$ is called
the\emph{ probability density function} of $\nu$ with respect to
$\mu.$
\end{rem}

\begin{cor}
\label{cor:densityIntegral}Let $\mu,\nu$ be measures on $\measurablespace$
and let $f$ be a density of $\nu$ with respect to $\mu.$ The unique
Lebesgue integral associated with the measure $\nu$ is given by 
\[
\lebInt{\nu}g=\lebInt{\mu}{fg},
\]
for any function $g\in\measurableFunctions$, provided the right-hand
side is defined.
\end{cor}

\begin{proof}
Note that for any set $A\in\F$, $\lebInt{\nu}{\indicate_{A}}=\lebInt{\mu}{f\indicate_{A}}=\nu\left(A\right)$.
Further, $\bar{\nu}$ inherits linearity and monotone convergence
on $\nonnegMeasurableFunctions$ from $\bar{\mu}$ and so by Theorem
\ref{thm:existenceUniquenessLebesgueIntegral}, $\bar{\nu}$ is the
unique integral (with respect to measure $\nu$) on $\nonnegMeasurableFunctions,$
which of course extends uniquely to $\measurableFunctions$ through
Definition \ref{def:integral}.
\end{proof}
\begin{prop}
\label{prop:imageMeasures}Let $\left(\X,\F,\mu\right)$ be a measure
space and let $\left(\mathcal{Y},\mathcal{G}\right)$ be a measurable
space . For any $\F/\mathcal{G}$ measurable function $f:\X\to\mathcal{Y}$,
the function $f\mu:\mathcal{G}\longrightarrow\left[0,\infty\right]$
given by 
\[
f\mu\left(B\right):=\mu\left(f^{-1}\left[B\right]\right)
\]
is a measure on $\borel\left(\R\right).$
\end{prop}

\begin{proof}
Note that $f\mu\left(\emptyset\right)=\mu\left(f^{-1}\left[\emptyset\right]\right)=\mu\left(\emptyset\right).$
Next, let $\left\{ B_{i}\right\} _{i\in\N}\in\mathcal{G}$ be disjoint
and observe that
\begin{align*}
f\mu\left(\bigcup_{i\in\N}B_{i}\right) & =\mu\left(f^{-1}\left[\bigcup_{i\in\N}B_{i}\right]\right)\\
 & =\mu\left(\bigcup_{i\in\N}f^{-1}\left[B_{i}\right]\right)\\
 & =\sum_{i=1}^{\infty}\mu\left(f^{-1}\left[B_{i}\right]\right)\\
 & =\sum_{i=1}^{\infty}f\mu\left(B_{i}\right)
\end{align*}
where the second equality is a property of inverse maps and the third
equality is due to the countable additivity of $\mu$.
\end{proof}
\begin{rem*}
Here the measure $f\mu$ is called the \emph{image measure }generated
by $f$ via $\mu.$ If $\mu$ is a probability measure, then $f\mu$
is called the \emph{probability distribution }of $f$.
\end{rem*}
\begin{cor}
\label{cor:changeOfVariables}Let $\left(\X,\F,\mu\right)$ be a measure
space and let $\left(\mathcal{Y},\mathcal{G}\right)$ be a measurable
space. Define $f\mu$ to be the image measure of a measurable function
$f:\X\to\mathcal{Y}$ with respect to measure $\mu.$ Then, for any
function $g\in\mathcal{M}\left(\mathcal{Y},\mathcal{G}\right)$, the
Lebesgue integral associated with measaure $f\mu$ is given by 
\[
\lebInt{f\mu}g:=\lebInt{\mu}{g\circ f}
\]
provided the right hand side is defined.
\end{cor}

\begin{proof}
First note that for any $B\in\mathcal{G}$
\begin{align*}
\lebInt{f\mu}{\indicate_{B}} & =\lebInt{\mu}{\indicate_{B}\circ f}\\
 & =\lebInt{\mu}{\indicate_{f^{-1}\left[B\right]}}\\
 & =\mu\left(f^{-1}\left[B\right]\right)\\
 & =f\mu\left(B\right)
\end{align*}
which satisfies the first required property of Theorem \ref{thm:existenceUniquenessLebesgueIntegral}.
Linearity and monotone convergence follow from $\bar{\mu}$ and so
the uniqueness criterion of the theorem tells us that $\bar{f\mu}$
is indeed the unique integral associated with the image measure $f\mu$.
\end{proof}
Recall from Proposition(\ref{prop:sumOfCountableMeasures}) that the
countable sum of measures on $\left(\X,\F\right)$ is a measure on
$\left(\X,\F\right)$. The integral resulting from this compound measure
can be decomposed into the integrals from constituent summand measures
in certain situations.
\begin{prop}
\label{prop:integralSumOfMeasures}Let $\left(\X,\F\right)$ be a
measurable space and let $\left\{ \mu_{i}\right\} _{i\in\N}$ be a
countable collection of measures on $\F$ with their respective integrals
$\left\{ \lebInt{\mu_{i}}{\cdot}\right\} _{i\in\N}$. Then the integral
associated with the sum measure
\[
\mu:=\sum_{i=1}^{\infty}\mu_{i}
\]
is given by
\[
\lebInt{\mu}f:=\sum_{i=1}^{\infty}\lebInt{\mu}f
\]
for any $f\in\nonnegMeasurableFunctions.$
\end{prop}

\begin{proof}
We first show this holds for $f=\indicate_{A}$for any $A\in\F$;
indeed, an application of Proposition \ref{prop:sumOfCountableMeasures}
is sufficient for this purpose. Next, for any $s\in M_{\textnormal{sim}}\left(\X,\mathcal{F}\right)$,
we have that
\[
s=\sum_{i=1}^{I}\alpha_{i}\indicate_{A_{i}}
\]
where $\alpha_{i}>0$ , $A_{i}:=\left\{ x\in\X:s\left(x\right)=\alpha_{i}\right\} $,
and $I\in\N$. Thus,
\begin{align*}
\lebInt{\mu}s & =\sum_{i=1}^{I}\alpha_{i}\mu\left(A_{i}\right)\\
 & =\sum_{i=1}^{I}\alpha_{i}\sum_{j=1}^{\infty}\mu_{j}\left(A_{i}\right)\\
 & =\sum_{j=1}^{\infty}\sum_{i=1}^{I}\mu_{j}\left(A_{i}\right)\\
 & =\sum_{j=1}^{\infty}\lebInt{\mu_{j}}s
\end{align*}
where the first equality is by the definition of integrals on simple
functions and the third is by the linearity of limits of sequences.
This establishes the result for simple functions. Finally, let $f\in\nonnegMeasurableFunctions$
be arbitrary and observe by Proposition \ref{prop:simpleFunctionMonotoneConvergence}
that there exists some increasing sequence $\left\{ s_{n}\right\} \in M_{\textnormal{sim}}\left(\X,\mathcal{F}\right)$
such that $s_{n}\nearrow f$ and 
\begin{align*}
\lebInt{\mu}f & =\lebInt{\mu}{\lim_{n\to\infty}s_{n}}\\
 & =\lim_{n\to\infty}\lebInt{\mu}{s_{n}}\\
 & =\lim_{n\to\infty}\sum_{i=1}^{\infty}\lebInt{\mu_{i}}{s_{n}}\\
 & =\sum_{i=1}^{\infty}\lebInt{\mu_{i}}{\lim_{n\to\infty}s_{n}}\\
 & =\sum_{i=1}^{\infty}\lebInt{\mu_{i}}f
\end{align*}
where the second equality follows by monotone convergence and the
fourth equality follows by two applications of monotone convergence
(you can think of the countable sum as integration with respect to
the counting measure). This completes the proof.
\end{proof}


\subsection{Equivalence of integrals and measures}

Note that while in Theorem \ref{thm:existenceUniquenessLebesgueIntegral}
we constructed the integral from a seemingly more primitive concept
of a measure, it turns out that a measure can be constructed out a
linear functional on the space of measurable functions such that the
functional is the integral with respect to the measure we have constructed.
In this sense, measures and integrals are really equivalent. At this
point, this may seem to be a trivial observation (and the proof of
this result is indeed trivial); this shift of perspective, however,
offers powerful simplifications to questions concerning the existence
and uniqueness of measures. Indeed, our approach to product measures
in Chapter \ref{chap:productMeasures} would rely on this insight,
allowing us to prove both the existence of product measures, and the
representation of integrals with respect to product measures as iterated
integrals, in a single stroke.
\begin{thm}
\label{thm:integralMeasureEquivalence}Let $\left(\X,\F\right)$ be
a measurable space and let 
\[
\Lambda:\nonnegMeasurableFunctions\longrightarrow\left[0,\infty\right]
\]
be a linear functional that satisfies monotone convergence i.e for
any sequence $f_{n}\in\measurableFunctions$ such that $f_{n}\leq f_{n+1}$
and $f:=\lim_{n\to\infty}f_{n}\in\F$ we have 
\[
\Lambda\left(f\right)=\lim_{n\to\infty}\Lambda\left(f_{n}\right).
\]
Further, suppose there exists a function $g\in\nonnegMeasurableFunctions$
such that $\Lambda\left(g\right)<\infty$. Then the function 
\[
\lambda:\F\longrightarrow\left[0,\infty\right]
\]
given by 
\[
\lambda\left(A\right):=\Lambda\left(\indicate_{A}\right)
\]
for any $A\in\F$ is a measure and $\Lambda$ is the integral with
respect to $\lambda$.
\end{thm}

\begin{proof}
First, note that 
\begin{align*}
\Lambda\left(g\right) & =\Lambda\left(g+\indicate_{\emptyset}\right)\\
 & =\Lambda\left(g\right)+\Lambda\left(\indicate_{\emptyset}\right)\\
 & =\Lambda\left(g\right)+\lambda\left(\emptyset\right)
\end{align*}
by linearity and since $\Lambda\left(g\right)<\infty$, we can subtract
it from both sides to deduce $\lambda\left(\emptyset\right)=0.$ To
establish countable additivity, note that for a disjoint collection
$A_{n}\in\F$, 
\begin{align*}
\lambda\left(\bigcup_{n\in\N}A_{n}\right) & =\Lambda\left(\sum_{n=1}^{\infty}\indicate_{A_{n}}\right)\\
 & =\sum_{n=1}^{\infty}\Lambda\left(\indicate_{A_{n}}\right)\\
 & =\sum_{n=1}^{\infty}\lambda\left(A_{n}\right)
\end{align*}
where the first equality is by (an extension of) Proposition \ref{prop:indicatorFunctionsArbitraryOperations}
and the second by linearity and monotone convergence. Of course, since
$\Lambda$ satisfies the properties of Theorem \ref{thm:existenceUniquenessLebesgueIntegral},
it is the unique integral induced by the measure $\lambda$; this
completes the proof.
\end{proof}

\section{Null sets}
\begin{defn}
\label{def:nullSet}Let $\left(\X,\F,\mu\right)$ be a measure space.
A set $A\in\F$ is called a $\mu-$null set if it's a measure zero
set with respect to $\mu$ i.e.
\[
\mu\left(A\right)=0.
\]
The collection of all $\mu-$null sets in $\F$ is denoted $N_{\mu}$.
\end{defn}

Note we will often omit the ``$\mu$'' when describing a $\mu-$null
set and simply say ``null set'' if the measure is clear from context.
\begin{prop}
\label{prop:nullClosureCountableUnion}Let $\left(\X,\F,\mu\right)$
be a measure space. The set of all null sets $N_{\mu}$ is closed
under countable unions.
\end{prop}

\begin{proof}
Let $\left\{ A_{i}\right\} _{i\in\N}\in N_{\mu}$ be arbitrary. Then,
by \hyperref[cor:countableSubadditivity]{countable subadditivity}
\begin{align*}
\mu\left(\bigcup_{i\in\N}A_{i}\right) & \leq\sum_{i=1}^{\infty}\mu\left(A_{i}\right)\\
 & =\lim_{n\to\infty}\sum_{i=1}^{n}\mu\left(A_{i}\right)\\
 & =0.
\end{align*}
\end{proof}
\begin{prop}
\label{prop:intZeroFuncZero}For any function $f\in\nonnegMeasurableFunctions$
and a measure $\mu$ on $\F$, if the integral $\lebInt{\mu}f=0$
then the set $\left\{ x\in\X\mid f\left(x\right)>0\right\} \in N_{\mu}.$
\end{prop}

\begin{proof}
Observe that for any $n\in\N$ we have the pointwise inequality
\[
h_{n}:=\indicate_{\left\{ f\geq\frac{1}{n}\right\} }\leq nf
\]
where both sides are measurable since $f$ is Borel-measurable. By
the monotonicity and linearity of the integral, we have 
\[
\mu\left(\left\{ x\in\X\mid f\left(x\right)\geq\frac{1}{n}\right\} \right)=\lebInt{\mu}{h_{n}}\leq n\lebInt{\mu}f=0.
\]
Since $f\left(x\right)\geq\frac{1}{n}\Longrightarrow f\left(x\right)\geq\frac{1}{n+1}$
and $\bigcup_{n\in\N}\left\{ x\in\X\mid f\left(x\right)\geq\frac{1}{n}\right\} =\left\{ x\in\X\mid f\left(x\right)>0\right\} ,$
by the \hyperref[prop:measureProperties]{continuity from below of measures}
(or, equivalently, the monotone convergence theorem)
\[
\mu\left(\left\{ x\in\X\mid f\left(x\right)>0\right\} \right)=\lim_{n\to\infty}\mu\left(\left\{ x\in\X\mid f\left(x\right)\geq\frac{1}{n}\right\} \right)=0
\]
which completes the proof.
\end{proof}
\begin{rem*}
When two function $f,g\in\measurableFunctions$ are such that $f=g$
on $\X\setminus A$ for some $A\in N_{\mu}$ , we say that the functions
are equal $\mu-$almost everywhere. In the literature, this is often
shortened to writing $\mu-$a.e or simply a.e if the measure is clear
from context. If $\mu$ is a probability measure, then we say $f=g$
almost surely, which is often shortened to a.s in the literature.\footnote{Sometimes, this is denoted even more compactly as $f\stackrel{\textnormal{a.e}}{=}g.$}
From this it's clear that the set $\left\{ x\in\X\mid f\left(x\right)\neq g\left(x\right)\right\} \subseteq A.$
Of course, if we can show that $\left\{ x\in\X\mid f\left(x\right)\neq g\left(x\right)\right\} \in\F$
then $\left\{ x\in\X\mid f\left(x\right)\neq g\left(x\right)\right\} \in N_{\mu}$
by the monotonicity of measures.

Corresponding to the notion of almost-everywhere equality of measurable
functions, there's a notion of almost-everywhere equality of sets.
Recall from set theory that the symmetric difference of two sets $A,B$
is given
\[
A\Delta B:=\left(A\setminus B\right)\bigcup\left(B\setminus A\right)
\]
Given a measure $\mu$on some measurable space $\measurablespace$
and sets $A,B\in\F$, we say $A\stackrel{\mu-\text{a.e}}{=}B$ if
$\mu\left(A\Delta B\right)=0.$ Of course, in the spirit of Theorem
\ref{thm:integralMeasureEquivalence}, this definition must coincide
with the one for measurable functions in that we must have that $\indicate_{A}\stackrel{\mu-\text{a.e}}{=}\indicate_{B}$.
This is easily shown in the following result.
\end{rem*}
\begin{prop}
\label{prop:almostEverywhereEqualSets}Let $\measurespace$ be a measure
space. For any $A,B\in\F$, $A\stackrel{\mu-\text{a.e}}{=}B$ if and
only if $\indicate_{A}\stackrel{\mu-\text{a.e}}{=}\indicate_{B}$
\end{prop}

\begin{proof}
Note that $\mu\left(A\Delta B\right)=0\Longleftrightarrow\mu\left(A\cap B^{C}\right)=\mu\left(B\cap A^{C}\right)=0.$
Further, observe that
\[
\left\{ x\in\X\mid\indicate_{A}\left(x\right)\neq\indicate_{B}\left(x\right)\right\} =\underbrace{\left\{ x\in\X\mid\indicate_{A}\left(x\right)>\indicate_{B}\left(x\right)\right\} }_{=A\cap B^{C}}\bigcup\underbrace{\left\{ x\in\X\mid\indicate_{A}\left(x\right)<\indicate_{B}\left(x\right)\right\} }_{=B\cap A^{C}}
\]
which completes the proof.
\end{proof}
\begin{lem}
\label{lem:setFuncNotEqualMeasurable}Let $f,g\in\measurableFunctions$
be arbitrary. Then, the set
\[
A:=\left\{ x\in\X\mid f\left(x\right)\neq g\left(x\right)\right\} 
\]
is measurable i.e. $A\in\F.$
\end{lem}

\begin{proof}
Let $h:=\left(f-g\right)\indicate_{\left\{ f=g=\infty\textnormal{or}f=g=-\infty\right\} ^{C}}$and
observe that $h$ is measurable by Corollary \ref{cor:examplesBinaryOpsMeasFunc}.
Further, $A\subseteq\left\{ x\in\X\mid f\left(x\right)=g\left(x\right)=\infty\textnormal{ or }f\left(x\right)=g\left(x\right)=-\infty\right\} ^{C}$
and so
\[
A=\left\{ x\in\X\mid h\left(x\right)\neq0\right\} =\left\{ x\in\X\mid h\left(x\right)>0\right\} \bigcup\left\{ x\in\X\mid h\left(x\right)<0\right\} 
\]
is in $\F$ by the measurability of $h$.
\end{proof}
\begin{prop}
\label{prop:intFiniteFuncFinite}For any function $f\in\nonnegMeasurableFunctions$
and a measure $\mu$ on $\F$, if the integral $\lebInt{\mu}f<\infty$
then $f<\infty$ $\mu-$almost everywhere.
\end{prop}

\begin{proof}
Observe that the pointwise equality
\[
\indicate_{\left\{ f=\infty\right\} }\leq\frac{f}{n}
\]
where both sides are measurable since $f$ is Borel-measurable. By
the monotonicity and linearity of the integral and our assumptions,
\[
0\leq\mu\left(\left\{ x\in\X\mid f\left(x\right)=\infty\right\} \right)\leq\frac{1}{n}\lebInt{\mu}f<\infty.
\]
Since weak inequalities are preserved under limits, we have that 
\[
0\leq\mu\left(\left\{ x\in\X\mid f\left(x\right)=\infty\right\} \right)\leq\lim_{n\to\infty}\frac{1}{n}\lebInt{\mu}f=0
\]
which completes the proof.
\end{proof}
\begin{prop}
\label{prop:funcEqualityAlmostEverywhere}Let $f,g\in\nonnegMeasurableFunctions$
be arbitrary and let $\mu$be a measure on $\F$. If $f=g$ $\mu-$almost
everywhere, then 
\[
\lebInt{\mu}f=\lebInt{\mu}g.
\]
\end{prop}

\begin{proof}
Define the set 
\[
A:=\left\{ x\in\X\mid f\left(x\right)\neq g\left(x\right)\right\} 
\]
which has measure zero by Lemma \ref{lem:setFuncNotEqualMeasurable}
and our assumption. Then, I claim that the pointwise inequality
\[
\min\left\{ g,n\right\} \leq n\indicate_{A}+f
\]
holds for every $n\in\N.$ To see this, we can look at the following
cases:

\begin{enumerate}

\item$g\left(x\right)>n$ and $g\left(x\right)\neq f\left(x\right)$:
In this case, we see that the inequality resolves to 
\[
n\leq n+f\left(x\right)
\]
which is true by the non-negativity of $f.$

\item$g\left(x\right)\leq n$ and $g\left(x\right)\neq f\left(x\right):$
In this case, the inequality resolves to 
\[
g\left(x\right)\leq n+f\left(x\right)
\]
which is again true by the non-negativity of $f.$

\item$g\left(x\right)>n$ and $g\left(x\right)=f\left(x\right):$
In this case, the inequality resolves to 
\[
n\leq f\left(x\right)
\]
which is true since $f\left(x\right)=g\left(x\right).$

\item$g\left(x\right)\leq n$ and $g\left(x\right)=f\left(x\right)$:
In this case, the inequality resolves to 
\[
g\left(x\right)\leq f\left(x\right)
\]
which is true by assumption.

\end{enumerate}

Note that both the left and right hand side are measurable by Proposition
\ref{prop:binaryOperationsMeasurableFunctions}and so, integrating
both sides, we have
\begin{align*}
\lebInt{\mu}{\min\left\{ g,n\right\} } & \leq n\mu\left(A\right)+\lebInt{\mu}f\\
 & =\lebInt{\mu}f
\end{align*}
since $\mu\left(A\right)=0.$ Finally, observe that $\min\left\{ g,n\right\} \leq\min\left\{ g,n+1\right\} $
and that $\lim_{n\to\infty}\min\left\{ g,n\right\} =g.$ Then, applying
the monotone convergence theorem, we have 
\[
\lebInt{\mu}g=\lim_{n\to\infty}\lebInt{\mu}{\min\left\{ g,n\right\} }\leq\lebInt{\mu}f.
\]
We can deduce the reverse inequality with the analagous pointwise
inequality
\[
\min\left\{ f,n\right\} \leq n\indicate_{A}+g
\]
which proves the result.
\end{proof}
\begin{cor}
\label{cor:LPfuncEqualityAlmostEverywhere}Let $f,g\in L^{1}\left(\X,\F,\mu\right)$
be such that they are equal almost everywhere. Then
\[
\lebInt{\mu}f=\lebInt{\mu}g.
\]
\end{cor}

\begin{proof}
If $f=g$ a.e then
\[
f^{+}-f^{-}\stackrel{\textnormal{a.e}}{=}g^{+}-g^{-}\Longleftrightarrow f^{+}+g^{-}\stackrel{\textnormal{a.e}}{=}g^{+}+f^{-}
\]
and so, integrating both sides and applying Proposition \ref{prop:funcEqualityAlmostEverywhere}
we have
\[
\lebInt{\mu}{f^{+}+g^{-}}=\lebInt{\mu}{g^{+}+f^{-}}.
\]
Applying linearity and observing that the integral of each component
is finite by \ref{prop:L1conditions}, we get 
\[
\lebInt{\mu}f=\lebInt{\mu}{f^{+}}-\lebInt{\mu}{f^{-}}=\lebInt{\mu}{g^{+}}-\lebInt{\mu}{g^{-}}=\lebInt{\mu}g
\]
as desired.
\end{proof}
\begin{prop}
\label{prop:intEqualFuncEqual} Let $f,g\in\nonnegMeasurableFunctions$ (or  let $f,g \in \Lp{1}{\mu}$)
where $\mu$ is a measure on $\F$ such that 
\[
\lebInt{\mu}{f\indicate_{F}}=\lebInt{\mu}{g\indicate_{F}}
\]
for every $F\in\F$. Then
\[
f=g
\]
$\mu-$almost everywhere.
\end{prop}

\begin{proof}
Assume the hypothesis is true and define
\[
A:=\left\{ x\in\X\mid f\left(x\right)\neq g\left(x\right)\right\} =\underbrace{\left\{ x\in\X\mid f\left(x\right)>g\left(x\right)\right\} }_{A_{1}}\bigcup\underbrace{\left\{ x\in\X\mid f\left(x\right)<g\left(x\right)\right\} }_{A_{2}}
\]
and suppose for contradiction that $\mu\left(A\right)>0.$ Then one
of $A_{1}$ or $A_{2}$ has positive measure. Assume, without loss
of generality, that $\mu\left(A_{1}\right)>0$ and further define
\[
A_{1,n}:=\left\{ x\in\X\mid f\left(x\right)-\frac{1}{n}\geq g\left(x\right)\right\} =\left\{ x\in\X\mid z\left(x\right)\geq\frac{1}{n}\right\} 
\]
where $z=f-g$ and so is measurable, which in turn implies that $A_{1,n}\in\F$
for all $i\in\N.$ Further, $A_{1,n}\subseteq A_{1,n+1}$ and $\bigcup_{n\in\N}A_{1,n}=A$,
and so by \hyperref[prop:measureProperties]{continuity from below}
\[
\lim_{n\to\infty}\mu\left(A_{1,n}\right)=\mu\left(A_{1}\right)>0.
\]
By the definition of limts, there exists some $n_{0}\in\N$ such that
\[
\mu\left(A_{1,n}\right)>0\ \forall n\geq n_{0}
\]
and for such $n$ we also have, by the monotonicty of integration
\[
\lebInt{\mu}{z\indicate_{A_{1,n}}}\geq\lebInt{\mu}{\frac{1}{n}\indicate_{A_{1,n}}}=\frac{1}{n}\mu\left(A_{1,n}\right)>0
\]
which implies, by the linearity of integration, that
\[
\lebInt{\mu}{f\indicate_{A_{1,n}}}>\lebInt{\mu}{g\indicate_{A_{1,n}}}
\]
which contradicts our hypotehsis. This completes the proof.
\end{proof}
\begin{defn}[Convergence almost everywhere]
\label{def:convAlmostEverywhere}Let $\left(\X,\F,\mu\right)$ be
a measure space and define $\left\{ f_{n}\right\} _{n\in\N}\in\measurableFunctions$
to be a sequence of functions. The sequence is $f_{n}$ is said to
\emph{converge almost everywhere} to a function $f\in\measurableFunctions$
if 
\[
\mu\left(\left\{ x\in\X\mid\lim_{n\to\infty}f_{n}\left(x\right)\neq f\left(x\right)\right\} \right)=0.
\]
In this case, we write
\[
\lim_{n\to\infty}f_{n}\stackrel{\textnormal{a.e}}{=}f
\]
or say $f_{n}\longrightarrow f$ $\mu-$a.e (or $f_{n}\stackrel{\textnormal{a.e}}{\longrightarrow}f$).
\end{defn}

\begin{thm}[Generalized monotone convergence theorem]
\label{thm:generalizedMonotoneConvergence} Let $\left\{ f_{n}\right\} _{n\in\N}\in\nonnegMeasurableFunctions$
be such that $f_{n}\leq f_{n+1}$ $\mu-$almost everyhwere; that is,
there is some $A\in N_{\mu}$ such that $f_{n}\left(x\right)\leq f_{n+1}\left(x\right)$
for all $x\in\X\setminus A.$ Then if 
\[
\lim_{n\to\infty}f_{n}=f
\]
 on $\X\setminus A$, we have that
\[
\lim_{n\to\infty}\lebInt{\mu}{f_{n}}=\lebInt{\mu}f.
\]
\end{thm}

\begin{proof}
Define $g_{n}=f_{n}\indicate_{\X\setminus A}$ and observe that 
\[
g_{n}\leq g_{n+1}
\]
pointwise for all $n\in\N$and 
\[
\lim_{n\to\infty}g_{n}=f\indicate_{\X\setminus A}.
\]
By the standard \hyperref[thm:monotoneConvergenceLebInt]{monotone convergence theorem},
\[
\lim_{n\to\infty}\lebInt{\mu}{g_{n}}=\lebInt{\mu}{f\indicate_{\X\setminus A}}.
\]
But note that since $g_{n}\stackrel{\textnormal{a.e}}{=}f_{n}$ and
$f\stackrel{\textnormal{a.e}}{=}f\indicate_{\X\setminus A}$, by Proposition
\ref{prop:funcEqualityAlmostEverywhere},
\[
\lim_{n\to\infty}\lebInt{\mu}{f_{n}}=\lim_{n\to\infty}\lebInt{\mu}{g_{n}}=\lebInt{\mu}{f\indicate_{\X\setminus A}}=\lebInt{\mu}f
\]
which completes the proof.
\end{proof}
We can similarly strengthen the dominated convergence theorem.
\begin{thm}[Generalized dominated convergence theorem]
\label{thm:generalizedDominatedConvergence}Let $\left\{ f_{n}\right\} _{n}\in\Lp 1{\X,\F,\mu}$
and suppose there exists some $g\in\Lp 1{\mu}$ be such that
\[
\lvert f_{n}\rvert\leq g
\]
$\mu-$almost everywhere. Then, if there exists some $f\in\mathcal{M}\left(\X,\F\right)$
such that 
\[
\lim_{n\to\infty}f_{n}\stackrel{\textnormal{a.e}}{=}f
\]
we have that $f\in\Lp 1{\mu}$ and
\[
\lim_{n\to\infty}\lebInt{\mu}{\lvert f_{n}-f\rvert}=0
\]
and 
\[
\lim_{n\to\infty}\lebInt{\mu}{f_{n}}=\lebInt{\mu}f.
\]
\end{thm}

\begin{proof}
Without loss of generality\footnote{This can be justfied by Proposition \ref{prop:nullClosureCountableUnion}},
assume that there exists some set $A\in N_{\mu}$ such that 
\[
\lvert f_{n}\left(x\right)\rvert\leq g\left(x\right)
\]
for all $x\in\X\setminus A$ and every $n\in\N$, and that
\[
\lim_{n\to\infty}f_{n}\left(x\right)=f\left(x\right)
\]
for all $x\in\X\setminus A.$ Then, consider the functions $h_{n}:=f_{n}\indicate_{\X\setminus A}$and
observe that $g^{*}:=g\indicate_{\X\setminus A}\in\Lp 1{\mu}$ since
$g^{*}\stackrel{\text{a.e}}{=}g\Longrightarrow\lvert g^{*}\rvert\stackrel{\text{a.e}}{=}\lvert g\rvert\Longrightarrow\lebInt{\mu}{\lvert g^{*}\rvert}=\lebInt{\mu}{\lvert g\rvert}<\infty.$
Next, note that 
\[
\lvert h_{n}\rvert\leq g^{*}
\]
everywhere for each $n\in\N$ and 
\[
\lim_{n\to\infty}h_{n}=f\indicate_{\X\setminus A}
\]
pointwise and so, applying the usual \hyperref[thm:dominatedConvergenceLebInt]{dominated convergence theorem}
\[
\lim_{n\to\infty}\lebInt{\mu}{h_{n}}=\lebInt{\mu}{f\indicate_{\X\setminus A}}.
\]
As, before, by Corollary \ref{cor:LPfuncEqualityAlmostEverywhere}
\[
\lim_{n\to\infty}\lebInt{\mu}{f_{n}}=\lim_{n\to\infty}\lebInt{\mu}{h_{n}}=\lebInt{\mu}{f\indicate_{\X\setminus A}}=\lebInt{\mu}f.
\]
\end{proof}
In the context of our discussion on the equivalence between integrals
and measures, we had foreshadowed how the monotone convergence theorem
and the continuity from below of measures were basically the same
concept. We had also said that the \hyperref[thm:borelCantelli]{Borel-Cantelli lemma}
could be understood through from both an integration and measure-theoretic
perspective. We can make this precise with the following result.
\begin{thm}[Generalized Borel-Cantelli lemma]
\label{thm:generalizedBorelCantelli}Let $\left\{ f_{n}\right\} _{n\in\N}\in\nonnegMeasurableFunctions$
and let $\mu$ be measure on $\F$. If 
\[
\sum_{n=1}^{\infty}\lebInt{\mu}{f_{n}}<\infty
\]
then
\[
\sum_{n=1}^{\infty}f_{n}<\infty
\]
$\mu-$almost everywhere.
\end{thm}

\begin{proof}
Note that 
\begin{align*}
\lebInt{\mu}{\sum_{n=1}^{\infty}f_{n}} & =\lebInt{\mu}{\lim_{N\to\infty}\sum_{n=1}^{N}f_{n}}\\
 & =\lim_{N\to\infty}\lebInt{\mu}{\sum_{n=1}^{N}f_{n}}\\
 & =\sum_{n=1}^{\infty}\lebInt{\mu}{f_{n}}<\infty
\end{align*}
where the second equality follows from the monotone convergence theorem
and the third equality due to the linearity of the integral. By Proposition
\ref{prop:intFiniteFuncFinite}
\[
\sum_{n=1}^{\infty}f_{n}<\infty
\]
$\mu-$almost everywhere.
\end{proof}
It should be clear that we can recover our original Borel-Cantelli
lemma by letting $f_{n}=\indicate_{A_{n}}$for sets $\left\{ A_{n}\right\} _{n\in\N}\in\F.$

\section{Convergence of measurable functions}

So far, we have discussed two modes of convergence for measurable
functions explicitly: pointwise convergence and \hyperref[def:convAlmostEverywhere]{almost-everywhere convergence},
the latter of which is implied by the former. We have also implicitly
defined another type of convergence through the \hyperref[thm:dominatedConvergenceLebInt]{dominated convergence theorem}.
We can make this explicit with the following definition
\begin{defn}[Convergence in $L^{1}$]
\label{def:L1Convergence}A sequence of functions $\left\{ f_{n}\right\} _{n\in\N}\in\Lp 1{\X,\F,\mu}$
is said to converge to a function $f\in\Lp 1{\X,\F,\mu}$ in $\mathcal{L}^{1}$
if 
\[
\lim_{n\to\infty}\lebInt{\mu}{\lvert f_{n}-f\rvert}=0.
\]
In this case, we write 
\[
f_{n}\stackrel{\mathcal{L}^{1}}{\longrightarrow}f.
\]
\end{defn}

\begin{defn}[Convergence in measure]
\label{def:convergenceInMeasure}A sequence of functions $\left\{ f_{n}\right\} _{n\in\N}\in\measurableFunctions$
is said to \emph{converge in measure }with respect to measure $\mu$
on $\F$ if for every $\epsilon>0$
\[
\lim_{n\to\infty}\mu\left(\left\{ x\in\X\mid\lvert f_{n}\left(x\right)-f\left(x\right)\rvert>\epsilon\right\} \right)=0.
\]
In this case, we write
\[
f_{n}\stackrel{\mu}{\longrightarrow}f.
\]

Immediately, we would like to know if these ``limits'' are well
behaved in some sense. That is, we woud like to ensure that they satisfy
some basic properties that we expect limits to satisfy.
\end{defn}

\begin{prop}
\label{prop:convFconvAbsFnMinusF}Let $(\X,\F,\mu)$ be a measure
space and let $\left\{ f_{n}\right\} _{n\in\N}\in\measurableFunctions$
converge to $f\in\measurableFunctions$ in any of the modes described
earlier. Then $\lvert f_{n}-f\rvert\to0$ in the same mode.
\end{prop}

\begin{proof}
We first show this for almost everywhere convergence. Suppose $f_{n}\stackrel{\text{a.e}}{\longrightarrow}f$
and consider the function $g_{n}:=\lvert f_{n}-f\rvert$. Note that
if for any $x\in\X$
\[
\lim_{n\to\infty}g_{n}(x)=0\Longleftrightarrow\lim_{n\to\infty}f_{n}(x)=f(x)
\]
by the definition of convergence of sequences. The result then follows
by observing that the points where this does not occur are identical
(and thus so are their measures). The result for the other two modes
are trivial.
\end{proof}
\begin{prop}[Linearity of convergence]
\label{prop:linearityConvergence}Let $(\X,\F,\mu)$ be a measure
space and let $\left\{ f_{n}\right\} _{n\in\N},\left\{ g_{n}\right\} \in\measurableFunctions$
converge to $f,g\in\measurableFunctions$ respectively in any of the
modes described earlier. For any $c\in\R$
\[
cf_{n}+g_{n}\to cf+g
\]
in the same mode of convergence.
\end{prop}

\begin{proof}
In the case of almost sure convergence, let the null sets where $\lim_{n\to\infty}f_{n}(x)\neq f(x)$
and $\lim_{n\to\infty}g_{n}(x)\neq g(x)$ be $N_{f}$ and $N_{g}$
respectively. Then $N=N_{f}\cup N_{g}$ is a null set and so
\[
\lim_{n\to\infty}cf_{n}+g_{n}=cf+g
\]
on $N^{C}$ which establishes linearity.

In the case of convergence in measure, fix $\epsilon>0$ and observe
that 
\begin{align*}
\left\{ x\in\X\mid\lvert cf_{n}(x)+g(x)-cf(x)-g(x)\rvert>\epsilon\right\}  & \subseteq\left\{ x\in\X\mid\lvert c\rvert\lvert f_{n}(x)-f(x)\rvert+\lvert g_{n}(x)-g(x)\rvert>\epsilon\right\} \\
 & \subseteq\left\{ x\in\X\mid\lvert f_{n}(x)-f(x)\rvert>\frac{\epsilon}{2c}\right\} \cup\left\{ x\in\X\mid\lvert g_{n}(x)-g(x)\rvert>\frac{\epsilon}{2}\right\} .
\end{align*}
Subadditvity and monotonicty implies
\begin{align*}
\mu\left(\left\{ x\in\X\mid\lvert cf_{n}(x)+g(x)-cf(x)-g(x)\rvert>\epsilon\right\} \right) & \leq\mu\left(\left\{ x\in\X\mid\lvert f_{n}(x)-f(x)\rvert>\frac{\epsilon}{2c}\right\} \right)\\
 & +\mu\left(\left\{ x\in\X\mid\lvert g_{n}(x)-g(x)\rvert>\frac{\epsilon}{2}\right\} \right).
\end{align*}
Taking limits yields the result.

The case for $L^{1}$ convergence will be established in chapter 4
where we show that $L^{1}$ convergence corresponds to convergence
in a (semi) norm, which automatically implies the result; for now
we take it as given.
\end{proof}
\begin{prop}[Squeeze theorem]
\label{prop:squeezeThm}Let $\left(\X,\F,\mu\right)$ be a measure
space and let $f_{n}\leq g_{n}\leq h_{n}$ be in $\measurableFunctions$
such that $f_{n},h_{n}\to\psi$ in one of the three modes of convergence.
Then $g_{n}\to\psi$ in the same mode of convergence.
\end{prop}

\begin{proof}
Suppose that $f_{n},h_{n}\stackrel{\text{a.s}}{\longrightarrow}\psi$.
Let $N_{f}$ and $N_{h}$ be the null sets where pointwise convergence
fails for the sequences $f_{n}$ and $h_{n}$ respectively. On the
complement of their union, we have pointwise convergence. Take any
point $x\in N_{f}^{C}\cap N_{h}^{C};$for any $\epsilon>0$ there's
some $n_{x,\epsilon}\in\N$ such that for all $n\geq n_{x,\epsilon}$
\[
\lvert f_{n}(x)-\psi(x)\rvert<\epsilon
\]
and
\[
\lvert h_{n}(x)-\psi(x)\rvert<\epsilon
\]
and so
\[
-\epsilon<f_{n}(x)-\psi(x)\leq g_{n}(x)-g(x)\leq h_{n}(x)<\epsilon\Longleftrightarrow\lvert g_{n}(x)-\psi(x)\rvert<\epsilon
\]
which completes the proof.

Now for the convergence in measure, observe that for any $\epsilon>0$,
\[
\left\{ x\in\X\mid\lvert h_{n}(x)-\psi(x)\rvert\leq\epsilon\right\} \bigcap\left\{ x\in\X\mid\lvert f_{n}(x)-\psi(x)\rvert\leq\epsilon\right\} \subseteq\left\{ x\in\X\mid\lvert g_{n}(x)-\psi(x)\rvert\leq\epsilon\right\} 
\]
and so
\[
\left\{ x\in\X\mid\lvert g_{n}(x)-\psi(x)\rvert>\epsilon\right\} \subseteq\left\{ x\in\X\mid\lvert h_{n}(x)-\psi(x)\rvert>\epsilon\right\} \bigcup\left\{ x\in\X\mid\lvert f_{n}(x)-\psi(x)\rvert>\epsilon\right\} 
\]
and so
\[
\mu\left(\left\{ x\in\X\mid\lvert g_{n}(x)-\psi(x)\rvert>\epsilon\right\} \right)\leq\mu\left(\left\{ x\in\X\mid\lvert h_{n}(x)-\psi(x)\rvert>\epsilon\right\} \right)+\mu\left(\left\{ x\in\X\mid\lvert f_{n}(x)-\psi(x)\rvert>\epsilon\right\} \right)
\]
by monotonicty and sub-additivity. Taking limits yields the result.

For $L^{1}$ convergence, note that we have $f_{n}-\psi\leq g_{n}-\psi\leq h_{n}-\psi$
and so\footnote{If $a\leq b\leq c$ then,$\lvert b\rvert\leq\max\left\{ b,-b\right\} \leq\max\left\{ c,-a\right\} \leq\max\left\{ \lvert c\rvert,\lvert a\rvert\right\} .$}
\[
\lvert g_{n}-\psi\rvert\leq\max\left\{ \lvert f_{n}-\psi\rvert,\lvert h_{n}-\psi\rvert\right\} \leq\lvert f_{n}-\psi\rvert+\lvert h_{n}-\psi\rvert.
\]
The linearity of integrals and the squeeze theorem for real sequences
then implies the result.
\end{proof}
Finally, for our limiting operations to make sense, limits should
be unique. Unfortunately, this is not strictly true; however all limits
are almost everywhere unique. For the case of convergence in measure
and $L^{1}$ convergence, we shall show that this follows from the
fact that the topologies for these modes of convergence are given
by (pseudo) metrics (in the special case of $L^{1}$, by a semi-norm);
the case of almost-everywhere convergence, the result follows by the
fact that the union of null sets is null. Therefore we have the following.
\begin{prop}[Uniqueness of limits]
\label{prop:limitUnique}Let $\left(\X,\F,\mu\right)$ be a measure
space and let $\left\{ f_{n}\right\} _{n\in\N},f,g\in\measurableFunctions$
be such that $f_{n}\longrightarrow f$ and $f_{n}\longrightarrow g$
in any of the modes of convergence. Then $f\stackrel{\text{a.s}}{=}g.$
\end{prop}

It would be good to know the conditions under which one type of convergence
implies another, one of which we have already explored in the form
of the (generalized) dominated convergence theorem, which gives us
sufficient conditions for almost everywhere convergence implies $\mathcal{L}^{1}$
convergence. The next <few> results link the other types of convergence.
\begin{prop}
\label{prop:finiteMeasuresAEimpliesConvInMeasures}Let $\left(\X,\F,\mu\right)$
be a measure space such that $\mu\left(\X\right)<\infty$ and let
$\left\{ f_{n}\right\} _{n\in\N}\in\measurableFunctions$ such that
\[
f_{n}\stackrel{\text{a.e}}{\longrightarrow}f.
\]
Then
\[
f_{n}\stackrel{\mu}{\longrightarrow}f.
\]
\end{prop}

\begin{proof}
Define 
\[
A:=\left\{ x\in\X\mid\lim_{n\to\infty}f_{n}\left(x\right)\neq f\left(x\right)\right\} 
\]
and observe that by our hypothesis $\mu\left(A\right)=0.$ Let's unpack
what this means carefully. Fix any $x_{0}\in A$ and observe that
there exists some $\epsilon>0$ such that $\lvert f_{n}\left(x_{0}\right)-f\left(x_{0}\right)\rvert>\epsilon$
for infinitely many $n\in\N.$ Recall the discussion on convergence
of sets in Section 2.1 and notice that we can formalize our intuition
with the equalities
\begin{align*}
A & =\bigcup_{\epsilon>0}\bigcap_{n_{0}\in\N}\bigcup_{n\geq n_{0}}\left\{ x\in\X\mid\lvert f_{n}\left(x\right)-f\left(x\right)\rvert>\epsilon\right\} \\
 & =\bigcup_{\epsilon>0}\limsup_{n\to\infty}\left\{ x\in\X\mid\lvert f_{n}\left(x\right)-f\left(x\right)\rvert>\epsilon\right\} 
\end{align*}
where the set $\left\{ x\in\X\mid\lvert f_{n}\left(x\right)-f\left(x\right)\rvert>\epsilon\right\} \in\F$
by Lemma \ref{lem:compositionMeasurableFunctions} and so $\limsup_{n\to\infty}\left\{ x\in\X\mid\lvert f_{n}\left(x\right)-f\left(x\right)\rvert>\epsilon\right\} \in\F$
by closure under countable unions and intersections. Then,
\[
0\leq\mu\left(\limsup_{n\to\infty}\left\{ x\in\X\mid\lvert f_{n}\left(x\right)-f\left(x\right)\rvert>\epsilon\right\} \right)\leq\mu\left(A\right)=0
\]
for any $\epsilon>0$ by the monotonicity of measures. Finally, recall
that by Proposition \ref{prop:limSupInfIndicator}
\[
\indicate_{\limsup_{n\to\infty}\left\{ x\in\X\mid\lvert f_{n}\left(x\right)-f\left(x\right)\rvert>\epsilon\right\} }=\limsup_{n\to\infty}\indicate_{\left\{ x\in\X\mid\lvert f_{n}\left(x\right)-f\left(x\right)\rvert>\epsilon\right\} }\leq1
\]
where $\lebInt{\mu}1=\mu\left(\X\right)<\infty$. Then, by the \hyperref[cor:reverseFatouLemma]{reverse Fatou Lemma}
\[
0\leq\limsup_{n\to\infty}\mu\left(\left\{ x\in\X\mid\lvert f_{n}\left(x\right)-f\left(x\right)\rvert>\epsilon\right\} \right)\leq\mu\left(\limsup_{n\to\infty}\left\{ x\in\X\mid\lvert f_{n}\left(x\right)-f\left(x\right)\rvert>\epsilon\right\} \right)=0.
\]
By the non-negativity of measures
\begin{align*}
0 & \leq\liminf_{n\to\infty}\mu\left(\left\{ x\in\X\mid\lvert f_{n}\left(x\right)-f\left(x\right)\rvert>\epsilon\right\} \right)\\
 & \leq\lim_{n\to\infty}\mu\left(\left\{ x\in\X\mid\lvert f_{n}\left(x\right)-f\left(x\right)\rvert>\epsilon\right\} \right)\\
 & \leq\limsup_{n\to\infty}\mu\left(\left\{ x\in\X\mid\lvert f_{n}\left(x\right)-f\left(x\right)\rvert>\epsilon\right\} \right)\\
 & =0
\end{align*}
which completes the proof.
\end{proof}
Note that if we drop the finiteness assumption the result does not
hold: take $\indicate_{[n,n+1]}\longrightarrow0$ pointwise and the
Lebesgue measure $\lambda$ on $\R$ and note that for any $\epsilon\in\left(0,1\right)$
\[
\lim_{n\to\infty}\lambda\left(\left\{ x\in\R\mid\indicate_{[n,n+1]}>\epsilon\right\} \right)=\lim_{n\to\infty}\lambda\left(\left[n,n+1\right]\right)=1.
\]
Note that crucial failure here is that we cannot dominate $\indicate_{[n,n+1]}$
uniformly (i.e. for all $n$) with any integrable function $g$ and
so a violation of dominated convergence leads to a failure of convergence
in measure.

The converse to this result is not true either, that is, convergence
in measure does not imply almost everywhere convergence, even when
the measure in question in is finite.
\begin{example}
\label{exa:convMeasureDoesntImplyAEConv} Observe that for any $n\in\N$,
there exists some $k\in\N\cup\{0\}$ such that $2^{k}\leq n<2^{k+1}.$
By construction, there's only one such $k$ and so we can denote it
$k(n).$ Consider the following collection of sets in $\borel(\R)$
\[
E_{n}:=\left[\frac{n-2^{k(n)}}{2^{k(n)}},\frac{n+1-2^{k(n)}}{2^{k(n)}}\right]
\]
and observe that $E_{n}\subseteq[0,1]$ for all $n\in\N$. To see
this, note that the lower bound is smallest when $n=2^{k(n)}$ and
the upper bound is largest when $n=2^{k(n)+1}-1$. which correspond
to 0 and 1 respectively. Now for $\epsilon\in(0,1)$
\[
\lim_{n\to\infty}\lambda\left(\left\{ x\in\R\mid\indicate_{E_{n}}>\epsilon\right\} \right)=\lim_{n\to\infty}\lambda\left(E_{n}\right)=\lim_{n\to\infty}\frac{1}{2^{k(n)}}=0
\]
as $k(n)\to\infty.$ Further, for $\epsilon\geq1$
\[
\lambda\left(\left\{ x\in\R\mid\indicate_{E_{n}}>\epsilon\right\} \right)=0
\]
for every $n\in\N$ and so 
\[
\indicate_{E_{n}}\stackrel{\mu}{\longrightarrow}0.
\]
Now for $x\in[0,1]$, we have that $x\in E_{n}$ for infinitely many
$n\in\N$. We can establish this by noting that 
\[
\bigcup_{2^{k(n)}\leq n\leq2^{k(n)}-1}E_{n}=[0,1]
\]
and so for each $k\in\N\cup\{0\}$ there's some $n_{k}$ such that
$x\in E_{n_{k}}$. Then the subsequence $\indicate_{E_{n_{k}}}=1$
and so the our sequence $\indicate_{E_{n}}$doesnt converge pointwise
to zero anywhere in $[0,1].$ In particular, it doesn't converge almost
everywhere.
\end{example}

However, convergence in measure implies almost everywhere subsequential
convergence. To show this, we will first establish a useful lemma.
\begin{lem}
\label{lem:borelCantelliAEConvArgument}Let $\left(\X,\F,\mu\right)$
be a measure space and let $\left\{ f_{n}\right\} _{n\in\N}\in\measurableFunctions$
be such that 
\[
\sum_{n=1}^{\infty}\mu\left(\left\{ x\in\X\mid\lvert f_{n}\left(x\right)-f\left(x\right)\rvert>\epsilon\right\} \right)<\infty
\]
for every $\epsilon>0$ and some $f\in\measurableFunctions.$ Then
\[
f_{n}\stackrel{\text{a.e}}{\longrightarrow}f.
\]
\end{lem}

\begin{proof}
Applying the Borel-Cantelli lemma, we know that for each $\epsilon>0$
\[
\mu\left(\limsup_{n\to\infty}\left\{ x\in\X\mid\lvert f_{n}\left(x\right)-f\left(x\right)\rvert>\epsilon\right\} \right)=0.
\]
In particular, this is true for $\epsilon=\frac{1}{p}$ for every
$p\in\N.$ By the Archimedean property of natural numbers and our
discussion in the proof of Proposition \ref{prop:finiteMeasuresAEimpliesConvInMeasures},
the set
\[
\left\{ x\in\X\mid\lim_{n\to\infty}f_{n}\left(x\right)\neq f\left(x\right)\right\} =\bigcup_{p\in\N}\limsup_{n\to\infty}\left\{ x\in\X\mid\lvert f_{n}\left(x\right)-f\left(x\right)\rvert>\frac{1}{p}\right\} 
\]
and so by Proposition \ref{prop:nullClosureCountableUnion}
\[
\mu\left(\left\{ x\in\X\mid\lim_{n\to\infty}f_{n}\left(x\right)\neq f\left(x\right)\right\} \right)=0
\]
which completes the proof.
\end{proof}
\begin{prop}
\label{prop:convInMeasureImpliesSubsequenceAE}Let $\left(\X,\F,\mu\right)$
be a measure space and let $\left\{ f_{n}\right\} _{n\in\N}\in\measurableFunctions$
be a sequence of functions such that 
\[
f_{n}\stackrel{\mu}{\longrightarrow}f
\]
for some $f\in\measurableFunctions.$ Then there exists a sequence
of natural numbers $\left\{ n_{k}\right\} _{k\in\N}$such that
\[
f_{n_{k}}\stackrel{\text{a.e}}{\longrightarrow}f.
\]
\end{prop}

\begin{proof}
Note that by convergence in measure, for any $\epsilon,\delta>0$
there exists some $n_{\epsilon,\delta}^{*}\in\N$ such that for all
$n\geq n_{\epsilon,\delta}^{*}$
\[
\mu\left(\left\{ x\in\X\mid\lvert f_{n}\left(x\right)-f\left(x\right)\rvert>\epsilon\right\} \right)<\delta.
\]
In particular, this is true for $\epsilon=\delta=2^{-k}$ for all
$k\in\N.$ Define the sequence $\left\{ n_{k}\right\} _{k\in\N}$by
\[
n_{k}:=\begin{cases}
n_{1,1}^{*}, & k=1\\
n_{2^{-k},2^{-k}}^{*}, & n_{k-1}<n_{2^{-k},2^{-k}}^{*}\\
n_{2^{-k},2^{-k}}^{*}+1, & n_{k-1}=n_{2^{-k},2^{-k}}^{*}
\end{cases}
\]
and observe that $n_{k}$ is a strictly increasing sequence of natural
numbers and that 
\[
\sum_{k=1}^{\infty}\mu\left(\left\{ x\in\X\mid\lvert f_{n_{k}}\left(x\right)-f\left(x\right)\rvert>2^{-k}\right\} \right)<\sum_{k=1}^{\infty}2^{-k}=1<\infty.
\]
Applying Lemma \ref{lem:borelCantelliAEConvArgument} furnishes the
result.
\end{proof}
\begin{prop}[Markov's inequality]
\label{prop:markovInequality} Let $\left(\X,\F,\mu\right)$ be a
measure space. For any function $f\in\Lp 1{\mu}$ and any $a>0$
\[
\mu\left(\left\{ x\in\X\mid\lvert f\left(x\right)\rvert>a\right\} \right)\leq\frac{\lebInt{\mu}{\lvert f\rvert}}{a}.
\]
\end{prop}

\begin{proof}
Note the pointwise inequality
\[
a\indicate_{\left\{ x\in\X\mid\lvert f\left(x\right)\rvert>a\right\} }\leq\lvert f\left(x\right)\rvert
\]
and observe that by the monotonicity of the integral
\[
a\lebInt{\mu}{\indicate_{\{x\in\X\mid\lvert f\left(x\right)\rvert>a\}}}\leq\lebInt{\mu}{\lvert f\rvert}
\]
which finishes the proof.
\end{proof}
\begin{prop}
\label{prop:L1impliesConvergenceInMeasure}For a sequence of functions
$\left\{ f_{n}\right\} _{n\in\N}\in\Lp 1{\X,\F,\mu}$such that 
\[
f_{n}\stackrel{\mathcal{L}^{1}}{\longrightarrow}f
\]
for some $f\in\Lp 1{\mu},$ we have 
\[
f_{n}\stackrel{\mu}{\longrightarrow}f.
\]
\end{prop}

\begin{proof}
Note that for any fixed $\epsilon>0$
\[
0\leq\mu\left(\left\{ x\in\X\mid\lvert f_{n}\left(x\right)-f\left(x\right)\rvert>\epsilon\right\} \right)\leq\frac{1}{\epsilon}\lebInt{\mu}{\lvert f_{n}-f\rvert}
\]
by the Markov inequality. Taking the limit proves the result.
\end{proof}
Again, the converse of this result is false.
\begin{example}
\label{exa:convMeasureDoesNotImplyConvLp}The sequence of functions
$f_{n}:=n\indicate_{\left(0,\frac{1}{n}\right]}$ on the measure space
$\left(\left[0,1\right],\borel\left(\left[0,1\right]\right),\lambda\right)$
converges pointwise to zero: for any $x\in\R$, eventually $\indicate\left(x\right)=0$
for large $n$and so $f_{n}\to0$ which means that $f_{n}\stackrel{\lambda}{\to}0$
as well by Proposition \ref{prop:finiteMeasuresAEimpliesConvInMeasures}.
But $\lebInt{\lambda}{\lvert f_{n}\rvert}=1$ which doesn't converge
to 0.
\end{example}

\begin{thm}[Egorov's theorem]
\label{thm:egorovTheorem}Let $\left(\X,\F,\mu\right)$ be a measure
space such that $\mu\left(\X\right)<\infty$ and fix some $\epsilon>0$.
If $\left\{ f_{n}\right\} _{n\in\N}\in\measurableFunctions$ is a
sequence functions such that
\[
f_{n}\stackrel{\text{a.e}}{\longrightarrow}f
\]
where $f\in\measurableFunctions$ then there exists some $A\in\F$
such that $\mu\left(A\right)<\epsilon$ and 
\[
\lim_{n\to\infty}\sup_{x\in\X\setminus A}\lvert f_{n}\left(x\right)-f\left(x\right)\rvert=0
\]
\end{thm}

\begin{proof}
Define the set 
\[
A_{n,k}:=\bigcup_{i=n}^{\infty}\left\{ x\in\X\mid\lvert f_{i}\left(x\right)-f\left(x\right)\rvert>\frac{1}{k}\right\} 
\]
and recall from our earlier discussion that 
\[
\mu\left(\bigcap_{n\in\N}A_{n,k}\right)=0
\]
for every $k\in\N$ by almost everywhere convergence. Then, since
$\mu\left(\X\right)<\infty,$ we can apply the \hyperref[prop:equivalenceContinuityMeasures]{continuity from above of measures}
to deduce that 
\[
\lim_{n\to\infty}\mu\left(A_{n,k}\right)=0.
\]
This means there exists some $n_{k}\in\N$ such that for all $n\geq n_{k}$
\[
\mu\left(A_{n,k}\right)<\frac{\epsilon}{2^{k}}
\]
and so
\[
\mu\left(\bigcup_{k\in\N}A_{n_{k},k}\right)\leq\sum_{k=1}^{\infty}\mu\left(A_{n_{k},k}\right)\leq\epsilon.
\]
If we let $A:=\bigcup_{k\in\N}A_{n_{k},k}$ then
\[
\X\setminus A=\bigcap_{k\in\N}\bigcap_{i=n_{k}}^{\infty}\left\{ x\in\X\mid\left\lvert f_{i}\left(x\right)-f\left(x\right)\right\rvert \leq\frac{1}{k}\right\} .
\]
Fix any $k\in\N$, and observe that for any $n\geq n_{k}$ and any
$x\in\X\setminus A$
\[
\lvert f_{n}\left(x\right)-f\left(x\right)\rvert\leq\frac{1}{k}
\]
which means 
\[
\sup_{x\in\X\setminus A}\lvert f_{n}\left(x\right)-f\left(x\right)\rvert\leq\frac{1}{k}.
\]
Since $k$ was unspecified, we have uniform convergence.
\end{proof}

\subsection{Topologies of various convergence notions}
\begin{defn}
\label{def:pseudoMetric}Let $X$ be a set and let 
\[
d:X\times X\to\R
\]
be a function that satisfies\begin{enumerate}[label=(\roman*),leftmargin=.1\linewidth,rightmargin=.4\linewidth]
	\item $ d(x,y) = d(y,x) \forall x,y \in X$ (Symmetry)
	\item $ d(x,z) \leq d(x,y) + d(y,z) \forall x,y,z \in X $ (Triangle inequality)
\end{enumerate}

Then $(X,d)$ is called a \emph{pseudometric space }and $d$ is called
a \emph{pseudo-metric.}
\end{defn}

\begin{prop}
\label{prop:pseudoMetricConvMeasure}Let $\left(\X,\F,\mu\right)$
be a finite measure space i.e. $\mu(\X)<\infty$. Then the function
\[
d(f,g):=\lebInt{\mu}{\frac{\lvert f-g\rvert}{1+\lvert f-g\rvert}}
\]
defines a pseudo-metric. Moreover, for any $\left\{ f_{n}\right\} _{n\in\N},f\in\measurableFunctions$
\[
f_{n}\stackrel{\mu}{\longrightarrow}f
\]
if and only if
\[
d(f_{n},f)\longrightarrow0.
\]
\end{prop}

\begin{proof}
Note that 
\[
0\leq\frac{\lvert f-g\rvert}{1+\lvert f-g\rvert}\leq1
\]
and so the function $d$ is well defined as an integral on a finite
measure space of bounded a function. Note that the symmetry of $d$
is obvious given the symmetry of the absolute value function. To see
the triangle inequality, first observe that the function $g(x)=\frac{x}{1+x}$
is increasing on $x>0$ (it has a derivative $\frac{1}{(1+x)^{2}}$)
and so, by the triangle inequality for absolute value
\begin{align*}
\frac{\lvert f-h\rvert}{1+\lvert f-h\rvert} & \leq\frac{\lvert f-g\rvert+\lvert g-h\rvert}{1+\lvert f-g\rvert+\lvert g-h\rvert}\\
 & =\frac{\lvert f-g\rvert}{1+\lvert f-g\rvert+\lvert g-h\rvert}+\frac{\lvert g-h\rvert}{1+\lvert f-g\rvert+\lvert g-h\rvert}\\
 & \leq\frac{\lvert f-g\rvert}{1+\rvert f-g\rvert}+\frac{\lvert g-h\rvert}{1+\rvert g-h\rvert}.
\end{align*}
Linearity and monotonicity of integration yields the triangle inequality
for $d.$ Now suppose $f_{n}\stackrel{\mu}{\to}f$ , let $\epsilon>0$
and observe that 
\begin{align*}
d(f_{n},f) & =\lebInt{\mu}{\frac{\lvert f_{n}-f\rvert}{1+\lvert f_{n}-f\rvert}}\\
 & =\lebInt{\mu}{\frac{\lvert f_{n}-f\rvert}{1+\lvert f_{n}-f\rvert}\indicate_{\{\lvert f_{n}-f\rvert>\epsilon\}}}+\lebInt{\mu}{\frac{\lvert f_{n}-f\rvert}{1+\lvert f_{n}-f\rvert}\indicate_{\{\lvert f_{n}-f\rvert\leq\epsilon\}}}\\
 & \leq\mu\left(\lvert f_{n}-f\rvert>\epsilon\right)+\frac{\epsilon}{1+\epsilon}\mu\left(\lvert f_{n}-f\rvert\leq\epsilon\right)\\
 & \leq\mu\left(\lvert f_{n}-f\rvert>\epsilon\right)+\frac{\epsilon}{1+\epsilon}\mu\left(\X\right).
\end{align*}
Therefore by convergence in measure 
\[
\lim_{n\to\infty}d(f_{n},f)\leq\frac{\epsilon}{1+\epsilon}\mu\left(\X\right).
\]
Of course, $\epsilon$ can be arbitrarily small and so we have convergence
in the pseudo-metric.

Conversely, suppose $\lim_{n\to\infty}d(f_{n},f)=0$ and fix some
$\epsilon>0$. Then
\[
0\leq\frac{\epsilon}{1+\epsilon}\mu\left(\lvert f_{n}-f\rvert>\epsilon\right)\leq\lebInt{\mu}{\frac{\lvert f_{n}-f\rvert}{1+\lvert f_{n}-f\rvert}\indicate_{\{\lvert f_{n}-f\rvert>\epsilon\}}}\leq\lebInt{\mu}{\frac{\lvert f_{n}-f\rvert}{1+\lvert f_{n}-f\rvert}}
\]
and in the limit 
\[
\frac{\epsilon}{1+\epsilon}\mu\left(\lvert f_{n}-f\rvert>\epsilon\right)\longrightarrow0\implies\mu\left(\lvert f_{n}-f\rvert>\epsilon\right)\longrightarrow0
\]
which completes the proof.
\end{proof}
Note that $d(f,g)=0\Longleftrightarrow f\stackrel{\text{a.e}}{=}g$.
This is a simple corrolary of Proposition \ref{prop:intZeroFuncZero}.
Propositions \ref{prop:convInMeasureImpliesSubsequenceAE} and \ref{prop:finiteMeasuresAEimpliesConvInMeasures}
gives us a sort of equivalence between almost everywhere equivalence
as a corollary of a basic fact about sequences in topological spaces.
\begin{lem}
\label{lem:subsubSequence}Let $\left(\X,\tau\right)$ be a topological
space. The sequence$\left\{ x_{n}\right\} _{n\in\N}\in\X$ converges
to $x\in\X$ if and only if every subsequence of $x_{n}$ has a further
subsequence that converges to $x.$
\end{lem}

\begin{proof}
Convergence implies subsequential convergence for every subseqeuence
(and in particulary, subsubsequences). Conversely, suppose that $x_{n}\nrightarrow x$.
Then there's an open set $\mathcal{U\in\tau}$ that contains $x$
such that there are infinitely many elements of $x_{n}$ outside $\mathcal{U}$.
In other words, there exists some subsequence $x_{n_{k}}\notin\mathcal{U}$
for all $k\in\N$. Then no subsequence of $x_{n_{k}}$ converges to
$x$.
\end{proof}
\begin{prop}
\label{prop:subsubsequenceProbAEConv}Let $\left(\X,\F,\mu\right)$
be a finite measure space i.e $\mu(\X)<\infty.$ Then for any sequence
$\left\{ f_{n}\right\} _{n\in\N}\in\measurableFunctions$ and $f\in\measurableFunctions$
\[
f_{n}\stackrel{\mu}{\longrightarrow}f
\]
if and only if every subsequence $f_{n_{k}}$ has a further subsequence
$f_{n_{k_{j}}}$ such that 
\[
f_{n_{k_{j}}}\stackrel{\text{a.e}}{\longrightarrow}f.
\]
\end{prop}

\begin{proof}
Suppose $f_{n}\stackrel{\mu}{\longrightarrow}f$ and let $f_{n_{k}}$
be an arbitrary subsequence. This subsequence also converges to $f$
in measure and so by Proposition \ref{prop:convInMeasureImpliesSubsequenceAE}
there's some sub-subsequence $f_{n_{k_{j}}}\stackrel{\text{a.e}}{\longrightarrow}f$.
Conversely, suppose every subsequence $f_{n_{k}}$ has a further subsequence
$f_{n_{k_{j}}}\stackrel{\text{a.e}}{\longrightarrow}f$. Then by Proposition
\ref{prop:finiteMeasuresAEimpliesConvInMeasures} $f_{n_{k_{j}}}\stackrel{\mu}{\longrightarrow}f$.
By Proposition \ref{prop:subsubsequenceProbAEConv}, convergence in
measure is (pseudo)-metrizable and thus topological and so by Lemma
\ref{lem:subsubSequence} $f_{n}\stackrel{\mu}{\longrightarrow}f.$
\end{proof}
\begin{cor}
\label{cor:noTopologyForAEConv}There is no topology for almost everywhere
convergence.
\end{cor}

\begin{proof}
Note that if almost everywhere convergence was topological, then by
Lemma \ref{lem:subsubSequence} and Proposition \ref{prop:subsubsequenceProbAEConv},
$f_{n}\stackrel{\mu}{\longrightarrow}f\implies f_{n}\stackrel{\text{a.e}}{\longrightarrow}f$.
But this has been ruled out by Example \ref{exa:convMeasureDoesntImplyAEConv}.
\end{proof}
Thus we have established that the topology of convergence in measure
comes from a (pseudo)-metric, whereas almost everywhere convergence
is not topologizable at all! Convergence in $L^{1}$ , on the other
hand, corresponds to a (semi)-normed convergence as shown in Chapter
\ref{chap:spaces_of_functions}. Proposition \ref{prop:subsubsequenceProbAEConv}
implies a form of dominated convergence theorem for convergence in
measure.
\begin{prop}
\label{prop:convMeasureDomConv}Let $\left(\X,\F,\mu\right)$ be a
finite measure space i.e $\mu(\X)<\infty.$ Suppose $\left\{ f_{n}\right\} _{n\in\N}\in\measurableFunctions$
such that $\lvert f_{n}\rvert\leq g$ where $g\in\Lp 1{\X,\F,\mu}.$
Then
\[
f_{n}\stackrel{\mu}{\longrightarrow}f\implies f_{n}\stackrel{\mathcal{L}^{1}}{\longrightarrow}f.
\]
\end{prop}

\begin{proof}
Note that if $f_{n}\stackrel{\mu}{\longrightarrow}f$, then for any
subsequence $f_{n_{j}}$ of $f_{n}$ there exists a further subseqeunce
$f_{n_{j_{k}}}$ such that 
\[
f_{n_{j_{k}}}\stackrel{\text{a.e}}{\longrightarrow}f.
\]
Since $\lvert f_{n_{j_{k}}}\rvert\leq g\in\Lp 1{\mu}$, the \hyperref[thm:generalizedDominatedConvergence]{dominated convergence theorem}
implies that 
\[
f_{n_{j_{k}}}\stackrel{\mathcal{L}^{1}}{\longrightarrow}f.
\]
But $L^{1}$ convergence is topological and so by Lemma \ref{lem:subsubSequence},
\[
f_{n}\stackrel{\mathcal{L}^{1}}{\longrightarrow}f.
\]
\end{proof}

\subsection{Uniform integrability and uniform absolute continuity}

The dominated convergence theorem says that for sequence of random
variables that are dominated by an integrable function, pointwise
convergence implies convergence in $L^{1}$. We would like to extend
this notion more broadly.
\begin{defn}
\label{def:uniformIntegrability}Let $\left(\X,\F,\mu\right)$ be
a measure space and let $\mathcal{C}\subseteq\measurableFunctions$
be a collection of measurable functions. We say $\mathcal{C}$ is
\emph{uniformly integrable }if for every $\epsilon>0$, there exists
some $M\in\N$ such that 
\[
\lebInt{\mu}{\lvert f\rvert\indicate_{\{\lvert f\rvert>M\}}}<\epsilon
\]
for every $f\in\mathcal{C}$.
\end{defn}

\begin{defn}
\label{def:uniformAbsoluteContinuity}Let $\left(\X,\F,\mu\right)$
be a measure space and let $\mathcal{C}\subseteq\measurableFunctions$
be a collection of measurable functions. We say $\mathcal{C}$ is
\emph{uniformly absolutely continuous }if for every $\epsilon>0$
there exists a $\delta>0$ such that for any $f\in\mathcal{C}$
\[
F\in\F,\mu(F)<\delta\implies\lvert\lebInt{\mu}{f\indicate_{F}}\rvert<\epsilon.
\]
\end{defn}

\begin{rem}
\label{rem:uniformAbsoluteContinuity}Note that if $\mathcal{C}$
is a trivial class i.e. $\mathcal{C}=\{f\}$, then the absolute continuity
property always holds as observed in Proposition \ref{prop:densities}
and Remark \ref{rem:absoluteContinuity}. Of course, it is not clear
that this $\epsilon-\delta$ characterization of absolute continuity
is the same as the one discussed in that remark; later on in Chapter
\ref{chap:Differentiation} (see Proposition \ref{prop:epsdeltaAbsContinuity})
we show they are indeed equiavelent for finite measures.
\end{rem}

\begin{prop}
\label{prop:finiteMeasureUniformIntegrability}Let $\left(\X,\F,\mu\right)$
be a finite measure space i.e. $\mu(\X)<\infty.$ Then a class of
functions $\mathcal{C}\subseteq\measurableFunctions$ is uniformly
integrable if and only if it is uniformly absolutely continuous and
\[
\sup_{f\in\mathcal{C}}\lebInt{\mu}{\lvert f\rvert}<\infty.
\]
\end{prop}

\begin{proof}
Suppose $\mathcal{C}\subseteq\measurableFunctions$ is uniformly integrable.
Fix some $\epsilon>0$ and observe that by uniform integrability there
is some $M\in\N$ such that 
\[
\lebInt{\mu}{\lvert f\rvert\indicate_{\{\lvert f\rvert>M\}}}<\frac{\epsilon}{2}.
\]
Then for any $F\in\F$,
\begin{align*}
\lvert\lebInt{\mu}{f\indicate_{F}}\rvert & \leq\lebInt{\mu}{\lvert f\rvert\indicate_{F}}\\
 & =\lebInt{\mu}{\lvert f\rvert\left(\indicate_{\{\lvert f\rvert>M\}}+\indicate_{\{\lvert f\rvert\leq M\}}\right)\indicate_{F}}\\
 & =\lebInt{\mu}{\lvert f\rvert\indicate_{\{\lvert f\rvert>M\}}\indicate_{F}}+\lebInt{\mu}{\lvert f\rvert\indicate_{\{\lvert f\rvert\leq M\}}\indicate_{F}}\\
 & \leq\lebInt{\mu}{\lvert f\rvert\indicate_{\{\lvert f\rvert>M\}}}+\lebInt{\mu}{\lvert f\rvert\indicate_{\{\lvert f\rvert\leq M\}}\indicate_{F}}\\
 & \leq\lebInt{\mu}{\lvert f\rvert\indicate_{\{\lvert f\rvert>M\}}}+\lebInt{\mu}{M\indicate_{\{\lvert f\rvert\leq M\}}\indicate_{F}}\\
 & \leq\frac{\epsilon}{2}+M\mu(\{\lvert f\rvert\leq M\}\cap F)\\
 & \leq\frac{\epsilon}{2}+M\mu(F)
\end{align*}
where all the inequalities follow by the monotonicity of the integral.
Letting $\delta=\frac{\epsilon}{2M}$ then shows uniform absolute
continuity. Letting $F=\X$ then yields 
\[
\lebInt{\mu}{\lvert f\rvert}\leq\frac{\epsilon}{2}+M\mu(\X)
\]
for any $f\in\mathcal{C}$ which gives us the boundedness in $L^{1}$.

Conversely, assume that $\mathcal{C}$ is uniformly absolutely continuous
and $\sup_{f\in\mathcal{C}}\lebInt{\mu}{\lvert f\rvert}<\infty.$
Let $\epsilon>0$ and note that there exists some $\delta>0$ such
that $F\in\F,\mu(F)<\delta\implies\lvert\lebInt{\mu}{f\indicate_{F}}\rvert<\epsilon$.
Note that by Markov's inequality
\[
\mu\left(\lvert f\rvert>M\right)\leq\frac{\lebInt{\mu}{\lvert f\rvert}}{M}.
\]
for any $M\in\N$. Since $\lebInt{\mu}{\lvert f\rvert}<\infty,$we
can take $M$ large enough to so that $\mu\left(\lvert f\rvert>M\right)<\delta$
which by uniform absolute continuity implies $\lvert\lebInt{\mu}{\lvert f\lvert\indicate_{\{\lvert f\rvert>M\}}}\rvert=\lebInt{\mu}{\lvert f\lvert\indicate_{\{\lvert f\rvert>M\}}}<\epsilon.$
This shows uniform integrability, completing the proof.
\end{proof}
Note that uniform integrability generalizes the notion of the domination
by an integrable function, which is an essential ingredient for the
dominated convergence theorem.
\begin{prop}
\label{prop:dominatedImpliesUniformlyIntegrable}Let $\left(\X,\F,\mu\right)$
be a measure space and let $\mathcal{C}\subseteq\measurableFunctions$
be a collection of measurable functions. If for every $f\in\mathcal{C}$,
we have that $\lvert f\rvert\leq g$ such that $g\in\Lp 1{\mu}$ then
$\mathcal{C}$ is uniformly integrable.
\end{prop}

\begin{proof}
Let $\epsilon>0$ be fixed. Since $g$ is integrable, Proposition
\ref{prop:intFiniteFuncFinite} tells us that $\lvert g\rvert<\infty$
almost everywhere. Then
\[
\indicate\left\{ \lvert g\rvert>M\right\} \leq\frac{\lvert g\rvert}{M}\stackrel{\text{a.e}}{\to}0
\]
and so 
\[
\lvert g\rvert\indicate\left\{ \rvert g\rvert>M\right\} \stackrel{\text{a.e}}{\to}0.
\]
Since $\lvert g\rvert\geq\lvert g\rvert\indicate\left\{ \rvert g\rvert>M\right\} $,
dominated convergence implies that 
\[
\lebInt{\mu}{\lvert g\rvert\indicate\left\{ \lvert g\rvert>M\right\} }<\epsilon
\]
for some large enough $M.$ Then, since $\lvert f\rvert\leq\lvert g\rvert,$
\[
\lebInt{\mu}{\lvert f\rvert\indicate\left\{ \lvert f\rvert>M\right\} }\leq\lebInt{\mu}{\lvert g\rvert\indicate\left\{ \lvert g\rvert>M\right\} }<\epsilon.
\]
\end{proof}
Classes of uniformly integrable functions are preserved under linear
combinations.
\begin{prop}
\label{prop:uniformlyIntegrableLinearCombination}Let $\left(\X,\F,\mu\right)$
be a measure space and let $c\in\R$ be a scalar. Further, let $\mathcal{C},\mathcal{D}\subseteq\measurableFunctions$
be uniformly integrable collections of functions. Then 
\[
\mathcal{C}+c\mathcal{D}:=\left\{ f+cg\mid f\in\mathcal{C},g\in\mathcal{D}\right\} 
\]
is a uniformly integrable collection.
\end{prop}

\begin{proof}
Note that when $c=0$ then the result is triviall true so let $c\neq0.$
Fix $\epsilon>0$ and note that there exists some $M$ such that for
any $g\in\mathcal{D}$
\[
\lebInt{\mu}{\lvert g\rvert\indicate_{\{\lvert g\rvert>M\}}}<\frac{\epsilon}{\lvert c\rvert}.
\]
Then, for $N=M\lvert c\rvert$
\[
\lebInt{\mu}{\lvert cg\rvert\indicate_{\{\lvert cg\rvert>N\}}}=\lvert c\rvert\lebInt{\mu}{\lvert g\rvert\indicate_{\{\lvert g\rvert>M\}}}<\epsilon.
\]
Next, observe that \hl{COMPLETE LATER}
\end{proof}
In the main result for this section, we generalize the dominated convergence
theorem.
\begin{thm}[Vitali's Convergence Theorem]
\label{thm:vitaliConvergence}Let $\left(\X,\F,\mu\right)$ be a
finite measure space i.e. $\mu(\X)<\infty.$ If $\left\{ f_{n}\right\} _{n\in\N}\in\measurableFunctions$
are uniformly integrable and $f_{n}\stackrel{\mu}{\longrightarrow}f$
then $f\in\Lp 1{\mu}$ and 
\[
f_{n}\stackrel{\mathcal{L}^{1}}{\longrightarrow}f.
\]
\end{thm}

\begin{proof}
Note that if $f_{n}\stackrel{\mu}{\longrightarrow}f$ then by Proposition
\ref{prop:convInMeasureImpliesSubsequenceAE} there exists some subsequence
$f_{n_{j}}\stackrel{\text{a.e}}{\longrightarrow}f$. By Fatou's lemma
and the fact that absolute value is continuous, we have
\[
\lebInt{\mu}{\lvert f\rvert}=\lebInt{\mu}{\liminf_{j\to\infty}\lvert f_{n_{j}}\rvert}\leq\liminf_{j\to\infty}\lebInt{\mu}{\lvert f_{n_{j}}\rvert}\leq\sup_{n\in\N}\lebInt{\mu}{\lvert f_{n}\rvert}<\infty
\]
where the last inequality follows by Proposition \ref{prop:finiteMeasureUniformIntegrability},
which shows $f\in\Lp 1{\mu}.$ Since $\left\{ f_{n}\right\} _{n\in\N},\left\{ f\right\} $
are uniformly integrable, $g_{n}:=f_{n}-f$ is uniormly integrable
by Proposition \ref{prop:uniformlyIntegrableLinearCombination}. Note
that by uniform absolute continuity, for any $\epsilon>0$, there
exists some $\delta>0$ such that for any $A\in\F:P(A)<\delta\implies\lebInt{\mu}{\lvert g_{n}\rvert\indicate_{A}}<\frac{\epsilon}{2}.$
Note that by convergence in probability, there exists some $n_{0}\in\N$
$\mu\left(\lvert g_{n}\rvert>\frac{\epsilon}{2}\right)<\delta$ and
so
\[
\lebInt{\mu}{\lvert g_{n}\rvert}=\lebInt{\mu}{\lvert g_{n}\rvert\indicate_{\{\lvert g_{n}\rvert\leq\frac{\epsilon}{2}\}}}+\lebInt{\mu}{\lvert g_{n}\rvert\indicate_{\{\lvert g_{n}\rvert>\frac{\epsilon}{2}\}}}\leq\frac{\epsilon}{2}+\frac{\epsilon}{2}=\epsilon
\]
which shows $L^{1}$ convergence.
\end{proof}

\section{The Riemann integral\label{sec:riemannIntegral}}

The theory of integration that we have developed so far is only useful
if (1) it is, in some sense, a suitable generalization of Riemann's
theory of integration, and (2) if it obeys the fundamental theorem
of calculus. We leave (2) to section \ref{sec:ftc} and in this section
review and characterize the Riemann integral. In particular, we give
a precise characterization of Riemann integrable functions using the
Lebesgue theory we have developed so far, and show that for such functions,
their Lebesgue integrals (with respect to the Lebesgue measure) and
Riemann integrals coincide.

We first consider the classical construction of the Riemann integral,
as articulated by Darboux and Riemann. After we have developed this
baisc material carefully, we can present the Riemann integral under
the more modern Lebesgue formulation. We do this by, a more primitive
version of the Lebesgue measure, called the \emph{Jordan content }or
\emph{Jordan measure}, which coincides with the Lebesgue measure for
sufficiently well behaved sets.

\subsection{The classical construction of the Riemann integral}
\begin{defn}
\label{def:partitionInterval}Let $\left[a,b\right]\subset\R$ be
a closed and bounded interval. A \emph{partition $\pi$ }of $\left[a,b\right]$
is a finite set
\[
\pi=\left\{ t_{i}\mid t_{0}=a,t_{k}=b,t_{i}<t_{i+1}\right\} 
\]
where $k\in\N$ is the \emph{size }of the partition. The size of a
partition $\pi$is often denoted $k\left(\pi\right).$
\end{defn}

A second notion of size is given by th concept of a mesh (which is
closely related to, and a special case of, the Lebesgue measure).
\begin{defn}
\label{def:meshPartition}Given a partition $\pi$of $\left[a,b\right]$,
the\emph{ mesh }of $\pi$is given by 
\[
\mesh\left(\pi\right):=\max_{1\leq i\leq k}t_{i}-t_{i-1}.
\]
\end{defn}

As one can see, the mesh is basically given by $\max_{i}\lambda\left(\left[t_{i-1},t_{i}\right]\right)$
where $\lambda$ is the Lebesgue measure. We now introduce the notion
of a \emph{Darboux }sum, which are the building blocks of Riemann
integrals much in the way that integrals of simple functions are in
the Lebesgue theory. In fact, Darboux sums are the Lebesgue integrals
of a special type of signed simple function called a step function.
\begin{defn}
\label{def:darbouxSums}Let $f:\left[a,b\right]\to\R$ be a bounded
function and let $\pi$ be a partition of $\left[a,b\right]$ with
size $k.$ Then the \emph{upper Darboux sum }is given by 
\[
U\left(f,\pi\right):=\sum_{i=1}^{k}\sup_{x\in\left[t_{i-1},t_{i}\right)}f\left(x\right)\left(t_{i}-t_{i-1}\right)
\]
whereas the \emph{lower Darboux sum }is given
\[
L\left(f,\pi\right):=\sum_{i=1}^{k}\inf_{x\in\left[t_{i-1},t_{i}\right)}f\left(x\right)\left(t_{i}-t_{i-1}\right).
\]
\end{defn}

\begin{prop}
\label{prop:darbouxSumInequality}Let $\pi^{\prime}\supseteq\pi$
be a refinement of $\pi$; that is a partition of $\left[a,b\right]$
that contains the coarser partition $\pi$ as a subset. Then
\[
L\left(f,\pi\right)\leq L\left(f,\pi^{\prime}\right)\leq U\left(f,\pi^{\prime}\right)\leq U\left(f,\pi\right).
\]
\end{prop}

\begin{proof}
Note that the second inequality is obvious and so we focus on proving
the first (the third inequality is analagous). First suppose $\pi^{\prime}\setminus\pi=\left\{ t^{*}\right\} $;
that is, the refinement contains only one additional element $t^{*}.$
Then, there is some $1\leq i\leq k$ such that $t_{i-1}<t^{*}<t_{i}.$
Then
\begin{align*}
\inf_{x\in\left[t_{i-1},t_{i}\right)}f\left(x\right)\left(t_{i}-t_{i-1}\right) & \leq\min\left\{ \inf_{x\in\left[t_{i-1},t^{*}\right)}f\left(x\right),\inf_{x\in\left[t^{*},t_{i}\right)}f\left(x\right)\right\} \left(t_{i}-t_{i-1}\right)\\
 & \leq\inf_{x\in\left[t_{i-1},t^{*}\right)}f\left(x\right)\left(t^{*}-t_{i-1}\right)+\inf_{x\in\left[t^{*},t_{i}\right)}f\left(x\right)\left(t_{i}-t^{*}\right)
\end{align*}
and so the first inequality follows since
\begin{align*}
L\left(f,\pi\right) & =\sum_{j\neq i,1\leq j\leq k}\inf_{x\in\left[t_{j-1},t_{j}\right)}f\left(x\right)\left(t_{j}-t_{j-1}\right)+\inf_{x\in\left[t_{i-1},t_{i}\right)}f\left(x\right)\left(t_{i}-t_{i-1}\right)=L\left(f,\pi^{\prime}\right).\\
 & \leq\sum_{j\neq i,1\leq j\leq k}\inf_{x\in\left[t_{j-1},t_{j}\right)}f\left(x\right)\left(t_{j}-t_{j-1}\right)+\inf_{x\in\left[t_{i-1},t^{*}\right)}f\left(x\right)\left(t^{*}-t_{i-1}\right)+\inf_{x\in\left[t^{*},t_{i}\right)}f\left(x\right)\left(t_{i}-t^{*}\right)\\
 & =L\left(f,\pi^{\prime}\right)
\end{align*}
The general case then follows by forming nested sequence $\pi\subset\pi_{1}\subset\ldots\subset\pi_{m}\subset\pi^{\prime}$,
each of which contain one additional element, and then applying the
above result in sequence.
\end{proof}
\begin{defn}
\label{def:riemannIntegrable}A bounded real-valued function $f:\left[a,b\right]\to\R$is
said to be \emph{Riemann integrable }(or just \emph{integrable, }in
context) if
\[
L^{*}\left(f\right):=\sup_{\pi}L\left(f,\pi\right)=\inf_{\pi}U\left(f,\pi\right)=:U^{*}\left(f\right)
\]
where the supremum and infimum are taken over the collection of all
finite partitions $\pi$ of $\left[a,b\right]$. The \emph{Riemann
integral }of an integrable $f$ on $\left[a,b\right]$ is then denoted
\[
\int_{a}^{b}f\left(x\right)dx:=L^{*}\left(f\right)=U^{*}\left(f\right).
\]
The collection of Riemann integrable functions on $\left[a,b\right]$
is denoted $\mathcal{R}\left[a,b\right].$
\end{defn}

\begin{prop}
\label{prop:upperAndLowerSumRelation}Let $f:\left[a,b\right]\to\R$
be a bounded function. Then 
\[
L^{*}\left(f\right)\leq U^{*}\left(f\right)
\]
and
\[
L^{*}\left(f\right)=-U^{*}\left(-f\right).
\]
\end{prop}

\begin{proof}
For the first claim, note that for any partitions $\pi$and $\pi^{\prime}$
, Proposition \ref{prop:darbouxSumInequality} implies that
\[
L\left(f,\pi\right)\leq L\left(f,\pi^{\prime}\cup\pi\right)\leq U\left(f,\pi^{\prime}\cup\pi\right)\leq U\left(f,\pi^{\prime}\right)
\]
and so 
\[
L\left(f,\pi\right)\leq\inf_{\pi^{\prime}}U\left(f,\pi^{\prime}\right)
\]
and similarly
\[
\sup_{\pi}L\left(f,\pi\right)\leq\inf_{\pi^{\prime}}U\left(f,\pi^{\prime}\right).
\]
For the second claim, notice that $\inf_{x\in\left[t_{i-1},t_{i}\right)}f\left(x\right)=-\sup_{x\in\left[t_{i-1},t_{i}\right)}-f\left(x\right)$
and so for any partition $\pi$
\[
L\left(f,\pi\right)=-U\left(-f,\pi\right)
\]
and the result follows by noting that 
\[
L^{*}\left(f\right)=\sup_{\pi}L\left(f,\pi\right)=\sup_{\pi}-U\left(-f,\pi\right)=-\inf_{\pi}U\left(-f,\pi\right)=-U^{*}\left(-f\right).
\]
\end{proof}
\begin{thm}
\label{thm:riemannIntegrability}The following are equivalent for
any bounded function $f:\left[a,b\right]\to\R$

\begin{enumerate}[label=(\roman*),leftmargin=.1\linewidth,rightmargin=.4\linewidth]
	\item $ f \in \mathcal{R}\left[a,b\right]$
	\item For every $\epsilon > 0$ there exists some partition $\pi$ such that 
		\[
			U\left(f,\pi\right) - L\left(f,\pi\right) < \epsilon.
		\]
	\item For every $\epsilon >0$ there exists some $\delta > 0$ such that for any partition $\pi$ with $\mesh \left(\pi\right) < \delta$
		\[
			U\left(f,\pi\right) - L\left(f,\pi\right) < \epsilon
		\]
	\item The limit 
	\[
		I := \lim_{\mesh\left(\pi\right) \to 0}\sum_{i: t_i \in \pi } f\left( c_i \right)\left(t_i - t_{i-1}\right)
	\]
	exists for any choice of intermediate values $t_{i-1} \leq c_i \leq t_i $. Further, this limit is equal to $I = L^*(f)=U^*(f)$.
\end{enumerate}
\end{thm}

\begin{proof}
We first prove that (i)$\implies$(ii). Suppose that $f\in\mathcal{R}\left[a,b\right]$,
and so $L^{*}\left(f\right)=U^{*}\left(f\right).$ Fix $\epsilon>0$
and observe that by the definition of supremum and infimum, there
exist partitions $\pi$ and $\pi^{\prime}$ such that $L^{*}\left(f\right)-L\left(f,\pi\right)<\frac{\epsilon}{2}$
and $U\left(f,\pi^{\prime}\right)-L^{*}\left(f\right)<\frac{\epsilon}{2}$.
Then clearly, for the common refinement $\pi\cup\pi^{\prime}$
\[
\lvert L\left(f,\pi\cup\pi^{\prime}\right)-U\left(f,\pi\cup\pi^{\prime}\right)\rvert\leq\lvert L\left(f,\pi\cup\pi^{\prime}\right)-L^{*}\left(f\right)\rvert+\lvert L^{*}\left(f\right)-U\left(f,\pi\cup\pi^{\prime}\right)\rvert<\epsilon.
\]

Next, we prove (ii) $\implies$(iii). Fix $\epsilon>0$ and note that
by (ii) there exists some partition $\pi:=\left\{ a=t_{0}<t_{1}<\ldots<t_{k}=b\right\} $such
that 
\begin{equation}
U\left(f,\pi\right)-L\left(f,\pi\right)=\sum_{i=1}^{k}\left(\sup_{x\in\left[t_{i-1},t_{i}\right)}f\left(x\right)-\inf_{x\in\left[t_{i-1},t_{i}\right)}f\left(x\right)\right)\left(t_{i}-t_{i-1}\right)<\frac{\epsilon}{2}.\label{eq:assumption(ii)}
\end{equation}
Now $\pi^{\prime}:=\left\{ a=t_{0}^{\prime}<t_{1}^{\prime}<\ldots<t_{l}^{\prime}=b\right\} $
be a partition such that $\mesh\left(\pi\right)<\delta:=\frac{\epsilon}{4k\lVert f\rVert_{\infty}}$
where $\lVert f\rVert_{\infty}=\sup_{x\in\left[a,b\right]}\lvert f\left(x\right)\rvert$
. Then for each $t_{i}\in\pi$, there can be at most one interval
$\left[t_{j-1}^{\prime},t_{j}^{\prime}\right)$ with $1\leq j\leq l$
that contains $t_{i}$ since these intervals form a partition of $\left[a,b\right].$
Thus the set $S:=\left\{ j\in\left\{ 1,\ldots,l\right\} \mid\text{\ensuremath{\exists i\in\left\{ 0,1,\ldots,k\right\} } }\text{s.t }t_{i}\in\left[t_{j-1}^{\prime},t_{j}^{\prime}\right)\right\} $
has size at most $k.$ Now consider the difference of Darboux sums
\begin{align*}
U\left(f,\pi^{\prime}\right)-L\left(f,\pi^{\prime}\right) & =\sum_{j=1}^{l}\left(\sup_{x\in\left[t_{j-1}^{\prime},t_{j}^{\prime}\right)}f\left(x\right)-\inf_{x\in\left[t_{j-1}^{\prime},t_{i}^{\prime}\right)}f\left(x\right)\right)\left(t_{j}^{\prime}-t_{j-1}^{\prime}\right)\\
 & =\sum_{j\in S}\left(\sup_{x\in\left[t_{j-1}^{\prime},t_{j}^{\prime}\right)}f\left(x\right)-\inf_{x\in\left[t_{j-1}^{\prime},t_{i}^{\prime}\right)}f\left(x\right)\right)\left(t_{j}^{\prime}-t_{j-1}^{\prime}\right)\\
 & \ \ \ +\sum_{j\in S^{C}}\left(\sup_{x\in\left[t_{j-1}^{\prime},t_{j}^{\prime}\right)}f\left(x\right)-\inf_{x\in\left[t_{j-1}^{\prime},t_{i}^{\prime}\right)}f\left(x\right)\right)\left(t_{j}^{\prime}-t_{j-1}^{\prime}\right).
\end{align*}
We control each of these sums separately. For the first sum, note
that there at most $k$ terms in the sum, where $\left(\sup_{x\in\left[t_{j-1}^{\prime},t_{j}^{\prime}\right)}f\left(x\right)-\inf_{x\in\left[t_{j-1}^{\prime},t_{i}^{\prime}\right)}f\left(x\right)\right)\leq2\lVert f\rVert_{\infty}$
and $\left(t_{j}^{\prime}-t_{j-1}^{\prime}\right)<\delta$ for every
$j\in S.$ Thus the first term is bounded above by $\frac{\epsilon}{2}.$
Next, note that for the second sum, by definition for every $j\in S^{C},$
there exists some $i_{j}\in\left\{ 1,\ldots,k\right\} $ such that
$\left[t_{j-1}^{\prime},t_{j}^{\prime}\right)\subset\left[t_{i_{j}-1},t_{i_{j}}\right)$
and that
\[
\sum_{j\in S^{C}}\left(\sup_{x\in\left[t_{j-1}^{\prime},t_{j}^{\prime}\right)}f\left(x\right)-\inf_{x\in\left[t_{j-1}^{\prime},t_{i}^{\prime}\right)}f\left(x\right)\right)\left(t_{j}^{\prime}-t_{j-1}^{\prime}\right)\leq\sum_{j\in S^{C}}\left(\sup_{x\in\left[t_{i_{j}-1},t_{i_{j}}\right)}f\left(x\right)-\inf_{x\in\left[t_{i_{j}-1},t_{i_{j}}\right)}f\left(x\right)\right)\left(t_{i_{j}}-t_{i_{j-1}}\right)<\frac{\epsilon}{2}
\]
where the final inequality is due to (\ref{eq:assumption(ii)}). This
means 
\[
\lvert U\left(p,\pi^{\prime}\right)-L\left(f,\pi^{\prime}\right)\rvert<\epsilon
\]
 which completes this part of the proof.

Next, assume that (iii) holds and let $\epsilon>0$ . Observe that
there exists some $\delta>0$ such that for any partition $\pi:=\left\{ a=t_{0}<t_{1}<\ldots t_{k\left(\pi\right)}=b\right\} $
with mesh less than $\delta$, 
\[
0\leq\sum_{i=1}^{k\left(\pi\right)}f\left(c_{i}\right)\left(t_{i}-t_{i-1}\right)-L\left(f,\pi\right)\leq U\left(f,\pi\right)-L\left(f,\pi\right)<\frac{\epsilon}{2}
\]
where $c_{i}\in\left[t_{i-1},t_{i}\right).$ Now note that since $L\left(f,\pi_{n}\right)\leq L^{*}\left(f\right)\leq U^{*}\left(f\right)\leq U\left(f,\pi_{n}\right)$,
we have that $L^{*}\left(f\right)-L\left(f,\pi_{n}\right)<\frac{\epsilon}{2}$
(which shows that $L\left(f,\pi_{n}\right)\to L^{*}\left(f\right)$).
Then
\[
\left\lvert \sum_{i=1}^{k_{n}}f\left(c_{i}^{n}\right)\left(t_{i}^{n}-t_{i-1}^{n}\right)-L^{*}\left(f\right)\right\rvert \leq\left\lvert \sum_{i=1}^{k_{n}}f\left(c_{i}^{n}\right)\left(t_{i}^{n}-t_{i-1}^{n}\right)-L\left(f,\pi_{n}\right)\right\rvert +\left\lvert L\left(f,\pi\right)-L^{*}\left(f\right)\right\rvert <\epsilon.
\]
Similarly, we know tht 
\[
U\left(f,\pi\right)-\sum_{i=1}^{k\left(\pi\right)}f\left(c_{i}\right)\left(t_{i}-t_{i-1}\right)\leq U\left(f,\pi\right)-L\left(f,\pi\right)<\frac{\epsilon}{2}
\]
and so $U\left(f,\pi\right)-U^{*}\left(f\right)<\frac{\epsilon}{2}$
and
\[
\left\lvert \sum_{i=1}^{k_{n}}f\left(c_{i}^{n}\right)\left(t_{i}^{n}-t_{i-1}^{n}\right)-U^{*}\left(f\right)\right\rvert \leq\left\lvert \sum_{i=1}^{k_{n}}f\left(c_{i}^{n}\right)\left(t_{i}^{n}-t_{i-1}^{n}\right)-U\left(f,\pi_{n}\right)\right\rvert +\left\lvert U\left(f,\pi\right)-U^{*}\left(f\right)\right\rvert <\epsilon
\]
and so the uniqueness of limits shows (iv).

Finally, Suppose that (iv) hods i.e. that $I=\lim_{\mesh\left(\pi\right)\to0}\sum_{i=1}^{k\left(\pi\right)}f\left(c_{i}^{\pi}\right)\left(t_{i}^{\pi}-t_{i-1}^{\pi}\right)$
exists and is independent of any choice $t_{i-1}^{\pi}\leq c_{i}^{\pi}<t_{i}^{\pi}$.
Fix $\epsilon>0$ and notice that by assumption there exists some
$\delta>0$ such that for any partition $\pi$ with $\mesh\left(\pi\right)<\delta$
\[
\left\lvert I-\sum_{i=1}^{k\left(\pi\right)}f\left(c_{i}^{\pi}\right)\left(t_{i}^{\pi}-t_{i-1}^{\pi}\right)\right\rvert <\frac{\epsilon}{3}.
\]
Crucially, the choice of intermediate values $c_{i}^{\pi}$\emph{
does not }influence our choice of $\delta.$ Then, by Proposition
\ref{prop:darbouxSumInequality} and the definition of a supremum,
we can actually choose a partition $\pi_{0}$ fine enough, such that
its mesh is less than $\delta$ and
\[
\lvert L\left(f,\pi_{0}\right)-L^{*}\left(f\right)\rvert<\frac{\epsilon}{3}.
\]
Finally, notice that by the definition of infimum, we can choose the
intermediates $c_{i}^{\pi_{0}}$ such that that $f\left(c_{i}^{\pi_{0}}\right)-\inf_{x\in\left[t_{i-1}^{\pi_{0}},t_{i}^{\pi_{0}}\right)}f\left(x\right)<\frac{\epsilon}{3k\left(\pi_{0}\right)\mesh\left(\pi_{0}\right)}$
and so 
\begin{align*}
\left\lvert I-L^{*}\left(f\right)\right\rvert  & \leq\left\lvert I-\sum_{i=1}^{k\left(\pi_{0}\right)}f\left(c_{i}^{\pi_{0}}\right)\left(t_{i}^{\pi_{0}}-t_{i-1}^{\pi_{0}}\right)\right\rvert +\left\lvert \sum_{i=1}^{k\left(\pi_{0}\right)}f\left(c_{i}^{\pi_{0}}\right)\left(t_{i}^{\pi_{0}}-t_{i-1}^{\pi_{0}}\right)-L\left(f,\pi_{0}\right)\right\rvert +\left\lvert L\left(f,\pi_{0}\right)-L^{*}\left(f\right)\right\rvert \\
< & \epsilon.
\end{align*}
Since $\epsilon$is arbitrary, we have $I=L^{*}\left(f\right).$ A
similar argument shows that $I=U^{*}\left(f\right)$ which completes
the proof.
\end{proof}
Note that the existence of \emph{particular }sums of the form $\sum f\left(c_{i}^{\pi}\right)\left(t_{i}^{\pi}-t_{i-1}^{\pi}\right)$
that converge to a limit when the mesh goes to zero does not imply
the integrability of $f.$ We need the convergence to hold for \emph{any
}choice of intermediate values and \emph{any }set of partitions with
mesh going to zero. This is highlighted in the next example, which
is also the canonical example of a function that is not Riemann integrable.
\begin{example}
\label{exa:nonRiemannIntegrableFunction}Let $f:\left[0,1\right]\to\R$
be given by $f\left(x\right)=\indicate\left\{ x\in\left[0,1\right]\cap\mathds{Q}\right\} .$Then
$L\left(f,\pi\right)=0$ for any partition $\pi$given the density
of the rationals and the irrationals in $\left[0,1\right].$ Therefore
$L^{*}\left(f\right)=0.$ A similar argument shows that $U^{*}\left(f\right)=1$
and so $f$ is not integrable. On the other hand, the Riemann sums
given by 
\[
\sum_{i=1}^{k}f\left(c_{i}\right)\left(\frac{i}{k}-\frac{i-1}{k}\right)=\frac{1}{k}\sum_{i=1}^{k}f\left(c_{i}\right)
\]
can be made to converge to 0, for instance, by choosing all intermediate
points as irrational (this is possible to do again by the density
of irrationals). Nevertheless the integral does not exist.
\end{example}

\begin{example}
\label{exa:isi2008samplepsb2} For $k\geq1$, let 
\[
a_{k}=\lim_{n\rightarrow\infty}\frac{1}{n}\sum_{m=1}^{kn}\exp\left(-\frac{1}{2}\frac{m^{2}}{n^{2}}\right).
\]

Find $\lim_{k\rightarrow\infty}a_{k}$.\hl{TODO}
\end{example}

\begin{thm}
\label{thm:lebesgueRiemannEqual}Suppose $f:\left[a,b\right]\to\R$
is Riemann integrable then $f\in L^{1}\left(\left[a,b\right],\borel\left(\left[a,b\right]\right),\lambda\right)$
and 
\[
\lebInt{\lambda}f=L^{*}\left(f\right)=U^{*}\left(f\right).
\]
\end{thm}

\begin{proof}
Let $\left\{ \pi_{i}\right\} _{i=1}^{\infty}$ be a sequences of partitions
such that $\pi_{i}\subset\pi_{i+1}$ and $\mesh\left(\pi_{i}\right)\to0$.
Notice we can define the Darboux functions -- which are step functions
-- as
\[
l\left(f,\pi_{i}\right)\left(x\right):=\sum_{j=1}^{k\left(\pi_{i}\right)}\inf_{c\in\left[t_{j-1}^{\pi_{i}},t_{j}^{\pi_{i}}\right)}f\left(c\right)\indicate\left\{ x\in\left[t_{j-1}^{\pi_{i}},t_{j}^{\pi_{i}}\right)\right\} 
\]
and
\[
u\left(f,\pi_{i}\right)\left(x\right):=\sum_{j=1}^{k\left(\pi_{i}\right)}\sup_{c\in\left[t_{j-1}^{\pi_{i}},t_{j}^{\pi_{i}}\right)}f\left(c\right)\indicate\left\{ x\in\left[t_{j-1}^{\pi_{i}},t_{j}^{\pi_{i}}\right)\right\} .
\]
For brevity we will write $l_{i}:=l\left(f,\pi_{i}\right)$ and $u_{i}:=u\left(f,\pi_{i}\right)$.
Note then that for all $i\in\N$
\[
l_{i}\leq l_{i+1}\leq\sup_{i}l_{i}\leq f\leq\inf_{i}u_{i}\leq u_{i+1}\leq u_{i}
\]
pointwise by our construction and the nested nature of the partitions
$\pi_{i}$ (see Proposition \ref{prop:darbouxSumInequality} for the
argument). Notice further that step functions are simple functions
on a finite measure space and hence integrable. Moreover, since $f$
is bounded (Riemann integrable functions are bounded by definition),
$\max\left\{ \lvert u_{i}\rvert,\lvert l_{i}\rvert\right\} \leq\lVert f\rVert_{\infty}$
and so by the dominated convergence theorem
\[
\lebInt{\lambda}{l_{i}}\to\lebInt{\lambda}{\sup l_{i}}
\]
and 
\[
\lebInt{\lambda}{u_{i}}\to\lebInt{\lambda}{\inf u_{i}}.
\]
But notice that $\lebInt{\lambda}{l_{i}}=L\left(f,\pi_{i}\right)$
and $\lebInt{\lambda}{u_{i}}=U\left(f,\pi_{i}\right)$ and so by Riemann
integrability
\[
\lebInt{\lambda}{\sup l_{i}}=L^{*}\left(f\right)=U^{*}\left(f\right)=\lebInt{\lambda}{\inf u_{i}}.
\]
Note further that $\inf_{i}u_{i}-\sup_{i}l_{i}\geq0$ and so $\lebInt{\lambda}{\inf_{i}u_{i}-\sup_{i}l_{i}}=0$
implies that $\inf_{i}u_{i}\stackrel{\text{a.e}}{=}\sup_{i}l_{i}$
by Proposition \ref{prop:intEqualFuncEqual}. Of course, then $f\stackrel{\text{a.e}}{=}\inf_{i}u_{i}$
and so $f$ is measurable (and Lebesgue integrable since its bounded
in a finite measure space). Moreover, Proposition \ref{prop:funcEqualityAlmostEverywhere}
implies
\[
\lebInt{\lambda}f=L^{*}\left(f\right)=U^{*}\left(f\right)
\]
which completes the proof.
\end{proof}
\begin{cor}
\label{cor:charRiemannIntegrable}Let $f:\left[a,b\right]\to\R$ be
a bounded functon. Then $f$ is Riemann integrable if and only if
for any sequence of nested partitions $\left\{ \pi_{i}\right\} _{i=1}^{\infty}$
with $\pi_{i}\subseteq\pi_{i+1}$ and $\mesh\left(\pi_{i}\right)\to0$
\[
\sup_{i}l\left(f,\pi_{i}\right)\stackrel{\text{a.e}}{=}\inf_{i}u\left(f,\pi_{i}\right).
\]
\end{cor}

\begin{proof}
Note that the ``only if'' direction follows from Proposition \ref{thm:lebesgueRiemannEqual}.
To see the converse, fix $\epsilon>0$ and let $N:=\left\{ x\in\left[a,b\right]\mid\sup_{i}l_{i}\left(x\right)\neq\inf_{i}u_{i}\left(x\right)\right\} $.
Note that for any $x\in[a,b]\setminus N$ we have pointwise convergence
\[
\lim_{i\to\infty}l_{i}\left(x\right)=\sup_{i}l_{i}\left(x\right)
\]
and
\[
\lim_{i\to\infty}u_{i}\left(x\right)=\inf_{i}u_{i}\left(x\right).
\]
Then 
\[
\lim_{i\to\infty}\underbrace{u_{i}\left(x\right)-l_{i}\left(x\right)}_{=:g_{i}\left(x\right)}=0
\]
for $x\in\left[a,b\right]\setminus N$. But notice that $\lvert g_{i}\rvert\leq2\lVert f\rVert_{\infty}$
and so by (generalized) dominated convergence
\[
U^{*}\left(f\right)=L^{*}\left(f\right)
\]
which completes the proof.
\end{proof}
Now we are finally ready to characterize Riemann integrability in
terms of continuity.
\begin{thm}
\label{thm:riemannIntegrableAEContinuous}Let $f:\left[a,b\right]\to\R$
be a bounded function. Then $f$ is Riemann integrable if and only
if $f$ is continuous almost everywhere.
\end{thm}

\begin{proof}
Let $\left\{ \pi_{i}\right\} _{i=1}^{\infty},$ $l_{i},$ and $u_{i}$
be defined as before and let $c\in\left[a,b\right]$ be arbitrary.
First, suppose that $f$ is Riemann integrable. Fix $\epsilon>0$
and notice that by the definition of a partition, for every $i\in\N$
there exists some $1\leq j_{i}\leq k\left(\pi_{i}\right)$ such that
$c\in\left[t_{j_{i}-1}^{\pi_{i}},t_{j_{i}}^{\pi_{i}}\right)$. Further,
note that there is some $i_{c}\in\N$ such that for all $i\geq i_{c}$
\[
u_{i}\left(c\right)-\inf_{i}u_{i}\left(c\right)<\frac{\epsilon}{2}
\]
and
\[
\sup_{i}l_{i}\left(c\right)-l_{i}\left(c\right)<\frac{\epsilon}{2}.
\]
Adding the two inequalities and re-arranging, we have that 
\[
u_{i}\left(c\right)-l_{i}\left(c\right)<\epsilon+\inf_{i}u_{i}\left(c\right)-\sup_{i}l_{i}\left(c\right).
\]
For any $x$ such that $\lvert x-c\rvert<\delta:=\frac{\min\left\{ t_{j_{i_{c}}}^{\pi_{i_{c}}}-c,c-t_{j_{i_{c}}-1}^{\pi_{i_{c}}}\right\} }{2}$
we have that $x,c\in\left[t_{j_{i_{c}}-1}^{\pi_{i_{c}}},t_{j_{i_{c}}}^{\pi_{i_{c}}}\right)$
and so
\[
\lvert f\left(x\right)-f\left(c\right)\rvert\leq u_{i_{c}}\left(c\right)-l_{i_{c}}\left(c\right)<\epsilon+\inf_{i}u_{i}\left(c\right)-\sup_{i}l_{i}\left(c\right).
\]
Thus $f$ is continuous at $c$ if $c\notin N$ which is a null set
if and only if $f$ is Riemann integrable. So we have proved that
a Riemann integrable function is continuous almost everywhere.

Conversely, suppose that the set of discontinuities $D:=\left\{ x\in\left[a,b\right]\mid f\ \text{is not continuous at }x\right\} $
is such that $\lambda\left(D\right)=0.$ Then, for any $c\in\left[a,b\right]\setminus D$
and any $\epsilon>0$, there is a $\delta_{c}>0$ such that $\lvert x-c\rvert<\delta_{c}\implies\lvert f\left(x\right)-f\left(c\right)\rvert<\frac{\epsilon}{4}.$
Let $\left\{ \pi\right\} _{i=1}^{\infty}$ be as before, and note
that 
\[
\inf_{i}u_{i}\left(c\right)-\sup_{i}l_{i}\left(c\right)\leq u_{i}\left(c\right)-l_{i}\left(c\right)=\sup_{x\in\left[t_{j_{i}-1}^{\pi_{i}},t_{j_{i}}^{\pi_{i}}\right)}f\left(x\right)-\inf_{x\in\left[t_{j_{i}-1}^{\pi_{i}},t_{j_{i}}^{\pi_{i}}\right)}f\left(x\right)
\]
for any $i\in\N$where $j_{i}$ and $t_{j_{i}}^{\pi_{i}}$ etc are
defined as before. Note that there exists some $i_{0}$ such that
for all $i\geq i_{0}$ $\mesh\left(\pi_{i}\right)<\delta_{c}$. Then,
note that we can find $y,z\in\left[t_{j_{i}-1}^{\pi_{i}},t_{j_{i}}^{\pi_{i}}\right)$
such that
\[
\sup_{x\in\left[t_{j_{i}-1}^{\pi_{i}},t_{j_{i}}^{\pi_{i}}\right)}f\left(x\right)-f\left(y\right)<\frac{\epsilon}{4}
\]
and
\[
f\left(z\right)-\inf_{x\in\left[t_{j_{i}-1}^{\pi_{i}},t_{j_{i}}^{\pi_{i}}\right)}f\left(x\right)<\frac{\epsilon}{4}.
\]
Further, $\lvert y-c\rvert<\delta_{c}$ and $\lvert z-c\rvert<\delta_{c}$
and so
\begin{align*}
\inf_{i}u_{i}\left(c\right)-\sup_{i}l_{i}\left(c\right) & \leq\sup_{x\in\left[t_{j_{i}-1}^{\pi_{i}},t_{j_{i}}^{\pi_{i}}\right)}f\left(x\right)-\inf_{x\in\left[t_{j_{i}-1}^{\pi_{i}},t_{j_{i}}^{\pi_{i}}\right)}f\left(x\right)\\
 & \leq\sup_{x\in\left[t_{j_{i}-1}^{\pi_{i}},t_{j_{i}}^{\pi_{i}}\right)}f\left(x\right)-f\left(y\right)+\lvert f\left(y\right)-f\left(c\right)\rvert+\lvert f\left(c\right)-f\left(z\right)\rvert+f\left(z\right)-\inf_{x\in\left[t_{j_{i}-1}^{\pi_{i}},t_{j_{i}}^{\pi_{i}}\right)}f\left(x\right)\\
 & <\epsilon.
\end{align*}
Since $\epsilon$ and $c$ were arbitrary,
\[
\inf_{i}u_{i}\stackrel{\text{a.e}}{=}\sup_{i}l_{i}
\]
which implies, by Corollary \ref{cor:charRiemannIntegrable}, that
$f$ is Riemann integrable. 
\end{proof}

\subsection{The improper Riemann integral}

\subsection{The Jordan content}



\chapter{Spaces of functions\label{chap:spaces_of_functions}}

\section{$\mathcal{L}^{p}$ spaces as almost Banach spaces over $\protect\R$}

The central objects of study in analysis are functions, and the study
of sets of functions is what broadly characterizes functional analysis.
Often, the types of function spaces studied in analysis are \emph{vector
spaces }over some field (usually $\R$ or $\mathds{C})$. We have
already seen one such space in Chapter 3: the $\mathcal{L}^{1}$ space
or the space of integrable functions. $\mathcal{L}^{1}$ spaces can
be suitably generalized to allow for $p-$th power integrability
\begin{defn}
\label{def:LpSpace}Let $\left(\X,\F,\mu\right)$ be a measure space.
For any $p\in\left[1,\infty\right),$ the spaces 
\[
\mathcal{L}^{p}\left(\X,\F,\mu\right):=\left\{ f\in\measurableFunctions\mid\lebInt{\mu}{\lvert f\rvert^{p}}<\infty\right\} 
\]
are called $\mathcal{L}^{p}$ spaces over $\R$.
\end{defn}

For $p=1$ we get our original integrable functions. We will soon
show that $\mathcal{L}^{p}$ spaces are vector spaces; in-fact, they
are (semi)-normed vector spaces with the (semi)-norm given by the
function
\[
\pnorm fp:=\lebInt{\mu}{\lvert f\rvert^{p}}^{\frac{1}{p}}.
\]
It should be clear that the function defined above satisfies the absolute
homogeneity aspect of norms i.e. for any $\alpha\in\R$ 
\[
\pnorm{\alpha f}p=\lvert\alpha\rvert\pnorm fp.
\]
However, our function in question fails being positive definite in
that $\pnorm fp=0\not\implies f=0.$ In fact, all functions that are
almost everywhere equal to zero are mapped to zero under this function,
which means that it cannot be a bonafide norm. Hence, if we can show
that the function satisfies the triangle inequality, we can show that
$\mathcal{L}^{p}$ is a semi-normed vector space. To do this, we will
establish a series of results that will be important in their own
right, but shall also help us get the triangle inequality
\begin{lem}
\label{lem:amGmInequality}Let $\left\{ a_{i}\right\} _{i=1}^{k}\in\left[0,\infty\right)$
and let $\theta_{1},\ldots,\theta_{k}>0$ be real numbers such that
\[
\sum_{i=1}^{k}\theta_{i}=1.
\]
Then,
\[
\prod_{i=1}^{k}a_{i}^{\theta_{i}}\leq\sum_{i=1}^{k}a_{i}\theta_{i}.
\]
\end{lem}

\begin{proof}
Note that since the $\log\left(\cdot\right)$ function is concave,
by induction on the definition of concavity, we have that
\[
\log\left(\prod_{i=1}^{k}a_{i}^{\theta_{i}}\right)=\sum_{i=1}^{k}\theta_{i}\log\left(a_{i}\right)\leq\log\left(\sum_{i=1}^{k}\theta_{i}a_{i}\right).
\]
Since $e^{x}$ is a monotonic function, the inequality is preserved
under exponentiation which yields are result.
\end{proof}
\begin{rem*}
This lemma is often referred to as the inequality between arithmetic
and geometric means as the right hand side is the standard weighted
average i.e. the arithmetic mean whereas the left hand side is the
geometric mean.
\end{rem*}
\begin{prop}
\label{prop:generalizedHolder}Let functions $f_{1},\ldots,f_{k}\in\measurableFunctions$
and let $\theta_{1},\ldots,\theta_{k}>0$ be real numbers such that
\[
\sum_{i=1}^{k}\theta_{i}=1.
\]
Then, for any measure $\mu$ on $\X$
\[
\lebInt{\mu}{\prod_{i=1}^{k}f_{i}^{\theta_{i}}}\leq\prod_{i=1}^{k}\lebInt{\mu}{f_{i}}^{\theta_{i}}.
\]
\end{prop}

\begin{proof}
First notice that if $\lebInt{\mu}{f_{i}}=0$ for any $i\in\left\{ 1,\ldots,k\right\} $
then by Proposition \ref{prop:intZeroFuncZero}, $f_{i}=0$ almost
everywhere, which would imply that the left hand side is identically
zero. This would make the inequality hold trivially. Conversely, if
any $\lebInt{\mu}{f_{i}}=\infty$ and all $\mu\left(f_{i}\right)>0$
then the right hand side is identically $\infty$ which would again
let the inequality hold trivially.

Thus, without loss of generality, assume that $0<\lebInt{\mu}{f_{i}}<\infty$
for all $i$ and define 
\[
f_{i}^{*}=\frac{f_{i}}{\lebInt{\mu}{f_{i}}}.
\]
Clearly, $\lebInt{\mu}{f_{i}^{*}}=1$ and moreover if
\[
\lebInt{\mu}{\prod_{i=1}^{k}f_{i}^{*\theta_{i}}}=\frac{\lebInt{\mu}{\prod_{i=1}^{k}f_{i}^{\theta_{i}}}}{\prod_{i=1}^{k}\lebInt{\mu}{f_{i}}^{\theta_{i}}}\leq\prod_{i=1}^{k}\lebInt{\mu}{f_{i}^{*}}^{\theta_{i}}=1
\]
then our claim follows. To show this, note that by Lemma \ref{lem:amGmInequality}
\[
\prod_{i=1}^{k}f_{i}^{*\theta_{i}}\leq\sum_{i=1}\theta_{i}f_{i}^{*}
\]
pointwise. Integrating both sides, we have 
\begin{align*}
\lebInt{\mu}{\prod_{i=1}^{k}f_{i}^{*\theta_{i}}} & \leq\lebInt{\mu}{\sum_{i=1}^{k}\theta_{i}f_{i}^{*}}\\
 & =\sum_{i=1}^{k}\theta_{i}\lebInt{\mu}{f_{i}^{*}}\\
 & =\sum_{i=1}^{k}\theta_{i}\\
 & =1
\end{align*}
where the first inequality follows from the monotonicty of the integral
and the first equality from the linearity of integration. This completes
the proof.
\end{proof}
\begin{cor}[H\"{o}lder's inequality]
\label{cor:holdersInequality}Let $\left(\X,\F,\mu\right)$ be a
measure space. For any real numbers $p,q\in\left(1,\infty\right)$
such that $\frac{1}{p}+\frac{1}{q}=1$ and functions $g,h\in\measurableFunctions,$
we have that
\[
\pnorm{gh}1\leq\pnorm gp\pnorm hq.
\]
\end{cor}

\begin{proof}
Let $k=2$ , $f_{1}=\lvert g\rvert^{p},f_{2}=\lvert h\rvert^{q},$
$\theta_{1}=\frac{1}{p},$ and $\theta_{2}=\frac{1}{q}$ and apply
Proposition \ref{prop:generalizedHolder}.
\end{proof}
Now we can finally establish the triangle inequality for the so-called
$p-$norms.
\begin{thm}[Minkowski's inequality]
\label{thm:minkowskiInequality}Let $f,g\in\Lp p{\X,\F,\mu}$ for
some $p\in\left[1,\infty\right).$ Then, $f+g\in\Lp p{\mu}$ and 
\[
\pnorm{f+g}p\leq\pnorm fp+\pnorm gp.
\]
\end{thm}

\begin{proof}
Note that if $\lebInt{\mu}{\lvert f+g\rvert^{p}}=0$ then the inequality
follows trivially so let's assume that $\lebInt{\mu}{\lvert f+g\rvert^{p}}>0$.
Then
\begin{align*}
\lebInt{\mu}{\lvert f+g\lvert^{p}} & =\lebInt{\mu}{\lvert f+g\rvert\lvert f+g\rvert^{p-1}}\\
 & \leq\lebInt{\mu}{\lvert f\rvert\lvert f+g\rvert^{p-1}+\lvert g\rvert\lvert f+g\rvert^{p-1}}\\
 & =\lebInt{\mu}{\lvert f\rvert\lvert f+g\rvert^{p-1}}+\lebInt{\mu}{\lvert g\rvert\lvert f+g\rvert^{p-1}}\\
 & =\pnorm{f\lvert f+g\rvert^{p-1}}1+\pnorm{g\lvert f+g\rvert^{p-1}}1\\
 & \leq\pnorm fp\pnorm{\lvert f+g\rvert^{p-1}}{\frac{p}{p-1}}+\pnorm gp\pnorm{\lvert f+g\rvert^{p-1}}{\frac{p}{p-1}}
\end{align*}
where the first inequality follows from the triangle inequality of
$\lvert\cdot\rvert$ and the monotonicty of integration and the second
inequality from H\"{o}lder's inequality above. Dividing both sides
by $\pnorm{\lvert f+g\rvert^{p-1}}{\frac{p}{p-1}}=\lebInt{\mu}{\lvert f+g\rvert^{p}}^{1-\frac{1}{p}}=\frac{\lebInt{\mu}{\lvert f+g\lvert^{p}}}{\pnorm{f+g}p}$
yields the result. Of course, this then shows that $f+g\in\Lp p{\mu}.$
\end{proof}
For finite measures , $\mathcal{L}^{p}$ spaces enjoy a nesting property
\begin{prop}
\label{prop:nestingLpSpace}Let $(\X,\F,\mu)$ be a measure space
such that $0<\mu(\X)<\infty.$ Then for $1\leq q<p<\infty$
\[
\Lp p{\mu}\subseteq\Lp q{\mu}
\]
and there exists some $C>0$ such that for any $f\in\Lp p{\mu}$
\[
C\pnorm fp\geq\pnorm fq.
\]
\end{prop}

\begin{proof}
By H\"{o}lder's inequality, we have for any$f\in\Lp p{\mu}$
\[
\pnorm{\lvert f\rvert{}^{q}}1\leq\pnorm{\lvert f\rvert{}^{q}}s\pnorm{\indicate_{\X}}{\frac{s}{s-1}}
\]
for any $s\in(1,\infty)$. In particular, for $s=\frac{p}{q}$, the
inequality is
\[
\pnorm{\lvert f\rvert{}^{q}}1\leq\lebInt{\mu}{\lvert f\rvert^{p}}^{\frac{q}{p}}\mu(\X)^{\frac{s-1}{s}}.
\]
Taking the $q$th root yields
\[
\pnorm fq\leq\pnorm fp\mu(\X)^{\frac{s-1}{sq}}<\infty
\]
which completes the proof with $C=\mu(\X)^{\frac{s-1}{sq}}.$
\end{proof}
\begin{defn}
\label{def:banachSpace}A normed vector space $\left(V,\|\cdot\|\right)$
is called a \emph{Banach space }if it's complete with respect to the
metric induced by its norm.
\end{defn}

We would like to prove that our $\mathcal{L}^{p}$ spaces are actually
Banach spaces but the problem is that they are not normed spaces to
begin with; as we noted earlier, the $p-$norms map non-zero functions
to zero, violating the definiteness condition for norms. This does
not actually turn out to be a major impediment in practice, as it
is easy to transform our $\mathcal{L}^{p}$ spaces into actual normed
spaces.

To see this, define a relation $\sim$ on $\mathcal{L}^{p}\left(\X,\F,\mu\right)$
such that $f\sim g$ if $f=g$ on all but a null set. It's straightforward
to verify that this is in fact an equivalence relation and so the
quotient space
\[
L^{p}\left(\X,\F,\mu\right):=\mathcal{L}^{p}\left(\X,\F,\mu\right)/\sim
\]
consisting of all equivalence classes in $\mathcal{L}^{p}$ generated
by $\sim$ is actually a normed space, where 
\[
\|[u]\|_{p}:=\inf\left\{ \pnorm wp\mid w\in\mathcal{L}^{p}\text{such that}\ w\sim u\right\} =\pnorm up\ \forall u\in\mathcal{L}^{p}
\]
Here the norm of an equivalence class is simply the norm of a element
of the equivalence class since the norm is invariant if the function
changes only on a null set. In this case
\[
\pnorm{[u]}p=0\implies[u]=[0].
\]
While this construction is useful to illustrate the fact that $p-$norms
can be transformed into proper norms, in practice people do not think
of spaces of functions as collections of equivalence classes of functions.
For now, shall adopt the more pragmatic approach of not worrying about
whether our norm is a semi norm or a proper norm and explore the more
substantive questions. When we begin our investigation of continuous
time stochastic processes, the distinction between spaces of functions
and spaces of equivalence classes of functions shall become more important.

In order to discuss completeness, we need a good notion of limits.
In the context of a normed vector space, limits can be defined naturally
like in Euclidean spaces. However, in $\mathcal{L}^{p}$ spaces, limits
may not be unique unless we adopt the quotient space construction
above.
\begin{thm}
\label{thm:LpNormedVectorSpace}Let $\measurespace$ be a measure
space. For $p\in[1,\infty)$ , $\Lp p{\X,\F,\mu}$ is a semi-normed
vector space and $L^{p}\measurespace$ is a normed vector space.
\end{thm}

\begin{defn}[Convergence in $\mathcal{L}^{p}$]
\label{def:convergenceLp}A sequence of functions $\left\{ f_{n}\right\} _{n\in\N}\in\mathcal{L}^{p}\left(\X,\F,\mu\right)$
converges to a function $f\in\measurableFunctions$ in $\mathcal{L}^{p}$
if 
\[
\lim_{n\to\infty}\pnorm{f_{n}-f}p=0.
\]
In this case, we write
\[
f_{n}\stackrel{\mathcal{L}^{p}}{\longrightarrow}f.
\]
\end{defn}

Note that this definition strictly subsumes Definition \ref{def:L1Convergence}
which discussed limits in $\mathcal{L}^{1}$. There, we had implictly
assumed that the limiting function $f$ was also $\mathcal{L}^{1}$;
now, we shall show that this is in fact always true for all $\mathcal{L}^{p}$
spaces, that is to say, $\mathcal{L}^{p}$ spaces contain their limit
points. But we can show more, as every Cauchy sequence converges to
some limit that is $\mathcal{L}^{p}$. This is the main result of
this section.
\begin{thm}[Completeness of $\mathcal{L}^{p}$]
\label{thm:completenessLp} Let $\left\{ f_{n}\right\} _{n\in\N}\in\mathcal{L}^{p}\left(\X,\F.\mu\right)$
be a Cauchy sequence in $\mathcal{L}^{p}$; that is to say, for any
$\epsilon>0$ there exists some $n_{\epsilon}\in\N$ such that for
all $m,n\geq n_{\epsilon}$
\[
\pnorm{f_{m}-f_{n}}p<\epsilon.
\]
Then, there exists some function $f\in\Lp p{\X,\F,\mu}$ such that
\[
f_{n}\stackrel{\mathcal{L}^{1}}{\longrightarrow}f.
\]
\end{thm}

\begin{proof}
Note that by the definition of a Cauchy sequence and the well ordering
principle of natural numbers, for any $k\in\N$ there exists some
smallest natural number $n_{k}$ such that for all $m,n\geq n_{k}$
\[
\pnorm{f_{m}-f_{n}}p<2^{-k}.
\]
In particular, this implies that 
\[
\pnorm{f_{n_{k+1}}-f_{n_{k}}}p<2^{-k}
\]
as $n_{k+1}\geq n_{k}$. Further, observe that we can rewrite the
elements of our subsequence of functions as 
\[
f_{n_{k}}=\sum_{i=0}^{k-1}\left(f_{n_{i+1}}-f_{n_{i}}\right)
\]
where $f_{n_{0}}=0.$ Then, note that 
\begin{align*}
\pnorm{\sum_{i=0}^{k-1}\lvert f_{n_{i+1}}-f_{n_{i}}\rvert}p & \leq\sum_{i=0}^{k-1}\pnorm{f_{n_{i+1}}-f_{n_{i}}}p\\
 & \leq\pnorm{f_{n_{1}}}p+\sum_{i=1}^{k-1}2^{-i}
\end{align*}
where the first inequality follows from Theorem \ref{thm:minkowskiInequality}.
Then, applying limits, we have
\begin{align}
\lim_{k\to\infty}\pnorm{\sum_{i=0}^{k-1}\lvert f_{n_{i+1}}-f_{n_{i}}\rvert}p & \leq\pnorm{f_{n_{1}}}p+\sum_{i=1}^{\infty}2^{-i}\nonumber \\
 & =\pnorm{f_{n_{1}}}p+1.\label{eq:boundCompleteness}
\end{align}
Observe that the limit on the left hand side exists because the sequence
is increasing and bounded above. Finally, note that the sequence $g_{k}:=\sum_{i=0}^{k-1}\lvert f_{n_{i+1}}-f_{n_{i}}\rvert$
is a sequence of non-negative and increasing measurable functions
and since $p\geq1$, $g_{k}^{p}$ is also non-negative, increasing
and measurable. Therefore, we can apply the \hyperref[thm:generalizedMonotoneConvergence]{monotone convergence theorem}
to deduce
\begin{align}
\lim_{k\to\infty}\pnorm{g_{k}}p & =\lim_{k\to\infty}\lebInt{\mu}{g_{k}^{p}}^{\frac{1}{p}}\nonumber \\
 & =\lebInt{\mu}{\lim_{k\to\infty}g_{k}^{p}}^{\frac{1}{p}}\nonumber \\
 & =\lebInt{\mu}{\left(\lim_{k\to\infty}g_{k}\right)^{p}}^{\frac{1}{p}}\nonumber \\
 & =\pnorm{\lim_{k\to\infty}g_{k}}p\label{eq:monotoneConvergenceLp}
\end{align}
where we have also used the fact that the maps $x\to x^{p}$ and $x\to x^{\frac{1}{p}}$
are continuous for $p\geq1.$ Together, equations (\ref{eq:boundCompleteness})
and (\ref{eq:monotoneConvergenceLp}) tell us that 
\[
\pnorm{\lim_{k\to\infty}g_{k}}p<\infty
\]
which, by Proposition \ref{prop:intFiniteFuncFinite} shows that $\lim_{k\to\infty}g_{k}^{p}<\infty$
$\mu-$a.e and so $\lim_{k\to\infty}g_{k}<\infty$ $\mu-$a.e. In
other words, since absolute summability implies summability
\[
\sum_{i=0}^{\infty}\lvert f_{n_{i+1}}-f_{n_{i}}\rvert<\infty\implies f:=\sum_{i=0}^{\infty}\left(f_{n_{i+1}}-f_{n_{i}}\right)<\infty\ \mu-\text{a.e}
\]

Now we have a candidate function $f\in\mathcal{L}^{p}\left(\mu\right)$
such that 
\[
f=\lim_{k\to\infty}f_{n_{k}}
\]
where the limit is taken pointwise. If we can show that $f_{n_{k}}\stackrel{\mathcal{L}^{p}}{\longrightarrow}f$
then it will imply that our original sequence $f_{n}\stackrel{\mathcal{L}^{p}}{\longrightarrow}f$
since by ``Cauchyness'' and Minkowski's inequality
\begin{align*}
\pnorm{f_{n}-f}p & \leq\pnorm{f_{n}-f_{n_{k}}}p+\pnorm{f_{n_{k}}-f}{}\\
 & \leq\frac{\epsilon}{2}+\frac{\epsilon}{2}=\epsilon
\end{align*}
for any $\epsilon>0$ and appropriately large values of $n$ and $k.$
To show subsequential convergence, observe that 
\begin{align*}
\pnorm{f-f_{n_{k}}}p & =\pnorm{\sum_{i=k}^{\infty}\left(f_{n_{i+1}}-f_{n_{i}}\right)}p\\
 & \leq\pnorm{\sum_{i=k}^{\infty}\lvert f_{n_{i+1}}-f_{n_{i}}\rvert}p\\
 & \leq\sum_{i=k}^{\infty}\pnorm{f_{n_{i+1}}-f_{n_{i}}}p\\
 & \leq\sum_{i=k}^{\infty}2^{-i}\\
 & =1-\sum_{i=1}^{k-1}2^{-i}
\end{align*}
where the second inequality follows from the monotone convergence
argument from earlier. Taking the limit in $k$ yields the result.
\end{proof}

\subsection{Convexity}

We take a little detour in this section to establish an important
result in analysis and probability theory: Jensen's inequality. The
proof of this theorem is remarkably simple once we develop a reasonable
understanding on the behavior of convex functions; in particular,
we need to show that convex functions always have support lines. This
is intuitively obvious, but showing this rigorously requires a bit
of work, which we do here.
\begin{defn}
\label{def:convexFunction}Let $X$ be a vector space. A function
$f:X\to\R$ is called convex if for any $\lambda\in\left[0,1\right]$
and any $x,y\in X$
\[
f\left(\lambda x+\left(1-\lambda\right)y\right)\leq\lambda f\left(x\right)+\left(1-\lambda\right)f\left(y\right).
\]
\end{defn}

In elementary calculus, we learnt that convex functions are those
that have a graph shaped like a smile (concave functions are frowns),
and that the second derivative of convex functions are positive. However,
convex functions in general need not be differentiable ($\lvert x\rvert$
is convex but not differentiable at zero). Convex functions do have
the property of \emph{subdifferentiability, }which is basically captures
the fact that convex functions with ``corners'' can have tangent
lines, even if they are not unique.
\begin{lem}
\label{lem:convexTangents}Let $f:\left[a,b\right]\to\R$ be a convex
function. Then for any $x\in\left(a,b\right)$
\[
\frac{f\left(x\right)-f\left(a\right)}{x-a}\leq\frac{f\left(b\right)-f\left(a\right)}{b-a}\leq\frac{f\left(b\right)-f\left(x\right)}{b-x}.
\]
\end{lem}

\begin{proof}
Let $\lambda=\frac{b-x}{b-a}$ and so $1-\lambda=\frac{x-a}{b-a}$
and 
\begin{align*}
f\left(x\right) & -f\left(a\right)=f\left(\lambda a+\left(1-\lambda\right)b\right)-f\left(a\right)\\
 & \leq\lambda f\left(a\right)+\left(1-\lambda\right)f\left(b\right)-f\left(a\right)\\
 & =\left(1-\lambda\right)\left(f\left(b\right)-f\left(a\right)\right)\\
 & =\frac{x-a}{b-a}\left(f\left(b\right)-f\left(a\right)\right)
\end{align*}
so rearranging yields the first inequality. The second inequality
is similarly deduced by noticing that $f\left(b\right)-f\left(x\right)\geq f\left(b\right)-\lambda f\left(a\right)-\left(1-\lambda\right)f\left(b\right)=\lambda\left(f\left(b\right)-f\left(a\right)\right)$.
\end{proof}
Now let $x_{0}\in\left(a,b\right)$ be fixed and define
\begin{align*}
m^{-}\left(x_{0}\right) & :=\sup_{\eta\in\left(a,b\right),\eta<x_{0}}\frac{f\left(x_{0}\right)-f\left(\eta\right)}{x_{0}-\eta}\\
m^{+}\left(x_{0}\right) & :=\inf_{\xi\in\left(a,b\right),\xi>x_{0}}\frac{f\left(\xi\right)-f\left(x_{0}\right)}{\xi-x_{0}}.
\end{align*}

\begin{prop}
\label{prop:convexLeftRightDerivative}For any $x_{0}\in\left(a,b\right)$
and any convex function $f:\left[a,b\right]\to\R$ , $m^{-}\left(x_{0}\right)$
and $m^{+}\left(x_{0}\right)$ are finite and $m^{-}\left(x_{0}\right)\leq m^{+}\left(x_{0}\right)$
with equality holding if and only if $f$ is differentiable at $x_{0}$
in which case 
\[
m^{-}\left(x_{0}\right)=m^{+}\left(x_{0}\right)=f^{\prime}\left(x_{0}\right).
\]
\end{prop}

\begin{proof}
Fix $x_{0}\in\left(a,b\right)$ and note that by our Lemma \footnote{in our lemma $a$ and $b$ are arbitrary. Here we are applying the
Lemma to $x_{0}\in\left(\eta,b\right)$ and $x_{0}\in\left(a,\xi\right)$}, for $a<\eta<x_{0}<\xi<b$
\[
\frac{f\left(x_{0}\right)-f\left(\eta\right)}{x_{0}-\eta}\leq\frac{f\left(b\right)-f\left(x_{0}\right)}{b-x_{0}}
\]
and so its supremum 
\[
m^{-}\left(x_{0}\right)\leq\frac{f\left(b\right)-f\left(x_{0}\right)}{b-x_{0}}
\]
and similarly 
\[
\frac{f\left(\xi\right)-f\left(x_{0}\right)}{\xi-x_{0}}\geq\frac{f\left(x_{0}\right)-f\left(a\right)}{x_{0}-a}
\]
and so 
\[
m^{+}\left(x_{0}\right)\geq\frac{f\left(x_{0}\right)-f\left(a\right)}{x_{0}-a}.
\]
Next, note that applying the Lemma to $x_{0}\in\left(\eta,\xi\right)$
tells us that 
\[
\frac{f\left(x_{0}\right)-f\left(\eta\right)}{x_{0}-\eta}\leq\frac{f\left(\xi\right)-f\left(x_{0}\right)}{\xi-x_{0}}
\]
and so
\[
m^{-}\left(x_{0}\right)\leq m^{+}\left(x_{0}\right).
\]
Finally, apply the Lemma to $\eta_{2}\in\left(\eta_{1},x_{0}\right)$
to deduce that 
\[
\frac{f\left(x_{0}\right)-f\left(\eta_{2}\right)}{x_{0}-\eta_{2}}\geq\frac{f\left(x_{0}\right)-f\left(\eta_{1}\right)}{x_{0}-\eta_{1}}
\]
which means that $\frac{f\left(x_{0}\right)-f\left(\eta\right)}{x_{0}-\eta}$
is increasing in $\eta<x_{0}$ and is bounded above and so
\[
\lim_{\eta\to x_{0}^{-}}\frac{f\left(x_{0}\right)-f\left(\eta\right)}{x_{0}-\eta}=m^{-}\left(x_{0}\right).
\]
Similarly, 
\[
\lim_{\xi\to x_{0}^{+}}\frac{f\left(\xi\right)-f\left(x_{0}\right)}{\xi-x_{0}}=m^{+}\left(x_{0}\right)
\]
and so our $m^{-}$and $m^{+}$ are simply left and right hand derivatives
and our claim follows.
\end{proof}
Now we can construct tangent lines to convex functions in the following
sense
\begin{prop}
\label{prop:subderivatives}Let $f:\left[a,b\right]\to\R$ be a convex
function. For any $x_{0}\in\left(a,b\right),$ for any $m\in\left[m^{-}\left(x_{0}\right),m^{+}\left(x_{0}\right)\right],$we
have that
\[
l\left(x\right):=f\left(x_{0}\right)+m\left(x-x_{0}\right)\leq f\left(x\right).
\]
\end{prop}

\begin{proof}
Note that for $x>x_{0}$
\[
\frac{f\left(x\right)-f\left(x_{0}\right)}{x-x_{0}}\geq m^{+}\left(x_{0}\right)\geq m
\]
and if $x<x_{0}$ then 
\[
\frac{f\left(x_{0}\right)-f\left(x\right)}{x_{0}-x}\leq m^{-}\left(x_{0}\right)\leq m.
\]
 The case of $x=x_{0}$ is trivial.
\end{proof}
\begin{rem*}
The interval $\left[m^{-}\left(x_{0}\right),m^{+}\left(x_{0}\right)\right]$
is called the set of \emph{subderivatives }of $f$ at $x_{0}$. It
should be clear that $f$ is differentiable at $x_{0}$ if and only
if the set is a singleton. The function $l$ is called a \emph{supporting
line }at $x_{0}$. More generally, a supporting line $l$ at $x_{0}$
is any affine function that satisfies $l\left(x_{0}\right)=f\left(x_{0}\right)$
and $l\left(x\right)\leq f\left(x\right)$ for every $x\in\left(a,b\right)$.
A convex function on $\left[a,b\right]$ has at least one supporting
line at every point in $\left(a,b\right)$
\end{rem*}
Note that the existence of left and right derivatives at a point implies
continuity at that point. This implies that convex functions on an
interval are continuous.
\begin{prop}
\label{prop:leftRightDerivativeImpliesContinuity}Let $f:\left[a,b\right]\to\R$
be a function such that for a point $x_{0}\in\left(a,b\right)$ the
left and right hand derivatives 
\[
\lim_{x\to x_{0}^{-}}\frac{f\left(x\right)-f\left(x_{0}\right)}{x-x_{0}}=L
\]
and
\[
\lim_{x\to x_{0}^{+}}\frac{f\left(x\right)-f\left(x_{0}\right)}{x-x_{0}}=M
\]
then $f$ is continuous $x_{0}.$
\end{prop}

\begin{proof}
Note that 
\[
\lim_{x\to x_{0}^{-}}f\left(x\right)-f\left(x_{0}\right)=L\lim_{x\to x_{0}^{-}}x-x_{0}=0
\]
and
\[
\lim_{x\to x_{0}^{+}}f\left(x\right)-f\left(x_{0}\right)=M\lim_{x\to x_{0}^{+}}x-x_{0}=0
\]
 which completes the proof.
\end{proof}
\begin{cor}
\label{cor:convexContinuous}A convex function $f:\left[a,b\right]\to\R$
is continuous on $\left(a,b\right).$
\end{cor}

We have proved that a convex function on $\left[a,b\right]$ has supporting
lines; it turns out that if a function on $\left[a,b\right]$ has
supporting lines everywhere on $\left(a,b\right)$ then it is convex
as well.
\begin{prop}
\label{prop:convexSupportingLines}Let $f:\left[a,b\right]\to\R$
be a function. Then $f$ is convex if and only if it has a supporting
line at each point $x_{0}\in\left(a,b\right)$.
\end{prop}

\begin{proof}
The ``only if'' part is Proposition \ref{prop:subderivatives}.
For the converse, let $\lambda\in\left[0,1\right]$ be arbitrary and
let $x,y\in\left(a,b\right)$. Let $x_{0}=\lambda x+\left(1-\lambda\right)y$
and note that 
\begin{align*}
\lambda f\left(x\right)+\left(1-\lambda\right)f\left(y\right) & \geq\lambda l\left(x\right)+\left(1-\lambda\right)f\left(y\right)\\
 & =l\left(x_{0}\right)\\
 & =f\left(x_{0}\right)
\end{align*}
where the first equality follows from the fact that affine functions
are both convex and concave.
\end{proof}
Finally, we recover the result that a twice continuously differentiable
convex function has non-negative second derivative, a fact that follows
from Taylor's theorem with remainder.
\begin{prop}
\label{prop:convexPositiveSecondDerivative}Let $f:\left[a,b\right]\to\R$
be a twice continuously differentiable function. Then $f$ is a convex
function if and only if at every point $x_{0}\in\left(a,b\right)$
$f^{\prime\prime}\left(x_{0}\right)\geq0$.
\end{prop}

\begin{proof}
Note that by Taylor's theorem
\begin{equation}
f\left(x\right)=f\left(x_{0}\right)+f^{\prime}\left(x_{0}\right)\left(x-x_{0}\right)+f^{\prime\prime}\left(c\right)\frac{\left(x-x_{0}\right)^{2}}{2}\label{eq:taylorConvex}
\end{equation}
where $c$ is between $x$ and $x_{0}$. Then if $f^{\prime\prime}\left(c\right)\geq0$
then
\[
f\left(x\right)\geq f\left(x_{0}\right)+f^{\prime}\left(x_{0}\right)\left(x-x_{0}\right)
\]
and so by Proposition \ref{prop:convexSupportingLines} our $f$ is
convex. Conversely, if $f$ is convex then
\end{proof}
We are finally ready to present the main result from this section.
\begin{thm}[Jensen's inequality]
\label{thm:jensenInequality}Let $\measurespace$ be a measure space
wth $\mu\left(\X\right)=1.$ For any integrable function $f\in\Lp 1{\mu}$
and a convex function $g:\R\to\R$, we have that
\[
\lebInt{\mu}{g\circ f}\geq g\left(\lebInt{\mu}f\right).
\]
\end{thm}

\begin{proof}
First note that for any restriction of $g$ to any interval $\left[-n,n\right],$$g$
has supporting lines everywhere on $\left(-n,n\right)$ by Proposition
\ref{prop:convexSupportingLines}. Since $n$can be arbitrarily large,
$g$ admits a supporting line at $x_{0}=\lebInt{\mu}f.$That is to
say, there exists an affine function $l$ such that 
\[
l\left(\lebInt{\mu}f\right)=g\left(\lebInt{\mu}f\right)
\]
and
\[
l\left(f\left(x\right)\right)\leq g\left(f\left(x\right)\right)
\]
for all $x\in\R$. Any affine function $l:\R\to\R$ can be written
as $l\left(x\right)=a+bx$ where $a,b\in\R.$ Therefore, by the monotonicty
of integration
\begin{align*}
\lebInt{\mu}{g\circ f} & \geq\lebInt{\mu}{l\circ f}\\
 & =a+b\lebInt{\mu}f\\
 & =l\left(\lebInt{\mu}f\right)\\
 & =g\left(\lebInt{\mu}f\right)
\end{align*}
where in the second line we have used the linearity of integration
along with the fact that $\mu\left(\X\right)=1.$
\end{proof}
\begin{example}[ISI 2005 Sample PSB 4]
\label{exa:isi2005samplepsb4}Let $f$ be a non-decreasing, integrable
function defined on $[0,1]$. Show that 
\[
\left(\int_{0}^{1}f(x)dx\right)^{2}\leq2\int_{0}^{1}x(f(x))^{2}dx.
\]
Note that Jensen's inequality immediately implies that 
\[
\left(\int_{0}^{1}f(x)dx\right)^{2}\leq\int_{0}^{1}(f(x))^{2}dx.
\]
 Further, note that $f\left(x\right)^{2}2x\indicate\left\{ \frac{1}{2}\leq x\leq1\right\} \geq f\left(x\right)^{2}\indicate\left\{ \frac{1}{2}\leq x\leq1\right\} $
but $f\left(x\right)^{2}2x\indicate\left\{ 0\leq x\leq\frac{1}{2}\right\} \leq f\left(x\right)^{2}\indicate\left\{ 0\leq x\leq\frac{1}{2}\right\} $.
However, since $f$ is non decreasing, we know that 
\[
\lvert f\left(x\right)^{2}\left(2x-1\right)\indicate\left\{ \frac{1}{2}\leq x\leq1\right\} \rvert\geq\lvert f\left(x\right)^{2}\left(2x-1\right)\indicate\left\{ 0\leq x\leq\frac{1}{2}\right\} \rvert
\]
and so 
\begin{align*}
\int_{0}^{1}(f(x))^{2}dx & =\int_{0}^{\frac{1}{2}}\left(f\left(x\right)\right)^{2}dx+\int_{\frac{1}{2}}^{1}\left(f\left(x\right)\right)^{2}dx\\
 & \leq\int_{0}^{\frac{1}{2}}2x\left(f\left(x\right)\right)^{2}dx+\int_{\frac{1}{2}}^{1}2x\left(f\left(x\right)\right)^{2}dx\\
 & =\int_{0}^{1}2x\left(f\left(x\right)\right)^{2}dx.
\end{align*}
\end{example}


\subsection{The space $L^{\infty}$}

In the context of $L^{p}$ spaces, we replace the notion of boundedness
(i.e. the quality of a function in a space being bounded) with the
related notion of \emph{essentially boundedness. }A function is said
to be essentially bounded if it is bounded except on a null set. Then
the space of essentially bounded functions from a measure space $\measurespace$
to $\left(\R,\borel\left(\R\right)\right)$is denoted $\Lp{\infty}{\X,\F,\mu}$.
This space comes equipped with a (semi) norm $\pnorm{\cdot}{\infty}$
so that for $f\in\Lp{\infty}{\mu},$we have that 
\[
\pnorm f{\infty}:=\inf\left\{ C\in\R\mid\mu\left(\left\{ \lvert f\rvert>C\right\} \right)=0\right\} .
\]
To see this is indeed a semi-norm, we can verify that non-negativity
is satisfied by definition. To see homogeneity, observe that for any
$\alpha\neq0$, the collection $\left\{ C\in\R\mid\mu\left(\lvert\alpha f\rvert>C\right)\right\} =\left\{ \lvert\alpha\rvert C\in\R\mid\mu\left(\lvert f\rvert>C\right)\right\} $
and so we have homogeneity. For the triangle inequality, consider
the fact that for any $C\in\R$ and any $f,g\in\Lp{\infty}{\mu}$,
we have $\lvert f\left(x\right)\rvert\leq\pnorm f{\infty},\lvert g\left(x\right)\rvert\leq\pnorm g{\infty}$
for almost all $x\in\X$ and so
\[
\lvert f\left(x\right)+g\left(x\right)\rvert\leq\lvert f\left(x\right)\rvert+\lvert g\left(x\right)\rvert\leq\pnorm f{\infty}+\pnorm g{\infty}
\]
almost everywhere. In words, $\pnorm f{\infty}+\pnorm g{\infty}$
is an \emph{essential upper-bound }for $\lvert f\left(x\right)+g\left(x\right)\rvert$
and so $\pnorm f{\infty}+\pnorm g{\infty}\in\left\{ C\in\R\mid\mu\left(\left\{ \lvert f+g\rvert>C\right\} \right)=0\right\} .$
The inequality then follows since $\pnorm{f+g}{\infty}$ is the infimum
of that set. The notation $\Lp{\infty}{\mu}$ of course deserves some
scrutiny; one potential justification for the choice of notation is
the following result.
\begin{prop}
\label{prop:limLp}Let $\measurespace$ be a measure space and let
$f\in\cap_{p\in\N}\Lp p{\mu}.$Then 
\[
\lim_{p\to\infty}\pnorm fp=\pnorm f{\infty}.
\]
\end{prop}

\begin{proof}
First, suppose that $\pnorm f{\infty}<\infty.$ Then, for any sequence
$q_{n}\to\infty,$ we have that 
\begin{align*}
\pnorm f{p+q_{n}} & =\lebInt{\mu}{\lvert f\rvert^{p+q_{n}}}^{\frac{1}{p+q_{n}}}\\
 & \leq\lebInt{\mu}{\lVert f\rVert_{\infty}^{q_{n}}\lvert f\rvert^{p}}^{\frac{1}{p+q_{n}}}\\
 & =\lVert f\rVert_{\infty}^{\frac{q_{n}}{p+q_{n}}}\lebInt{\mu}{\lvert f\rvert^{p}}^{\frac{1}{p+q_{n}}}
\end{align*}
where we used the fact that $\lvert f\rvert\leq\pnorm f{\infty}$
almost everywhere and the monotonicity and linearity of the integral.
Taking lim-sups\footnote{We don't know if the limit on the left side exists; the right-hand
side has a proper limit and so is equal to its lim-sup.} on both sides we have 
\[
\limsup_{q_{n}\to\infty}\pnorm f{p+q_{n}}\leq\pnorm f{\infty}.
\]
Conversely, note that for any $\epsilon>0$, we have the fact that
$\lvert f\rvert\geq\left(\pnorm f{\infty}-\epsilon\right)$$\indicate\left\{ \lvert f\rvert>\pnorm f{\infty}-\epsilon\right\} $
and so since p-norms respect monotonicty, we have that 
\begin{align*}
\pnorm fp & \geq\pnorm{\left(\pnorm f{\infty}-\epsilon\right)\indicate\left\{ \lvert f\rvert>\pnorm f{\infty}-\epsilon\right\} }p\\
 & =\lvert\pnorm f{\infty}-\epsilon\rvert\mu\left(\lvert f\rvert>\pnorm f{\infty}-\epsilon\right)^{\frac{1}{p}}
\end{align*}
where we have used homogeneity in the second line. Note that the right
hand side is finite by Markov's inequality since $f\in\Lp 1{\mu}$
and $\pnorm f{\infty}<\infty$\footnote{$\mu\left(\lvert f\rvert>\pnorm f{\infty}-\epsilon\right)\leq\frac{\lebInt{\mu}{\lvert f\rvert}}{\pnorm f{\infty}-\epsilon}$}
and positive by the fact that $\pnorm f{\infty}$is an essential supremum
so any smaller real number cannot be an essential upper bound. Letting
$p\to\infty,\epsilon\to0$ on the right hand side ( while taking a
lim-inf on the left), we have
\[
\liminf_{p\to\infty}\pnorm fp\geq\pnorm f{\infty}
\]
which completes the proof for the essentially bounded case.
\end{proof}

\subsubsection{Uniform convergence and $L^{\infty}$.}

Recall from basic analysis (or our discussion on Egorov's theorem)
the notion of uniform convergence. We say
\begin{defn}
\label{def:uniformConvergence}A sequence of functions $f_{n}\in\measurableFunctions$
converges uniformly to a function $f\in\measurableFunctions$
\end{defn}


\section{Hilbert spaces over $\protect\R$}

\subsection{Introduction to inner product spaces}

The inner product is a generalization of the dot product of Euclidean
spaces. Recall from calculus that the dot product is defined $\bullet:\R^{n}\times\R^{n}\longrightarrow\R$
where
\[
x\bullet y:=\sum_{i=1}^{n}x_{i}y_{i}
\]
for any $x,y\in\R^{n}.$ We can generalize this with the following
definition.
\begin{defn}
\label{def:innerProduct}Let $V$ be a vector space over $\R$. The
function $\innerproduct{\cdot}{\cdot}:V\times V\longrightarrow\R$
is called an\emph{ inner product }if

\begin{enumerate}[label=(\roman*),leftmargin=.1\linewidth,rightmargin=.4\linewidth]
	\item $\innerproduct{v}{v} \geq 0$ for all $ v \in V $
	\item $\innerproduct{v}{v} =0 \implies v = 0$ for all $ v \in V $
	\item For any $v,w \in V: \innerproduct{v}{w} = \innerproduct{w}{v}$
	\item For any $\alpha,\beta \in \R$ and any $v,w,u \in V$
	\[
			\innerproduct{\alpha v + \beta w}{u} = \alpha\innerproduct{v}{u} + \beta\innerproduct{w}{u}.
	\]
\end{enumerate}

If $\innerproduct{\cdot}{\cdot}$ satsifies these properties then
$\left(V,\innerproduct{\cdot}{\cdot}\right)$ is called an \emph{inner
product space}.
\end{defn}

The first and second properties are together called the \emph{definiteness
}condition of inner products, analagous to the one for norms. The
third property is called \emph{symmetry, }and the the final property
is \emph{linearity in the first argument.} Note that symmetry and
linearity in the first argument together imply linearity in the second
argument. Since we are ultimately concerned with basic probability
theory, we shall always assume the field over which $V$ is defined
is $\R$; if the field were $\mathds{C}$ then the second property
above would be replaced by \emph{skew-symmetry }i.e. $\innerproduct vw=\overline{\innerproduct wv}$
where the $\overline{c}$ for any $c\in\mathds{C}$ denotes the complex
conjugate. In this case, proper linearity in the second argument would
not hold.

We can immediately use these properties to derive a familiar result
from a different context.
\begin{prop}[Cauchy-Schwarz inequality]
\label{prop:innerCauchySchwarz}Let $\left(V,\innerproduct{\cdot}{\cdot}\right)$
be an inner product space. For any $v,w\in V$
\[
\innerproduct vw{}^{2}\leq\innerproduct vv\innerproduct ww
\]
and moreover, strict equality holds if and only if $v=\alpha w$ for
some $\alpha\in\R.$
\end{prop}

\begin{proof}
The claim is trivially true if either $v=0$ or $w=0$ as $\innerproduct 0w=0\innerproduct vw$
by linearity. Thus we can assume without loss of generality both are
non-zero. Write $v=\left(v-\alpha w\right)+\alpha w$ where $\alpha=\frac{\innerproduct vw}{\innerproduct ww}.$
Then,
\begin{align}
\innerproduct vv & =\innerproduct{\left(v-\alpha w\right)+\alpha w}{\left(v-\alpha w\right)+\alpha w}\nonumber \\
 & =\innerproduct{v-\alpha w}{v-\alpha w}+2\innerproduct{\alpha w}{v-\alpha w}+\innerproduct{\alpha w}{\alpha w}\label{eq:innerProdExpansion}
\end{align}
where the second equality follows from an application of linearity
and symmetry. Note that the term
\begin{align}
\innerproduct{\alpha w}{v-\alpha w} & =\alpha\innerproduct wv-\alpha^{2}\innerproduct ww\nonumber \\
 & =\alpha\left(\innerproduct vw-\alpha\innerproduct ww\right)\nonumber \\
 & =0.\label{eq:crossProdZero}
\end{align}
Note that equations (\ref{eq:innerProdExpansion}) and (\ref{eq:crossProdZero}),
together with the definiteness of inner products, imply that 
\begin{align*}
\innerproduct vv & \geq\innerproduct{\alpha w}{\alpha w}\\
 & =\alpha^{2}\innerproduct ww\\
 & =\frac{\innerproduct vw^{2}}{\innerproduct ww}
\end{align*}
and rearranging yields the inequality.

To see the equality result, note that $\innerproduct vv=\innerproduct{\alpha w}{\alpha w}$
in equation (\ref{eq:innerProdExpansion}) when $\innerproduct{v-\alpha w}{v-\alpha w}=0\implies v=\alpha w$
which shows the necessity. To show sufficiency, let $v=\beta w$ for
some $\beta\in\R$ and note that 
\begin{align*}
\innerproduct vw^{2} & =\innerproduct{\beta w}w^{2}\\
 & =\beta^{2}\innerproduct ww^{2}\\
 & =\beta^{2}\innerproduct ww\innerproduct ww\\
 & =\innerproduct{\beta w}{\beta w}\innerproduct ww\\
 & =\innerproduct vv\innerproduct ww
\end{align*}
where the second equality follows from linearity in the first argument
and the fourth equality follows from linearity in both arguments.
\end{proof}
The relationship between inner-products and norms is a tight one;
every inner-product induces a norm.
\begin{prop}
\label{prop:normInducedByInnerProd}Let $\left(V,\innerproduct{\cdot}{\cdot}\right)$
be an inner product space and let the function $\pnorm{\cdot}{}:V\longrightarrow\R$
be defined by 
\[
\pnorm v{}=\sqrt{\innerproduct vv}
\]
for any $v\in V$. Then the function $\pnorm{\cdot}{}$ is a norm.
\end{prop}

\begin{proof}
Note that the definiteness condition of norms corresponds to the definiteness
condition of inner-products, and so is trivially satisfied. Next,
observe that for any $\alpha\in\R$
\begin{align*}
\pnorm{\alpha v}{} & =\sqrt{\innerproduct{\alpha v}{\alpha v}}\\
 & =\sqrt{\alpha^{2}\innerproduct vv}\\
 & =\lvert\alpha\rvert\pnorm v{}
\end{align*}
which gives us absolute homogeneity. Finally, for any $v,w\in V$
\begin{align*}
\pnorm{v+w}{}^{2} & =\innerproduct{v+w}{v+w}\\
 & =\innerproduct vv+2\innerproduct vw+\innerproduct ww\\
 & \leq\pnorm v{}^{2}+2\pnorm v{}\pnorm w{}+\pnorm w{}^{2}\\
 & =\left(\pnorm v{}+\pnorm w{}\right)^{2}
\end{align*}
where the inequality follows from the Cauchy-Schwarz inequality. This
completes the proof.
\end{proof}
We call such a norm a norm \emph{induced by }an inner product. It
turns out that one can recover a norm from an inner product precisely
when a particular identity is satisfied.
\begin{defn}
\label{def:parallelogram}Let $\left(V,\pnorm{\cdot}{}\right)$ be
a normed vector space. The norm $\pnorm{\cdot}{}$ is said to satisfy
the \emph{parallelogram identity }if for any $v,w\in V$
\[
\pnorm{\frac{v+w}{2}}{}^{2}+\pnorm{\frac{v-w}{2}}{}^{2}=\frac{1}{2}\left(\pnorm v{}^{2}+\pnorm w{}^{2}\right).
\]
\end{defn}

Next, we need a few lemmas to aid in the proof of the main result.
\begin{lem}
\label{lem:polarization}Let $\left(V,\pnorm{\cdot}{}\right)$ be
a normed vector space where the norm $\pnorm{\cdot}{}$ satisfies
the parallelogram identity. Then, for any $u,v,w\in V$
\[
\pnorm{u+v+w}{}^{2}=\pnorm{u+v}{}^{2}+\pnorm{v+w}{}^{2}+\pnorm{u+w}{}^{2}-\pnorm u{}^{2}-\pnorm v{}^{2}-\pnorm w{}^{2}.
\]
\end{lem}

\begin{proof}
Note that by the parallelogram identity
\begin{equation}
\pnorm{\left(u+v\right)+w}{}^{2}=2\pnorm{u+v}{}^{2}+2\pnorm w{}^{2}-\pnorm{\left(u+v\right)-w}{}^{2}\label{eq:parallel1}
\end{equation}
and 
\begin{equation}
\pnorm{u+\left(v+w\right)}{}^{2}=2\pnorm u{}^{2}+2\pnorm{v+w}{}^{2}-\pnorm{u-v-w}{}^{2}.\label{eq:parallel2}
\end{equation}
Adding equations (\ref{eq:parallel1}) and (\ref{eq:parallel2}),
then dividing by two, we have that 
\begin{align*}
\pnorm{u+v+w}{}^{2} & =\pnorm{u+v}{}^{2}+\pnorm w{}^{2}-\frac{1}{2}\pnorm{u+v-w}{}^{2}+\pnorm u{}^{2}+\pnorm{v+w}{}^{2}-\frac{1}{2}\pnorm{u-v-w}{}^{2}\\
 & =\pnorm{u+v}{}^{2}+\pnorm w{}^{2}+\pnorm u{}^{2}+\pnorm{v+w}{}^{2}-\pnorm{u-w}{}^{2}-\pnorm v{}^{2}\\
 & =\pnorm{u+v}{}^{2}+\pnorm w{}^{2}+\pnorm u{}^{2}+\pnorm{v+w}{}^{2}+\pnorm{u+w}{}^{2}-2\pnorm u{}^{2}-2\pnorm w{}^{2}-\pnorm v{}^{2}\\
 & =\pnorm{u+v}{}^{2}+\pnorm{v+w}{}^{2}+\pnorm{u+w}{}^{2}-\pnorm u{}^{2}-\pnorm v{}^{2}-\pnorm w{}^{2}
\end{align*}
where the second equality follows from an application of the parallelogram
identity on $\frac{1}{2}\pnorm{u+v-w}{}^{2}+\frac{1}{2}\pnorm{u-v-w}{}^{2}$
and the third equality follows from another application of the identity
to $\pnorm{u-w}{}^{2}$.
\end{proof}
\begin{lem}
\label{lem:continuityNorm}For any norm $\pnorm{\cdot}{}$ on some
vector space $V$, the map $\phi_{v,w}:\R\longrightarrow\R$ given
by
\[
\phi_{v,w}\left(t\right):=\frac{1}{4}\left(\pnorm{tv+w}{}^{2}-\pnorm{tv-w}{}^{2}\right)
\]
is continuous for any $v,w\in V$.
\end{lem}

\begin{proof}
Note that for any $v,w\in V$, our function $\phi_{v,w}$ is the composition
of continous functions and thus the result follows.\footnote{The norm is continuous by the reversed triangle inequality $\lvert\pnorm x{}-\pnorm y{}\rvert\leq\pnorm{x-y}{}$}
\end{proof}
\begin{thm}
\label{thm:parallelogram}Let $\left(V,\pnorm{\cdot}{}\right)$ be
a normed vector space. The norm $\pnorm{\cdot}{}$ is induced by an
inner product if and only if the parallelogram identity is satisfied.
\end{thm}

\begin{proof}
First suppose that the norm $\pnorm{\cdot}{}$ is induced by the inner
product $\innerproduct{\cdot}{\cdot}$. Then, 
\begin{align*}
\pnorm{\frac{v+w}{2}}{}^{2}+\pnorm{\frac{v-w}{2}}{}^{2} & =\innerproduct{\frac{v+w}{2}}{\frac{v+w}{2}}+\innerproduct{\frac{v-w}{2}}{\frac{v-w}{2}}\\
 & =\frac{1}{4}\left(\innerproduct vv+2\innerproduct vw+\innerproduct ww\right)+\frac{1}{4}\left(\innerproduct vv-2\innerproduct vw+\innerproduct ww\right)\\
 & =\frac{1}{2}\innerproduct vv+\frac{1}{2}\innerproduct ww\\
 & =\frac{1}{2}\left(\pnorm v{}^{2}+\pnorm w{}^{2}\right).
\end{align*}
The harder part is showing that if a norm that satisfies the parallelogram
identity, there exists an inner product that induces such a norm.
To do so, first define a map
\[
\left(v,w\right):=\frac{1}{4}\left(\pnorm{v+w}{}^{2}-\pnorm{v-w}{}^{2}\right)
\]
 and observe that 
\[
\left(v,v\right)=\pnorm v{}^{2}\geq0
\]
and that $\left(v,v\right)=0$ if and only if $v=0$ by the definiteness
condition of norms. Moreover, observe that $\left(v,w\right)=\left(w,v\right)$
since$\pnorm{v-w}{}=\pnorm{w-v}{}$. Next, note that for any $v,u,w\in V$
\begin{align*}
\left(v+u,w\right) & =\frac{1}{4}\left(\pnorm{v+u+w}{}^{2}-\pnorm{v+u-w}{}^{2}\right)\\
 & =\frac{1}{4}\left(\pnorm{u+v}{}^{2}+\pnorm{v+w}{}^{2}+\pnorm{u+w}{}^{2}-\pnorm u{}^{2}-\pnorm v{}^{2}-\pnorm w{}^{2}\right.\\
 & \ \ \ \ \left.\ \ \ -\pnorm{u+v}{}^{2}-\pnorm{v-w}{}-\pnorm{u-w}{}^{2}+\pnorm u{}^{2}+\pnorm v{}^{2}+\pnorm w{}^{2}\right)\\
 & =\frac{1}{4}\left(\pnorm{v+w}{}^{2}-\pnorm{v-w}{}^{2}\right)+\frac{1}{4}\left(\pnorm{u+w}{}^{2}-\pnorm{u-w}{}^{2}\right)\\
 & =\left(v,w\right)+\left(u,w\right)
\end{align*}
where the second equality follows from Lemma \ref{lem:polarization}
which proves additive linearity.

In order to establish multiplicative linearity on $\R$, we shall
first demonstrate it on natural numbers $\N$, then on integers $\mathds{Z},$then
on the rationals $\mathds{Q}$ and finally the entire real line. First,
we show that for any $n\in\N:\left(nv,w\right)=n\left(v,w\right).$
This follows from induction on additive linearity since
\begin{align*}
\left(nv,w\right) & =\left(\sum_{i=1}^{n}v,w\right)\\
 & =\sum_{i=1}^{n}\left(v,w\right)\\
 & =n\left(v,w\right).
\end{align*}
Next, to show that multiplicative linearity holds over $\mathds{Z},$
all we have to show is that 
\[
\left(0v,w\right)=0
\]
and
\[
\left(-v,w\right)=-\left(v,w\right).
\]
The first follows on inspection; for the second, note
\begin{align*}
\left(-v,w\right) & =\frac{1}{4}\left(\pnorm{w-v}{}^{2}-\pnorm{-\left(v+w\right)}{}^{2}\right)\\
 & =\frac{1}{4}\left(\pnorm{v-w}{}^{2}-\pnorm{v+w}{}^{2}\right)\\
 & =-\left(v,w\right)
\end{align*}
where the second equality follows from absolute homogeneity of norms.
To extend the multiplicative linearity to $\mathds{Q}$, consider
an arbitrary $q\in\mathds{Q}$ and note that by definition $q=\frac{a}{b},\left(a,b\right)\in\mathds{Z}\times\left(\mathds{Z}\setminus\left\{ 0\right\} \right)$
and so
\begin{align*}
\left(qv,w\right) & =\left(\frac{a}{b}v,w\right)\\
 & =a\left(\frac{v}{b},w\right)\\
 & =\frac{a}{4}\left(\pnorm{\frac{v}{b}+w}{}^{2}-\pnorm{\frac{v}{b}-w}{}^{2}\right)\\
 & =\frac{a}{4}\left(\pnorm{\frac{b}{b}\left(\frac{v}{b}+w\right)}{}^{2}-\pnorm{\frac{b}{b}\left(\frac{v}{b}-w\right)}{}^{2}\right)\\
 & =\frac{a}{4b^{2}}\left(\pnorm{v+bw}{}^{2}-\pnorm{v-bw}{}^{2}\right)\\
 & =\frac{a}{b^{2}}\left(v,bw\right)\\
 & =\frac{a}{b}\left(v,w\right)\\
 & =q\left(v,w\right)
\end{align*}
where we used the linearity on $\mathds{Z}$ (and symmetry) in the
second and second-to-last equalities, and the absolute homogeneity
of norms in the fifth equality. Finally, to extend this linearity
to all of $\R$, let $\alpha\in\R$ be unspecified and observe that
by the density of rational numbers in $\R$, there exists a sequence
$\left\{ q_{n}\right\} \in\mathds{Q}$ such that $\lim_{n\to\infty}q_{n}=\alpha$
and so, 
\begin{align*}
\left(\alpha v,w\right) & =\left(\lim_{n\to\infty}q_{n}v,w\right)\\
 & =\lim_{n\to\infty}\left(q_{n}v,w\right)\\
 & =\lim_{n\to\infty}q_{n}\left(v,w\right)\\
 & =\alpha\left(v,w\right)
\end{align*}
where the second equality follows by Lemma \ref{lem:continuityNorm}
and the third by our linearity result on $\mathds{Q}.$

Together, we have shown that our function $\left(\cdot,\cdot\right):V\times V\longrightarrow\R$
satisfies all the properties of the inner product, thus completing
the proof.
\end{proof}
\begin{cor}
\label{cor:L2Hilbert}Let $\left(\X,\F,\mu\right)$ be a measure space.
The space $\Lp 2{\X,\F,\mu}$ is an inner-product space with inner
product $\innerproduct{\cdot}{\cdot}:\X\times\X\to\R$ given by
\[
\innerproduct xy=\lebInt{\mu}{xy}
\]
\end{cor}

\begin{proof}
Let us verify that $\Lp 2{\mu}$ satisfies the parallelogram identity.
\hl{COMPLETE LATER}
\end{proof}

\subsection{Hilbert spaces}

Corollary \ref{cor:L2Hilbert} showed that $\Lp 2{\mu}$ is an inner-product
space but we also know, from Theorem \ref{thm:completenessLp} that
$\Lp 2{\mu}$ is a complete metric space with respect to the metric
induced by its norm. These types of spaces play a special role in
analysis and are important objects of study in of themselves in functional
analysis. In the context of probability theory, they play a key role
in the development of the theory of conditional expectations and the
existence of probability density functions.
\begin{defn}
\label{def:hilbertSpace}An inner product space $\left(\mathcal{H},\innerproduct{\cdot}{\cdot}\right)$
is called a \emph{Hilbert space }if it is complete with respect to
the norm induced by the inner product.
\end{defn}

\begin{thm}[Projection]
\label{thm:projectionThm}Let $\left(\mathcal{H},\innerproduct{\cdot}{\cdot}\right)$
be a Hilbert space and let $G\subseteq\mathcal{H}$ be a nonempty
closed convex subset. Then, for any $h\in\mathcal{H},$ there exists
a unique $g_{0}\in G$ such that 
\[
\pnorm{h-g_{0}}{}=\inf_{g\in G}\pnorm{h-g}{}
\]
where $\pnorm{\cdot}{}$ is the norm induced by $\innerproduct{\cdot}{\cdot}.$
\end{thm}

\begin{proof}
Let $\delta_{h}:=\inf_{g\in G}\pnorm{h-g}{}$ and note that by the
definition of the infimum, for every $n\in\N,$ there exists some
$g_{n}\in G$ such that 
\[
\delta_{h}\leq\pnorm{h-g_{n}}{}<\delta_{h}+\frac{1}{n}
\]
and so $\lim_{n\to\infty}\pnorm{h-g_{n}}{}=\delta_{h}.$ It turns
out that the sequence $\left\{ g_{n}\right\} _{n\in\N}$ is a Cauchy
sequence in $\mathcal{H}.$ To see this, fix $\epsilon>0$ observe
that by the definition of a limit and the continuity of $x\to x^{2}$,
there exists some $n_{0}$ such that $\pnorm{h-g_{n}}{}^{2}-\delta_{h}^{2}<\frac{\epsilon^{2}}{4}$
for every $n\geq n_{0}.$ By the parallelogram identity \ref{def:parallelogram}with
$v=h-g_{n}$ and $w=h-g_{m}$ where $m,n\geq n_{0}$
\[
\pnorm{h-\left(\frac{g_{m}+g_{n}}{2}\right)}{}^{2}+\pnorm{\frac{g_{n}-g_{m}}{2}}{}^{2}=\frac{1}{2}\left(\pnorm{h-g_{n}}{}^{2}+\pnorm{h-g_{m}}{}^{2}\right).
\]
Since $G$ is convex, $\frac{g_{m}+g_{n}}{2}\in G$ and so, again
by the definition of an infimum
\[
\delta_{h}^{2}\leq\pnorm{h-\left(\frac{g_{m}+g_{n}}{2}\right)}{}^{2}
\]
which implies that
\begin{align*}
\delta_{h}^{2}+\pnorm{\frac{g_{n}-g_{m}}{2}}{}^{2} & \leq\frac{1}{2}\left(\pnorm{h-g_{n}}{}^{2}+\pnorm{h-g_{m}}{}^{2}\right)\\
\implies\pnorm{g_{n}-g_{m}}{}^{2} & \leq2\left(\pnorm{h-g_{n}}{}^{2}-\delta_{h}^{2}+\pnorm{h-g_{m}}{}^{2}-\delta_{h}^{2}\right)\\
 & <2\left(\frac{\epsilon^{2}}{4}+\frac{\epsilon^{2}}{4}\right)\\
 & =\epsilon^{2}
\end{align*}
which then shows that $g_{n}$ is Cauchy and so by the completeness
of $\mathcal{H}$ converges to a limit in $\mathcal{H}$. However,
since $G$ is closed, it contains this limit and thus $g_{0}:=\lim_{n\to\infty}g_{n}\in G.$
Finally, by Minkowski's inequality
\begin{align*}
\pnorm{h-g_{0}}{} & \leq\pnorm{h-g_{n}}{}+\pnorm{g_{n}-g_{0}}{}\\
\end{align*}
and by taking limits on the RHS we have
\[
\pnorm{h-g_{0}}{}\leq\delta_{h}.
\]
The definition of the infimum then implies that 
\[
\delta_{h}\leq\pnorm{h-g_{0}}{}
\]
which gives equality.

Now suppose there exists some $g_{1}\in G$ such that $\pnorm{h-g_{1}}{}=\delta_{h}$.
Then, applying the parallelogram identity \ref{def:parallelogram}
again with $v=h-g_{0}$ and $w=h-g_{1}$ we have
\begin{align*}
\pnorm{h-\left(\frac{g_{0}+g_{1}}{2}\right)}{}^{2}+\pnorm{\frac{g_{0}-g_{1}}{2}}{}^{2} & =\frac{1}{2}\left(\pnorm{h-g_{0}}{}^{2}+\pnorm{h-g_{1}}{}^{2}\right)\\
 & =\delta_{h}^{2}
\end{align*}
Note again by convexity, $\frac{g_{0}+g_{1}}{2}\in G$ and so $\pnorm{h-\left(\frac{g_{0}+g_{1}}{2}\right)}{}^{2}\geq\delta_{h}^{2}$
which implies that
\[
\pnorm{\frac{g_{0}-g_{1}}{2}}{}^{2}\leq0
\]
and since norms cant be negative, we have that $\pnorm{\frac{g_{0}-g_{1}}{2}}{}^{2}=0\implies g_{0}=g_{1}$
by the definiteness of norms.
\end{proof}
\begin{rem*}
The unique vector $g_{0}\in G$ described above is called the \emph{projection
}of $h$ into $G$ and is often denoted as $P_{G}h.$
\end{rem*}
Note that the projection theorem holds in general for any nonempty,
closed, and convex subset $G$ of any Hilbert space $\mathcal{H}$
but in particular it holds for any nonempty closed \emph{subspace
}of $\mathcal{H}$, since every subspace is automatically convex.
However, in the case, of a subspace, the projection is \emph{orthogonal}.
We make this precise with the following result.
\begin{cor}
\label{cor:orthProjection}Let $\left(\mathcal{H},\innerproduct{\cdot}{\cdot}\right)$
be a Hilbert space and let $\mathcal{G\subset\mathcal{H}}$ be a closed
subspace. Then for any $h\in\mathcal{H}$, $k\in\mathcal{G}$ we have
that
\[
\innerproduct{h-k}g=0
\]
for every $g\in\mathcal{G}$ if and only if $k=P_{\mathcal{G}}h$
and so
\[
\pnorm h{}^{2}=\pnorm{P_{\mathcal{G}}h}{}^{2}+\pnorm{h-P_{\mathcal{G}}h}{}^{2}.
\]
\end{cor}

\begin{proof}
Fix $h\in\mathcal{H}$ and observe that $\tilde{g}=P_{\mathcal{G}}h+tg\in\mathcal{G}$
for any $g\in\mathcal{G}$ and $t\in\R$ since $\mathcal{G}$ is a
subspace and so by the projection theorem \ref{thm:projectionThm}
\begin{align}
\pnorm{h-P_{\mathcal{G}}h}{}^{2} & \leq\pnorm{h-\tilde{g}}{}^{2}\nonumber \\
 & =\innerproduct{h-\tilde{g}}{h-\tilde{g}}\nonumber \\
 & =\innerproduct{(h-P_{\mathcal{G}}h)-tg}{(h-P_{\mathcal{G}}h)-tg}\nonumber \\
 & =\pnorm{h-P_{\mathcal{G}}h}{}^{2}-2t\innerproduct{h-P_{\mathcal{G}}h}g+t^{2}\pnorm g{}^{2}\label{eq:convxPoly}
\end{align}
where the last equality follows by linearity of inner products. Now
if $\pnorm g{}=0$ then $g=0$ by definiteness and so $\innerproduct{h-P_{\mathcal{G}}h}g=0$
by linearity. Thus, assume that $\pnorm g{}>0$ and so notice that
the right hand side of (\ref{eq:convxPoly}) is a convex polynomial
in $t$ which is minimized at $t=\frac{\innerproduct{h-P_{\mathcal{G}}h}g}{\pnorm g{}^{2}}$
with minimum 
\[
\pnorm{h-P_{\mathcal{G}}h}{}^{2}-\frac{\innerproduct{h-P_{\mathcal{G}}h}g^{2}}{\pnorm g{}^{2}}.
\]
But since the inequality in (\ref{eq:convxPoly}) holds for any $t\in\R$,
we have 
\[
\pnorm{h-P_{\mathcal{G}}h}{}^{2}\leq\pnorm{h-P_{\mathcal{G}}h}{}^{2}-\frac{\innerproduct{h-P_{\mathcal{G}}h}g^{2}}{\pnorm g{}^{2}}
\]
 and so
\[
\innerproduct{h-P_{\mathcal{G}}h}g=0
\]
since all the terms are non-negative. Conversely, assume that $\innerproduct{h-k}g=0$
for some $k\in\mathcal{G}$ and every $g\in\mathcal{G}$. Then, we
have that
\begin{align*}
\pnorm{h-g}{}^{2} & =\pnorm{h-k+k-g}{}^{2}\\
 & =\innerproduct{(h-k)+(k-g)}{(h-k)+(k-g)}\\
 & =\pnorm{h-k}{}^{2}+\pnorm{k-g}{}^{2}+2\innerproduct{h-k}{k-g}.
\end{align*}
where we have again used the linearity and symmetry of inner products.
But notice that $k-g\in\mathcal{G},$ so by our assumption, we have
$\innerproduct{h-k}{k-g}=0$ and so for every $g\in\mathcal{G}$
\begin{align*}
\pnorm{h-g}{}^{2} & =\pnorm{h-k}{}^{2}+\pnorm{k-g}{}^{2}\\
 & \geq\pnorm{h-k}{}^{2}
\end{align*}
by the non-negativity of norms. Then by the uniqueness clause of Theorem
\ref{thm:projectionThm} $k=P_{\mathcal{G}}h$.

Finally, observe that
\begin{align*}
\pnorm h{}^{2} & =\pnorm{P_{\mathcal{G}}h+\left(h-P_{\mathcal{G}}h\right)}{}^{2}\\
 & =\pnorm{P_{\mathcal{G}}h}{}^{2}+2\innerproduct{P_{\mathcal{G}}h}{h-P_{\mathcal{G}}h}+\pnorm{h-P_{\mathcal{G}}h}{}^{2}\\
 & =\pnorm{P_{\mathcal{G}}h}{}^{2}+\pnorm{h-P_{\mathcal{G}}h}{}^{2}
\end{align*}
since $\innerproduct{h-P_{\mathcal{G}}h}{P_{\mathcal{G}}h}=0.$
\end{proof}
The following result establishes some standard properties of Hilbert
projections.
\begin{prop}[Properties of projections]
\label{prop:propertiesHilbertProjection} Let $(\mathcal{H},\innerproduct{\cdot}{\cdot})$
be a Hilbert space. For any closed subspace $\mathcal{G\subseteq\mathcal{H}}$,
the projection operator $P_{\mathcal{G}}$ has the following properties

\begin{enumerate}[label=(\roman*),leftmargin=.1\linewidth,rightmargin=.4\linewidth]
	\item (Linearity) For any $f,g \in \mathcal{H}$ and any $\alpha,\beta \in \R$
			\[
					P_{\mathcal{G}}\left(\alpha f + \beta g\right) = \alpha P_{\mathcal{G}}f + \beta P_{\mathcal{G}}. 
			\]
	\item (Tower) For any closed subspaces $\mathcal{G}_1 \subseteq \mathcal{G}_2 \subseteq \mathcal{H}$ and any $h \in \mathcal{H}$
			\[
					P_{\mathcal{G}_1} P_{\mathcal{G}_2}h = P_{\mathcal{G}_1}h
			\]
\end{enumerate}
\end{prop}

\begin{proof}
For (i), observe that for any $k\in\mathcal{G}$
\begin{align*}
\innerproduct{\alpha f+\beta g-(\alpha P_{\mathcal{G}}f+\beta P_{\mathcal{G}}g)}k & =\alpha\innerproduct{f-P_{\mathcal{G}}f}k-\beta\innerproduct{g-P_{\mathcal{G}}g}k\\
 & =0
\end{align*}
where the first equality follows from the linearity of inner products
(in the first argument) and the second from Corollary\ref{cor:orthProjection}.
Then, an application of the uniqueness clause of the same result furnishes
the linearity result.

Next, pick an arbitrary $k\in\mathcal{G}_{1}$ and note that for any
$h\in\mathcal{H}$
\begin{align*}
\innerproduct{P_{\mathcal{G}_{2}}h-P_{\mathcal{G}_{1}}h}k & =\innerproduct{P_{\mathcal{G}_{2}}h-h+h-P_{\mathcal{G}_{1}}h}k\\
 & =\innerproduct{h-P_{\mathcal{G}_{1}}h}k-\innerproduct{h-P_{\mathcal{G}_{2}}h}k\\
 & =0
\end{align*}
where the second equality is due to linearity of inner products and
the third is due the fact that both the inner products on the second
line are zero due to the same uniqueness of orthogonal projections
(the second inner product is zero because of $k\in\mathcal{G}_{1}\implies k\in\mathcal{G}_{2}$).
This is sufficient to deduce (ii) by yet another application of the
uniqueness of orthogonal projections.
\end{proof}
The projection theorem for subspaces has important consequences in
the theory of linear functionals on Hilbert spaces.
\begin{prop}
\label{prop:linearFunctionalContinuity}Let $\left(V,\pnorm{\cdot}{}\right)$
be a normed vector space and let
\[
\Gamma:V\to\R
\]
be a linear functional on $V.$ The functional $\Gamma$ is continuous
(with respect to the usual topologies) if and only if there exists
a constant $C\in\R$ such that
\begin{equation}
\lvert\Gamma\left(h\right)\rvert\leq C\pnorm h{}\label{eq:continuityConditionLinFunc}
\end{equation}
for every $h\in V.$
\end{prop}

\begin{proof}
First, suppose that a functional $\Gamma$ on satisfies the condition
in(\ref{eq:continuityConditionLinFunc}). Fix $\epsilon>0$ and notice
that for any $h,\tilde{h}\in V$, $h-\tilde{h}\in V$ and so if $\pnorm{h-\tilde{h}}{}<\frac{\epsilon}{C}$,
then by (\ref{eq:continuityConditionLinFunc})
\[
\lvert\Gamma\left(h-\tilde{h}\right)\rvert=\lvert\Gamma\left(h\right)-\Gamma\left(\tilde{h}\right)\rvert<\epsilon
\]
which proves continuity.

For the converse, assume that $\Gamma$ is continuous and fix $\epsilon=1.$
By continuity at $h=0$, there exists some $\delta_{\epsilon,0}>0$
such that
\[
\pnorm h{}\leq\delta_{\epsilon,0}\implies\lvert\Gamma\left(h\right)\rvert\leq1
\]
for any nonzero $h\in V.$ Let $g=\frac{\delta_{\epsilon,0}h}{\pnorm h{}}$
and notice that since $\pnorm g{}=\delta_{\epsilon_{0},}$
\[
\lvert\Gamma\left(g\right)\rvert\leq1
\]
which by linearity of $\Gamma$ implies that
\[
\frac{\delta_{\epsilon,0}}{\pnorm h{}}\lvert\Gamma\left(h\right)\rvert\leq1\Longleftrightarrow\lvert\Gamma\left(h\right)\rvert\leq\frac{1}{\delta_{\epsilon,0}}\pnorm h{}
\]
for every nonzero $h\in V$. Of course, if $h=0$ then the final (in)
equality follows trivially since both sides are identically zero.
This completes our proof with $C=\frac{1}{\delta_{\epsilon,0}}.$
\end{proof}
\begin{cor}
\label{cor:integrationContinuousLinearFunctional} Let $\left(\X,\F,\mu\right)$
be a finite measure space i.e. $\mu\left(\X\right)<\infty$ and define
\[
\Gamma:\Lp 2{\mu}\longrightarrow\R
\]
as
\[
\Gamma\left(f\right):=\lebInt{\mu}f
\]
is a continuous linear functional.
\end{cor}

\begin{proof}
Note that by the Cauchy-Schwarz (or H\"{o}lder's inequality)
\[
\lvert\Gamma\left(f\right)\rvert\leq\lebInt{\mu}{\lvert f\rvert}=\pnorm f1\leq\pnorm f2\pnorm{\indicate_{\X}}2=\sqrt{\mu\left(\X\right)}\pnorm f2
\]
which completes the proof.
\end{proof}
It turns out that every continuous linear functional on a Hilbert
space can be recovered as an inner product on the space. This result
is key to proving the existence of conditional expectations and density
functions in probability theory.
\begin{thm}[Riesz representation theorem]
\label{thm:rieszRep}Let $\left(\mathcal{H},\innerproduct{\cdot}{\cdot}\right)$
be a Hilbert space and let 
\[
\Gamma:\mathcal{H}\to\R
\]
be a continuous linear functional on $\mathcal{H}.$ Then there exists
a unique element $k\in\mathcal{H}$ such that
\[
\Gamma\left(h\right)=\innerproduct hk
\]
for every $h\in\mathcal{H}$.
\end{thm}

\begin{proof}
Let $\mathcal{G:=}\Gamma^{-1}\left(\left\{ 0\right\} \right)$ and
notice that $\mathcal{G}$ is a closed subspace\footnote{This type of a subspace is called the \emph{kernel }of the operator
and is often denoted\hl{Add macro}} of $\mathcal{H}$ by the continuity and linearity of $\Gamma$ and
the fact that singletons are closed in the usual topology on $\R$.
Note that if $\Gamma$ is zero everywhere, then $k=0$ uniquely satisfies
the condition of the theorem. Thus, assuming that $\Gamma$ is not
identically zero, we can find some $h_{0}\in\mathcal{H}$ such that
$\Gamma\left(h_{0}\right)\neq0$. Define $\tilde{h}:=\frac{h_{0}}{\Gamma(h_{0})}$
so that $\Gamma\left(\tilde{h}\right)=1$ and observe that by Corollary
\ref{cor:orthProjection} that
\[
\innerproduct{\tilde{h}-P_{\mathcal{G}}\tilde{h}}g=0
\]
for all $g\in\mathcal{\mathcal{G}}.$ Let $k_{0}:=\tilde{h}-P_{\mathcal{G}}\tilde{h}$
and note that by linearity
\begin{align*}
\Gamma\left(k_{0}\right) & =\Gamma\left(\tilde{h}\right)-\Gamma\left(P_{\mathcal{G}}\tilde{h}\right)\\
 & =1
\end{align*}
since $P_{\mathcal{G}}\tilde{h}\in\mathcal{G}.$ By continuity and
Proposition \ref{prop:linearFunctionalContinuity}
\[
1=\Gamma\left(k_{0}\right)\leq C\pnorm{k_{0}}{}
\]
for some real $C$ which implies that $\pnorm{k_{0}}{}>0.$

Next, observe that for any $h\in\mathcal{H}$, 
\[
h-\Gamma\left(h\right)k_{0}\in\mathcal{G}
\]
since
\begin{align*}
\Gamma\left(h-\Gamma\left(h\right)k_{0}\right) & =\Gamma\left(h\right)-\Gamma\left(h\right)\Gamma\left(k_{0}\right)\\
 & =\Gamma\left(h\right)-\Gamma\left(h\right)\\
 & =0
\end{align*}
where the second equality follows from the fact that $\Gamma\left(k_{0}\right)=1.$
This implies that 
\[
\innerproduct{k_{0}}{h-\Gamma\left(h\right)k_{0}}=0
\]
which by linearity of inner products reduces to
\[
\innerproduct{k_{0}}h=\Gamma\left(h\right)\pnorm{k_{0}}{}^{2}.
\]
Recall that $\pnorm{k_{0}}{}>0$ and so we can rearrange and apply
linearity once again to deduce
\[
\Gamma\left(h\right)=\innerproduct h{\frac{k_{0}}{\pnorm{k_{0}}{}^{2}}}
\]
which completes the existence part of the proof.

To see that the representation is unique, note that if $k_{1},k_{2}\in\mathcal{H}$
are both valid representers then
\begin{align*}
0 & =\innerproduct h{k_{1}}-\innerproduct h{k_{2}}\\
 & =\innerproduct h{k_{1}-k_{2}}
\end{align*}
for any $h\in\mathcal{H}.$ In particular, this holds $h=k_{1}-k_{2}$
in which case
\[
\pnorm{k_{1}-k_{2}}{}^{2}=0\implies k_{1}=k_{2}
\]
completing the proof.
\end{proof}
This is the classical Hilbert-space theorem that describes the duality
of linear operators and the vector spaces on which they act. This
theorem has considerable power in probability because it allows us
to construct conditional expectations as orhogonal projections of
functions into lower dimensional subspaces, tbe full power of which
will become apparent in Chapter \ref{chap:conditioning}

\subsection{Orthonormal bases}

\hl{Use proposition 19.14 in Bass for proof of existence. Use Schilling for other parts}

\section{Banach spaces over $\protect\R$\label{sec:banach-R}}

We have discussed some important special cases of complete normed
vector spacs in the form of the $L^{p}$ spaces and Hilbert spaces.
In this section, we discuss some of the properties of complete normed
vector spaces over the reals that hold without reference to any measure
or any notion of orthogonality. Without the additional structure afforded
by these ideas, the study of normed spaces becomes considerably more
complicated. Nevertheless, there are some important results in the
theory of general Banach spaces that serve as important tools in analysis,
probability, statistics, and economics. As we have already discussed
the notion of a normed vector space in the previous sections, we first
start with a discussion on the basic properties of such spaces. Most
of these should be already known to the reader (indeed, we have implicitly
used these concepts throughout this chapter).

\subsection{Review of the basic properties of normed vector spaces}
\begin{defn}
\label{def:normedVectorSpace}Let $V$ be a \hyperref[def:vectorSpace]{vector space}
over $\R$. A function $\lVert\cdot\rVert:V\times V\to\left[0,\infty\right]$
is called a norm if for any $u,v\in V$ and any $\alpha\in\R$
\end{defn}

\begin{enumerate}
\item $\lVert v\rVert=0\implies v=0$ (Positive definiteness)
\item $\lVert\alpha v\rVert=\lvert\alpha\rvert\lVert v\rVert$ (Absolute
homogeneity)
\item $\lVert u+v\rVert\leq\lVert u\rVert+\lVert v\rVert$ (Triangle inequality).
\end{enumerate}
\begin{prop}
\label{prop:normMetric}Let $\left(V,\lVert\rVert\right)$ be a normed
vector space over $\R$. Then $\left(V,d\right)$ where $d\left(u,v\right):=\lVert u-v\rVert$
is a metric space.
\end{prop}

\begin{proof}
Note that $d\left(u,v\right)=0\implies\lVert u-v\rVert=0\implies u=v$
by definiteness. Symmetry is follows from absolute homogeneity and
the fact that $v-u=-1\left(u-v\right).$ The triangle inequality follows
from the fact that for any $u,v,w\in V$
\begin{align*}
d\left(u,w\right) & =\lVert u-w\rVert\\
 & =\lVert u-v+v-w\rVert\\
 & \leq\lVert u-v\rVert+\lVert v-w\rVert\\
 & =d\left(u,v\right)+d\left(v,w\right)
\end{align*}
where the inequality is the triangle inequality for norms.
\end{proof}
\begin{defn}
\label{def:linearCombintation}Let $V$ be a vector space over $\R$.
Given a finite collection of vectors $v_{1},\ldots v_{n}$ and scalars
$\alpha_{1},\ldots,\alpha_{n}$, the sum
\[
\sum_{i=1}^{n}a_{i}v_{i}\in V
\]
is called a \emph{linear combination }of $V.$ For any subset $S\subseteq V$,
the $\spans\left(S\right)$ is the set of all linear combinations
of $S$.
\end{defn}

\begin{prop}
\label{prop:spanIsSubspace}Let $V$ be a vector space over $\R$
and let $S\subseteq V$ be a subset. Then $\spans\left(S\right)\subseteq V$
and is a vector space i.e. $\spans\left(S\right)$ is a subspace of
$V.$
\end{prop}

\begin{proof}
Note that we only need to show closure under addition and scalar multiplication,
along the existence of the additive identity; the other properties
are inherited from $V$. Let $u,v\in\spans\left(S\right).$Then $u=\sum_{i=1}^{m}a_{i}u_{i}$
and $v=\sum_{i=1}^{n}b_{i}v_{i}$ where $u_{i},v_{i}\in S$ and $a_{i},b_{i}\in\R.$
Then $u+v=\sum_{i=1}^{m}a_{i}u_{i}+\sum_{i=1}^{n}b_{i}v_{i}$ which
is another linear combination of vectors in $S$ and so $u+v\in\spans\left(S\right).$
Similarly, $\alpha v=\sum_{i=1}^{m}\left(\alpha a_{i}\right)v_{i}$
which is another linear combination of vectors in $S$. The additive
identity $\mathbf{0}\in\spans\left(S\right)$ because $0\times v=\mathbf{0}$
where $v$
\end{proof}
\begin{defn}
\label{def:equivalentNorm}Let $V$ be a vector space over $\R$.
Two norms $\lVert\cdot\rVert_{a}$ and $\lVert\cdot\rVert_{b}$ are
considered equivalent if there exist $c,C>0$ such that for any $v\in V$
\[
c\lVert v\rVert_{a}\leq\lVert v\rVert_{b}\leq C\lVert v\rVert_{b}
\]
\end{defn}


\subsection{Finite-dimensional normed vector spaces}

\section{Duality\label{sec:Duality}}



\chapter{Measure and Topology\label{chap:measureAndTopology}}

\hl{Lebesgue covering dimensions}

\hl{Hausdorff measure}

\section{Topological approximation theorems}

\hl{Use Tao (epsilon of room) + Royden (chap 12). Background in appendix chapter on topological spaces}

\section{Convergence of measures}

\hl{Ash + Klenke + Royden background on weak* convergence covered in the section on Duality}

\hl{Schilling chapter 21}



\chapter{\label{chap:Differentiation}Differentiation}

In elementary calculus, the theory of the Riemann integral was developed
with a corresponding theory of differentiation, and these two operations
were shown to be the inverse of each other for a suitable class of
functions, namely the differentiable functions; this is the fundamental
theorem of calculus. The Lebesgue theory also implies a fundamental
theorem of calculus for a broader class of functions. The goal of
this chapter is to lay the groundwork for this result -- the full
proof of which will have to be deferred to Chapter \ref{chap:productMeasures}--
and in order to do so, we will develop some theory that is very important
in its own right. Before delving in this material, you should review differentiation on the line in Appendix section \ref{subsec:reviewDifferentiation}. In the first section here, I will suitably generalize the notion of a derivative by defining
differentiability as a property of functions on arbitrary normed vector
spaces.

\section{Differentiation in normed vector spaces\label{sec:differentiation}}



\section{Decomposition of measures}

Recall that in Proposition \ref{prop:densities}, we showed that for
a measure space $\left(\X,\F,\mu\right)$ and a non-negative measurable
function $f\in\nonnegMeasurableFunctions$ , the function $\nu\left(A\right):=\lebInt{\mu}{f\indicate_{A}}$is
a measure. Moreover, in the remark following the proposition, we observed
that such a measure $\nu$ was \emph{absolutely continuous }with respect
to $\mu$, that is to say, for any $A\in\F,\mu(A)=0\implies\nu(A)=0.$
Further, we also claimed that every absolutely continuous measure
could be represented this way i.e as an integral with respect to the
dominating measure. In other words, we claimed that there was always
an $f\in\nonnegMeasurableFunctions$ such that this relation would
hold. It turns out we can establish a slightly stronger result, which
in turn is key to establishing the fundamental theorem of calculus
for the Lebesgue measure. What we can show, using the Hilbert space
machinery we developed earlier, is that we can decompose a measure
(relative to another measure) into two measures that are in some sense
orthogonal. This is in direct correspondence with the decomposition
of Hilbert spaces given to us by the projection theorem (Theorem \ref{thm:projectionThm}).

To motivate how the Hilbert space theory can be useful in this context,
we can begin with a change of perspective in the spirit of Theorem
\ref{thm:integralMeasureEquivalence}. Note that saying $\nu(A)=\lebInt{\mu}{f\indicate_{A}}$
for every $A\in\F$ is equivalent to saying $\lebInt{\nu}g=\lebInt{\mu}{fg}\forall g\in\Lp 1{\X,\F,\nu}.$
The right hand side is the inner product associated with the space
$\Lp 2{\mu}$ and so we wish to show that a linear functional $g\to\lebInt{\nu}g$
can be represented as an inner product on $\Lp 2{\mu}.$ At this juncture,
one can guess that the Riesz representation theorem for Hilbert spaces
is clearly right tool to finish off the proof; however, several hurdles
remain. For one, $g\to\lebInt{\nu}g$ a linear functional on $\Lp 1{\nu}$
and it is not $\Lp 2{\mu}.$ Further, it is unclear whether our functional
is bounded, which is a necessary condition for the theorem to apply.
The following lemma provides the conditions under which the theorem
finds purchase.
\begin{lem}
\label{lemma:LebesgueRadonNikodym}Let $\left(\X,\F\right)$ be a
measurable space. If $\mu,\nu$ are two $\sigma-$finite measures
on $\F$, then there exists some non-negative (almost everywhere with
respect to $\mu$) measurable function $k\in\nonnegMeasurableFunctions$
and a set $G\in\F$ such that $\mu(G)=0$ and
\[
\nu\left(A\right)=\lebInt{\mu}{k\indicate_{A}}+\nu(A\cap G)
\]
for every $A\in\F$. Moreover, the function $k$ is unique $\mu-$almost-everywhere
and the set $G$ is unique $\mu+\nu$-almost everywhere.
\end{lem}

\begin{proof}
Suppose first that the measures $\mu$and $\nu$are finite. Define
the measure $\psi:=\mu+\nu$ and observe that by Proposition \ref{prop:integralSumOfMeasures}
\[
\lebInt{\psi}f=\lebInt{\mu}f+\lebInt{\nu}f.
\]
Next define an operator 
\[
\Gamma:\Lp 2{\psi}\longrightarrow\R
\]
by
\[
\Gamma\left(f\right):=\lebInt{\nu}f
\]
and notice that $\Gamma$ is a continuous linear functional by a slight
variant of the argument presented in the proof of Corollary \ref{cor:integrationContinuousLinearFunctional},
using the fact that $\mu\left(\X\right),\nu\left(\X\right)<\infty$.
Since $\mathcal{L}^{2}\left(\psi\right)$ is a Hilbert space by Corollary
\ref{cor:L2Hilbert} and Theorem \ref{thm:completenessLp}, the \hyperref[thm:rieszRep]{Riesz representation theorem} tells
us there exists an (almost $\psi-$everywhere unique) function $h\in\Lp 2{\psi}$
such that
\begin{align*}
\Gamma\left(f\right) & =\innerproduct fh
\end{align*}
for every $f\in\Lp 2{\psi}.$ Of course, in the context of $\mathcal{L}^{2}$,
\begin{equation}
\innerproduct fh=\pnorm{fh}1=\lebInt{\psi}{fh}=\lebInt{\mu}{fh}+\lebInt{\nu}{fh}.\label{eq:rieszRepresenterL2}
\end{equation}
Next consider the following measurable partition of $\X$
\[
N:=\left\{ x\in\X\mid h\left(x\right)<0\right\} ,M:=\left\{ x\in\X\mid0\leq h\left(x\right)<1\right\} ,G:=\left\{ x\in\X\mid h\left(x\right)\geq1\right\} .
\]
Note that by the fact that $0\geq h\indicate_{N}$ and the monotonicity
of integration
\begin{align*}
0 & \geq\lebInt{\psi}{h\indicate_{N}}\\
 & =\lebInt{\mu}{h\indicate_{N}}+\lebInt{\nu}{h\indicate_{N}}\\
 & =\lebInt{\nu}{\indicate_{N}}
\end{align*}
where the last equality is due to (\ref{eq:rieszRepresenterL2}).
But since $\lebInt{\nu}{\indicate_{N}}=\nu\left(N\right),$we have
$\nu\left(N\right)=0$ by non-negativity of measures. Then
\[
\lebInt{\mu}{h\indicate_{N}}+\lebInt{\nu}{h\indicate_{N}}=0
\]
where the second term is automatically zero since $h\indicate_{N}\stackrel{\nu-\text{a.e}}{=}0$.
Therefore, 
\[
\lebInt{\mu}{h\indicate_{N}}=0
\]
which by Proposition \ref{prop:intZeroFuncZero} implies that
\[
h\indicate_{N}\stackrel{\mu-\text{a.e}}{=}0\implies\mu\left(N\right)=0
\]
and so both $\mu\left(N\right)=\nu\left(N\right)=0.$

On the other hand, observe that 
\begin{align*}
\Gamma\left(\indicate_{G}\right) & =\nu\left(G\right)\\
 & =\lebInt{\mu}{h\indicate_{G}}+\lebInt{\nu}{h\indicate_{G}}\\
 & \geq\mu\left(G\right)+\nu\left(G\right)
\end{align*}
where the inequality follows from the fact that $h\indicate_{G}\geq\indicate_{G}$
and the monotonicity of the integral. Since $\nu\left(G\right)<\infty$
as $\nu$ is a finite measure, we can subtract if from both sides
of the inequality to deduce
\[
\mu\left(G\right)\leq0
\]
 which reduces to equality by the non-negativity of measures.

To control the final piece of the partition, observe that we can use
the definition $\Gamma\left(f\right)=\lebInt{\nu}f$ and Eq. (\ref{eq:rieszRepresenterL2})
together to deduce (since all terms are finite) that
\[
\lebInt{\nu}f-\lebInt{\nu}{fh}=\lebInt{\mu}{fh}
\]
for $f\in\Lp 2{\psi}.$ By linearity this reduces to~
\begin{equation}
\lebInt{\nu}{\left(1-h\right)f}=\lebInt{\mu}{hf}.\label{eq:oneminusH}
\end{equation}
Next we define the increasing sequence of sets
\[
M_{n}:=\left\{ x\in\X\mid0\leq h\left(x\right)\leq1-\frac{1}{n}\right\} 
\]
and note that
\[
M=\bigcup_{n\in\N}M_{n}.
\]
Further, observe that for any $A\in\F$ the functions
\[
f_{n}:=\frac{\indicate_{M_{n}}\indicate_{A}}{1-h}\leq n\indicate_{M_{n}}\indicate_{A}
\]
and
\[
0\leq f_{n}\leq f_{n+1}
\]
pointwise for all $n\in\N$. This shows us that our functions $f_{n}\in\Lp 2{\psi}$\footnote{$n\indicate_{M_{n}}$is square integrable since $\psi$ is a finite
measure; $f_{n}$ then is integrable since both functions are non-negative
and so 
\[
\lebInt{\psi}{\lvert f_{n}\rvert^{2}}\leq\lebInt{\psi}{\lvert n\indicate_{M_{n}}\indicate_{A}\rvert^{2}}\leq n^{2}\psi\left(\left\{ 0\leq h\leq1-\frac{1}{n}\right\} \right)
\]
} and are monotonically increasing. Therefore,
\begin{align*}
\nu\left(M\cap A\right) & =\lebInt{\nu}{\indicate_{M}\indicate_{A}\frac{1-h}{1-h}}\\
 & =\lim_{n\to\infty}\lebInt{\nu}{\frac{\indicate_{M_{n}}\indicate_{A}}{1-h}1-h}\\
 & =\lim_{n\to\infty}\lebInt{\mu}{\frac{\indicate_{M_{n}}\indicate_{A}}{1-h}h}\\
 & =\lebInt{\mu}{\indicate_{M}\indicate_{A}\frac{h}{1-h}}+\lebInt{\mu}{\indicate_{N\cup G}\indicate_{A}\frac{h}{1-h}}\\
 & =\lebInt{\mu}{\frac{h}{1-h}\indicate_{A}}
\end{align*}
where the second equality follows by the \hyperref[thm:generalizedMonotoneConvergence]{monotone convergence theorem},
the third equality is due to (\ref{eq:oneminusH}), the fourth equality
is due to the fact that $\mu\left(G\right)=\mu\left(N\right)=0$ and
the last equality is due to the linearity of integration. 

Letting $k=\frac{h}{1-h}$ we have that 
\[
\mu\left(\left\{ x\in\X\mid k\left(x\right)<0\right\} \right)=0
\]
since $\mu\left(N\right)=\mu\left(G\right)=0.$ Therefore
\begin{align*}
\nu\left(A\right) & =\nu\left(\left(A\cap N\right)\cup\left(A\cap M\right)\cup\left(A\cap G\right)\right)\\
 & =\nu\left(A\cap N\right)+\nu\left(A\cap M\right)+\nu\left(A\cap G\right)\\
 & =\lebInt{\mu}{k\indicate_{A}}+\nu(A\cap G)
\end{align*}
where the first equality follows from the fact that $N,M,$ and $G$
form a partition of $\X$, the second equality by finite additivity,
and the final equality by the fact that $N$ is a $\nu-$null set.

The extension to $\sigma-$finite measures is relatively straightforward.
Let $\left\{ E_{i}\right\} _{i\in\N}\in\F$ be a partition of $\X$
such that $\mu$and $\nu$are finite on each $E_{i}$ (this is possible
by Propositions \ref{prop:equivSigmaFinite} and \ref{prop:sumSigmaFiniteMeasures}).
Then the measures given by
\begin{align*}
\nu_{i}\left(A\right) & :=\nu\left(A\cap E_{i}\right),\\
\mu_{i}\left(A\right) & :=\mu\left(A\cap E_{i}\right)
\end{align*}
are finite on $\F$ and so the finite version of theorem implies the
existence of functions $k_{i}\in\nonnegMeasurableFunctions$ and $\mu_{i}-$null
sets $B_{i}$ such that 
\[
\nu_{i}\left(A\right)=\lebInt{\mu_{i}}{k_{i}\indicate_{A}}+\nu_{i}\left(A\cap B_{i}\right)
\]
which is equivalent to saying that 
\[
\nu\left(A\cap E_{i}\right)=\lebInt{\mu}{k_{i}\indicate_{A}\indicate_{E_{i}}}+\nu\left(A\cap B_{i}\cap E_{i}\right)
\]
for each $i\in\N$. Summing over $i$ we have 
\begin{align*}
\nu\left(A\right) & =\sum_{i=1}^{\infty}\nu\left(A\cap E_{i}\right)\\
 & =\sum_{i=1}^{\infty}\lebInt{\mu}{k_{i}\indicate_{E_{i}}\indicate_{A}}+\sum_{i=1}^{\infty}\nu\left(A\cap B_{i}\cap E_{i}\right)\\
 & =\lebInt{\mu}{\sum_{i=1}^{\infty}k_{i}\indicate_{E_{i}}\indicate_{A}}+\nu\left(A\cap\left(\cup_{i\in\N}\left(B_{i}\cap E_{i}\right)\right)\right).
\end{align*}
where the last equality follows by monotone convergence and countable
additivity. Note that $\mu\left(\bigcup_{i\in\N}\left(B_{i}\cap E_{i}\right)\right)=\sum_{i=1}^{\infty}\mu\left(B_{i}\cap E_{i}\right)=\sum_{i=1}^{\infty}\mu_{i}\left(B_{i}\right)=0.$
This completes the proof of existence with $k=\sum_{i=1}^{\infty}k_{i}\indicate_{E_{i}}$
and $B=\bigcup_{i\in\N}\left(B_{i}\cap E_{i}\right)$.

Finally, for uniqueness, consider two pairs $\left(k_{1},G_{1}\right)$
and $\left(k_{2},G_{2}\right)$ such that $k_{1},k_{2}\in\nonnegMeasurableFunctions,$
and $G_{1},G_{2}\in N_{\mu},$ and
\[
\nu\left(A\right)=\lebInt{\mu}{k_{1}\indicate_{A}}+\nu\left(A\cap G_{1}\right)=\lebInt{\mu}{k_{2}\indicate_{A}}+\nu\left(A\cap G_{2}\right)
\]
for every $A\in\F.$ Then,
\begin{align*}
\nu\left(G_{2}\cap G_{1}^{C}\right) & =\lebInt{\mu}{k_{1}\indicate_{G_{2}}\indicate_{G_{1}^{C}}}+\nu\left(\underbrace{G_{2}\cap G_{1}^{C}\cap G_{1}}_{=\emptyset}\right)\\
 & =0
\end{align*}
where the first term is zero because $G_{1}\in N_{\mu}.$ Similarly,
we can show that 
\begin{align*}
\nu\left(G_{1}\cap G_{2}^{C}\right) & =\lebInt{\mu}{k_{2}\indicate_{G_{1}}\indicate_{G_{2}^{C}}}+\nu\left(\underbrace{G_{1}\cap G_{2}^{C}\cap G_{2}}_{=\emptyset}\right)\\
 & =0.
\end{align*}
In other words, $\nu\left(G_{1}\Delta G_{2}\right)=0$ which by Proposition
\ref{prop:almostEverywhereEqualSets} implies that $G_{1}\stackrel{\nu-\text{a.e}}{=}G_{2}.$
Of course, since $G_{1},G_{2}\in N_{\mu},$ we have that $\mu\left(A\Delta B\right)=0$
and so $G_{1}\stackrel{\mu+\nu-\text{a.e}}{=}G_{2}.$ Note that since
$G_{1}^{C}\Delta G_{2}^{C}=G_{1}\Delta G_{2}$, we have that $G_{1}^{C}\stackrel{\mu+\nu-\text{a.e}}{=}G_{2}^{C}$
and so for any $A\in\F$
\[
\nu\left(A\cap G_{1}^{C}\right)=\nu\left(A\cap G_{2}^{C}\right).
\]
But
\begin{align*}
\nu\left(A\cap G_{1}^{C}\right) & =\lebInt{\mu}{k_{1}\indicate_{A}\indicate_{G_{1}^{C}}}+\nu\left(\underbrace{A\cap G_{1}^{C}\cap G_{1}}_{=\emptyset}\right)\\
\nu\left(A\cap G_{2}^{C}\right) & =\lebInt{\mu}{k_{2}\indicate_{A}\indicate_{G_{2}^{C}}}+\nu\left(\underbrace{A\cap G_{2}^{C}\cap G_{2}}_{=\emptyset}\right)
\end{align*}
This combined with the fact that $G_{1}$ and $G_{2}$ are $\mu-$null
show that for every $A\in\F$ we have $\lebInt{\mu}{k_{1}\indicate_{A}}=\lebInt{\mu}{k_{2}\indicate_{A}}$
which by Proposition \ref{prop:intEqualFuncEqual} implies that $k_{1}\stackrel{\mu-\text{a.e}}{=}k_{2}.$
\end{proof}
This is a powerful lemma as it gives us two very important results
as simple corollaries.
\begin{defn}
\label{def:absoluteContinuityMeasures}Let $(\X,\F)$ be a measurable
space and let $\mu,\nu$ be two measures on this space. We say $\nu$
is absolutely continuous with respect to $\mu$ if for any $A\in F:\mu(F)=0\implies\nu(F)=0$.
We denote this relation by $\nu<<\mu$.

Why is this relation labeled continuity? There are two good reasons
for this: first, at least for finite measures, the above definition
is equivalent to an $\epsilon-\delta$ definition that resembles the
continuity definitions that we have seen before. More importantly,
we will see that measures that are absolutely continuous with respect
to the Lebesgue measure give rise to a particular class of real-valued
\emph{functions }with the property of absolute continuity. Such functions
are exactly the class of functions for which the Lebesgue fundamental
theorem of calculus applies.
\end{defn}

\begin{prop}
\label{prop:epsdeltaAbsContinuity}Let $(\X,\F)$ be a measurable
space and let $\mu,\nu$ be two measures on this space such that $\nu(X)<\infty$.
Then $\nu<<\mu$ if and only if for every $\epsilon>0$ there exists
a $\delta>0$ such that for any $A\in\F$, $\mu(A)<\delta\implies\nu(A)<\epsilon$
\end{prop}

\begin{proof}
Suppose that $\epsilon-\delta$ characterization holds. Take a sequence
$\epsilon_{n}>0$ such that $\epsilon_{n}\to0$ and find the corresponding
$\delta_{n}>0$ such that $\mu(A)<\delta_{n}\implies\nu(A)<\epsilon_{n}$.
If $\mu(A)=0$ then $\mu(A)<\delta_{n}$ for all $n\in\N$. Then $\nu(A)<\epsilon_{n}$
for every $n.$ Taking limits yields $\nu(A)=0.$

Conversely, assume that $\epsilon-\delta$ property does not hold.
Then, there exists some $\epsilon_{0}>0$ such that for any $\delta>0$,
there's some $A\in\F$ such that $\mu(A)<\delta$ but $\nu(A)\geq\epsilon$.
Let $\delta_{n}=\frac{1}{2^{n}}\to0$ and let $A_{n}\in\F$ be the
corresponding sequence of sets such that $\mu(A_{n})<\delta_{n}$
but $\nu(A_{n})\geq\epsilon.$ Note then $\sum_{n=1}^{\infty}\mu(A_{n})<1$
and so by the \hyperref[thm:borelCantelli]{Borel-Cantelli lemma},
\[
\mu(\limsup_{n\to\infty}A_{n})=0.
\]
where $\limsup_{n\to\infty}A_{n}=\cap_{n=1}^{\infty}\cup_{i=1}^{n}A_{i}\in\F$.
Since $\nu(X)<\infty,$ by the \hyperref[cor:reverseFatouLemma]{reverse Fatou lemma}
\[
\nu(\limsup_{n\to\infty}A_n)\geq\limsup_{n\to\infty}\nu(A_{n})\geq\epsilon
\]
which completes the proof.
\end{proof}
While this is one way to characterize absolutely continuous measures,
a different characterization provides the shortest route to the fundamental
theorem of calculus.
\begin{thm}[Radon-Nikodym]
\label{thm:radonNikodym}Let $\left(\X,\F\right)$ be a measurable
space. If $\mu,\nu$ are two $\sigma$-finite measures on $\F$ then
$\nu<<\mu$ if and only if there exists almost-everywhere unique (with
respect to both measures) non-negative function $f\in\nonnegMeasurableFunctions$
such that 
\[
\nu(A)=\lebInt{\mu}{f\indicate_{A}}
\]
for every $A\in\F$
\end{thm}

\begin{proof}
Note that the existence of $f$ implies absolute continuity by Proposition
\ref{prop:densities}. The converse is far more challenging; fortunately
for us, Lemma \ref{lemma:LebesgueRadonNikodym} does all of the work.
To see this, note that by our lemma

\[
\nu(A)=\lebInt{\mu}{\tilde{f}\indicate_{A}}+\nu(A\cap B)
\]
where $A\in\F$, $B\in N_{\mu}$ and $\tilde{f}$ is $\mu-$almost
everywhere non-negative and unique. If $\nu<<\mu$ then $\nu(A\cap B)=0$.
Letting $f=\tilde{f}\indicate_{{\tilde{f}\geq0}}$, our result follows
by Proposition \ref{prop:funcEqualityAlmostEverywhere}.
\end{proof}
\begin{cor}
\label{cor:radonNikodymIntegral}Let $\left(\X,\F\right)$ be a measurable
space. If $\mu,\nu$ are two $\sigma$-finite measures on $\F$ such
that $\mu<<\nu$ then there exists a unique $f\in\nonnegMeasurableFunctions$
such that for any $g\in\nonnegMeasurableFunctions$ 
\[
\lebInt{\mu}g=\lebInt{\nu}{gf}.
\]
\end{cor}

\begin{proof}
(Sketch) This is the standard approximation argument. For indicator
functions the result follows by the Radon-Nikodym theorem. For any
simple function, it follows by the linearity of the integral (and
the result for indicators). Finally for non-negative measurable function
it follows by monotone convergence (and the result for simple functions).
Uniqueness follows by Proposition \ref{prop:intEqualFuncEqual}.
\end{proof}
The Radon-Nikodym theorem builds on the Hilbert space theory we described
earlier in Chapter \ref{chap:spaces_of_functions} and is the proper
converse of the comparatively simple result on constructing new measures
using non-negative measurable functions that we saw in Proposition
\ref{prop:densities}. This theorem is also fundamental to developing
conditional expectations; indeed the existence and uniqueness of conditional
expectations is an almost trivial corollary of the Radon-Nikodym theorem.
In the context of probability theory, the functions described by the
Radon-Nikodym theorem are probability density functions. Recall from
undergraduate probability that probability density functions could
be recovered as derivatives of cumulative distribution functions,
provided those distribution functions were sufficiently well behaved.
It turns out such densities can be thought of as derivatives generally
in an admittedly contrived sense: if we write the Radon Nikodym theorem
in traditional notation, then we have can write for $\nu<<\mu$
\[
\nu\left(A\right)=\int_{A}fd\mu.
\]
Then, in Leibniz notation, we could write $f$ as $\frac{d\nu}{d\mu}$
and refer to $f$ as the \emph{Radon-Nikodym derivative }of $\nu$
with respect to $\mu$. Of course, for function to actually be some
sort of derivative of measures, we need to be able to represent it
as a limiting ratio of the two measures; in the special case of the
Lebesgue measure, we can indeed do this. However, the notation is
useful more generally, partly due to the following facts
\begin{prop}
\label{prop:RadonNikodymFacts}Let $\measurablespace$ be a measurable
space and let $\mu,\nu$, and $\gamma$be $\sigma-$finite measures
on the space. Then

\begin{enumerate}[label=(\roman*),leftmargin=.1\linewidth,rightmargin=.4\linewidth]
	\item If $\mu<<\nu$ and $\nu<<\gamma$  then $\mu << \gamma$  and
\begin{equation}
\tag{Chain rule}
	\frac{d\mu}{d\gamma}\stackrel{\text{a.e}}{=}\frac{d\mu}{d\nu}\frac{d\nu}{d\gamma}
\label{eq:chainRule}
\end{equation}
	\item  If $\mu << \gamma $ and $\nu << \gamma $ then the sum measure $\mu + \nu$ is $\sigma-$finite, $\mu + \nu << \gamma$, and 
\begin{equation}
\tag{Sum Rule}
	\frac{d\mu+\nu}{d\gamma}\stackrel{\gamma-\text{a.e}}{=}\frac{d\mu}{d\gamma}+\frac{d\nu}{d\gamma}
\label{eq:sumRule}
\end{equation}
	\item If $\mu << \nu$ then $\nu << \mu$ if and only if $\frac{d\mu}{d\nu}>0$ $\nu-$a.e and then
\begin{equation}
\tag{Inverse ``function" rule}
	\frac{d\nu}{d\mu}=\frac{1}{\frac{d\mu}{d\nu}}.
\label{eq:invFunctionRule}
\end{equation}
\end{enumerate}
\end{prop}

\begin{proof}
To show \emph{(i),} let $A\in\F$ be such that $\gamma\left(A\right)=0.$Then
$\nu\left(A\right)=0$ since $\nu<<\gamma.$ Then $\mu\left(A\right)=0$
since $\mu<<\nu$ which establishes that absolute continuity is transitive.
Note that since the measures are $\sigma-$finite, by the Radon Nikodym
theorem there exists some almost everywhere unique non-negative measurable
functions $f,g,h$ such that for any $A\in\F$
\begin{align*}
\mu\left(A\right) & =\lebInt{\gamma}{h\indicate_{A}}\\
\mu\left(A\right) & =\lebInt{\nu}{f\indicate_{A}}\\
\nu\left(A\right) & =\lebInt{\gamma}{g\indicate_{A}}
\end{align*}
and so by Corollary \ref{cor:densityIntegral}, we have that - for
any $A\in\F$
\[
\mu\left(A\right)=\lebInt{\nu}{f\indicate_{A}}=\lebInt{\gamma}{fg\indicate_{A}}.
\]
Since the function $fg$ is non-negative measurable, we have by the
uniqueness of Radon-Nikodym derivatives that 
\[
h\stackrel{\text{a.e}}{=}fg
\]
which completes the proof.

For \emph{(ii), }note that $\mu+\nu$ is $\sigma-$finite by Proposition
\ref{prop:sumSigmaFiniteMeasures}. Absolute continuity follows trivially.
To show the almost sure equivalence of the Radon-Nikodym derivatives,
let $f,g,h\in\nonnegMeasurableFunctions$ be Radon-Nikodym derivatives
such that for every $A\in\F$
\begin{align*}
\mu\left(A\right)+\nu\left(A\right) & =\lebInt{\gamma}{h\indicate_{A}}\\
\mu\left(A\right) & =\lebInt{\gamma}{f\indicate_{A}}\\
\nu\left(A\right) & =\lebInt{\gamma}{g\indicate_{A}}.
\end{align*}
By linearity of the Lebesgue integral, we have that for every $A\in\F$
\[
\lebInt{\gamma}{\left(f+g\right)\indicate_{A}}=\lebInt{\gamma}{h\indicate_{A}}.
\]
By Proposition \ref{prop:intEqualFuncEqual} we have that 
\[
h\stackrel{\gamma-\text{a.e}}{=}f+g.
\]
Finally, to show \emph{(iii), }let $f$ denote $\frac{d\mu}{d\nu}$
and suppose that that $f\stackrel{\nu-\text{a.e}}{>}0$ and that $\mu\left(A\right)=0$
for some $A\in\F.$ The Radon-Nikodym theorem implies then that $\lebInt{\nu}{f\text{}\indicate_{A}}=0$.
Since $f$ is positive almost everywhere, Proposition \ref{prop:intZeroFuncZero}
implies that $\indicate_{A}\stackrel{\nu-\text{a.e}}{=}0\Longleftrightarrow\nu\left(A\right)=0.$
Conversely, suppose $\nu<<\mu$ and there's a set of positive (with
respect to both measures) mass $B\in\F$ such that $f=0$ on $B$.
Then $\mu\left(B\right)=\lebInt{\nu}{f\indicate_{B}}=0$ which is
a contradiction. To conclude, observe that when $\frac{d\mu}{d\nu}\stackrel{\text{a.e}}{>}0$,
both measures are \emph{mutually }absolutely continuous (or equivalent)
and so $\frac{d\nu}{d\mu}$ exists and by part \emph{(i) }above
\[
1=\frac{d\mu}{d\mu}=\frac{d\mu}{d\nu}\frac{d\nu}{d\mu}.
\]
Taking reciprocals yields the result.
\end{proof}
Absolute continuity is a special property between pairs of measures
characterized by the notion that the null sets of a measure are a
subset of the other; that is to say $\mu<<\nu\Longleftrightarrow N_{\nu}\subseteq N_{\mu}$.
Of course, if the two measures $\mu,\nu$ are mututally absolutely
continuous (or equivalent), then $N_{\mu}=N_{\nu}.$ Dual to this
notion of absolute continuity, we can define the concept of mutual
singularity, which corresponds to the situation where $N_{\nu}$ and
$N_{\mu}$ can together cover $\X$
\begin{defn}
\label{def:concentrationOfMeasure}Let $\measurespace$ be a measure
space. The measure $\mu$ is said to \emph{concentrate }on a set $A\in\F$
if $\mu\left(E\right)=\mu\left(A\cap E\right).$ Equivalenty, one
can say that $\mu$ concentrates on $A$ if $\mu\left(E\right)=0$
if and only if $E\cap A=\emptyset.$
\end{defn}

\begin{defn}
\label{def:mutualSingularity}Let $\measurablespace$ be a measurable
space. Two measures $\mu,\nu$ on $\F$ are said to be \emph{mutually
singular }if there exists two disjoint sets $A,B\in\F$ such that
$\mu$ concentrates on $A$ and $\nu$ concentrates on $B$. We can
then write $\mu\perp\nu$.
\end{defn}

In the next result we summarize some basic properties of mutual singularity
and its relationship with absolute continuity.
\begin{prop}
\label{prop:propertiesSingularityAbsContinuity}Let $\measurablespace$
be a measurable space and let $\mu,\nu,$ and $\gamma$ be three measures
on $\F$. Then

\begin{enumerate}[label=(\roman*),leftmargin=.1\linewidth,rightmargin=.4\linewidth]
	\item $\mu \perp \nu$ if and only if there exist two disjoint sets $A,B \in \F$ such that $A \cup B= \X$ and $\mu(A) = 0 , \nu(A) = 0 $ .
	\item  If $\mu \perp \gamma $ and $\nu \perp \gamma $ then $\mu + \nu \perp \gamma$.
	\item If $\mu << \gamma $ and $\nu \perp \gamma$ then $\mu \perp \nu$.
	\item If $\mu << \nu $ and $\mu \perp \nu$ then $\mu = 0$.
\end{enumerate}
\end{prop}

\begin{proof}
For \emph{(i), }observe that if $\mu\perp\nu$ then there exist disjoint
sets $C,D\in\F$ such that $\mu$ concentrates on $C$ and $\nu$
concentrates on $D.$ By definition, we have that $\mu\left(C^{C}\right)=\mu\left(C\cap C^{C}\right)=0$
and $\nu\left(D^{C}\right)=\nu\left(D\cap D^{C}\right)=0.$ Note that
$\nu\left(C\right)=\nu\left(C\cap D\right)=0$ and so $A=C^{C}$and
$B=D^{C}$ does the trick. Conversely, let $A,B$ be as given in the
hypothesis. Then $\mu\left(F\right)=\mu\left(F\cap A\right)+\mu\left(F\cap B\right)=\mu\left(F\cap B\right)$
and similarly $\nu\left(F\right)=\nu\left(F\cap A\right)$ for every
$F\in\F$.

For \emph{(ii), }let $A_{1},B_{1}\in\F$ be a partition of $\X$ such
that $\mu\left(A_{1}\right)=0$ and $\gamma\left(B_{1}\right)=0.$
Similarly, let $A_{2},B_{2}\in\F$ be a partition of $\X$ such that
$\nu\left(A_{2}\right)=0$ and $\gamma\left(B_{2}\right)=0.$ Let
$C=A_{1}\cap A_{2}$ and $D=\left(A_{1}\cap A_{2}\right)^{C}=B_{1}^{C}\cup B_{2}^{C}.$
Clearly, $C$ and $D$ form a partition of $\X$ such that $\mu+\nu\left(A_{1}\cap A_{2}\right)=\mu\left(A_{1}\cap A_{2}\right)+\nu\left(A_{1}\cap A_{2}\right)=0$
by subadditivity. Similarly, $\gamma\left(B_{1}\cup B_{2}\right)\leq\gamma\left(B_{1}\right)+\gamma\left(B_{2}\right)=0.$

Next, for \emph{(iii) }let $A\in\F$ such that $\nu\left(A\right)=0$
and $\gamma\left(A^{C}\right)=0$. Absolute continuity implies that
$\mu\left(A^{C}\right)=0.$ This shows $\mu\perp\nu.$

Finally, suppose that $A\in\F$ such that $\mu\left(A\right)=0$ and
$\nu\left(A^{C}\right)=0.$ By absolute continuity, $\mu\left(A\right)=0.$
Then, $\mu\left(\X\right)=\mu\left(A\right)+\mu\left(A^{C}\right)=0.$
\end{proof}
It turns out that absolute continuity and singularity represent a
sort of \emph{orthogonality }notion for measures. Considering that
the spaces of measures is a subset of the dual space of $L^{1}$ functions,
this does not correspond to the canonical notion of orthogonality.
\hl{MAKE PRECISE USING TARCSAY 2014. }Again, this result falls trivially
out of Lemma \ref{lemma:LebesgueRadonNikodym}.
\begin{thm}[Lebesgue Decomposition]
\label{thm:lebesgueDecomposition}Let $\measurablespace$ be a measurable
space and let $\mu$ and $\nu$ be two $\sigma-$finite measures on
$\F$. Then there exist two unique measures $\mu_{1},\mu_{2}$ such
that 
\[
\mu=\mu_{1}+\mu_{2}
\]
and $\mu_{1}<<\nu$ and $\mu_{2}\perp\nu$. Such a pair $\left(\mu_{1},\mu_{2}\right)$
is called a Lebesgue decomposition of $\mu$ with respect to $\nu$.
\end{thm}

\begin{proof}
Note that existence follows from Lemma \ref{lemma:LebesgueRadonNikodym}.
To see this, write 
\[
\mu\left(A\right)=\lebInt{\nu}{k\indicate_{A}}+\mu\left(A\cap B\right)
\]
where $B\in N_{\nu}$ and $k$ is non-negative almost-everywhere with
respect to $\nu.$ Then setting $\text{\ensuremath{\mu_{1}\left(A\right):=\lebInt{\nu}{k\indicate_{A}}}}$we
have that $\mu_{1}<<\nu$. Similarly, setting $\mu_{2}\left(A\right):=\mu\left(A\cap B\right)$
which is a measure since it satisfies countable additivity and is
null on the empty set. Then note that $\nu\left(B\right)=0$ and $\mu_{2}\left(B^{C}\right)=0$
which implies that $\mu_{2}\perp\nu$.
\end{proof}
Note that we postpone the proof of the uniqueness of the decomposition
to the following subsection, since we shall need the concept of a
signed measure.

\subsection{Signed measures, Duality of $L^{p}$ spaces, and the Riesz representation
theorem.}

\hl{Tao (epsilon of room vol 1, section 1.3.2)?}

\section{Absolutely continuous functions}

The absolute continutiy of measures is in some sense a generalization
of the notion of absolute continuity of a real-valued function on
$\left[a,b\right]\subset\R$.
\begin{defn}
\label{def:absolutelyContinuousFunction}A function $f:\left[a,b\right]\to\R$
is said to be \emph{absolutely continuous }if for every $\epsilon>0$,
there exists some $\delta>0$ such that for any finite collection
of disjoint open intervals $\left\{ \left(a_{i},b_{i}\right)\right\} _{i=1}^{n}\subset\left[a,b\right]$,
\[
\sum_{i=1}^{n}\left(b_{i}-a_{i}\right)<\delta\implies\sum_{i=1}^{n}\lvert f\left(b_{i}\right)-f\left(a_{i}\right)\rvert<\epsilon.
\]
\end{defn}

It should be immediately clear that every absolutely continuous function
is uniformly continuous (and hence continuous). The converse is not
true, however: the \hyperref[def:cantorFunction]{Cantor function}
is a canonical example of a function that is uniformly continuous
but not absolutely continuous. The Cantor function is continuous on
$\left[0,1\right]$ by Proposition \ref{prop:cantorFunctionContinuous}
and hence uniformly continuous since $\left[0,1\right]$ is compact
(see Theorem\ref{thm:compactUniformContinuity}). One the other hand,
the Cantor function ``concentrates'' on $C$, in that the image
of the Cantor set under the Cantor function is all of $\left[0,1\right]$,
and so for any finite collection $\left\{ \left(a_{i}b_{i}\right)\right\} _{i=1}^{n}$
of disjoint intervals that covers the entire Cantor set, the corresponding
image sequence $\sum_{i=1}^{n}\lvert f\left(b_{i}\right)-f\left(a_{i}\right)\rvert\geq1.$
Of course, since the Cantor set has measure zero, it can be covered
by countably many open intervals whose total length is arbitrarily
small. Moreover, since the Cantor set is compact, we can extract a
finite subcollection of such intervals (disjointly, without loss of
generality) and so absolute continuity fails for any $0<\epsilon<1$.

The fact that the Cantor function concentrates on $C$ should be telling;
the measure that the Cantor function corresponds to is singular with
respect to the Lebesgue measure. As we shall see, the duality of singular
and absolutely continuous functions mimics the duality of singular
and absolutely continuous measures with respect to the Lebesgue measure.
We start with the following fact that links absolute continuity of
Stieljes functions to the absolute continuity of the measures induced
by such functions with respect to the Lebesgue measure.
\begin{prop}
\label{prop:absoluteContinuityStieljesFunctions}Let $F:\left[a,b\right]\to\R$
be a Stieljes function. Then the measure $\mu_{F}\left(\left(x,y\right]\right):=F\left(y\right)-F\left(x\right)$
for $a\leq x\leq y\leq b$ on $\borel\left(\left[a,b\right]\right)$
is absolutely continuous with respect to the Lebesgue measure if and
only if $F$ is absolutely continuous as a function.
\end{prop}

\begin{proof}
Note that both $\mu_{F}$ and $\lambda$ are finite measures on $\borel\left(\left[a,b\right]\right)$
and so if $\mu_{F}\ll\lambda$ then (by Proposition \ref{prop:epsdeltaAbsContinuity})
for any $\epsilon>0$ there exists some $\delta>0$ such that for
any $B\in\borel\left(\left[a,b\right]\right):\lambda\left(B\right)<\delta\implies\mu_{F}\left(B\right)<\epsilon.$
In particular, for $B=\bigcup_{i=1}^{n}\left(a_{i},b_{i}\right)$
where $\left(a_{i},b_{i}\right)\subset\left[a,b\right]$ are disjoint,
we have that 
\[
\lambda\left(\bigcup_{i=1}^{n}\left(a_{i},b_{i}\right)\right)>\delta\implies\mu_{F}\left(\bigcup_{i=1}^{n}\left(a_{i},b_{i}\right)\right)<\epsilon
\]
and applying finite additivity yields one side of the result. Conversely,
assume $F$ is absolutely continuous as a function and so for any
$\epsilon>0$ there exists some $\delta>0$ such that for any finite
collection of disjoint open intervals $\left\{ \left(a_{i},b_{i}\right)\right\} _{i=1}^{n}\subset\left[a,b\right]$
\[
\sum_{i=1}^{n}\left(b_{i}-a_{i}\right)<\delta\implies\sum_{i=1}^{n}f\left(b_{i}\right)-f\left(a_{i}\right)<\epsilon.
\]
Taking limits, we can extend this result to countable disjoint collections
so that for any $\left\{ \left(a_{i},b_{i}\right)\right\} _{i\in\N}$
disjoint
\[
\sum_{i=1}^{\infty}\left(b_{i}-a_{i}\right)<\delta\implies\sum_{i=1}^{\infty}f\left(b_{i}\right)-f\left(a_{i}\right)<\epsilon.
\]
 Note that by Lemma \ref{lem:openSetDisjointUnionInterval}, each
open set $O\subset\left[a,b\right]$ can be written as a countable
union of disjoint open intervals and by Proposition \ref{prop:borelApproximateLebesgue},
every Borel set $B\in\borel\left(\left[a,b\right]\right)$ (except
sets that contain the endpoints) is contained in some open set $O\subset\R$
such that 
\[
\lambda\left(O\setminus B\right)=\lambda\left(O\right)-\lambda\left(B\right)<\frac{\delta}{2}.
\]
Choosing $B$ such that $\lambda\left(B\right)<\frac{\delta}{2}$,
we can assume without loss of generality that $O\subset\left[a,b\right]$
and so 
\[
\mu_{F}\left(B\right)\leq\mu_{F}\left(O\right)<\epsilon
\]
where the inequality is due to monotonicity of measures. This completes
the proof by yet another application of Proposition \ref{prop:epsdeltaAbsContinuity}.
\end{proof}
\begin{cor}
\label{cor:stieljesAbsContRepresentation}A Stieljes function $F:\left[a,b\right]\to\R$
is absolutely continuous if and only if there exists some unique $f\in\mathcal{M}^{+}\left(\left[a,b\right],\borel\left(\left[a,b\right]\right)\right)$
such that for any $x\in\left[a,b\right]$
\[
F\left(x\right)=F\left(a\right)+\lebInt{\lambda}{f\indicate_{\left[a,x\right]}}.
\]
\end{cor}

\begin{proof}
First suppose that $F$ is absolutely continuous as a function. Then
by Proposition \ref{prop:absoluteContinuityStieljesFunctions}, the
measure $\mu_{F}$ extended from $\mu_{F}\left(\left(a,x\right]\right):=F\left(x\right)-F\left(a\right)$
is absolutely continuous with respect to the Lebesgue measure on $\left[a,b\right].$
By the Radon-Nikodym theorem, there exists some $f\in\mathcal{M}^{+}\left(\left[a,b\right],\borel\left(\left[a,b\right]\right)\right)$
such that
\[
\mu_{F}\left(A\right)=\lebInt{\lambda}{f\indicate_{A}}
\]
for any $A\in\borel\left(\left[a,b\right]\right).$ In particular
this works for $A=\left[a,x\right]$ where $x\in\left[a,b\right]$
is a arbitrary. Conversely, if $F$ has this integral representation
then $\mu_{F}$ and $\gamma\left(A\right):=\lebInt{\lambda}{f\indicate_{A}}$
agree on all sets of the form $\left[a,x\right]$ which means they
agree on intersections of such sets which is the collection of all
closed intervals in $\left[a,b\right].$ This collection is a $\pi-$system
that generates $\borel\left(\left[a,b\right]\right)$ and so by Theorem
\ref{thm:uniquenessMeasures} $\mu_{F}=\gamma$ on all Borel sets
which implies absolute continuity of the measure $\mu_{F}$ and thus
of the function $F$ by the previous Proposition.

Uniqueness of $f$ follows by the standard argument we used to prove
that Radon Nikodym derivatives are unique.
\end{proof}
\begin{rem*}
Note that the function $f$ here actually turns out to be the (almost
everywhere) derivative of $F$, a fact that requires considerable
effort to prove. We leave the proof of this result (and thus the fundamental
theorems of calculus) to the chapter on product spaces, where we would
be able to show some remarkable results related to the geometry of
the Euclidean space $\R^{n}$, and use those results to complete the
proof started here.
\end{rem*}
Next we extend this representation result to absolutely continuous
functions that are not necessarily non-decreasing. As usual, we do
this by decomposing an arbitrary absolutely continuous function as
a difference of absolutely continuous Stieljes functions and apply
the above result separately to each component.
\begin{lem}
\label{lem:decomposeAbsolutelyContinuous}Let $f:\left[a,b\right]\to\R$
be absolutely continuous. Then there exist non-negative, non-decreasing,
and absolutely continuous functions $f_{1},f_{2}:\left[a,b\right]\to\R$
such that 
\[
f\left(x\right)-f\left(a\right)=f_{1}\left(x\right)-f_{2}\left(x\right)
\]
for all $x\in\left[a,b\right]$.
\end{lem}

\begin{proof}
Define 
\begin{align*}
f_{1}\left(x\right) & :=\sup_{\pi\left[a,x\right]}\sum_{i=1}^{k\left(\pi\right)}\left(f\left(t_{i}\right)-f\left(t_{i-1}\right)\right)^{+}\\
f_{2}\left(x\right) & :=\sup_{\pi\left[a,x\right]}\sum_{i=1}^{k\left(\pi\right)}\left(f\left(t_{i}\right)-f\left(t_{i-1}\right)\right)^{-}
\end{align*}
where $\pi\left[a,x\right]:\left\{ t_{i}\mid t_{0}=a,t_{k\left(\pi\right)}=x,t_{i}<t_{i+1}\right\} $
is a \hyperref[def:partitionInterval]{partition} of $\left[a,x\right].$
Note that the non-negativity of these functions follows by definition.
Next, notice that for $x_{2}\geq x_{1}$ , $f_{j}\left(x_{2}\right)\geq f_{j}\left(x_{1}\right)$
for $j\in\left\{ 1,2\right\} .$To see this, notice that any partition
$\pi$ of $\left[a,x_{1}\right]$ can be extended to a partition $\pi^{\prime}$
of $\left[a,x_{2}\right]$ by adding $t_{k\left(\pi^{\prime}\right)}=t_{k\left(\pi\right)+1}=x_{2}.$
Then,
\[
\sum_{i=1}^{k\left(\pi\right)}\left(f\left(t_{i}\right)-f\left(t_{i-1}\right)\right)^{\pm}\leq\sum_{i=0}^{k\left(\pi\right)+1}\left(f\left(t_{i}\right)-f\left(t_{i-1}\right)\right)^{\pm}
\]
since the summands are non-negative. Next, observe that for any partition
$\pi$ of $\left[a,x\right]\subset\left[a,b\right]$, we have that
\begin{align*}
f\left(x\right)-f\left(a\right) & =\sum_{i=1}^{k\left(\pi\right)}\left(f\left(t_{i}\right)-f\left(t_{i-1}\right)\right)\\
 & =\sum_{i=1}^{k\left(\pi\right)}\left(f\left(t_{i}\right)-f\left(t_{i-1}\right)\right)^{+}-\sum_{i=1}^{k\left(\pi\right)}\left(f\left(t_{i}\right)-f\left(t_{i-1}\right)\right)^{-}
\end{align*}
and so taking supremums we get the equality $f\left(x\right)-f\left(a\right)=f_{1}\left(x\right)-f_{2}\left(x\right).$
Next, we shall show absolute continuity for $f_{1}$; the proof for
$f_{2}$ is identical. Fix $\epsilon>0$ and note that by the absolute
continuity of $f$ we have that there exists some $\delta>0$ such
that for any disjoint collection $\left\{ \left(a_{i},b_{i}\right)\right\} _{i=1}^{n}$$\subset\left[a,b\right]$,
$\sum_{i=1}^{n}b_{i}-a_{i}<\delta$ implies $\sum_{i=1}^{n}\lvert f\left(b_{i}\right)-f\left(a_{i}\right)\rvert<\epsilon$.
Note that 
\begin{align*}
\sum_{i=1}^{n}\lvert f_{1}\left(b_{i}\right)-f_{1}\left(a_{i}\right)\rvert & =\sum_{i=1}^{n}\lvert\sup_{\pi\left[a,b_{i}\right]}\sum_{j=1}^{k\left(\pi\right)}\left(f\left(t_{j}^{\pi}\right)-f\left(t_{j-1}^{\pi}\right)\right)^{+}-\sup_{\pi^{\prime}\left[a,a_{i}\right]}\sum_{j=1}^{k\left(\pi^{\prime}\right)}\left(f\left(t_{j}^{\pi^{\prime}}\right)-f\left(t_{j-1}^{\pi^{\prime}}\right)\right)^{+}\rvert\\
 & =\sum_{i=1}^{n}\sup_{\pi\left[a_{i},b_{i}\right]}\sum_{j=1}^{k\left(\pi\right)}\left(f\left(t_{j}^{\pi}\right)-f\left(t_{j-1}^{\pi}\right)\right)^{+}\\
 & \leq\sum_{i=1}^{n}\sup_{\pi\left[a_{i},b_{i}\right]}\sum_{j=1}^{k\left(\pi\right)}\left(f\left(t_{j}^{\pi}\right)-f\left(t_{j-1}^{\pi}\right)\right)^{+}+\left(f\left(t_{j}^{\pi}\right)-f\left(t_{j-1}^{\pi}\right)\right)^{-}\\
 & =\sum_{i=1}^{n}\sup_{\pi\left[a_{i},b_{i}\right]}\sum_{j=1}^{k\left(\pi\right)}\lvert f\left(t_{j}^{\pi}\right)-f\left(t_{j-1}^{\pi}\right)\rvert\\
 & \leq\epsilon
\end{align*}
where in the second equality we have used the fact that we can enlarge
any partition $\pi$on $\left[a,b_{i}\right]$ to contain any given
partition $\pi^{\prime}$ on $\left[a,a_{i}\right]$ since $\left[a,a_{i}\right]\subset\left[a,b_{i}\right].$
The final inequality then follows since for any partition $\pi_{i}$
of $\left[a_{i},b_{i}\right]$, entire collection $\left\{ \left(t_{j-1}^{\pi_{i}},t_{j}^{\pi_{i}}\right)\right\} _{1\leq j\leq k\left(\pi_{i}\right),1\leq i\leq n}$
is a disjoint collection with total length
\[
\sum_{i=1}^{n}\sum_{j=1}^{k\left(\pi_{i}\right)}t_{j}^{\pi_{i}}-t_{j-1}^{\pi_{i}}=\sum_{i=1}^{n}b_{i}-a_{i}<\delta
\]
and so by absolute continuity
\[
\sum_{i=1}^{n}\sum_{j=1}^{k\left(\pi_{i}\right)}\lvert f\left(t_{j}^{\pi}\right)-f\left(t_{j-1}^{\pi}\right)\rvert\leq\epsilon.
\]
Taking supremums over partitions preserves the inequality.
\end{proof}
\begin{thm}
\label{thm:integralRepresentationAbsoluteContinuity}Let $f:\left[a,b\right]\to\R$
be a measurable function. Then $f$ is absolutely continuous if and
only if there exists an (almost-everywhere) unique function $h\in\Lp 1{\left[a,b\right],\borel\left[a,b\right],\lambda}$
such that for each $x\in [a,b]$,
\[
f\left(x\right)=f\left(a\right)+\lambda\left(h\indicate_{\left[a,x\right]}\right).
\]
\end{thm}

\begin{proof}
First suppose that $f$ is absolutely continuous and so by Lemma \ref{lem:decomposeAbsolutelyContinuous}
we can write 
\[
f\left(x\right)-f\left(a\right)=f_{1}\left(x\right)-f_{2}\left(x\right)
\]
where $f_{i}$ are non-negative, non-decreasing, and absolutely continuous
(and thus Stieljes functions). By Corollary \ref{cor:stieljesAbsContRepresentation}
there exist functions $h_{1},h_{2}\in\mathcal{M}^{+}\left(\left[a,b\right],\borel\left(\left[a,b\right]\right)\right)$
such that
\[
f_{i}\left(x\right)=f_{i}\left(a\right)+\lambda\left(h_{i}\indicate_{\left[a,x\right]}\right)
\]
and so by the linearity of integration we have that 
\[
f\left(x\right)-f\left(a\right)=\underbrace{f_{1}\left(a\right)-f_{2}\left(a\right)}_{=0}+\lambda\left(\left(h_{1}-h_{2}\right)\indicate_{\left[a,x\right]}\right).
\]
Note that $h:=h_{1}-h_{2}$ is integrable since $f\left(b\right)-f\left(a\right)<\infty$.
Uniqueness is a generating class argument as usual. To spell it out, notice that for  $h,g\in\Lp{1}{\left[a,b\right],\borel\left(\left[a,b\right]\right),\lambda}$, where the representation result holds we have that 
\[
	\mathcal{D} : = \{ B \in \borel \left(\left[a,b\right] \right) \mid \lebInt{\lambda}{h\indicate_{B}} = \lebInt{\lambda}{g\indicate_{B}} \}
\]
is a $\lambda-$system: $\left[a,b\right]$ is clearly in $\mathcal{D}$, and if $B \in \mathcal{D}$ then $B^C \in \mathcal{D}$ because $\indicate_{B^C} = \indicate_{[a,b]} - \indicate_{B}$, the linearity of integration, and the fact that $[a,b] \in \mathcal{D}$. Finally, if $\{B_i\}_{i\in\N} \in \mathcal{D}$ are disjoint then 
\begin{align*}
	\lebInt{\lambda}{h\indicate_{\bigcup_{i\in\N}B_i}} &= \lebInt{\lambda}{\sum_{i=1}^\infty h \indicate_{B_i}} \\
	&= \sum_{i=1}^\infty \lebInt{\lambda}{h\indicate_{B_i}} \\
	&=  \sum_{i=1}^\infty \lebInt{\lambda}{g\indicate_{B_i}} \\
	&= \lebInt{\lambda}{\sum_{i=1}^{\infty}g\indicate_{B_i}} \\
	&= 	\lebInt{\lambda}{g\indicate_{\bigcup_{i\in\N}B_i}}
\end{align*}
where in the second and fourth equalities leverage dominated convergence. Note that $\mathcal{E}$ the collection of all closed interval subsets of $[a,b]$ generates $\borel\left([a,b]\right)$ and the equality of representations holds on $\mathcal{E}$ by definition. By the $\pi-\lambda$ theorem, the equality extends to the entirety of $\borel\left([a,b]\right)$. By Proposition \ref{prop:intEqualFuncEqual} we have that $h \stackrel{\text{a.e}}{=} g $.

Finally, suppose that $f$ has the integral representation
\[
f(x) - f(a) = \lebInt{\lambda}{h\indicate_{[a,x]}}
\] 
and fix $\epsilon > 0$ . Note that by the linearity of integration, for any $\left[a_i, b_i\right] \subseteq \left[a,b\right]$ 
\[
f(b_i) - f(a_i) = \lebInt{\lambda}{h\indicate_{[a_i,b_i]}}
\]
and so for any disjoint collection $\{[a_i,b_i]\}_{i=1}^n$
\begin{align*}
	\sum_{i=1}^{n} \lvert f(b_i)-f(a_i) \rvert &= \sum_{i=1}^n\lvert \lebInt{\lambda}{h\indicate_{[a_i,b_i]}}\rvert \\
	&\leq \sum_{i=1}^n\lebInt{\lambda}{\lvert h\rvert \indicate_{[a_i,b_i]}} 
\end{align*}
 by Corollary \ref{cor:triangleIneqLebIntL1}. Then, using Proposition \ref{prop:epsdeltaAbsContinuity}, we have that there exists some $\delta > 0$ such that for $\lambda\left(\bigcup_{i=1}^n[a_i,b_i]\right) < \delta $ we have $ \lebInt{\lambda}{\lvert h \rvert \indicate_{\bigcup_{i=1}^n[a_i,b_i]}} < \epsilon $ which completes the proof.
\end{proof}
%

\section{Optimization\label{sec:optimization}}
\begin{example}
\label{exa:isi2006samplepsb2}Maximize $x+y$ subject to the condition
that $2x^{2}+3y^{2}\leq1$.
\end{example}





\chapter{Product measures}

In calculus, we learnt that the theory of integration readily extends
from from real valued functions on $\R$ to real valued functions
on the Euclidean space $\R^{n}.$ The extension is usually motivated
geometrically by studying the volume under the surface of a sufficiently
smooth function $f:\R^{2}\to\R$. The idea is that if a function is
sufficiently well behaved, then one can recover the volume under the
surface by looking at the areas under various ``slices'' of the function
and then summing up those areas. Importantly, under the requisite
smoothness conditions, the ``slices'' could have been made horizontally
or vertically, and we would get the same result. This intuition leads
to the Fubini theorem for multiple integration in the Riemann setting:
\[
\int_{a}^{b}\int_{c}^{d}f(x,y)dydx=\int_{c}^{d}\int_{a}^{b}f(x,y)dxdy.
\]
The main goal of this chapter is to recover this result for the Lebesgue
integral. While important in its own right for the study of analysis
on Euclidean spaces, in the context of probability theory, this result
takes a far more important role. In particular, the ability to write
a multiple integral as an iterated integral corresponds directly with
the ability to factor the joint distribution of random variables into
the marginal distribution; that is, it underpins the theory of \emph{independent
}random variables. Independence, and the departures from independence,
constitute the central concepts of probabiliy theory.

\section{Product measures on finite product spaces}

\subsection{Iterated integrals on non-negative measurable function}

Let $\left(\X,\F,\mu\right)$ and $\left(\mathcal{Y},\mathcal{G},\nu\right)$
be measure spaces. Our interest is in defining measurable functions
on the product space $\X\times\mathcal{Y}.$ The principle hurdle
that is immediately apparent here is that the product of the $\sigma-$algebras
$\F\times\mathcal{G}:=\left\{ F\times G\mid F\in\F,G\in\mathcal{G}\right\} $
is not necessarily a $\sigma-$algebra. \hl{ADD CE}. It is easy to
see, however, that it is a $\pi-$system (this fact will turn out
to be important!). Any $\sigma-$algebra we use should certainly \emph{contain
}$\mathcal{F}\times\mathcal{G}$ and so the canonical choice is given
by $\mathcal{F}\otimes\mathcal{G:=\sigma\left(F\times G\right)}$.
A series of natural questions follow if we want iterated integrals
to make sense. In particular, for a measurable map $f\in\mathcal{M}^{+}\left(\X\times\mathcal{Y},\mathcal{F}\otimes\mathcal{G}\right),$we
want to know if the projections $x\to f(x,y)$ and $y\to\mu^{x}\left(f\left(x,y\right)\right)$
are $\mathcal{F}$ and $\mathcal{G}$ measurable, respectively. Thankfully,
this turns out to be the case. We will use a long lost result from
Chapter 2: the \hyperref[thm:piLambdaThmFunctions]{$\pi-\lambda$ theorem for functions}.
Go over the hypotheses of this theorem before reading the following
results.
\begin{lem}
\label{lem:partialFunctionMeasurability}For every $f\in\mathcal{M}^{+}\left(\X\times\mathcal{Y},\mathcal{F}\otimes\mathcal{G}\right)$,
the maps $x\to f(x,y)$ and $y\to f(x,y)$ are $\mathcal{F}/\borel\left(\R\right)$
and $\mathcal{G}/\borel\left(\R\right)$ measurable for every $y\in\mathcal{Y}$
and $x\in\X$, respectively.
\end{lem}

\begin{proof}
First we note, due to the fact that $\left(A_{1}\times B_{1}\right)\cap\left(A_{2}\times B_{2}\right)=\left(A_{1}\cap A_{2}\right)\times\left(B_{1}\times B_{2}\right),$that
$\mathcal{\mathcal{E:=}F}\times\mathcal{G}$ is a $\pi-$system. Next,
we claim that the space
\[
\mathcal{H:=}\left\{ f\in\mathcal{M_{\text{bdd}}}\left(\X\times\mathcal{Y},\mathcal{F}\otimes\mathcal{G}\right)\mid\forall y\in\mathcal{Y}:x\to f\left(x,y\right)\text{is }\mathcal{G}/\borel\left(\R\right)\text{ measurable }\right\} 
\]
is a $\lambda-$space of functions. Note that $\indicate_{\X\times\mathcal{Y}}$is
constant and bounded (and evidently partially measurable in our sense).
Further, our space $\mathcal{H}$ is a vector space, since linear
combinations of bounded measurable functions is bounded, and the partial
measurability condition is also preserved under linear combinations
(both results of Proposition \ref{prop:binaryOperationsMeasurableFunctions}).
Finally, $\mathcal{H}$ is closed under monotone limits (if they exist)
since measurability (resp. partial measurability) is preserved under
limits.

Now, observe that $\left\{ \indicate_{A}\mid A\in\mathcal{E}\right\} \subseteq\mathcal{H}$,
since $\indicate_{F\times G}=\indicate_{F}\indicate_{G}$which is
clearly bounded, measurable, and partially measurable in our sense.
Thus applying the $\pi-\lambda$ theorem as discussed, we have that

\[
\mathcal{M_{\text{bdd}}}\left(\X\times\mathcal{Y},\mathcal{F}\otimes\mathcal{G}\right)\subseteq\mathcal{H}
\]
which completes the proof for bounded functions.

Now we can simply take a function $f\in\mathcal{M}^{+}\left(\X\times\mathcal{Y},\mathcal{F}\otimes\mathcal{G}\right)$
and construct the bounded monotone sequence $f_{n}:=\min\left\{ f,n\right\} \in\mathcal{H}$
by our results. We complete the proof by taking limits and noting
that partial measurability is preserved. The same argument, of course,
holds for $y\to f\left(x,y\right)$
\end{proof}
Virtually the same argument shows that $y\to\mu^{x}\left(f\left(x,y\right)\right)$
is measurable (although we need this previous lemma to show that this
function is indeed well-defined). We include this result here for
completeness
\begin{lem}
\label{lem:partialIntegralMeasurability}Let $\left(\X,\F,\mu\right)$
and $\left(\mathcal{Y},\mathcal{G},\nu\right)$ be measure spaces.
For every function $f\in\mathcal{M}^{+}\left(\X\times\mathcal{Y},\mathcal{F}\otimes\mathcal{G}\right)$,
the maps $x\to\nu^{y}\left(f\left(x,y\right)\right)$ and $y\to\mu^{x}\left(f\left(x,y\right)\right)$
are $\mathcal{F}/\borel\left(\R\right)$ and $\mathcal{G}/\borel\left(\R\right)$
measurable respectively.
\end{lem}

\begin{proof}
Let $\mathcal{E}$ be as before and this time let
\[
\mathcal{H:=}\left\{ f\in\mathcal{M_{\text{bdd}}}\left(\X\times\mathcal{Y},\mathcal{F}\otimes\mathcal{G}\right)\mid y\to\mu^{x}f\left(x,y\right)\text{is }\mathcal{G}/\borel\left(\R\right)\text{ measurable }\right\} ,
\]
noting that $\indicate_{\X\times\mathcal{Y}}\in\mathcal{H}$ since
$\mu^{x}\left(\indicate_{\X\times\mathcal{Y}}\right)=\indicate_{\mathcal{Y}}\mu\left(\X\right)$
which is $\mathcal{G}$ measurable. $\mathcal{H}$ is a vector space
closed under monotone limits with the same arguments as before along
with the linearity and monotone convergence properties of $\mu$.
Applying the $\pi-\lambda$ theorem for functions, we have that 
\[
\mathcal{M_{\text{bdd}}}\left(\X\times\mathcal{Y},\mathcal{F}\otimes\mathcal{G}\right)\subseteq\mathcal{H}.
\]
Finally, take some $f\in\mathcal{M}^{+}\left(\X\times\mathcal{Y},\mathcal{F}\otimes\mathcal{G}\right)$
and construct $f_{n}:=\min\left\{ f,n\right\} \in\mathcal{H}$ and
note that $y\to\mu^{x}\left(f_{n}\left(x,y\right)\right)$ is measurable
and by monotone convergence $y\to\mu^{x}\left(f\left(x,y\right)\right)$
is measurable completing the proof.
\end{proof}
%
With the technicalities out of the way, we can show that iterated
integrals give the same result under certain conditions. This is a
generalization of the result we saw all the way back in Lemma \ref{lem:TonelliForSeries}.
\begin{thm}[Tonelli]
\label{thm:tonelli}Let $\left(\X,\F,\mu\right)$ and $\left(\mathcal{Y},\mathcal{G},\nu\right)$
be measure spaces. Then, for any $f\in\mathcal{M}^{+}\left(\X\times\mathcal{Y},\mathcal{F}\otimes\mathcal{G}\right)$,
the functions 
\[
\gamma_{1}\left(f\right):=\nu^{y}\mu^{x}\left(f\right)
\]
and
\[
\gamma_{2}\left(f\right):\mu^{x}\nu^{y}\left(f\right)
\]
are integrals on $\mathcal{M}^{+}\left(\X\times\mathcal{Y},\mathcal{F}\otimes\mathcal{G}\right).$
Moreover, if $\mu$and $\nu$ are $\sigma-$finite, $\gamma_{1}\left(f\right)=\gamma_{2}\left(f\right)$
for every $f\in\mathcal{M}^{+}\left(\X\times\mathcal{Y},\mathcal{F}\otimes\mathcal{G}\right).$
\end{thm}

\begin{proof}
We show that $\gamma_{1}$ is an integral; the argument for $\gamma_{2}$
is the analagous. First, observe that $\gamma_{1}\left(0\right)=\nu^{y}\mu^{x}\left(0\right)=\nu^{y}\left(0\right)=0$,
given that $\nu$ and $\mu$ are integrals. Second, note that for
$\alpha,\beta\geq0$, and $f,g\in\mathcal{M}^{+}\left(\X\times\mathcal{Y},\mathcal{F}\otimes\mathcal{G}\right)$
\begin{align*}
\gamma_{1}\left(\alpha f+\beta g\right) & =\nu^{y}\mu^{x}\left(\alpha f+\beta g\right)\\
 & =\nu^{y}\left(\alpha\mu^{x}f+\beta\mu^{x}g\right)\\
 & =\alpha\nu^{y}\mu^{x}\left(f\right)+\beta\nu^{y}\mu^{x}\left(g\right)\\
 & =\alpha\gamma_{1}\left(f\right)+\beta\gamma_{1}\left(g\right).
\end{align*}
Finally, observe that for $f_{n}\in\mathcal{M}^{+}\left(\X\times\mathcal{Y},\mathcal{F}\otimes\mathcal{G}\right)$
such that $f_{n}\leq f_{n+1}$ and $f_{n}\to f$, we have
\begin{align*}
\lim_{n\to\infty}\gamma_{1}\left(f_{n}\right) & =\lim_{n\to\infty}\nu^{y}\mu^{x}\left(f_{n}\right)\\
 & =\nu^{y}\mu^{x}\left(\lim_{n\to\infty}f_{n}\right)\\
 & =\gamma_{1}\left(f\right)
\end{align*}
by applying monotone convergence twice.

Therefore, by Theorem \ref{thm:integralMeasureEquivalence}, $\gamma_{1}$
and $\gamma_{2}$ are integrals on $\mathcal{M}^{+}\left(\X\times\mathcal{Y},\mathcal{F}\otimes\mathcal{G}\right)$
with respect to measures defined by integrating indicator functions
in $\mathcal{F\otimes\mathcal{G}}.$ If $\mu$ and $\nu$ are $\sigma-$finite,
the two integrals can be shown to be equal by showing that the corresponding
measures are equal on a generating $\pi-$system that can approximate
the full space (courtesy of our \hyperref[thm:uniquenessMeasures]{uniqueness theorem}).
Of course, since $\mathcal{F\times\mathcal{G}}$ is a $\pi-$system,
for $F\in\F,G\in\mathcal{G}$
\begin{align*}
\gamma_{1}\left(\indicate_{F\times G}\right) & =\nu^{y}\mu^{x}\left(\indicate_{F\times G}\left(x,y\right)\right)\\
 & =\nu^{y}\mu^{x}\left(\indicate_{F}\left(x\right)\indicate_{G}\left(y\right)\right)\\
 & =\nu\left(G\right)\mu\left(F\right)\\
 & =\mu^{x}\nu^{y}\left(\indicate_{F}\left(x\right)\indicate_{G}\left(y\right)\right)\\
 & =\gamma_{2}\left(\indicate_{F\times G}\right)
\end{align*}
completing the proof. Since our measures $\mu,\nu$ are $\sigma-$finite,
we know that there exist sets $E_{i}\in\mathcal{F\times\mathcal{G}}$
such that $\bigcup E_{i}=\X\times\mathcal{Y}.$ This completes the
proof.
\end{proof}
Note that the $\sigma-$finiteness condition is actually necessary
for the uniqueness of the integrals, as the following example illustrates.
\begin{example}
\label{exa:tonelliFailNonSigmaFinite}Let $\left(\left[0,1\right],\borel\left(\left[0,1\right]\right),\lambda\right)$
and $\left(\left[0,1\right],2^{\left[0,1\right]},\mu_{0}\right)$
where $\lambda$ and $\mu_{0}$ are the Lebesgue and counting measures,
respectively. Now consider the product space $ $
\end{example}


\subsection{Product measures}
\begin{defn}
\label{def:productMeasure}Let $\measurespace$ and $\left(\mathcal{Y},\mathcal{G},\nu\right)$
be measure spaces and let $\left(\X\times\mathcal{Y},\F\otimes\mathcal{G}\right)$
be the product measurable space. Then a measure $\mu\otimes\nu:\F\otimes\mathcal{G}\to\left[0,\infty\right]$
is called a \emph{product measure }if
\[
\mu\otimes\nu\left(F\times G\right)=\mu\left(F\right)\nu\left(G\right)
\]
for all $F\in\F,G\in\mathcal{G}.$
\end{defn}

Of course, the way we have set this up, Theorem \ref{thm:tonelli}(Tonelli)
guarantees existence of this measure, since measures and integrals
are equivalent. Uniqueness, when $\mu$ and $\nu$ are $\sigma-$finite,
also follows from the uniqueness theorem, as outlined in the proof
of Tonelli.

\hl{Prove that $\sigma(\left(f,g\right)) = \sigma(f,g)$ }

\subsection{Convolutions}

\hl{SEE SCHiLLING THEOREM 13.10 FOR ALTERNATIVE CONSTRUCTION OF PRODUCT SIGMA ALGEBRAS}

\section{The Lebesgue measure on $\protect\R^{n}$}

\hl{FUBINI IMPLIES YOUNGS THEOREM ON SYMMETRY}

\section{Kernels}

\hl{Refer to Tatikonda notes}

\section{Extension to infinite product spaces}

\hl{ASH or KLENKE?}

\section{Disintegration}



\chapter{Fourier transforms, Laplace transforms, and generating functions}




\part{Probability}


\chapter{Independence}

\section{{*}Foundations}

\hl{Use Tao blog}

\section{Exchangeability}

\hl{Tatikonda notes}



\chapter{Conditioning\label{chap:conditioning}}



\chapter{Martingales}

\section{Stopping times}

\section{Optional stopping}

\section{Martingale convergence theorems}

\section{Uniformly integrable martingales}

\section{Backwards martingales and exchangeability}



\chapter{Ergodic Theory and Laws of Large Numbers}


\include{central_limit_theorems}

\part{Optimization}


\chapter{Basic convex analysis and optimization}


\part{Statistics}

\part{Topics in Economics}

\appendix

\part{Appendices}


\chapter{Naive set theory}

\section{Construction of basic number systems}

\section{Basic order theory}

\section{Cardinality}

\section{Equivalent forms of choice}



\chapter{\label{chap:finiteVectorSpace}Finite Dimensional Vector Spaces}
\begin{defn}
\label{def:vectorSpace}A \emph{vector space }is a non-empty set \emph{$V$
}over a field $\mathbb{K}$ together with two binary operations $+:V\times V\to V$
and $\cdot:\mathbb{K}\times V\to V$ , called \emph{vector addition
}and \emph{scalar multiplication, }respectively, that satisfy
\end{defn}

\begin{enumerate}
\item Closure under vector addition and scalar multiplication: For any $\alpha\in K$
and any $u,v\in V$: $\alpha u\in V$ and $u+v\in V.$
\item Associativity of vector addition: For any $u,v,w\in V:u+\left(v+w\right)=\left(v+u\right)+w$
\item Commutativity of vector addition: For any $u,v\in V:u+v=v+u$
\item Identity element of vector addition: There exists some element $\mathbf{0}\in V$,
called the \emph{zero vector }such that for any $v\in V:\mathbf{0}+v=v.$
\item Inverse element of vector addition: For any $v\in V$, there exists
a vector $-v\in V$ such that $v+\left(-v\right)=\mathbf{0}$
\item Compatibility of scalar multiplication with field multiplication:
For any $\alpha,\beta\in\mathbb{K}$ and $u\in V:\left(\alpha\beta\right)\cdot u=\alpha\cdot\left(\beta\cdot u\right).$
\item Identity element of scalar multiplication: There exists some element
$1\in\mathbb{K}$ such that for any $u\in V:1v=v.$
\item Distributivity of scalar multiplication with respect to vector addition:
For any $\alpha\in\mathbb{K}$ and any $u,v\in V:\alpha\cdot\left(u+v\right)=\alpha\cdot u+\alpha\cdot v$
\item Distributivity of scalar multiplication with respect to field addition:
For all $\alpha,\beta\in\mathbb{K}$ and $u\in V:\left(\alpha+\beta\right)\cdot u=\alpha\cdot u+\beta\cdot u.$
\end{enumerate}
\begin{xca}[ISI 2013 PSB-1]
\label{exer:isi2013psb1}Let $E=\{1,2,\ldots,n\}$, where \$n\$ is
an odd positive integer. Let $V$ be the vector space of all functions
from $E$ to $\mathbb{R}^{3}$, where the vector space operations
are given by 
\[
\begin{aligned}(f+g)(k) & =f(k)+g(k),\quad\text{ for }f,g\in V,k\in E,\\
(\lambda f)(k) & =\lambda f(k),\quad\text{ for }f\in V,\lambda\in\mathbb{R},k\in E.
\end{aligned}
\]
\end{xca}

\begin{enumerate}
\item Find the dimension of V
\item Let $T:V\to V$ be the map given by 
\[
T\left(f\right)\left(k\right):=\frac{1}{2}\left(f\left(k\right)+f\left(n+1-k\right)\right),\ k\in E
\]
is linear.
\item Find the null space of T.
\end{enumerate}

\section{Determinants\label{sec:Determinants}}

We shall construct the determinant of linear map $T:\R^{n}\to\R^{n}$axiomatically.
Determinants are typically defined by a given matrix representation
$M_{T}=\left[\begin{array}{cccc}
| & | &  & |\\
v_{1} & v_{2} & \ldots & v_{n}\\
| & | &  & |
\end{array}\right]$ of $T$ with respect to the standard basis on $\R^{n}$. Here $v_{i}$
are vectors in $\R^{n}$ with respect to the standard basis. As we
shall later see, the choice of basis does not matter and so we can
think of the determinant as acting on the linear map itself rather
the matrix representation.

\subsection{Desiderata of the determinant}

What properties should our determinant function have? Why do we even
need such a function? What does it actually represent? Alas, the determinant
is one of those concepts whose usefulness only becomes apparent \emph{after
}you are done constructing it and relaying its properties. In fact,
the geometric interpretation of the determinant is often not even
covered in a course on Linear Algebra; we shall in fact use the $n-$dimensional
Lebesgue measure in Section \ref{subsec:lebesgueRNproperties}to understand
the geometry of determinants in the main text rather than this appendix.

Thus. without much motivation (for now), we set out on a goal to construct
a map $\det:M_{n\times n}\to\R$ (where $M_{n\times n}$ is the vector
space of all $n\times n$ real valued matrices) that satisfies the
following properties:
\begin{enumerate}
\item \textbf{Linearity in each argument: }Let $\left\{ v_{i}\right\} _{i=1}^{n},u\in\R^{n}$\textbf{
}and let $\alpha\in\R$. Then\textbf{
\[
\det\left(\left[\begin{array}{ccccc}
| &  & | &  & |\\
v_{1} & \ldots & v_{k}+\alpha u & \ldots & v_{n}\\
| &  & | &  & |
\end{array}\right]\right)=\det\left(\left[\begin{array}{ccccc}
| &  & | &  & |\\
v_{1} & \ldots & v_{k} & \ldots & v_{n}\\
| &  & | &  & |
\end{array}\right]\right)+\alpha\det\left(\left[\begin{array}{ccccc}
| &  & | &  & |\\
v_{1} & \ldots & u & \ldots & v_{n}\\
| &  & | &  & |
\end{array}\right]\right).
\]
}
\item \textbf{Preservation under column replacement:} Let $\left\{ v_{i}\right\} _{i=1}^{n}\in\R^{n}$\textbf{
}and let $\alpha\in\R$. Then
\[
\det\left(\left[\begin{array}{ccccccc}
| &  & | &  & | &  & |\\
v_{1} & \ldots & v_{l}+\alpha v_{k} & \ldots & v_{k} & \ldots & v_{n}\\
| &  & | &  & | &  & |
\end{array}\right]\right)=\det\left(\left[\begin{array}{ccccccc}
| &  & | &  & | &  & |\\
v_{1} & \ldots & v_{l} & \ldots & v_{k} & \ldots & v_{n}\\
| &  & | &  & | &  & |
\end{array}\right]\right).
\]
\item \textbf{Antisymmetry: }Let $\left\{ v_{i}\right\} _{i=1}^{n}\in\R^{n}$.
Then
\[
\det\left(\left[\begin{array}{ccccccc}
| &  & | &  & | &  & |\\
v_{1} & \ldots & v_{l} & \ldots & v_{k} & \ldots & v_{n}\\
| &  & | &  & | &  & |
\end{array}\right]\right)=-\det\left(\left[\begin{array}{ccccccc}
| &  & | &  & | &  & |\\
v_{1} & \ldots & v_{k} & \ldots & v_{l} & \ldots & v_{n}\\
| &  & | &  & | &  & |
\end{array}\right]\right).
\]
\item \textbf{Normalization: $\det\left(I_{n}\right)=1$ }where $I_{n}$
is the $n\times n$ identity matrix.
\end{enumerate}
It turns out that these properties completely characterize the determinant,
in that there exists a unique function that possesses these properties.
We shall postpone the proof of existence and uniqueness to the end
of this section and derive the properties of the determinant function.

\subsection{Properties of the determinant}

All the properties of the determinant can be recovered using the axioms
above, although some results are harder to prove than others. We start
with a few simple results.
\begin{prop}
\label{prop:IzeroColumnDetZero}If $A\in M_{n\times n}$ has a zero
column, then $\det\left(A\right)=0$.
\end{prop}

\begin{proof}
This fact follows from linearity since a zero column can be written
as $0.v$ for any $v\in\R^{n}$.
\end{proof}
\begin{prop}
\label{prop:linearlyDependentColumnsDetZero}If $A\in M_{n\times n}$
has linearly dependent columns then $\det\left(A\right)=0$.
\end{prop}

\begin{proof}
Let 
\[
A=\left[\begin{array}{ccccccc}
| &  & | &  & | &  & |\\
v_{1} & \ldots & v_{l} & \ldots & v_{k} & \ldots & v_{n}\\
| &  & | &  & | &  & |
\end{array}\right]
\]
and notice that if the columns of $A$ are linearly dependent then
there exist constants $\left\{ \alpha_{i}\right\} _{i=1}^{n}\in\R$,,
such that $\sum_{i=1i\neq l}^{n}\alpha_{i}v_{i}=v_{l}.$ Now, we can
write
\begin{align*}
\det\left(A\right) & =\det\left(\left[\begin{array}{ccccccc}
| &  & | &  & | &  & |\\
v_{1} & \ldots & \sum_{i=1i\neq l}^{n}\alpha_{i}v_{i} & \ldots & v_{k} & \ldots & v_{n}\\
| &  & | &  & | &  & |
\end{array}\right]\right)\\
 & =\sum_{i\neq l}\alpha_{i}\det\left(\left[\begin{array}{ccccccc}
| &  & | &  & | &  & |\\
v_{1} & \ldots & v_{i} & \ldots & v_{k} & \ldots & v_{n}\\
| &  & | &  & | &  & |
\end{array}\right]\right)\\
 & =0
\end{align*}
where in the second equality we have used linearity and in the last
equality we have used anti-symmetry and the fact that if $\det\left(B\right)=-\det\left(B\right)$
then $\det\left(B\right)=0.$
\end{proof}
Now we have sufficiently many properties to describe the determinants
of the most elementary types of matrices. Recall that the elementary
linear maps $T:\R^{n}\to\R^{n}$ are
\begin{enumerate}
\item \textbf{Row scaling: $T\left(x_{1,}\ldots,x_{k},\ldots,x_{n}\right)^{T}=\left(x_{1},\ldots cx_{k},\ldots x_{n}\right)^{T}$
}for some $c\in\R$. The matrix with respect to the standard basis
of this transformation is $\diag\left(1,1,\ldots,c,\ldots,1\right).$
By linearity and the normalization property of determinants, $\det\left(T\right)=c.$
More generally, for any diagonal matrix $D=\diag\left(a_{1},a_{2},\ldots,d_{n}\right)$
we have that $\det\left(D\right)=\prod_{i=1}^{n}a_{i}$ by linearity
and normalization. $T$ is invertible with $T^{-1}$ being represented
by $\diag\left(1,1,\ldots,\frac{1}{c},\ldots,1\right)$. The tranpose
of $T^{T}=T$
\item \textbf{Row switching: $T\left(x_{1},\ldots x_{l},\ldots x_{k},\ldots,x_{n}\right)^{T}=\left(x_{1},\ldots,x_{k},\ldots,x_{l},\ldots,x_{n}\right)^{T}$
}for any $1\leq l\leq k\leq n.$ The matrix of this operator wrt to
the standard basis is $\left[e_{1},\ldots e_{k},\ldots,e_{l},\ldots,e_{n}\right]$
where $e_{i}$ is the $i$th standard basis element. Clearly, $\det\left(T\right)=-1$
by antisymmetry. Again, $T$ is invertible and is its own inverse.
The tranpose of $T$ is $T$ itself.
\item \textbf{Row replacement$T\left(x_{1},\ldots x_{l},\ldots x_{k},\ldots,x_{n}\right)^{T}=\left(x_{1},\ldots,x_{l}+cx_{k},\ldots,x_{k},\ldots,x_{n}\right)^{T}$}for
any $1\leq l<l\leq n$ and any $c\in\R$. The matrix representation
of this map is given by $\left[e_{1},\ldots,e_{l},\ldots,e_{k}+ce_{l},\ldots,e_{n}\right]$
with determinant $\det\left(T\right)=1$ by linearity and Proposition
\ref{prop:linearlyDependentColumnsDetZero}. This function is also
clearly invertible and its inverse is an operations of the same type.
The tranpose is also an operation of the same type.
\end{enumerate}
\begin{prop}
\label{prop:detMatrixTimesElementary}Let $A\in M_{n\times n}$ and
let $T\in M_{n\times n}$ be the standard basis representation of
an elementary row operation as above. Then
\[
\det\left(AT\right)=\det\left(A\right)\det\left(T\right).
\]
\end{prop}

\begin{proof}
First consider the case when $T$ is the scaling operator for the
$k$th coordinate. Then, letting $A=\left[\begin{array}{ccccccc}
| &  & | &  & | &  & |\\
v_{1} & \ldots & v_{l} & \ldots & v_{k} & \ldots & v_{n}\\
| &  & | &  & | &  & |
\end{array}\right]$, we have $ST=\left[\begin{array}{ccccccc}
| &  & | &  & | &  & |\\
v_{1} & \ldots & v_{l} & \ldots & cv_{k} & \ldots & v_{n}\\
| &  & | &  & | &  & |
\end{array}\right]$ and so 
\[
\det\left(AT\right)=c\det\left(A\right)=\det\left(A\right)\det\left(T\right).
\]
Next, let $T$ is the row switching operator. Then $AT=\left[\begin{array}{ccccccc}
| &  & | &  & | &  & |\\
v_{1} & \ldots & v_{k} & \ldots & v_{l} & \ldots & v_{n}\\
| &  & | &  & | &  & |
\end{array}\right]$ and by antisymmetry
\[
\det\left(AT\right)=-\det\left(A\right)=\det\left(A\right)\det\left(T\right).
\]
Finally, if $T$ is the row replacement operator, then $AT=\left[\begin{array}{ccccccc}
| &  & | &  & | &  & |\\
v_{1} & \ldots & v_{l} & \ldots & v_{k}+cv_{l} & \ldots & v_{n}\\
| &  & | &  & | &  & |
\end{array}\right]$ and 
\[
\det\left(AT\right)=\det\left(A\right)+c\det\left(\left[\begin{array}{ccccccc}
| &  & | &  & | &  & |\\
v_{1} & \ldots & v_{l} & \ldots & v_{l} & \ldots & v_{n}\\
| &  & | &  & | &  & |
\end{array}\right]\right)=\det\left(A\right)
\]
 by linear dependence.
\end{proof}
Note that by our work on simultaneous linear equations and their solutions,
every invertible matrix $A\in M_{n\times n}$ can be reduced to the
identity matrix by multiplication with elementary matrices from the
left i.e. $I=T_{1}T_{2}..T_{k}A.$ Then,
\[
A=T_{k}^{-1}T_{k-1}^{-1}\ldots T_{1}^{-1}
\]
where each $T_{i}^{-1}$ is an elementary operation and so $\det\left(A\right)=\prod_{k=1}^{n}\det\left(T_{i}^{-1}\right)$
by inducting on Proposition \ref{prop:detMatrixTimesElementary}.
\begin{prop}
\label{prop:detProduct}Let $A,B\in M_{n\times n}$. Then, 
\[
\det\left(AB\right)=\det\left(BA\right)=\det\left(A\right)\det\left(B\right).
\]
\end{prop}

\begin{proof}
First suppose that $A,B$ are both invertible. Then $A$ and $B$
are products of elementary transformations and so $\det\left(AB\right)=\det\left(A\right)\det\left(B\right).$
If either, $A$ or $B$ is not invertible, then neither $AB$ nor
$BA$ are invertible since neither operator is either injective or
surjective (by rank-nullity) but have the same domains. Suppose without
loss of generality that $B$ is not invertible. Then by the rank nullity
theorem, $\ker\left(B\right)$ is trivial, which is equivalent to
the fact $B$ has linearly dependent columns and so $\det\left(B\right)=0$.
The same argument applied to $AB$ or $BA$ shows that $\det\left(AB\right)=\det\left(BA\right)=0.$
This completes the proof.
\end{proof}
\begin{cor}
\label{cor:invertibleDetZero}A matrix $A\in M_{n\times n}$ is invertible
if and only if $\det\left(A\right)\neq0.$
\end{cor}

\begin{proof}
We have already shown that if $A$ is not invertible, then $A$ has
linearly dependent columns and so $\det\left(A\right)=0.$ Conversely,
suppose that $\det\left(A\right)=0$ and $A$ is invertible. Then
$1=\det\left(I\right)=\det\left(AA^{-1}\right)=\det\left(A\right)\det\left(A^{-1}\right)=0$
which is a contradiction.
\end{proof}
\begin{cor}
\label{cor:detOfInverse}Let $A\in M_{n\times n}$ be invertible.
Then
\[
\det\left(A^{-1}\right)=\frac{1}{\det\left(A\right)}.
\]
\end{cor}

\begin{proof}
Note that $1=\det\left(I\right)=\det\left(AA^{-1}\right)=\det\left(A\right)\det\left(A^{-1}\right).$
\end{proof}
%
\begin{cor}
\label{cor:detBasisInvariance}If $A,B\in M_{n\times n}$ are similar
matrices, then 
\[
\det\left(A\right)=\det\left(B\right).
\]
\end{cor}

\begin{proof}
Let $P\in M_{n\times n}$ be invertible such that 
\[
A=PBP^{-1}.
\]
Then, 
\begin{align*}
\det\left(A\right) & =\det\left(PBP^{-1}\right)\\
 & =\det\left(P\right)\det\left(B\right)\det\left(P^{-1}\right)\\
 & =\det\left(P\right)\det\left(B\right)\frac{1}{\det\left(P\right)}\\
 & =\det\left(B\right).
\end{align*}
\end{proof}
Note that this tells us that the determinant of a matrix doesn't change
under a change of basis. In other words, we can think of the determinant
as acting on the operator that the matrix represents rather than the
matrix itself.
\begin{prop}
\label{prop:detTranspose}For any $A\in M_{n\times n}$
\[
\det\left(A\right)=\det\left(A^{T}\right).
\]
\end{prop}

\begin{proof}
If $A$ is not invertible then $A^{T}$ is also not invertible and
so both determinants are zero. On the other hand, if $A$ is invertible
then $A=T_{1}T_{2}\ldots T_{k}$ for some $k$ where $T_{i}$ are
elementary matrices. Then, $A^{T}=T_{k}^{T}T_{k-1}^{T}\ldots T_{1}^{T}$,
where the determinant of $T_{i}$ is the same as the determinant of
$T_{i}^{T}$. The result then follows by Proposition \ref{prop:detProduct}.
\end{proof}

\subsection{Existence and uniqueness}

\subsection{The cofactor expansion}
\begin{example}[ISI 2023 PSB 1]
\label{exa:isi2023psb1}Let $A_{n}=\left(\left(a_{ij}\right)\right)$
be the $n\times n$ matrix defined by 
\[
a_{ij}=\begin{cases}
0 & \text{ if }|i-j|>1\\
1 & \text{ if }|i-j|=1\\
2 & \text{ if }i=j
\end{cases}
\]
What is the determinant of $A_{n}$ for $n\geq1$?\hl{TODO}
\end{example}

\begin{example}
\label{exa:isi2004psbsample2}Is the following system of equations
always consistent for real $k$ ? Justify your answer. 
\[
\begin{aligned}x+y+kz & =2,\\
3x+4y+2z & =k,\\
2x+3y-z & =1.
\end{aligned}
\]

Find the value of $k$ for which this system admits more than one
solution? Express the general solution for the system of equations
for this value of $k$.\hl{TODO}
\end{example}

\begin{example}
\label{exa:isi2005samplepsb1}Let $A$ be a $n\times n$ upper triangular
matrix such that $AA^{T}=A^{T}A$. Show that $A$ is a diagonal matrix.\hl{TODO}
\end{example}

\begin{example}
\label{exa:isi2005samplepsb3}Let $A$ be a $n\times n$ orthogonal
matrix, where $n$ is even and suppose \$|A|=\$ -1, where $|A|$ denotes
the determinant of $A$. Show that $|I-A|=0$, where $I$ denotes
the $n\times n$ identity matrix.\hl{TODO}
\end{example}

\begin{example}
\label{exa:isi2006samplepsb1}Let $A$ and $B$ be two invertible
$n\times n$ real matrices. Assume that $A+B$ is invertible. Show
that $A^{-1}+B^{-1}$ is also invertible.\hl{TODO}
\end{example}

\begin{example}
\label{exa:isi2007samplepsb1}Let $A$ be a $2\times2$ matrix with
real entries such that$A^{2}=0$. Find the determinant of $I+A$ where
$I$ denotes the identity matrix.\hl{TODO}
\end{example}

\begin{example}
\label{exa:isi2007samplepsb2}Let $A$ and $B$ be $n\times n$ real
matrices such that $A^{2}=A$ and $B^{2}=B$. Suppose that $I-(A+B)$
is invertible. Show that $\operatorname{rank}(A)=\operatorname{rank}(B)$.\hl{TODO}
\end{example}

\begin{example}
\label{exa:isi2008samplepsb1}1. Let 
\[
A=\frac{1}{3}\left(\begin{array}{rrr}
2 & -1 & -1\\
-1 & 2 & -1\\
-1 & -1 & 2
\end{array}\right).
\]

Which of the following statements are false. In each case, justify
your answer. (a) $A$ has only one real eigenvalue. (b) $\operatorname{Rank}(A)=\operatorname{Trace}(A)$.
(c) Determinant of $A$ equals the determinant of $A^{n}$ for each
integer $n>1$.\hl{TODO}
\end{example}


\section{Eigenvalues and eigenvectors\label{sec:finiteDimEigenvalues}}


\include{topological_spaces}


\chapter{Basic combinatorics\label{chap:Combinatorics}}

\section{Binomial and multinomial coefficients}

The following is an \emph{algebraic }proof of Pascal's rule, a basic
combinatorial identity. In the remark that follows, we provide a more
combinatorial interpretation of the identity. Throughout this appendix,
we will try to provide both algebraic and combinatorial arguments
for results, unless the algebraic arguments are too cumbersome.
\begin{prop}[Pascal's rule]
\label{prop:pascalRule}For $n\geq k\geq1$
\[
\left(\begin{array}{c}
n\\
k
\end{array}\right)=\left(\begin{array}{c}
n-1\\
k-1
\end{array}\right)+\left(\begin{array}{c}
n-1\\
k
\end{array}\right).
\]
\end{prop}

\begin{proof}
Observe that 
\begin{align*}
\left(\begin{array}{c}
n-1\\
k-1
\end{array}\right)+\left(\begin{array}{c}
n-1\\
k
\end{array}\right) & =\frac{\left(n-1\right)!}{\left(k-1\right)!\left(n-k\right)!}+\frac{\left(n-1\right)!}{k!\left(n-k-1\right)!}\\
 & =\left(n-1\right)!\left[\frac{1}{\left(k-1\right)!\left(n-k\right)!}+\frac{1}{k!\left(n-k-1\right)!}\right]\\
 & =\left(n-1\right)!\left[\frac{k}{k!\left(n-k\right)!}+\frac{n-k}{k!\left(n-k\right)!}\right]\\
 & =\left(n-1\right)!\frac{n}{k!\left(n-k\right)!}\\
 & =\frac{n!}{k!\left(n-k\right)!}\\
 & =\left(\begin{array}{c}
n\\
k
\end{array}\right).
\end{align*}
\end{proof}
\begin{rem*}
The combinatorial idea for the above result is simple. Suppose you
want to select $k$ items from a collection of $n$ items, without
any consideration for order. How does one do so? . First we arbitrarily
pick some item and label it $x$. How many ways are ways are there
to pick items excluding $x$? Well, there are $\left(\begin{array}{c}
n-1\\
k
\end{array}\right)$ ways to pick $k$ items if we explicitly exclude $x$. What if we
insist on including $x?$ Then we have to pick $k-1$ items from $n-1$
remaining total items, that is $\left(\begin{array}{c}
n-1\\
k-1
\end{array}\right).$
\end{rem*}
\begin{example}[ISI 2021 PSA-11]
\label{exa:isi2021psa11}We can use Pascal's formula to recover a
more succinct formula for the expression
\[
\prod_{i=1}^{n}\left(\left(\begin{array}{c}
n\\
i
\end{array}\right)+\left(\begin{array}{c}
n\\
i-1
\end{array}\right)\right).
\]
We can apply Pascal's rule to each term in the product so that 
\begin{align*}
\prod_{i=1}^{n}\left(\left(\begin{array}{c}
n\\
i
\end{array}\right)+\left(\begin{array}{c}
n\\
i-1
\end{array}\right)\right) & =\prod_{i=1}^{n}\left(\begin{array}{c}
n+1\\
i
\end{array}\right)\\
 & =k\prod_{i=1}^{n}\left(\begin{array}{c}
n\\
i
\end{array}\right)
\end{align*}
where $k=\frac{\left(n+1\right)^{n}}{n!}.$
\end{example}

Pascal's rule helps us establish the binomial theorem.
\begin{thm}[Binomial theorem]
\label{thm:binomialTheorem}Let $n\geq1$ be an integer and let $x,y\in\R$.
Then,
\[
\left(x+y\right)^{n}=\sum_{k=0}^{n}\left(\begin{array}{c}
n\\
k
\end{array}\right)x^{k}y^{n-k}.
\]
\end{thm}

\begin{proof}
The identity is easily verified for $n=1$ and $n=2$. Supopose that
it holds for $n-1$ and observe
\begin{align*}
\left(x+y\right)^{n} & =x\left(x+y\right)^{n-1}+y\left(x+y\right)^{n-1}\\
 & =x\sum_{k=0}^{n-1}\left(\begin{array}{c}
n-1\\
k
\end{array}\right)x^{k}y^{n-1-k}+y\sum_{k=0}^{n-1}\left(\begin{array}{c}
n-1\\
k
\end{array}\right)x^{k}y^{n-1-k}\\
 & =\sum_{k=0}^{n-1}\left(\begin{array}{c}
n-1\\
k
\end{array}\right)x^{k+1}y^{n-1-k}+\sum_{k=0}^{n-1}\left(\begin{array}{c}
n-1\\
k
\end{array}\right)x^{k}y^{n-k}\\
 & =\sum_{k=1}^{n}\left(\begin{array}{c}
n-1\\
k-1
\end{array}\right)x^{k}y^{n-k}+\sum_{k=0}^{n-1}\left(\begin{array}{c}
n-1\\
k
\end{array}\right)x^{k}y^{n-k}\\
 & =\left(\begin{array}{c}
n-1\\
n-1
\end{array}\right)x^{n}+\left(\begin{array}{c}
n-1\\
0
\end{array}\right)y^{n}+\sum_{k=1}^{n-1}\left(\begin{array}{c}
n\\
k
\end{array}\right)x^{k}y^{n-k}\\
 & =\left(\begin{array}{c}
n\\
n
\end{array}\right)x^{n}+\left(\begin{array}{c}
n\\
0
\end{array}\right)y^{n}+\sum_{k=1}^{n-1}\left(\begin{array}{c}
n\\
k
\end{array}\right)x^{k}y^{n-k}\\
 & =\sum_{k=0}^{n}\left(\begin{array}{c}
n\\
k
\end{array}\right)x^{k}y^{n-k}
\end{align*}
where in the third line we used a change of variables, and in the
fourth we used Pascal's rule.
\end{proof}
\begin{rem*}
The combinatorial argument is again simple. We know that the coefficient
for $x^{k}y^{n-k}$ in the expansion of $\left(x+y\right)^{n}$ involves
choosing $k$ out of $n$ available $x'$s and the number of ways
to do that is $\left(\begin{array}{c}
n\\
k
\end{array}\right)$.
\end{rem*}
\begin{example}[ISI 2016 PSA 1]
\label{exa:isi2016psa1}How many terms in the binomial expansion
of $\left(3x^{2}+\frac{1}{x}\right)^{5}$ are independent of $x$?
Note that the binomial expansion of this expression is of the form
\begin{align*}
\left(3x^{2}+\frac{1}{x}\right)^{5} & =\sum_{i=0}^{5}\left(\begin{array}{c}
5\\
i
\end{array}\right)3x^{2i}x^{i-5}\\
 & =\sum_{i=0}^{5}\left(\begin{array}{c}
5\\
i
\end{array}\right)3x^{3i-5}.
\end{align*}
Note that $3i-5=0\implies i=\frac{5}{3}$ which is not an integer.
Thus no terms are independent of $x.$
\end{example}

The binomial theorem gives us a simple formula for the sum of binomial
coefficients.
\begin{cor}
\label{cor:sumOfBinomialCoefficients}Let $n\geq1$ be a fixed integer.
Then,
\[
\sum_{k=0}^{n}\left(\begin{array}{c}
n\\
k
\end{array}\right)=2^{n}.
\]
\end{cor}

\begin{proof}
Let $x=y=1$ in the binomial theorem.
\end{proof}
\begin{rem*}
Think about a set with $n$ elements. You want to construct all possible
subsets. You know the number of all possible subsets is $2^{n}$.
How do you count all possible subsets of a finite set? Well, order
doesn't matter in a set so all you have to do is count the empty set,
the sets with one element (singletons), the set with two elements,
those with three and so on.
\end{rem*}
\begin{example}[ISI 2013 PSB 3]
\label{exa:isi2013psb3}Let $S=\left\{ 1,2,\ldots,n\right\} .$ How
many ways can we choose two subsets $B\subseteq A\subseteq S$ such
that $B\neq\emptyset$? How many subsets $B\subsetneq A\subseteq S$
can we choose? For the first question, we know that for each subset
$A$ of size $k$ we can select $2^{k}-1$ nonempty subsets. Further,
there are $\left(\begin{array}{c}
n\\
k
\end{array}\right)$ subsets of size $k$ and so the number of such sets is $\sum_{k=1}^{n}\left(\begin{array}{c}
n\\
k
\end{array}\right)\left(2^{k}-1\right)$.
\end{example}

There are a number of interesting identities regarding binomial coefficients
which have simple combinatorial interpretations. We list a few here.
\begin{prop}
\label{prop:mdmActivity94}Let $n$ be a non-negative integer. Then,
\[
\left(\begin{array}{c}
2n\\
2
\end{array}\right)=2\left(\begin{array}{c}
n\\
2
\end{array}\right)+n^{2}.
\]
\end{prop}

\begin{proof}
Note that 
\begin{align*}
\left(\begin{array}{c}
2n\\
2
\end{array}\right) & =\frac{\left(2n\right)!}{2!\left(2n-2\right)!}\\
 & =\frac{2n\left(2n-1\right)}{2!}\\
 & =2n^{2}-n\\
 & =n(n-1)+n^{2}\\
 & =2\left(\begin{array}{c}
n\\
2
\end{array}\right)+n^{2}.
\end{align*}
\end{proof}
\begin{rem*}
Suppose you have $n$distinguihsable red balls and $n$ distinguishble
green balls. How many ways are there to choose two balls from this
collection? Well, you can get one red and one green and there are
$n^{2}$ ways of getting such pairs. Alternatively, you can get two
of red or two of green and there are $\left(\begin{array}{c}
n\\
2
\end{array}\right)$ possible ways to select two reds (or two greens).
\end{rem*}
\begin{prop}
\label{prop:mdmActivity95}Let $m,n\geq1$ be positive integers. Then,
\[
\left(\begin{array}{c}
m+n\\
2
\end{array}\right)-\left(\begin{array}{c}
m\\
2
\end{array}\right)-\left(\begin{array}{c}
n\\
2
\end{array}\right)=mn.
\]
\end{prop}

\begin{proof}
Note that 
\begin{align*}
\left(\begin{array}{c}
m+n\\
2
\end{array}\right)-\left(\begin{array}{c}
m\\
2
\end{array}\right)-\left(\begin{array}{c}
n\\
2
\end{array}\right) & =\frac{\left(m+n\right)!}{2!\left(m+n-2\right)!}-\frac{m!}{2!\left(m-2\right)!}-\frac{n!}{2!\left(n-2\right)!}\\
 & =\frac{\left(m+n\right)\left(m+n-1\right)-m(m-1)-n(n-1)}{2}\\
 & =\frac{m^{2}+n^{2}+2mn-m^{2}+m-n^{2}+n}{2}\\
 & =mn.
\end{align*}
\end{proof}
\begin{rem*}
Suppose you want to count the number of ways in which you can choose
2 items from $m$ distinguishable green balls and $n$ distinguishable
red balls and you want one of each. Then you can count the number
of ways you can choose two of those in aggregate and subtract out
the number of ways in which you could choose both of one color. On
the other hand there are clearly $mn$ such ways (by the product rule
if you will).
\end{rem*}
\begin{prop}
\label{prop:prodSumBinomialCoefficients}Let $n\geq1$ be an integer,.
Then,
\[
\sum_{i=1}^{n}i\left(\begin{array}{c}
n\\
i
\end{array}\right)=n2^{n-1}.
\]
\end{prop}

\begin{proof}
Note that by the binomial theorem,
\[
\left(1+x\right)^{n}=\sum_{k=0}^{n}\left(\begin{array}{c}
n\\
k
\end{array}\right)x^{k}.
\]
Taking derivatives on both sides, we have that 
\[
n\left(1+x\right)^{n-1}=\sum_{k=0}^{n}\left(\begin{array}{c}
n\\
k
\end{array}\right)kx^{k-1}.
\]
Letting $x=1$ yields the result.
\end{proof}
\begin{rem*}
\hl{Fill later}
\end{rem*}
\begin{prop}
\label{prop:squaredSumBinomialCoefficients}Let $n\geq1$ be a fixed
integer. Then,
\[
\sum_{k=0}^{n}\left(\begin{array}{c}
n\\
k
\end{array}\right)^{2}=\left(\begin{array}{c}
2n\\
n
\end{array}\right).
\]
\end{prop}

\begin{proof}
Consider the algebraic identity
\[
\left[\left(1+x\right)^{n}\right]^{2}=\left(1+x\right)^{2n}.
\]
We can expand the left hand side as 
\[
\left(\sum_{k=0}^{n}\left(\begin{array}{c}
n\\
k
\end{array}\right)x^{k}\right)^{2}=\sum_{k=0}^{n}\sum_{j=0}^{n}\left(\begin{array}{c}
n\\
k
\end{array}\right)\left(\begin{array}{c}
n\\
j
\end{array}\right)x^{j+k}.
\]
The coefficient to the term $x^{n}$ is $\sum_{k=0}^{n}\left(\begin{array}{c}
n\\
k
\end{array}\right)\left(\begin{array}{c}
n\\
n-k
\end{array}\right)=\sum_{k=0}^{n}\left(\begin{array}{c}
n\\
k
\end{array}\right)^{2}.$On the other hand, a simple application of binomial theorem tells
us that the coefficient of $x^{n}$ on the right hand side is $\left(\begin{array}{c}
2n\\
n
\end{array}\right)$ which completes the proof.
\end{proof}
\begin{rem*}
Suppose you have $n$ distinguishable red balls and $n$distinguishable
green balls. How many ways can you collect $n$ items from this group?
Well the answer is clearly $\left(\begin{array}{c}
2n\\
n
\end{array}\right)$ , but we can break this up by considering that we can select 1 red
ball and $n-1$ green balls, or $2$ red balls and $n-2$ green balls...
\end{rem*}
The above result is a special case of Vandermonde's identity, which
was known to mathematicians as early as the 14th century.
\begin{prop}
\label{prop:vandermondeIdentity}Let $m,n,$ and $r$ be non-negative
integers. Then,
\[
\left(\begin{array}{c}
m+n\\
r
\end{array}\right)=\sum_{k=0}^{r}\left(\begin{array}{c}
m\\
k
\end{array}\right)\left(\begin{array}{c}
n\\
r-k
\end{array}\right).
\]
\end{prop}

\begin{proof}
The algebraic proof is similar to the one for the special case in
Proposition \ref{prop:squaredSumBinomialCoefficients}. The combinatorial
argument is more illuminating and significantly less cumbersome. Consider
a committee consisting of $m$ men and $n$ women. We want to form
a subcommittee of $r$ members. Clearly the number of such subcommittes
is $\left(\begin{array}{c}
m+n\\
r
\end{array}\right)$. One way to form such a committee is to consider a committee with
one man and $r-1$ women, or two men and $r-2$ women, or ...
\end{proof}
\begin{prop}
\label{prop:mdmActivity103}Let $n\geq1$ be an integer. Then,
\[
\sum_{i=1}^{n}i\left(n+1-i\right)=\left(\begin{array}{c}
n+2\\
3
\end{array}\right).
\]
\end{prop}

\begin{proof}
For $n=1$ both the left and right hand sides of the identity above
are 1 and so the base case holds. Suppose the identity holds for $n$.
Then,
\begin{align*}
\sum_{i=1}^{n+1}i\left(n+1+1-i\right) & =\sum_{i=1}^{n+1}i\left(n+1-i\right)+\sum_{i=1}^{n+1}i\\
 & =\sum_{i=1}^{n}i\left(n+1-i\right)+\left(\begin{array}{c}
n+2\\
2
\end{array}\right)\\
 & =\left(\begin{array}{c}
n+2\\
3
\end{array}\right)+\left(\begin{array}{c}
n+2\\
2
\end{array}\right)\\
 & =\left(\begin{array}{c}
n+2\\
3
\end{array}\right)
\end{align*}
where the second equality is the Gaussian formula for the sum of consecutive
natural numbers, the third is the induction hypothesis, and the last
is Pascal's rule.
\end{proof}
\begin{prop}
\label{prop:sumOfBinomialCoefficients2}Let $n\geq2$ be an integer.
Then
\[
\sum_{k=2}^{n}\left(\begin{array}{c}
k\\
2
\end{array}\right)=\left(\begin{array}{c}
n+1\\
3
\end{array}\right).
\]
\end{prop}

\begin{proof}
$n=2$ the result follows easily since $\left(\begin{array}{c}
2\\
2
\end{array}\right)=\left(\begin{array}{c}
3\\
3
\end{array}\right)=1$. For $n=3$ we have $\left(\begin{array}{c}
2\\
2
\end{array}\right)+\left(\begin{array}{c}
3\\
2
\end{array}\right)=4=\left(\begin{array}{c}
4\\
3
\end{array}\right).$ Now suppose the result holds for $n-1$ and note that 
\begin{align*}
\sum_{k=2}^{n}\left(\begin{array}{c}
k\\
2
\end{array}\right) & =\sum_{k=2}^{n-1}\left(\begin{array}{c}
k\\
2
\end{array}\right)+\left(\begin{array}{c}
n\\
2
\end{array}\right)\\
 & =\left(\begin{array}{c}
n\\
3
\end{array}\right)+\left(\begin{array}{c}
n\\
2
\end{array}\right)\\
 & =\left(\begin{array}{c}
n+1\\
3
\end{array}\right)\\
\end{align*}
where the second equality uses the induction hypothesis and the last
uses Pascal's rule.
\end{proof}
We can generalize the notion of binomial coefficients to that of multinomial
coefficients with the following result.
\begin{prop}
\label{prop:multinomialCoefficients}Let $r_{1},r_{2},\ldots,r_{k}$
be positive integers and let $n=\sum_{i=1}^{k}r_{i}$. Then the number
of ways to split a set of size $n$ into $k$ ordered subsets where
the $i$th subset contains $r_{i}$ elements is given by
\[
\left(\begin{array}{cccc}
 & n\\
r_{1}, & r_{2}, & \ldots & r_{k}
\end{array}\right):=\frac{n!}{r_{1}!r_{2}!\ldots r_{k}!}.
\]
\end{prop}

\begin{proof}
Note that first we choose $r_{1}$ items from $n$ items, then $r_{2}$
items from $n-r_{1}$ remaining items, then $r_{3}$ from the remaining
$n-r_{1}-r_{2}$ and so on, yielding
\begin{align*}
\left(\begin{array}{cccc}
 & n\\
r_{1}, & r_{2}, & \ldots & r_{k}
\end{array}\right) & =\prod_{i=1}^{k}\left(\begin{array}{c}
n-\sum_{j=0}^{i-1}r_{j}\\
r_{i}
\end{array}\right)\\
 & =\prod_{i=1}^{k}\frac{\left(n-\sum_{j=0}^{i-1}r_{j}\right)!}{r_{i}!\left(n-\sum_{j=0}^{i}r_{j}\right)!}\\
 & =\frac{n!}{r_{1}!r_{2}!\ldots r_{k}!}.
\end{align*}
The last equality follows because there's a ``telescoping'' product
and so the all $\left(n-\sum_{j=0}^{i-1}r_{j}\right)!$ terms except
the first (which is is $n!$) cancel.
\end{proof}
This gives us the multinomial theorem.
\begin{thm}[Multinomial theorem]
\label{thm:multinomialTheorem}Let $m,n\geq1$ be fixed integers.
Then, for $x_{1},x_{2},\ldots,x_{m}\in\R$
\[
\left(\sum_{i=1}^{m}x_{i}\right)^{n}=\sum_{k_{1}+\ldots+k_{m}=n,k_{i}\geq0}\left(\begin{array}{cccc}
 & n\\
k_{1}, & k_{2}, & \ldots & k_{m}
\end{array}\right)\prod_{i=1}^{m}x_{i}^{k_{i}}.
\]
\end{thm}

\begin{proof}
For $m=1$ the result is trivial. For $m=2$, we shall verify that
it is in fact the binomial theorem. To see this, note that for $m=2$,
the binomial theorem implies that 
\begin{align*}
\left(x_{1}+x_{2}\right)^{n} & =\sum_{k=0}^{n}\left(\begin{array}{c}
n\\
k
\end{array}\right)x_{1}^{k}x_{2}^{n-k}\\
 & =\sum_{k=0}^{n}\left(\begin{array}{c}
n\\
k,n-k
\end{array}\right)x_{1}^{k}x_{2}^{n-k}\\
 & =\sum_{k_{1}+k_{2}=n,k_{i}\geq0}\left(\begin{array}{c}
n\\
k_{1},k_{2}
\end{array}\right)x_{1}^{k_{1}}x_{2}^{k_{2}}
\end{align*}
where the second equality uses Proposition \ref{prop:multinomialCoefficients}.
To extend this result to arbitrary $m$, we assume that it holds for
$m-1$ and then note that 
\begin{align*}
\left(\sum_{i=1}^{m-1}x_{i}+x_{m}\right)^{n} & =\sum_{K+k_{m}=n,K,k_{m}\geq0}\left(\begin{array}{c}
n\\
K,k_{m}
\end{array}\right)\left(\sum_{i=1}^{m-1}x_{i}\right)^{K}x_{m}^{k_{m}}\\
 & =\sum_{K+k_{m}=n,K,k_{m}\geq0}\left(\begin{array}{c}
n\\
K,k_{m}
\end{array}\right)\sum_{k_{1}+\ldots+k_{m-1}=K,k_{i}\geq0}\left(\begin{array}{c}
K\\
k_{1},\ldots,k_{m-1}
\end{array}\right)\prod_{i=1}^{m}x_{i}^{k_{i}}\\
 & =\sum_{K+k_{m}=n,K,k_{m}\geq0}\sum_{k_{1}+\ldots+k_{m-1}=K,k_{i}\geq0}\left(\begin{array}{c}
n\\
K,k_{m}
\end{array}\right)\left(\begin{array}{c}
K\\
k_{1},\ldots,k_{m-1}
\end{array}\right)\prod_{i=1}^{m}x_{i}^{k_{i}}\\
 & =\sum_{K+k_{m}=n,K,k_{m}\geq0}\sum_{k_{1}+\ldots+k_{m-1}=K,k_{i}\geq0}\left(\begin{array}{cccc}
 & n\\
k_{1}, & k_{2}, & \ldots & k_{m}
\end{array}\right)\prod_{i=1}^{m}x_{i}^{k_{i}}\\
 & =\sum_{k_{1}+\ldots+k_{m}=n,k_{i}\geq0}\left(\begin{array}{cccc}
 & n\\
k_{1}, & k_{2}, & \ldots & k_{m}
\end{array}\right)\prod_{i=1}^{m}x_{i}^{k_{i}}
\end{align*}
which completes the proof.

There is an obvious generalization of Vandermonde's identity that
should occur to you at this point.
\end{proof}
\begin{prop}[Generalized Vandermonde identity]
\label{prop:generalizedVandermondeIdentity}Let $\left\{ n_{i}\right\} _{i=1}^{k},r$
be non-negative integers. Then,
\[
\left(\begin{array}{c}
\sum_{i=1}^{k}n_{i}\\
r
\end{array}\right)=\sum_{i_{1}+i_{2}+\ldots+i_{k}=r}\prod_{j=1}^{k}\left(\begin{array}{c}
n_{j}\\
i_{j}
\end{array}\right).
\]
\end{prop}


\subsubsection{Stars and bars}

Sums like $\sum_{i_{1}+i_{2}+\ldots+i_{k}=r}$ should trouble you,
since it's not obvious how many terms are in such a sum. Fortunately,
we have the tools we need to be able to answer such questions. To
do so, we make an analogy with a different type of question: suppose
you have $k$ urns and $r$ indistinguishable balls. How many ways
could you allocate balls to urns? William Feller's famous textbook
on probability provided an ingenious framework to think about this
problem: suppose you have $k+1$ bars and $r$ stars, you could then
fix two bars at two ends and put all the stars and remaining bars
in the middle. The gap between bars would act as urns and the stars
would be balls. Suppose you have four bars and ten stars, one arrangment
could be like $|**|*****|**|$. This corresponds to the solution where
there are $3$ urns and the first urn has $2$ balls, the second has
$5$ balls, and the last again has 2 balls. How many other possibilities
are there? Well we can rearrange the bars and stars in the interior
of by looking at the number of slots in the interior (in this case
12) and choosing the slots for either the bars or the stars (in this
case 2 or 10). So the answer is $\left(\begin{array}{c}
12\\
2
\end{array}\right)=\left(\begin{array}{c}
12\\
10
\end{array}\right)$. More generally, there are $k-1$ bars in the interior along with
$r$ stars, leading to $\left(\begin{array}{c}
r+k-1\\
k-1
\end{array}\right)=\left(\begin{array}{c}
r+k-1\\
r
\end{array}\right)$ ways to arrange the interior bars and stars. Equivalently, there
are $\left(\begin{array}{c}
r+k-1\\
k
\end{array}\right)$ ways to put $r$ balls in $k$ urns. This simple idea has remarkable
power in its ability to resolve combinatorial problems. For instance,
if we apply the restriction that each urn needs to have at least $1$
ball, then our result becomes $\left(\begin{array}{c}
k-1+(r-k)\\
r-k
\end{array}\right)=\left(\begin{array}{c}
r-1\\
k-1
\end{array}\right).$
\begin{example}[ISI 2018 PSA 13 (variant)]
\label{exa:countingFunctions}How many functions $f:\left\{ 1,2,\ldots,n\right\} \to\left\{ 1,2,\ldots,m\right\} $
are strictly increasing? How many are non-decreasing? To answer questions
like this, we can use some of the ideas from the discussion above.
First, you should notice that there are $\left(\begin{array}{c}
m\\
n
\end{array}\right)$ ways to choose the range of the function (it has to be injective
after all), and exactly one way for each of those ways to arrange
them in increasing order. This gives us the total number of strictly
increasing functions. To count non-decreasing functions, we
\end{example}

\begin{example}[ISI 2016 PSA 21]
\label{exa:isi2016psa21}Let $A=\{1,2,\ldots n,\ldots,m\}$. How
many functions $f:A\rightarrow A$ are there such that $f(1)<f(2)<\ldots,<f(n)$
? Well there are $\left(\begin{array}{c}
m\\
n
\end{array}\right)$ ways of selecting the strictly increasing elements and $\left(m-n\right)^{\left(m-n\right)}$
ways of choosing how the remaining elements are arranged. Thus the
total number of of such functions is $\left(\begin{array}{c}
m\\
n
\end{array}\right)\left(m-n\right)^{\left(m-n\right)}.$
\end{example}

\begin{example}[ISI 2016 PSA 11]
\label{exa:isi2016psa11}The number of ordered pairs $\left(a,b\right)\in\N^{2}$
such that $a+b\leq n$ where $n\in\N$ is $\sum_{i=2}^{n}\left(\begin{array}{c}
i-1\\
2-1
\end{array}\right)=1+2+3+\ldots+n-1=\frac{n\left(n-1\right)}{2}$
\end{example}


\section{Inclusion-exclusion and its consequences}
\begin{lem}
\label{lem:inclusionExclusion}Let $A$ denote the union of sets $A_{1},A_{2},\ldots,A_{n}$,
all of which are subsets of some ambient set $\X$. Then,
\begin{align}
\indicate_{A} & =\sum_{i=1}^{n}\left(-1\right)^{i-1}\sum_{J\subset\left\{ 1,2,\ldots n\right\} ,\lvert J\rvert=i}\indicate_{\bigcap_{j\in J}A_{j}}.\label{eq:inclusionExclusionIndicator}
\end{align}
\end{lem}

\begin{proof}
Consider the function 
\[
g\left(x\right)=\prod_{i=1}^{n}\left(\indicate_{A}\left(x\right)-\indicate_{A_{i}}\left(x\right)\right).
\]
We claim that $g\left(x\right)=0$ for all $x\in\X$. To see this,
notice that if $x\in A_{i}$ for any $1\leq i\leq n$ then that particular
factor is zero. Conversely if $x\notin A$ then all the factors are
zero. Rearranging the equation $g\left(x\right)=0$ and using Fact
(\ref{fact:indicatorFunctionsFiniteOperations})yields the result.
\end{proof}
\begin{thm}[Inclusion-Exclusion]
\label{thm:inclusionExclusionCardinality}Let $A_{1},A_{2},\ldots,A_{n}$
be finite sets and let $A=\bigcup A_{i}$ . Then,
\[
\lvert A\rvert=\sum_{i=1}^{n}\left(-1\right)^{i-1}\sum_{J\subset\left\{ 1,2,\ldots n\right\} ,\lvert J\rvert=i}\lvert\bigcap_{j\in J}A_{j}\rvert.
\]
\end{thm}

\begin{proof}
We are going to cheat here and use measure theory. We can take the
integral with respect to the counting measure on (\ref{eq:inclusionExclusionIndicator})and
recover the result by linearity.
\end{proof}
The inclusion-exclusion principle is very useful in counting the number
of \emph{derangements} of a given set. A derangement of a finite set
$A$ is a permutation on that set with no fixed points.
\begin{prop}
\label{prop:numDerangements}Let $A$ be a finite set such that $\lvert A\rvert=n.$
The number of derangements $\sigma:A\to A$ is given by 
\[
!n:=n!\sum_{i=0}^{n}\frac{\left(-1\right)^{i}}{i!}.
\]
\end{prop}

\begin{proof}
Without loss of generality, we can assume that $A=\left\{ 1,2,\ldots,n\right\} .$
Let $S_{k}:=\left\{ \sigma\in\mathrm{Perm}\left(A\right)\mid\sigma\left(k\right)=k\right\} $
for $1\leq k\leq n.$ That is, each $S_{k}$ fixes $k$ and may or
may not fix any other elements. Then, for any $J\subset A$, we have
that 
\[
\sum_{J\subset\left\{ 1,2,\ldots n\right\} ,\lvert J\rvert=i}\lvert\bigcap_{j\in J}S_{j}\rvert=\left(\begin{array}{c}
n\\
i
\end{array}\right)\left(n-i\right)!
\]
because the intersection of any $i$ elements of $\left\{ S_{k}\right\} _{1\leq k\leq n}$
consists of permutations which fix at least $i$ points and there
are $\left(\begin{array}{c}
n\\
i
\end{array}\right)$ ways to pick $i$ fixed points and $\left(n-i\right)!$ ways to permute
all the other elements. Then, the number of ways in which you can
have at least one fixed point is given by the inclusion-exclusion
formula
\begin{align*}
\lvert\bigcup_{i=1}^{n}S_{i}\rvert & =\sum_{i=1}^{n}\left(-1\right)^{i-1}\sum_{J\subset\left\{ 1,2,\ldots n\right\} ,\lvert J\rvert=i}\lvert\bigcap_{j\in J}A_{j}\rvert\\
 & =\sum_{i=1}^{n}\left(-1\right)^{i-1}\left(\begin{array}{c}
n\\
i
\end{array}\right)\left(n-i\right)!\\
 & =\sum_{i=1}^{n}\left(-1\right)^{i-1}\frac{n!}{i!\left(n-i\right)!}\left(n-i\right)!\\
 & =n!\sum_{i=1}^{n}\frac{\left(-1\right)^{i-1}}{i!}.
\end{align*}
The number of derangements is then simply the difference between the
total number of permutations and the number of permutations that has
at least one fixed point i.e.
\begin{align*}
!n & =n!-n!\sum_{i=1}^{n}\frac{\left(-1\right)^{i-1}}{i!}\\
 & =n!\left(1-\sum_{i=1}^{n}\frac{\left(-1\right)^{i-1}}{i!}\right)\\
 & =n!\left(1+\sum_{i=1}^{n}\frac{\left(-1\right)^{i}}{i!}\right)\\
 & =n!\sum_{i=0}^{n}\frac{\left(-1\right)^{i}}{i!}.
\end{align*}
\end{proof}
\begin{rem*}
There's a recursive formulation for counting derangements as well.
To see this, first think about \hl{Finish later}
\end{rem*}
\begin{cor}
\label{cor:recontresNumbers}Let $A$ be a finite set with cardinality
$n.$ The number of permutations $\sigma:A\to A$ with exactly $k$
fixed points, where $1\leq k\leq n$. is 
\[
D_{n,k}:=\left(\begin{array}{c}
n\\
k
\end{array}\right)!\left(n-k\right).
\]
\end{cor}

It's worth noting down the values of small derangements so that one
isn't forced to compute these when solving problems (much in the way
we often memorize small factorials).

\begin{table}

\caption{Values of derangements}

\begin{centering}
\begin{tabular}{ccc}
\hline 
$n$ & $n!$ & $!n$\tabularnewline
\hline 
\hline 
1 & 1 & 0\tabularnewline
2 & 2 & 1\tabularnewline
3 & 6 & 2\tabularnewline
4 & 24 & 9\tabularnewline
5 & 120 & 44\tabularnewline
6 & 720 & 265\tabularnewline
7 & 5040 & 1854\tabularnewline
8 & 40320 & 14833\tabularnewline
9 & 362880 & 133496\tabularnewline
10 & 3628800 & 1334961\tabularnewline
\end{tabular}
\par\end{centering}
\end{table}

\begin{example}[JAM 2022 P-54]
\label{exa:jam2022p54}Suppose that five men go to a restaurant together
and each of them orders a dish that is different from the dishes ordered
by the other members of the group. However, the waiter serves the
dishes randomly. Then what is the number of ways in which exactly
one of them gets the dish he ordered? The answer is $D_{5,1}=\left(\begin{array}{c}
5\\
1
\end{array}\right)!4=5\times9=45$.
\end{example}

We can generalize the idea of derangements by considering permutations
that don't have \emph{cycles. }A cycle is itself a sort of generalization
of a fixed point. A permutation $f$ on a finite set $A$ of size
$n$ is said to have a $k-$cycle if there exists some $x\in A$ such
that $f^{k}\left(x\right)=x$ where the exponent is denoting repeated
composition rather than multiplication. Of course, here $k$ represents
the \emph{smallest }positive integer such that the equality holds
true. It should be clear that it also holds true for any \emph{multiple
}of $k$. To understand cycles in permutations, it's useful to adopt
a notation for describing permutations that can help clarify their
cyclic structure. This is the so called \emph{cyclic notation}. For
example, we can write a permutation on $\left\{ 1,2,3,4,5\right\} $
\[
f=\left(351\right)\left(24\right)
\]
which basically tells us that the permutation consists of two cycles
$\left(351\right)$ and $\left(24\right)$. The cycles here are \emph{ordered},
in that the first cycle represents the fact that $f\left(3\right)=5,f\left(5\right)=1$
and $f\left(1\right)=3$ and the second cycle tells us that $f\left(2\right)=4$
and $f\left(4\right)=2$. Every permutation can be decomposed into
cycles, and \emph{non-cyclic }permutations are those that consist
of only the trivial cycle: for instance, a non-cyclic permutation
on $\left\{ 1,2,3,4,5\right\} $ would be the permutation $\left(14235\right).$
Note that two cycles of length $k$ $\left(x_{1}x_{2}\ldots x_{k}\right)$
and $\left(y_{1}y_{2}\ldots y_{k}\right)$ are equivalent if there
exists some $p\in\N$ such that $y_{i}=f^{p}\left(x_{i}\right)$.
This defines an equivalence relation (as should be clear), so we can
talk about \emph{equivalence classes} of cycles, denoted $\left[\left(x_{1}x_{2}\ldots x_{k}\right)\right]$.
Such an equivalence class consists of exactly $k$ distinct members
since $f^{k}\left(x_{i}\right)=x_{i}.$ Thus the total number of cycles
of length $k$ that can be formed out of $k$ fixed elements is $\frac{k!}{k}$.
We can use this to count the total number of permutations without
any cycles using an inclusion-exclusion argument analagous to the
one used for counting derangements in Proposition \ref{prop:numDerangements}.
\begin{example}[ISI 2021 PSA 14]
\label{exa:isi2021psa14}What is the total number of permutations
$\sigma:\left\{ 1,2,\ldots,6\right\} \to\left\{ 1,2,\ldots,6\right\} $
such that $\sigma\left(\sigma\left(i\right)\right)\neq i$ for any
$i\in\left\{ 1,2,\ldots,6\right\} $?\hl{TODO}
\end{example}


\section{Miscellaneous problems}
\begin{example}[ISI 2019 PSA 11]
\label{exa:isi2019psa11}What are the total number of divisors of
$2^{5}5^{3}11^{4}$ that are perfect squares? The prime square divisors
are $2^{2},5^{2},$ and $11^{2}$, where the first and last divisors
appear twice. Thus the first product can appear at most twice, the
second at most once, and the third at most twice in any square divisor
and so the number is $(2+1)(1+1)(2+1)=18$. More generally, for any
positive integer $n$ with prime factorization $n=\prod_{i=1}^{k}p_{i}^{r_{i}}$,
we can decompose the product into the largest square divisor 
\[
n=\prod_{i=1}^{k}\left(p_{i}^{2}\right)^{\lfloor\frac{r_{i}}{2}\rfloor}\prod_{i=1}^{k}p_{i}^{\indicate\left\{ \lfloor\frac{r_{i}}{2}\rfloor\neq\frac{r_{i}}{2}\right\} }
\]
where the second product is square-free and so the number of square
divisors is given $\prod_{i=1}^{k}\left(\lfloor\frac{r_{i}}{2}\rfloor+1\right)$.
\end{example}

\begin{example}[ISI 2019 PSA 18]
\label{exa:isi2019psa18}Draw one observation $N$ at random from
the set $\{1,2,\ldots,100\}$. What is the probability that the last
digit of $N^{2}$ is 1 ? Well note that only if the units digit of
$N$ is 1 or 9 does the units digit of $N^{2}$ equal 1, which tells
us that the probability is $\frac{1}{5}.$
\end{example}

\begin{example}[ISI 2019 PSA 6]
\label{exa:isi2019psa6} How many times does the digit '2' appear
in the set of integers $\{1,2,..,1000\}$ ? In the units digit, '2'
appears $10\times10=100$ times; in the tens digit, it appears 10
times and in the 100s digit it appears once. Thus in total it appears
111 times.
\end{example}

\begin{example}[ISI 2019 PSA 5]
\label{exa:isi2019psa5}Let the sum $3+33+333+\cdots+\underbrace{33\ldots3}_{200\text{ times }}$
be $...zyx$ in the decimal system, i.e., $x$ is the unit's digit,
$y$ the ten's digit, and so on. What is $z$ ?
\end{example}

\begin{example}[ISI 2021 PSA 12]
\label{exa:isi2021psa12} Let $\pi=\left(a_{1},a_{2},\cdots,a_{2021}\right)$
be a permutation of $(1,2,\cdots,2021)$. For every such permutation
$\pi$, 
\[
P(\pi)=\prod_{j=1}^{2021}\left(a_{j}-j\right).
\]

Is $P\left(\pi\right)$ always even for any permutation $\pi$? Yes.
To see this, note that $\sum_{j=1}^{2021}\left(a_{j}-j\right)=0$
and so 
\[
\sum_{j\neq i}\left(a_{j}-j\right)=-\left(a_{i}-i\right)
\]
where for any $i\in\left\{ 1,2,\ldots,2021\right\} $ the sum on the
LHS has an even number of terms. If the terms are all odd, the sum
(and therefore the RHS) is even and thus the product is even.
\end{example}

\begin{example}[ISI 2020 PSA 14]
\label{exa:isi2020psa14}Let $S$ be the set of all $3\times3$ matrices$A$
such that among the 9 entries of $A$, there are exactly three 0 's,
exactly three 1 's and exactly three 2 's. What is the number of matrices
in $S$ that have trace divisible by 3? \hl{TODO}
\end{example}

\begin{example}[ISI 2023 PSB 2]
\label{exa:isi2023psb2}How many permutations of the numbers $1,2,\ldots,n$
where $n$is even exist such that no two adjacent numbers have an
odd product? Let's first count the number of such permutations where
the first slot is taken up by an even number. The number of positive
even integers less or equal to $n$ is $\frac{n}{2}$ and so answer
is for these is $\frac{n}{2}!^{2}$. The odd first answer is $\frac{n}{2}!^{2}$
so together the answer is $2\frac{n}{2}!^{2}$.
\end{example}

\begin{example}[ISI 2015 PSB 4]
\label{exa:isi2015psb4}Suppose 15 identical balls are placed in
3 boxes labeled A, B and C. What is the number of ways in which Box
A can have more balls than Box C?\hl{TODO}
\end{example}





\chapter{Common probability distributions\label{chap:probabilityDistributions}}

\section{General families of distributions}

\subsection{Location-scale families}

\subsection{General exponential families}

\subsection{Stable distributions}

\subsection{Infinitely divisible distributions}

\subsection{Power series distributions}

\section{Special parametric families of distributions}

\subsection{Normal distributions and their associates}

\subsubsection{The univariate normal distribution}

\subsubsection{The multivariate normal distribution}
\begin{example}
\label{exa:isi2008samplepsb8}Let $\underline{Y}=\left(Y_{1},Y_{2}\right)^{\prime}$
have the bivariate normal distribution $N_{2}(\underline{0},\Sigma)$,
where 
\[
\Sigma=\left(\begin{array}{cc}
\sigma_{1}^{2} & \rho\sigma_{1}\sigma_{2}\\
\rho\sigma_{1}\sigma_{2} & \sigma_{2}^{2}
\end{array}\right).
\]

Obtain the mean and variance of $U=\underline{Y^{\prime}}\Sigma^{-1}\underline{Y}-Y_{1}^{2}/\sigma_{1}^{2}$.\hl{TODO}
\end{example}


\subsubsection{The lognormal distribution}

\subsubsection{The folded normal distribution}

\subsubsection{The Rayleigh distribution}

\subsubsection{The Maxwell distribution}

\subsubsection{The Levy distribution}

\subsection{Distributions useful for basic statistical inference}

\subsubsection{The Gamma distribution}

\begin{example}
\label{exa:isi2009samplepsb3}
Using an appropriate probability distribution or otherwise show that
$$
\lim _{n \rightarrow \infty} \int_0^n \frac{\exp (-x) x^{n-1}}{(n-1)!} d x=\frac{1}{2} .
$$
\hl{TODO}
\end{example}

\subsubsection{The Chi-squared distributtion}

\subsubsection{Student's $t$ distribution}

\subsubsection{The $F$ distribution}

\subsection{Continuous distributions with bounded support}

\subsubsection{The uniform distribution\label{subsec:uniformDistribution}}

The uniform distribution on the interval $\left[a,b\right]$ is the
simplest example of a distribution that is absolutely continuous with
respect to the Lebesgue measure. In fact, its density is $\frac{1}{b-a}\indicate_{\left[a,b\right]}$
which means the distribution is simply the restriction of the Lebesgue
measure to the interval $\left[a,b\right]$ , with an appropriate
normalization to ensure that it is a probability measure. A random
variable $X$ distributed uniformly on $\left[a,b\right]$ is often
denoted $X\sim U\left[a,b\right]$. The CDF of such an $X$ is $F_{X}\left(x\right)=\lebInt{\lambda}{\frac{1}{b-a}\indicate_{\left[a,x\right]}}=\frac{x-a}{b-a}\indicate\left\{ a\leq x\leq b\right\} +\indicate\left\{ x>b\right\} .$
The moments can be computed easily.
\begin{prop}
\label{prop:momentsUniformDistribution}Let $\probabilityspace$ be
a probability space and let $X\sim U\left[a,b\right]$. Then,
\begin{align*}
\E\left[X^{k}\right] & =\frac{b^{k+1}-a^{k+1}}{(k+1)\left(b-a\right)}.\\
\end{align*}
\end{prop}

\begin{proof}
Note that by \hyperref[cor:changeOfVariables]{change of variables}
and the fundamental theorem of calculus
\begin{align*}
\E\left[X^{k}\right] & =\lebInt{\lambda_{x}}{x^{k}\frac{1}{b-a}\indicate\left[a,b\right]}\\
 & =\frac{1}{b-a}\left.\frac{x^{k+1}}{k+1}\right|_{a}^{b}\\
 & =\frac{b^{k+1}-a^{k+1}}{(k+1)\left(b-a\right)}.
\end{align*}
\end{proof}
\begin{prop}
\label{prop:mgfUniformDistribution}Let $\probabilityspace$ be a
probability space and let $X\sim U\left[a,b\right]$. Then,
\begin{align*}
M_{X}\left(t\right) & =\frac{e^{bt}-e^{at}}{t\left(b-a\right)}\indicate\left\{ t\neq0\right\} +\indicate\left\{ t=0\right\} .
\end{align*}
\end{prop}

\begin{proof}
Suppose $t\neq0$, in which case
\begin{align*}
M_{X}\left(t\right) & =\E\left[e^{tX}\right]\\
 & =\frac{1}{b-a}\lebInt{\lambda_{x}}{e^{tx}\indicate\left[a,b\right]}\\
 & =\frac{1}{b-a}\frac{1}{t}\left.e^{tx}\right|_{a}^{b}\\
 & =\frac{e^{tb}-e^{ta}}{t\left(b-a\right)}.
\end{align*}
 The other case is trivial.
\end{proof}
Other interesting moments of the distribution are the variance, skewness
and kurtosis.The variance is found easily by the identity 
\begin{align*}
\Var\left[X\right] & =\E\left[X^{2}\right]-\left(\E\left[X\right]\right)^{2}\\
 & =\frac{b^{3}-a^{3}}{3\left(b-a\right)}-\left(\frac{a+b}{2}\right)^{2}\\
 & =\frac{\left(a+b\right)^{2}-ab}{3}-\frac{\left(a+b\right)^{2}}{4}\\
 & =\frac{\left(b-a\right)^{2}}{12}.
\end{align*}
Similarly, the skewness is given
\begin{align*}
\mathrm{skew}\left(X\right) & =\E\left[\left(\frac{X-\mu}{\sigma}\right)^{3}\right]\\
 & =
\end{align*}


\subsubsection{The Beta distribution}

\subsubsection{The Beta Prime distribution}

\subsubsection{The arcsine distribution}

\subsubsection{The semicircle distribution}

\subsubsection{The triangle distribution}

\subsubsection{The Irwin-Hall distribution}

\subsection{Continuous distributions with positve support}

\subsubsection{Exponential-logarithmic distribution}

\subsubsection{The Gompertz distribution}

\subsubsection{The Log-logistic distribution}

\subsubsection{The Pareto distribution}

\subsubsection{The Wald distribution}

\subsubsection{The Weibull distribution}

\subsection{Continuous distributions supported on the real line}

\subsubsection{The Laplace distribution}

\subsubsection{The logistic distribution}

\subsubsection{The exreme value distribution}

\subsubsection{The hyperbolic secant distribution}

\subsubsection{The Cauchy distribution}

\subsection{Distributions associated with modeling Bernoulli trials}

\subsubsection{The Bernoulli distribution\label{subsec:bernoulliDistribution}}

The Bernoulli distribution is the simplest of all discrete probability
distributions. It represents the mathematical abstraction of coin
tossing with not-necessarily fair coins.
\begin{defn}
\label{def:bernoulliDistribution}A random variable $X$ on probability
space $\probabilityspace$ is said to have a Bernoulli distribution
with parameter $p\in\left[0,1\right]$ if $\P\left(X=1\right)=p$
and $\P\left(X=0\right)=1-p$.
\end{defn}

Formally, any indicator variable of an event $A\in\F$ is a Bernoulli
random variable. However, we typically reserve this description for
random variables used to model the outcome of a binary experiment
trial. Usually we are interested in the \emph{sequence }of such trials.
\begin{prop}
\label{prop:bernoulliMoments}The raw moments of a Bernoulli random
variable $X$ with parameter $p$ is given 
\[
\E\left[X^{k}\right]=p
\]
for any $k\in\N$. The central moments are
\[
\E\left[\left(X-p\right)^{k}\right]=\left(1-p\right)^{k}p+\left(1-p\right)\left(-p\right)^{k}
\]
\end{prop}

\begin{proof}
Note that 
\begin{align*}
\E\left[\left(X-p\right)^{k}\right] & =\left(1-p\right)^{k}\P\left(X=1\right)+\left(0-p\right)^{k}\P\left(X=0\right)\\
 & =\left(1-p\right)^{k}p+\left(1-p\right)\left(-p\right)^{k}.
\end{align*}
The raw moment case is simpler and follows exactly in the same way.
\end{proof}
\begin{prop}
\label{prop:mgfBernoulli}The moment generating function of a Bernoulli
random variable $X$ with parameter $p$ is given
\[
M_{X}\left(t\right)=1-p+pe^{t}.
\]
\end{prop}

\begin{proof}
Again follow the same approach as in \ref{prop:bernoulliMoments}.
\end{proof}

\subsubsection{The Binomial distribution\label{subsec:binomialDistribution}}

The Binomial distribution is perhaps the most well known of all discrete
distributions. It models the number of successes in a fixed number
of trials (say coin tosses) and as such is a defined as a sum of Bernoulli
trials.
\begin{defn}
\label{def:binomialDistribution}A random variable $Y$ has a Binomial
distribution with parameters $n$ and $p$ if 
\[
Y=\sum_{i=1}^{n}X_{i}
\]
where $X_{i}$ are independent and identically distributed Bernoulli
random variables with parameter $p$. In this case we write $Y\sim\textrm{Bin}\left(n,p\right)$
\end{defn}

\begin{prop}
\label{prop:binomialCDF}The mass function of a random variable $Y$
with a Binomial distribution with parameters $n$ and $p$ is given
by
\[
\P\left(Y=k\right)=\left(\begin{array}{c}
n\\
k
\end{array}\right)p^{k}\left(1-p\right)^{n-k}\indicate\left\{ 0\leq k\leq n\right\} 
\]
\end{prop}

\begin{proof}
For $n=2$, we use the result about convolutions. More specifically,
by Corollary \ref{cor:discreteConvolution}
\begin{align*}
\P\left(Y=k\right) & =\sum_{x\in\left\{ 0,1\right\} }\P\left(X_{1}=x\right)\P\left(X_{2}=k-x\right)\\
 & =\left(1-p\right)\left[p\indicate\left\{ k=1\right\} +\left(1-p\right)\indicate\left\{ k=0\right\} \right]\\
 & \ \ \ \!\ +p\left[p\indicate\left\{ k=2\right\} +\left(1-p\right)\indicate\left\{ k=1\right\} \right]\\
 & =p^{2}\indicate\left\{ k=2\right\} +2p\left(1-p\right)\indicate\left\{ k=1\right\} +\left(1-p\right)^{2}\indicate\left\{ k=0\right\} \\
 & =\left(\begin{array}{c}
2\\
k
\end{array}\right)p^{k}\left(1-p\right)^{2-k}\indicate\left\{ 0\leq k\leq2\right\} .
\end{align*}
Now for the induction step, assume that the result holds for $n$
and then we write $Y=\sum_{i=1}^{n+1}X_{i}=Z+X_{n+1}$ where $Z\sim\textrm{Bin}\left(n-1,p\right)$.
Then, another convolution argument shows
\begin{align*}
\P\left(Y=k\right) & =\sum_{z=0}^{n}\P\left(Z=z\right)\P\left(X_{n+1}=k-z\right)\\
 & =\sum_{z=0}^{n}\left(\begin{array}{c}
n\\
z
\end{array}\right)p^{z}\left(1-p\right)^{n-z}\indicate\left\{ 0\leq z\leq n\right\} \left(p\indicate\left\{ z=k-1\right\} +\left(1-p\right)\indicate\left\{ z=k\right\} \right)\\
 & =\sum_{z=0}^{n}\left(\begin{array}{c}
n\\
z
\end{array}\right)p^{z+1}\left(1-p\right)^{n-z}\indicate\left\{ 0\leq z\leq n,z=k-1\right\} \\
 & \ \ \!\ +\sum_{z=0}^{n}\left(\begin{array}{c}
n\\
z
\end{array}\right)p^{z}\left(1-p\right)^{n-z+1}\indicate\left\{ 0\leq z\leq n,z=k\right\} \\
 & =\left(\begin{array}{c}
n\\
k-1
\end{array}\right)p^{k}\left(1-p\right)^{n-k+1}+\left(\begin{array}{c}
n\\
k
\end{array}\right)p^{k}\left(1-p\right)^{n-k+1}\\
 & =\left(\begin{array}{c}
n+1\\
k
\end{array}\right)p^{k}\left(1-p\right)^{n+1-k}
\end{align*}
 where we used \hyperref[prop:pascalRule]{Pascal's rule} in the last
equality.
\end{proof}
The moments of the Binomial distribution are easily characterized
using the Bernoulli moments and the multinomial theorem
\begin{prop}
\label{prop:momentsBinomial}Let $\probabilityspace$ be a probability
space and let $Y\sim\mathrm{Bin}\left(n,p\right)$ for some $n\in\N$
and $p\in\left[0,1\right]$. Then,
\[
\E\left[Y^{k}\right]=\sum_{i_{1}+i_{2}+\ldots+i_{n}=k,i_{j}\geq0}\left(\begin{array}{cccc}
 & k\\
i_{1}, & i_{2}, & \ldots & i_{n}
\end{array}\right)p^{\sum_{j=1}^{n}\indicate\left\{ i_{j}\geq1\right\} }.
\]
\end{prop}

\begin{proof}
Note that $Y=\sum_{i=1}^{n}X_{i}$ where $X_{i}$ are i.i.d Bernoulli
random variables with parameter $p$. Then
\begin{align*}
\E\left[Y^{k}\right] & =\E\left[\left(\sum_{i=1}^{n}X_{i}\right)^{k}\right]\\
 & =\E\left[\sum_{i_{1}+i_{2}+\ldots+i_{n}=k,i_{j}\geq0}\left(\begin{array}{cccc}
 & k\\
i_{1}, & i_{2}, & \ldots & i_{n}
\end{array}\right)\prod_{j=1}^{n}X_{j}^{i_{j}}\right]\\
 & =\sum_{i_{1}+i_{2}+\ldots+i_{n}=k,i_{j}\geq0}\left(\begin{array}{cccc}
 & k\\
i_{1}, & i_{2}, & \ldots & i_{n}
\end{array}\right)\E\left[\prod_{j=1}^{n}X_{j}^{i_{j}}\right]\\
 & =\sum_{i_{1}+i_{2}+\ldots+i_{n}=k,i_{j}\geq0}\left(\begin{array}{cccc}
 & k\\
i_{1}, & i_{2}, & \ldots & i_{n}
\end{array}\right)\prod_{j=1}^{n}\E\left[X_{j}^{i_{j}}\right]\\
 & =\sum_{i_{1}+i_{2}+\ldots+i_{n}=k,i_{j}\geq0}\left(\begin{array}{cccc}
 & k\\
i_{1}, & i_{2}, & \ldots & i_{n}
\end{array}\right)p^{\sum_{j=1}^{n}\indicate\left\{ i_{j}\geq1\right\} }
\end{align*}
where in the second equality we have used the \hyperref[thm:multinomialTheorem]{multinomial theorem},
the third is linearity of integration, the fourth is Proposition \ref{prop:indepExpectationFactors},
and the last equality uses Proposition \ref{prop:bernoulliMoments}.
\end{proof}
The first moment is $\E\left[Y\right]=np$ which can be seen just
by linearity and the definition. The second moment is
\begin{align*}
\E\left[Y^{2}\right] & =\E\left[\sum_{i=1}^{n}X_{i}^{2}+\sum_{i=1}^{n}\sum_{j=1,i\neq j}^{n}X_{i}X_{j}\right]\\
 & =np+n\left(n-1\right)p^{2}.
\end{align*}
We are more interested in higher \emph{central }moments like the variance.
Of course, $\Var\left[Y\right]=\E\left[Y^{2}\right]-\left(\E\left[Y\right]\right)^{2}=np+n\left(n-1\right)p^{2}-n^{2}p^{2}=np-np^{2}=np\left(1-p\right).$
The moment generating function is can be computed in a similar way.
\begin{prop}
\label{prop:mgfBinomial}Let $\probabilityspace$ be a probability
space and let $Y\sim\mathrm{Bin}\left(n,p\right)$ for some $n\in\N$
and $p\in\left[0,1\right]$. Then,
\[
M_{Y}\left(t\right)=\left(1-p+pe^{t}\right)^{n}.
\]
\end{prop}

\begin{proof}
Note that $Y=\sum_{i=1}^{n}X_{i}$ where as before $X_{i}$ are iid
Bernoulli with parameter $p$ and so
\begin{align*}
M_{Y}\left(t\right) & =\E\left[e^{tY}\right]\\
 & =\E\left[e^{t\sum X_{i}}\right]\\
 & =\E\left[\prod_{i=1}^{n}e^{tX_{i}}\right]\\
 & =\prod_{i=1}^{n}\E\left[e^{tX_{i}}\right]\\
 & =\left(1-p+pe^{t}\right)^{n}
\end{align*}
wheree the fourth equality is again due to Proposition \ref{prop:indepExpectationFactors}
and the last is Proposition \ref{prop:mgfBernoulli}.
\end{proof}

\subsubsection{The geometric distribution}

\subsubsection{The negative binomial distribution}

\subsubsection{The multinomial distribution}

\subsubsection{The discrete arcine distribution}

\subsubsection{The Beta-binomial distribution}

\subsubsection{The Beta-negative binomial distribution}

\subsubsection{The discrete uniform distribution\label{subsec:discreteUniformDistribution}}

The discrete uniform distribution is the generalization of the distribution
we encountered in Example \ref{exa:floorFuncCDF}. The idea is that
it is a distribution supported on any finite set, where each element
of the set has equal probability. Thus for a finite set $A$ and a
discrete uniform random variable $X$ supported on $A$ we have that
$\P\left(X=k\right)=\frac{1}{\lvert A\rvert}\indicate\left\{ k\in A\right\} .$
Typically, the set $A=\left\{ a,a+1,a+2,\ldots,b\right\} $ in which
case the $\lvert A\rvert=b-a+1$. In the discussion below we will
focus on the special case where $a=0$ and $b=n-1$ for some $n\in\N$.
\begin{prop}
\label{prop:discreteUniformCDF}Let $\probabilityspace$ be a probability
space and suppose $X$ is a discrete uniform distribution supported
on $A=\left\{ 0,1,\ldots,n-1\right\} $. Then, the CDF of $X$ is
given
\[
F_{X}\left(x\right)=\frac{\lfloor x\rfloor+1}{n}\indicate\left\{ x\in\left[0,n-1\right]\right\} +\indicate\left\{ x>n-1\right\} .
\]
\end{prop}

\begin{proof}
Observe
\begin{align*}
F_{X}\left(x\right) & =\E\left[\indicate\left\{ X\leq x\right\} \right]\\
 & =\frac{1}{n}\sum_{a=0}^{n-1}\indicate\left\{ a\leq x\right\} \\
 & =\frac{\lfloor x\rfloor+1}{n}\indicate\left\{ x\in\left[0,n-1\right]\right\} +\indicate\left\{ x>n-1\right\} .
\end{align*}
\end{proof}
\begin{prop}
\label{prop:momentsDiscreteUniform}Let $\probabilityspace$ be a
probability space and suppose $X$ is a discrete uniform distribution
supported on $A=\left\{ 0,1,\ldots,n-1\right\} $. Then for any $k\in\N$
\[
\E\left[X^{k}\right]=\frac{1}{n}\sum_{a=0}^{n-1}a^{k}.
\]
In particular 
\[
\E\left[X\right]=\frac{1}{2}\left(n-1\right)
\]
and 
\[
\E\left[X^{2}\right]=\frac{\left(n-1\right)\left(2n-1\right)}{6}
\]
and so
\[
\Var\left[X\right]=\frac{n^{2}-1}{12}.
\]
\end{prop}

\begin{proof}
This is a standard application of Corollary \ref{cor:changeOfVariables}.
The specific formulas for the first and second moments follow from
partial sum formulas for consecutive integers and squares, respectively.
The variance then is given by $\Var\left[X\right]=\E\left[X^{2}\right]-\left(\E\left[X\right]\right)^{2}$.
\end{proof}
\begin{prop}
\label{prop:mgfDiscreteUniform}Let $\probabilityspace$ be a probability
space and let $X$ be a discrete uniform random variable supported
on $A=\left\{ 0,1,\ldots,n-1\right\} $. Then, the MGF of $X$ is
given
\[
M_{X}\left(t\right)=\frac{1-e^{nt}}{n\left(1-e^{t}\right)}\indicate\left\{ t\neq0\right\} +\indicate\left\{ t=0\right\} .
\]
\end{prop}

\begin{proof}
Observe that
\begin{align*}
M_{X}\left(t\right) & =\E\left[e^{tX}\right]\\
 & =\frac{1}{n}\sum_{a=0}^{n-1}e^{ta}\\
 & =\frac{1}{n}\sum_{a=0}^{n-1}\left(e^{t}\right)^{a}\\
 & =\frac{1-e^{nt}}{n\left(1-e^{t}\right)}\indicate\left\{ t\neq0\right\} +\indicate\left\{ t=0\right\} 
\end{align*}
by the geometric partial sum formula.
\end{proof}

\subsection{Distributions associated with finite sampling models}

\subsubsection{The hypergeometric distribution}

\subsubsection{The multivariate hypergeometric distribution}

\subsubsection{The matching distribution}

\subsubsection{The birthday distribution}

\subsubsection{The coupon collector distribution}

\subsubsection{The Polya distribution}

\subsection{Distributions associated with the Poisson process}

\subsubsection{The exponential distribution}

\subsubsection{The Erlang distribution}

\subsubsection{The Poisson distribution}

\end{document}
