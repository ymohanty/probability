
\chapter{Measurable functions\label{chap:measurableFunctions}}

\section{Limits of sets and their indicator functions}

Before we embark on a general description of measurable functions,
it's useful to look at a special kind of function: the indicator function
of a set. These functions are special, because they are essentially
the building blocks of all important functions in measure theory.
In fact, indicator functions of sets are the key to linking abstract
measure theory on one hand, to the theory of integration on the other.
As we shall later see, this link is actually an equivalence: measures
and integrals are equivalent objects, and so, in the context of this
theory, sets and their indicator functions are also in some sense
equivalent. While the full scope of this equivalence will only become
salient when we discuss integration, this section will shed some light
on why we perhaps should expect this \emph{ex-ante}.
\begin{defn}
\label{def:indicatorFunction}Let $\mathcal{X}$ be a set and let
$A\subseteq\mathcal{X}$ be an arbitrary subset. The function 
\[
\mathds{1}_{A}:\mathcal{X}\longrightarrow\left\{ 0,1\right\} 
\]
defined by
\[
\mathds{1}_{A}\left(x\right)=\begin{cases}
1 & x\in A\\
0 & x\notin A
\end{cases}
\]
is called the \emph{indicator function }of set $A$.
\end{defn}

The algebra of sets implies a corresponding Boolean algebra for indicator
functions.
\begin{fact}
\label{fact:indicatorFunctionsFiniteOperations}Let $A,B\subseteq\mathcal{X}$
be arbitrary and let $\mathds{1}_{A},\mathds{1}_{B}$ be their respective
indicator functions. Then the indicator function of the set $C:=A\cup B$
is given by
\[
\mathds{1}_{C}=\max\left\{ \mathds{1}_{A},\mathds{1}_{B}\right\} =\mathds{1}_{A}+\mathds{1}_{B}-\mathds{1}_{A}\mathds{1}_{B}
\]
where the maximum is taken pointwise. Similarly, the indicator function
for set $D:=A\cap B$ is given by
\[
\mathds{1}_{D}=\min\left\{ \mathds{1}_{A},\mathds{1}_{B}\right\} =\mathds{1}_{A}\mathds{1}_{B}.
\]
The indicator function for $A^{C}$ is given by
\[
\mathds{1}_{A^{C}}=1-\mathds{1}_{A}.
\]
\end{fact}

Note that if $A,B$ are disjoint, then the indicator function of their
union is simply the sum of their individual indicators, i.e.
\[
\mathds{1}_{A\cup B}=\mathds{1}_{A}+\mathds{1}_{B}.
\]
We can extend these facts to describe indicator functions of arbitrary
unions and intersections of sets in the obvious way
\begin{prop}
\label{prop:indicatorFunctionsArbitraryOperations}Let $\mathcal{I}$
be an arbitrary index set and let $\left\{ A_{i}\right\} _{i\in\mathcal{I}}\subseteq\mathcal{X}$
be subsets with indicator functions $\left\{ \mathds{1}_{A_{i}}\right\} _{i\in\mathcal{I}}$.
Then, the indicator function for $B:=\bigcup_{i\in\mathcal{I}}A_{i}$
is given by
\[
\mathds{1}_{B}=\sup_{i\in\mathcal{I}}\mathds{1}_{A_{i}}
\]
where the supremum is taken pointwise. Similarly, the indicator function
for $C:=\bigcap_{i\in\mathcal{I}}A_{i}$ is given by
\[
\mathds{1}_{C}=\inf_{i\in\mathcal{I}}\mathds{1}_{A_{i}}.
\]
\end{prop}

\begin{proof}
We provide the argument for $B$; the argument for $C$ is analagous.
Observe that
\begin{align*}
\mathds{1}_{B}\left(x\right)=1 & \Longleftrightarrow x\in\bigcup_{i\in\mathcal{I}}A_{i}\\
 & \Longleftrightarrow x\in A_{i_{0}}\textrm{ for some }i_{0}\in\mathcal{I}\\
 & \Longleftrightarrow\mathds{1}_{A_{i_{0}}}=1\text{ for some }i_{0}\in\mathcal{I}\\
 & \Longleftrightarrow\sup_{i\in\mathcal{I}}\mathds{1}_{A_{i}}=1
\end{align*}
which completes the argument.
\end{proof}
These arguments appear to be rather pedantic, but they are key to
defining limiting operations on sets. With a background in undergraduate
calculus, it can be quite cumbersome to think of a sequence of sets
converging to another set. However, it is quite straightforward to
imagine the pointwise convergence of a sequence of \emph{indicator
functions }of sets. For example, we have what appears to be a fairly
daunting definition for the limit of a sequence of sets.
\begin{defn}
\label{def:limSupInfSets} Let $\left\{ A_{i}\right\} _{i\in\N}$
be a sequence of sets in $2^{\mathcal{X}}$. Then the limit superior
of the sequence is given by
\[
\limsup_{n\to\infty}A_{n}:=\bigcap_{n=1}^{\infty}\bigcup_{i=n}^{\infty}A_{i}.
\]
Similarly, the limit inferior of the sequence is given by
\[
\liminf_{n\to\infty}A_{n}:=\bigcup_{n=1}^{\infty}\bigcap_{i=n}^{\infty}A_{i}.
\]
If $\limsup_{n\to\infty}A_{n}=\liminf_{n\to\infty}A_{n}$ then the
limit of the sequence is defined and is equal to the limit superior
and inferior i.e.
\[
\lim_{n\to\infty}A_{n}:=\limsup_{n\to\infty}A_{n}=\liminf_{n\to\infty}A_{n}.
\]

While these definitions appear arbitrary, they demarcate important
concepts in both analysis and probability. To unpack the intuition,
let's try to understand what it means for an element $x\in\mathcal{X}$
to be in $\limsup_{n\to\infty}A_{n}$. If $x\in\bigcap_{n=1}^{\infty}\bigcup_{i=n}^{\infty}A_{i}$,
then $x\in\bigcup_{i=n}^{\infty}A_{i}$ for every $n\in\N$. That
it is to say, for any $n\in\N$, there exists an $i\geq n$ such that
$x\in A_{i}$. This essentially says that $x$ is in infinitely many
of the sets $\left\{ A_{i}\right\} _{i\in\N}$. In the language of
probability, the event $\limsup_{n\to\infty}A_{n}$ is the event of
outcomes that occur infinitely often in the collection of events $\left\{ A_{i}\right\} _{i\in\N}.$

On the other hand, if $x\in\liminf_{n\to\infty}A_{n},$then there
exists some $n_{0}\in\N$ such that $x\in A_{i}$ for every $i\geq n_{0}$.
Clearly then, $\liminf_{n\to\infty}A_{n}\subseteq\limsup_{n\to\infty}A_{n}$
which mirrors the domination condition for limit superiors and inferiors
of sequences of real numbers or real functions. So when does equality
hold? Note that if $\left\{ A_{i}\right\} _{i\in\N}$ is an increasing
sequence of sets
\begin{align*}
\limsup_{n\to\infty}A_{n} & =\bigcap_{n=1}^{\infty}\bigcup_{i=n}^{\infty}A_{i}\\
 & =\bigcap_{n=1}^{\infty}\bigcup_{i=1}^{\infty}A_{i}\\
 & =\bigcup_{i=1}^{\infty}A_{i}\\
 & =\bigcup_{i=1}^{\infty}\bigcap_{n=i}^{\infty}A_{n}\\
 & =\liminf_{i\to\infty}A_{i}
\end{align*}
where the second and fourth equalitiies follow from the increasing
nature of the sets $A_{i}$. This shows that the continuity from below
condition described in Proposition \ref{prop:measureProperties} is
in fact bona-fide continuity. After developing the theory of integration,
we will (seemingly) generalize this continuity result to measurable
functions in the form of the famous \hyperref[thm:monotoneConvergenceLebInt]{\emph{monotone convergence theorem}}
. Of course, once we know that measures and integrals are essentially
the same objects, it will be clear that continuity from below and
monotone convergence are two sides of the same coin. While this result
is better known in it's integral formulation, there's another result
that is perhaps better known in its measure-theoretic formulation:
the Borel-Cantelli lemma.
\end{defn}

\begin{thm}[First Borel-Cantelli lemma]
\label{thm:borelCantelli}Let $\left(\mathcal{X},\mathcal{F},\mu\right)$
be an arbitrary measure space and let $\left\{ A_{i}\right\} _{i\in\N}\in\mathcal{F}$
be a sequence of sets. If
\[
\sum_{i=1}^{\infty}\mu\left(A_{i}\right)<\infty
\]
then
\[
\mu\left(\limsup_{i\to\infty}A_{i}\right)=0.
\]
\end{thm}

\begin{proof}
Define by $B_{n}:=\bigcup_{i=n}^{\infty}A_{i}$. It's clear that $\left\{ B_{n}\right\} _{n\in\N}$
is a decreasing sequence of sets. More over $\mu\left(B_{1}\right)=\mu\left(\bigcup_{i\in\N}A_{i}\right)\leq\sum_{i=1}^{\infty}\mu\left(A_{i}\right)<\infty$
by \hyperref[cor:countableSubadditivity]{subadditivity} and so, since
$\mu$ is finite on $\left\{ B_{n}\right\} _{n\in\N}$, we can apply
Propositions \ref{prop:measureProperties} and \ref{prop:equivalenceContinuityMeasures}
to establish
\begin{align*}
\mu\left(\bigcap_{n=1}^{\infty}B_{n}\right) & =\lim_{n\to\infty}\mu\left(B_{n}\right)\\
 & =\lim_{n\to\infty}\text{\ensuremath{\mu\left(\bigcup_{i=n}^{\infty}A_{i}\right)}}\\
 & \leq\lim_{n\to\infty}\sum_{i=n}^{\infty}\mu\left(A_{i}\right)\\
 & =\lim_{n\to\infty}\left[\sum_{i=1}^{\infty}\mu\left(A_{i}\right)-\sum_{i=1}^{n-1}\mu\left(A_{i}\right)\right]\\
 & =0
\end{align*}
where the inequality follows from subadditivity and the last equality
is due to the the assumption that $\sum_{i\in\N}\mu\left(A_{i}\right)<\infty$
along with the fact that a sequence and its tail have the same limit.
\end{proof}
\begin{rem*}
This version of the Borel-Cantelli lemma is sometimes called the \emph{first
}Borel-Cantelli lemma since its converse, which is true under certain
conditions, is also called the Borel-Cantelli lemma in the literature.
To prevent ambiguity, we refer to the conversee result as the \emph{second
}Borel-Cantelli lemma. The second Borel-Cantelli lemma uses the probabilistic
concept of independence and is covered in Theorem\ref{thm:secondBorelCantelli}
in the chapter on independence and as such, we will relegate the discussion
of the second Borel-Cantelli lemma to when we formally delve into
probabiility theory in the second part of these notes.
\end{rem*}
By now you should be sufficiently convinced that our definitions of
the limiting behavior sets indeed make sense. However, if you any
doubts, our treatment of indicator functions should help resolve them
completely
\begin{prop}
\label{prop:limSupInfIndicator}Let $\left\{ A_{i}\right\} _{i\in\N}$
be a collection of subsets of $\mathcal{X}$ and let $\left\{ \indicate_{A_{i}}\right\} _{i\in\N}$
be their corresponding indicator functions. Then
\[
\limsup_{n\to\infty}\indicate_{A_{n}}=\indicate_{\limsup_{n\to\infty}A_{n}}
\]
and
\[
\liminf_{n\to\infty}\indicate_{A_{n}}=\indicate_{\liminf_{n\to\infty}A_{n}}.
\]
\end{prop}

\begin{proof}
We prove the first assertion as the second one follows by the same
argument. Note that
\begin{align*}
\limsup_{n\to\infty}\indicate_{A_{n}} & =\inf_{n\in\N}\left\{ \sup_{i\geq n}\indicate_{A_{i}}\right\} \\
 & =\inf_{n\in\N}\left\{ \indicate_{\bigcup_{i\geq n}A_{i}}\right\} \\
 & =\indicate_{\bigcap_{n\in\N}\bigcup_{i\geq n}A_{i}}\\
 & =\indicate_{\limsup_{n\to\infty}A_{n}}
\end{align*}
where the second and third equalities follow by Proposition \ref{prop:indicatorFunctionsArbitraryOperations}.
\end{proof}
By now, you should have begun to appreciate that indicator functions
are essentially functional equivalents of the sets they indicate:
sets and their indicator functioins are just two representations of
the same object. This is not particularly suprising, given that sets
are defined by their membership and indicator functions describe membership.
In this context, it should then not be suprising that indicator functions
of \emph{measurable sets }are \emph{measurable functions},\emph{ }even
though we have not yet described the latter concept yet. This is indeed
true, and moreover, any non-negative measurable function can be built
by taking a limit of a linear combination of indicators of sets. But
before we can show this, we should first define what a measurable
function is!
\begin{defn}
\label{def:measurableFunction}Let $\left(\X,\mathcal{F}\right)$
and $\left(\mathcal{Y},\mathcal{G}\right)$ be two measurable spaces.
A function 
\[
f:\mathcal{X}\longrightarrow\mathcal{Y}
\]
is called \emph{$\mathcal{F}/\mathcal{G}-$measurable }if for any
$G\in\mathcal{G}$
\[
f^{-1}\left[G\right]\in\mathcal{F}.
\]
\end{defn}

\begin{rem*}
This definition also resembles a continuity condition; indeed, if
$\mathcal{F}$ and $\mathcal{G}$ were topologies rather than $\sigma-$algebras,
this would be the definition of a continuous function. It turns out
that if $\mathcal{F}$ and $\mathcal{G}$ are Borel $\sigma-$algebras,
then continuity implies measurability: this is a fact that we establish
in the next section. Moreover, all measurable functions are in some
sense \emph{almost }continuous: we make this notion precise when we
discuss the deep connections between topology and measure theory in
Chapter \ref{chap:measureAndTopology}.
\end{rem*}
Later in these notes, we will stop writing $\mathcal{F}/\mathcal{G}$
explicitly and let the reader infer the $\sigma-$algebras in play
from the context.

\section{Properties of measurable functions}

Armed with our definition of measurable functions, we are ready to
discuss interesting examples of such functions along with their properties.
First, we establish that the measurability of a sets and its indicator
function is indeed equivalent, as we had guessed earlier
\begin{prop}
\label{prop:measurableSetsFunctions}Let $\left(\X,\mathcal{F}\right)$
be a measurable space. Then for any $A\subseteq\mathcal{X}$, $A$
is measurable (i.e. $A\in\mathcal{F}$) if and only if 
\[
\indicate_{A}:\mathcal{X}\longrightarrow\left\{ 0,1\right\} 
\]
is $\mathcal{F}/2^{\left\{ 0,1\right\} }-$measurable.
\end{prop}

\begin{proof}
First assume that $A\in\mathcal{F}$ and observe that if $B=\left\{ 0,1\right\} $
then $\indicate_{A}^{-1}\left[B\right]=\X\in\mathcal{F}$, if $B=\{1\}$
then $\indicate_{A}^{-1}\left[B\right]=A\in\mathcal{F}$, if $B=\left\{ 0\right\} $
then $\indicate_{A}^{-1}\left[B\right]=A^{C}\in\mathcal{F}$, and
if $B=\emptyset$ then $\indicate_{A}^{-1}\left[B\right]=\emptyset\in\mathcal{F}$.

Conversely, assume that $\indicate_{A}$ is measurable and notice
how $A=f^{-1}\left[\left\{ 1\right\} \right]\in\mathcal{F}$ which
completes the proof.
\end{proof}
\begin{cor}
\label{cor:borelMeasurableSetsFunctions}Let $\left(\X,\F\right)$
be a measurable space. For any $A\subseteq\X$, $A\in\F$ if and only
if
\[
\indicate_{A}:\X\longrightarrow\R
\]
is $\F/\borel\left(\R\right)-$measurable.
\end{cor}

\begin{proof}
First assume that $A\in\F$ and observe that for any set $B\in\borel\left(\R\right)$
\begin{align*}
\indicate_{A}^{-1}\left[B\right] & =\indicate_{A}^{-1}\left[\left(B\setminus\left\{ 0,1\right\} \right)\bigcup\left(B\cap\left\{ 0,1\right\} \right)\right]\\
 & =\indicate_{A}^{-1}\left[B\setminus\left\{ 0,1\right\} \right]\bigcup\indicate_{A}^{-1}\left[B\cap\left\{ 0,1\right\} \right]\\
 & =\indicate_{A}^{-1}\left[B\cap\left\{ 0,1\right\} \right]\in\F
\end{align*}
where the second equality follows by the property of preimages and
the last equality follows by the fact that $\indicate_{A}^{-1}\left[B\setminus\left\{ 0,1\right\} \right]=\emptyset$.
The inclusion on the last line then is a consequence of Proposition
\ref{prop:measurableSetsFunctions}.

Conversely, assume that $\indicate_{A}$ is measurable and observe
that since $\left\{ 1\right\} \in\borel\left(\R\right)$, the results
follows trivially.
\end{proof}
In this case, measurability of our function was easy to establish
because the $\sigma-$algebra $2^{\left\{ 0,1\right\} }$ could be
explicitly enumerated. Generally, this is not possible as $\sigma-$algebras
can be extremely large. Nevertheless, it is possible establish measurability
using a smaller class of sets in the target $\sigma-$algebra; this
is the crux of generating class arguments which we discusss more abstractly
in the next section. Even the simplest of such arguments can be quite
powerful, as we shall see with the following result.
\begin{thm}[Generic generating class argument]
\label{thm:genericGeneratingClassArgument}Let $\left(\X,\mathcal{F}\right)$
and $\left(\mathcal{Y},\mathcal{G}\right)$ be measurable spaces and
let $\mathcal{E}\subseteq\mathcal{G}$ be a collection of sets such
that $\sigma\left(\mathcal{E}\right)=\mathcal{G}.$ A function
\[
f:\X\longrightarrow\mathcal{Y}
\]
is $\mathcal{F}/\mathcal{G}-$measurable if and only if 
\[
f^{-1}\left[E\right]\in\mathcal{F}
\]
for every $E\in\mathcal{E}.$
\end{thm}

\begin{proof}
If $f$ is measurable, then by definition, $f^{-1}\left[E\right]\in\mathcal{F}$
for every $E\in\mathcal{E}$ since $\mathcal{E}\subseteq\mathcal{G}$.
Conversely, suppose that $f^{-1}\left[E\right]\in\mathcal{F}$ for
every $E\in\mathcal{E}$ and define 
\[
\mathcal{D}=\left\{ G\in\mathcal{G}\mid f^{-1}\left[G\right]\in\mathcal{F}\right\} .
\]
By assumption, $\mathcal{E}\subseteq\mathcal{D}$ and with a little
effort we can show that $\mathcal{D}$ is in fact a $\sigma-$algebra,
which then shows that $\mathcal{G}=\sigma\left(\mathcal{E}\right)\subseteq\sigma\left(\mathcal{D}\right)=\mathcal{D}.$
First, it is clear that $\emptyset\in\mathcal{D}$ as $f^{-1}\left[\emptyset\right]=\emptyset\in\mathcal{F}$.
Next, for any $A\in\mathcal{D}$, observe that $f^{-1}\left[A^{c}\right]=\left(f^{-1}\left[A\right]\right)^{C}\in\mathcal{F}$
since $\mathcal{F}$ is a $\sigma-$algebra. Finally, for any collection
$\left\{ A_{i}\right\} _{i\in\N}\in\mathcal{D},$ $f^{-1}\left[\bigcup_{i\in\N}A_{i}\right]=\bigcup_{i\in\N}f^{-1}\left[A_{i}\right]\in\mathcal{F}$
again because $\mathcal{F}$ is a $\sigma-$algebra. This completes
the proof.
\end{proof}
Now we can show that continuous functions between two spaces equipped
with Borel $\sigma-$algebras are indeed measurable.
\begin{cor}
\label{cor:continuousFunctionsAreMeasurable}Let $\left(\X,\borel\left(\X\right)\right)$
and $\left(\mathcal{Y},\borel\left(\mathcal{Y}\right)\right)$ be
measurable spaces and let 
\[
f:\mathcal{X}\longrightarrow\mathcal{Y}
\]
be continuous. Then $f$ is $\borel\left(\X\right)/\borel\left(\mathcal{Y}\right)-$measurable.
\end{cor}

\begin{proof}
Let $\mathcal{O}_{\mathcal{Y}}$ be the topology on $\mathcal{Y}$.
By definition
\[
\sigma\left(\mathcal{O}_{\mathcal{Y}}\right)=\borel\left(\mathcal{Y}\right)
\]
and by continuity, for any open set $O\in\mathcal{O}_{\mathcal{Y}}$
\[
f^{-1}\left[O\right]\in\mathcal{O}_{\mathcal{X}}\subseteq\borel\left(\X\right)
\]
where $\mathcal{O}_{\mathcal{X}}$ is the topology on $\X$. Thus
by Theorem \ref{thm:genericGeneratingClassArgument}, $f$ is measurable.
\end{proof}
Next we can use the generic generating class argument to establish
routine properties of measurable functions. First, we prove the following
useful lemma, which is an important property on its own.
\begin{lem}
\label{lem:compositionMeasurableFunctions}Let $\left(\X,\mathcal{F}\right),\left(\mathcal{Y},\mathcal{G}\right),$
and $\left(\mathcal{Z},\mathcal{H}\right)$ be measurable spaces.
If the functions
\begin{align*}
f & :\X\longrightarrow\mathcal{Y}\\
g & :\mathcal{Y}\longrightarrow\mathcal{Z}
\end{align*}
are $\mathcal{F}/\mathcal{G}$ and $\mathcal{G}/\mathcal{H}-$measurable
respectively, then the composition function $\phi:=f\circ g$ is $\mathcal{F}/\mathcal{H}-$measurable.
\end{lem}

\begin{proof}
Let $H\in\mathcal{H}$ be arbitrary and note that
\begin{align*}
\phi^{-1}\left[H\right] & =f^{-1}\left[g^{-1}\left[H\right]\right]\\
 & =f^{-1}\left[G\right]\\
 & =F\in\mathcal{F}
\end{align*}
where $G:=g^{-1}\left[H\right]\in\mathcal{G}$ since $g$ is measurable.
\end{proof}
\begin{prop}
\label{prop:binaryOperationsMeasurableFunctions} Let $\left(\X,\mathcal{F}\right)$
be a measurable space and let $f,g$ be real-valued Borel-measurable
functions on $\X$; that is $f,g:\X\longrightarrow\R$ and are $\mathcal{F}/\borel\left(\R\right)-$measurable.
Define $T:\X\longrightarrow\R^{2}$ as
\[
T\left(x\right):=\left(\begin{array}{c}
f\left(x\right)\\
g\left(x\right)
\end{array}\right)
\]
where $\R^{2}$ is equipped with its Borel $\sigma-$algebra $\borel\left(\R^{2}\right).$
Finally, let the function $\psi:\R^{2}\longrightarrow\R$ be continuous
with respect to the standard topologies. Then the function 
\begin{align*}
h & :=\psi\circ T:\X\longrightarrow\R
\end{align*}
is Borel measurable.
\end{prop}

\begin{proof}
First, note that by Corollary \ref{cor:continuousFunctionsAreMeasurable}
and Lemma \ref{lem:compositionMeasurableFunctions}, we only need
to prove the measurability of $T$ in order to deduce the measurability
of $h$. Let $R$ be an open rectangle in $\R^{2}$ i.e. $R=I_{1}\times I_{2}$
where
\[
I_{j}=\left(a_{j},b_{j}\right)
\]
for $a_{j}>b_{j}\in\R$. Then consider 
\begin{align*}
T^{-1}\left[R\right] & =\left\{ x\in\X\mid T\left(x\right)\in R\right\} \\
 & =\left\{ x\in\X\mid\left(f\left(x\right),g\left(x\right)\right)\in I_{1}\times I_{2}\right\} \\
 & =\left\{ x\in\X\mid f\left(x\right)\in I_{1}\text{ and }g\left(x\right)\in I_{2}\right\} \\
 & =\left\{ x\in\X\mid f\left(x\right)\in I_{1}\right\} \bigcap\left\{ x\in\X\mid g\left(x\right)\in I_{2}\right\} \\
 & =f^{-1}\left[I_{1}\right]\bigcap g^{-1}\left[I_{2}\right]\in\mathcal{F}
\end{align*}
due to the measurability of $f,g$ and the fact that $\sigma-$algebras
are closed under intersection.

Let $\mathcal{R}$ denote the collection of all open rectanges in
$\R^{2}$. Since $\mathcal{R}$ is a subset of all open sets in $\R^{2}$,
we know that $\sigma\left(\mathcal{R}\right)\subseteq\borel\left(\R^{2}\right).$
To deduce the converse inclusion, recall that a small modification
of Lemma \ref{lem:openSetDisjointUnionInterval} will show that any
open set in $\R^{2}$ can be written as a countable union of disjoint
sets in $\mathcal{R}$ i.e. for any open set $O\subseteq\R^{2}$
\[
O=\bigcup_{i\in\N}R_{i}
\]
where $R_{i}\in\mathcal{R}.$ This implies that $O\in\sigma\left(\mathcal{R}\right)$
and so $\borel\left(\R^{2}\right)\subseteq\sigma\left(\mathcal{R}\right).$
Applying a \hyperref[thm:genericGeneratingClassArgument]{generating class argument}
with $\mathcal{R}$ then shows that $T$ is measurable.
\end{proof}
\begin{cor}
\label{cor:examplesBinaryOpsMeasFunc}For any real-valued Borel-measurable
functions $f,g$ on $\left(\X,\mathcal{F}\right)$, the following
functions are also measurable

\begin{enumerate}[label=(\roman*),leftmargin=.1\linewidth,rightmargin=.4\linewidth]
	\item $ h(x) := f(x) + g(x) $
	\item $ h(x) := f(x)g(x) $
	\item $ h(x) := f(x)/g(x)\ \mathrm{where} \ g(x)\neq 0 $
\end{enumerate}
\end{cor}

\begin{proof}
For (i), let $\psi\left(x,y\right):=x+y$ (which is a continuous function)
and apply Proposition \ref{prop:binaryOperationsMeasurableFunctions}.
The other cases are simiilar.
\end{proof}
So we know that measurabiltiy is preserved under addition and multiplication,
but the principal concept in analysis is the limit, and we would like
measurability of functions to be preserved under limiting operations.
The following results help us establish that measurability of real-valued
functions is indeed preserved under pointwise limits, when one exists.
\begin{lem}
\label{lem:collectionIntervalsMeasurable}The following sets are generating
classes for $\borel\left(\R\right):$
\[
\left\{ \left(a,\infty\right)\mid\forall a\in\R\right\} ,\ \left\{ \left[a,\infty\right)\mid\forall a\in\R\right\} ,\ \left\{ \left(-\infty,b\right)\mid\forall b\in\R\right\} ,\ \left\{ \left(-\infty,b\right]\mid\forall a\in\R\right\} .
\]
\end{lem}

\begin{proof}
\hyperref[prop:sigmaAlgebraGeneratedbyLisBorel]{Recall} that the
collection of half-open intervals $\mathcal{L}=\left\{ \left(a,b\right]\mid-\infty<a\leq b<\infty\right\} $
is a generating class for $\borel\left(\R\right).$ Observe that for
any $a\leq b\in\R$,
\[
\left(a,b\right]=\bigcap_{n\in\N}\left(\left(a,\infty\right)\cap\left(b+\frac{1}{n},\infty\right)\right)
\]
and so $\mathcal{L}\subseteq\sigma\left(\left\{ \left(a,\infty\right)\mid\forall a\in\R\right\} \right)$
which implies that $\borel\left(\R\right)\subseteq\sigma\left(\left\{ \left(a,\infty\right)\mid\forall a\in\R\right\} \right).$
To see the reverse inclusion, note that for any $a\in\R$
\[
\left(a,\infty\right)=\bigcup_{b\in\N}\left(a,b\right]
\]
which shows that $\sigma\left(\left\{ \left(a,\infty\right)\mid\forall a\in\R\right\} \right)\subseteq\borel\left(\R\right)$.
The other sets can be shown to be generators in a similar fashion.
\end{proof}
\begin{prop}
\label{prop:supInfMeasurable}For a sequence of real-valued Borel-measurable
functions $\left\{ f_{n}\right\} _{n\in\N}$ on $\left(\X,\mathcal{F}\right)$,
the functions
\[
g:=\sup_{n\in\N}f_{n}
\]
and 
\[
h:=\inf_{n\in\N}f_{n}
\]
are $\mathcal{F}/\borel\left(\overline{\R}\right)$ measurable where
$\overline{\R}$ represents the extended real line.
\end{prop}

\begin{proof}
Let $a\in\R$ be arbitrary. Note that the set
\[
g^{-1}\left[\left(a,\infty\right)\right]=\left\{ x\in\X\mid g\left(x\right)>a\right\} =\bigcup_{n\in\N}\left\{ x\in\X\mid f_{n}\left(x\right)>a\right\} =\bigcup_{n\in\N}f_{n}^{-1}\left[\left(a,\infty\right)\right]
\]
since if $g\left(x\right)>a$ then there's at least one $n\in\N$
such that $f_{n}\left(x\right)>a$. Note that by the measurability
of $f_{n}$, $f_{n}^{-1}\left[\left(a,\infty\right)\right]\in\mathcal{F}$
for every $n\in\N.$ Since $\mathcal{F}$ is a $\sigma-$algebra,
$g^{-1}\left[\left(a,\infty\right)\right]\in\mathcal{F}$ by closure
under countable unions. This establishes the measurability of $g$
by Lemma \ref{lem:collectionIntervalsMeasurable}. A similar argument
establishes the measurability of $h$.
\end{proof}
\begin{cor}
\label{cor:limSupLimInfMeasurable}For a sequence of real-valued Borel-measurable
functions $\left\{ f_{n}\right\} _{n\in\N}$ on $\left(\X,\mathcal{F}\right)$,
the functions
\[
g:=\limsup_{n\to\infty}f_{n}
\]
and 
\[
h:=\liminf_{n\to\infty}f_{n}
\]
are measurable (provided they exist) and if $h=g$ then 
\[
\lim_{n\to\infty}f_{n}=h=g
\]
is measurable.
\end{cor}

\begin{proof}
Recall that 
\[
g=\limsup_{n\to\infty}f_{n}=\inf_{n\in\N}\left\{ \sup_{k\geq n}f_{k}\right\} 
\]
and apply Proposition \ref{prop:supInfMeasurable}. The other results
follow in the same fashion.
\end{proof}
Of course, we are often more interested in establishing the measurability
of vector-valued function i.e functions whose range is some subset
of the Euclidean space $\R^{n}$. This limit theorems for real-valued
functions proved here extend naturally to arbitrary finite dimensional
spaces.
\begin{prop}
Let $\left(\X,\mathcal{F}\right)$ be a measurable space and let the
sequence of functions $\left\{ f_{m}\right\} _{m\in\N}$
\[
f_{m}:\X\longrightarrow\R^{n}
\]
be $\mathcal{F}/\borel\left(\R^{n}\right)-$measurable. The pointwise
limit function $f:=\lim_{m\to\infty}f_{m}$ (if it exists) is $\mathcal{F}/\borel\left(\R^{n}\right)-$measurable
if and only if the projection functions $f_{1},\ldots,f_{n}$ are
$\mathcal{F}/\borel\left(\R\right)-$measurable.
\end{prop}

\begin{proof}
First assume that the projections $f_{1},\ldots,f_{n}$ are measurable.
Then, applying the same proof we used to prove the measurability of
the function $T$ in Proposition \ref{prop:binaryOperationsMeasurableFunctions},
we have that
\[
f=\left(\begin{array}{c}
f_{1}\\
\vdots\\
f_{n}
\end{array}\right)
\]
is measurable.

Conversely, suppose that $f$ is measurable, then recall that the
projection functions $\pi_{k}:\R^{n}\longrightarrow\R$ given by
\[
\pi_{k}\left(\begin{array}{c}
x_{1}\\
\vdots\\
x_{k}\\
\vdots\\
x_{n}
\end{array}\right)=x_{k}
\]
is a continuous function and hence measurable for any $1\leq k\leq n$.
Applying Lemma \ref{lem:compositionMeasurableFunctions}, we have
that $f_{k}=\pi_{k}\circ f$ is measurable for any $1\leq k\leq n.$
\end{proof}
In general, it appears that a measurability is preserved under a wide
range of operations on functions. Additionally, the richness of commonly
seen $\sigma-$algebras makes it the case that almost all functions
we encounter in everyday mathematics are measurable. This heuristic
is a fine one, but it begs the question: precisely how rich does the
$\sigma-$algebra of the domain have to be for a given function to
be measurable? This has a relatively straightforward answer.
\begin{prop}
\label{prop:sigmaAlgebraGeneratedByFunction}Let $\X$ be a set and
let $\left(\mathcal{Y},\mathcal{G}\right)$ be a measurable space.
Let
\[
T:\X\longrightarrow\mathcal{Y}
\]
be a function. Then the ``smallest'' $\sigma-$algebra on $\mathcal{X}$
that makes $T$ measurable is 
\[
\sigma\left(T\right):=\left\{ T^{-1}\left[G\right]\mid G\in\mathcal{G}\right\} .
\]
That is, $\sigma\left(T\right)$ is the intersecion of all $\sigma-$algebras
on $\mathcal{X}$ that makes T measurable.
\end{prop}

\begin{proof}
Let $\mathcal{R}$ be the collection of all $\sigma-$algebras on
$\X$ that makes $T$ measurable. Clearly, $\sigma\left(T\right)\subseteq R$
for every $R\in\mathcal{\mathcal{R}}$ by the definition of measurability.
Thus 
\[
\sigma\left(T\right)\subseteq\bigcap_{R\in\mathcal{R}}R
\]
and so all we have to do is show that $\sigma\left(T\right)\in\mathcal{\mathcal{R}}$
to show the reverse inclusion. To do this, we simply show that $\sigma\left(T\right)$
is a $\sigma-$algebra.

Note that $\emptyset=T^{-1}\left[\emptyset\right]\in\sigma\left(T\right).$
Next, let $A\in\sigma\left(T\right)$ be arbitrary and observe that
$A=T^{-1}\left[G\right]$ for some $G\in\mathcal{G}$. Then $A^{C}=\left(T^{-1}\left[G\right]\right)^{C}=T^{-1}\left[G^{C}\right]\in\sigma\left(T\right)$
as $G^{C}\in\mathcal{G}$ since $\mathcal{G}$ is a $\sigma-$algebra.
Finally, suppose $\left\{ A_{i}\right\} _{i\in\N}\in\sigma\left(T\right)$
and note that $A_{i}=T^{-1}\left[G_{i}\right]$ for $G_{i}\in\mathcal{G}$
and so
\[
\bigcup_{i\in\N}A_{i}=\bigcup_{i\in\N}T^{-1}\left[G_{i}\right]=T^{-1}\left[\bigcup_{i\in\N}G_{i}\right]\in\sigma\left(T\right)
\]
 since $\mathcal{G}$ is closed under countable unions. This completes
the proof.
\end{proof}
\begin{cor}
\label{cor:generatorPreimage}Let $\X$ be a set and let $\left(\mathcal{Y},\mathcal{G}\right)$
be a measurable space. Let $\mathcal{H}\subset\mathcal{G}$ be a generating
class such that $\sigma\left(\mathcal{H}\right)=\mathcal{G}$. Then
for a function $T:\mathcal{X}\to\mathcal{Y}$
\[
\sigma\left(T^{-1}\left(\mathcal{H}\right)\right)=\sigma\left(T\right).
\]
\end{cor}

\begin{proof}
Note that $T^{-1}\left(\mathcal{H}\right)\subseteq\sigma\left(T\right)$
and so $\sigma\left(T^{-1}\left(\mathcal{H}\right)\right)\subseteq\sigma\left(T\right)$.
Conversely, note that $T$ is $\sigma\left(T^{-1}\left(\mathcal{H}\right)\right)/\mathcal{G}$
measurable by the generating class argument in Theorem \ref{thm:genericGeneratingClassArgument}.
But since by Proposition \ref{prop:sigmaAlgebraGeneratedByFunction},$\sigma\left(T\right)$
is the smallest $\sigma-$algebra on which $T$ is measurable, we
have that $\sigma\left(T\right)\subseteq\sigma\left(T^{-1}\left(\mathcal{H}\right)\right)$.
\end{proof}
\begin{prop}
\label{prop:sigmaFunctionOfMeasurableFunctionSubset}Let $\left(\X,\mathcal{F}\right)$
, $\left(\mathcal{Y},\mathcal{G}\right),$ and $\left(\mathcal{Z},\mathcal{H}\right)$
be measurable spaces and let $f:\X\to\mathcal{Y}$ be $\mathcal{F}/\mathcal{G}$
measurable. Then for any $\mathcal{G}/\mathcal{H}$ measurable function
$g:\mathcal{Y}\to\R$
\[
\sigma\left(g\circ f\right)\subseteq\sigma\left(f\right).
\]
\end{prop}

\begin{proof}
Note that for any $H\in\mathcal{H}$, $\left(g\circ f\right)^{-1}\left[H\right]=f^{-1}\left[g^{-1}\left[H\right]\right]\in\sigma\left(f\right)$
which implies the claim.
\end{proof}
Note that $\sigma-$algebras generated by measurable functions constitute
an important subclass of all $\sigma-$algebras.
\begin{prop}
\label{prop:countablyGeneratedSigmaAlgebrasMeasFunc}Let $\X$ be
a set and let $\{B_{i}\}_{i\in\N}\subseteq\X$ be a countable collection
of sets. Then there exists a Borel-measurable function $f:\X\to\R$
such that 
\[
\sigma\left(\{B_{i}\}_{i\in\N}\right)=\sigma\left(f\right).
\]
\end{prop}


\section{Constructing measurable functions}

\subsection{Simple functions}

In the beginning of this chapter, we foreshadowed how the indicator
functions of measurable sets were in fact the building blocks of all
real-valued non-negative measurable functions; now we have developed
just enough of the theory to show that this is true. First, we should
introduce some useful notation to save space in the future. We denote
by $\mathcal{M}\left(\X,\mathcal{F}\right)$ the set of (extended)
real-valued Borel-measurable functions on $\left(\X,\mathcal{F}\right)$.
We write $\mathcal{M}^{+}\left(\X,\mathcal{F}\right)\subset\mathcal{M}\left(\X,\mathcal{F}\right)$
to denote the set of all non-negative (extended) real-valued Borel-measurable
functions on $\left(\X,\mathcal{F}\right)$. Note that the properties
we described for the set of real-valued measurable functions on $\left(\X,\F\right)$
in the previous section carry over to the set of extended real valued
measurable functions, as long as no weird $\infty-\infty$ or $\infty\cdot0$
situations arise.
\begin{defn}
\label{def:simpleFunction} A function $s:\mathcal{X}\longrightarrow\left[0,\infty\right)$
is called a \emph{simple function }if its range is a finite subset
of $\left[0,\infty\right)$ It can be represented as an aggregation
of indicator functions as
\[
s=\sum_{i=1}^{k}\alpha_{i}\indicate_{A_{i}}
\]
where $\left\{ \alpha_{i}\right\} _{i=1}^{k}\in\left[0,\infty\right)$
is the range of $s$ and $A_{i}:=\left\{ x\in\X\mid s\left(x\right)=\alpha_{i}\right\} $
partitions the domain $\X$ into preimages of the singletons in the
range. Such a representation is called the \emph{standard representation
}of $s$
\end{defn}

\begin{rem*}
Note that in general, a simple function could be written as a finite
linear combination of indicator functions in more than one way. For
example, the function $s=3\indicate_{A}+7\indicate_{B}$ can also
be expressed as $3\indicate_{A\setminus B}+10\indicate_{A\cap B}+7\indicate_{B\setminus A}$,
where the latter is the \emph{standard representation }since the sets
$A\cap B,A\setminus B,B\setminus A$ ( and the implicitly included
$\X\setminus\left(A\cup B\right)$) form a partition of the domain
given by preimages of the singletons in the range (namely $\{0,3,7,10\}$).
Standard representations are, of course, unique.
\end{rem*}
It should be clear that a non-negative simple function $s\in\mathcal{M^{+}}\left(\mathcal{X},\mathcal{F}\right)$
if and only if $\left\{ A_{i}\right\} _{i=1}^{k}\in\mathcal{F}$;
the ``if'' part follows directly from Corollary \ref{cor:examplesBinaryOpsMeasFunc}.
To see the ``only if'' part, note that if $s$ is measurable then
$s^{-1}\left[\left\{ \alpha_{i}\right\} \right]=A_{i}\in\mathcal{F}$
since singletons in $\R$ are Borel sets (they are closed). Denote
the collection of these measurable simple functions on $\left(\X,\mathcal{F}\right)$
as $M_{\textnormal{sim}}\left(\X,\mathcal{F}\right)$. Then, we have
shown that $M_{\textnormal{sim}}\left(\X,\mathcal{F}\right)\subset\mathcal{M^{+}}\left(\mathcal{X},\mathcal{F}\right)$.
It turns out that we can make a stronger claim: the measurable simple
functions are in some sense ``dense'' in the space of non-negative
measurable functions.
\begin{prop}
\label{prop:simpleFunctionMonotoneConvergence}Let $f\in\mathcal{M^{+}}\left(\mathcal{X},\mathcal{F}\right)$
be arbitrary. Then there exists a sequence of simple measurable functions
$\left\{ f_{n}\right\} _{n\in\N}\in M_{\textnormal{sim}}\left(\X,\mathcal{F}\right)$
such that
\[
f_{n}\leq f_{n+1}
\]
pointwise for every $n\in\N$ and
\[
\lim_{n\to\infty}f_{n}=f
\]
where the limit is taken pointwise.
\end{prop}

\begin{proof}
It turns out that we can establish the existence of such functions
$f_{n}$ constructively. Define
\[
f_{n}\left(x\right):=\sum_{k=0}^{4^{n}-1}\frac{k}{2^{n}}\indicate_{\left\{ \frac{k}{2^{n}}\leq f\left(x\right)<\frac{k+1}{2^{n}}\right\} }+2^{n}\indicate_{\left\{ f\left(x\right)\geq2^{n}\right\} }
\]
and observe that $f_{n}$ are Borel-measurable simple functions (given
the measurability of $f$) and that $f_{n}\leq f$ pointwise by definition.
Next, note that $\left\{ \frac{k}{2^{n}}\right\} _{k=0}^{4^{n}}\subset\left\{ \frac{k}{2^{n+1}}\right\} _{k=0}^{4^{n+1}}$
and fix $x_{0}\in\X$ to be arbitrary. If $\frac{k_{0}}{2^{n}}\leq f\left(x_{0}\right)<\frac{k_{0}+1}{2^{n}}$
for some $k_{0}\in\left\{ 0,1,\ldots,4^{n}-1\right\} $ then either
$\frac{k_{0}}{2^{n}}=\frac{2k_{0}}{2^{n+1}}\leq f\left(x_{0}\right)<\frac{2k_{0}+1}{2^{n+1}}$
or $\frac{2k_{0}+1}{2^{n+1}}\leq f\left(x_{0}\right)<\frac{2k_{0}+2}{2^{n+1}}=\frac{k_{0}+1}{2^{n}}$.
In either case,
\[
f_{n}\left(x_{0}\right)=\frac{k_{0}}{2^{n}}\leq f_{n+1}\left(x_{0}\right)\leq\frac{2k_{0}+1}{2^{n+1}}
\]
by the definitions of our simple functions $f_{n}.$ Conversely, if
$f\left(x_{0}\right)\geq2^{n}$ then either $f\left(x_{0}\right)\geq2^{n+1}$
or there exists some $k_{0}\in\left\{ 2^{2n+1},\ldots,4^{n+1}-1\right\} $
such that
\[
\frac{k_{0}}{2^{n+1}}\leq f\left(x_{0}\right)<\frac{k_{0}+1}{2^{n+1}}.
\]
Again, in either case,
\[
f_{n}\left(x_{0}\right)=2^{n}\leq\frac{k_{0}}{2^{n+1}}\leq f_{n+1}\left(x_{0}\right)\leq2^{n+1}
\]
which shows that $f_{n}\leq f_{n+1}$ pointwise.

To show convergence, again pick an arbitrary $x_{0}\in\X$ and observe
that if $f\left(x_{0}\right)=\infty$ then $f_{n}\left(x_{0}\right)=2^{n}\nearrow\infty=f\left(x_{0}\right).$
Conversely, suppose that $f\left(x_{0}\right)<\infty$. Then, since
the natural numbers are not bounded above in $\R$, we know that there
exists some $n_{x_{0}}\in\N$ such that for every $n\geq n_{x_{0}}$:
$f\left(x_{0}\right)<2^{n}$. Thus for each such $n,$ there exists
some $k_{n}\in\left\{ 0,1,\ldots,4^{n}-1\right\} $ such that 
\begin{equation}
\frac{k_{n}}{2^{n}}\leq f\left(x_{0}\right)<\frac{k_{n}+1}{2^{n}}\label{eq:pointwiseInequalitySimpleFunc}
\end{equation}
which would imply that $f_{n}\left(x_{0}\right)=\frac{k_{n}}{2^{n}}$
for all $n\geq n_{x_{0}}.$ Then
\[
0\leq f\left(x_{0}\right)-f_{n}\left(x_{0}\right)=f\left(x_{0}\right)-\frac{k_{n}}{2^{n}}\leq\frac{k_{n}+1}{2^{n}}-\frac{k_{n}}{2_{n}}=\frac{1}{2^{n}}
\]
where the first inequality is due to the fact that $f_{n}\leq f$
pointwise and the second inequality is due to (\ref{eq:pointwiseInequalitySimpleFunc}).
Taking limits then yields the result.
\end{proof}
\begin{prop}
\label{prop:minMaxMeasurable}For any function $f\in\mathcal{M}\left(\X,\mathcal{F}\right)$,
the derived functions\footnote{Again, maxima are taken pointwise.}
\begin{align*}
f^{+} & :=\max\left\{ f,0\right\} \\
f^{-} & :=\max\left\{ -f,0\right\} 
\end{align*}
are contained in $\mathcal{M}^{+}\left(\X,\mathcal{F}\right)$
\end{prop}

\begin{proof}
Define for any $a\in\R$ , $F_{a}:=\left\{ x\in\X\mid f^{+}\left(x\right)\in\left(a,\infty\right)\right\} $
and notice that if $a>0$, then
\[
F_{a}=f^{-1}\left[\left(a,\infty\right)\right]\in\mathcal{F}
\]
by the measurability of $f.$ Conversely, if $a\leq0$ then
\[
F_{a}=\X\in\mathcal{F}
\]
and so, by Lemma \ref{lem:collectionIntervalsMeasurable} and a generating
class argument, $f^{+}\in\mathcal{M}^{+}\left(\X,\mathcal{F}\right)$.

To see that $f^{-1}\in\mathcal{M}^{+}\left(\X,\mathcal{F}\right)$,
notice that by Corollary \ref{cor:examplesBinaryOpsMeasFunc}, $-f\in\mathcal{M}\left(\X,\mathcal{F}\right)$
and apply the same argument.
\end{proof}
\begin{rem*}
Another way to prove the above proposition would be to observe that
\[
\max\left\{ x,y\right\} =\frac{x+y+\left|x-y\right|}{2}
\]
and apply Proposition \ref{prop:binaryOperationsMeasurableFunctions}.
\end{rem*}
Proposition \ref{prop:minMaxMeasurable} is important because any
function $f\in\measurableFunctions$ can be decomposed as
\[
f=f^{+}-f^{-}
\]
and since both $f^{+},f^{-}\in\nonnegMeasurableFunctions$, by Proposition
\ref{prop:simpleFunctionMonotoneConvergence}, there are non-negative
simple functions $\left\{ s_{n}^{+}\right\} _{n\in\N},\left\{ s_{n}^{-}\right\} _{n\in\N}$
such that $s_{n}^{+}\nearrow f^{+}$and $s_{n}^{-}\nearrow f^{-}$.
By the linearity of limits, the sequence of functions $h_{n}:=s_{n}^{+}-s_{n}^{-}$
converges (although not monotonically) to $f$. This fact will prove
important in the chapter on integration.
\begin{prop}
\label{prop:simpleFunctionsAddMultiply}Let $s,t\in\mathcal{M}_{\textnormal{sim}}\measurablespace$.
Then
\[
h:=s+t\in\mathcal{M}_{\textnormal{sim}}\measurablespace
\]
and 
\[
g:=st\in\mathcal{M}_{\textnormal{sim}}\measurablespace.
\]
Moreover, if $s$ and $t$ be given by the standard representations
\begin{align*}
s & =\sum_{i=1}^{I}\alpha_{i}\indicate_{A_{i}}\\
t & =\sum_{j=1}^{J}\beta_{j}\mathds{1}_{B_{j}}
\end{align*}
then
\begin{align*}
h & =\sum_{i=1}^{I}\sum_{j=1}^{J}\left(\alpha_{i}+\beta_{j}\right)\indicate_{A_{i}\cap B_{j}}\\
g & =\sum_{i=1}^{I}\sum_{j=1}^{J}\left(\alpha_{i}\beta_{j}\right)\indicate_{A_{i}\cap B_{j}}
\end{align*}
\end{prop}

\begin{proof}
First we prove that 
\[
\sum_{i=1}^{I}\alpha_{i}\indicate_{A_{i}}+\sum_{j=1}^{J}\beta_{j}\mathds{1}_{B_{j}}=\sum_{i=1}^{I}\sum_{j=1}^{J}\left(\alpha_{i}+\beta_{j}\right)\indicate_{A_{i}\cap B_{j}}.
\]
To see this, let $x_{0}\in\X$ be arbitrary and observe that since
$\left\{ A_{i}\right\} _{i=1}^{I}$ and $\left\{ B_{j}\right\} _{j=1}^{J}$
are partitions of $\X$, there exists exactly one $1\leq i_{0}\leq I$
and one $1\leq j_{0}\leq J$ such that $x_{0}\in A_{i_{0}}$ and $x\in B_{j_{0}}$.
Then, $x_{0}\in A_{i_{0}}\cap B_{j_{0}}$ and $x\notin A_{i}\cap B_{j}$
if either $i\neq i_{0}$ or $j\neq j_{0}$.\footnote{In other words, $\left\{ A_{i}\cap B_{j}\right\} _{\left(i,j\right)\in\left\{ 1,\ldots,I\right\} \times\left\{ 1,\ldots,J\right\} }$forms
a partition of $\X$.}Then 
\begin{align*}
\left(\sum_{i=1}^{I}\alpha_{i}\indicate_{A_{i}}+\sum_{j=1}^{J}\beta_{j}\mathds{1}_{B_{j}}\right)\left(x_{0}\right) & =\sum_{i=1}^{I}\alpha_{i}\indicate_{A_{i}}\left(x_{0}\right)+\sum_{j=1}^{J}\beta_{j}\mathds{1}_{B_{j}}\left(x_{0}\right)\\
 & =\alpha_{i_{0}}+\beta_{j_{0}}\\
 & =\sum_{i=1}^{I}\sum_{j=1}^{J}\left(\alpha_{i}+\beta_{j}\right)\indicate_{A_{i}\cap B_{j}}\left(x_{0}\right)
\end{align*}
which establishes our claim. Note that the representation above need
not be standard as the partition $\left\{ A_{i}\cap B_{j}\right\} _{\left(i,j\right)\in\left\{ 1,\ldots,I\right\} \times\left\{ 1,\ldots,J\right\} }$
need not be the collection of preimages of singletons in $\Ran\left(h\right)$;
in fact, it is a \emph{refinement }of the collection of preimages,
and so the preimage of every singleton in $\Ran\left(h\right)$ can
be written as a union of sets in $\left\{ A_{i}\cap B_{j}\right\} _{\left(i,j\right)\in\left\{ 1,\ldots,I\right\} \times\left\{ 1,\ldots,J\right\} }$.

Next, for closure under multiplication, observe that
\begin{align*}
st & =\left(\sum_{i=1}^{I}\alpha_{i}\indicate_{A_{i}}\right)\left(\sum_{j=1}^{J}\beta_{j}\mathds{1}_{B_{j}}\right)\\
 & =\sum_{i=1}^{I}\sum_{j=1}^{J}\left(\alpha_{i}\beta_{j}\right)\indicate_{A_{i}}\indicate_{B_{j}}\\
 & =\sum_{i=1}^{I}\sum_{j=1}^{J}\left(\alpha_{i}\beta_{j}\right)\indicate_{A_{i}\cap B_{j}}
\end{align*}
where the last equality follows from Fact \ref{fact:indicatorFunctionsFiniteOperations}.
For the same reason as before, this representation need not be standard.
\end{proof}
\begin{prop}
\label{prop:doobDynkin}Let $\X$ be an arbitrary set and let $f:\X\to\mathcal{\R}$
be a function, then for any $\sigma\left(f\right)/\mathcal{\borel}\left(\R\right)-$measurable
function $g:\X\longrightarrow\R$, there exists a Borel-measurable
map $h:\R\to\R$ such that
\[
g=h\circ f.
\]
\end{prop}

\begin{proof}
First suppose that $g=\indicate_{A}$ for some $A\in\sigma\left(f\right).$
Then
\[
g=\indicate_{A}\left(x\right)=\indicate_{f^{-1}\left[B\right]}\left(x\right)=\indicate_{B}\left(f(x)\right)
\]
for some $B\in\mathcal{G}.$ Thus in this case, $h=\indicate_{B}$.

Now consider the case of a simple function $g=\sum_{i=1}^{n}\alpha_{i}\indicate_{A_{i}}$
for $\alpha_{i}\in\R,A_{i}\in\sigma\left(F\right).$ Note that by
the above example for indicator functions, we can find $h_{i}:\R\to\R$
such that 
\[
g=\sum_{i=1}^{n}\alpha_{i}h_{i}\left(f(x)\right)
\]
and so the function $h=\sum_{i=1}^{n}\alpha_{i}h_{i}$ does the trick.

Next, for $g\in\nonnegMeasurableFunctions,$ by Proposition \ref{prop:simpleFunctionMonotoneConvergence}we
can find a sequence of simple functions $s_{n}$ such that $s_{n}\nearrow g$
pointwise. Note from our work above that we can write $s_{n}=h_{n}\circ f$
where $h_{n}$ is Borel-measurable as above. Let $h:=\lim_{n\to\infty}h_{n}$
which exists pointwise since it is a monotone limit ($s_{n}\leq s_{n+1}\implies h_{n+1}\leq h_{n}$)
. In particular, $h$ is Borel-measurable by Corollary \ref{cor:limSupLimInfMeasurable}
and so for any fixed $x\in\X$
\[
\lim_{n\to\infty}s_{n}\left(x\right)=\lim_{n\to\infty}h_{n}\left(f\left(x\right)\right)=h\left(f\left(x\right)\right)=g\left(x\right).
\]

Finally, for general measurable $g\in\measurableFunctions$, we can
write $g=g^{+}-g^{-}$ and by the previous results, we know there
exist Borel-measurable real-valued maps $h_{1},h_{2}$ such that $g^{+}=h_{1}\circ f$,
$g^{-}=h_{2}\circ f.$ Then
\begin{align*}
g & =h_{1}\circ f-h_{2}\circ f\\
 & =(h_{1}-h_{2})\circ f
\end{align*}
and $h:=h_{1}-h_{2}$ is Borel-measurable by Corollary \ref{cor:examplesBinaryOpsMeasFunc}.
This completes the proof.
\end{proof}

\subsection{Images of measurable sets under measurable functions}

We have the shown (or rather, defined) that for measurable functions,
preimages of measurable sets are measurable. Does this hold for images
under measurable functions as well? That is, for any pair of measurable
spaces $\left(\X,\F\right)$ and $\left(\mathcal{Y},\mathcal{G}\right)$
and a measurable function $f:\X\to\mathcal{G}$, is it true that $f\left[A\right]\in\mathcal{G}$
for any $A\in\F$? The answer to this question is ``no'' and to
show this, we continue on a thread we began in Subsection \ref{subsec:cantorSets}
(read that section again). We shall first construct a strictly increasing
analogue of the Cantor function $\psi_{C}$ from that section.
\begin{prop}
\label{prop:strictlyIncreasingCantorFunction}Let $\phi\left(x\right):=\psi_{C}\left(x\right)+x$.
Then $\phi$ is a strictly increasing, continuous function that maps
$\left[0,1\right]$ onto $\left[0,2\right]$. Moreover, $\phi\left[C\right]\in\borel\left(\R\right)$
is a set of positive Lebesgue measure.
\end{prop}

\begin{proof}
Note that the sum of a non-decreasing and strictly increasing function
is strictly increasing and the sum of two continuous functions is
continuous. That gives us the first two properties. Now since $\phi\left(0\right)=0$
and $\phi\left(1\right)=2$, by the intermediate value theorem $\phi\left[\left[0,1\right]\right]=\left[0,2\right].$
Next, note that $\phi\left[C\right]=\left(\phi^{-1}\right)^{-1}\left[C\right]$
which is Borel measurable since $C$ is closed (and so Borel measurable)
and $\phi^{-1}$ is a continuous (and hence Borel measurable) function
by Proposition \ref{prop:strictlyIncreasingContinuousFunctionInverse}.
Finally, consider the image $\phi\left[\left[0,1\right]\setminus C\right]$.
We know from Proposition \ref{prop:cantorFunctionConstantOutside}that
there exists a countable collection of disjoint open intervals $\left\{ O_{j}\right\} _{j\in\N}\subset\left[0,1\right]$
such that $\bigcup_{j\in\N}O_{j}=\left[0,1\right]\setminus C$ such
that $\psi_{C}\left(x\right)=c_{j}$ for any $x\in O_{j}$. Then,
\begin{align*}
\phi\left[\left[0,1\right]\setminus C\right] & =\phi\left[\bigcup_{j\in\N}O_{j}\right]\\
 & =\bigcup_{j\in\N}\phi\left[O_{j}\right]\\
 & =\bigcup_{j\in\N}O_{j}+c_{j}
\end{align*}
where each $O_{j}+c_{j}$ is disjoint since $\phi$ is a bijection
and so its image preserves intersections. Then,
\begin{align*}
\lambda\left(\phi\left[\left[0,1\right]\setminus C\right]\right) & =\lambda\left(\bigcup_{j\in\N}O_{j}+c_{j}\right)\\
 & =\sum_{i=1}^{\infty}\lambda\left(O_{j}\right)\\
 & =\lambda\left(\bigcup_{j\in\N}O_{j}\right)\\
 & =\lambda\left(\left[0,1\right]\setminus C\right)\\
 & =1
\end{align*}
where in the inequality we used countable additivity and translation
invariance. Then, by additivity, finiteness of $\lambda$ on bounded
intervasls, and the fact that the image of a bijective function preserves
set differences-
\begin{align*}
1=\lambda\left(\phi\left[\left[0,1\right]\setminus C\right]\right) & =\lambda\left(\phi\left[\left[0,1\right]\right]\setminus\phi\left[C\right]\right)\\
 & =\lambda\left(\phi\left[\left[0,1\right]\right]\right)-\lambda\left(\phi\left[C\right]\right)\\
 & =\lambda\left(\left[0,2\right]\right)-\lambda\left(\phi\left[C\right]\right)\\
 & =2-\lambda\left(\phi\left[C\right]\right).
\end{align*}
Therefore 
\[
\lambda\left(\phi\left[C\right]\right)=1
\]
which completes the proof.
\end{proof}
\begin{thm}
\label{thm:continuousImageLebesgueMeasurable}There exists a measurable
set $A\in\mathcal{C}\left(\lambda^{*}\right)$ such that $\phi\left[A\right]\notin\mathcal{C}\left(\lambda^{*}\right).$
\end{thm}

\begin{proof}
Note that by Theorem \ref{thm:positiveMeasureNonMeasurable} and Proposition
\ref{prop:strictlyIncreasingCantorFunction}, $\phi\left[C\right]$
contains a set $E$ such that $E\notin\mathcal{C}\left(\lambda^{*}\right).$
Let $A=C\cap\phi^{-1}\left[E\right]$ and notice that $A$ is measurable
as it is a subset of a measure-zero set and since $\mathcal{C}\left(\lambda^{*}\right)$
is complete. Then, $\phi\left[A\right]=E$ which is not measurable.
\end{proof}
\begin{cor}
\label{cor:lebesgueNotBorel}There exists a set $A\in\mathcal{C}\left(\lambda^{*}\right)$
such that $A\notin\borel\left(\R\right)$.
\end{cor}

\begin{proof}
Let $A=C\cap\phi^{-1}\left[E\right]$ as in Theorem \ref{thm:continuousImageLebesgueMeasurable}.
Then $A\in\mathcal{C}\left(\lambda^{*}\right)$ but $\phi\left[A\right]\notin\mathcal{C}\left(\lambda^{*}\right).$
But recall that $\phi^{-1}$ is measurable since it is continuous
and $\phi\left[A\right]=\left(\phi^{-1}\right)^{-1}\left[A\right]$
and so if $A\in\borel\left(\R\right)$ then $\phi\left[A\right]\in\borel\left(\R\right)$
which would be a contradiction.
\end{proof}

\subsection{An iterative construction of the Cantor function}

\section{Types of generating class arguments\label{sec:genClassArgs}}

So far we have seen how a generating class argument, such as the one
formalized in Theorem \ref{thm:genericGeneratingClassArgument}, can
help establish the measurability of functions. In general, measurability
can be replaced by any property, opening up other avenues to use such
arguments. The abstract analogue of Theorem \ref{thm:genericGeneratingClassArgument}
goes as follows: Let $\mathcal{F}$ be a $\sigma-$algebra on $\X$.
We want to show that all sets in $\mathcal{F}$ enjoy some property
$\left(*\right)$. In order to prove that this is indeed the case
we
\begin{enumerate}
\item Find a subclass of sets $\mathcal{E}\subseteq\mathcal{F}$ that enjoys
property $\left(*\right)$ such that $\sigma\left(\mathcal{E}\right)=\mathcal{F}.$
\item Define $\mathcal{D}=\left\{ F\in\mathcal{F}\mid F\ \text{enjoys property }\text{\ensuremath{\left(*\right)}}\right\} $.
\item Observe that if $\mathcal{D}$ is a $\sigma-$algebra then
\[
\mathcal{E}\subseteq\mathcal{D}\subseteq\mathcal{F}\Longrightarrow\mathcal{F}=\sigma\left(\mathcal{E}\right)\subseteq\sigma\left(\mathcal{D}\right)=\mathcal{D}\subseteq\mathcal{F}\Longrightarrow\mathcal{F}=\mathcal{D}.
\]
\end{enumerate}
In this argument, we do not assume any structure on the special class
$\mathcal{E}.$ However, imposing structure can add power to generating
class arguments, since then we can drop the requirement that subclass
$\mathcal{D}$ of ``desirable'' sets be a $\sigma-$algebra. The
fact that we can do this is not \emph{a-priori }obvious, and the point
of this section is to develop the theory to show that this can be
done with different types of structural assumptions. The results of
this section form the backbone of standard techniques in measure theory,
and have important applications, one of which we show here.

\subsection{Dynkin's $\pi-\lambda$ Theorem}
\begin{defn}
\label{def:dynkinSystem}A collection of sets $\mathcal{D}\subseteq2^{\X}$
is called a \emph{Dynkin system }or a $\lambda-$system if

\begin{enumerate}[label=(\roman*),leftmargin=.1\linewidth,rightmargin=.4\linewidth]
	\item $ \X \in \mathcal{D} $
	\item $ A_1, A_2 \in \mathcal{D} \text{ s.t. } A_2 \subseteq A_1 \Longrightarrow A_2\setminus A_1 \in \mathcal{D} $
	\item $\{A_i\}_{i\in\N} \in \mathcal{D} \text{ s.t. } A_i \subseteq A_{i+1} \Longrightarrow \bigcup_{i\in\N} A_i \in \mathcal{D} $
\end{enumerate}
\end{defn}

\begin{prop}
\label{prop:dynkinSystemEquivDefn}A collection of sets $\mathcal{D\subseteq}2^{\mathcal{X}}$
is a $\lambda-$system if and only if

\begin{enumerate}[label=(\roman*'),leftmargin=.1\linewidth,rightmargin=.4\linewidth]
	\item $ \X \in \mathcal{D} $
	\item $ A \in \mathcal{D} \Longrightarrow A^C \in \mathcal{D} $
	\item $\{A_i\}_{i\in\N} \in \mathcal{D}$ s.t $A_i \cap A_j = \emptyset \Longrightarrow \bigcup_{i\in\N} A_i \in \mathcal{D}$
\end{enumerate}
\end{prop}

\begin{proof}
First suppose that $\mathcal{D}$ is a $\lambda-$system. Then, property
$(i')$ above is automatically satisfied. For property $(ii')$, observe
that $\text{(i),(ii})$ together imply $(ii')$ since $A^{C}=\X\setminus A.$
Next, let $\left\{ A_{i}\right\} _{i\in\N}\in\mathcal{D}$ be pairwise
disjoint and define 
\[
B_{n}:=\bigcup_{i=1}^{n}A_{i},
\]
observing that $B_{n}\subseteq B_{n+1}$ and that $\bigcup_{n\in\N}B_{n}=\bigcup_{i\in\N}A_{i}.$
Then $(iii')$ follows from (iii).

Conversely, assume that $\mathcal{D}$ satisfies $(i')-(iii')$ above.
Again, (i) follows from $(i')$. Next, pick $A_{1},A_{2}\in\mathcal{D}$
such that $A_{2}\subseteq A_{1}$. By $(ii')$ , $A_{1}^{C}\in\mathcal{D}$
and we know that$A_{1}^{C}\cap A_{2}=\emptyset$ and so by $\left(ii'\right),\left(iii'\right)$,
we have that
\[
A_{1}\setminus A_{2}=\left(A_{1}^{C}\cup A_{2}\right)^{C}\in\mathcal{D}
\]
which proves (ii). Finally, let $\left\{ A_{i}\right\} _{i\in\N}\in\mathcal{D}$
be an increasing sequence of sets (that is, $A_{i}\subseteq A_{i+1}$
for every $i\in\N$). By (ii), 
\[
B_{i}:=A_{i}\setminus A_{i-1}\in\mathcal{D}
\]
 and are pairwise disjoint. Applying $(iii')$ we have that
\[
\bigcup_{i\in\N}A_{i}=\bigcup_{i\in\N}B_{i}\in\mathcal{D}
\]
completing the proof.
\end{proof}
It is easy to see that a $\sigma-$algebra is a $\lambda-$system,
and arguments akin to those in Proposition \ref{prop:ringGeneratedByClassIsRing}
will prove that
\[
\lambda\left(\mathcal{E}\right):=\bigcap\left\{ \mathcal{A}\subseteq2^{\X}\text{ is a }\lambda-\text{system}\mid\mathcal{E}\subseteq\mathcal{A}\right\} 
\]
is itself a $\lambda-$system, and these two facts together show that
\[
\mathcal{E}\subseteq\lambda\left(\mathcal{E}\right)\subseteq\sigma\left(\mathcal{E}\right).
\]

\begin{defn}
\label{def:piSystem}A collection of sets $\mathcal{E}\subseteq2^{\X}$
is called a $\pi-$system if it is closed under finite intersections
i.e. if $A,B\in\mathcal{E}$ then $A\cap B\in\mathcal{E}$.
\end{defn}

Properties of $\pi-$systems and $\lambda-$systems can be combined
to yield a $\sigma-$algebra.
\begin{lem}
\label{lem:piLambdaIsSigma}A collection of sets $\mathcal{F}\subseteq2^{\X}$
is a $\sigma-$algebra if and only if it is both a $\pi-$system and
a $\lambda-$system.
\end{lem}

\begin{proof}
Clearly, if $\mathcal{F}$ is a $\sigma-$algebra, then it is both
a $\pi-$system and a $\lambda-$system. To see the converse, note
that Proposition \ref{prop:dynkinSystemEquivDefn} gives us (i) containing
$\X$ and (ii) closure under complements for free. Thus we need to
prove that for any countable (not necessarily disjoint) collection
of sets $\left\{ E_{i}\right\} _{i\in\N}\in\mathcal{E}$
\[
\bigcup_{i\in\N}E_{i}\in\mathcal{F}.
\]
As we usually do, we shall look at the ``disjointification''
\begin{align*}
F_{i} & :=E_{i}\setminus\bigcup_{j=1}^{i-1}E_{j}\\
 & =E_{i}\cap\left(\bigcap_{j=1}^{i-1}E_{j}^{C}\right)
\end{align*}
and observe that closure under finite intersections (from being a
$\pi-$system) and complements implies that $F_{i}\in\mathcal{F}$.
Moreover, the $F_{i}$ are pairwise disjoint, and so by closure under
countable \emph{disjoint }unions, we have that
\[
\bigcup_{i\in\N}E_{i}=\bigcup_{i\in\N}F_{i}\in\mathcal{F}
\]
which finishes the proof.
\end{proof}
\begin{thm}[Dynkin's $\pi-\lambda$ Theorem]
\label{thm:piLambdaThm}Let $\mathcal{E}\subseteq2^{\X}$ be a $\pi-$system.
Then
\[
\lambda\left(\mathcal{E}\right)=\sigma\left(\mathcal{E}\right).
\]
\end{thm}

\begin{proof}
We had shown earlier that $\lambda\left(\mathcal{E}\right)\subseteq\sigma\left(\mathcal{E}\right)$
and so we now we use the additional hypothesis of closure under finite
intersections to show the reverse inclusion. By Lemma \ref{lem:piLambdaIsSigma},
we only need to show that $\lambda\left(\mathcal{E}\right)$ inherits
closure under finite intersections from $\mathcal{E}$, since then
$\lambda\left(\mathcal{E}\right)$ would be a $\sigma-$algebra that
contains $\mathcal{E}$ and so $\sigma\left(\mathcal{E}\right)\subseteq\lambda\left(\mathcal{E}\right)$.

In order to show that $\lambda\left(\mathcal{E}\right)$ is indeed
closed under finite intersections, define for every $B\in\lambda\left(\mathcal{E}\right)$
\[
\mathcal{D}_{B}:=\left\{ A\subseteq\X\mid A\cap B\in\lambda\left(\mathcal{E}\right)\right\} .
\]
It turns out that $\mathcal{D}_{B}$ is a $\lambda-$system for every
$B\in\lambda\left(\mathcal{E}\right)$. To see this, note that $\X\in\mathcal{D}_{B}$
since $\X\cap B=B\in\lambda\left(\mathcal{E}\right).$ Moreover, for
any sets $A_{1},A_{2}\in\mathcal{D}_{B}$ such that $A_{2}\subseteq A_{1}$
\[
\left(A_{1}\setminus A_{2}\right)\cap B=A_{1}\cap A_{2}^{C}\cap B=A_{1}\cap B\setminus A_{2}=A_{1}\cap B\setminus A_{2}\cap B\in\lambda\left(\mathcal{E}\right)
\]
 since $A_{1}\cap B,A_{2}\cap B\in\lambda\left(\mathcal{E}\right)$
and $A_{2}\cap B\subseteq A_{1}\cap B$ (see property (ii) in Definition
\ref{def:dynkinSystem}). This shows that $A_{1}\setminus A_{2}\in\mathcal{D}_{B}.$
Finally, let $\left\{ A_{i}\right\} _{i\in\N}\in\mathcal{D}_{B}$
be an increasing sequence of sets, then
\[
A_{i}\cap B\subseteq A_{i+1}\cap B
\]
and so using property (iii) of $\lambda-$systems 
\[
\left(\bigcup_{i\in\N}A_{i}\right)\cap B=\bigcup_{i\in\N}\left(A_{i}\cap B\right)\in\lambda\left(\mathcal{E}\right)
\]
which shows that $\bigcup_{i\in\N}A_{i}\in\mathcal{D}_{B}$, proving
our claim that $\mathcal{D}_{B}$ is a $\lambda-$system for any $B\in\lambda\left(\mathcal{E}\right).$
In particular, for any $E\in\mathcal{E}$, $\mathcal{D}_{E}$ is a
$\lambda-$system such that $\mathcal{E}\subseteq\mathcal{D}_{E}$
because of $\mathcal{E}$ is closed under finite intersections. Since
$\mathcal{D}_{E}$ is a $\lambda-$system,
\[
\lambda\left(\mathcal{E}\right)\subseteq\mathcal{D}_{E}
\]
for every $E\in\mathcal{E}$. Now we have proved that for any $B\in\lambda\left(\mathcal{E}\right)$
and any $E\in\mathcal{E}$, their intersection $A\cap E\in\lambda\left(\mathcal{E}\right)$,
which seems like it is just a little bit short of what we need. But
notice that this means $\mathcal{E}\subseteq\mathcal{D}_{B}$ for
every $B\in\lambda\left(\mathcal{E}\right)$, which implies that 
\[
\lambda\left(\mathcal{E}\right)\subseteq\mathcal{D}_{B}
\]
for every $B\in\lambda\left(\mathcal{E}\right).$ Thus we have proved
that for any $A,B\in\mathcal{\lambda\left(E\right)},A\cap B\in\mathcal{\lambda\left(E\right)}$
which completes the proof.
\end{proof}
\begin{cor}
\label{cor:piLambdaGeneratingClassArg}Let $\mathcal{D}\subseteq2^{\X}$
be a $\lambda-$system and let $\mathcal{E}\subseteq\mathcal{D}$
be a $\pi-$system. Then
\[
\sigma\left(\mathcal{E}\right)\subseteq\mathcal{D}.
\]
\end{cor}

\begin{proof}
By the $\pi-\lambda$ theorem
\[
\sigma\left(\mathcal{E}\right)=\lambda\left(\mathcal{E}\right)\subseteq\mathcal{D}
\]
since $\mathcal{D}$ is a $\lambda-$system that contains $\mathcal{E}$.
\end{proof}
This corollary is useful because it adds structural assumptions on
the generating class (by requiring that it be a $\pi-$system), which
allows us to loosen the restriction that $\mathcal{D}$ be a $\sigma-$algebra
like we used to assume in standard generating class arguments. This
technique is powerful, and has important applications such as finding
sufficient conditions to show that two measures on a $\sigma-$algebra
are equal without having to explicitly check that equality holds on
the entire $\sigma-$algebra.
\begin{thm}[Uniqueness of Measures]
 \label{thm:uniquenessMeasures}Let $\left(\X,\mathcal{F}\right)$
be a measurable space and let $\mu,\nu$ be measures on $\mathcal{F}$.
If there exists a $\pi-$system $\mathcal{E\subseteq}2^{\X}$ such
that $\sigma\left(\mathcal{E}\right)=\mathcal{F}$ and 
\[
\mu\left(E\right)=\nu\left(E\right)
\]
for every $E\in\mathcal{E}$, and there exists increasing sequence
$\left\{ E_{i}\right\} _{i\in\N}\in\mathcal{E}$ such that $\X=\bigcup_{i\in\N}E_{i}$
and
\[
\mu\left(E_{i}\right)=\nu\left(E_{i}\right)<\infty
\]
for all $i\in\N$ then
\[
\mu\left(F\right)=\nu\left(F\right)
\]
for every $F\in\mathcal{F}$.
\end{thm}

\begin{proof}
Define
\[
\mathcal{D}_{i}:=\left\{ F\in\mathcal{F}\mid\mu\left(F\cap E_{i}\right)=\nu\left(F\cap E_{i}\right)\right\} 
\]
and note that $\X\in\mathcal{D}_{i}$ for every $i\in\N$ since $\mu\left(E_{i}\right)=\nu\left(E_{i}\right)$.
Next, let $F\in\mathcal{D}_{i}$ and observe that 
\begin{align*}
\mu\left(F^{C}\cap E_{i}\right) & =\mu\left(E_{i}\setminus\left(E_{i}\cap F\right)\right)\\
 & =\mu\left(E_{i}\right)-\mu\left(E_{i}\cap F\right)\\
 & =\nu\left(E_{i}\right)-\nu\left(E_{i}\cap F\right)\\
 & =\nu\left(E_{i}\setminus\left(E_{i}\cap F\right)\right)\\
 & =\nu\left(F^{C}\cap E_{i}\right)
\end{align*}
where the second and fourth equalities follow from (finite) additivity
of measures and the fact that $\mu\left(E_{i}\cap F\right)\leq\mu\left(E_{i}\right)<\infty$.
This proves that for any $F\in\mathcal{D}_{i}$, $F^{C}\in\mathcal{D}_{i}$
for every $i\in\N$. Finally, let $\left\{ F_{j}\right\} _{j\in\N}$be
a pairwise disjoint collection of sets in $\mathcal{D}_{i}$. Then
\begin{align*}
\mu\left(\left(\bigcup_{k\in\N}F_{j}\right)\cap E_{i}\right) & =\mu\left(\bigcup_{j\in\N}\left(F_{j}\cap E_{i}\right)\right)\\
 & =\sum_{j=1}^{\infty}\mu\left(F_{j}\cap E_{i}\right)\\
 & =\sum_{j=1}^{\infty}\nu\left(F_{j}\cap E_{i}\right)\\
 & =\nu\left(\bigcup_{j\in\N}\left(F_{j}\cap E_{i}\right)\right)\\
 & =\nu\left(\left(\bigcup_{k\in\N}F_{j}\right)\cap E_{i}\right)
\end{align*}
where the second and fourth equality follow from countable additivity
and the third equality follows from the uniqueness of limits. This
proves that $\bigcup_{j\in\N}F_{j}\in\mathcal{D}_{i}$ for every $i\in\N,$
which shows (through Proposition \ref{prop:dynkinSystemEquivDefn})
that every $\mathcal{D}_{i}$ is a $\lambda-$system.

Now note that since $\mathcal{E}$ is a $\pi-$system (and so closed
under finite intersections), we have that 
\[
\mathcal{E}\subseteq\mathcal{D}_{i}
\]
for every $i\in\N$. By Corollary \ref{cor:piLambdaGeneratingClassArg},
we have that 
\[
\mathcal{F}=\sigma\left(\mathcal{E}\right)\subseteq\mathcal{D}_{i}
\]
for every $i\in\N$. This means that for any $F\in\mathcal{F}$ and
any $i\in\N$
\[
\mu\left(F\cap E_{i}\right)=\nu\left(F\cap E_{i}\right).
\]
Let $F\in\mathcal{F}$ be arbitrary and observe that since $\left\{ E_{i}\right\} _{i\in\N}\nearrow\X$,
$\left\{ E_{i}\cap F\right\} _{i\in\N}\nearrow\X\cap F=F.$ Then,
by the \hyperref[prop:measureProperties]{continuity of measures},
\[
\mu\left(F\right)=\lim_{i\to\infty}\mu\left(F\cap E_{i}\right)=\lim_{i\to\infty}\nu\left(F\cap E_{i}\right)=\nu\left(F\right),
\]
again by the uniqueness of limits.
\end{proof}

\subsection{Monotone class theorem}
\begin{defn}
\label{def:monotoneClass} A collection of sets $\mathcal{M\subseteq}2^{\X}$
is called a \emph{monotone class }if

\begin{enumerate}[label=(\roman*),leftmargin=.1\linewidth,rightmargin=.4\linewidth]
	\item For an increasing sequence of sets $ \{A_i\}_{i\in\N} \in \mathcal{M}: \bigcup_{i\in\N}A_i \in \mathcal{M} $
	\item For a decreasing sequence of sets $ \{A_i\}_{i\in\N} \in \mathcal{M}: \bigcap_{i\in\N}A_i \in \mathcal{M} $
\end{enumerate}
\end{defn}

Note that every $\sigma-$algebra is already a monotone class. However,
we can make an even stronger claim: every $\lambda-$system is monotone
class.
\begin{prop}
\label{prop:dynkinSystemIsMonotoneClass} Let $\mathcal{D}\subseteq2^{\X}$
be a $\lambda-$system. Then $\mathcal{D}$ is a monotone class
\end{prop}

\begin{proof}
Property (i) above follows directly from Definition \ref{def:dynkinSystem}.
To see property (ii), let $\left\{ A_{i}\right\} _{i\in\N}\in\mathcal{D}$
be a decreasing sequence of sets and observe that 
\[
B_{i}=\X\setminus A_{i}\in\mathcal{D}
\]
is an inccreasing sequence of sets such that
\[
\bigcup_{i\in\N}B_{i}\in\mathcal{D}.
\]
However, 
\begin{align*}
\bigcap_{i\in\N}A_{i} & =\left(\bigcup_{i\in\N}A_{i}^{C}\right)^{C}\\
 & =\left(\bigcup_{i\in\N}B_{i}\right)^{C}\\
 & =\X\setminus\left(\bigcup_{i\in\N}B_{i}\right)
\end{align*}
which is in $\mathcal{D}$. This completes the proof.
\end{proof}
As usual, we define the monotone class generated by class of sets
$\mathcal{A}$ as 
\[
\mathscr{M}\left(\mathcal{A}\right):=\bigcap\left\{ \mathcal{M}\subseteq2^{\X}\text{ is a monotone class}\mid\mathcal{A}\subseteq\mathcal{M}\right\} 
\]
which is itself a monotone class by the usual arguments. Again, since
$\sigma-$algebras and $\lambda-$systems are monotone classes, we
have that
\[
\mathcal{A}\subseteq\mathscr{M}\left(\mathcal{A}\right)\subseteq\lambda\left(\mathcal{A}\right)\subseteq\sigma\left(\mathcal{A}\right)
\]
where we also use the fact that a $\sigma-$algebra is a $\lambda-$system.
Finally, note that an \hyperref[def:algebra]{\emph{algebra} of sets}
$\mathcal{A}$ is also a $\pi-$system and so, by the $\pi-\lambda$
theorem, 
\[
\sigma\left(\mathcal{A}\right)=\lambda\left(\mathcal{A}\right).
\]

\begin{thm}[Monotone class theorem]
\label{thm:monotoneClassThm} Let $\mathcal{A}\subseteq2^{\X}$ be
an algebra of sets. Then
\[
\mathscr{M}\left(\mathcal{A}\right)=\sigma\left(\mathcal{A}\right).
\]
\end{thm}

\begin{proof}
We have shown that $\mathscr{M}\left(\mathcal{A}\right)\subseteq\sigma\left(\mathcal{A}\right)$
and so we need to show the reverse inclusion. Since $\mathcal{A}$
is also a $\pi-$system, this is equivalent to showing that $\lambda\left(\mathcal{A}\right)\subseteq\mathscr{M}\left(\mathcal{A}\right)$.
To this end, it is sufficient to show that $\mathscr{M}\left(\mathcal{A}\right)$
is a $\lambda-$system. Note that $\X\in\mathscr{M}\left(\mathcal{A}\right)$
since $\X\in\mathcal{A}$. Further, $\mathscr{M}\left(\mathcal{A}\right)$
is closed under limits of increasing sequences of sets since it's
a monotone class. Thus we need to prove that for any $A_{1},A_{2}\in\mathscr{M}\left(\mathcal{A}\right)$
such that $A_{2}\subseteq A_{1}$ we have that $A_{1}\setminus A_{2}\in\mathscr{M}\mathcal{\left(A\right)}.$

For any set $E\in\mathscr{M}\left(\mathcal{A}\right)$, define
\[
M_{E}:=\left\{ F\subseteq\mathscr{M}\left(\mathcal{A}\right)\mid E\setminus F,F\setminus E\in\mathscr{M}\left(\mathcal{A}\right)\right\} 
\]
and note that for an increasing collection of sets $\left\{ F_{i}\right\} _{i\in\N}\in M_{E}$
\begin{align*}
E\setminus\bigcup_{i\in\N}F_{i} & =E\cap\left(\bigcap_{i\in\N}F_{i}^{C}\right)\\
 & =\bigcap_{i\in\N}\left(E\cap F_{i}^{C}\right)\\
 & =\bigcap_{i\in\N}E\setminus F_{i}\in M_{E}
\end{align*}
since $E\setminus F_{i}\supseteq E\setminus F_{i+1}$ and $E\setminus F_{i}\in\mathscr{M}\left(\mathcal{A}\right)$
for all $i\in\N$. A similar aragument shows that $\bigcup_{i\in\N}F_{i}\setminus E\in\mathscr{M}\left(\mathcal{A}\right)$
which shows that $\bigcup_{i\in\N}F_{i}\in M_{E}.$ We can apply the
same argument to show that for a decreasing sequence of sets $\left\{ F_{i}\right\} _{i\in\N}\in M_{E}$,
the intersection $\bigcap_{i\in\N}F_{i}\in M_{E}$. This proves that
$M_{E}$ is in fact a monotone class, and since $\mathcal{A}$ is
an algebra (and so it's closed under set differences), $\mathcal{A}\subseteq M_{E}$
which implies that $\mathscr{M}\left(\mathcal{A}\right)\subseteq M_{E}$
for any $E\in\mathscr{M}\left(\mathcal{A}\right).$ In other words,
this proves that for any $E,F\in\mathscr{M}\left(\mathcal{A}\right):$
\[
E\setminus F,F\setminus E\in\mathscr{M}\left(\mathcal{A}\right)
\]
 which completes the proof.
\end{proof}
Note how we used our previous work with the $\pi-\lambda$ theorem
to establish the monotone class theorem; it turns out that the monotone
class also implies theorem implies the $\pi-\lambda$ theorem, and
so these two theorems are in fact equivalent. This should not be particularly
surprising to you at this juncture since they are both generating
class arguments which make slightly different structural assumptions
to yield the same result. We can present these assumptions in a concise
and organized fashion, using the notation defined in the abstract
generating class argument described at the beginning of this section.
In Table \ref{tab:typesGenClassArg} $\mathcal{E}$ is the generating
class which satisfies the desired property $\left(*\right)$ and $\mathcal{D}$
is all the sets in the $\sigma-$algebra $\mathcal{F}$ which satisfies
property $\left(*\right)$. In the generic argument, we put very strong
structural assumptions on $\mathcal{D}$ and none on $\mathcal{E}$.
In the $\pi-\lambda$ case, we weaken the assumptions on $\mathcal{D}$
in exchange for imposing (weak) assumptions on $\mathcal{E}$. Finally,
in the monotone class approach, we impose a strong structural assumption
on $\mathcal{E}$, whereas $\mathcal{D}$ has the minimal structure
of a monotone class. The type of argument one uses in practice depends
on context, and in many situations one can pick to use either Dynkin's
theorem or the monotone class theorem.

\begin{table}[H]
\caption{\label{tab:typesGenClassArg}Types of generating class arguments}

\centering{}%
\begin{tabular}{ccc}
\hline 
Name & Structure of $\mathcal{E}$ & Structure of $\mathcal{D}$\tabularnewline
\hline 
\hline 
Generic & No structure & $\sigma-$algebra\tabularnewline
Dynkin's theorem & $\pi-$system & $\lambda-$system\tabularnewline
Monotone class theorem & Algebra  & Monotone class\tabularnewline
\hline 
\end{tabular}
\end{table}


\subsection{Generating classes of functions}
\begin{defn}
\label{def:lambdaSpaceOfFunctions}A set $\mathcal{H}$ of bounded
real functions on $\X$ is called a $\lambda-$\emph{space of functions
}if

\begin{enumerate}[label=(\roman*),leftmargin=.1\linewidth,rightmargin=.4\linewidth]
	\item The constant function $\indicate_{\X} \in \mathcal{H} $
	\item $\mathcal{H}$ is a vector space over $\R$
	\item If $h_n \in \mathcal{H} $ such that $h_n \leq h_{n+1} $ pointwise and
		\[
					\lim_{n\to\infty} h_n = h
		\]
		 is well defined and bounded then $h \in \mathcal{H}$
\end{enumerate}
\end{defn}

$\lambda-$spaces of functions are named so because they generalize
$\lambda-$systems of sets, as we show in the following result.
\begin{prop}
\label{prop:lambdaFuncsGeneralizeLambdaSystems} If $\mathcal{H}$
is a $\lambda-$space of functions on $\X$ then
\[
\mathcal{D:=}\left\{ A\subseteq\X\mid\indicate_{A}\in\mathcal{H}\right\} 
\]
is a $\lambda-$system.
\end{prop}

\begin{proof}
Note that since $\indicate_{\X}\in\mathcal{H}$ we have that $\X\in\mathcal{D}.$
Next, assume that $A_{1},A_{2}\in\mathcal{D}$ such that $A_{2}\subseteq A_{1}$
and observe that
\begin{align*}
\indicate_{A_{1}\setminus A_{2}} & =\indicate_{A_{1}\cap A_{2}^{C}}\\
 & =\indicate_{A_{1}}\left(1-\indicate_{A_{2}}\right)\\
 & =\indicate_{A_{1}}-\indicate_{A_{2}}\in\mathcal{H}
\end{align*}
where the second and third equalities follows from Fact \ref{fact:indicatorFunctionsFiniteOperations},
and the inclusion follows from the fact that $\mathcal{H}$ is closed
under linear combinations. This prove that $A_{1}\setminus A_{2}\in\mathcal{H}$.
Fiinally, let $\left\{ A_{i}\right\} _{i\in\N}$ be an increasing
sequence of sets in $\mathcal{D}$ and observe that
\begin{align*}
\indicate_{\bigcup_{i\in\N}A_{i}} & =\sup_{i\in\N}\indicate_{A_{i}}\\
 & =\lim_{n\to\infty}\indicate_{A_{i}}\in\mathcal{H}
\end{align*}
where the first equality follows from \ref{prop:indicatorFunctionsArbitraryOperations}
and the second equality is due to the fact that $A\subseteq B\Longrightarrow\indicate_{A}\leq\indicate_{B}$,
and so $\indicate_{A_{i}}$ converges pointwise to its supremum. This
proves that $\bigcup_{i\in\N}A_{i}\in\mathcal{D}$ which implies that
$\mathcal{D}$ is a $\lambda-$system.
\end{proof}
In the next theorem and elsewhere, let $\mathcal{M}_{\text{bdd}}\left(\X,\mathcal{F}\right)$
and $\mathcal{M}_{\text{bdd}}^{+}\left(\X,\mathcal{F}\right)$ denote
the set of all bounded real-valued Borel-measurable functions on $\left(\mathcal{X},\mathcal{F}\right),$and
the set of all non-negative and bounded real-valued Borel-measurable
functions on $\left(\mathcal{X},\mathcal{F}\right)$ respectively.
\begin{thm}[$\pi-\lambda$ theorem for functions]
\label{thm:piLambdaThmFunctions} Let $\mathcal{G}$ be a $\pi-$system
of sets and let $\mathcal{H}$ be a $\lambda-$space of functions
on $\X$. If
\[
\left\{ \indicate_{A}\mid A\in\mathcal{G}\right\} \subseteq\mathcal{H}
\]
then $\mathcal{M}_{\textnormal{bdd}}\left(\X,\sigma\left(\mathcal{G}\right)\right)\subseteq\mathcal{H}.$
\end{thm}

\begin{proof}
Define $\mathcal{D:=}\left\{ D\subseteq\X\mid\indicate_{D}\in\mathcal{H}\right\} $.
Then, by Proposition \ref{prop:lambdaFuncsGeneralizeLambdaSystems},
$\mathcal{D}$ is a $\lambda-$system such that $\mathcal{G}\subseteq\mathcal{D}.$
By the \hyperref[cor:piLambdaGeneratingClassArg]{$\pi-\lambda$ theorem for sets},
$\sigma\left(\mathcal{G}\right)\subseteq\mathcal{D}$. This means
that the indicator functions of sets in $\sigma\left(\mathcal{G}\right)$
is contained in $\mathcal{H}$.

Next, let $f\in\mathcal{M}_{\text{bdd}}^{+}\left(\X,\sigma\left(\mathcal{G}\right)\right)$
be arbitrary. Since $f\in\mathcal{M}^{+}\left(\X,\sigma\left(\mathcal{G}\right)\right),$
by Proposition \ref{prop:simpleFunctionMonotoneConvergence} there
exists an increasing sequence of simple functions $\left\{ s_{n}\right\} _{n\in\N}\in\mathcal{M}^{+}\left(\X,\sigma\left(\mathcal{G}\right)\right)$
such that 
\[
f=\lim_{n\to\infty}s_{n}.
\]
Since, $f$ is bounded, $s_{n}\leq f$ are all bounded. Moreover,
since every $s_{n}$ is a measurable simple function, it's a finite
linear combination of indicator functions of sets in $\sigma\left(\mathcal{G}\right)$.
Therefore, $\left\{ s_{n}\right\} _{n\in\N}\in\mathcal{H}$ and since
$f$ is the monotone limit of $s_{n}$, $f\in\mathcal{H}$.

Finally, assume that $f\in\mathcal{M}_{\textnormal{bdd}}\left(\X,\sigma\left(\mathcal{G}\right)\right)$
and note that $f^{+},f^{-}\in\mathcal{H}$ by Proposition \ref{prop:minMaxMeasurable}
and the last step, and so $f=f^{+}-f^{-}\in\mathcal{H}$.
\end{proof}

