
\chapter{Limits and continuity\label{chap:limitsAndContinuity}}

For completeness, we state and prove all theorems from basic real
analysis that are used in the main text.

\section{Elementary analysis on the line}

\subsection{Sequences and series}

\subsubsection{Properties of Limit Inferiors and Superiors}

We collect here the following properties of the limit superiors and
limit inferiors. In all of the following, $\left\{ a_{n}\right\} $
and $\left\{ b_{n}\right\} $ are real sequences. Recall that $\limsup_{n\to\infty}a_{n}=\inf_{n\geq1}\left\{ \sup_{i\geq n}\left\{ a_{i}\right\} \right\} $
and $\liminf_{n\to\infty}a_{n}=\sup_{n\geq1}\left\{ \inf_{i\geq n}\left\{ a_{i}\right\} \right\} $.
Note that these always exist (in the extended reals) by the completeness
of the real numbers.
\begin{prop}
\label{prop:.limInfNegLimSupNeg}$\liminf_{n\to\infty}a_{n}=-\limsup_{n\to\infty}-a_{n}$.
\end{prop}

\begin{proof}
Note that $\inf_{i\geq n}a_{i}=-\sup_{i\geq n}-a_{i}$ and so taking
limits completes the proof.
\end{proof}
\begin{prop}
\label{prop:limSupDominatesLimInf}$\liminf_{n\to\infty}a_{n}\leq\limsup_{n\to\infty}a_{n}$.
\end{prop}

\begin{proof}
$\inf_{i\geq n}a_{i}\leq\sup_{i\geq n}a_{i}$ and so the result follows
by taking limits.
\end{proof}
\begin{prop}
\label{prop:limSupSubsequential}$\liminf_{n\to\infty}a_{n}$ and
$\limsup_{n\to\infty}a_{n}$ are subsequential limits of $a_{n}$
such that for any other subsequential limit $L$ we have that 
\[
\liminf_{n\to\infty}a_{n}\leq L\leq\limsup_{n\to\infty}a_{n}.
\]
\end{prop}

\begin{proof}
Note that if $L<\liminf_{n\to\infty}a_{n}$ , then there are infinitely
many elements of $\left\{ a_{n}\right\} $ in $\left(L-\epsilon,L+\epsilon\right)$where
$\epsilon>0$ is small enough that $L+\epsilon<\liminf_{n\to\infty}a_{n}$.
Note that since $\left\{ \inf_{i\geq n}a_{i}\right\} $ is a nondecreasing
sequence, there exists some $n_{0}\in\N$ such that for all $n\geq n_{0}$,
$\inf_{i\geq n}a_{i}>L+\epsilon$. But then $a_{i}\geq\inf_{i\geq n_{0}}a_{i}>L+\epsilon$
for all every $i\geq n_{0}$ which contradicts the fact that infinitely
many of our $\left\{ a_{n}\right\} $ were in $\left[L,L+\epsilon\right)$.
The proof for the limit superior case is analagous.

Next fix $\epsilon>0$ , let $c_{n}:=\inf_{i\geq n}a_{i}$., and let
$I:=\lim c_{n}.$By the definition of infimum, there exists some $k_{n}\geq n$
such that $\lvert c_{n}-a_{k_{n}}\rvert<\frac{\epsilon}{2}$. By definition
of limit inferiors, we have that for large enough $n$, $\lvert c_{n}-I\rvert<\frac{\epsilon}{2}$
and so
\[
\lvert a_{k_{n}}-I\rvert\leq\lvert c_{n}-a_{k_{n}}\rvert+\lvert c_{n}-I\rvert<\epsilon
\]
which completes the proof. The proof for limit superiors is analagous
(or you can use Proposition \ref{prop:.limInfNegLimSupNeg}).
\end{proof}
\begin{prop}
\label{prop:limSupEqualLimInf}The sequence $\left\{ a_{n}\right\} $
converges if and only if $\limsup_{n\to\infty}a_{n}=\liminf_{n\to\infty}a_{n}$
in which case 
\[
\lim_{n\to\infty}a_{n}=\limsup_{n\to\infty}a_{n}=\liminf_{n\to\infty}a_{n}.
\]
\end{prop}

\begin{proof}
If $\lim_{n\to\infty}a_{n}=L$, then all subsequential limits are
also equal to $L$, which implies (by Proposition \ref{prop:limSupSubsequential})
that $\lim_{n\to\infty}a_{n}=\limsup_{n\to\infty}a_{n}=\liminf_{n\to\infty}a_{n}.$
Conversely, assuming that $L:=\limsup_{n\to\infty}a_{n}=\liminf_{n\to\infty}a_{n},$
we can show that for any $\epsilon>0,$there's some large enough $n_{0}\in\N$,
$\lvert$such that
\[
\sup_{i\geq n_{0}}x_{i}-\inf_{i\geq n_{0}}x_{i}\leq\lvert\inf_{i\geq n_{0}}x_{i}-L\rvert+\lvert\sup_{i\geq n_{0}}x_{i}-L\rvert<\epsilon.
\]
Then for any $m,n\geq n_{0}$
\[
\rvert x_{m}-x_{n}\rvert\leq\sup_{i\geq n_{0}}x_{i}-\inf_{i\geq n_{0}}x_{i}\leq\epsilon
\]
and thus the sequence is Cauchy and converges to $L$ by Proposition
\ref{prop:limSupSubsequential}.
\end{proof}
\begin{prop}
\label{prop:SumLimInf}$\liminf_{n\to\infty}a_{n}+\liminf_{n\to\infty}b_{n}\leq\liminf_{n\to\infty}\left(a_{n}+b_{n}\right)\leq\limsup_{n\to\infty}\left(a_{n}+b_{n}\right)\leq\limsup_{n\to\infty}a_{n}+\limsup_{n\to\infty}b_{n}.$
\end{prop}

\begin{proof}
Note that $\inf_{i\geq n}a_{i}+\inf_{i\geq n}b_{i}\leq\inf_{i\geq n}\left(a_{i}+b_{i}\right)$
and take limits. The other inequalities follow by Propositions \ref{prop:limSupDominatesLimInf}
and \ref{prop:.limInfNegLimSupNeg}.
\end{proof}
\begin{rem*}
We are implicitly excluding the $\infty-\infty$ situations in the
limits. This should be assumed throughout this section.
\end{rem*}
\begin{prop}
\label{prop:prodLimInf}For $a_{n},b_{n}\geq0$ we have that $\liminf a_{n}\liminf b_{n}\leq\liminf a_{n}b_{n}\leq\limsup a_{n}b_{n}\leq\limsup a_{n}\limsup b_{n}.$
\end{prop}

\begin{proof}
Note that we have the inequalities 
\begin{align*}
0 & \leq\inf_{i\geq n}a_{i}\leq a_{i}\\
0 & \leq\inf_{i\geq n}b_{i}\leq b_{i}
\end{align*}
for all $i\geq n$ and so by multiplying the inequalities we get 
\[
0\leq\inf_{i\geq n}a_{i}\inf_{i\geq n}b_{i}\leq a_{i}b_{i}\implies0\leq\inf_{i\geq n}a_{i}\inf_{i\geq n}b_{i}\leq\inf_{i\geq n}a_{i}b_{i}
\]
and taking limits finishes the proof.
\end{proof}
\begin{rem}
Note that the condition that $a_{n},b_{n}\geq0$ is necessary: see
$a_{n}=\left(-1\right)^{n},b_{n}=(-1)^{n+1}$.
\end{rem}

\begin{prop}
\label{prop:limInfMid}$\liminf_{n\to\infty}\left(a_{n}+b_{n}\right)\leq\liminf_{n\to\infty}a_{n}+\limsup_{n\to\infty}b_{n}\leq\limsup_{n\to\infty}\left(a_{n}+b_{n}\right)$
\end{prop}

\begin{proof}
Note that 
\[
a_{n}+b_{n}\leq a_{n}+\sup_{i\geq n}b_{i}\implies\inf_{i\geq n}\left(a_{n}+b_{n}\right)\leq\inf_{i\geq n}\left(a_{n}+\sup_{i\geq n}b_{n}\right)=\inf_{i\geq n}a_{i}+\sup_{i\geq n}b_{i}
\]
where the last equality follows because $\sup b_{i}$ is a constant
for a fixed $n.$ Taking limits then yields the result.
\end{proof}
\begin{prop}
\label{prop:sumLimInfLim}If $\lim_{n\to\infty}a_{n}=L$ then 
\begin{align*}
\liminf_{n\to\infty}\left(a_{n}+b_{n}\right) & =L+\liminf_{n\to\infty}b_{n}\\
\limsup_{n\to\infty}\left(a_{n}+b_{n}\right) & =L+\limsup b_{n}.
\end{align*}
\end{prop}

\begin{proof}
Note first that 
\begin{align*}
L+\liminf_{n\to\infty}b_{n} & =\liminf_{n\to\infty}a_{n}+\liminf_{n\to\infty}b_{N}\\
 & \leq\liminf_{n\to\infty}\left(a_{n}+b_{n}\right)\\
 & \leq\limsup_{n\to\infty}a_{n}+\liminf_{n\to\infty}b_{n}\\
 & =L+\liminf_{n\to\infty}b_{n}
\end{align*}
by Propositions\ref{prop:limSupEqualLimInf}, \ref{prop:SumLimInf},
and \ref{prop:limInfMid}.
\end{proof}
\begin{prop}
If $a_{n},b_{n}\geq0$ then
\[
\liminf_{n\to\infty}a_{n}b_{n}\leq\liminf_{n\to\infty}a_{n}\limsup_{n\to\infty}b_{n}\leq\limsup_{n\to\infty}a_{n}b_{n}.
\]
\end{prop}

\begin{proof}
The proof proceeds in the exact same way as in Proposition \ref{prop:limInfMid}.
\end{proof}
%
\begin{prop}
\label{prop:prodLimInfExists}If $a_{n},b_{n}\geq0$ , $\lim a_{n}=L$
and $\limsup_{n\to\infty}b_{n}<\infty$ then 
\begin{align*}
\liminf\left(a_{n}b_{n}\right) & =L\liminf b_{n}\\
\limsup\left(a_{n}b_{n}\right) & =L\limsup b_{n}.
\end{align*}
\end{prop}

\begin{proof}
This is akin to Proposition \ref{prop:prodLimInfExists}; the following
chain of inequalities establish the claim
\begin{align*}
L\liminf b_{n} & =\liminf a_{n}\liminf b_{n}\\
 & \leq\liminf a_{n}b_{n}\\
 & \leq\limsup a_{n}\liminf b_{n}\\
 & =L\liminf b_{n}.
\end{align*}
\end{proof}
\begin{prop}
\label{prop:limInfContinuity}Let $a_{n}$ be a real sequence and
let $f:\R\to\R$ be a continuous and increasing function, then
\begin{align*}
f\left(\liminf_{n\to\infty}a_{n}\right) & =\liminf_{n\to\infty}f\left(a_{n}\right)\\
f\left(\limsup_{n\to\infty}a_{n}\right) & =\limsup_{n\to\infty}f\left(a_{n}\right).
\end{align*}
\end{prop}

\begin{proof}
First observe that since $f$ is increasing, we have that $\inf_{i\geq n}a_{i}\leq a_{i}\implies f\left(\inf_{i\geq n}a_{i}\right)\leq f\left(a_{i}\right)$
for all $i\geq n$ and so
\[
f\left(\inf_{i\geq n}a_{i}\right)\leq\inf_{i\geq n}f\left(a_{i}\right).
\]
Conversely, fix $n$ and suppose that $f\left(\inf_{i\geq n}a_{i}\right)<\inf_{i\geq n}f\left(a_{i}\right)$.
Fixing $\epsilon=\frac{\inf_{i\geq n}f\left(a_{i}\right)-f\left(\inf_{i\geq n}a_{i}\right)}{2}$
, note that by continuity there exists a $\delta>0$ such that $\lvert\inf_{i\geq n}a_{i}-x\rvert<\delta\implies\lvert f\left(\inf_{i\geq n}a\right)-f\left(x\right)\rvert<\epsilon$.
By the definition of infimum, there is some $i_{0}\geq n$ where $a_{i_{0}}-\inf_{i\geq n}a_{i}<\delta$
and so 
\[
f\left(a_{i_{0}}\right)-f\left(\inf_{i\geq n}a_{i}\right)<\epsilon
\]
which is a contradiction and hence 
\[
f\left(\inf_{i\geq n}a_{i}\right)\geq\inf_{i\geq n}f\left(a_{i}\right).
\]
Taking limits and applying the continuity of $f$ once again yields
the result.
\end{proof}
\begin{prop}
\[
\limsup_{n\to\infty}\max\left\{ a_{n},b_{n}\right\} =\max\left\{ \limsup_{n\to\infty}a_{n},\limsup_{n\to\infty}b_{n}\right\} 
\]
 and 
\[
\liminf_{n\to\infty}\min\left\{ a_{n},b_{n}\right\} =\min\left\{ \liminf_{n\to\infty}a_{n},\liminf_{n\to\infty}b_{n}\right\} .
\]
\end{prop}

\begin{xca}
Let $\left\{ a_{n}\right\} $ and $\left\{ b_{n}\right\} $ be positive
real sequences with such that $\limsup_{n\to\infty}\frac{\log\left(b_{n}\right)}{n}=-\infty$.
Prove that 
\[
\liminf_{n\to\infty}\frac{\log\left(a_{n}+b_{n}\right)}{n}=\liminf_{n\to\infty}\frac{\log\left(a_{n}\right)}{n}.
\]
\end{xca}

\begin{sol*}
Note that since exponential function is increasing and continuous,
by Proposition \ref{prop:limInfContinuity}
\[
\exp\left(\limsup_{n\to\infty}\log\left(b_{n}^{\frac{1}{n}}\right)\right)=\limsup_{n\to\infty}b_{n}^{\frac{1}{n}}=0
\]
which implies that $\lim_{n\to\infty}b_{n}^{\frac{1}{n}}=0$ since
$b_{n}>0$ for all $n\in\N.$ Since eventually, $b_{n}<b_{n}^{\frac{1}{n}}$
we have that $\lim_{n\to\infty}b_{n}=0.$ 
\[
\frac{\log\left(a_{n}+b_{n}\right)}{n}=\frac{\log\left(a_{n}\right)}{n}+\frac{\log\left(1+\frac{b_{n}}{a_{n}}\right)}{n}\leq\frac{\log\left(a_{n}\right)}{n}+
\]

\hl{INCOMPLETE}
\end{sol*}

\subsubsection{\label{subsec:baseBRepresentations}Base-$b$ representations}
\begin{thm}
\label{thm:realDecimalExpansion}For any real number $x\in\left[0,1\right)$
and any positive integer $b\geq2,$ there exists a sequence of non-negative
integers $\left\{ a_{i}\right\} _{i=1}^{\infty}\in\left\{ 0,1,2,\ldots,b-1\right\} $
such that 
\[
\sum_{i=1}^{\infty}\frac{a_{i}}{b^{i}}=x.
\]
Moreover, if there is no $n_{0}\in\mathbb{N}$ such that for all $n\geq n_{0}:a_{n}=b-1,$
then the sequence $\left\{ a_{i}\right\} $ is unique.
\end{thm}

\begin{proof}
Note that we can partition\hl{TODO}
\end{proof}
\begin{example}
\label{exa:isi2004samplepsb11}For $n\geq1$ let $x_{n}=\frac{1}{n\alpha_{n}}$
where $\alpha_{n}$ is such that $2^{\alpha_{n}-2}<n\leq2^{\alpha_{n}-1}$.
Is the series $\sum_{n=1}^{\infty}x_{n}$ convergent or divergent?
Justify your answer.
\end{example}

\subsubsection{Series}

Series are limits of sequences of partial sums of the form $\{\sum_{i=1}^n a_i \}_{n\in\N}$


\subsection{Limits of functions}

\subsection{Differentiation in $\R$\label{subsec:reviewDifferentiation}}

We first review some basic material from single variable calculus
for completeness; note that this section is technically presumed knowledge
and so results appearing previously in the text use some of the facts
established here. We will avoid circularity by keeping this section
self-contained without any reference to the material we have developed
so far in the body of the text (we will use results from the appendices).
\begin{defn}
	\label{def:differentiable}A function $f:\left[a,b\right]\to\R$ is
	said to be \emph{differentiable at a point }$c\in\left(a,b\right)$
	if 
	\[
	\lim_{x\to a}\frac{f\left(x\right)-f\left(a\right)}{x-a}
	\]
	exists as a real number. The function is said to be \emph{differentiable
	}if it is differentiable at every point in the interior of its domain.
	If a function $f$ is differentiable everywhere, the derivative function
	is denoted $f^{\prime}$as in 
	\[
	f^{\prime}\left(y\right)=\lim_{x\to y}\frac{f\left(x\right)-f\left(y\right)}{x-y}.
	\]
\end{defn}

\begin{rem*}
	The derivative is also denoted $\frac{df}{dy}=f^{\prime}\left(y\right)$
	which captures the heuristic that the derivative can be thought as
	the ratio of small quantities.
\end{rem*}
\begin{example}
	\label{exa:derivativeConstant}It should be immediately clearly that
	for any constant function $f\left(x\right)=a$, its derivative is
	zero everywhere, and for the function $f\left(x\right)=x$, its derivative
	is 1 is everywhere.
\end{example}

\begin{prop}
	\label{prop:differentiableImpliesContinuous}Let $f:\left[a,b\right]\to\R$
	be differentiable at a point $c\in\left(a,b\right).$Then it is continuous
	at $c.$
\end{prop}

\begin{proof}
	Recall that the limit of the product of two functions is the product
	of the limit, provided those individual limits exist. 
	\begin{align*}
		\lim_{x\to c}f\left(x\right)-f\left(c\right) & =\lim_{x\to c}\frac{f\left(x\right)-f\left(c\right)}{x-c}x-c\\
		& =\lim_{x\to c}\frac{f\left(x\right)-f\left(c\right)}{x-c}\lim_{x\to c}x-c\\
		& =f^{\prime}\left(c\right)0\\
		& =0
	\end{align*}
	which completes the proof.
\end{proof}
\begin{rem*}
	The converse of this theorem is not true; for instance, consider that
	$\lvert x\rvert$ is continuous (this follows by the ``reverse''
	triangle inequality $\lvert\lvert x\rvert-\lvert y\rvert\rvert\leq\lvert x-y\rvert)$.
	That it is not differentiable can be seen by looking at 
	\[
	\lim_{x\to0^{-}}\frac{\lvert x\rvert}{x}=-1\neq1=\lim_{x\to0^{+}}\frac{\lvert x\rvert}{x}.
	\]
\end{rem*}
\begin{prop}
	\label{prop:algebraOfDifferentiableFunctions}Let $f,g$ be real-valued
	functions on $\left[a,b\right]$ that are differentiable at some $x\in\left[a,b\right]$
	. Then $f+g$ and $fg$ are differentiable at $x.$ If $g\left(x\right)\neq0$
	then $\frac{f}{g}$is also differentaible. The derivatives are given
	\begin{align}
		\left(f+g\right)^{\prime}\left(x\right) & =f^{\prime}\left(x\right)+g^{\prime}\left(x\right)\\
		\left(fg\right)^{\prime}\left(x\right) & =f^{\prime}\left(x\right)g\left(x\right)+g^{\prime}\left(x\right)f\left(x\right)\\
		\left(\frac{f}{g}\right)^{\prime} & \left(x\right)=\frac{f^{\prime}\left(x\right)g\left(x\right)-g^{\prime}\left(x\right)f\left(x\right)}{\left(g\left(x\right)\right)^{2}}
	\end{align}
\end{prop}

\begin{proof}
	The first one, called the sum rule, follows by algebra of limits.
	The second, known as the product rule, follows by the same trick that
	is used to prove that the product of two sequences converges to the
	product of their limits. To reiterate, note that 
	\begin{align*}
		\frac{f\left(y\right)g\left(y\right)-f\left(x\right)g\left(x\right)}{y-x} & =\frac{f\left(y\right)g\left(y\right)-f\left(y\right)g\left(x\right)+f\left(y\right)g\left(x\right)-f\left(x\right)g\left(x\right)}{x-c}\\
		& =\frac{f\left(y\right)\left(g\left(y\right)-g\left(x\right)\right)}{y-x}+\frac{g\left(x\right)\left(f\left(y\right)-f\left(x\right)\right)}{y-x}.
	\end{align*}
	Taking limits and applying Proposition \ref{prop:differentiableImpliesContinuous}
	gets the result.
	
	Finally, the third result, called the quotient rule, follows as 
	\begin{align*}
		\frac{\frac{f\left(y\right)}{g\left(y\right)}-\frac{f\left(x\right)}{g\left(x\right)}}{y-x} & =\frac{1}{g\left(x\right)g\left(y\right)}\left(\frac{f\left(y\right)g\left(x\right)-f\left(x\right)g\left(y\right)}{y-x}\right)\\
		& =\frac{1}{g\left(x\right)g\left(y\right)}\left(\frac{f\left(y\right)g\left(x\right)-f\left(y\right)g\left(y\right)+f\left(y\right)g\left(y\right)-f\left(x\right)g\left(y\right)}{y-x}\right)\\
		& =\frac{1}{g\left(x\right)g\left(y\right)}\left(\frac{f\left(y\right)\left[g\left(x\right)-g\left(y\right)\right]}{y-x}+\frac{g\left(y\right)\left[f\left(y\right)-f\left(x\right)\right]}{y-x}\right)
	\end{align*}
	and taking limits and again applying the continuity of differentiable
	functions completes the proof.
\end{proof}
\begin{cor}[Power rule]
	\label{cor:powerRule}The derivative of a real-valued function on
	$\R$ defined $f\left(x\right)=x^{n}$ where $n\in\mathds{Z}$ is
	given by $f^{\prime}\left(x\right)=nx^{n-1}.$
\end{cor}

\begin{proof}
	First we examine the case when $n\in\N$. For the base cases of $n=0$
	and $n=1$ look at the remark above. Now suppose that the rule holds
	for $n\in\N$. Then, writing $x^{n+1}=xx^{n},$ we apply the product
	rule to yield
	\[
	\frac{dx^{n+1}}{dx}=xnx^{n-1}+x^{n}=\left(n+1\right)x^{n}
	\]
	which completes the proof.
	
	To extend the result to negative integers, we can apply the quotient
	rule. Note that for a negative integer $m$, we have some positive
	integern $n$ such that $m=-n$ and so $f\left(x\right)=x^{m}=\frac{1}{x^{n}}$
	which has derivative (by applying the power rule for positive integers
	and the quoteint rule)
	\[
	f^{\prime}\left(x\right)=\frac{-nx^{n-1}}{x^{2n}}=mx^{m-1}.
	\]
\end{proof}
\begin{rem*}
	The power rule also holds for rational exponents; we shall establish
	this fact after we state the chain rule. For real exponents, the power
	rule holds if $x\geq0$ since we define $x^{a}:=\exp\left(a\log\left(x\right)\right)$
	when $a\in\R$. Since the logarithm is only defined on the positive
	reals, the result only extends to that case (the case of $x=0$ is
	trivial). We shall prove this after we construct the exponential and
	natural logarithm functions in Appendix section \ref{sec:specialFunctions}.
\end{rem*}
\begin{prop}
	\label{prop:differentiabilityOfPolynomials}Let $f\in P_{n}\left(x\right)$
	the space of $n-$degree polynomials. Then $f$ is differentiable
	and its derivative $f^{\prime}\in P_{n-1}\left(x\right).$
\end{prop}

\begin{proof}
	Let $f\left(x\right)=\sum_{i=0}^{n}a_{i}x^{i}.$ By the ``sum rule'',''product
	rule'' and the ``power rule'', 
	\[
	f^{\prime}\left(x\right)=\sum_{i=0}^{n}a_{i}ix^{i-1}\in P_{n-1}\left(x\right).
	\]
\end{proof}
\begin{rem*}
	An immediate corrolary is that every polynomial is infinitely differentiable.
	Infinitely differentiable functions are called \emph{smooth.}
\end{rem*}
\begin{thm}[Chain rule]
	\label{thm:chainRuleR}Let $g:C\subseteq\R\to\R$ be differentiable
	at some $c\in D$ and let $f:D\subseteq\R\to\R$ be differentiable
	at $g\left(c\right)\in D.$ Then the composition $f\circ g$ is differentiable
	at $c$ with derivative 
	\[
	\left(f\circ g\right)^{\prime}\left(c\right)=f^{\prime}\left(g\left(c\right)\right)g^{\prime}\left(c\right).
	\]
\end{thm}

We omit the proof of this theorem right now; we will prove it in the
more general setting of derivatves on functions between Banach spaces
in the next section.
\begin{prop}
	\label{prop:rationalPowerRule}A function $f\left(x\right)=x^{r}$
	where $r\in\mathds{Q}$ and $x\in\R$ is differentiable with derivative
	$f^{\prime}\left(x\right)=rx^{r-1}.$
\end{prop}

\begin{proof}
	We will establish the result when $x\neq0$ (when $x=0$, the derivative
	is 0 as established in an earlier remark). Let $r=\frac{p}{q}$ where
	$p,q\in\mathds{Z}$ and note that $g\left(x\right):=x^{q}$ has derivative
	$g^{\prime}\left(x\right)=qx^{q-1}$ by the integral power rule\ref{cor:powerRule}.
	Then, observe that $g\left(f\left(x\right)\right)=x^{p}$ and so applying
	the chain rule on the left hand hand side and the integral power rule
	on the right hand side, we have
	\begin{align*}
		qf\left(x\right)^{q-1}f^{\prime}\left(x\right) & =px^{p-1}
	\end{align*}
	which can be re-arranged to yield
	\begin{align*}
		f^{\prime}\left(x\right) & =\frac{px^{p-1}}{qf\left(x\right)^{q-1}}\\
		& =\left(\frac{p}{q}\right)x^{p-1-\frac{p}{q}\left(q-1\right)}\\
		& =\frac{p}{q}x^{\frac{p}{q}-1}\\
		& =rx^{r-1}
	\end{align*}
	which completes the proof.
\end{proof}
\begin{example}[ISI 2013 PSB 2]
	\label{exa:isi2013psb2} Let $a_{1}<a_{2}<\cdots<a_{m}$ and $b_{1}<b_{2}<\cdots<b_{n}$
	be real numbers such that 
	\[
	\sum_{i=1}^{m}\left|a_{i}-x\right|=\sum_{j=1}^{n}\left|b_{j}-x\right|\text{ for all }x\in\mathbb{R}.
	\]
	We can use differentiability to show that $m=n$ and $a_{j}=b_{j}$
	for$1\leq j\leq n$. To see this, note that $f\left(x\right):=\sum_{i=1}^{m}\lvert a_{i}-x\rvert$
	is not differentiable exactly on the set $F:=\left\{ a_{1},\ldots,a_{n}\right\} .$
	This is because eacj component $\lvert a_{i}-x\rvert$ is non-differentiable
	only at $a_{i}$and the sum of differentiable and non-differentiable
	functions are not differentiable. Similarly, the function $g\left(x\right):=\sum_{i=1}^{n}\lvert b_{i}-x\rvert$
	is exactly not differentiable on the set $G$ which implies $F=G$.
\end{example}


\subsubsection{Mean value theorems}
\begin{defn}
	\label{def:localMax}A function $f:\left[a,b\right]\to\R$ is set
	to have a local maximum at $c\in\left[a,b\right]$ if there exists
	some $\delta>0$ such that for every $x\in\left(c-\delta,c+\delta\right)\cap\left[a,b\right]$
	\[
	f\left(c\right)\geq f\left(x\right).
	\]
	Local minima are defined analagously.
\end{defn}

\begin{prop}
	\label{prop:firstOrderConditionR}Let $f:\left[a,b\right]\to\R$ have
	a local maximum or minimum at $c\in\left(a,b\right).$ If $f^{\prime}\left(c\right)$
	exists then $f^{\prime}\left(c\right)=0.$
\end{prop}

\begin{proof}
	We consider the case of the local maximum; the case of the local minimum
	follows by applying the maximum result to $-f.$ Suppose $c\in\left(a,b\right)$
	is a local maximum for $f$, in which case we can choose some $\delta>0$
	small enough that 
	\[
	\left(c-\delta,c+\delta\right)\subseteq\left(a,b\right)
	\]
	and $f\left(c\right)\geq f\left(x\right)$ on $\left(c-\delta,c+\delta\right).$
	Note that for $c-\delta<x<c$ we have that 
	\[
	\frac{f\left(x\right)-f\left(c\right)}{x-c}\geq0
	\]
	and so 
	\[
	f^{\prime}\left(c^{-}\right):=\lim_{x\to c^{-}}\frac{f\left(x\right)-f\left(c\right)}{x-c}\geq0.
	\]
	Similarly, for $c<x<c+\delta$
	\[
	\frac{f\left(x\right)-f\left(c\right)}{x-c}\leq0
	\]
	and so 
	\[
	f^{\prime}\left(c^{+}\right):=\lim_{x\to c^{+}}\frac{f\left(x\right)-f\left(c\right)}{x-c}\leq0.
	\]
	But of course, $f^{\prime}\left(c\right)=f^{\prime}\left(c^{-}\right)=f^{\prime}\left(c^{+}\right)$
	since left and right hand limits are always equal to the limit when
	it exists and thus 
	\[
	f^{\prime}\left(c\right)=0.
	\]
\end{proof}
\begin{thm}[Cauchy's mean value theorem]
	\label{thm:cauchyMeanValue}If $f,g$ are real-valued continuous
	functions on $\left[a,b\right]$ such that the derivative functions
	$f^{\prime},g^{\prime}$ exist on $\left(a,b\right)$ , there exists
	some $x\in\left(a,b\right)$ such that 
	\[
	\left(f\left(b\right)-f\left(a\right)\right)g^{\prime}\left(x\right)=\left(g\left(b\right)-g\left(a\right)\right)f^{\prime}\left(x\right)
	\]
\end{thm}

\begin{proof}
	Write 
	\[
	h\left(t\right):=\left(f\left(b\right)-f\left(a\right)\right)g\left(t\right)-\left(g\left(b\right)-g\left(a\right)\right)f\left(t\right)
	\]
	and observe that since it's built out of differeniable functions by
	adding and multiplying them, $h$ is continuous and differentiable.
	Since $\left[a,b\right]$ is compact, $h$ has a minimum and maximum
	by Weierstrass'. If the minimum and maximum are both at the end points
	$a$ and $b$, then the function is constant and the derivative is
	zero everywhere, which yields the result. If at least one of the minimum
	or maximum is interior, say at $x\in\left(a,b\right)$, then by Proposition
	\ref{prop:firstOrderConditionR}, $f^{\prime}\left(x\right)=0$ which
	again yields the result.
\end{proof}
\begin{cor}[Mean~ value theorem]
	\label{cor:meanValueThm}Let $f:\left[a,b\right]\to\R$ be continuous
	on $\left[a,b\right]$ and differentiable on $\left(a,b\right).$
	Then there exists some $x\in\left(a,b\right)$ such that
	\[
	f^{\prime}\left(c\right)=\frac{f\left(b\right)-f\left(a\right)}{b-a}.
	\]
\end{cor}

\begin{proof}
	Apply Cauchy's mean value theorem with $g\left(x\right)=x$.
\end{proof}
\begin{prop}
	\label{prop:derivativeAndMonotonicity}If $f:\left[a,b\right]\to\R$
	is differentiable on $\left(a,b\right)$ then
	\begin{align*}
		\forall x\in\left(a,b\right):f^{\prime}\left(x\right) & \geq0\implies\left(x_{1}\geq x_{2}\implies f\left(x_{1}\right)\geq f\left(x_{2}\right)\right)\\
		\forall x\in\left(a,b\right):f^{\prime}\left(x\right) & \leq0\implies\left(x_{1}\geq x_{2}\geq f\left(x_{1}\right)\leq f\left(x_{2}\right)\right)\\
	\end{align*}
\end{prop}

\begin{proof}
	Applying the mean value theorem, we have 
	\[
	f\left(x_{1}\right)-f\left(x_{2}\right)=f^{\prime}\left(c\right)\left(x_{1}-x_{2}\right)
	\]
	for $c\in\left(a,b\right)$ and if $f^{\prime}\left(c\right)\geq0$
	then $x_{1}\geq x_{2}$ implies that $f\left(x_{!}\right)\geq f\left(x_{2}\right)$
	and the other implication follows in the same way.
\end{proof}
\begin{rem*}
	An obvious corollary of the above result is that if $\forall x\in\left(a,b\right)$$:f^{\prime}\left(x\right)=0\implies f$
	is constant on $\left(a,b\right)$.
\end{rem*}

\subsubsection{Continuity of derivatives}

For a function $f$ on $\left[a,b\right]$ that is differentiable
everywhere on $\left(a,b\right)$, it need not be the case that the
derivative function $f^{\prime}$ is continuous everywhere. In fact,
the points of discontinuity of a derivative can be dense in the domain
and have positive Lebesgue measure. We postpone the discussion of
these more complicated examples until we prove the fundamental theorem
of calculus. The simplest canonical example of a discontinuous derivative
is the following
\begin{example}
	\label{exa:discontinuousDerivative}Let $f\left(x\right)=x^{2}\sin\left(\frac{1}{x}\right)\indicate\left\{ x\neq0\right\} .$We
	take on faith the fact that the function $\sin\left(x\right)$ is
	bounded between $-1$ and 1, is infinitely differentiable and that
	its derivative is $\cos\left(x\right)$. We construct these functions
	from first principles in Appendix section \ref{sec:specialFunctions}.
	Note that the derivative of this function away from zero is given
	by the chain and product rules as 
	\[
	f^{\prime}\left(x\right)=2x\sin\left(\frac{1}{x}\right)-\cos\left(\frac{1}{x}\right).
	\]
	The derivative at zero can be computed from first principles by looking
	at the limit
	\[
	\lim_{x\to0}\frac{x^{2}\sin\left(\frac{1}{x}\right)\indicate\left\{ x\neq0\right\} }{x}=\lim_{x\to0}x\sin\left(\frac{1}{x}\right)\indicate\left\{ x\neq0\right\} =0
	\]
	since $\lvert\sin\left(\frac{1}{x}\right)\rvert\leq1$. Then we have
	that 
	\[
	f^{\prime}\left(x\right)=\begin{cases}
		2x\sin\left(\frac{1}{x}\right)-\cos\left(\frac{1}{x}\right), & x\neq0\\
		0, & x=0
	\end{cases}
	\]
	but $\lim_{x\to0}f^{\prime}\left(x\right)$ does not exist since $\cos\left(\frac{1}{x}\right)$
	oscillates rapidly near zero.
\end{example}

While derivatives can be discontinuous, they share the intermediate
value property with continuous functions.
\begin{prop}[Darboux's theorem]
	\label{prop:darbouxTheorem}Let $f:\left[a,b\right]\to\R$ be differentiable
	on $\left(a,b\right).$ Then the derivative function $f^{\prime}:\left(a,b\right)\to\R$
	has the intermediate value property that the image $f\left[\left(a,b\right)\right]$
	is an interval.
\end{prop}

\begin{proof}
	Let $c,d\in\left(a,b\right)$ be such that, without loss generality,
	$f^{\prime}\left(c\right)>f^{\prime}\left(d\right).$ Take any $y\in\left(f^{\prime}\left(c\right),f^{\prime}\left(d\right)\right)$.
	The function $g\left(x\right)=f\left(x\right)-yx$ is continuous on
	$\left[\min\left\{ c,d\right\} ,\max\left\{ c,d\right\} \right]$
	and thus achieves a maximum and a minimum by Weierstrass' theorem.
	If both the minimum and maximum occur at the end points then the function
	$g$ is constant on $\left[\min\left\{ c,d\right\} ,\max\left\{ c,d\right\} \right]$
	and so its derivative is zero everywhere, completing the proof. If
	there's at least one interior extrema, say at $t\in\left(\min\left\{ c,d\right\} ,\max\left\{ c,d\right\} \right),$
	then
	\[
	f^{\prime}\left(t\right)=y
	\]
	by Proposition \ref{prop:firstOrderConditionR}.
\end{proof}
\begin{lem}
	\label{lem:IVTJump}Any function that satisifies the intermediate
	value property cannot have removable or jump discontinuities.
\end{lem}

\begin{proof}
	Let $f:\left[a,b\right]\to\R$ be a function that satisfies the intermediate
	value property and let $c\in\left(a,b\right)$ such that $\lim_{x\to c^{-}}f\left(x\right)=L$
	and $\lim_{x\to c^{+}}f\left(x\right)=M$. Suppose, without loss of
	generality, that $L\neq f\left(c\right)$ and let $\epsilon=\frac{\lvert L-f\left(c\right)\rvert}{2}.$
	Note that by definition, there exists some $\delta>0$ such that for
	any $x$ with $0<c-x<\delta\implies\lvert f\left(x\right)-L\rvert<\epsilon$
	which implies that $\lvert f\left(x\right)-f\left(c\right)\rvert\geq\epsilon.$
	By the intermediate value property, for any $y$ in between $f\left(x\right)$
	and $f\left(c\right)$, there exists some $z\in\left(x,c\right)$
	such that $f\left(z\right)=y.$ In particular, this holds true for
	any $\text{y}.$such that $\lvert y-f\left(c\right)\rvert<\epsilon.$
	But then $0<c-z<\delta$ which implies that $\lvert y-f\left(c\right)\rvert\geq\epsilon$
	which is a contradiciton.
\end{proof}

\subsubsection{Differentiability classes and local approximation by polynomials}
\begin{example}
	\label{exa:isi2007samplepsb3}Let $f$ be a function such that $f(0)=0$
	and $f$ has derivatives of all order. Show that 
	\[
	\lim_{h\rightarrow0}\frac{f(h)+f(-h)}{h^{2}}=f^{\prime\prime}(0)
	\]
	where $f^{\prime\prime}(0)$ is the second derivative of $f$ at
	0 .\hl{TODO}
\end{example}


\section{Metric spaces}

\subsection{Topology of metric spaces}

\subsection{Complete metric spaces}

\subsection{Compact metric spaces}

\subsection{Continuity of functions between metric spaces}
\begin{defn}
\label{def:continousFunction}Let $\left(X,d_{X}\right)$ and $\left(Y,d_{Y}\right)$
be a metric space. A function $f:X\to Y$ is said to be \emph{continuous
at a point} $c\in X$ if for any sequence $x_{n}\to c$ we have that
$f\left(x_{n}\right)\to f\left(c\right)$. The function $f$ is called
\emph{continuous }if it is continuous at every point $c\in X$.
\end{defn}

\begin{prop}
\label{prop:equivalentContinuity}Let $\left(X,d_{X}\right)$ and
$\left(Y,d_{Y}\right)$ be a metric space. A function $f:X\to Y$
is said to be \emph{continuous at a point} $c\in X$ if and only if
for every $\epsilon>0$ there exists a $\delta$ such that for any
$x\in X$ $d_{X}\left(x,c\right)<\delta\implies d_{Y}\left(f\left(x\right),f\left(c\right)\right)<\epsilon$.
\end{prop}

\begin{proof}
Suppose that the $\epsilon-\delta$ definition doesnt hold i.e. there
exists some $\epsilon_{0}>0$ such that for all $\delta>0$ there
exists some $x$ such that $d_{X}\left(x,c\right)<\delta$ but $d_{Y}\left(f\left(x\right),f\left(c\right)\right)\geq\epsilon_{0}.$
Letting $\delta=1/n,$ we can find a corresponding $x_{n}$ such that
$d_{X}\left(x_{n},c\right)<\frac{1}{n}$ but $d_{Y}\left(f\left(x_{n}\right),f\left(c\right)\right)\geq\epsilon_{0}$.
Since this can be done for any $n\in\N$, we have a seqeunce $x_{n}\to c$
but $f\left(x_{n}\right)\not\to c$.

Conversely, suppose the $\epsilon-\delta$ definition holds and let
$x_{n}\to c$ by arbitrary. For a large enough $N,$ $d_{X}\left(x_{n},c\right)<\delta$
for all $n\geq N$ and so $d_{Y}\left(f\left(x_{n}\right),f\left(c\right)\right)<\epsilon$
which completes the proof.
\end{proof}
\begin{prop}
\label{prop:openPreImageMetric}Let $\left(X,d_{X}\right)$ and $\left(Y,d_{Y}\right)$
be a metric space. A function $f:X\to Y$ is said to be \emph{continuous
if and only if for every open set $O_{Y}\in\tau_{Y}$ (here $\tau_{y}$
denotes the topology on $Y$) the preimage $f^{-1}\left[o_{Y}\right]\in\tau_{X}$.}
\end{prop}

\begin{proof}
First assume that $f$ is continuous and let $O_{Y}$ be an open set
in $Y$. We wish to show that for any $x\in f^{-1}\left[O_{Y}\right],$there
is some open ball $B\left(x,\delta\right)\subseteq f^{-1}\left[O_{y}\right].$
Note that since $O_{Y}$ is open, there exists some ball $B\left(f\left(x\right),\epsilon\right)\subseteq O_{Y}.$
Note that by continuity, there exists some $\delta>0$ such that $f\left[B\left(x,\delta\right)\right]\subseteq B\left(f\left(x\right),\epsilon\right)$.
Then by the property of preimages
\[
B\left(x,\delta\right)\subseteq f^{-1}f\left[B\left(x,\delta\right)\right]\subseteq f^{-1}\left[o_{Y}\right]
\]
which proves the claim.

Conversely, suppose that preimages of open sets are open. Fix $\epsilon>0$
and note that for any $x\in X$, the preimage of the ball $f^{-1}\left[B\left(f\left(x\right),\epsilon\right)\right]$
is open and thus there exists some $\delta>0$ such that $B\left(x,\delta\right)\subseteq f^{-1}\left[B\left(f\left(x\right),\epsilon\right)\right].$
Then
\[
f\left[B\left(x,\delta\right)\right]\subseteq ff^{-1}\left[B\left(f\left(x\right),\epsilon\right)\right]\subseteq B\left(f\left(x\right),\epsilon\right)
\]
which completes the proof.
\end{proof}
\begin{prop}
Let $\left(X,d_{X}\right),\left(Y,d_{Y}\right),$ and $\left(Z,d_{Z}\right)$
be metric spaces. Then for any continuous functions $f:X\to Y$ and
$g:Y\to Z$, the composition $g\circ f:X\to Z$ given by
\[
g\circ f\left(x\right)=g\left(f\left(x\right)\right)
\]
is continuous.
\end{prop}

\begin{proof}
Let $O_{Z}$ be an open set in $Z$. Note that $\left(g\circ f\right)^{-1}\left[O_{Z}\right]=f^{-1}\left[g^{-1}\left[O_{Z}\right]\right]$
and $g^{-1}\left[O_{Z}\right]$ is open since $g$ is continuous and
so $f^{-1}\left[g^{-1}\left[O_{Z}\right]\right]$ is open.
\end{proof}
\begin{rem*}
This proof applies \emph{mutis mutandis }to continuous functions between
general topological spaces, of which metric spaces are a special case.
\end{rem*}
\begin{defn}
\label{def:lipschitz}Let $\left(X,d_{X}\right)$ and $\left(Y,d_{Y}\right)$
be a metric spaces. A function $f:X\to Y$ is called \emph{Lipschitz
continuous }if there exists some $K>0$ such that 
\[
d_{Y}\left(f\left(x\right),f\left(y\right)\right)\leq Kd_{X}\left(x,y\right).
\]
\end{defn}

\begin{prop}
\label{prop:lipschitzContinuous}Let $\left(X,d_{X}\right)$ and $\left(Y,d_{Y}\right)$
be a metric spaces. Every Lipschitz continuous function $f:X\to Y$
is uniformly continuous.
\end{prop}

\begin{proof}
Fix $\epsilon>0$ and note that if $\delta=\frac{\epsilon}{K}$ then
$d_{X}\left(x,y\right)<\delta\implies d_{Y}\left(f\left(x\right),f\left(y\right)\right)<\epsilon.$
\end{proof}
\begin{prop}
\label{prop:lipschitzBoundedDerivative}$Let$$f:D\subseteq\R\to\R$
be a differentiable function. Then $f$ is Lipschitz if and only if
it has a bounded derivative.
\end{prop}

\begin{proof}
First suppose that the function has a bounded derivative. Then there
exists some $M>0$ such that for any $x\in D$, $\lvert f^{\prime}\left(x\right)\rvert\leq M.$
Then for any $x,y\in D$, the mean value theorem implies that there
exists some $c\in\left(\min\left\{ x,y\right\} ,\max\left\{ x,y\right\} \right)$
such that 
\[
\left\lvert \frac{f\left(x\right)-f\left(y\right)}{x-y}\right\rvert =\lvert f^{\prime}\left(c\right)\rvert\leq M
\]
Rearranging yields the result.

Conversely, suppose that the function is Lipschitz with constant $K$.
Then
\[
\left\lvert \frac{f\left(x\right)-f\left(y\right)}{x-y}\right\rvert \leq K
\]
and taking limits $x\to y$ gives the result.
\end{proof}
\begin{defn}
\label{def:cauchyContinuity}Let $\left(X,d_{X}\right)$ and $\left(Y,d_{Y}\right)$
be metric spaces. A function $f:X\to Y$ is called \emph{Cauchy-continuous
}if for any Cauchy sequence $\left\{ x_{n}\right\} \in X$, the image
$\left\{ f\left(x_{n}\right)\right\} $ is Cauchy.
\end{defn}

\begin{prop}
\label{prop:CauchyContinuityImpliesContinuity}Let $\left(X,d_{X}\right)$
and $\left(Y,d_{Y}\right)$ be metric spaces. If $f:X\to Y$ is Cauchy
continuous then it is continuous. Moreover, if $X$ is complete then
every continuous function $f:X\to Y$ is also Cauchy-continuous.
\end{prop}

\begin{proof}
Fix $\epsilon>0$ and let $c\in X$ be arbitrary. Then there exists
at least one (possibly eventually constant) Cauchy sequence $x_{m}\to c.$.
Now consider a new sequence $y_{n}$ such that $y_{n}=x_{n}$ for
even $n$and $y_{n}=c$ for odd $n.$ Then $y_{n}\to c$ (and is Cauchy)
and so $f\left(y_{n}\right)$ is Cauchy. This implies that for large
$n$
\[
d\left(f\left(x_{n}\right),f\left(c\right)\right)<\epsilon
\]
which implies $f\left(y_{n}\right)\to f\left(c\right).$ Extracting
the even indexed subsequence shows that $f\left(x_{n}\right)\to c.$

Next, suppose that $X$ is complete and $f$ is continuous. Let $\left\{ x_{n}\right\} $
be a Cauchy sequence. Then by completenes $x_{n}\to c\in X$. By continuity,
$f\left(x_{n}\right)\to f\left(c\right)$ and so $\left\{ f\left(x_{n}\right)\right\} $
is Cauchy.
\end{proof}
\begin{prop}
\label{prop:uniformContinuityImpliesCauchyContinuity}Let $\left(X,d_{X}\right)$
and $\left(Y,d_{Y}\right)$ be a metric spaces. If a function $f:X\to Y$
is uniformly continuous, it is Cauchy continuous.
\end{prop}

\begin{proof}
Fix $\epsilon>0$ and let $\left\{ x_{n}\right\} \in X$ be a Cauchy
sequence. By uniform continuity, there exists some $\delta>0$ such
that for any $m,n\in\N$ $d_{X}\left(x_{n},x_{m}\right)<\delta\implies d_{Y}\left(f\left(x_{n}\right),f\left(x_{m}\right)\right)<\epsilon.$
Since $\left\{ x_{n}\right\} $ is Cauchy, there exists some $N$
such that for any $m,n\geq N$ $d_{X}\left(x_{n},x_{m}\right)<\delta$
which completes the proof.
\end{proof}
\begin{thm}
\label{thm:compactUniformContinuity}Let $\left(X,d_{X}\right)$ be
a compact metric space and let $\left(Y,d_{Y}\right)$ be an arbitrary
metric space. The following are equivalent for a function $f:X\to Y$

\begin{enumerate}[label=(\roman*),leftmargin=.1\linewidth,rightmargin=.4\linewidth]
	\item f is continuous
	\item f is uniformly continuous
	\item f is Cauchy continuous.
\end{enumerate}
\end{thm}

\begin{proof}
We prove $(i)\implies(ii)$; $(ii)\implies(iii)$ is Proposition \ref{prop:uniformContinuityImpliesCauchyContinuity}
and $(iii)\implies(i)$ is Proposition \ref{prop:CauchyContinuityImpliesContinuity}.

Let $\epsilon>0$ be fixed. By continuity, we know that for every
$c\in X$ there exists some $\delta_{c}>0$ such that when $d_{X}\left(x,c\right)<\delta_{c}\implies d_{Y}\left(f\left(x\right),f\left(c\right)\right)<\frac{\epsilon}{2}$.
Note that the balls $\left\{ B_{d_{X}}\left(c,\frac{\delta_{c}}{2}\right)\right\} _{c\in X}$
form a cover of $X$ and by compactness we can extract a finite subcover
$\left\{ B_{d_{X}}\left(c_{i},\frac{\delta_{c_{i}}}{2}\right)\right\} _{i=1}^{n}$.
Let $\delta:=\min_{i}\frac{\delta_{c_{i}}}{2}$ and note that for
any $x,y\in X$ , there's some $1\leq i_{0}\leq n$ such that $x\in B_{d_{X}}\left(c_{i_{0}}\frac{d_{c_{i_{0}}}}{2}\right).$
Then, if $d_{X}\left(x,y\right)<\delta$
\begin{align*}
d_{X}\left(c_{i_{0}},y\right) & \leq d_{X}\left(x,y\right)+d_{X}\left(x,c_{i_{0}}\right)\\
 & <\delta+\frac{\delta_{c_{i_{0}}}}{2}\\
 & \leq\delta_{c_{i_{0}}}
\end{align*}
Then,
\begin{align*}
d_{Y}\left(f\left(x\right),f\left(y\right)\right) & \leq d_{Y}\left(f\left(x\right),f\left(c_{i_{0}}\right)\right)+\delta_{Y}\left(f\left(y\right),f\left(c_{i_{0}}\right)\right)\\
 & <\frac{\epsilon}{2}+\frac{\epsilon}{2}\\
 & =\epsilon
\end{align*}
which completes the proof.
\end{proof}
\begin{defn}
Let $\left(X,d_{X}\right)$ and $\left(Y,d_{Y}\right)$ be a metric
spaces. A function $f:X\to Y$ is called a contraction if it is Lipschitz
with a Lipschitz constant $0<K<1$. In other words, for any $x,y\in X$
\[
d_{Y}\left(f\left(x\right),f\left(y\right)\right)\leq Kd_{X}\left(x,y\right).
\]
\end{defn}

\begin{thm}[Banach Fixed Point Theorem]
\label{thm:banachFixedPoint}Let $\left(X,d\right)$ be a complete
metric space and let a function $f:X\to X$ be a contraction. Then
there exists a unique $x\in X$ such that $f\left(x\right)=x$.
\end{thm}

\begin{proof}
Let $x_{0}\in X$ be arbitrary, define a sequence $x_{n+1}=f\left(x_{n}\right)$
and fix $\epsilon>0$. For any $n\geq m\geq1$
\begin{align}
d\left(x_{n},x_{m}\right) & \leq d\left(x_{n},x_{n-1}\right)+d\left(x_{n-1},x_{n-2}\right)+\ldots+d\left(x_{m+1},x_{m}\right)\nonumber \\
 & \leq K^{n-1}d\left(x_{1},x_{0}\right)+K^{n-2}d\left(x_{1},x_{0}\right)+\ldots+K^{m}d\left(x_{1},x_{0}\right)\nonumber \\
 & =d\left(x_{1},x_{0}\right)\sum_{i=m}^{n-1}K^{i}\label{eq:contractionCauchy}
\end{align}
where the first inequalitity is simply the triangle inequality and
the second is induction on the contraction property. Note that since
$\left\{ \sum_{i=1}^{n}K^{i}\right\} _{i=1}^{\infty}$ is a convergent
geometric series, it's Cauchy as in we can find an $N\in\N$ such
that for all $n,m\geq N$ we have that $\lvert\sum_{i=1}^{n}K_{i}-\sum_{i=1}^{m}K_{i}\rvert=\sum_{i=\min\left\{ n.m\right\} -1}^{\max\left\{ n.m\right\} }K^{i}<\epsilon.$
Therefore, for $m,n\geq N$, the inequality (\ref{eq:contractionCauchy})
implies
\[
d\left(x_{m},x_{n}\right)\leq d\left(x_{1},x_{0}\right)\epsilon.
\]
Since $\epsilon$ can be arbitrarily small, our sequence $x_{n}$
is Cauchy and so by the completeness of $X$ it converges to some
limit $x.$ Then, by the continuity of $f$
\begin{align*}
x & =\lim_{n\to\infty}x_{n}\\
 & =\lim_{n\to\infty}f(x_{n-1})\\
 & =f(x).
\end{align*}
which yields our fixed point.

Now suppose that there were two fixed points i.e. $x\neq y$ such
that $f\left(x\right)=x$ and $f\left(y\right)=y$. By the contraction
property,
\[
d\left(x,y\right)\leq Kd\left(x,y\right)
\]
which is a contradiction.
\end{proof}

\subsubsection{Continuity of real valued function on $\protect\R$}
\begin{prop}
\label{prop:strictlyIncreasingContinuousFunctionInverse}Let $f:\left[a,b\right]\to\R$
be a strictly increasing, continuous function. Then $f$ is bijective
on its range, and its inverse is strictly increasing and continuous.
\end{prop}

\begin{proof}
Note that injectivity is straightforward since if $x\neq y$ then
either $x>y$ or $y<x$ in which case $f\left(x\right)>f\left(y\right)$
or $f\left(x\right)<f\left(y\right)$, respectively. To see that the
inverse is strictly increasing, let $c,d\in R$ where $R\subseteq\R$
is the range of $f$. If $c>d$, then since $f$ is strictly increasing
it must be that $f^{-1}\left(c\right)>f^{-1}\left(d\right)$. Finally,
notice that $f^{-1}$ is a strictly increasing function that is onto
$\left[a,b\right]$. Suppose that $y_{0}\in R$ is a point of discontinuity
for $f^{-1}$, then $f^{-1}\left(y_{0}^{-}\right)<f^{-1}\left(y_{0}^{+}\right)$
where both limits are in $\left[a,b\right].$ Then of course there's
some $f^{-1}\left(y_{0}^{-}\right)<c<f^{-1}\left(y_{0}^{+}\right)$
such that there's no $y\in R$ such that $f^{-1}\left(y\right)=c$
which is a contradiction since $f^{-1}$ is onto $\left[a,b\right].$
\end{proof}
\begin{example}
\label{exa:isi2008samplepsb3} Let $g$ be a continuous function with
$g(1)=1$ such that 
\[
g(x+y)=5g(x)g(y)
\]
 for all $x,y$. Find $g(x)$. {[}Hint: You may use the following
result. If $f$ is a continuous function that satisfies $f(x+y)=f(x)+f(y)$
for all $x,y$, then $f(x)=xf(1)$.\hl{TODO}
\end{example}


\subsection{Separable metric spaces}

\section{Special functions\label{sec:specialFunctions}}
