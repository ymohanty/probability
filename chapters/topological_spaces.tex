
\chapter{Topological spaces}

For completeness, we state and prove all theorems from basic real
analysis and general topology that were used in the main text.

\section{Elementary real analysis}

\subsection{Sequences and series}

\subsubsection{Properties of Limit Inferiors and Superiors}

We collect here the following properties of the limit superiors and
limit inferiors. In all of the following, $\left\{ a_{n}\right\} $
and $\left\{ b_{n}\right\} $ are real sequences. Recall that $\limsup_{n\to\infty}a_{n}=\inf_{n\geq1}\left\{ \sup_{i\geq n}\left\{ a_{i}\right\} \right\} $
and $\liminf_{n\to\infty}a_{n}=\sup_{n\geq1}\left\{ \inf_{i\geq n}\left\{ a_{i}\right\} \right\} $.
Note that these always exist (in the extended reals) by the completeness
of the real numbers.
\begin{prop}
\label{prop:.limInfNegLimSupNeg}$\liminf_{n\to\infty}a_{n}=-\limsup_{n\to\infty}-a_{n}$.
\end{prop}

\begin{proof}
Note that $\inf_{i\geq n}a_{i}=-\sup_{i\geq n}-a_{i}$ and so taking
limits completes the proof.
\end{proof}
\begin{prop}
\label{prop:limSupDominatesLimInf}$\liminf_{n\to\infty}a_{n}\leq\limsup_{n\to\infty}a_{n}$.
\end{prop}

\begin{proof}
$\inf_{i\geq n}a_{i}\leq\sup_{i\geq n}a_{i}$ and so the result follows
by taking limits.
\end{proof}
\begin{prop}
\label{prop:limSupSubsequential}$\liminf_{n\to\infty}a_{n}$ and
$\limsup_{n\to\infty}a_{n}$ are subsequential limits of $a_{n}$
such that for any other subsequential limit $L$ we have that 
\[
\liminf_{n\to\infty}a_{n}\leq L\leq\limsup_{n\to\infty}a_{n}.
\]
\end{prop}

\begin{proof}
Note that if $L<\liminf_{n\to\infty}a_{n}$ , then there are infinitely
many elements of $\left\{ a_{n}\right\} $ in $\left(L-\epsilon,L+\epsilon\right)$where
$\epsilon>0$ is small enough that $L+\epsilon<\liminf_{n\to\infty}a_{n}$.
Note that since $\left\{ \inf_{i\geq n}a_{i}\right\} $ is a nondecreasing
sequence, there exists some $n_{0}\in\N$ such that for all $n\geq n_{0}$,
$\inf_{i\geq n}a_{i}>L+\epsilon$. But then $a_{i}\geq\inf_{i\geq n_{0}}a_{i}>L+\epsilon$
for all every $i\geq n_{0}$ which contradicts the fact that infinitely
many of our $\left\{ a_{n}\right\} $ were in $\left[L,L+\epsilon\right)$.
The proof for the limit superior case is analagous.

Next fix $\epsilon>0$ , let $c_{n}:=\inf_{i\geq n}a_{i}$., and let
$I:=\lim c_{n}.$By the definition of infimum, there exists some $k_{n}\geq n$
such that $\lvert c_{n}-a_{k_{n}}\rvert<\frac{\epsilon}{2}$. By definition
of limit inferiors, we have that for large enough $n$, $\lvert c_{n}-I\rvert<\frac{\epsilon}{2}$
and so
\[
\lvert a_{k_{n}}-I\rvert\leq\lvert c_{n}-a_{k_{n}}\rvert+\lvert c_{n}-I\rvert<\epsilon
\]
which completes the proof. The proof for limit superiors is analagous
(or you can use Proposition \ref{prop:.limInfNegLimSupNeg}).
\end{proof}
\begin{prop}
\label{prop:limSupEqualLimInf}The sequence $\left\{ a_{n}\right\} $
converges if and only if $\limsup_{n\to\infty}a_{n}=\liminf_{n\to\infty}a_{n}$
in which case 
\[
\lim_{n\to\infty}a_{n}=\limsup_{n\to\infty}a_{n}=\liminf_{n\to\infty}a_{n}.
\]
\end{prop}

\begin{proof}
If $\lim_{n\to\infty}a_{n}=L$, then all subsequential limits are
also equal to $L$, which implies (by Proposition \ref{prop:limSupSubsequential})
that $\lim_{n\to\infty}a_{n}=\limsup_{n\to\infty}a_{n}=\liminf_{n\to\infty}a_{n}.$
Conversely, assuming that $L:=\limsup_{n\to\infty}a_{n}=\liminf_{n\to\infty}a_{n},$
we can show that for any $\epsilon>0,$there's some large enough $n_{0}\in\N$,
$\lvert$such that
\[
\sup_{i\geq n_{0}}x_{i}-\inf_{i\geq n_{0}}x_{i}\leq\lvert\inf_{i\geq n_{0}}x_{i}-L\rvert+\lvert\sup_{i\geq n_{0}}x_{i}-L\rvert<\epsilon.
\]
Then for any $m,n\geq n_{0}$
\[
\rvert x_{m}-x_{n}\rvert\leq\sup_{i\geq n_{0}}x_{i}-\inf_{i\geq n_{0}}x_{i}\leq\epsilon
\]
and thus the sequence is Cauchy and converges to $L$ by Proposition
\ref{prop:limSupSubsequential}.
\end{proof}
\begin{prop}
\label{prop:SumLimInf}$\liminf_{n\to\infty}a_{n}+\liminf_{n\to\infty}b_{n}\leq\liminf_{n\to\infty}\left(a_{n}+b_{n}\right)\leq\limsup_{n\to\infty}\left(a_{n}+b_{n}\right)\leq\limsup_{n\to\infty}a_{n}+\limsup_{n\to\infty}b_{n}.$
\end{prop}

\begin{proof}
Note that $\inf_{i\geq n}a_{i}+\inf_{i\geq n}b_{i}\leq\inf_{i\geq n}\left(a_{i}+b_{i}\right)$
and take limits. The other inequalities follow by Propositions \ref{prop:limSupDominatesLimInf}
and \ref{prop:.limInfNegLimSupNeg}.
\end{proof}
\begin{rem*}
We are implicitly excluding the $\infty-\infty$ situations in the
limits. This should be assumed throughout this section.
\end{rem*}
\begin{prop}
\label{prop:prodLimInf}For $a_{n},b_{n}\geq0$ we have that $\liminf a_{n}\liminf b_{n}\leq\liminf a_{n}b_{n}\leq\limsup a_{n}b_{n}\leq\limsup a_{n}\limsup b_{n}.$
\end{prop}

\begin{proof}
Note that we have the inequalities 
\begin{align*}
0 & \leq\inf_{i\geq n}a_{i}\leq a_{i}\\
0 & \leq\inf_{i\geq n}b_{i}\leq b_{i}
\end{align*}
for all $i\geq n$ and so by multiplying the inequalities we get 
\[
0\leq\inf_{i\geq n}a_{i}\inf_{i\geq n}b_{i}\leq a_{i}b_{i}\implies0\leq\inf_{i\geq n}a_{i}\inf_{i\geq n}b_{i}\leq\inf_{i\geq n}a_{i}b_{i}
\]
and taking limits finishes the proof.
\end{proof}
\begin{rem}
Note that the condition that $a_{n},b_{n}\geq0$ is necessary: see
$a_{n}=\left(-1\right)^{n},b_{n}=(-1)^{n+1}$.
\end{rem}

\begin{prop}
\label{prop:limInfMid}$\liminf_{n\to\infty}\left(a_{n}+b_{n}\right)\leq\liminf_{n\to\infty}a_{n}+\limsup_{n\to\infty}b_{n}\leq\limsup_{n\to\infty}\left(a_{n}+b_{n}\right)$
\end{prop}

\begin{proof}
Note that 
\[
a_{n}+b_{n}\leq a_{n}+\sup_{i\geq n}b_{i}\implies\inf_{i\geq n}\left(a_{n}+b_{n}\right)\leq\inf_{i\geq n}\left(a_{n}+\sup_{i\geq n}b_{n}\right)=\inf_{i\geq n}a_{i}+\sup_{i\geq n}b_{i}
\]
where the last equality follows because $\sup b_{i}$ is a constant
for a fixed $n.$ Taking limits then yields the result.
\end{proof}
\begin{prop}
\label{prop:sumLimInfLim}If $\lim_{n\to\infty}a_{n}=L$ then 
\begin{align*}
\liminf_{n\to\infty}\left(a_{n}+b_{n}\right) & =L+\liminf_{n\to\infty}b_{n}\\
\limsup_{n\to\infty}\left(a_{n}+b_{n}\right) & =L+\limsup b_{n}.
\end{align*}
\end{prop}

\begin{proof}
Note first that 
\begin{align*}
L+\liminf_{n\to\infty}b_{n} & =\liminf_{n\to\infty}a_{n}+\liminf_{n\to\infty}b_{N}\\
 & \leq\liminf_{n\to\infty}\left(a_{n}+b_{n}\right)\\
 & \leq\limsup_{n\to\infty}a_{n}+\liminf_{n\to\infty}b_{n}\\
 & =L+\liminf_{n\to\infty}b_{n}
\end{align*}
by Propositions\ref{prop:limSupEqualLimInf}, \ref{prop:SumLimInf},
and \ref{prop:limInfMid}.
\end{proof}
\begin{prop}
If $a_{n},b_{n}\geq0$ then
\[
\liminf_{n\to\infty}a_{n}b_{n}\leq\liminf_{n\to\infty}a_{n}\limsup_{n\to\infty}b_{n}\leq\limsup_{n\to\infty}a_{n}b_{n}.
\]
\end{prop}

\begin{proof}
The proof proceeds in the exact same way as in Proposition \ref{prop:limInfMid}.
\end{proof}
%
\begin{prop}
\label{prop:prodLimInfExists}If $a_{n},b_{n}\geq0$ , $\lim a_{n}=L$
and $\limsup_{n\to\infty}b_{n}<\infty$ then 
\begin{align*}
\liminf\left(a_{n}b_{n}\right) & =L\liminf b_{n}\\
\limsup\left(a_{n}b_{n}\right) & =L\limsup b_{n}.
\end{align*}
\end{prop}

\begin{proof}
This is akin to Proposition \ref{prop:prodLimInfExists}; the following
chain of inequalities establish the claim
\begin{align*}
L\liminf b_{n} & =\liminf a_{n}\liminf b_{n}\\
 & \leq\liminf a_{n}b_{n}\\
 & \leq\limsup a_{n}\liminf b_{n}\\
 & =L\liminf b_{n}.
\end{align*}
\end{proof}
\begin{prop}
\label{prop:limInfContinuity}Let $a_{n}$ be a real sequence and
let $f:\R\to\R$ be a continuous and increasing function, then
\begin{align*}
f\left(\liminf_{n\to\infty}a_{n}\right) & =\liminf_{n\to\infty}f\left(a_{n}\right)\\
f\left(\limsup_{n\to\infty}a_{n}\right) & =\limsup_{n\to\infty}f\left(a_{n}\right).
\end{align*}
\end{prop}

\begin{proof}
First observe that since $f$ is increasing, we have that $\inf_{i\geq n}a_{i}\leq a_{i}\implies f\left(\inf_{i\geq n}a_{i}\right)\leq f\left(a_{i}\right)$
for all $i\geq n$ and so
\[
f\left(\inf_{i\geq n}a_{i}\right)\leq\inf_{i\geq n}f\left(a_{i}\right).
\]
Conversely, fix $n$ and suppose that $f\left(\inf_{i\geq n}a_{i}\right)<\inf_{i\geq n}f\left(a_{i}\right)$.
Fixing $\epsilon=\frac{\inf_{i\geq n}f\left(a_{i}\right)-f\left(\inf_{i\geq n}a_{i}\right)}{2}$
, note that by continuity there exists a $\delta>0$ such that $\lvert\inf_{i\geq n}a_{i}-x\rvert<\delta\implies\lvert f\left(\inf_{i\geq n}a\right)-f\left(x\right)\rvert<\epsilon$.
By the definition of infimum, there is some $i_{0}\geq n$ where $a_{i_{0}}-\inf_{i\geq n}a_{i}<\delta$
and so 
\[
f\left(a_{i_{0}}\right)-f\left(\inf_{i\geq n}a_{i}\right)<\epsilon
\]
which is a contradiction and hence 
\[
f\left(\inf_{i\geq n}a_{i}\right)\geq\inf_{i\geq n}f\left(a_{i}\right).
\]
Taking limits and applying the continuity of $f$ once again yields
the result.
\end{proof}
\begin{prop}
\[
\limsup_{n\to\infty}\max\left\{ a_{n},b_{n}\right\} =\max\left\{ \limsup_{n\to\infty}a_{n},\limsup_{n\to\infty}b_{n}\right\} 
\]
 and 
\[
\liminf_{n\to\infty}\min\left\{ a_{n},b_{n}\right\} =\min\left\{ \liminf_{n\to\infty}a_{n},\liminf_{n\to\infty}b_{n}\right\} .
\]
\end{prop}

\begin{xca}
Let $\left\{ a_{n}\right\} $ and $\left\{ b_{n}\right\} $ be positive
real sequences with such that $\limsup_{n\to\infty}\frac{\log\left(b_{n}\right)}{n}=-\infty$.
Prove that 
\[
\liminf_{n\to\infty}\frac{\log\left(a_{n}+b_{n}\right)}{n}=\liminf_{n\to\infty}\frac{\log\left(a_{n}\right)}{n}.
\]
\end{xca}

\begin{sol*}
Note that since exponential function is increasing and continuous,
by Proposition \ref{prop:limInfContinuity}
\[
\exp\left(\limsup_{n\to\infty}\log\left(b_{n}^{\frac{1}{n}}\right)\right)=\limsup_{n\to\infty}b_{n}^{\frac{1}{n}}=0
\]
which implies that $\lim_{n\to\infty}b_{n}^{\frac{1}{n}}=0$ since
$b_{n}>0$ for all $n\in\N.$ Since eventually, $b_{n}<b_{n}^{\frac{1}{n}}$
we have that $\lim_{n\to\infty}b_{n}=0.$ 
\[
\frac{\log\left(a_{n}+b_{n}\right)}{n}=\frac{\log\left(a_{n}\right)}{n}+\frac{\log\left(1+\frac{b_{n}}{a_{n}}\right)}{n}\leq\frac{\log\left(a_{n}\right)}{n}+
\]
\end{sol*}
\begin{thm}
\label{thm:realDecimalExpansion}For any real number $x\in\left[0,1\right)$
and any positive integer $b\geq2,$ there exists a sequence of non-negative
integers $\left\{ a_{i}\right\} _{i=1}^{\infty}\in\left\{ 0,1,2,\ldots,b-1\right\} $
such that 
\[
\sum_{i=1}^{\infty}\frac{a_{i}}{b^{i}}=x.
\]
Moreover, if there is no $n_{0}\in\mathbb{N}$ such that for all $n\geq n_{0}:a_{n}=b-1,$
then the sequence $\left\{ a_{i}\right\} $ is unique.
\end{thm}

\begin{proof}
Note that we can partition
\end{proof}

\subsection{Limits of functions}


\section{Metric spaces}

\subsection{Topology of metric spaces}

\subsection{Complete metric spaces}

\subsection{Compact metric spaces}

\subsection{Continuity of functions between metric spaces}
\begin{defn}
\label{def:continousFunction}Let $\left(X,d_{X}\right)$ and $\left(Y,d_{Y}\right)$
be a metric space. A function $f:X\to Y$ is said to be \emph{continuous
at a point} $c\in X$ if for any sequence $x_{n}\to c$ we have that
$f\left(x_{n}\right)\to f\left(c\right)$. The function $f$ is called
\emph{continuous }if it is continuous at every point $c\in X$.
\end{defn}

\begin{prop}
\label{prop:equivalentContinuity}Let $\left(X,d_{X}\right)$ and
$\left(Y,d_{Y}\right)$ be a metric space. A function $f:X\to Y$
is said to be \emph{continuous at a point} $c\in X$ if and only if
for every $\epsilon>0$ there exists a $\delta$ such that for any
$x\in X$ $d_{X}\left(x,c\right)<\delta\implies d_{Y}\left(f\left(x\right),f\left(c\right)\right)<\epsilon$.
\end{prop}

\begin{proof}
Suppose that the $\epsilon-\delta$ definition doesnt hold i.e. there
exists some $\epsilon_{0}>0$ such that for all $\delta>0$ there
exists some $x$ such that $d_{X}\left(x,c\right)<\delta$ but $d_{Y}\left(f\left(x\right),f\left(c\right)\right)\geq\epsilon_{0}.$
Letting $\delta=1/n,$ we can find a corresponding $x_{n}$ such that
$d_{X}\left(x_{n},c\right)<\frac{1}{n}$ but $d_{Y}\left(f\left(x_{n}\right),f\left(c\right)\right)\geq\epsilon_{0}$.
Since this can be done for any $n\in\N$, we have a seqeunce $x_{n}\to c$
but $f\left(x_{n}\right)\not\to c$.

Conversely, suppose the $\epsilon-\delta$ definition holds and let
$x_{n}\to c$ by arbitrary. For a large enough $N,$ $d_{X}\left(x_{n},c\right)<\delta$
for all $n\geq N$ and so $d_{Y}\left(f\left(x_{n}\right),f\left(c\right)\right)<\epsilon$
which completes the proof.
\end{proof}
\begin{prop}
\label{prop:openPreImageMetric}Let $\left(X,d_{X}\right)$ and $\left(Y,d_{Y}\right)$
be a metric space. A function $f:X\to Y$ is said to be \emph{continuous
if and only if for every open set $O_{Y}\in\tau_{Y}$ (here $\tau_{y}$
denotes the topology on $Y$) the preimage $f^{-1}\left[o_{Y}\right]\in\tau_{X}$.}
\end{prop}

\begin{proof}
First assume that $f$ is continuous and let $O_{Y}$ be an open set
in $Y$. We wish to show that for any $x\in f^{-1}\left[O_{Y}\right],$there
is some open ball $B\left(x,\delta\right)\subseteq f^{-1}\left[O_{y}\right].$
Note that since $O_{Y}$ is open, there exists some ball $B\left(f\left(x\right),\epsilon\right)\subseteq O_{Y}.$
Note that by continuity, there exists some $\delta>0$ such that $f\left[B\left(x,\delta\right)\right]\subseteq B\left(f\left(x\right),\epsilon\right)$.
Then by the property of preimages
\[
B\left(x,\delta\right)\subseteq f^{-1}f\left[B\left(x,\delta\right)\right]\subseteq f^{-1}\left[o_{Y}\right]
\]
which proves the claim.

Conversely, suppose that preimages of open sets are open. Fix $\epsilon>0$
and note that for any $x\in X$, the preimage of the ball $f^{-1}\left[B\left(f\left(x\right),\epsilon\right)\right]$
is open and thus there exists some $\delta>0$ such that $B\left(x,\delta\right)\subseteq f^{-1}\left[B\left(f\left(x\right),\epsilon\right)\right].$
Then
\[
f\left[B\left(x,\delta\right)\right]\subseteq ff^{-1}\left[B\left(f\left(x\right),\epsilon\right)\right]\subseteq B\left(f\left(x\right),\epsilon\right)
\]
which completes the proof.
\end{proof}
\begin{prop}
Let $\left(X,d_{X}\right),\left(Y,d_{Y}\right),$ and $\left(Z,d_{Z}\right)$
be metric spaces. Then for any continuous functions $f:X\to Y$ and
$g:Y\to Z$, the composition $g\circ f:X\to Z$ given by
\[
g\circ f\left(x\right)=g\left(f\left(x\right)\right)
\]
is continuous.
\end{prop}

\begin{proof}
Let $O_{Z}$ be an open set in $Z$. Note that $\left(g\circ f\right)^{-1}\left[O_{Z}\right]=f^{-1}\left[g^{-1}\left[O_{Z}\right]\right]$
and $g^{-1}\left[O_{Z}\right]$ is open since $g$ is continuous and
so $f^{-1}\left[g^{-1}\left[O_{Z}\right]\right]$ is open.
\end{proof}
\begin{rem*}
This proof applies \emph{mutis mutandis }to continuous functions between
general topological spaces, of which metric spaces are a special case.
\end{rem*}
\begin{defn}
\label{def:lipschitz}Let $\left(X,d_{X}\right)$ and $\left(Y,d_{Y}\right)$
be a metric spaces. A function $f:X\to Y$ is called \emph{Lipschitz
continuous }if there exists some $K>0$ such that 
\[
d_{Y}\left(f\left(x\right),f\left(y\right)\right)\leq Kd_{X}\left(x,y\right).
\]
\end{defn}

\begin{prop}
\label{prop:lipschitzContinuous}Let $\left(X,d_{X}\right)$ and $\left(Y,d_{Y}\right)$
be a metric spaces. Every Lipschitz continuous function $f:X\to Y$
is uniformly continuous.
\end{prop}

\begin{proof}
Fix $\epsilon>0$ and note that if $\delta=\frac{\epsilon}{K}$ then
$d_{X}\left(x,y\right)<\delta\implies d_{Y}\left(f\left(x\right),f\left(y\right)\right)<\epsilon.$
\end{proof}
\begin{prop}
\label{prop:lipschitzBoundedDerivative}$Let$$f:D\subseteq\R\to\R$
be a differentiable function. Then $f$ is Lipschitz if and only if
it has a bounded derivative.
\end{prop}

\begin{proof}
First suppose that the function has a bounded derivative. Then there
exists some $M>0$ such that for any $x\in D$, $\lvert f^{\prime}\left(x\right)\rvert\leq M.$
Then for any $x,y\in D$, the mean value theorem implies that there
exists some $c\in\left(\min\left\{ x,y\right\} ,\max\left\{ x,y\right\} \right)$
such that 
\[
\left\lvert \frac{f\left(x\right)-f\left(y\right)}{x-y}\right\rvert =\lvert f^{\prime}\left(c\right)\rvert\leq M
\]
Rearranging yields the result.

Conversely, suppose that the function is Lipschitz with constant $K$.
Then
\[
\left\lvert \frac{f\left(x\right)-f\left(y\right)}{x-y}\right\rvert \leq K
\]
and taking limits $x\to y$ gives the result.
\end{proof}
\begin{defn}
\label{def:cauchyContinuity}Let $\left(X,d_{X}\right)$ and $\left(Y,d_{Y}\right)$
be metric spaces. A function $f:X\to Y$ is called \emph{Cauchy-continuous
}if for any Cauchy sequence $\left\{ x_{n}\right\} \in X$, the image
$\left\{ f\left(x_{n}\right)\right\} $ is Cauchy.
\end{defn}

\begin{prop}
\label{prop:CauchyContinuityImpliesContinuity}Let $\left(X,d_{X}\right)$
and $\left(Y,d_{Y}\right)$ be metric spaces. If $f:X\to Y$ is Cauchy
continuous then it is continuous. Moreover, if $X$ is complete then
every continuous function $f:X\to Y$ is also Cauchy-continuous.
\end{prop}

\begin{proof}
Fix $\epsilon>0$ and let $c\in X$ be arbitrary. Then there exists
at least one (possibly eventually constant) Cauchy sequence $x_{m}\to c.$.
Now consider a new sequence $y_{n}$ such that $y_{n}=x_{n}$ for
even $n$and $y_{n}=c$ for odd $n.$ Then $y_{n}\to c$ (and is Cauchy)
and so $f\left(y_{n}\right)$ is Cauchy. This implies that for large
$n$
\[
d\left(f\left(x_{n}\right),f\left(c\right)\right)<\epsilon
\]
which implies $f\left(y_{n}\right)\to f\left(c\right).$ Extracting
the even indexed subsequence shows that $f\left(x_{n}\right)\to c.$

Next, suppose that $X$ is complete and $f$ is continuous. Let $\left\{ x_{n}\right\} $
be a Cauchy sequence. Then by completenes $x_{n}\to c\in X$. By continuity,
$f\left(x_{n}\right)\to f\left(c\right)$ and so $\left\{ f\left(x_{n}\right)\right\} $
is Cauchy.
\end{proof}
\begin{prop}
\label{prop:uniformContinuityImpliesCauchyContinuity}Let $\left(X,d_{X}\right)$
and $\left(Y,d_{Y}\right)$ be a metric spaces. If a function $f:X\to Y$
is uniformly continuous, it is Cauchy continuous.
\end{prop}

\begin{proof}
Fix $\epsilon>0$ and let $\left\{ x_{n}\right\} \in X$ be a Cauchy
sequence. By uniform continuity, there exists some $\delta>0$ such
that for any $m,n\in\N$ $d_{X}\left(x_{n},x_{m}\right)<\delta\implies d_{Y}\left(f\left(x_{n}\right),f\left(x_{m}\right)\right)<\epsilon.$
Since $\left\{ x_{n}\right\} $ is Cauchy, there exists some $N$
such that for any $m,n\geq N$ $d_{X}\left(x_{n},x_{m}\right)<\delta$
which completes the proof.
\end{proof}
\begin{thm}
\label{thm:compactUniformContinuity}Let $\left(X,d_{X}\right)$ be
a compact metric space and let $\left(Y,d_{Y}\right)$ be an arbitrary
metric space. The following are equivalent for a function $f:X\to Y$

\begin{enumerate}[label=(\roman*),leftmargin=.1\linewidth,rightmargin=.4\linewidth]
	\item f is continuous
	\item f is uniformly continuous
	\item f is Cauchy continuous.
\end{enumerate}
\end{thm}

\begin{proof}
We prove $(i)\implies(ii)$; $(ii)\implies(iii)$ is Proposition \ref{prop:uniformContinuityImpliesCauchyContinuity}
and $(iii)\implies(i)$ is Proposition \ref{prop:CauchyContinuityImpliesContinuity}.

Let $\epsilon>0$ be fixed. By continuity, we know that for every
$c\in X$ there exists some $\delta_{c}>0$ such that when $d_{X}\left(x,c\right)<\delta_{c}\implies d_{Y}\left(f\left(x\right),f\left(c\right)\right)<\frac{\epsilon}{2}$.
Note that the balls $\left\{ B_{d_{X}}\left(c,\frac{\delta_{c}}{2}\right)\right\} _{c\in X}$
form a cover of $X$ and by compactness we can extract a finite subcover
$\left\{ B_{d_{X}}\left(c_{i},\frac{\delta_{c_{i}}}{2}\right)\right\} _{i=1}^{n}$.
Let $\delta:=\min_{i}\frac{\delta_{c_{i}}}{2}$ and note that for
any $x,y\in X$ , there's some $1\leq i_{0}\leq n$ such that $x\in B_{d_{X}}\left(c_{i_{0}}\frac{d_{c_{i_{0}}}}{2}\right).$
Then, if $d_{X}\left(x,y\right)<\delta$
\begin{align*}
d_{X}\left(c_{i_{0}},y\right) & \leq d_{X}\left(x,y\right)+d_{X}\left(x,c_{i_{0}}\right)\\
 & <\delta+\frac{\delta_{c_{i_{0}}}}{2}\\
 & \leq\delta_{c_{i_{0}}}
\end{align*}
Then,
\begin{align*}
d_{Y}\left(f\left(x\right),f\left(y\right)\right) & \leq d_{Y}\left(f\left(x\right),f\left(c_{i_{0}}\right)\right)+\delta_{Y}\left(f\left(y\right),f\left(c_{i_{0}}\right)\right)\\
 & <\frac{\epsilon}{2}+\frac{\epsilon}{2}\\
 & =\epsilon
\end{align*}
which completes the proof.
\end{proof}
\begin{defn}
Let $\left(X,d_{X}\right)$ and $\left(Y,d_{Y}\right)$ be a metric
spaces. A function $f:X\to Y$ is called a contraction if it is Lipschitz
with a Lipschitz constant $0<K<1$. In other words, for any $x,y\in X$
\[
d_{Y}\left(f\left(x\right),f\left(y\right)\right)\leq Kd_{X}\left(x,y\right).
\]
\end{defn}

\begin{thm}[Banach Fixed Point Theorem]
\label{thm:banachFixedPoint}Let $\left(X,d\right)$ be a complete
metric space and let a function $f:X\to X$ be a contraction. Then
there exists a unique $x\in X$ such that $f\left(x\right)=x$.
\end{thm}

\begin{proof}
Let $x_{0}\in X$ be arbitrary, define a sequence $x_{n+1}=f\left(x_{n}\right)$
and fix $\epsilon>0$. For any $n\geq m\geq1$
\begin{align}
d\left(x_{n},x_{m}\right) & \leq d\left(x_{n},x_{n-1}\right)+d\left(x_{n-1},x_{n-2}\right)+\ldots+d\left(x_{m+1},x_{m}\right)\nonumber \\
 & \leq K^{n-1}d\left(x_{1},x_{0}\right)+K^{n-2}d\left(x_{1},x_{0}\right)+\ldots+K^{m}d\left(x_{1},x_{0}\right)\nonumber \\
 & =d\left(x_{1},x_{0}\right)\sum_{i=m}^{n-1}K^{i}\label{eq:contractionCauchy}
\end{align}
where the first inequalitity is simply the triangle inequality and
the second is induction on the contraction property. Note that since
$\left\{ \sum_{i=1}^{n}K^{i}\right\} _{i=1}^{\infty}$ is a convergent
geometric series, it's Cauchy as in we can find an $N\in\N$ such
that for all $n,m\geq N$ we have that $\lvert\sum_{i=1}^{n}K_{i}-\sum_{i=1}^{m}K_{i}\rvert=\sum_{i=\min\left\{ n.m\right\} -1}^{\max\left\{ n.m\right\} }K^{i}<\epsilon.$
Therefore, for $m,n\geq N$, the inequality (\ref{eq:contractionCauchy})
implies
\[
d\left(x_{m},x_{n}\right)\leq d\left(x_{1},x_{0}\right)\epsilon.
\]
Since $\epsilon$ can be arbitrarily small, our sequence $x_{n}$
is Cauchy and so by the completeness of $X$ it converges to some
limit $x.$ Then, by the continuity of $f$
\begin{align*}
x & =\lim_{n\to\infty}x_{n}\\
 & =\lim_{n\to\infty}f(x_{n-1})\\
 & =f(x).
\end{align*}
which yields our fixed point.

Now suppose that there were two fixed points i.e. $x\neq y$ such
that $f\left(x\right)=x$ and $f\left(y\right)=y$. By the contraction
property,
\[
d\left(x,y\right)\leq Kd\left(x,y\right)
\]
which is a contradiction.
\end{proof}

\subsubsection{Continuity of real valued function on $\protect\R$}


\section{Topological spaces}

\section{Special functions\label{sec:specialFunctions}}
