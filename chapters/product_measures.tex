
\chapter{Measures on product spaces\label{chap:productMeasures}}

In calculus, we learnt that the theory of integration readily extends
from from real valued functions on $\R$ to real valued functions
on the Euclidean space $\R^{n}.$ The extension is usually motivated
geometrically by studying the volume under the surface of a sufficiently
smooth function $f:\R^{2}\to\R$. The idea is that if a function is
sufficiently well behaved, then one can recover the volume under the
surface by looking at the areas under various ``slices'' of the function
and then summing up those areas. Importantly, under the requisite
smoothness conditions, the ``slices'' could have been made horizontally
or vertically, and we would get the same result. This intuition leads
to the Fubini theorem for multiple integration in the Riemann setting:
\[
\int_{a}^{b}\int_{c}^{d}f(x,y)dydx=\int_{c}^{d}\int_{a}^{b}f(x,y)dxdy.
\]
In our linear functional notation, this can be written as
\[
\lambda^{x}\left(\lambda^{y}\left(f\left(x,y\right)\indicate_{\left[c,d\right]}\right)\indicate_{\left[a,b\right]}\right)=\lambda^{y}\left(\lambda^{x}\left(f\left(x,y\right)\indicate_{\left[a,b\right]}\right)\indicate_{\left[c,d\right]}\right).
\]
Here $\lambda$ is the Lebesgue measure as usual, and the superscripts
denote the variable of integration. Note that we have dropped the
bars on top of the $\lambda$ to denote integration to clean up notation;
for comfort, you can just think of integration as an extension of
a measure from the space of measurable indicators to the space of
non-negative measurable or integrable functions, in the spirit of
Theorem \ref{thm:integralMeasureEquivalence}.

The main goal of this chapter is to recover this result for the general
measures. While important in its own right for the study of analysis
on Euclidean spaces, in the context of probability theory, this result
takes a far more important role. In particular, the ability to write
a multiple integral as an iterated integral corresponds directly with
the ability to factor the joint distribution of random variables into
their marginal distributions; that is, it underpins the theory of
\emph{independent }random variables. Independence, and the departures
from independence, constitute the central concepts of probabiliy theory.

\section{Product measures on finite product spaces}

\subsection{Iterated integrals}

Let $\left(\X,\F,\mu\right)$ and $\left(\mathcal{Y},\mathcal{G},\nu\right)$
be measure spaces. Our interest is in defining measurable functions
on the product space $\X\times\mathcal{Y}.$ The principle hurdle
that is immediately apparent here is that the product of the $\sigma-$algebras
$\F\times\mathcal{G}:=\left\{ F\times G\mid F\in\F,G\in\mathcal{G}\right\} $
is not necessarily a $\sigma-$algebra. \hl{ADD CE}. It is easy to
see, however, that it is a $\pi-$system (this fact will turn out
to be important!). Any $\sigma-$algebra we use should certainly \emph{contain
}$\mathcal{F}\times\mathcal{G}$ and so the canonical choice is given
by $\mathcal{F}\otimes\mathcal{G:=\sigma\left(F\times G\right)}$.
A series of natural questions follow if we want iterated integrals
to make sense. In particular, for a measurable map $f\in\mathcal{M}^{+}\left(\X\times\mathcal{Y},\mathcal{F}\otimes\mathcal{G}\right),$we
want to know if the projections $x\to f(x,y)$ and $y\to\mu^{x}\left(f\left(x,y\right)\right)$
are $\mathcal{F}$ and $\mathcal{G}$ measurable, respectively. Thankfully,
this turns out to be the case. We will use a long lost result from
Chapter \ref{chap:measurableFunctions}: the \hyperref[thm:piLambdaThmFunctions]{$\pi-\lambda$ theorem for functions}.
Go over the hypotheses of this theorem before reading the following
results.
\begin{lem}
\label{lem:partialFunctionMeasurability}For every $f\in\mathcal{M}\left(\X\times\mathcal{Y},\mathcal{F}\otimes\mathcal{G}\right)$,
the maps $x\to f(x,y)$ and $y\to f(x,y)$ are $\mathcal{F}/\borel\left(\R\right)$
and $\mathcal{G}/\borel\left(\R\right)$ measurable for every $y\in\mathcal{Y}$
and $x\in\X$, respectively.
\end{lem}

\begin{proof}
First we note, due to the fact that $\left(A_{1}\times B_{1}\right)\cap\left(A_{2}\times B_{2}\right)=\left(A_{1}\cap A_{2}\right)\times\left(B_{1}\times B_{2}\right),$that
$\mathcal{\mathcal{E:=}F}\times\mathcal{G}$ is a $\pi-$system. Next,
we claim that the space
\[
\mathcal{H:=}\left\{ f\in\mathcal{M_{\text{bdd}}}\left(\X\times\mathcal{Y},\mathcal{F}\otimes\mathcal{G}\right)\mid\forall y\in\mathcal{Y}:x\to f\left(x,y\right)\text{is }\mathcal{G}/\borel\left(\R\right)\text{ measurable }\right\} 
\]
is a $\lambda-$space of functions. Note that $\indicate_{\X\times\mathcal{Y}}$is
constant and bounded (and evidently partially measurable in our sense).
Further, our space $\mathcal{H}$ is a vector space, since linear
combinations of bounded measurable functions is bounded, and the partial
measurability condition is also preserved under linear combinations
(both results of Proposition \ref{prop:binaryOperationsMeasurableFunctions}).
Finally, $\mathcal{H}$ is closed under monotone limits (if they exist)
since measurability (resp. partial measurability) is preserved under
limits.

Now, observe that $\left\{ \indicate_{A}\mid A\in\mathcal{E}\right\} \subseteq\mathcal{H}$,
since $\indicate_{F\times G}=\indicate_{F}\indicate_{G}$which is
clearly bounded, measurable, and partially measurable in our sense.
Thus applying the $\pi-\lambda$ theorem as discussed, we have that

\[
\mathcal{M_{\text{bdd}}}\left(\X\times\mathcal{Y},\mathcal{F}\otimes\mathcal{G}\right)\subseteq\mathcal{H}
\]
which completes the proof for bounded functions.

Now we can simply take a function $f\in\mathcal{M}\left(\X\times\mathcal{Y},\mathcal{F}\otimes\mathcal{G}\right)$
and construct the bounded monotone sequences $f_{n}^{\pm}:=\min\left\{ f^{\pm},n\right\} \in\mathcal{H}$
by our results. We complete the proof by taking limits and noting
that partial measurability is preserved. The same argument, of course,
holds for $y\to f\left(x,y\right)$.
\end{proof}
Note that the converse of this result is not necessarily true; that
is, for a function $f:\X\times\mathcal{Y}\to\R$ with measurable sections
$x\to f\left(x,y\right)$ and $y\to f\left(x,y\right)$, it need not
be the case that $f$ is product measurable.
\begin{example}
\label{exa:sectionMeasurableNotProductMeasurable}Let $V$ be a Vitali
set in $\R$ and let $E=\left\{ \left(x,x\right)\mid x\in V\right\} .$
Then let $f\left(x,y\right)=\indicate_{E}\left(x,y\right)$ and notice
that
\[
x\to f\left(x,y\right)=\begin{cases}
0, & y\notin V\\
\indicate\left\{ x=y\right\}  & y\in V.
\end{cases}
\]
In the first case, the section $x\to f\left(x,y\right)$ is constantly
zero and hence measurable. In the second case, its an indicator function
for the singleton $\left\{ y\right\} $ which is measurable since
singletons are in $\borel\left(\R\right).$Therefore, for every $y\in\R$
the section $x\to f\left(x,y\right)$ is $\borel\left(\R\right)/\borel\left(\R\right)$
measurable. A similar argument shows that for every $x\in\R$, $y\to f\left(x,y\right)$
is Borel measurable. But notice that $f$ is measurable if and only
if $E$ is measurable, but the map $g\left(x\right)=\left(x,x\right)$is
measurable and $g^{-1}\left[E\right]=V$ is not measurable and so
$E$ is not measurable.
\end{example}

\begin{lem}
\label{lem:partialIntegralMeasurability}Let $\left(\X,\F,\mu\right)$
and $\left(\mathcal{Y},\mathcal{G},\nu\right)$ be $\sigma-$finite
measure spaces. For every function $f\in\mathcal{M}^{+}\left(\X\times\mathcal{Y},\mathcal{F}\otimes\mathcal{G}\right)$,
the maps $x\to\nu^{y}\left(f\left(x,y\right)\right)$ and $y\to\mu^{x}\left(f\left(x,y\right)\right)$
are $\mathcal{F}/\borel\left(\R\right)$ and $\mathcal{G}/\borel\left(\R\right)$
measurable respectively.
\end{lem}

\begin{proof}
We will prove that $x\to\nu^{y}f\left(x,y\right)$ is measurable;
the other case follows by symmetry. First, let $F_{i}\in\mathcal{\F}$
be an increasing sequence of sets with finite measure such that $\bigcup_{i\in\N}F_{i}=\mathcal{\X}.$
Observe that for the indicator of any rectangle in $F\times G\in\mathcal{E},$we
have that $\nu^{y}\left(F\times G\right)=\indicate_{F}\nu\left(G\right)$
which is measurable. Next, define
\[
\mathcal{D}_{i}:=\left\{ D\in\mathcal{F}\otimes\mathcal{G}\mid x\to\nu^{y}\left(\indicate_{D}\indicate_{F_{i}}\right)\text{ is }\mathcal{\F}/\borel\left(\R\right)\ \text{measurable}\right\} 
\]
using Lemma \ref{lem:partialFunctionMeasurability} to ensure that
the expression is well defined. Notice that $\mathcal{E}\subseteq\mathcal{D}$
(and so in particular, $\X\times\mathcal{Y}\in\mathcal{D}$). Now
suppose $D_{1},D_{2}\in\mathcal{D}$ such that $D_{1}\subseteq D_{2}$.
Then, letting $\nu_{i}$ denoting the measure that has density $\indicate_{F_{i}}$
with respect to $\nu$
\[
\nu_{i}^{y}\left(\indicate_{D_{2}\setminus D_{1}}\right)=\nu_{i}^{y}\left(\indicate_{D_{2}}\right)-\nu_{i}^{y}\left(\indicate_{D_{1}}\right)
\]
 which is measurable by Proposition \ref{prop:binaryOperationsMeasurableFunctions}.
Note that the expression on the right is well defined since $\nu_{i}$
is a finite measure and indicator functions are bounded (and so no
$\infty-\infty$ situations arise). Similarly, suppose we have an
increasing sequence of sets $D_{n}\in\mathcal{D}_{i}$, with $D:=\bigcup_{n\in\N}D_{n}$,
we have 
\begin{align*}
\nu_{i}^{y}\left(\indicate_{D}\right) & =\nu_{i}^{y}\left(\sup_{n\in\N}\indicate_{D_{n}}\right)\\
 & =\nu_{i}^{y}\left(\lim_{n\to\infty}\indicate_{D_{n}}\right)\\
 & =\lim_{n\to\infty}\nu_{i}^{y}\left(\indicate_{D_{n}}\right)
\end{align*}
where we have used Proposition \ref{prop:indicatorFunctionsArbitraryOperations}
in the first equality, the nested nature of $D_{n}$ in the second
equality, and the monotone convergence theorem in the final equality.
By Corollary \ref{cor:limSupLimInfMeasurable}, the limit is measurable
and so $D\in\mathcal{D}_{i}$. Thus we have shown that each $\mathcal{D}_{i}$
is a $\lambda-$system containing $\mathcal{E}$, a $\pi-$system,
and so by the $\pi-\lambda$ theorem, $\F\otimes\mathcal{G}=\sigma\left(\mathcal{E}\right)\subseteq\mathcal{D}_{i}$
for all $i\in\N$. Finally, for any $D\in\F\otimes\mathcal{G}$, we
know that $x\to\nu_{i}^{y}\left(\indicate_{D}\right)$ is measurable,
and by yet another application of monotone convergence and Corollary
\ref{cor:limSupLimInfMeasurable}, $x\to\nu^{y}\left(\indicate_{D}\right)$
is also measurable.

This completes the proof for indicator functions. We can then show
that measurability holds for non-negative measurable simple functions
by using the linearity of integration and the fact that measurability
is preserved under linear combinations. Finally, we show the result
for general non-negative measurable functions by approximating them
by simple functions from below, applying monotone convergence, and
using Corollary \ref{cor:limSupLimInfMeasurable} yet again.
\end{proof}
\hl{Add remarks under each measurability lemma to extend to n dimensions via induction}

With the technicalities out of the way, we can show that iterated
integrals give the same result under certain conditions. This is a
generalization of the result we saw all the way back in Lemma \ref{lem:TonelliForSeries}.
\begin{thm}[Tonelli]
\label{thm:tonelli}Let $\left(\X,\F,\mu\right)$ and $\left(\mathcal{Y},\mathcal{G},\nu\right)$
be measure spaces. Then, for any $f\in\mathcal{M}^{+}\left(\X\times\mathcal{Y},\mathcal{F}\otimes\mathcal{G}\right)$,
the functions 
\[
\gamma_{1}\left(f\right):=\nu^{y}\mu^{x}\left(f\right)
\]
and
\[
\gamma_{2}\left(f\right):\mu^{x}\nu^{y}\left(f\right)
\]
are integrals on $\mathcal{M}^{+}\left(\X\times\mathcal{Y},\mathcal{F}\otimes\mathcal{G}\right).$
Moreover, if $\mu$and $\nu$ are $\sigma-$finite, $\gamma_{1}\left(f\right)=\gamma_{2}\left(f\right)$
for every $f\in\mathcal{M}^{+}\left(\X\times\mathcal{Y},\mathcal{F}\otimes\mathcal{G}\right).$
\end{thm}

\begin{proof}
We show that $\gamma_{1}$ is an integral; the argument for $\gamma_{2}$
is the analagous. First, observe that $\gamma_{1}\left(0\right)=\nu^{y}\mu^{x}\left(0\right)=\nu^{y}\left(0\right)=0$,
given that $\nu$ and $\mu$ are integrals. Second, note that for
$\alpha,\beta\geq0$, and $f,g\in\mathcal{M}^{+}\left(\X\times\mathcal{Y},\mathcal{F}\otimes\mathcal{G}\right)$
\begin{align*}
\gamma_{1}\left(\alpha f+\beta g\right) & =\nu^{y}\mu^{x}\left(\alpha f+\beta g\right)\\
 & =\nu^{y}\left(\alpha\mu^{x}f+\beta\mu^{x}g\right)\\
 & =\alpha\nu^{y}\mu^{x}\left(f\right)+\beta\nu^{y}\mu^{x}\left(g\right)\\
 & =\alpha\gamma_{1}\left(f\right)+\beta\gamma_{1}\left(g\right).
\end{align*}
Finally, observe that for $f_{n}\in\mathcal{M}^{+}\left(\X\times\mathcal{Y},\mathcal{F}\otimes\mathcal{G}\right)$
such that $f_{n}\leq f_{n+1}$ and $f_{n}\to f$, we have
\begin{align*}
\lim_{n\to\infty}\gamma_{1}\left(f_{n}\right) & =\lim_{n\to\infty}\nu^{y}\mu^{x}\left(f_{n}\right)\\
 & =\nu^{y}\mu^{x}\left(\lim_{n\to\infty}f_{n}\right)\\
 & =\gamma_{1}\left(f\right)
\end{align*}
by applying monotone convergence twice.

Therefore, by Theorem \ref{thm:integralMeasureEquivalence}, $\gamma_{1}$
and $\gamma_{2}$ are integrals on $\mathcal{M}^{+}\left(\X\times\mathcal{Y},\mathcal{F}\otimes\mathcal{G}\right)$
with respect to measures defined by integrating indicator functions
in $\mathcal{F\otimes\mathcal{G}}.$ If $\mu$ and $\nu$ are $\sigma-$finite,
the two integrals can be shown to be equal by showing that the corresponding
measures are equal on a generating $\pi-$system that can approximate
the full space (courtesy of our \hyperref[thm:uniquenessMeasures]{uniqueness theorem}).
Of course, since $\mathcal{F\times\mathcal{G}}$ is a $\pi-$system,
for $F\in\F,G\in\mathcal{G}$
\begin{align*}
\gamma_{1}\left(\indicate_{F\times G}\right) & =\nu^{y}\mu^{x}\left(\indicate_{F\times G}\left(x,y\right)\right)\\
 & =\nu^{y}\mu^{x}\left(\indicate_{F}\left(x\right)\indicate_{G}\left(y\right)\right)\\
 & =\nu\left(G\right)\mu\left(F\right)\\
 & =\mu^{x}\nu^{y}\left(\indicate_{F}\left(x\right)\indicate_{G}\left(y\right)\right)\\
 & =\gamma_{2}\left(\indicate_{F\times G}\right)
\end{align*}
completing the proof. Since our measures $\mu,\nu$ are $\sigma-$finite,
we know that there exist sets $E_{i}\in\mathcal{F\times\mathcal{G}}$
such that $\bigcup E_{i}=\X\times\mathcal{Y}.$ This completes the
proof.
\end{proof}
Note that the $\sigma-$finiteness condition is actually necessary
for the uniqueness of the integrals, as Example \ref{exa:tonelliFailNonSigmaFinite}
illustrates.
\begin{cor}[Fubini's Theorem]
Let $\left(\X,\F,\mu\right)$ and $\left(\mathcal{Y},\mathcal{G},\nu\right)$
be $\sigma-$finite measure spaces and let $f\in\mathcal{M}\left(\X\times\mathcal{Y},\F\otimes\mathcal{G}\right).$
If one of $\gamma_{1}\left(\lvert f\rvert\right):=\nu^{y}\mu^{x}\left(\lvert f\rvert\right)$
or $\gamma_{2}\left(\lvert f\rvert\right):\mu^{x}\nu^{y}\left(\lvert f\rvert\right)$
is finite, then

\begin{enumerate}[label=(\roman*),leftmargin=.1\linewidth,rightmargin=.4\linewidth]
\item $x \to f\left(x,y\right) \in \Lp{1}{\mu}$
\item $y \to f\left(x,y\right) \in \Lp{1}{\nu}$
\item $x \to \nu^y\left(f\left(x,y\right)\right)\in \Lp{1}{\mu}$
\item $y \to \mu^x\left(f\left(x,y\right)\right) \in \Lp{1}{\nu}$
\item $\gamma_1\left(f\right) = \gamma_2\left(f\right)$
\end{enumerate}
\end{cor}

\begin{proof}
Note that since $\lvert f\rvert\in\mathcal{M}^{+}\left(\X\times\mathcal{Y},\F\otimes\mathcal{G}\right)$,
by Tonelli's theorem,
\[
\gamma_{1}\left(\lvert f\rvert\right)=\gamma_{2}\left(\lvert f\rvert\right)<\infty
\]
which implies, by Proposition \ref{prop:intFiniteFuncFinite}, that
\[
\nu^{y}\left(\lvert f\left(x,y\right)\rvert\right)<\infty
\]
and
\[
\mu^{x}\left(\lvert f\left(x,y\right)\rvert\right)<\infty
\]
which gives us $\left(i\right)$ and $\left(ii\right)$. Next, note
that
\[
\lvert\nu^{y}\left(f\left(x,y\right)\right)\rvert\leq\nu^{y}\left(\lvert f\left(x,y\right)\rvert\right)<\infty
\]
and
\[
\lvert\mu^{x}\left(f\left(x,y\right)\right)\rvert\leq\mu^{x}\left(\lvert f\left(x,y\right)\rvert\right)<\infty
\]
for all $y\in\mathcal{Y}$ by $\left(i\right)$ , $\left(ii\right)$
and Corollary \ref{cor:triangleIneqLebIntL1}. The measurabiltiy of
$x\to\lvert\nu^{y}\left(f\left(x,y\right)\right)\rvert$ and $y\to\lvert\mu^{x}\left(f\left(x,y\right)\right)\rvert$
follows from the fact that $x\to\nu^{y}\left(f^{\pm}\right)$ and
$y\to\mu^{x}\left(f^{\pm}\right)$ are measurable (and finite) by
Lemma \ref{lem:partialIntegralMeasurability} and so $x\to\nu^{x}\left(f\left(x,y\right)\right)=x\to\nu^{x}\left(f^{+}\right)-\nu^{x}\left(f^{-}\right)$
and $y\to\mu^{x}\left(f\left(x,y\right)\right)=y\to\mu^{x}\left(f^{+}\right)-\mu^{x}\left(f^{-}\right)$
are measurable by Corollary \ref{cor:examplesBinaryOpsMeasFunc}.
Since the absolute value is continuous, our measurability results
hold. Then, monotonicity of integration implies that 
\[
\mu^{x}\lvert\nu^{y}\left(f\left(x,y\right)\right)\rvert<\infty\ \text{and}\ \nu^{y}\lvert\mu^{x}\left(f\left(x,y\right)\right)\rvert<\infty
\]
which gives is $\left(iii\right)$ and $\left(iv\right).$ Finally,
to see $\left(v\right)$, you note that $\gamma_{1}\left(f^{+}\right)=\gamma_{2}\left(f^{+}\right)$
and $\gamma_{1}\left(f^{-}\right)=\gamma_{2}\left(f^{-}\right)$ by
Tonelli. Then the result follows by subtracting these two equalities
and apply linearity.
\end{proof}

\subsection{Product sigma algebras and measures}
\begin{defn}
\label{def:productMeasure}Let $\measurespace$ and $\left(\mathcal{Y},\mathcal{G},\nu\right)$
be measure spaces and let $\left(\X\times\mathcal{Y},\F\otimes\mathcal{G}\right)$
be the product measurable space. Then a measure $\mu\otimes\nu:\F\otimes\mathcal{G}\to\left[0,\infty\right]$
is called a \emph{product measure }if
\[
\mu\otimes\nu\left(F\times G\right)=\mu\left(F\right)\nu\left(G\right)
\]
for all $F\in\F,G\in\mathcal{G}.$
\end{defn}

Of course, the way we have set this up, Theorem \ref{thm:tonelli}(Tonelli)
guarantees existence of this measure, since measures and integrals
are equivalent. Uniqueness, when $\mu$ and $\nu$ are $\sigma-$finite,
follows from the uniqueness theorem, as outlined in the proof of Tonelli.
Note also the fact that the product of sigma finite measures is sigma
finite. Tonelli (or Fubini) also tell us that for non-negative measurable
(or integrable) functions, the integral with respect to the product
measure coincides with iterated integrals, at least when the measure
spaces are $\sigma-$finite. That is to say
\[
\lebInt{\mu\otimes\nu}f=\gamma_{1}\left(f\right)=\gamma_{2}\left(f\right)
\]
for any $f\in\mathcal{M}^{+}\left(\X\times\mathcal{Y},\mathcal{F}\otimes\mathcal{G}\right)$
($f\in\Lp 1{\mu\otimes v}$).

To get some of the standard results on multiple integration in $\R^{n}$,
we first need to do some legwork to characterize products of Borel
$\sigma-$algebras. We begin first by generalizing the concept of
a $\sigma-$algebra generated by a measurable function.
\begin{defn}
\label{def:sigmaMultipleMeasurableFunctions}Let $\X$ be a set and
let $\left(\mathcal{Y}_{i},\mathcal{G}_{i}\right)$ be a measurable
spaces, for $i\in I$ where $I$ is an index set. Further, let $\left\{ f_{i}\right\} _{i\in I}$
, be a collection of functions $f_{i}:\X\to\mathcal{Y}_{i}$. Then,
the $\sigma-$algebra generated by $\left\{ f_{i}\right\} _{i\in I}$
is 
\[
\sigma\left(\left\{ f_{i}\right\} _{i\in I}\right):=\sigma\left(\bigcup_{i\in I}\sigma\left(f_{i}\right)\right)
\]
where $\sigma\left(f_{i}\right)$ is the usual definition of the sigma
algebra generated by a function.
\end{defn}

\begin{rem*}
When the $I$ is finite or countable, we simply write $\sigma\left(f_{1},f_{2},\ldots\right)$.
\end{rem*}
\begin{prop}
\label{prop:sigmaAlgebraGenCollectionSmallest}Let $\X$ be a set
and let $\left(\mathcal{Y},\mathcal{G}\right)$ be a measurable space.
For any collection $\left\{ f_{i}\right\} _{i\in I}$ where $I$ is
just some index set, $\sigma\left(\left\{ f_{i}\right\} _{i\in I}\right)$
is the smallest $\sigma-$algebra that makes every $f_{i}$ measurable.
\end{prop}

\begin{proof}
Note that if $\mathcal{F}$ is some $\sigma-$algebra that makes $f_{i}$
measurable for all $i\in I$, then $\sigma\left(f_{i}\right)\subseteq\mathcal{F}$
which implies that $\bigcup_{i\in I}\sigma\left(f_{i}\right)\subseteq\F$
and the result follows.
\end{proof}
\begin{rem*}
We can use the definition above to provide an alternate characterization
of product $\sigma-$algebras. First note for measurable spaces $\left(\X_{1},\mathcal{F}_{1}\right)$
and $\left(\X_{2},\mathcal{F}_{2}\right)$ with the product space
$\left(\X_{1}\times\mathcal{\X}_{2},\F_{1}\otimes\F_{2}\right)$,
we have that the coordinate projections $\pi_{1}$ and $\pi_{2}$
defined as
\[
\pi_{i}:\X_{1}\times\mathcal{\X}_{2}\to\X_{i}
\]
where $\pi_{i}\left(x_{1},x_{2}\right)=x_{i}$ for any $i=1,2.$
\end{rem*}
\begin{prop}
\label{prop:altProductSigma}Let $\left(\X_{1},\F_{1}\right),\left(\X_{2},\F_{2}\right),$
and $\left(\mathcal{Y},\mathcal{G}\right)$ be measurable spaces.
Then

\begin{enumerate}[label=(\roman*),leftmargin=.1\linewidth,rightmargin=.4\linewidth]
\item $\sigma \left( \pi_1, \pi_2 \right) = \mathcal{F}_1 \otimes \mathcal{F}_2$
\item  $T:\left(\mathcal{Y},\mathcal{G}\right) \to \left(\X_{1}\times\mathcal{\X}_{2},\F_{1}\otimes\F_{2}\right)$ is measurable if and only if the maps $\pi_i \circ T :\left(\mathcal{Y},\mathcal{G}\right) \to \left(\mathcal{X}_i, \mathcal{F}_i\right)$ are measurable for $i=1,2$.
\item If $ S:\left(\X_{1}\times\mathcal{\X}_{2},\F_{1}\otimes\F_{2}\right) \to \left(\mathcal{Y},\mathcal{G}\right)$ is measurable then the maps $x_1 \to S(x_1,x_2)$ and $x_2 \to S(x_1, x_2)$ are measurable for all $x_2 \in \mathcal{X}_2$ and $x_1 \in \mathcal{X}_1$, respectively.
\end{enumerate}
\end{prop}

\begin{proof}
For $\left(i\right)$, note that for any $F\in\mathcal{F}_{1}$, $\pi_{1}^{-1}\left[F\right]=F\times\X_{2}\in\mathcal{F}_{1}\otimes\mathcal{F}_{2}$
and so $\pi_{1}$ is measurable with respect to $\mathcal{F}_{1}\otimes\mathcal{F}_{2}$
and so by definition $\sigma\left(\pi_{1}\right)\subseteq\mathcal{F}_{1}\otimes\mathcal{F}_{2}$.
Similarly, $\sigma\left(\pi_{2}\right)\subseteq\mathcal{F}_{1}\otimes\mathcal{F}_{2}$
and so $\sigma\left(\pi_{1},\pi_{2}\right)\subseteq\mathcal{F}_{1}\otimes\mathcal{F}_{2}.$
Conversely, for any $F_{1}\in\mathcal{F}_{1}$ and $F_{2}\in\mathcal{F}_{2}$,
\[
F_{1}\times F_{2}=\left(F_{1}\times\X_{2}\right)\cap\left(\X_{1}\times F_{2}\right)\in\sigma\left(\pi_{1},\pi_{2}\right)
\]
 and so $\mathcal{F}_{1}\otimes\mathcal{F}_{2}=\sigma\left(\left\{ F_{1}\times F_{2}\mid F_{1}\in\F_{1},F_{2}\in\F_{2}\right\} \right)\subseteq\sigma\left(\pi_{1},\pi_{2}\right).$

Next, for $\left(ii\right)$, first note that if $T$ is measurable,
then $\pi_{i}\circ T$ is measurable since $\pi_{i}\in\mathcal{M}\left(\X_{1}\times\X_{2},\mathcal{F}_{1}\otimes\mathcal{F}_{2}\right)$
by part $\left(i\right)$ and the composition of measurable functions
is measurable. Conversely, if both $\pi_{i}$ and $\pi\circ T$ are
measurable, then for any $F_{1}\in\F_{1}$ and $F_{2}\in\F_{2}$
\begin{align*}
T^{-1}\left[F_{1}\times F_{2}\right] & =T^{-1}\left[\left(F_{1}\times\X_{2}\right)\cap\left(\X_{1}\times F_{2}\right)\right]\\
 & =T^{-1}\left[\pi_{1}^{-1}\left[F_{1}\right]\cap\pi_{2}^{-1}\left[F_{2}\right]\right]\\
 & =\left(\pi_{1}\circ T\right)^{-1}\left[F_{1}\right]\cap\left(\pi_{2}\circ T\right)^{-1}\left[F_{2}\right]
\end{align*}
where both terms in the intersection are in $\mathcal{G}$ by the
measurability of $\pi_{i}\circ T$. Then a standard generating class
argument (Theorem \ref{thm:genericGeneratingClassArgument}) yields
the result.

Finally, note that $x_{1}\to S\left(x_{1},x_{2}\right)$ for some
fixed $x_{2}$ can be thought of as the composition $S\circ i_{x_{2}}$
where $i_{x_{2}}:\X_{1}\to\X_{1}\times\X_{2}$ is given by $i_{x_{2}}\left(x_{1}\right)=\left(x_{1},x_{2}\right)$
. $S$ is measurable by assumption and $i_{x_{2}}^{-1}\left[F_{1}\times F_{2}\right]=F_{1}$
and so $i_{x_{2}}$ is measurable by the standard generating class
argument. A composition of measurable maps is measurable and the result
follows. The argument for $x_{2}\to S\left(x_{1},x_{2}\right)$ is
analagous.
\end{proof}
This result allows us to show that our definition of the $\sigma-$algebra
generated by multiple functions is quite natural.
\begin{cor}
\label{cor:sigmaTupleisSigmaUnpack}Let $\X$ be a set, let $\left(\mathcal{Y}_{i},\mathcal{G}_{i}\right)$
be a measurable spaces, for $i\in\left\{ 1,2\right\} $, and let $\left(\mathcal{Y}_{1}\times\mathcal{Y}_{2},\mathcal{G}_{1}\otimes\mathcal{G}_{2}\right)$
be the product measurable space. For any functions $f_{1}:\X\to\mathcal{Y}_{1}$
and $f_{2}:\X\to\mathcal{Y}_{2}$ , along with a function $h\left(x\right)=\left(f_{1}\left(x\right),f_{2}\left(x\right)\right)$,
we have that 
\[
\sigma\left(h\right)=\sigma\left(f_{1},f_{2}\right).
\]
\end{cor}

\begin{proof}
Note by Proposition \ref{prop:altProductSigma}, $h$ is measurable
with respect to $\sigma\left(f_{1},f_{2}\right)$ since $f_{1}=\pi_{1}\circ h$
and $f_{2}=\pi_{2}\circ h$ are measurable with respect to $\sigma\left(f_{1},f_{2}\right)$
and so $\sigma\left(h\right)\subseteq\sigma\left(f_{1},f_{2}\right).$
Conversely, again by Proposition \ref{prop:altProductSigma}, both
$f_{1}$ and $f_{2}$ are measurable with respect to $\sigma\left(h\right)$
since $h$ is measurable with respect to $\sigma\left(h\right)$.
Thus $\sigma\left(f_{1},f_{2}\right)\subseteq\sigma\left(h\right).$
\end{proof}
\begin{prop}
\label{prop:productGenerator}Let $\left(\X_{i},\F_{i}\right)$ $i=1,2$
be measurable spaces where there exist some collection of sets $\mathcal{E}_{i}\subset\F_{i}$
such that $\X_{i}\in\mathcal{E}_{i}$ and 
\[
\sigma\left(\mathcal{E}_{i}\right)=\F_{i}.
\]
Then, letting $\mathcal{E}:=\left\{ E_{1}\times E_{2}\mid E_{1}\in\mathcal{E}_{1},E_{2}\right\} $,
we have that 
\[
\sigma\left(\mathcal{E}\right)=\F_{1}\otimes\F_{2}.
\]
\end{prop}

\begin{proof}
Let $\F:=\left\{ F_{1}\times F_{2}\mid F_{1}\in\F_{1},F_{2}\in\F_{2}\right\} .$Obviously,
$\mathcal{E}\subseteq\mathcal{F}$ and so $\sigma\left(\mathcal{E}\right)\subseteq\sigma\left(\F\right)=\F_{1}\otimes\F_{2}.$
To see the reverse inclusion, take the projections $\pi_{i}:\X_{1}\times\X_{2}\to\X_{i}$.
We wish to show these functions are $\sigma\left(\mathcal{E}\right)/\F_{i}$
measurable. To that end, note that for any $E\in\mathcal{E}_{1}$,
$\pi_{1}^{-1}\left[E\right]=E\times\X_{2}\in\mathcal{E}\subseteq\sigma\left(\mathcal{E}\right)$
and so a generating class argument tells us that $\pi_{1}$ is measurable
as needed. An analagous argument establishes the measurability of
$\pi_{2}$ and so 
\[
\F_{1}\otimes\F_{2}=\sigma\left(\pi_{1},\pi_{2}\right)\subseteq\sigma\left(\mathcal{E}\right)
\]
where the equality is due to Proposition \ref{prop:altProductSigma}.
\end{proof}
\begin{example}
\label{exa:XinEnecessaryForProductGeneration}In the proposition above,
the requirement that $\X_{i}\in\mathcal{E}_{i}$ is necessary. Let
$\X_{1}=\X_{2}=\left\{ 1,2,3\right\} $. Suppose then that $\mathcal{E}_{1}=\left\{ \left\{ 1\right\} \right\} $and
$\mathcal{E}_{2}=\left\{ \left\{ 2\right\} \right\} $ and so
\begin{align*}
\F_{1} & :=\sigma\left(\mathcal{E}_{1}\right)=\left\{ \left\{ 1,2,3\right\} ,\emptyset,\left\{ 1\right\} ,\left\{ 2,3\right\} \right\} \\
\F_{2} & :=\sigma\left(\mathcal{E}_{2}\right)=\left\{ \left\{ 1,2,3\right\} ,\emptyset,\left\{ 2\right\} ,\left\{ 1,3\right\} \right\} .
\end{align*}
Meanwhile, $\mathcal{E}=\left\{ \left\{ \left(1,2\right)\right\} \right\} $
and so
\[
\sigma\left(\mathcal{E}\right)=\left\{ \left\{ 1,2,3\right\} \times\left\{ 1,2,3\right\} ,\emptyset,\left\{ \left(1,2\right)\right\} ,\left\{ \left(1,2\right)\right\} ^{C}\right\} 
\]
which doesnt contain $\left\{ 1\right\} \times\left\{ 1,2,3\right\} \in\F_{1}\otimes\F_{2}.$
\end{example}

\begin{prop}
\label{prop:borelProductInclusion}Let $\left(\X_{1},\tau_{1}\right)$
and $\left(\X_{2},\tau_{2}\right)$ be topological spaces with Borel
sigma algebras $\borel\left(\X_{1}\right)=\sigma\left(\tau_{1}\right)$
and $\borel\left(\X_{2}\right)=\sigma\left(\tau_{2}\right).$ Then,
letting $\tau_{1}\otimes\tau_{2}$ denote the box topology\footnote{which is equivalent to the product topology for finite products}i.e.
the topology generated by $\left\{ O_{1}\times O_{2}\mid O_{1}\in\tau_{1},O_{2}\in\tau_{2}\right\} $,
we have that 
\[
\borel\left(\X_{1}\right)\otimes\borel\left(\X_{2}\right)\subseteq\borel\left(\X_{1}\times\X_{2}\right):=\sigma\left(\tau_{1}\otimes\tau_{2}\right).
\]
Moreover, in the case of Euclidean spaces
\[
\borel\left(\R\right)\otimes\borel\left(\R\right)=\borel\left(\R^{2}\right).
\]
\end{prop}

\begin{proof}
Observe that since $\X_{i}\in\tau_{i}$ $,i=1,2$, $\sigma\left(\left\{ O_{1}\times O_{2}\mid O_{1}\in\tau_{1},O_{2}\in\tau_{2}\right\} \right)=\borel\left(\X_{1}\right)\otimes\borel\left(\X_{2}\right)$
by Proposition \ref{prop:productGenerator}. Note that the box topology
$\tau_{1}\otimes\tau_{2}=\tau\left(\left\{ O_{1}\times O_{2}\mid O_{1}\in\tau_{1},O_{2}\in\tau_{2}\right\} \right)$;
that is, it is the \emph{topology generated by $\left\{ O_{1}\times O_{2}\mid O_{1}\in\tau_{1},O_{2}\in\tau_{2}\right\} $
}(exactly analagous to a $\sigma-$algebra generated by a collection
of sets). Since $\left\{ O_{1}\times O_{2}\mid O_{1}\in\tau_{1},O_{2}\in\tau_{2}\right\} \subseteq\tau\left(\left\{ O_{1}\times O_{2}\mid O_{1}\in\tau_{1},O_{2}\in\tau_{2}\right\} \right)=\tau_{1}\otimes\tau_{2}$,
\[
\borel\left(\X_{1}\right)\otimes\borel\left(\X_{2}\right)=\sigma\left(\left\{ O_{1}\times O_{2}\mid O_{1}\in\tau_{1},O_{2}\in\tau_{2}\right\} \right)\subseteq\sigma\left(\tau_{1}\otimes\tau_{2}\right)=\borel\left(\X_{1}\times\X_{2}\right)
\]
which is the first claim.

The second claim makes use of the separability of $\R$. We extend
the argument from Lemma \ref{lem:openSetDisjointUnionInterval} to
$\R^{2}$ to show that any open set in $\R^{2}$ can be written as
a countable union of open rectangles in $\R^{2}$. Take any open set
$U\subseteq\R^{2}$. By definition, for any norm $\lVert\cdot\rVert$
on $\R^{2}$, for any $x:=\left(x_{1},x_{2}\right)\in U$ there exists
some $r>0$ such that 
\[
B_{\lVert\cdot\rVert}\left(x,r\right)\subset U.
\]
Since norms on $\R^{2}$ are all equivalent, let $\lVert\cdot\rVert=\lVert\cdot\rVert_{\infty}$
and notice that $y\in B_{\lVert\cdot\rVert_{\infty}}\left(x,r\right)$
if and ony if $y\in\left(x_{1}-r,x_{1}+r\right)\times\left(x_{2}-r,x_{2}+r\right)$.
Now note that since the rationals are dense in $\R$ , we can find
rational endpoints $a_{i,x},b_{i,}$ such that $\left(a_{i,x},,b_{i,x}\right)\subseteq\left(x_{i}-r,x_{i}+r\right)$
and $x_{i}\in\left(a_{i,x},b_{i,x}\right).$ Then, 
\[
U=\bigcup_{x\in U}\left(a_{1,x},,b_{1,x}\right)\times\left(a_{2,x},,b_{2,x}\right).
\]
Since the endpoints are rational, there can only be countably many
distinct such rectangles and so our claim follows. Now note that open
rectangles are in $\borel\left(\R\right)\otimes\borel\left(\R\right)$
by definition, and so every open set in $\R^{2}$ is a countable union
of sets in $\borel\left(\R\right)\otimes\borel\left(\R\right).$ Let
the collection of open sets in $\R^{2}$ be denoted $\mathcal{O}$.
Then
\[
\mathcal{O}\subseteq\borel\left(\R\right)\otimes\borel\left(\R\right)\implies\borel\left(\R^{2}\right)\subseteq\borel\left(\R\right)\otimes\borel\left(\R\right)
\]
which completes the proof.
\end{proof}
We can use the above result to construct classic counterexamples to
Tonelli and Fubini when one of their hypotheses is not satsfied.
\begin{example}
\label{exa:tonelliFailNonSigmaFinite}Let $\left(\left[0,1\right],\borel\left(\left[0,1\right]\right),\lambda\right)$
and $\left(\left[0,1\right],2^{\left[0,1\right]},\mu_{0}\right)$
where $\lambda$ and $\mu_{0}$ are the Lebesgue and counting measures,
respectively. Now consider the product space $\left(\left[0,1\right]^{2},\borel\left(\left[0,1\right]\right)\otimes2^{\left[0,1\right]},\lambda\otimes\mu_{0}\right)$
and observe that by a variant of the argument in Proposition \ref{prop:borelProductInclusion},
the diagonal set $D=\left\{ \left(x,y\right)\in\left[0,1\right]^{2}\mid x=y\right\} \in\borel\left(\left[0,1\right]^{2}\right)=\borel\left(\left[0,1\right]\right)\otimes\borel\left(\left[0,1\right]\right)\subseteq\borel\left(\left[0,1\right]\right)\otimes2^{\left[0,1\right]}$
since it is closed. Then
\begin{align*}
\lambda^{x}\mu_{0}^{y}\left(\indicate_{D}\right) & =\lambda^{x}\left(1\left\{ x\in\left[0,1\right]\right\} \right)\\
 & =1\\
 & \neq0\\
 & =\mu_{0}^{y}\left(0\indicate\left\{ y\in\left[0,1\right]\right\} \right)\\
 & =\mu_{0}^{y}\lambda^{x}\left(\indicate_{D}\right).
\end{align*}
Note that $\mu_{0}$ is not $\sigma-$finite.
\end{example}


\subsection{Radon-Nikodym derivatives}
\begin{prop}
\label{prop:prodRadonNikodymDerivatives}Let $\left(\X,\F\right)$
and $\left(\mathcal{Y},\mathcal{G}\right)$ be measurable spaces with
product space $\left(\X\times\mathcal{Y},\F\otimes\mathcal{G}\right).$
Let $\mu_{1},\mu_{2}$ be $\sigma-$finite measures on $\F$ such
that $\mu_{1}<<\mu_{2}$. Similarly, let $\nu_{1},\nu_{2}$ be measures
on $\mathcal{G}$ such that $\nu_{1}<<\nu_{2}$. Then $\mu_{1}\otimes\nu_{1}$
and $\mu_{2}\otimes\nu_{2}$ are $\sigma-$finite, $\mu_{1}\otimes\mu_{2}<<\nu_{1}\otimes\nu_{2}$
and
\[
\frac{d\left(\mu_{1}\otimes\nu_{1}\right)}{d\left(\mu_{2}\otimes\nu_{2}\right)}\stackrel{\text{a.e}}{=}\frac{d\mu_{1}}{d\mu_{2}}\times\frac{d\nu_{1}}{d\nu_{2}}.
\]
\end{prop}

\begin{rem*}
Note that the almost everywhere equality here holds with respect to
both product measures.
\end{rem*}
\begin{proof}
Note that the $\sigma-$finiteness of the products follows from the
fact that if $\left\{ F_{i}\right\} _{i\in\N}\in\F$ such that $\bigcup_{i\in\N}F_{i}=\X$
and $\mu_{1}\left(F_{i}\right)<\infty$ (and similarly for $\left\{ G_{j}\right\} _{j\in\N}\in\mathcal{G})$
then $\bigcup_{i\in\N}\bigcup_{j\in\N}F_{i}\times G_{j}=\X\times\mathcal{Y}$and
$\mu_{1}\left(F_{i}\right)\nu_{1}\left(G_{j}\right)<\infty$. The
same argument works for the other pair of measures. Then, notice that
for any $F\in\F\otimes\mathcal{G}$, 
\begin{align*}
\mu_{1}\otimes\nu_{1}\left(F\right) & =\mu_{1}^{x}\nu_{1}^{y}\left(\indicate_{F}\left(x,y\right)\right)\\
 & =\mu_{1}^{x}\nu_{2}^{y}\left(\indicate_{F}\left(x,y\right)\frac{d\nu_{1}}{d\nu_{2}}\right)\\
 & =\mu_{2}^{x}\left(\nu_{2}^{y}\left(\indicate_{F}\left(x\right)\frac{d\nu_{1}}{d\nu_{2}}\right)\frac{d\mu_{1}}{d\mu_{2}}\right)\\
 & =\mu_{2}^{x}\nu_{2}^{y}\left(\indicate_{F}\left(x\right)\frac{d\nu_{1}}{d\nu_{2}}\frac{d\mu_{1}}{d\mu_{2}}\right)
\end{align*}
where the first equality is Tonelli (and the uniquenes of product
measures), the second is due the fact that $y\to\indicate_{F}\left(x,y\right)$
is a non-negative measurable function for every $x\in\X$ (by Lemma
\ref{lem:partialFunctionMeasurability}) and so Corollary \ref{cor:radonNikodymIntegral}
applies. The third equality is similar: now we have the function $x\to\nu_{2}^{y}\left(\indicate_{F}\left(x,y\right)\frac{d\nu_{1}}{d\nu_{2}}\right)$
which is non-negative since the the integrand is non-negative function
for every $x\in\X$. It's also measurable by Lemma \ref{lem:partialIntegralMeasurability}
and so Corollary \ref{cor:radonNikodymIntegral} again applies. Finally,
we apply linearity (since $\frac{d\mu_{1}}{d\mu_{2}}$ is constant
in $y$) in the last line. This completes the proof.
\end{proof}
This result has a converse, in that for any $\sigma-$finite measure
$\nu$ on $\F\otimes\mathcal{G}$ which is absolutely continuous with
respect to a $\sigma-$finite product measure and whose Radon Nikodym
derivative can be written as a product is itself a product measure.
We make this precise with the following result.
\begin{prop}
\label{prop:prodRNConverse}Let $\left(\X,\F\right)$ and $\left(\mathcal{Y},\mathcal{G}\right)$
be measurable spaces with product $\left(\X\times\mathcal{Y},\F\otimes\mathcal{G}\right)$.
Let $\nu_{1}$ and $\mu_{1}$ be $\sigma-$finite measures on $\F$
such that $\nu_{1}<<\mu_{1}$ . Similarly, let $\nu_{2}$ and $\mu_{2}$
be $\sigma-$finite measures on $\mathcal{G}$ such that $\nu_{2}<<\mu_{2}$.
Finally, let $\gamma$ be a measure on $\F\otimes\mathcal{G}$ such
that $\gamma<<\mu_{1}\otimes\mu_{2}$ and
\[
\frac{d\gamma}{d\mu_{1}\otimes\mu_{2}}\stackrel{\text{a.e}}{=}\frac{d\nu_{1}}{d\mu_{1}}\times\frac{d\nu_{2}}{d\mu_{2}}.
\]
Then $\gamma=\nu_{1}\otimes\nu_{2}$.
\end{prop}

\begin{proof}
Note that for any $A\in\F\otimes\mathcal{G}$
\begin{align*}
\gamma\left(A\right) & =\mu_{1}^{x}\mu_{2}^{y}\left(\indicate_{A}\left(x,y\right)\frac{d\nu_{1}}{d\mu_{1}}\left(x\right)\frac{d\nu_{2}}{d\mu_{2}}\left(y\right)\right)\\
 & =\mu_{1}^{x}\nu_{2}^{y}\left(\indicate_{A}\left(x,y\right)\frac{d\nu_{1}}{d\mu_{1}}\left(x\right)\right)\\
 & =\nu_{2}^{y}\mu_{1}^{x}\left(\indicate_{A}\left(x,y\right)\frac{d\nu_{1}}{d\mu_{1}}\left(x\right)\right)\\
 & =\nu_{2}^{y}\nu_{1}^{x}\left(\indicate_{A}\left(x,y\right)\right)\\
 & =\nu_{2}\otimes\nu_{1}\left(A\right)
\end{align*}
where the second equality follows by Corollary\ref{cor:radonNikodymIntegral},
the third equality by Tonelli, the fourth again by Corollary \ref{cor:radonNikodymIntegral}.
\end{proof}
\begin{rem*}
As we shall see, this results in this section are a measure theoretic
justification for the probabilistic fact that two \emph{absolutely
continuous }random variables $X$ and $Y$ are \emph{independent }if
and only if we can factor their probability density functions.
\end{rem*}

\subsection{Integration by parts}

Tonelli (or Fubini) gives us a generalization of integration by parts
from elementary calculus. To see this, let $F$ and $G$ be Stieljes
(that is, real valued, nondecreasing, and right continuous) functions
on a compact set $\left[a,b\right].$ By Theorem \ref{thm:stieljesMeasure},
we have that the set functions 
\begin{align*}
\mu\left(\left(x,y\right]\right) & :=F\left(y\right)-F\left(x\right)\\
\nu\left(\left(x,y\right]\right) & :=G\left(y\right)-G\left(x\right)
\end{align*}
for any $a\leq x\leq y\leq b$ extend to measures on $\borel\left(\left[a,b\right]\right)$.
We can show that under mild regularity conditions, 
\[
\lebInt{\mu}G+\lebInt{\nu}F=F\left(b\right)G\left(b\right)-F\left(a\right)G\left(a\right).
\]
This implies the ``usual'' integration by parts formula since if $F$ and $G$ were absolutely continuous (as functions in the sense of Definition \ref{def:absolutelyContinuousFunction} ), then by Proposition \ref{prop:absoluteContinuityStieljesFunctions}, the measures $\mu$ and $\nu$ would be absolutely continuous (in the sense of Definition \ref{def:absoluteContinuityMeasures}) with respect to the Lebesgue measure with Radon Nikodym derivatives $f$ and $g$, respectively. Then, by Corollary \ref{cor:radonNikodymIntegral} our formula would be (in the traditional integral notation)
\[
	\int_{a}^{b}G(x)f(x)dx + \int_{a}^{b}F(x)g(x)dx = F\left(b\right)G\left(b\right)-F\left(a\right)G\left(a\right).
\]
The only piece missing from our usual formula is the fact that $F^\prime = f$ and $G^\prime = g$. As we had promised back in Chapter \ref{chap:Differentiation}, we will prove this later in this chapter.
\begin{thm}[Integration by Parts]
\label{thm:generalizedIntegrationByParts}Let $F$ and $G$ be Stieljes
functions on a compact set $\left[a,b\right]$. If the set of discontinuities
of $F$ and $G$ are disjoint, then
\[
\lebInt{\mu}G+\lebInt{\nu}F=F\left(b\right)G\left(b\right)-F\left(a\right)G\left(a\right).
\]
\end{thm}

\begin{proof}
Note that 
\begin{align*}
\lebInt{\mu}G & =\mu^{x}\left(G\left(x\right)\right)\\
 & =\mu^{x}\left(\nu\left(\left(a,x\right]\right)+G\left(a\right)\right)\\
 & =\mu^{x}\nu^{y}\left(\indicate\left\{ a<y\leq x\leq b\right\} \right)+G\left(a\right)\left(F\left(b\right)-F\left(a\right)\right)
\end{align*}
where the third equality uses linearity and the fact that $\mu\left(\left(a,b\right]\right)=F\left(b\right)-F\left(a\right).$
Similarly, we can write
\begin{align*}
\lebInt{\nu}F & =\nu^{y}\left(F\left(y\right)\right)\\
 & =\nu^{y}\left(\mu\left(\left(a,y\right]\right)+F\left(a\right)\right)\\
 & =\nu^{y}\mu^{x}\left(\indicate\left\{ a<x\leq y\leq b\right\} \right)+F\left(a\right)\left(G\left(b\right)-G\left(a\right)\right).
\end{align*}
Now note that
\begin{align}
\lebInt{\mu}G+\lebInt{\nu}F & =\mu^{x}\nu^{y}\left(\indicate\left\{ a<y\leq x\leq b\right\} \right)+\nu^{y}\mu^{x}\left(\indicate\left\{ a<x\leq y\leq b\right\} \right)+G\left(a\right)\left(F\left(b\right)-F\left(a\right)\right)+F\left(a\right)\left(G\left(b\right)-G\left(a\right)\right)\nonumber \\
 & =\mu^{x}\nu^{y}\left(\indicate\left\{ a<y\leq x\leq b\right\} +\indicate\left\{ a<x\leq y\leq b\right\} \right)+G\left(a\right)\left(F\left(b\right)-F\left(a\right)\right)+F\left(a\right)\left(G\left(b\right)-G\left(a\right)\right)\label{eq:muGPlusNuF}
\end{align}
where the second equality follows by Tonelli and linearity. We can
expand upon the term inside the integral by noticing that $\left\{ a<y\leq x\leq b\right\} =\underbrace{\left\{ a<y=x\leq b\right\} }_{=:A}\cup\underbrace{\left\{ a<y<x\leq b\right\} }_{=:B}$
where the union is disjoint and so by Fact \ref{fact:indicatorFunctionsFiniteOperations},
we know that 
\[
\indicate\left\{ a<y\leq x\leq b\right\} =\indicate_{A}+\indicate_{B}.
\]
Further, $\left\{ a<x\leq y\leq b\right\} =\left(a,b\right]^{2}\setminus B$.
Then
\begin{align}
\mu^{x}\nu^{y}\left(\indicate\left\{ a<y\leq x\leq b\right\} +\indicate\left\{ a<x\leq y\leq b\right\} \right) & =\mu^{x}\nu^{y}\left(\indicate_{A}+\indicate_{B}+\indicate_{\left(a,b\right]^{2}\setminus B}\right)\nonumber \\
 & =\mu^{x}\nu^{y}\left(\indicate_{A}\right)+\mu\otimes\nu\left(\left(a,b\right]^{2}\right)\nonumber \\
 & =\mu^{x}\nu\left(\left\{ x\right\} \right)+\left(F\left(a\right)-F\left(b\right)\right)\left(G\left(a\right)-G\left(b\right)\right)\nonumber \\
 & =\left(F\left(a\right)-F\left(b\right)\right)\left(G\left(a\right)-G\left(b\right)\right)\label{eq:muNuXEqualYZero}
\end{align}
where the third equality is due to the fact that $\mu\otimes\nu$
is a product measure, and the last equality follows by our assumption
that $F$ and $G$ contain no common points of discontinuity and Proposition
\ref{prop:stieljesSingleton}\hl{explain this better}. Then, substituting
(\ref{eq:muNuXEqualYZero}) into (\ref{eq:muGPlusNuF}) and doing
some algebra yields the result.
\end{proof}
An interesting application of this result is showing that the integral
of a Stieljes function $F$ under the measure induced by $F$ is simply
$\frac{F\left(b\right)^{2}-F\left(a\right)^{2}}{2},$ which follows
simply by applying the above theorem where $F=G.$ In probability
theory, this is a special case of the fact that a cumulative distribution
function of a random variable is itself uniformly distributed on $\left[0,1\right]$.
We shall establish this fact in Part II of these notes.

\subsection{Area under the graph of a function.}

In elementary calculus, we understood the ordinary (Riemann) integral
of a bounded function $f$ on a compact interval $\left[a,b\right]$
as representing the area of under the graph of $f$ on $\left[a,b\right].$
The Riemann theory formalized this by partitioning the interval $\left[a,b\right]$
into subintervals and computing the integral of step functions which
were constant on each subinterval in the partition. We can do this
more generally with Lebesgue integrals as well. That is to say, we
can show that for a non-negative measurable function $f:\R\to\R$
\[
\lebInt{\lambda}f=\lambda^{2}\left(\left\{ \left(x,t\right)\in\R^{2}\mid0\leq t<f\left(x\right)\right\} \right)
\]
where $\lambda^{2}$ is the 2-dimensional Lebesgue measure $\lambda\otimes\lambda$
on $\borel\left(\R^{2}\right)$. In fact, we can establish this result
more generally for other $\sigma-$finite measures using Tonelli's
theorem.
\begin{thm}
\label{thm:areaUnderGraphLebesgue}Let $\measurespace$ be a $\sigma-$finite
measure space and let $f\in\nonnegMeasurableFunctions.$ Then $\left\{ \left(x,t\right)\in\X\times\R\mid f\left(x\right)>t\geq0\right\} \in\F\otimes\borel\left(\R\right)$
and
\[
\lebInt{\mu}f=\lambda^{t}\left(\mu^{x}\left(\left\{ f\left(x\right)>t\right\} \right)\indicate\left\{ t\geq0\right\} \right)=\lambda\otimes\mu\left(\left\{ \left\{ f\left(x\right)>t\geq0\right\} \right\} \right)
\]
\end{thm}

\begin{proof}
First notice that function $T\left(x,t\right)=f\left(x\right)-t$
is $\F\otimes\mathcal{\borel\left(\R\right)}/\borel\left(\R\right)$
measurable. To see this, define $g\left(x,t\right)=f\left(x\right)$
and notice that for any $B\in\borel\left(\R\right):$$g^{-1}\left[B\right]=f^{-1}\left[B\right]\times\R\in\mathcal{F}\otimes\borel\left(\R\right)$.
Similarly, letting $h\left(x,t\right)=t$, we have that $h^{-1}\left[B\right]=\R\times B\in\F\otimes\borel\left(\R\right)$.
Then, using the fact that linear combinations of Borel measurable
functions are Borel measurable, the function $T$ is $\F\otimes\borel\left(\R\right)/\borel\left(\R\right)$
measurable. Notice then that $T^{-1}\left[\left(0,\infty\right)\right]=\left\{ \left(x,t\right)\in\R^{2}\mid f\left(x\right)>t\right\} $
is measurable. Further, the set $\left\{ \left(x,t\right)\in\R^{2}\mid t\geq0\right\} =\X\times\left[0,\infty\right)\in\F\otimes\borel\left(\R\right).$
Then, 
\[
\left\{ \left(x,t\right)\in\X\times\R\mid f\left(x\right)>t\geq0\right\} =T^{-1}\left[\left(0,\infty\right)\right]\cap\left(\X\times\left[0,\infty\right)\right)\in\F\otimes\borel\left(\R\right)
\]
and the first result follows. Then,
\begin{align*}
\lebInt{\mu}f & =\mu^{x}\left(\lambda^{t}\left(\indicate\left\{ 0\le t<f\left(x\right)\right\} \right)\right)\\
 & =\lambda^{t}\left(\indicate\left\{ t\geq0\right\} \mu^{x}\left(\indicate\left\{ f\left(x\right)>t\right\} \right)\right)\\
 & =\lambda^{t}\left(\mu^{x}\left(\left\{ f\left(x\right)>t\right\} \right)\indicate\left\{ t>0\right\} \right)
\end{align*}
where in the second equality we used Tonelli (\ref{thm:tonelli}),
linearity, and the fact that $\indicate_{A\cap B}=\indicate_{A}\indicate_{B}.$
Observe that Tonelli can be applied here since $\indicate\left\{ 0<t<f\left(x\right)\right\} \in\mathcal{M}^{+}\left(\X\times\R,\mathcal{F}\otimes\borel\left(\R\right)\right)$
and both $\mu$ and $\lambda$ are $\sigma-$finite. Note that Tonelli's
theorem also implies that 
\[
\lambda^{t}\left(\mu^{x}\left(\indicate\left\{ f\left(x\right)>t\geq0\right\} \right)\right)=\lambda\otimes\mu\left(\left\{ \left\{ f\left(x\right)>t\geq0\right\} \right\} \right).
\]
\end{proof}


\section{The Lebesgue measure on $\protect\R^{n}$}

\hl{Formalize extension of Fubini etc to $n$ dimension}

It should be easy to guess that by induction, we can extend the construction
of product measures to products of $n$$\sigma-$finite measure spaces.
That is, for spaces $\left\{ \left(\X_{i},\mathcal{F}_{i},\mu_{i}\right)\right\} _{i=1}^{n}$
, we can define the product space $\left(\prod_{i=1}^{n}\X_{i},\otimes_{i}\F_{i}\right)$
and show that for sigme finite measures $\mu_{i}$ on $\F_{i}$, there
exists a unique product measure $\otimes_{i=1}^{n}\mu_{i}$ such that
for $\prod F_{i}$, $F_{i}\in\F_{i}$
\[
\otimes_{i=1}^{n}\mu_{i}\left(\prod_{i=1}^{n}F_{i}\right)=\prod_{i=1}^{n}\mu_{i}\left(F_{i}\right).
\]
This requires us to use the canonical identification between $\prod_{i=1}^{n}\X_{i}$
and $\left(\prod_{i=1}^{n-1}\X_{i}\right)\times\X_{n}$ and the fact
that the product of $\sigma-$finite measures is $\sigma-$finite.
We can also similarly extend the Fubini-Tonelli theorems, so that
for any permutation $\sigma:\left\{ 1,2,\dots,n\right\} \to\left\{ 1,2,\ldots,n\right\} $
and $f\in\mathcal{M}^{+}\left(\prod_{i=1}^{n}\X_{i},\otimes_{i}\F_{i}\right)$
(or $f\in\Lp 1{\otimes_{i=1}^{n}\mu_{i}}$)
\[
\otimes_{i=1}^{n}\mu_{i}\left(f\right)=\mu_{\sigma\left(1\right)}\left(\mu_{\sigma\left(2\right)}\ldots\left(\mu_{\sigma\left(n\right)}\left(f\right)\right)\right).
\]

Using the above facts, we can easily construct the Lebesgue measure
$\lambda^{n}$ on $\borel\left(\R^{n}\right)=\otimes_{i=1}^{n}\borel\left(\R\right)$
as the product measure $\otimes_{i=1}^{n}\lambda$, where uniqueness
is guaranteed by the $\sigma-$finiteness of $\lambda$. Equivalently,
we can construct the Lebesgue measure using the Caratheodory approach,
by defining $\mathcal{L}^{n}=\left\{ \prod_{i=1}^{n}\left(a_{i},b_{i}\right]\mid\left(a_{i},b_{i}\right]\in\mathcal{L}\right\} $,
and restricting the canonical outer measure
\[
\lambda_{*}^{n}\left(A\right)=\inf\left\{ \sum_{j=1}^{\infty}\lambda_{1}^{n}\left(L_{j}\right)\mid A\subseteq\bigcup_{j\in\N}L_{j},L_{j}\in\mathcal{L}^{n}\right\} 
\]
to the Lebesgue $\sigma-$algebra $\mathcal{C}\left(\lambda_{*}^{n}\right)$.
This works because the collection $\mathcal{L}^{n}$ can be \hl{shown}
to be a semi-ring and $\lambda_{1}^{n}:\mathcal{L}^{n}\to\left[0,\infty\right]$
(here $\lambda_{1}^{n}\left(\prod_{i=1}^{n}L_{i}\right)=\prod_{i=1}^{n}\lambda_{1}\left(L_{i}\right)$where
$L_{i}\in\mathcal{L}$) can be shown to satisfy the requirements of
Theorem \ref{thm:semiRingCaratheodoryExtn}.

\subsection{Basic properties of the Lebesgue measure\label{subsec:lebesgueRNproperties}}

\subsubsection{Geometric properties}

The first property of the Lebesgue measure that we establish is a
generalization of Lemma \ref{lem:outerMeasureScales} to $n-$dimensions.
\begin{prop}
\label{prop:outerMeasureScalesNDim}For $\delta\neq0$ and any $A\in\borel\left(\R^{n}\right)$
we have $\delta B\in\borel\left(\R^{n}\right)$. Further, for any
$A\subseteq\R^{n}$
\[
\lambda_{*}^{n}\left(\delta A\right)=\lvert\delta\rvert^{n}\lambda_{*}^{n}\left(A\right).
\]
\end{prop}

\begin{proof}
Measurability follows by a simple generating class argument as in
the Lemma (or alternatively, let $f\left(x\right)=\frac{x}{\delta}$
and notice that $f^{-1}$is a continuous function and $f^{-1}\left[B\right]=\delta B\in\borel\left(\R\right)$).
Next, notice that for $n=1$ this result is result is Lemma \ref{lem:outerMeasureScales}.
Suppose the result holds for $n-1$. Let $B\in\borel\left(\R^{n-1}\right)$
and $A\in\borel\left(\R\right)$, then 
\begin{align*}
\lambda^{n}\left(\delta\left(B\times A\right)\right) & =\lambda^{n}\left(\left(\delta B\right)\times\left(\delta A\right)\right)\\
 & =\lambda^{n-1}\left(\delta B\right)\lambda\left(\delta A\right)\\
 & =\lvert\delta^{n}\rvert\lambda^{n-1}\left(B\right)\lambda\left(A\right)\\
 & =\lvert\delta^{n}\rvert\lambda^{n}\left(B\times A\right).
\end{align*}
By the identification between $\R^{n-1}\times\R$ and $\R^{n}$, our
scaling result holds on $\mathcal{L}^{n}$. Extending the proof to
any arbitrary set $A\subseteq\R^{n}$proceeds exactly as in Lemma
\ref{lem:outerMeasureScales}.
\end{proof}
\begin{cor}
\label{cor:dilationIntegral}Let $\delta\neq0$ be arbitrary. For
any $f\in\mathcal{M}^{+}\left(\R^{n},\borel\left(\R^{n}\right)\right)$
(or $f\in\Lp 1{\lambda^{n}})$
\[
\lebInt{\lambda^{n}}{f\left(\delta x\right)}=\frac{1}{\lvert\delta^{n}\rvert}\lebInt{\lambda^{n}}{f\left(x\right)}.
\]
\end{cor}

\begin{proof}
Let $g\left(x\right)=\delta x$ and notice that by Proposition \ref{prop:outerMeasureScalesNDim},
the image measure $\lambda^{n}g$ agrees with the measure $\frac{1}{\lvert\delta^{n}\rvert}\lambda^{n}$
on all Borel sets. By the change of variables formula in Corollary
\ref{cor:changeOfVariables}, $\lebInt{\lambda^{n}}{f\left(\delta x\right)}=\lebInt{\lambda^{n}g}f=\frac{1}{\lvert\delta^{n}\rvert}\lebInt{\lambda^{n}}f.$
\end{proof}
We also have translation invariance, which is the analogue of Lemma
\ref{lem:outerMeasureTranslationInvariant}.
\begin{prop}
\label{prop:outerMeasureTranslationInvariantNDim}Let $A\subseteq\R^{n}$
be an arbitrary set. Then the translated set $A+t:=\left\{ a+t\mid a\in A\right\} $
has Lebesgue outer measure
\[
\lambda_{*}^{n}\left(A+t\right)=\lambda_{*}^{n}\left(A\right)
\]
for all $t\in\R$. Moreover, for any set $A\in\borel\left(\R^{n}\right),$$A+t\in\borel\left(\R^{n}\right).$
\end{prop}

\begin{proof}
Again, the measurability result is a generating class argument (or
let $f\left(x\right)=x-t$...). The result holds for $n=1$ by the
Lemma. Suppose it holds for $n-1$ . Letting $B\in\borel\left(\R^{n-1}\right)$
and $A\in\borel\left(\R\right)$, we have 
\begin{align*}
\lambda^{n}\left(\left(B\times A\right)+t\right) & =\lambda^{n}\left(\left(B+t\right)\times\left(A+t\right)\right)\\
 & =\lambda^{n-1}\left(B+t\right)\lambda\left(A+t\right)\\
 & =\lambda^{n-1}\left(B\right)\lambda\left(A\right)\\
 & =\lambda^{n}\left(B\times A\right).
\end{align*}
Extending this to the general case follows exactly as in the Lemma.
\end{proof}
\begin{cor}
\label{cor:translationInvarianceIntegral}Let $t\in\R$ be arbitrary.
For any $f\in\mathcal{M}^{+}\left(\R^{n},\borel\left(\R^{n}\right)\right)$
(or $f\in\Lp 1{\lambda^{n}})$
\[
\lebInt{\lambda^{n}}{f\left(x+t\right)}=\lebInt{\lambda^{n}}{f\left(x\right)}.
\]
\end{cor}

\begin{proof}
Let $g\left(x\right)=x+t$ and notice that by Proposition \ref{prop:outerMeasureTranslationInvariantNDim},
the image measure $\lambda^{n}g$ and $\lambda^{n}$ agree on all
Borel sets. But recall that by the general change of variables formula
in Corollary\ref{cor:changeOfVariables}, $\lebInt{\lambda^{n}}{f\left(x+t\right)}=\lebInt{\lambda^{n}g}f=\lebInt{\lambda^{n}}f.$
\end{proof}
These two results are rather banal because they don't offer much additional
insight over and above the corresponding results for the usual Lebesgue
measure on the line. In order to derive some interesting properties
of $\lambda^{n}$, we need to exploit the rich geometric structure
of the Euclidean space and the functions that act on it. In order
to do so, we need to review some basic linear algebra. Appendix \ref{sec:Determinants}
on determinants is key here; please re-read this section carefully
before we embark on the next result.
\begin{lem}
\label{lem:changeOfVariablesAffineElementary}Let $T:\R^{n}\to\R^{n}$
be an elementary map and let $f\in\mathcal{M}^{+}\left(\R,\borel\left(\R\right)\right)$
(or $f\in\Lp 1{\lambda^{n}}$). Then $f\circ T\in\mathcal{M}^{+}\left(\R,\F\right)$
($f\circ T\in\Lp 1{\lambda^{n}}$ and 
\[
\lvert\det\left(T\right)\rvert\lebInt{\lambda^{n}}{f\circ T}=\lebInt{\lambda^{n}}f.
\]
\end{lem}

\begin{proof}
The measurability is trivial and so first, suppose $T$ is the row
scaling operator, and note that if $T\left(x_{1},\ldots,x_{k},\ldots,x_{n}\right)=\left(x_{1},\ldots,cx_{k},\ldots x_{n}\right),$with
$c\ne0,$then $\det\left(T\right)=c$ and so by Tonelli (Fubini)
\begin{align*}
\lambda^{n}\left(f\circ T\right) & =\lambda^{x_{-k}}\lambda^{x_{k}}\left(f\left(x_{1},\ldots,cx_{k},\ldots,x_{n}\right)\right)\\
 & =\frac{1}{\lvert c\rvert}\lambda^{x_{-k}}\lambda^{x_{k}}\left(f\left(x_{1},\ldots,x_{k},\ldots x_{n}\right)\right)\\
 & =\frac{1}{\lvert c\rvert}\lambda^{n}\left(f\right)
\end{align*}
where in the second line we used Corollary \ref{cor:dilationIntegral}
for $n=1$. Next, suppose that $T$ is the row switching operator
which as $\det\left(T\right)=-1$ and clearly Tonelli-Fubini imply
that the integral is invariant under any permutation of indices and
so the result follows. Finally, suppose that $T$ is the row replacement
operator i.e $T\left(x_{1},\ldots,x_{l},\ldots x_{k},\ldots,x_{n}\right)=\left(x_{1},\ldots,x_{l}+cx_{k},\ldots x_{k},\ldots,x_{n}\right)$.
Again, by Tonelli (Fubini)
\begin{align*}
\lambda^{n}\left(f\circ T\right) & =\lambda^{x_{-l}}\lambda^{x_{l}}\left(f\left(x_{1},\ldots,x_{l}+cx_{k},\ldots,x_{n}\right)\right)\\
 & =\lambda^{x_{-l}}\lambda^{x_{l}}\left(f\left(x_{1},\ldots,x_{l},\ldots x_{k},\ldots,x_{n}\right)\right)\\
 & =\lambda^{n}\left(f\right)
\end{align*}
which establishes the result since $\det\left(T\right)=1.$
\end{proof}
\begin{lem}
\label{lem:changeOfVariablesLinear}Let $T:\R^{n}\to\R^{n}$ be an
invertbe linear map and let $f\in\mathcal{M}^{+}\left(\R,\borel\left(\R\right)\right)$
(or $f\in\Lp 1{\lambda^{n}}$). Then $f\circ T\in\mathcal{M}^{+}\left(\R,\F\right)$
($f\circ T\in\Lp 1{\lambda^{n}}$ and
\[
\lvert\det\left(T\right)\rvert\lambda^{n}\left(f\circ T\right)=\lambda^{n}\left(f\right).
\]
\end{lem}

\begin{proof}
The measurability is again trivial. Since the matrices of arbitrary
invertible maps are the product of elementary matrices, all we need
to show is that for two elementary matrices $E_{1},E_{2}$ , the product
matrix $E=E_{1}E_{2}$ satisfies the conclusion of the theorem. The
general claim then follows by induction. Clearly, 
\begin{align*}
\lvert\det\left(E\right)\rvert\lambda^{n}\left(f\circ E\right) & =\lvert\det\left(E_{1}\right)\rvert\lvert\det\left(E_{2}\right)\rvert\lambda^{n}\left(\left(f\circ E_{1}\right)\circ E_{2}\right)\\
 & =\lvert\det\left(E_{1}\right)\rvert\lambda^{n}\left(f\circ E_{1}\right)\\
 & =\lambda^{n}\left(f\right)
\end{align*}
where the first equality uses the associativity of composition and
Proposition \ref{prop:detProduct}, the second equality uses Lemma
\ref{lem:changeOfVariablesAffineElementary} applied to $T=E_{2}$
and $f=f\circ E_{1}$. The last equality is yet another application
of the previous lemma.
\end{proof}
Note that a corollary of this result is the fact that the Lebesgue
measure is rotationally invariant.
\begin{cor}
\label{cor:lebesgueRotationalInvariance}Let $R:\R^{n}\to\R^{n}$
be a rotation. Then for any Borel set $B\in\borel\left(\R^{n}\right)$
\[
\lambda^{n}\left(R\left(B\right)\right)=\lambda^{n}\left(B\right).
\]
\end{cor}

\begin{proof}
Note that rotations are orthogonal linear maps in that $RR^{T}=I$
and so 
\[
\det\left(R\right)=\frac{1}{\det\left(R^{T}\right)}.
\]
But recall from Proposition \ref{prop:detTranspose} that $\det\left(R\right)=\det\left(R^{T}\right)$
and so
\[
\lvert\det\left(T\right)\rvert=1.
\]
We then apply Lemma  \ref{lem:changeOfVariablesLinear} with $f=\indicate_{B}$
and $T=R$.
\end{proof}

\subsubsection{Topological properties}

We have seen some (but not all!) of the nice \emph{geometric} properties of the Lebesgue measure (mainly that it's invariant under rigid motions). Before we can generalize the Lemma \ref{lem:changeOfVariablesLinear} to non-linear maps, we also need to establish some of the basic \emph{topological} properties of the Lebesgue measure ( we hinted at at this with Proposition \ref{prop:equivalentLebesgueMeasurability} for the one dimensional Lebesgue measure). Later, in Chapter \ref{chap:measureAndTopology}, we will explore the link between measures and topology more deeply.

\hl{REVIEW}
\begin{thm}
	\label{thm:lebesgueTopologicalProperty}
	Let $E \subset \R^n$ be Lebesgue measurable. Then
	\[
		\lambda^n\left(E\right) = \inf \left\{\lambda^n \left(O\right) \mid E \subseteq O, O \text{ is open }\right\}
	\]
	and 
	\[
		\lambda^n\left(E\right) = \sup \left\{\lambda^n \left(K\right) \mid K \subseteq E, K \text{ is compact }\right\}
	\]
\end{thm}
\begin{proof}
The proof of the first claim is almost identical to the proof of Proposition \ref{prop:borelApproximateLebesgue}. First assume that $\lambda(E) < \infty$ and note that by the outer measure construction of the $n$-dimensional Lebesgue measure, for any $\epsilon > 0$, there exist some $\{L_i\}_{i\in\N} \in \mathcal{L}$ with $E \subset \bigcup_{i\in\N} L_i$,  where $L_i = \prod_{j=1}^n\left(a_i^j, b_i^j\right]$ such that   
\[
	\lambda^n\left(E\right) \leq \sum_{i=1}^\infty \underbrace{\lambda^n \left(L_i\right)}_{=\prod_{j=1}^n \left(b_i^j - a_i^j\right) } < \lambda^n \left(E\right) + \frac{\epsilon}{2}.
\]
Note that we can for each $i \in \N$, we can choose some $\delta_i > 0$ such that
\[
	\prod_{j=1}^n \left(b_i^j - a_i^j + \delta_i\right) < \prod_{j=1}^n \left(b_i^j - a_i^j\right) + \frac{\epsilon}{4^i}.\footnote{We can do this by noting that that the expression on the left-hand side can be written as a symmetric elementary polynomial. Write $\Delta^j_i := b_i^j - a_i^j$ and note that 
		\[
			\prod_{j=1}^n \left( \Delta_i^j + \delta_i \right) = \sum_{k=0}^n e_k\left(\{\Delta_i^j\}_{j=1}^n\right)\delta_i^{n-k}
		\] where $e_0\left(\{\Delta_i^j\}_{j=1}^n\right) := 1$, $e_1\left(\{\Delta_i^j\}_{j=1}^n\right) := \sum_{j=1}^n \Delta_i^j$ and in general 
		\[
			e_k\left(\{\Delta_i^j\}_{j=1}^n\right) := \sum_{1 \leq j_1 < j_2 < \ldots < j_k \leq n} \prod_{l=1}^k \Delta_i^{j_l} 
		\] and so for small $\delta_i$ the higher power terms can effectively be ignored \hl{complete later} }
\]	Then writing
\[
 O := \bigcup_{i\in\N}\prod_{j=1}^n \left(b_i^j - a_i^j + \delta_i\right) 
\]
we have that $E \subseteq O$ and that
\begin{align*}
	\lambda^n \left(O\right) &\leq \sum_{i=1}^{\infty} \prod_{j=1}^n \left(b_i^j - a_i^j\right) +\frac{\epsilon}{4^i}\\
	&= \sum_{i=1}^\infty \lambda^n\left(L_i\right) + \frac{\epsilon}{4^i} \\
	&\leq \lambda^n\left(E\right)+ \epsilon
\end{align*}
and so the result follows when $E$ has finite measure. When it has infinite measure, we can simply follow the same kind of partitioning argument we saw in Proposition \ref{prop:borelApproximateLebesgue}. 

For the approximation by compact sets, begin once again by assuming that $\lambda^n(E) < \infty$. Note that the closure of $\bar{E}$ is compact by the Heine-Borel theorem since it is bounded\footnote{if $\forall x \in E: \lVert x \rVert < R$ then for any limit point $y$ of $E$ $\lVert y \rVert \leq R$ by the continuity of norms.} and closed. Now note that by the first part of this result, the measurable set $\bar{E} \setminus E$ is contained in some open set $O$ such that $\lambda^n(O \setminus \left(\bar{E} \setminus  \right)) < \epsilon $. Now let $K := \bar{E} \cap O^C$ which is closed since it is the intersection of two closed sets, and is bounded since $\bar{E}$ is bounded. Therefore $K$ is compact and since
\begin{align*}
	O^C &\subseteq \left(\bar{E} \setminus E\right)^C \\
	&= \left(\bar{E} \cap E^C \right)^C \\
	&= \bar{E}^C \cup E
\end{align*}
we have that $K = \bar{E} \cap O^C \subseteq \bar{E} \cap \left(\bar{E}^C \cup E\right) = E$. Finally, we have that 
\begin{align*}
	\lambda^n\left(E \setminus K\right) &= \lambda^n\left(E \cap K^C\right) \\
	&= \lambda^n\left(E \cap \left(\bar{E}^C \cup O\right)\right) \\
	&= \lambda^n\left(\left(E \cap \bar{E}^C\right) \cup  \left( E \cap O \right)\right) \\
	&= \lambda^n \left(E \cap O\right) \\
	&\leq \lambda^n\left(\bar{E}^C \cap O\right) + \lambda^n\left(O \cap E\right) \\
	&=\lambda^n\left(O \cap \left(\bar{E}^C \cup E\right)\right) \\
	&=\lambda^n \left(O \setminus \left(\bar{E}\setminus E\right)\right) \\
	&< \epsilon
\end{align*}  
which completes the proof for the finite measure case. The extension to the infinite measure case is the standard $\sigma-$finiteness argument we have always used.

\hl{Lusin's theorem: Rudin RCA (pp 55-56)}

\hl{Continuous compactly supported functions dense in Lp: Rudin RCA (pp 69)}

\end{proof}



\subsection{The Lebesgue differentiation theorem and the fundamental theorem
of calculus}

Establishing the change of variables theorem in full requires a difficult
measure theoretic result due to Lebesgue called the Lebesgue Differentiation
Theorem. It turns out that this result is also key to establishing
the fundamental theorem of calculus, and so we shall kill two birds with one stone and prove the 
fundamental theorem of calculus and the change of variables theorem together.

The Lebesgue Differentiation Theorem is built on top of lemmata that are important and interesting in their own right. We start with a \emph{covering lemma} 
which is a geometric argument that shows that a cover of a set $E$ that consists of bounded balls can be reduced to an enlargement of a countable subcollection of disjoint balls. Throughout, we shall be working in $\R^n$ where the standard notation $B_r\left(c\right)$ denotes an open ball of radius $r$ 
centered at $c$ under the standard Euclidean norm. We shall use superscripts to denote indices; that is to say, a ball $B^a$ -- where $A$ is an index set -- is short hand for
$B^a_r(c)$ where $r$ is the radius and $c$ is the center. We will use the notation $r(B^a)$ to denote the radius of the ball $B^a$ and $c(B^a)$ to denote its center. 

\begin{lem}
    \label{lem:coveringLemma} Let $E\subset \R^n$ be a subset of the Euclidean space such that there exists a  (possibly uncountable) collection $\{B^a\}_{a\in A}$ of balls 
    of radius at most $R$ that covers $E$. Then there exists a disjoint countable\footnote{or finite} subcollection $\{B^{a_{i}}\}_{i\in \N}$ of balls such that 
    \[
        \lambda^n(E) \leq 5^n \sum_{i=1}^\infty \lambda^n(B^{a_i}).
    \]
\end{lem}
\begin{proof}
    We can construct our subcollection inductively as follows: first, let $B^{a_1}$ be any ball in $\{B_a\}_{a\in A}$ such that its radius $r(B^{a_1}) \geq \frac{1}{2}\sup_{a\in A} r(B^a)$. This is possible because radii of the balls are bounded above
    by $R$. Then, we can define $B^{a_k}$ , $a_k \in A$ such that
    \[
        r(B^{a_k}) \geq \frac{1}{2}\sup \left\{ r(B^a) \mid B^a \cap B^{a_i} = \emptyset \text{ for } 1 \leq i \leq k-1 \right\}.
    \]
    This process can either terminate or continue indefinitely; this is 

    Next, note that if $\sum_{k=1}^\infty \lambda^n \left( B^{a_k} \right) = \infty $ then we are done. In the case that this sum is finite, we have that 
    $\lim_{k\to\infty} r\left( B^{a_k} \right) = 0$ and so for any $a^* \in A$ there exists some smallest $k^* \in \N$ such that for all $k \geq k^*$
    \[
        r\left(B^{a_k}\right) < \frac{1}{2}r\left( B^{a^*} \right).
    \]
    But by the construction of our sequence $B^{a_k}$, this means that there must exist some $i < k^*$ such that $B^{a_i} \cap B^{a^*} \neq \emptyset$. Let $x \in B^{a^*}$ be arbitrary, and pick some 
    $y \in B^{a_i} \cap B^{a^*}$. By our results so far, we have that 
    \begin{align*}
        \lVert x - c\left(B^{a_i}\right) \rVert &\leq \lVert x - y \rVert + \lVert y - c\left(B^{a_i}\right) \rVert \\
            &\leq 2 r\left(B^{a^*}\right) + r\left(B^{a_i}\right) \\
            & \leq 4 r\left(B^{a_i}\right) + r\left(B^{a_i}\right) \\
            &= 5r\left(B^{a_i}\right)
    \end{align*}
    where we have used the fact that for any $x,y \in B_r(c)$, $\lVert x - y \rVert \leq 2r $ in the first inequality, and the fact that $r(B^{a_i}) \geq \frac{1}{2}r\left( B^{a^*} \right)$ in the second
    inequality. Since $x$ was arbitrary, we have concluded that $B^{a^*} \subseteq 5B^{a_i}$. Since $a^* \in A$ was itself arbitrary, 
    we have that 
    \[
        E \subseteq \bigcup_{a \in A} B^a \subseteq \bigcup_{k=1}^\infty 5 B^{a_k}.
    \]
    Then, by Proposition \ref{prop:outerMeasureScalesNDim}, the monotonicity of measures, and countable subadditivity, our result follows.
\end{proof}

\begin{rem*}
	The bound is not sharp and the constant 5 is not necessarily optimal. Instead of choosing $r(B^{a_i}) \geq \frac{1}{2}\sup_{a\in A} r(B^a)$, we could have chosen $r(B^a_i) \geq c\sup_{a\in A} r(B^a)$ where $c \in (0,1)$. In this case, we get the inequality that
	\[
	 	\lambda^n (E) \leq \left(\frac{2+c}{c}\right)^n \sum_{i=1}^\infty \lambda^n \left( B^{a_i} \right).
	\]
	Since $c \to \frac{c+2}{c}$ is decreasing on $(0,1)$, it doesnt achieve a minimum, leaving us unable to optimize bound this way.
\end{rem*}

This geometric result appears to be a non-sequitur in our development of the fundamental theorem of calculus but it turns out to be indispensable to a key intermediate result
called the \emph{Hardy-Littlewood weak inequality}. Before we get to it, we need a special case of a result we shall later establish in Chapter \hl{Measure and Topology (?)}


\begin{defn}
\label{def:maximalFunction} A function $f \in \mathcal{M}\left(\R^n, \borel \left( \R^n \right) \right) $ is called \emph{locally integrable} if for any compact 
$K \subset \R^n$
\[
    \lambda^n \left( \lvert f \rvert \indicate_K \right) < \infty. 
\]
For any locally integrable function $f$, its \emph{maximal function} $Mf$ is defined 
\[
Mf(x) := \sup_{r > 0} \frac{1}{\lambda^n \left(B_r(x)\right) } \lambda^n \left( \lvert f \rvert \indicate_{B_r(x)} \right).
\] 
\end{defn}

\hl{REVIEW}
\begin{lem}
	\label{lem:ballBoundaryMeasureZero} For any $x \in \R$ and any $r > 0$, we have that 
	\[
		\lambda^n\left(\{y\in \R^n \mid \lVert y - x \rVert = r\}\right) = 0.
	\]
\end{lem}
\begin{proof}
	First let $x = 0$  and let $V := \lambda^n\left(B_1(0)\right)$. By the homogeneity of norms we have that $cB_1(0) = B_c(0)$ for any $c > 0$ and so by Proposition \ref{prop:outerMeasureScalesNDim}, $\lambda^n\left(B_c(0)\right) = c^nV$. Now notice that $\partial B_r(x) := \{y\in \R^n \mid \lVert y - x \rVert = r\}$ can be written as 
	\begin{align*}
		\partial B_r(x) &= \{y \in R^n \mid \lVert y - x \rVert \geq r \} \cap \{y \in R^n \mid \lVert y - x \rVert \leq r \} \\
		&= \bigcap_{m \in \N} \left\{y \in R^n \mid \lVert y - x \rVert \geq r - \frac{1}{m}  \right\} \cap \bigcap_{m \in \N}\left\{y \in R^n \mid \lVert y - x \rVert < r + \frac{1}{m} \right\} \\
		&=\bigcap_{m \in \N}  \left\{y \in R^n \mid \lVert y - x \rVert \geq r - \frac{1}{m}  \right\} \cap \left\{y \in R^n \mid \lVert y - x \rVert < r + \frac{1}{m} \right\} \\
		&= \bigcap_{m \in \N}  \left\{y \in R^n \mid \lVert y - x \rVert < r - \frac{1}{m}  \right\}^C \cap \left\{y \in R^n \mid \lVert y - x \rVert < r + \frac{1}{m} \right\} \\
		&= \bigcap_{m \in \N} B_{r+\frac{1}{m}}(x) \setminus B_{r-\frac{1}{m}}(x)
	\end{align*}
	which is the intersection of a decreasing sequence of (measurable) sets and thus is itself and measurable and so
	\begin{align*}
		\lambda^n\left(\partial B_r(0)\right) &= \lim_{m\to\infty} \lambda^n\left(B_{r+\frac{1}{m}}(0) \setminus B_{r-\frac{1}{m}}(0)\right) \\
		&= \lim_{m\to\infty}\left[ \lambda^n\left(B_{r+\frac{1}{m}}(0)\right) - \lambda^n\left(B_{r-\frac{1}{m}}(0)\right)\right] \\
		&= \lim_{m \to \infty} \left[ \left(r + \frac{1}{m}\right)^n -  \left(r - \frac{1}{m}\right)^n  \right] V \\
		&= 0
	\end{align*}
	where we have used the continuity of measures and finite additivity. 
	
	Of course, by translation invariance (\ref{prop:outerMeasureTranslationInvariantNDim}), this result holds for $\partial B_r(x) = \partial B_r(0) + x$.
	
\end{proof}

\begin{prop}
    \label{prop:maximalFunctionMeasurable} Let $f \in \mathcal{M}\left(\R^n, \borel \left( \R^n \right) \right) $ be locally integrable. Then
    $Mf:\R^n \to \R $ is Borel-measurable.
\end{prop}
\begin{proof}
    First, for a fixed $r>0$, observe that $x \to \lambda^n \left(\lvert f \rvert \indicate_{B_r(x)} \right)$ is continuous. To see this, let $\{x_n\}_{n \in \N} \in \R^n$ be a sequence
    converging to $x$.  Eventually, for large enough $n$ (say $n\geq n^*$), the sequence is in $B_r(x)$ and so for arbitrary $y \in B_r(x_n)$, $n \geq n^*$, we have that
    \begin{align*}
        \lVert y - x \rVert &\leq \lVert y - x_{n} \rVert + \lVert x_{n} - x \rVert  \\
        & \leq 2r 
    \end{align*}
    and so for large enough $n$, $B_r(x_n) \subseteq  B_{2r}(x)$. Writing $g_n := \lvert f \rvert \indicate_{B_r(x_n)}$, we claim that $g_n \to g:= \lvert f \rvert \indicate_{B_r(x)}$ almost everywhere. Indeed, notice that for any $y \in \R^n$, if $\lVert y - x\rVert > r$ then we can find some $\epsilon > 0$ such that $\lVert y - x \rVert = r - \epsilon$. Since $x_n \to x$, for large enough $n$, we have that $\lVert x - x_n \rVert < \frac{\epsilon}{2}$ and so 
    \begin{align*}
    	\lVert y - x_n \rVert &\leq \lVert y -x \rVert + \lVert x - x_n \\
    	&\leq r - \epsilon + \frac{\epsilon}{2}\\
    	&< r.
    \end{align*}
    Similarly, if $\lVert y - x \rVert > r$ we know there's some $\epsilon > 0$ such that $\lVert y - x \rVert = r + \epsilon$ and so with the same type of argument, we have for large $n$
    \begin{align*}
    	\lVert y - x_n \rVert &\geq \lVert y - x\rVert - \lVert x_n - x \rVert \\
    	&\geq r + \epsilon - \frac{\epsilon}{2} \\
    	&> \epsilon.
    \end{align*}
    Thus $\{ y \in \R^n \mid \lim g_n(y) \neq g(y)\} \subseteq \{y \in \R^n \mid \lVert y - x \rVert = r\}$ where this latter set has measure zero by Lemma \ref{lem:ballBoundaryMeasureZero}. Since $\overline{B_{2r}(x)}$ is compact, local integrability implies that the sequence of functions $g_n$ is dominated by the integrable function $h := \lvert f \rvert \indicate_{\overline{B_{2r}(x)}}$ and so by dominated convergence 
    \[
    	\lambda^n \left(g_n\right) \to \lambda^n \left(g\right).
    \]
    and so the continuity result follows.
    
    By a similar argument, $\frac{1}{\lambda^n \left(B_r(x)\right)}$ is continuous and thus so is their product. In other words, for any $a \in \R$ and $r>0$, $\{ x \in \R^n \mid \frac{1}{\lambda^n \left(B_r(x)\right) } \lambda^n \left( \lvert f \rvert \indicate_{B_r(x)} \right)  > a\}$ is open. Note that for any $x \in \R^n$ and fixed $ a \in \R$.
    \[
    	Mf(x) > a \Longleftrightarrow \exists r > 0 \text{ s.t } \frac{1}{\lambda^n \left(B_r(x)\right) } \lambda^n \left( \lvert f \rvert \indicate_{B_r(x)} \right)  > a
    \] and so 
    \[
    	\{ x \in \R^n \mid Mf(x) > a\} = \bigcup_{r > 0}  \left\{x \in \R^n \mid \frac{1}{\lambda^n \left(B_r(x)\right) } \lambda^n \left( \lvert f \rvert \indicate_{B_r(x)} \right)  > a \right\}
    \]
    which is open since arbitrary unions of open sets are open. Then, applying the standard generating class argument (Lemma \ref{lem:collectionIntervalsMeasurable}) yields the result.
\end{proof}


\hl{REVIEW}
\begin{thm}[Hardy-Littlewood]
	\label{thm:hardyLittlewoodWeakInequality} If $f \in \Lp{1}{\lambda^n}$ then 
	\[
		\lambda^n\left( \{\ x \in \R^n \mid Mf(x) > a \}\right) \leq \frac{5^n}{a}\lambda^n\left(\lvert f \rvert\right)
	\]
	for any $a > 0$.
\end{thm}

\begin{proof}
	Fix $a>0$ and consider, any $x_0 \in \{Mf(x) > a\}$. By the definition of the maximal function, there exists some $r_0>0$ such that
	\[
		\frac{1}{\lambda^n\left(B_{r_0}(x_0)\right)}\lambda^n\left(\lvert f \rvert \indicate_{B_{r_0}(x_0)}\right) > a
	\] 
	which is equivalent to saying
	\begin{align*}
		\lambda^n\left( B_{r_0}(x_0)\right) &< \frac{\lambda^n\left( \lvert f\rvert \indicate_{B_{r_0}(x_0)} \right)}{a} \\
		&\leq \frac{\lebInt{\lambda^n}{\lvert f \rvert }}{a}.
	\end{align*}
	Therefore $\{ x \in \R^n \mid Mf(x) > a\}$ can be covered by balls which are bounded above in measure by $R:=\frac{\lebInt{\lambda^n}{\lvert f \rvert }}{a}$. We can therefore apply Lemma \ref{lem:coveringLemma} and extract a countable subcollection of such balls $\{B_i\}_{i\in\N}$ which are pairwise disjoint such that
	\begin{align*}
		\lambda^n \left( \{ x \in \R^n \mid Mf(x) > a\} \right) &\leq  5^n \sum_{i=1}^n \lambda^n\left( B_i\right) \\
		&\leq \frac{5^n}{a} \sum_{i=1}^n \lambda^n\left(\lvert f \rvert \indicate_{B_i}\right) \\
		&= \frac{5^n}{a}\lambda^n\left( \lvert f \rvert \right)
	\end{align*}
where the equality at the end is via monotone convergence.
\end{proof}

We seem to be doing a lot of work to prove an $n-$dimensional generalization of the first fundamental theorem of calculus (FTC-1). The reason we need all this setup is that while the traditional FTC-1 applies to continuous functions, we are trying to extend the result to merely (locally) integrable functions. Fortunately for us, \hl{insert relevant theorem} tells us that continuous functions on compact support are dense in $\Lp{p}{\lambda^n}$.
\begin{defn}
	For any locally integrable $f \in \mathcal{M}\left(\R^n , \borel \left(\R^n\right)\right)$, a point $x \in \R$ is called a \emph{Lebesgue point} of $f$ if
	\[
		\lim_{r\to 0} \frac{1}{\lambda^n\left(B_r(x)\right)} \lambda^n_y\left(\lvert f(y) - f(x) \rvert \indicate_{B_r(x)}\right) = 0.
	\]
\end{defn}

It should be clear that the continuity points of $f$ are also Lebesgue points of $f$. Indeed, if $x$ is a continuity point, for any $\epsilon > 0$, there exists a $\delta > 0 $ such that when $y \in B_r(x)$ for $0< r \leq \delta$, we have that $\lvert f(y) - f(x) \rvert < \epsilon$. This ensures that for small enough $r$, 
\[
	\frac{1}{\lambda^n\left(B_r(x)\right)} \lambda^n_y\left(\lvert f(y) - f(x) \rvert \indicate_{B_r(x)}\right) \leq \frac{1}{\lambda^n\left(B_r(x)\right)} \lambda^n_y\left(\epsilon \indicate_{B_r(x)}\right) = \epsilon.
\]
Since $\epsilon$ was arbitrary the result follows. Using this fact, the density of continuous functions, and the Hardy-Littlewood inequality, we have the tools to show that almost every point in the domain of a locally integrable function is a Lebesgue point. We prove a slightly weaker result first.

\begin{lem}
	\label{lem:lebesgueDifferentiation}
	Let $f \in \mathcal{M}\left(\R^n , \borel \left(\R^n\right)\right)$ be locally integrable. Define
	\[
	f_r(x) := \frac{1}{\lambda^n \left(B_r(x)\right) } \lambda^n \left(  f \indicate_{B_r(x)} \right).
	\]
	Then $f_r \stackrel{\text{a.e}}{\to} f$.
\end{lem}
\begin{proof}
	Let $a > 0$ be fixed and first assume $f$ is integrable. By the \hl{insert theorem} density of continuous compactly supported functions in $\Lp{1}{\lambda^n}$, for any $\epsilon > 0$, there exists some function $g$ such that $\lVert f - g \rVert_1 < \epsilon$. Then, by the triangle inequality, monotonicity of measures, and finite subadditivity\footnote{We are using $\limsup$ rather than $\lim$ because we don't know the limit exists yet.}
	\begin{align*}
		\lambda^n\left(\limsup_{r\to 0}\lvert f_r(x) - f(x) \rvert > a\right) &\leq \lambda^n\left(\limsup_{r\to 0}\lvert f_r(x) - g_r(x) \rvert +  \limsup_{r\to 0}\lvert g_r(x) - g(x) \rvert + \lvert g(x) - f(x) \rvert > a\right) \\
		&\leq \lambda^n \left(\limsup_{r\to 0} \lvert f_r(x) - g_r(x) \rvert > \frac{a}{3}\right) + \lambda^n\left(\limsup_{r\to 0} \lvert g_r(x) - g(x) \rvert > \frac{a}{3}\right) + \\ & \hspace{4em}\lambda^n\left(\lvert f(x) - g(x) \rvert > \frac{a}{3}\right).
	\end{align*}
	
	Note that the second term on the right-hand side is 0 given our discussion on how the domain of a continuous functions consists of Lebesgue points. The third term is bounded above by $\frac{3\epsilon}{a}$ by Markov's inequality \hl{insert ref}. For the first term, note that $\lvert f_r - g_r \rvert \leq M\left(f-g\right)$ where $M$ is the maximal function operator from Definition \ref{def:maximalFunction}. This we can rewrite the inequality as 
	\begin{align*}
		\lambda^n\left(\limsup_{r\to 0}\lvert f_r(x) - f(x) \rvert > a\right) &\leq 	\lambda^n\left(M(f-g)(x) \rvert > \frac{a}{3}\right) + \frac{3\epsilon}{a} \\
		&\leq \frac{3\epsilon\left(5^n + 1\right)}{a}
	\end{align*}
	where the second inequality comes from Hardy-Littlewood's Theorem \ref{thm:hardyLittlewoodWeakInequality}. Since $\epsilon > 0$ can be arbitrarily small we have 
	\[
	\lambda^n\left(\limsup_{r\to 0}\lvert f_r(x) - f(x) \rvert > a\right) = 0.
	\]
	In particular, the result holds for $a = \frac{1}{m}$ for all $m \in \N$ and since
	\begin{align*}
		\lambda^n\left(\limsup_{r\to 0}\lvert f_r(x) - f(x) \rvert > 0\right) &= \lambda^n\left(\bigcup_{m\in\N}\limsup_{r\to 0}\lvert f_r(x) - f(x) \rvert > \frac{1}{m}\right) \\
		&\leq \sum_{m=1}^\infty \left(\limsup_{r\to 0}\lvert f_r(x) - f(x) \rvert > \frac{1}{m}\right) = 0
	\end{align*}
	our result for the integrable case follows by subadditivity and non-negativity of measures.
	
	The extension to the locally integrable case is routine; \hl{TODO}
\end{proof}
\hl{REVIEW}
\begin{thm}[Lebesgue Differentiation Theorem]
	\label{thm:lebesgueDifferentiation}
	For $f \in \mathcal{M}\left(\R^n , \borel \left(\R^n\right)\right)$  locally integrable, there exists a set $N \in \borel\left(\R^n\right)$ with $\lambda^n\left(N\right) = 0$ such at for every $x \notin N$, x is a Lebesgue point.
\end{thm}
\begin{proof}
	Fix $x \in \R^n$, let $\epsilon > 0 $ be arbitrary and note that by the density of rationals in the reals, there exists a rational $c_x \in \R$ such that $ \lvert f(x) - c_x \rvert < \frac{\epsilon}{2}$. Let $g^{x}(y) : = \lvert f(y) - c_x\rvert$ and notice that by Lemma \ref{lem:lebesgueDifferentiation}, there exists a null set $N_{c_x}$ such that $g_r^{x} \to g^x$ except on $N_{c_x}$. Since there are only countably many rationals, we can discard the subscript "$x$" in $c_x$ and so for any $x \notin \bigcup_{c \in \mathbb{Q}} N_c \in \borel\left(\R^n\right)$, we have that for small enough $r$
	\begin{align*}
		\frac{1}{\lambda^n\left(B_r(x)\right)} \lambda^n_y\left(\lvert f(y) - f(x) \rvert \indicate_{B_r(x)}\right) &\leq g^x(x) + g^x_r(x) \\
		&\leq \frac{\epsilon}{2} + \frac{\epsilon}{2} \\
		& = \epsilon
	\end{align*}
	where the first inequality is just the triangle inequality and linearity, and the second is given the density of rationals and our lemma. This completes the proof since $N := \bigcup_{c \in Q} N_c$ has measure zero by countable additivity.
\end{proof}

The second part of the fundamental theorem of calculus for integrable functions easily follows from this result. 

\begin{thm}
	For $f \in \Lp{1}{[a,b],\borel\left([a,b]\right), \lambda}$ and any $x \in [a,b]$, let 
	\[
		F(x):=\lambda\left(f\indicate_{[a,x]}\right).
	\]
	Then $F$ is differentiable at every Lebesgue point $x$ of $f$ with derivative $F^\prime(x) = f(x)$
\end{thm}
\begin{proof}
	Let $x$ be a Lebesgue point of $f$ and notice that 
	\begin{align*}
		\left\lvert \frac{F(x+r) - F(x)}{r} - f(x) \right\rvert &= \left\lvert \frac{1}{r}\lambda_y\left((f(y) -f(x))\indicate_{[x,x+r]}\right) \right\rvert \\
		&\leq \frac{1}{r}\lambda_y\left(\lvert f(y) - f(x) \rvert \indicate_{[x,x+r]} \right) \\
		& \leq 2\frac{1}{2r}\lambda_y\left(\lvert f(y) - f(x) \rvert \indicate_{[x-r,x+r]}	\right)
	\end{align*}
	where the equality comes from the definition of $F$ and the linearity of integration, the first inequality is the integral triangle inequality \ref{cor:triangleIneqLebIntL1}, and the last is monotonicity. The last term goes to zero by the Lebesgue Differentiation Theorem and the result follows. 
\end{proof}




\begin{example}
\label{exa:isi2007samplepsb4}Let $f:\mathbb{R}\rightarrow\mathbb{R}$
be a bounded continuous function. Define $g:[0,\infty)\rightarrow\R$
by, 
\[
g(x)=\int_{-x}^{x}(2xt+1)f(t)dt.
\]

Show that $g$ is differentiable on $(0,\infty)$ and find the derivative
of $g$.\hl{TODO}
\end{example}


\subsection{Non-linear change-of-variables}

\subsection{Integration with polar coordinates}
\hl{Folland / Schilling}

\subsection{Symmetry of partial derivatives}
\hl{Axler}

\subsection{Convolutions\label{subsec:convolutions}}

When measure spaces have a vector space structure such as $\R^{n}$,
we can define some interesting image measures which play an important
role in probability, fourier analysis, compressed sensing, signal
processing etc.
\begin{defn}
\label{def:convolution}Let $\mu,\nu$ be $\sigma-$finite Borel measures
on $\R^{n}$ (i.e. measures on $\borel\left(\R^{n}\right)$). The
\emph{convolution} $\mu\star\nu$ is is the image measure of the product
$\mu\otimes\nu$ under the map $T:\R^{n}\times\R^{n}\to\R$ given
by $T\left(x,y\right)=x+y$.

Since the product measure always exists and is unique, the convolution
always exists since $T$ is a continuous (and hence measurable) map.
Therefore for any bounded, continuous $f\in\mathcal{M}\left(\R^{n},\borel\left(\R^{n}\right)\right)$,
we have that
\begin{align*}
\lebInt{\mu\star\nu}f & =\lebInt{\mu\otimes\nu}{f\circ T}\\
 & =\mu^{x}\nu^{y}\left(f\left(x+y\right)\right)
\end{align*}
where we have used Corollary \ref{cor:changeOfVariables} and Fubini's
theorem. The definition implies that the convolution is symmetric
in that $\mu\star v=v\star\mu$. It is also possible to define the
convolution of a measure with a measurable function or of two measurable
functions on their own, as is illustrated in the following result.
\end{defn}

\begin{prop}
\label{prop:convolutionFunctions}Let $\mu,\nu$ be $\sigma-$finite
measures on $\borel\left(\R\right)$ such that $\mu$ is absolutely
continuous with respect to the Lebesgue measure with Radon-Nikodym
derivative $g\in\mathcal{M}^{+}\left(\R,\borel\left(\R\right)\right)$.
Then, for any bounded, continuous $f\in\mathcal{M}\left(\R,\borel\left(\R\right)\right)$
\[
\lebInt{\mu\star\nu}f=\lambda^{x}\nu^{y}\left(f\left(x\right)g\left(x-y\right)\right)
\]
 and so $\mu\star\nu$ is absolutely continuous with respect to the
Lebesgue measure with Radon-Nikodym derivative 
\[
g\star\nu:=\nu^{y}\left(g\left(x-y\right)\right).
\]
\end{prop}

\begin{proof}
Note that 
\begin{align*}
\lebInt{\mu\star\nu}f & =\mu\otimes\nu\left(f\left(x+y\right)\right)\\
 & =\nu^{y}\mu^{x}\left(f\left(x+y\right)\right)\\
 & =\nu^{y}\lambda^{x}\left(f\left(x+y\right)g\left(x\right)\right)\\
 & =\nu^{y}\lambda^{x}\left(f\left(x\right)g\left(x-y\right)\right)\\
 & =\lambda^{x}\left(f\left(x\right)\nu^{y}\left(g\left(x-y\right)\right)\right)
\end{align*}
where we have used the translation invariance of the Lebesgue measure
in the fourth equality.
\end{proof}
The Radon-Nikodym derivative $g\star\nu$ is called the convolution
of the function $g$ with the measure $\nu$. If $\nu$ too was absolutely
continuous and had some density $h$ then $\mu\star\nu$ would have
density $g\star h:=\lambda^{y}\left(g\left(x-y\right)h\left(y\right)\right)$.
Note that this definition can be expanded in the usual way to define
convolutions of functions that are not necessarily non-negative (as
Radon-Nikodym derivatives of positive measures are). We characterize
the symmetry and other properties of convolutions in this next result
\begin{prop}
\label{prop:propertiesConvolution}Let $\mu,\nu$ be $\sigma-$finite
measures on $\borel\left(\R^{n}\right)$ and let $f,g,h:\R^{n}\to\R$
be measurable functions. Then

\begin{enumerate}[label=(\roman*),leftmargin=.1\linewidth,rightmargin=.4\linewidth]
\item The convolution of two functions is symmetric in that $$ f \star g = g \star f $$. 
\item The convolution of two measures (or of a measure and a function, or of two functions) is bilinear in that 
$$ (\alpha f + \beta g) \star h = \alpha f \star h + \beta h \star h $$ with a similar result for the second argument.
\end{enumerate}
\end{prop}

\begin{proof}
For $\left(i\right)$, note that $f\star g=\lambda_{n}^{y}\left(f\left(x-y\right)g\left(y\right)\right)=\lambda^{y}\left(f\left(y\right)g\left(x-y\right)\right)=g\star f$
by using $T\left(y\right)=x-y$ \hl{TODO AFTER CHANGE OF VARIABLES}
\end{proof}
\begin{prop}
\label{prop:derivativeConvolution}
\end{prop}

\subsection{Laplace's method}

\begin{thm}[Laplace's method]
\label{thm:laplaceMethod}

\end{thm}



\section{Kernels and disintegration}

So far we have looked at constructing measures on product spaces by
taking individual measures and combining them in some way to create
a measure on the product space. Often we are interested in the reverse
process; that is, we are interested in taking a measure on a product
space and \emph{disintegrating }it into measures on the individual
spaces. It turns out that this process is more complicated than the
process we have seen for constructing product measures. As discussed
earlier, in the context of probability theory, the process of constructing
product measures corresponds to the process of constructing independent
random variables. In contrast, the process of disintegrating a measure
on a product space into component measures corresponds to the process
of computing conditional distributions. If you have some familiarity
with probability, you would know that conditioning is more complicated
than independence l

The central mathematical objects that underpin the theory of disintegration
are kernels. Kernels are important as they wear two hats: on one hand,
they act as measures, on the other they act as measurable functions.
\begin{defn}
\label{def:kernel}Let $\left(\X,\F\right)$ and $\left(\mathcal{Y},\mathcal{G}\right)$
be measure spaces A kernel $\kappa$ from $\left(\X,\F\right)$ to
$\left(\mathcal{Y},\mathcal{G}\right)$ is a function $\kappa:\mathcal{G}\times\X\to\left[0,\infty\right]$
such that for any fixed $G_{0}\in\mathcal{G}$ , the map
\[
x\to\kappa\left(G_{0},x\right)
\]
is $\F/\borel\left(\R\right)$ measurable and for every fixed $x_{0}\in\X$
, the map 
\[
G\to\kappa\left(G,x_{0}\right)
\]
is a measure on $\mathcal{G}.$
\end{defn}

\begin{rem*}
It is sometimes more intuitive to think of a kernel from $\left(\X,\F\right)$
to $\left(\mathcal{Y},\mathcal{G}\right)$ as a collection of measures
on $\mathcal{G}$ indexed by $\X$. So we can write
\[
K:=\left\{ \kappa_{x}:x\in\X\right\} 
\]
where each $\kappa_{x}$ is a measure on $\mathcal{G}$ and $x\to\kappa_{x}\left(G\right)$
is measurable for each $G\in\mathcal{G}$. When these measures are
restricted to the range $\left[0,1\right]$ the kernel is called a
\emph{Markov kernel.}
\end{rem*}
A generalization of the Tonelli theorem allows us to work with kernels.
\begin{thm}
\label{thm:tonelliKernels}Let $K=\left\{ \kappa_{x}:x\in\X\right\} $
be a kernel from $\left(\X,\F\right)$ and $\left(\mathcal{Y},\mathcal{G}\right)$
such that $\kappa_{x}$ is $\sigma-$finite for every $x\in\X$. Let
$\mu$be a $\sigma-$finite measure on $\F$. Then, for $f\in\mathcal{M}^{+}\left(\X\times\mathcal{Y},\mathcal{F}\otimes\mathcal{G}\right)$

\begin{enumerate}[label=(\roman*),leftmargin=.1\linewidth,rightmargin=.4\linewidth]
\item $y \to f\left(x,y\right)$ is $\mathcal{G}/\borel\left(\R\right)$ measurable
\item $x \to \kappa_x\left(f\left(x,y\right)\right)$ is $\mathcal{F}/\borel\left(\R\right)$ measurable
\item $\left(\mu \otimes K\right) := \mu^x\left(\kappa_x^y\left(f\left(x,y\right)\right)\right)$ is an integral on $\F\otimes\mathcal{G}$.
\end{enumerate}
\end{thm}

\begin{proof}
Note that $(i)$ is simply Lemma \ref{lem:partialFunctionMeasurability}.
For $\left(ii\right)$, we can treat
\end{proof}
\hl{Tatikonda}

\hl{Chang and Pollard, 1997}

\section{Extension to infinite product spaces}


