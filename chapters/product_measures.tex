
\chapter{Product measures}

In calculus, we learnt that the theory of integration readily extends
from from real valued functions on $\R$ to real valued functions
on the Euclidean space $\R^{n}.$ The extension is usually motivated
geometrically by studying the volume under the surface of a sufficiently
smooth function $f:\R^{2}\to\R$. The idea is that if a function is
sufficiently well behaved, then one can recover the volume under the
surface by looking at the areas under various ``slices'' of the function
and then summing up those areas. Importantly, under the requisite
smoothness conditions, the ``slices'' could have been made horizontally
or vertically, and we would get the same result. This intuition leads
to the Fubini theorem for multiple integration in the Riemann setting:
\[
\int_{a}^{b}\int_{c}^{d}f(x,y)dydx=\int_{c}^{d}\int_{a}^{b}f(x,y)dxdy.
\]
The main goal of this chapter is to recover this result for the Lebesgue
integral. While important in its own right for the study of analysis
on Euclidean spaces, in the context of probability theory, this result
takes a far more important role. In particular, the ability to write
a multiple integral as an iterated integral corresponds directly with
the ability to factor the joint distribution of random variables into
the marginal distribution; that is, it underpins the theory of \emph{independent
}random variables. Independence, and the departures from independence,
constitute the central concepts of probabiliy theory.

\section{Product measures on finite product spaces}

\subsection{Iterated integrals on non-negative measurable function}

Let $\left(\X,\F,\mu\right)$ and $\left(\mathcal{Y},\mathcal{G},\nu\right)$
be measure spaces. Our interest is in defining measurable functions
on the product space $\X\times\mathcal{Y}.$ The principle hurdle
that is immediately apparent here is that the product of the $\sigma-$algebras
$\F\times\mathcal{G}:=\left\{ F\times G\mid F\in\F,G\in\mathcal{G}\right\} $
is not necessarily a $\sigma-$algebra. \hl{ADD CE}. It is easy to
see, however, that it is a $\pi-$system (this fact will turn out
to be important!). Any $\sigma-$algebra we use should certainly \emph{contain
}$\mathcal{F}\times\mathcal{G}$ and so the canonical choice is given
by $\mathcal{F}\otimes\mathcal{G:=\sigma\left(F\times G\right)}$.
A series of natural questions follow if we want iterated integrals
to make sense. In particular, for a measurable map $f\in\mathcal{M}^{+}\left(\X\times\mathcal{Y},\mathcal{F}\otimes\mathcal{G}\right),$we
want to know if the projections $x\to f(x,y)$ and $y\to\mu^{x}\left(f\left(x,y\right)\right)$
are $\mathcal{F}$ and $\mathcal{G}$ measurable, respectively. Thankfully,
this turns out to be the case. We will use a long lost result from
Chapter 2: the \hyperref[thm:piLambdaThmFunctions]{$\pi-\lambda$ theorem for functions}.
Go over the hypotheses of this theorem before reading the following
results.
\begin{lem}
\label{lem:partialFunctionMeasurability}For every $f\in\mathcal{M}^{+}\left(\X\times\mathcal{Y},\mathcal{F}\otimes\mathcal{G}\right)$,
the maps $x\to f(x,y)$ and $y\to f(x,y)$ are $\mathcal{F}/\borel\left(\R\right)$
and $\mathcal{G}/\borel\left(\R\right)$ measurable for every $y\in\mathcal{Y}$
and $x\in\X$, respectively.
\end{lem}

\begin{proof}
First we note, due to the fact that $\left(A_{1}\times B_{1}\right)\cap\left(A_{2}\times B_{2}\right)=\left(A_{1}\cap A_{2}\right)\times\left(B_{1}\times B_{2}\right),$that
$\mathcal{\mathcal{E:=}F}\times\mathcal{G}$ is a $\pi-$system. Next,
we claim that the space
\[
\mathcal{H:=}\left\{ f\in\mathcal{M_{\text{bdd}}}\left(\X\times\mathcal{Y},\mathcal{F}\otimes\mathcal{G}\right)\mid\forall y\in\mathcal{Y}:x\to f\left(x,y\right)\text{is }\mathcal{G}/\borel\left(\R\right)\text{ measurable }\right\} 
\]
is a $\lambda-$space of functions. Note that $\indicate_{\X\times\mathcal{Y}}$is
constant and bounded (and evidently partially measurable in our sense).
Further, our space $\mathcal{H}$ is a vector space, since linear
combinations of bounded measurable functions is bounded, and the partial
measurability condition is also preserved under linear combinations
(both results of Proposition \ref{prop:binaryOperationsMeasurableFunctions}).
Finally, $\mathcal{H}$ is closed under monotone limits (if they exist)
since measurability (resp. partial measurability) is preserved under
limits.

Now, observe that $\left\{ \indicate_{A}\mid A\in\mathcal{E}\right\} \subseteq\mathcal{H}$,
since $\indicate_{F\times G}=\indicate_{F}\indicate_{G}$which is
clearly bounded, measurable, and partially measurable in our sense.
Thus applying the $\pi-\lambda$ theorem as discussed, we have that

\[
\mathcal{M_{\text{bdd}}}\left(\X\times\mathcal{Y},\mathcal{F}\otimes\mathcal{G}\right)\subseteq\mathcal{H}
\]
which completes the proof for bounded functions.

Now we can simply take a function $f\in\mathcal{M}^{+}\left(\X\times\mathcal{Y},\mathcal{F}\otimes\mathcal{G}\right)$
and construct the bounded monotone sequence $f_{n}:=\min\left\{ f,n\right\} \in\mathcal{H}$
by our results. We complete the proof by taking limits and noting
that partial measurability is preserved. The same argument, of course,
holds for $y\to f\left(x,y\right)$
\end{proof}
Virtually the same argument shows that $y\to\mu^{x}\left(f\left(x,y\right)\right)$
is measurable (although we need this previous lemma to show that this
function is indeed well-defined). We include this result here for
completeness
\begin{lem}
\label{lem:partialIntegralMeasurability}Let $\left(\X,\F,\mu\right)$
and $\left(\mathcal{Y},\mathcal{G},\nu\right)$ be measure spaces.
For every function $f\in\mathcal{M}^{+}\left(\X\times\mathcal{Y},\mathcal{F}\otimes\mathcal{G}\right)$,
the maps $x\to\nu^{y}\left(f\left(x,y\right)\right)$ and $y\to\mu^{x}\left(f\left(x,y\right)\right)$
are $\mathcal{F}/\borel\left(\R\right)$ and $\mathcal{G}/\borel\left(\R\right)$
measurable respectively.
\end{lem}

\begin{proof}
Let $\mathcal{E}$ be as before and this time let
\[
\mathcal{H:=}\left\{ f\in\mathcal{M_{\text{bdd}}}\left(\X\times\mathcal{Y},\mathcal{F}\otimes\mathcal{G}\right)\mid y\to\mu^{x}f\left(x,y\right)\text{is }\mathcal{G}/\borel\left(\R\right)\text{ measurable }\right\} ,
\]
noting that $\indicate_{\X\times\mathcal{Y}}\in\mathcal{H}$ since
$\mu^{x}\left(\indicate_{\X\times\mathcal{Y}}\right)=\indicate_{\mathcal{Y}}\mu\left(\X\right)$
which is $\mathcal{G}$ measurable. $\mathcal{H}$ is a vector space
closed under monotone limits with the same arguments as before along
with the linearity and monotone convergence properties of $\mu$.
Applying the $\pi-\lambda$ theorem for functions, we have that 
\[
\mathcal{M_{\text{bdd}}}\left(\X\times\mathcal{Y},\mathcal{F}\otimes\mathcal{G}\right)\subseteq\mathcal{H}.
\]
Finally, take some $f\in\mathcal{M}^{+}\left(\X\times\mathcal{Y},\mathcal{F}\otimes\mathcal{G}\right)$
and construct $f_{n}:=\min\left\{ f,n\right\} \in\mathcal{H}$ and
note that $y\to\mu^{x}\left(f_{n}\left(x,y\right)\right)$ is measurable
and by monotone convergence $y\to\mu^{x}\left(f\left(x,y\right)\right)$
is measurable completing the proof.
\end{proof}
%
With the technicalities out of the way, we can show that iterated
integrals give the same result under certain conditions. This is a
generalization of the result we saw all the way back in Lemma \ref{lem:TonelliForSeries}.
\begin{thm}[Tonelli]
\label{thm:tonelli}Let $\left(\X,\F,\mu\right)$ and $\left(\mathcal{Y},\mathcal{G},\nu\right)$
be measure spaces. Then, for any $f\in\mathcal{M}^{+}\left(\X\times\mathcal{Y},\mathcal{F}\otimes\mathcal{G}\right)$,
the functions 
\[
\gamma_{1}\left(f\right):=\nu^{y}\mu^{x}\left(f\right)
\]
and
\[
\gamma_{2}\left(f\right):\mu^{x}\nu^{y}\left(f\right)
\]
are integrals on $\mathcal{M}^{+}\left(\X\times\mathcal{Y},\mathcal{F}\otimes\mathcal{G}\right).$
Moreover, if $\mu$and $\nu$ are $\sigma-$finite, $\gamma_{1}\left(f\right)=\gamma_{2}\left(f\right)$
for every $f\in\mathcal{M}^{+}\left(\X\times\mathcal{Y},\mathcal{F}\otimes\mathcal{G}\right).$
\end{thm}

\begin{proof}
We show that $\gamma_{1}$ is an integral; the argument for $\gamma_{2}$
is the analagous. First, observe that $\gamma_{1}\left(0\right)=\nu^{y}\mu^{x}\left(0\right)=\nu^{y}\left(0\right)=0$,
given that $\nu$ and $\mu$ are integrals. Second, note that for
$\alpha,\beta\geq0$, and $f,g\in\mathcal{M}^{+}\left(\X\times\mathcal{Y},\mathcal{F}\otimes\mathcal{G}\right)$
\begin{align*}
\gamma_{1}\left(\alpha f+\beta g\right) & =\nu^{y}\mu^{x}\left(\alpha f+\beta g\right)\\
 & =\nu^{y}\left(\alpha\mu^{x}f+\beta\mu^{x}g\right)\\
 & =\alpha\nu^{y}\mu^{x}\left(f\right)+\beta\nu^{y}\mu^{x}\left(g\right)\\
 & =\alpha\gamma_{1}\left(f\right)+\beta\gamma_{1}\left(g\right).
\end{align*}
Finally, observe that for $f_{n}\in\mathcal{M}^{+}\left(\X\times\mathcal{Y},\mathcal{F}\otimes\mathcal{G}\right)$
such that $f_{n}\leq f_{n+1}$ and $f_{n}\to f$, we have
\begin{align*}
\lim_{n\to\infty}\gamma_{1}\left(f_{n}\right) & =\lim_{n\to\infty}\nu^{y}\mu^{x}\left(f_{n}\right)\\
 & =\nu^{y}\mu^{x}\left(\lim_{n\to\infty}f_{n}\right)\\
 & =\gamma_{1}\left(f\right)
\end{align*}
by applying monotone convergence twice.

Therefore, by Theorem \ref{thm:integralMeasureEquivalence}, $\gamma_{1}$
and $\gamma_{2}$ are integrals on $\mathcal{M}^{+}\left(\X\times\mathcal{Y},\mathcal{F}\otimes\mathcal{G}\right)$
with respect to measures defined by integrating indicator functions
in $\mathcal{F\otimes\mathcal{G}}.$ If $\mu$ and $\nu$ are $\sigma-$finite,
the two integrals can be shown to be equal by showing that the corresponding
measures are equal on a generating $\pi-$system that can approximate
the full space (courtesy of our \hyperref[thm:uniquenessMeasures]{uniqueness theorem}).
Of course, since $\mathcal{F\times\mathcal{G}}$ is a $\pi-$system,
for $F\in\F,G\in\mathcal{G}$
\begin{align*}
\gamma_{1}\left(\indicate_{F\times G}\right) & =\nu^{y}\mu^{x}\left(\indicate_{F\times G}\left(x,y\right)\right)\\
 & =\nu^{y}\mu^{x}\left(\indicate_{F}\left(x\right)\indicate_{G}\left(y\right)\right)\\
 & =\nu\left(G\right)\mu\left(F\right)\\
 & =\mu^{x}\nu^{y}\left(\indicate_{F}\left(x\right)\indicate_{G}\left(y\right)\right)\\
 & =\gamma_{2}\left(\indicate_{F\times G}\right)
\end{align*}
completing the proof. Since our measures $\mu,\nu$ are $\sigma-$finite,
we know that there exist sets $E_{i}\in\mathcal{F\times\mathcal{G}}$
such that $\bigcup E_{i}=\X\times\mathcal{Y}.$ This completes the
proof.
\end{proof}
Note that the $\sigma-$finiteness condition is actually necessary
for the uniqueness of the integrals, as the following example illustrates.
\begin{example}
\label{exa:tonelliFailNonSigmaFinite}Let $\left(\left[0,1\right],\borel\left(\left[0,1\right]\right),\lambda\right)$
and $\left(\left[0,1\right],2^{\left[0,1\right]},\mu_{0}\right)$
where $\lambda$ and $\mu_{0}$ are the Lebesgue and counting measures,
respectively. Now consider the product space $ $
\end{example}


\subsection{Product measures}
\begin{defn}
\label{def:productMeasure}Let $\measurespace$ and $\left(\mathcal{Y},\mathcal{G},\nu\right)$
be measure spaces and let $\left(\X\times\mathcal{Y},\F\otimes\mathcal{G}\right)$
be the product measurable space. Then a measure $\mu\otimes\nu:\F\otimes\mathcal{G}\to\left[0,\infty\right]$
is called a \emph{product measure }if
\[
\mu\otimes\nu\left(F\times G\right)=\mu\left(F\right)\nu\left(G\right)
\]
for all $F\in\F,G\in\mathcal{G}.$
\end{defn}

Of course, the way we have set this up, Theorem \ref{thm:tonelli}(Tonelli)
guarantees existence of this measure, since measures and integrals
are equivalent. Uniqueness, when $\mu$ and $\nu$ are $\sigma-$finite,
also follows from the uniqueness theorem, as outlined in the proof
of Tonelli.

\hl{Prove that $\sigma(\left(f,g\right)) = \sigma(f,g)$ }

\subsection{Convolutions}

\hl{SEE SCHiLLING THEOREM 13.10 FOR ALTERNATIVE CONSTRUCTION OF PRODUCT SIGMA ALGEBRAS}

\section{The Lebesgue measure on $\protect\R^{n}$}

\hl{FUBINI IMPLIES YOUNGS THEOREM ON SYMMETRY}

\section{Kernels}

\hl{Refer to Tatikonda notes}

\section{Extension to infinite product spaces}

\hl{ASH or KLENKE?}

\section{Disintegration}
