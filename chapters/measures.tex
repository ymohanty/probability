
\chapter{Measures\label{chap:measures}}

\section{Why is measurement hard?}

On the real line $\mathds{R}$, we may want our measure to satisfy
some properties that are consistent with our intuitive notion of ``length''.
Formally, we want a function
\[
\lambda:2^{\mathds{R}}\longrightarrow\left[0,\infty\right]
\]
that satisfies
\begin{enumerate}
\item $\lambda\left(\emptyset\right)=0$
\item $\lambda\left(\left[a,b\right]\right)=b-a$ for $a\leq b\in\mathds{R}$
\item Countable additivity: For a countable collection of pairwise-disjoint
sets $\left\{ A_{i}\right\} _{i\in\mathbb{N}}\subseteq\mathds{R}$
\begin{equation}
\lambda\left(\bigcup_{i\in\mathbb{N}}A_{i}\right)=\sum_{i\in\mathbb{N}}\lambda\left(A_{i}\right)\label{eq:countableAdditivity}
\end{equation}
\item Translation invariance: $\lambda\left(A+a\right)=\lambda\left(A\right)$
for any $a\in\mathds{R}$ where $A+a:=\left\{ \alpha+a\mid\alpha\in A\right\} $.
\end{enumerate}
Quite counterintuitively, it turns out that no such function exists!
To prove this assertion, we need to construct some special kinds of
sets that only exist if we assume the Axiom of Choice.
\begin{example}
\label{exa:vitaliSet} Define an equivalence relation $\sim$ on $\left[0,1\right]$
such that
\[
x\sim y\Leftrightarrow x-y\in\mathbb{Q}.
\]
Note that there are uncountably many classes in such a construction
as the equivalence class for any given irrational number can contain
at most countably many other irrational numbers. For example,
\[
\left[\frac{\pi}{4}\right]=\left\{ \frac{\pi}{4}+q\mod1\mid q\in\mathbb{Q}\right\} .
\]
Thus, using the Axiom of Choice, we can construct a set $E\subseteq\left[0,1\right]$
such that $E$ consists of exactly one ``representative'' from each
equivalence class. Next, we can define
\[
E_{q}:=\left\{ x+q\mod1\mid x\in E\right\} 
\]
so that $\left\{ E_{q}\right\} _{q\in\mathbb{Q}}$ is a partition
of $\left[0,1\right]$. To see that the sets are disjoint, suppose
for contradiction that for any distinct $q,\tilde{q}\in\mathbb{Q}\cap\left[0,1\right]$,
$E_{q}\cap E_{\tilde{q}}\neq\emptyset$. If $x\in E_{q}\cap E_{\tilde{q}}$,
then $x-q\in E$ and $x-\tilde{q}\in E$. But they clearly belong
to the same equivalence class and this is a contradiction given our
construction of $E.$ To see that the union of these sets is $\left[0,1\right]$,
consider an arbitrary $y\in\left[0,1\right]$ and observe that since
our equivalence relation $\sim$ partitions $\left[0,1\right]$, $y\in\left[x\right]$
for some $x\in E$. Then $q^{*}=y-x\in\mathbb{Q}$ and so $y=x+q^{*}\mod1\in E_{q^{*}}$.
Thus we have that
\[
\left[0,1\right]\subseteq\bigcup_{q\in\mathbb{Q}}E_{q}.
\]
Since the reverse inclusion follows by the definition of $E_{q}$,
we have that $\left\{ E_{q}\right\} _{q\in\mathbb{Q}}$ is a partition
of $\left[0,1\right]$.
\end{example}

\begin{prop}
\label{prop:vitalitSetNotMeasurable}There exists no function $\lambda:2^{\mathds{R}}\longrightarrow\left[0,\infty\right]$
that satisfies properties (1)-(4) described above
\end{prop}

\begin{proof}
Suppose, for contradiction, that such a function $\lambda$ exists.
We can define the collection of sets $\left\{ E\right\} _{q\in\mathbb{Q}}$
as in Example \ref{exa:vitaliSet} and observe that
\begin{align*}
1=\lambda\left(\left[0,1\right]\right) & =\lambda\left[\bigcup_{q\in\mathbb{Q}}E_{q}\right]\\
 & =\sum_{q\in\mathbb{Q}}\lambda\left[E_{q}\right]\\
 & =\sum_{q\in\mathbb{Q}}c
\end{align*}
where the first equality follows from property (2), the second equality
follows from the fact that $\left\{ E\right\} _{q\in\mathbb{Q}}$
is a partition of $\left[0,1\right]$, the third equality is due to
property (3). The last equality follows as a consequence of translation
invariance (property (4)). Since $c\in\left[0,1\right]$
\[
\sum_{q\in\mathbb{Q}}c=0\text{ or }\infty\neq1
\]
which is a contradiction. Thus no such function $\lambda$ exists.
\end{proof}
This particular example of a \emph{non-measurable }set is called a
\emph{Vitali set. }While we used the interval $\left[0,1\right]$
to construct such a set, it turns out that this contruction can be
extended to any set of positive length in the Lebesgue sense.

\section{Constructing measures on $\sigma-$algebras}

The key issue with our previous definition of a measure on $\mathds{R}$
is that one cannot have a set-valued function that both has our four
desired properties \emph{and }is defined on all subsets of the real
line. As a convention, the canonical construction of a measure retains
the desired properties in exchange for restricting the class of subsets
on which the measure is defined. These subsets are called $measurable$
and the standard construction of the Lebesgue measure leads to the
class of measurable subsets on the real line to have a special structure
of a \emph{$\sigma-$algebra. }Before we define this structure it
might be worthwhile looking at various types of structures a class
of sets could have

\subsection{Structures of sets}

In the rest of this chapter, we assume that $\left(\mathcal{X},\tau\right)$
is an abstract topological space.
\begin{defn}
\label{def:ring}Let $\mathcal{F}\subseteq2^{\mathcal{X}}$. We call
$\mathcal{F}$ a \emph{ring }if

\begin{enumerate}[label=(\roman*),leftmargin=.1\linewidth,rightmargin=.4\linewidth]
	\item $\emptyset \in \mathcal{F}$
	\item $A,B \in \mathcal{F} \Rightarrow A\cup B \in \mathcal{F}$
	\item $A,B \in \mathcal{F} \Rightarrow A\setminus B \in \mathcal{F}$.
\end{enumerate}
\end{defn}

\noindent Note that the above definition implies that $A\cap B=A\setminus\left(A\setminus B\right)\in\mathcal{F}$.
\begin{defn}
\label{def:algebra}Let $\mathcal{F}\subseteq2^{\mathcal{X}}$. We
call $\mathcal{F}$ an \emph{algebra }if

\begin{enumerate}[label=(\roman*),leftmargin=.1\linewidth,rightmargin=.4\linewidth]
	\item $\mathcal{F}$ is a ring
	\item $\mathcal{X} \in \mathcal{F}$.
\end{enumerate}
\end{defn}

\noindent For example, if we let $\mathcal{X}$ be an arbitrary infinite
set, the collection of all finite subsets of $\mathcal{X}$ forms
a ring but not an algebra.
\begin{defn}
\label{def:sigmaRing}Let $\mathcal{F}\subseteq2^{\mathcal{X}}$.
We call $\mathcal{F}$ a $\sigma$-ring if

\begin{enumerate}[label=(\roman*),leftmargin=.1\linewidth,rightmargin=.4\linewidth]
	\item $\mathcal{F}$ is a ring
	\item $\mathcal{F}$ is closed under countable unions.
\end{enumerate}
\end{defn}

\begin{defn}
\label{def:sigmaAlgebra}Let $\mathcal{F}\subseteq2^{\mathcal{X}}$.
We call $\mathcal{F}$ a $\sigma$-algebra if

\begin{enumerate}[label=(\roman*),leftmargin=.1\linewidth,rightmargin=.4\linewidth]
	\item $\mathcal{F}$ is an algebra.
	\item $\mathcal{F}$ is closed under countable unions.
\end{enumerate}
\end{defn}

\noindent Naturally, the power set $2^{\mathcal{X}}$ is a ring,
algebra, $\sigma$-ring, and $\sigma$-algebra all rolled into one.
\begin{rem*}
Algebras are sometimes referred to as \emph{fields} in the probability
literature.
\end{rem*}
As we said earlier, the notion of a $\sigma$-algebra is important
because the standard Lebesgue measurable sets form a $\sigma$-algebra
of subsets of $\mathds{R}$. However, the other structures we have
defined are also important; as we ``extend'' the notion of the length
of an interval on the real line to more complicated sets, we shall
first expand our class of measurable sets to a ring of sets.

\subsection{Lengths of intervals}

The mosts intuitive notion of a measure on $\mathds{R}$ arises from
the length of an interval. Thus, in our construction of the Lebesgue
measure, we start with the simplest class of sets which consists of
intervals in $\mathds{R}.$ Define $\mathcal{L}=\left\{ \left(a,b\right]\mid-\infty<a\leq b<\infty\right\} $
and let $\lambda_{1}:\mathcal{L}\longrightarrow\left[0,\infty\right]$
be given by $\lambda_{1}\left(\left(a,b\right]\right)=b-a$. It turns
out that our collection of half-open intervals in $\mathds{R}$ has
the structure of a \emph{semi-ring.}
\begin{defn}
\label{def:semiRing} Let $\mathcal{F}\subseteq2^{\mathcal{X}}$.
We call $\mathcal{F}$ a \emph{semi-ring }if

\begin{enumerate}[label=(\roman*),leftmargin=.1\linewidth,rightmargin=.4\linewidth]
	\item $\emptyset \in \mathcal{F}$.
	\item $A,B \in \mathcal{F} \Rightarrow A\cap B \in \mathcal{F}$.
	\item $A,B \in \mathcal{F} \Rightarrow \exists \left\{A_i\right\}_{i=1}^{n} \in \mathcal{F}$ such that $ A_i \cap A_j = \emptyset$ for $i \neq j$ and
	\[
		A \setminus B = \bigcup_{i=1}^{n}A_i 
	\]
\end{enumerate}
\end{defn}

\begin{prop}
\label{prop:intervalSigmaRing}$\mathcal{L}$ is a semi-ring.
\end{prop}

\begin{proof}
To see (i), note that $\emptyset=\left(a,a\right]\in\mathcal{L}$.
For (ii), note that for any intervals $A=\left(a_{1},b_{1}\right]$
and $B=\left(a_{2},b_{2}\right]$, \footnote{If $\max\left(a_{1},a_{2}\right)>\min\left(b_{1},b_{2}\right)$, then
$\left(\max\left(a_{1},a_{2}\right),\min\left(b_{1},b_{2}\right)\right]=\emptyset\in\mathcal{L}$}
\[
\left(a_{1},b_{1}\right]\cap\left(a_{2},b_{2}\right]=\left(\max\left(a_{1},a_{2}\right),\min\left(b_{1},b_{2}\right)\right]\in\mathcal{L}.
\]
To see (iii), we have to consider two possible cases. First, if $A,B$
are disjoint then $A\setminus B=A\in\mathcal{L}.$ If $A,B$ have
a non-trivial intersection, then
\begin{align*}
A\setminus B=A\cap B^{C} & =\left(a_{1},b_{1}\right]\cap\left\{ \left(-\infty,a_{2}\right]\cup\left(b_{2},\infty\right)\right\} \\
 & =\left(a_{1},b_{1}\right]\cap\left(-\infty,a_{2}\right]\bigcup\left(a_{1},b_{1}\right]\cap\left(b_{2},\infty\right)\\
 & =\left(a_{1},\min\left(b_{1},a_{2}\right)\right]\bigcup\left(\max\left(a_{1},b_{2}\right),b_{1}\right]
\end{align*}
where the components of the union expressed in the last equality are
in $\mathcal{L}$. This completes the proof.
\end{proof}
The fact that $\mathcal{L}$ is a semi-ring is important because there's
a relatively straightforward way to ``expand'' a semi-ring into
a ring.
\begin{thm}
\label{thm:expandSemiRing}Let $\mathcal{F}$ be a semi-ring and let
$\mathcal{B}$ be the set of all finite disjoint unions of sets in
$\mathcal{F}$. Then $\mathcal{B}$ is a ring.
\end{thm}

\begin{proof}
Property (i) in Definition \ref{def:ring} is trivially satisfied
thus we need to prove properties (ii) and (iii). Let $A,B\in\mathcal{B}$.
To prove property (iii), we first establish the weaker claim that
$A\cap B\in\mathcal{B}$. Observe that
\begin{align*}
A & =\bigcup_{i=1}^{n_{A}}A_{i},\ A_{i}\in\mathcal{F},A_{i}\cap A_{j}=\emptyset\mathrm{\ for\ }i\neq j,\\
B & =\bigcup_{i=1}^{n_{B}}B_{i},\ B_{i}\in\mathcal{F},B_{i}\cap B_{j}=\emptyset\mathrm{\ for\ }i\neq j,
\end{align*}
by the definition of $\mathcal{B}.$ Then
\begin{align*}
A\cap B & =\left(\bigcup_{i=1}^{n_{A}}A_{i}\right)\bigcap\left(\bigcup_{j=1}^{n_{B}}B_{j}\right)\\
 & =\bigcup_{i=1}^{n_{A}}\bigcup_{j=1}^{n_{B}}\left(A_{i}\cap B_{j}\right)
\end{align*}
where $\forall i,j:\ A_{i}\cap B_{j}\in\mathcal{F}$ as $\mathcal{F}$
is a semi-ring. Clearly, $A_{i}\cap B_{j}$ is disjoint from $A_{i^{\prime}}\cap B_{j^{\prime}}$,
thus proving the claim. Next, we establish property (iii) by noting
that
\begin{align*}
A\setminus B & =\left(\bigcup_{i=1}^{n_{A}}A_{i}\right)\setminus B\\
 & =\left(\bigcup_{i=1}^{n_{A}}A_{i}\right)\bigcap B^{C}\\
 & =\bigcup_{i=1}^{n_{A}}\left(A_{i}\cap B^{C}\right)\\
 & =\bigcup_{i=1}^{n_{A}}\left(A_{i}\cap\left(\bigcap_{j=1}^{n_{B}}B_{j}^{C}\right)\right)\\
 & =\bigcup_{i=1}^{n_{A}}\bigcap_{j=1}^{n_{B}}\left(A_{i}\cap B_{j}^{C}\right)\\
 & =\bigcup_{i=1}^{n_{A}}\bigcap_{j=1}^{n_{B}}A_{i}\setminus B_{j}
\end{align*}
where the $A_{i}\setminus B_{j}\in\mathcal{B}$ since $A_{i},B_{j}\in\mathcal{F}$.
By the closure under finite intersections property established earlier,
$E_{i}=\bigcap_{j=1}^{n_{B}}A_{i}\setminus B_{j}\in\mathcal{B}$ for
any $1\leq i\leq n_{A}$. Thus we can rewrite the chain of equalities
above as
\[
A\setminus B=\bigcup_{i=1}^{n_{A}}E_{i}
\]
where $E_{i}\cap E_{i^{\prime}}=\emptyset$ because $A_{i}\cap A_{i^{\prime}}=\emptyset.$
Since the finite disjoint union of elements of $\mathcal{B}$ is also
a finite disjoint union of elements of $\mathcal{F}$, our claim follows.
Finally, to establish property (ii), observe that
\[
A\cup B=\left(A\setminus B\right)\cup\left(A\cap B\right)\cup\left(B/A\right)
\]
which is a disjoint union of elements in $\mathcal{B}$ and so is
also in $\mathcal{B}$ by the same argument as earlier.
\end{proof}
\begin{cor}
Let $\mathcal{J}$ be the set of all finite disjoint unions of sets
in $\mathcal{L}$. Then $\mathcal{J}$ is a ring.
\end{cor}

\begin{proof}
By Proposition \propref{intervalSigmaRing}, $\mathcal{L}$ is a semi-ring.
The claim then follows by an application of Theorem \thmref{expandSemiRing}.
\end{proof}
Now we can extend our proto-measure $\lambda_{1}$ to a new proto-measure
$\lambda_{2}:\mathcal{J}\longrightarrow\left[0,\infty\right]$ as
follows:
\[
\lambda_{2}\left(A\right):=\begin{cases}
\lambda_{1}\left(A\right), & A\in\mathcal{L}\\
\sum_{i=1}^{n}\lambda_{1}\left(B_{i}\right), & A=\bigcup_{i=1}^{n}B_{i},\left\{ B_{i}\right\} _{i=1}^{n}\text{ are disjoint in }\mathcal{L}
\end{cases}
\]

\begin{prop}
\label{prop:ringMeasureFinitelyAdditive}$\lambda_{2}$ is fiinitely
additive on $\mathcal{J}$. That is, for any finite disjoint collection
of sets $\left\{ A_{i}\right\} _{i=1}^{n}\in\mathcal{J}$
\[
\lambda_{2}\left(\bigcup_{i=1}^{n}A_{i}\right)=\sum_{i=1}^{n}\lambda_{2}\left(A_{i}\right).
\]
\end{prop}

\begin{proof}
For clarity, we will prove finite additivity for two sets , since
the general case follows by induction. Let $A,B\in\mathcal{J}$ such
that $A\cap B=\emptyset$. By definition,
\begin{align*}
A & =\bigcup_{i=1}^{n_{A}}A_{i},\left\{ A_{i}\right\} _{i=1}^{n_{A}}\text{ are disjoint in }\mathcal{L}\\
B & =\bigcup_{i=1}^{n_{B}}B_{i},\left\{ B_{i}\right\} _{i=1}^{n_{B}}\text{ are disjoint in }\mathcal{L}
\end{align*}
and so we have that
\begin{align*}
\lambda_{2}\left(A\cup B\right) & =\lambda_{2}\left(\left(\bigcup_{i=1}^{n_{A}}A_{i}\right)\cup\left(\bigcup_{i=1}^{n_{B}}B_{i}\right)\right)\\
 & =\sum_{i=1}^{n_{A}}\lambda_{2}\left(A_{i}\right)+\sum_{i=1}^{n_{B}}\lambda_{2}\left(B_{i}\right)\\
 & =\lambda_{2}\left(A\right)+\lambda_{2}\left(B\right)
\end{align*}
where the second equality follows from associativity of addition along
with the fact that $\left(\bigcup_{i=1}^{n_{A}}A_{i}\right)\cup\left(\bigcup_{i=1}^{n_{B}}B_{i}\right)$
is a disjoint union of sets in $\mathcal{L}$.
\end{proof}

\subsection{Structures generated by a class of sets}

A key way to ``expand'' a particular class of sets into a larger
structure is to look at the structure \emph{generated }by the class
of sets. This idea can be formalized in the following definition,
which serves as particular example of this general concept of generation.
\begin{defn}
\label{def:ringGeneratedByClass}For any $\mathcal{A}\subseteq2^{\mathcal{X}},$
we refer to the intersection of all rings that contain $\mathcal{A}$
as the ring \emph{generated }by $\mathcal{A}$. Formally, we write
\[
\ring\left(\mathcal{A}\right)=\bigcap\left\{ \mathcal{R\subseteq}2^{\mathcal{X}}\text{ is a ring }\mid\mathcal{A\subseteq\mathcal{R}}\right\} .
\]
\end{defn}

\begin{prop}
\label{prop:ringGeneratedByClassIsRing}For any $\mathcal{A}\subseteq2^{\mathcal{X}},$
$\ring\left(\mathcal{A}\right)$ is a ring.
\end{prop}

\begin{proof}
First note that $\ring\left(\mathcal{A}\right)$ exists since $2^{\mathcal{X}}$
is a ring and so $\left\{ \mathcal{R\subseteq}2^{\mathcal{X}}\text{ is a ring }\mid\mathcal{A\subseteq\mathcal{R}}\right\} $
is non-empty. Next observe that $\emptyset\in\ring\left(\mathcal{A}\right)$
vacuously, so property (i) in Definition \defref{ring} is easily
satisfied. For property (ii), let $A,B\in\ring\left(\mathcal{A}\right)$
and observe that $A,B\in\mathcal{R}$ for every $\mathcal{R}\in\left\{ \mathcal{R\subseteq}2^{\mathcal{X}}\text{ is a ring }\mid\mathcal{A\subseteq\mathcal{R}}\right\} $.
Since $\mathcal{R}$ is a ring, $A\cup B\in\mathcal{R}$ for every
$\mathcal{R}$ and thus $A\cup B$ is in the intersection i.e. $\ring\left(\mathcal{A}\right)$.
A similar argument establishes property (iii) and thus we can conclude
that $\ring\left(\mathcal{A}\right)$ is a ring (as it should, given
its name).
\end{proof}
In our construction of the Lebesgue measure on $\mathds{R}$, we discovered
that $\mathcal{J}$, which is the set of all disjoint unions of half-open
intervals in $\mathds{R}$, is a ring. It turns out that we can make
a stronger statement using the language of generators developed here.
\begin{prop}
\label{prop:JisRingGeneratedByL}$\mathcal{J}=\ring\left(\mathcal{L}\right)$
\end{prop}

\begin{proof}
Let $A\in\mathcal{J}$ be arbitrary. Then we can write 
\[
A=\bigcup_{i=1}^{n_{A}}A_{i}
\]
where $A_{i}\in\mathcal{L}$ are pairwise disjoint. Let $\mathcal{R}$
be an arbitrary ring that contains $\mathcal{L}$ and observe that
since rings are closed under finite unions, $A\in\mathcal{R}.$ Since
$\mathcal{R}$ was arbitrary, $A$ is contained by every ring that
contains $\mathcal{L}$ and is thus contained in the intersection
of all such rings i.e. $\ring\left(\mathcal{L}\right).$ This proves
that $\mathcal{J}\subseteq\ring\left(\mathcal{L}\right).$

To see reverse inclusion, recall that $\mathcal{J}$ is a ring that
contains $\mathcal{L}$, and so the intersection of all rings that
contain $\mathcal{L}$ is certaintly contained in $\mathcal{J}$.
This completes the proof.
\end{proof}
In measure theory, the most important structure on sets is the $\sigma$-algebra,
and the $\sigma$-algebra generated by a class of sets $\mathcal{A}$,
defined analagously to Definition \ref{def:ringGeneratedByClass}
about rings and denoted as $\sigma\left(\mathcal{A}\right)$, plays
in an important role in this theory. Using a similar argument as the
one shown earlier, one can conclude that $\sigma\left(\mathcal{A}\right)$
is indeed a $\sigma$-algebra. In analysis and probability theory,
mathematicians are interested in $\sigma$-algebras generated by a
special class of sets.
\begin{defn}
\label{def:borelSigma}The $\sigma$-algebra generated by the topology
$\tau$ on set $\mathcal{X}$ is called the \emph{Borel $\sigma$-algebra
}on $\mathcal{X}$ and is denoted $\mathscr{B}\left(\mathcal{X}\right)$.
\end{defn}

The Borel $\sigma$-algebra is interesting because it turns that it
is the $\sigma$-algebra generated by $\mathcal{L}$ is indeed $\mathscr{B}\left(\mathds{R}\right)$,
where $\mathds{R}$ has the usual topology. To prove this fact, we
need a little lemma from an introductory course on analysis and topology.
\begin{lem}
\label{lem:openSetDisjointUnionInterval} Any open set in the usual
topology of $\R$ can be written as a countable disjoint union of
open intervals in $\R$.
\end{lem}

\begin{proof}
Let $O$ be an open set in $\R$ and let $x\in O$ be arbitrary. Define
$I_{x}\subseteq O$ to be the largest open interval that contains
$x$ (that is, $I_{x}$ is the union of all open intervals in $O$
that contain $x$). Note that at least one such interval exists because
$O$ is open and so there exists some $\varepsilon>0$ such that $\left(x-\varepsilon,x+\varepsilon\right)\subseteq O.$
Now for any distinct $x,y\in O$, $I_{x}$ and $I_{y}$ are either
disjoint or equal since if they were neither, $I_{x}\cup I_{y}\subseteq O$
would be a larger interval that contains both $x$ and $y$. Let $\mathcal{I}$
denote the collection of all disjoint such intervals (that is, we
get $\mathcal{I}$ by discarding all the ``redundant'' intervals
in $\left\{ I_{x}\right\} _{x\in O}$). We can do this without invoking
the Axiom of Choice since there are only countably many intervals
in $\mathcal{I}$: every interval $I\in\mathcal{I}$ contains at least
one rational number because the rationals are a countably dense subset
of $\R$. Thus, since the intervals are disjoint, $\mathcal{I}$ can
have at most countably many intervals. Of course
\[
O=\bigcup_{I\in\mathcal{I}}I
\]
and so our claim follows.
\end{proof}
\begin{prop}
\label{prop:sigmaAlgebraGeneratedbyLisBorel}$\sigma\left(\mathcal{L}\right)=\borel\left(\R\right)$
\end{prop}

\begin{proof}
Let $O$ be an open set in $\R$. Then, by Lemma \ref{lem:openSetDisjointUnionInterval}
\begin{align*}
O & =\bigcup_{i=1}^{\infty}\left(a_{i},b_{i}\right)\\
 & =\bigcup_{i=1}^{\infty}\bigcup_{n=1}^{\infty}\left(a_{i},b_{i}-\frac{1}{n}\right]
\end{align*}
which is in $\sigma\left(\mathcal{L}\right)$ by closure under countable
unions (property (ii) in Definition \ref{def:sigmaAlgebra}). Therefore
the topology of $\R$ is in $\sigma\left(\mathcal{L}\right)$ which
implies that $\borel\left(\R\right)\subseteq\sigma\left(\mathcal{L}\right)$.
The

To see the reverse inclusion, observe that for any $\left(a,b\right]\in\mathcal{L}$,
we can write
\[
\left(a,b\right]=\left(a,b\right)\cup\left\{ b\right\} \in\borel\left(\R\right)
\]
since $\left\{ b\right\} $ is closed in $\R$ and closed sets are
the complements of open sets and thus contained in $\borel\left(\R\right)$.\footnote{$\sigma$-algebras on $\mathcal{X}$ are closed under complements
since they are closed under set-differences and contain $\mathcal{X}$.} Therefore $\mathcal{L\subseteq\borel\left(\R\right)}$ and so $\sigma\left(\mathcal{L}\right)\subseteq\borel\left(\R\right)$,
completing the proof.
\end{proof}
Now we are ready to prove that our proto-measure $\lambda_{2}$ is
actually a countably-additive pre-measure on $\ring\left(\mathcal{L}\right)$.
But first, we need a lemma about double sums!
\begin{lem}[Tonelli for series]
\label{lem:TonelliForSeries}Let $\left\{ x_{ij}\right\} _{i,j\in\N\times\N}$
be a sequence of non-negative (extended) real numbers. Then

\[
\sum_{i,j\in\N^{2}}x_{ij}=\sum_{i=1}^{\infty}\sum_{j=1}^{\infty}x_{ij}=\sum_{j=1}^{\infty}\sum_{i=1}^{\infty}x_{ij}.
\]
\end{lem}

\begin{proof}
We will prove the first equality since the second then follows by
symmetry. Let $F\subset\N^{2}$ be arbitrary and finite. Then, there
exists some $N\in\N$ such that $F\subseteq\left\{ 1,2\ldots,N\right\} ^{2}$
and so, by the non-negativity of $x_{ij}$
\[
\sum_{i,j\in F}x_{ij}\leq\sum_{i,j\in\left\{ 1,2\ldots,N\right\} ^{2}}x_{ij}=\sum_{i=1}^{N}\sum_{j=1}^{N}x_{ij}\leq\sum_{i=1}^{\infty}\sum_{j=1}^{\infty}x_{ij}.
\]
This inequality holds for any finite $F\subset\N^{2}$ and so it holds
for the supremum of all such finite sums. That is to say,

\[
\sup_{F\subset\N^{2}\mid F\text{ is finite}}\sum_{i,j\in F}x_{ij}\leq\sum_{i=1}^{\infty}\sum_{j=1}^{\infty}x_{ij}.
\]
But recall that for any $\left\{ a_{i}\right\} _{i\in\mathcal{I}}\in\left[0,\infty\right]$
where $\mathcal{I}$ is any index set
\[
\sum_{i\in\mathcal{I}}a_{i}:=\sup_{I\subset\mathcal{I}\mid I\text{ is finite}}\sum_{i\in I}a_{i},
\]
and so we have that
\[
\sum_{i,j\in\N^{2}}x_{ij}\leq\sum_{i=1}^{\infty}\sum_{j=1}^{\infty}x_{ij}.
\]

To derive the other inequality, observe that it is sufficient to prove
that
\[
\sum_{i,j\in\N^{2}}x_{ij}\geq\sum_{i=1}^{I}\sum_{j=1}^{\infty}x_{ij}
\]
for every $I\in\N$. Fix $I=I_{0}$ and note that
\[
\sum_{i=1}^{I_{0}}\sum_{j=1}^{\infty}x_{ij}=\sum_{i=1}^{I_{0}}\lim_{J\to\infty}\sum_{j=1}^{J}x_{ij}=\lim_{J\to\infty}\sum_{i=1}^{I_{0}}\sum_{j=1}^{J}x_{ij}.
\]
Thus to prove $\sum_{i,j\in\N^{2}}x_{ij}\geq\sum_{i=1}^{I_{0}}\sum_{j=1}^{\infty}x_{ij}$
we need to prove that 
\[
\sum_{i,j\in\N^{2}}x_{ij}\geq\sum_{i=1}^{I_{0}}\sum_{j=1}^{J}x_{ij}
\]
for every $J\in\N$. Fix $J=J_{0}$ and then observe that
\[
\sum_{i=1}^{I_{0}}\sum_{j=1}^{J_{0}}x_{ij}=\sum_{i,j\in\left\{ 1,2,\ldots,I_{0}\right\} \times\left\{ 1,2,\ldots,J_{0}\right\} }x_{ij}\leq\sum_{i,j\in\N^{2}}x_{ij}
\]
where the inequality follows due to non-negativity of $x_{ij}$. This
concludes the proof.
\end{proof}
\begin{rem*}
This lemma is a special case of \hyperref[thm:tonelli]{Tonelli's theorem},
a fundamental theorem that allows us to construct measures on Cartesian
products of measure spaces from the measures on those spaces themselves.
This theorem will be motivated and proved in Chapter \ref{chap:productMeasures}.
\end{rem*}
\begin{prop}
\label{prop:ringMeasureCountablyAdditive} $\lambda_{2}$ is a countably
additive pre-measure on $\ring\left(\mathcal{L}\right)$, that is
to say,

\begin{enumerate}[label=(\roman*),leftmargin=.1\linewidth,rightmargin=.4\linewidth]
	\item $\lambda_2\left(\emptyset\right) = 0$ 
	\item For disjoint $\left\{A_i\right\}_{i=1}^{\infty}\in \mathcal{J}$ such that $\bigcup_{i=1}^{\infty}A_i \in \mathcal{J}$
	\[
			\lambda_2\left(\bigcup_{i=1}^{\infty}A_i\right) = \sum_{i=1}^{\infty}\lambda_2\left(A_i\right).
	\]
\end{enumerate}
\end{prop}

\begin{proof}
Property (i) is inherited from $\lambda_{1}$. To see property (ii),
let $\left\{ A_{i}\right\} _{i=1}^{\infty}\in\mathcal{J}$ be disjoint
and write $A:=\bigcup_{i=1}^{\infty}A_{i}$ where $A\in\mathcal{J}$
by assumption. First, note that if $\lambda_{2}\left(A_{i}\right)=\infty$
for any $i\in\N$, then $\infty=\lambda_{2}\left(A_{i}\right)\leq\lambda_{2}\left(\bigcup_{i=1}^{\infty}A_{i}\right)=\infty$
where the inequality is due to the monotonicity\footnote{For any $A,B\in\ring\left(\mathcal{L}\right)\text{ such that }A\subseteq B,\lambda_{2}\left(B\right)=\lambda_{2}\left(A\right)+\lambda_{2}\left(B\setminus A\right)\geq\lambda_{2}\left(A\right)$}
of $\lambda_{2}$. Thus, in this case, the claim follows vacuously.
So, without loss of generality, we can assume that $\lambda_{2}\left(A_{i}\right)<\infty$
for every $i\in\N$. First, note that for any $n\in\N$,$\bigcup_{i=1}^{n}A_{i}\subseteq A$
and so, by the monotonictity and finite additivity of $\lambda_{2}$,
we have that
\[
\lambda_{2}\left(A\right)\geq\lambda_{2}\left(\bigcup_{i=1}^{n}A_{i}\right)=\sum_{i=1}^{n}\lambda_{2}\left(A_{i}\right)
\]
for every $n\in\N$. Taking limits, we have countable superadditivity:
\[
\lambda_{2}\left(A\right)\geq\sum_{i=1}^{\infty}\lambda_{2}\left(A_{i}\right).
\]
In order to deduce the reverse inequality, first suppose that both
$A$ and $\left\{ A_{i}\right\} $ are in $\mathcal{L}.$ Then, we
can write 
\[
A:=\left(a,b\right]
\]
and
\[
A_{i}=\left(a_{i},b_{i}\right]
\]
for each $i\in\N.$ Pick an arbitrary $0<\epsilon<b-a$ and observe
that 
\[
\left[a+\epsilon,b\right]\subseteq\bigcup_{i=1}^{\infty}\left(a_{i},b_{i}+\frac{\epsilon}{2^{i}}\right)
\]
and so by the Heine-Borel theorem, there exists some finite $K$ such
that 
\[
\left[a+\epsilon,b\right]\subseteq\bigcup_{k=1}^{K}\left(a_{i_{k}},b_{i_{k}}+\frac{\epsilon}{2^{i_{k}}}\right).
\]
By the finite additivity established in Proposition \ref{prop:ringMeasureFinitelyAdditive}
and monotonicity, we have that
\[
\underbrace{b-a}_{\lambda_{2}\left(A\right)}-\epsilon\leq\sum_{k=1}^{K}b_{i_{k}}+\frac{\epsilon}{2^{i_{k}}}-a_{i_{k}}\leq\underbrace{\sum_{i=1}^{\infty}\left(b_{i}-a_{i}\right)}_{\sum_{i\in\N}\lambda_{2}\left(A_{i}\right)}+\epsilon
\]
and since $\epsilon$ can be arbitrary small the claim follows.

Deducing the general case from the special one outlined above is straightforward.
If $A,\left\{ A_{i}\right\} \in\mathcal{J}$ then 
\[
A=\bigcup_{j=1}^{J}B_{j}
\]
where $\left\{ B_{j}\right\} \in\mathcal{L}$ are pairwise disjoint.
Similarly, 
\[
A_{i}=\bigcup_{k=1}^{n_{i}}C_{ik}
\]
where $\left\{ C_{ij}\right\} _{i\in\N,j\in\N}\in\mathcal{L}$ are
pairwise disjoint and $n_{i}\in\N$. Note that then
\[
\lambda_{2}\left(A\right)=\sum_{j=1}^{J}\lambda_{2}\left(B_{j}\right)\leq\sum_{i=1}^{\infty}\sum_{k=1}^{n_{i}}\lambda_{2}\left(C_{ik}\right)=\sum_{i=1}^{\infty}\lambda_{2}\left(A_{i}\right)
\]
where the first equality follow from the finite additivity of $\lambda_{2}$
on $\mathcal{J},$ the inequality by the fact that for any $j\in\left\{ 1,2,\ldots,J\right\} ,$
there exists a partition of the collection $\left\{ C_{ik}\right\} $
into subcollections $\left\{ C_{ik}^{j}\right\} _{1\leq j\leq J}$such
that 
\[
B_{j}=\bigcup_{i,k}C_{ik}^{j}
\]
and so the special case of our result on $\mathcal{L}$ applies (along
with an application of Lemma \ref{lem:TonelliForSeries}). The final
equality again follows by finite additivity. 
\end{proof}

\subsection{Outer measures}
\begin{defn}
\label{def:outerMeasure}A set valued function 
\[
\mu^{*}:2^{\mathcal{X}}\longrightarrow\left[0,\infty\right]
\]
is called an outer measure on $\mathcal{X}$ if

\begin{enumerate}[label=(\roman*),leftmargin=.1\linewidth,rightmargin=.4\linewidth]
	\item $ \mu^*\left(\emptyset\right) = 0$ 
	\item $A\subseteq B \in 2^\mathcal{X} \Longrightarrow \mu^*\left(A\right) \leq \mu^*\left(B\right) $
	\item For $\left\{A_i\right\}_{i=1}^{\infty}\in 2^\mathcal{X}$ 
	\[
			\mu^*\left(\bigcup_{i=1}^{\infty}A_i\right) \leq \sum_{i=1}^{\infty}\mu^*\left(A_i\right).
	\]
\end{enumerate}
\end{defn}

\begin{example}
\label{exa:canonicalOuterMeasure}Given a non-negative extended-real
valued function $\mu$ on a collection $\mathcal{A\subseteq}2^{\mathcal{X}}$
such that $\mu\left(\emptyset\right)=0$, define for any $E\subseteq\mathcal{X}$
\[
\mu^{*}\left(E\right):=\inf\left\{ \sum_{i=1}^{\infty}\mu\left(A_{i}\right)\mid A_{i}\in\mathcal{A},E\subseteq\bigcup_{i=1}^{\infty}A_{i}\right\} 
\]
\end{example}

Note that this function is defined on $2^{\mathcal{X}}$ since every
bounded below subset of the (extended) real numbers has an infimum.
Now we prove that the set-function descibed above is indeed an outer
measure.
\begin{prop}
\label{prop:canonicalOuterMeasureIsOuterMeasure}The function $\mu^{*}:2^{\mathcal{X}}\longrightarrow\left[0,\infty\right]$
defined in Example \ref{exa:canonicalOuterMeasure} is an outer measure
\end{prop}

\begin{proof}
For (i), observe that $\emptyset\in\mathcal{A}$ and so $\mu^{*}\left(\emptyset\right)=\mu\left(\emptyset\right)=0.$
Next, let $A\subseteq B\subseteq\mathcal{X}$ and observe that 
\[
\left\{ \sum_{i=1}^{\infty}\mu\left(A_{i}\right)\mid A_{i}\in\mathcal{A},B\subseteq\bigcup_{i=1}^{\infty}A_{i}\right\} \subseteq\left\{ \sum_{i=1}^{\infty}\mu\left(A_{i}\right)\mid A_{i}\in\mathcal{A},A\subseteq\bigcup_{i=1}^{\infty}A_{i}\right\} 
\]
and so 
\[
\mu^{*}\left(B\right)=\inf\left\{ \sum_{i=1}^{\infty}\mu\left(A_{i}\right)\mid A_{i}\in\mathcal{A},B\subseteq\bigcup_{i=1}^{\infty}A_{i}\right\} \geq\inf\left\{ \sum_{i=1}^{\infty}\mu\left(A_{i}\right)\mid A_{i}\in\mathcal{A},A\subseteq\bigcup_{i=1}^{\infty}A_{i}\right\} =\mu^{*}\left(A\right)
\]
which gives us (ii). For (iii), let $\left\{ E_{i}\right\} _{i=1}^{\infty}\in2^{\mathcal{X}}$
and assume that $\sum_{i=1}^{\infty}\mu^{*}\left(E_{i}\right)<\infty$
since otherwise the claim is trivial. Fix $\epsilon>0$ and choose
$A_{ij}\in\mathcal{A}$ such that $E_{i}\subseteq\bigcup_{j=1}^{\infty}A_{ij}$
and

\[
\mu^{*}\left(E_{i}\right)\leq\sum_{j=1}^{\infty}\mu(A_{ij})<\mu^{*}\left(E_{i}\right)+\frac{\epsilon}{2^{i}}
\]
for every $i\in\N$\footnote{This is possible due to the assumption that $\mu^{*}\left(E_{i}\right)<\infty$,
which implies that the set $\left\{ \sum_{j=1}^{\infty}\mu\left(A_{ij}\right)\mid A_{ij}\in\mathcal{A},E_{i}\subseteq\bigcup_{j=1}^{\infty}A_{ij}\right\} $
is non-empty. The definition of an infimum then implies that such
a cover $\left\{ A_{ij}\right\} $ exists.}. Observe that
\begin{align*}
E & :=\bigcup_{i=1}^{\infty}E_{i}\subseteq\bigcup_{i=1}^{\infty}\bigcup_{j=1}^{\infty}A_{ij}
\end{align*}
and so 
\[
\mu^{*}\left(E\right)\leq\sum_{i,j\in\N^{2}}\mu\left(A_{ij}\right)=\sum_{i=1}^{\infty}\sum_{j=1}^{\infty}\mu\left(A_{ij}\right)\leq\sum_{i=1}^{\infty}\mu^{*}(E_{i})+\epsilon
\]
where the equality follows by Lemma \ref{lem:TonelliForSeries} and
the second inequality is due to properties of the geometric series.
Since $\epsilon$ was arbitrary, the claim folllows.
\end{proof}
\begin{rem}
The outer measure described above is called the \emph{canonical }outer-measure
as it as by far the most useful type of outer measure in measure theory.
Given a space $\X$, a collection of subsets $\mathcal{A}\subseteq2^{\X}$,
and a countably additive pre-measure $\mu$ on $\mathcal{A}$, we
can call
\[
\mu^{*}\left(E\right):=\inf\left\{ \sum_{i=1}^{\infty}\mu\left(A_{i}\right)\mid A_{i}\in\mathcal{A},E\subseteq\bigcup_{i=1}^{\infty}A_{i}\right\} 
\]
the canonical outer measure generated by $\left(\mu,\mathcal{A}\right)$.
\end{rem}

\begin{prop}
\label{prop:restrictionOfOuterMeasure}Let $\mathcal{A}$, $\mu,$
and $\mu^{*}$ be defined as in Example \ref{exa:canonicalOuterMeasure}.
Then, for any $A\in\mathcal{A}$
\[
\mu^{*}\left(A\right)=\mu\left(A\right).
\]
\end{prop}

\begin{proof}
First, observe that $A$ is a cover for itself and that $\emptyset\in\mathcal{A}$
and so 
\[
\mu^{*}\left(A\right)=\inf\left\{ \sum_{i=1}^{\infty}\mu(A_{i})\mid A_{i}\in\mathcal{A},A\subseteq\bigcup_{i=1}^{\infty}A_{i}\right\} \leq\sum_{i=1}^{\infty}\mu(A_{i})
\]
where $A_{1}=A$ and $A_{i}=\emptyset$ for $i\neq1.$ Therefore,
\[
\mu^{*}\left(A\right)\leq\mu\left(A\right).
\]

To see the reverse inequality, let $\left\{ A_{i}\right\} _{i\in\N}\in\mathcal{A}$
be an arbitrary cover of $A.$ Define,
\[
B_{i}:=A\cap\left(A_{i}\setminus\bigcup_{j=1}^{i-1}A_{j}\right)
\]
and notice that the $\left\{ B_{i}\right\} $ is a pairwise disjoint
collections whose union is $A$ such that $B_{i}\subseteq A_{i}$
for every $i\in\N$. By countable additivity and monotonicity,

\[
\mu\left(A\right)=\sum_{i=1}^{\infty}\mu\left(B_{i}\right)\leq\sum_{i=1}^{\infty}\mu\left(A_{i}\right).
\]
Since $\left\{ A_{i}\right\} \subseteq\mathcal{A}$ is an arbitrary
cover of $A$ , we have that
\[
\mu\left(A\right)\leq\inf\left\{ \sum_{i=1}^{\infty}\mu\left(A_{i}\right)\mid A_{i}\in\mathcal{A},A\subseteq\bigcup_{i=1}^{\infty}A_{i}\right\} =\mu^{*}\left(A\right)
\]
which completes the proof.
\end{proof}
Now we are (finally!!) ready to extend our pre-measure to a bona-fide
measure on a $\sigma$-algebra, using the following theorem.
\begin{thm}[Caratheodory's Extension Theorem]
\label{thm:caratheodoryExtn}Let $\X$ be a set. Given a countably-additive
pre-measure $\mu$ on ring $\mathcal{A\subseteq}2^{\mathcal{X}}$
with canonical outer measure $\mu^{*}$ generated by $\left(\mu,\mathcal{A}\right)$,
define the collection 
\[
\mathcal{C}\left(\mu^{*}\right):=\left\{ A\subseteq\mathcal{X}\mathrm{\ such\ that\ }\mu^{*}\left(E\right)=\mu^{*}\left(A\cap E\right)+\mu^{*}\left(A^{C}\cap E\right)\forall E\in2^{\mathcal{X}}\right\} .
\]
Then

\begin{enumerate}[label=(\roman*),leftmargin=.1\linewidth,rightmargin=.4\linewidth]
	\item $ \mathcal{A}\subseteq \mathcal{C}$.
	\item $ \mathcal{C}\left(\mu^*\right) $ is a $\sigma$-algebra.
	\item $\left.\mu^*\right|_{\mathcal{C}}$ is a countably additive measure on $\mathcal{C}$. 
\end{enumerate}
\end{thm}

\begin{proof}
First we will show (i). Let $A\in\mathcal{A}$ be arbitrary. By the
countable subadditivity of $\mu^{*}$, we know that 
\[
\mu^{*}\left(E\right)=\mu^{*}\left(\left(A\cap E\right)\bigcup\left(A^{C}\cap E\right)\right)\leq\mu^{*}\left(A\cap E\right)+\mu^{*}\left(A^{C}\cap E\right)
\]
for every $E\subseteq\mathcal{X}$. To deduce the reverse inequality,
fix $E$ such that $\mu^{*}\left(E\right)<\infty$ because otherwise
the claim follows trivially. Pick an $\epsilon>0$ and find a cover
$\left\{ A_{i}\right\} _{i=1}^{\infty}\in\mathcal{A}$ of $E$ such
that
\[
\mu^{*}\left(E\right)\leq\sum_{i=1}^{\infty}\mu\left(A_{i}\right)<\mu^{*}\left(E\right)+\epsilon
\]
As in the proof of Proposition \ref{prop:canonicalOuterMeasureIsOuterMeasure},
this is possible because $\mu^{*}\left(E\right)<\infty$ and the definition
of an infimum. Next, observe that
\begin{align*}
E\cap A & \subseteq\bigcup_{i=1}^{\infty}(A_{i}\cap A),\\
E\cap A^{C} & \subseteq\bigcup_{i=1}^{\infty}(A_{i}\cap A^{C})
\end{align*}
and so 
\begin{align*}
\mu^{*}\left(E\cap A\right) & \leq\mu^{*}\left(\bigcup_{i=1}^{\infty}(A_{i}\cap A)\right)\leq\sum_{i=1}^{\infty}\mu^{*}\left(A_{i}\cap A\right)\\
\mu^{*}\left(E\cap A^{C}\right) & \leq\mu^{*}\left(\bigcup_{i=1}^{\infty}(A_{i}\cap A^{C})\right)\leq\sum_{i=1}^{\infty}\mu^{*}\left(A_{i}\cap A^{C}\right)
\end{align*}
where the first inequality follows due to monotonicity and the second
due to subadditivity. Together, these inequalities imply that
\begin{align*}
\mu^{*}\left(A\cap E\right)+\mu^{*}\left(A^{C}\cap E\right) & \leq\sum_{i=1}^{\infty}\mu^{*}\left(A_{i}\cap A\right)+\mu^{*}\left(A_{i}\cap A^{C}\right)\\
 & =\sum_{i=1}^{\infty}\mu\left(A_{i}\cap A\right)+\mu\left(A_{i}\cap A^{C}\right)\\
 & =\sum_{i=1}^{\infty}\mu\left(A_{i}\right)\\
 & <\mu^{*}\left(E\right)+\epsilon
\end{align*}
where the first equality is due to the fact that rings are closed
under intersections and set-differences along with Proposition \ref{prop:restrictionOfOuterMeasure}
and the second equality is due to the countable additivity of $\mu.$
Since $\epsilon$ and $E$ are arbitrary, we have that 
\[
\mu^{*}\left(A\cap E\right)+\mu^{*}\left(A^{C}\cap E\right)\leq\mu^{*}\left(E\right)
\]
for every $E\subseteq\mathcal{X},$ establishing that $\mathcal{A}\subseteq\mathcal{C}$.

Next we show (ii); that is, we prove $\mathcal{C}$ is a $\sigma-$algebra.
Recall Definition \ref{def:sigmaAlgebra} and notice that it is sufficient
to prove that (1) $\emptyset,\mathcal{X}\in\mathcal{C}$; (2) if $A\in\mathcal{C}$
then $A^{C}\in\mathcal{C}$; (3) if $\left\{ A_{i}\right\} _{i=1}^{\infty}\in\mathcal{C}$
then $\bigcup_{i=1}^{\infty}A_{i}\in\mathcal{C}.$ Note that $\emptyset,\mathcal{X}\in\mathcal{C}$
because, trivially,
\[
\mu^{*}\left(E\cap\mathcal{X}\right)+\mu^{*}\left(E\cap\emptyset\right)=\mu^{*}\left(E\right).
\]
Symmetry between $A$ and $A^{C}$ in the definition of $\mathcal{C}$
establishes (2). For (3), we first establish closure under finite
unions and bootstrap this weaker result to yield the stronger claim.
Let $A,B\in\mathcal{C}$ and let $E\subseteq\mathcal{X}$ be arbitrary.
Then

\begin{align*}
\mu^{*}\left(E\right) & =\mu^{*}\left(E\cap A\right)+\mu^{*}\left(E\cap A^{C}\right)\\
 & =\mu^{*}\left(\left(E\cap A\right)\cap B\right)+\mu^{*}\left(\left(E\cap A\right)\cap B^{C}\right)+\mu^{*}\left(E\cap A^{C}\right)\\
 & =\mu^{*}\left(E\cap A\cap B\right)+\mu^{*}\left(E\cap\left(A\cap B\right)^{C}\cap A\right)+\mu^{*}\left(E\cap\left(A\cap B\right)^{C}\cap A^{C}\right)\\
 & =\mu^{*}\left(E\cap A\cap B\right)+\mu^{*}\left(E\cap\left(A\cap B\right)^{C}\right)
\end{align*}
where the second equality is due to the definition of $\mathcal{C}$
and the fact that $B\in\mathcal{\mathcal{C}}$, the third equality
is due to the identities
\begin{align*}
\left(A\cap B\right)^{C}\cap A & =\left(A^{C}\cup B^{C}\right)\cap A=A\cap B^{C}\\
\left(A\cap B\right)^{C}\cap A^{C} & =\left(A^{C}\cup B^{C}\right)\cap A^{C}=A^{C},
\end{align*}
and the fourth equality follows from the definition of $\mathcal{C}$
and that $A\in\mathcal{C}$. This proves that for any $A,B\in\mathcal{C}$,
$A\cap B\in\mathcal{C}.$ Property (2) then implies that $A\cup B\in\mathcal{C}.$

To establish closure under countable unions, fix $E\subseteq\mathcal{X}$
and let $\left\{ A_{i}\right\} _{i=1}^{\infty}\in\mathcal{C}$ be
arbitrary with $B=\bigcup_{i\in\N}A_{i}$ and define
\[
B_{n}:=\bigcup_{i=1}^{n}A_{i}
\]
where $B_{n}\in\mathcal{C}$ by our result on closure under finite
unions. Without loss of generality, we can assume that the $\left\{ A_{i}\right\} $
are pairwise disjoint (since we could otherwise replace $A_{i}$ with
$C_{i}:=A_{i}\setminus\bigcup_{j=1}^{i-1}A_{j}$ which are disjoint
such that $\bigcup_{i=1}^{\infty}A_{i}=\bigcup_{i=1}^{\infty}C_{i}$).
Then, we have that
\begin{align*}
\mu^{*}\left(E\right) & =\mu^{*}\left(E\cap B_{n}^{C}\right)+\mu^{*}\left(E\cap B_{n}\right)\\
 & =\mu^{*}\left(E\cap B_{n}^{C}\right)+\mu^{*}\left(E\cap B_{n}\cap A_{n}\right)+\mu^{*}\left(E\cap B_{n}\cap A_{n}^{C}\right)\\
 & =\mu^{*}\left(E\cap B_{n}^{C}\right)+\mu^{*}\left(E\cap A_{n}\right)+\mu^{*}\left(E\cap B_{n-1}\right)
\end{align*}
where we used the fact that $A_{n}\in\mathcal{C}$ for the second
equality and the disjointness of $A_{i}$ for the third equality.
Observe that the equality $\mu^{*}\left(E\cap B_{n}\right)=\mu^{*}\left(E\cap A_{n}\right)+\mu^{*}\left(E\cap B_{n-1}\right)$
is a recurrence relation that can be expanded as
\[
\mu^{*}\left(E\cap B_{n}\right)=\sum_{i=1}^{n}\mu^{*}\left(E\cap A_{i}\right)
\]
and so 
\begin{align*}
\mu^{*}\left(E\right) & =\mu^{*}\left(E\cap B_{n}^{C}\right)+\sum_{i=1}^{n}\mu^{*}\left(E\cap A_{i}\right)\\
 & \geq\mu^{*}\left(E\cap B^{C}\right)+\sum_{i=1}^{n}\mu^{*}\left(E\cap A_{i}\right)
\end{align*}
for every $n\in\N$ where the inequality is due to the the fact that
$B^{C}\subseteq B_{n}^{C}$ and the monotonicity of outer measures.
After taking limits, we have that
\begin{align*}
\mu^{*}\left(E\right) & \geq\mu^{*}\left(E\cap B^{C}\right)+\sum_{i=1}^{\infty}\mu^{*}\left(E\cap A_{i}\right)\\
 & \geq\mu^{*}\left(E\cap B^{C}\right)+\mu^{*}\left(\bigcup_{i\in\N}\left(E\cap A_{i}\right)\right)\\
 & =\mu^{*}\left(E\cap B^{C}\right)+\mu^{*}\left(E\cap B\right)
\end{align*}
where the second inequality follows by countable subadditivity. Another
application of countable subadditivity yields
\[
\mu^{*}\left(E\right)\leq\mu^{*}\left(E\cap B^{C}\right)+\mu^{*}\left(E\cap B\right)
\]
and together the two inequalities establish that $B\in\mathcal{C},$
finishing the proof of (ii).

Finally, in order to show that $\left.\mu^{*}\right|_{\mathcal{C}}$
is indeed a countably additive measure on $\mathcal{C}$, let $\left\{ A\right\} _{i=1}^{\infty}\in\mathcal{C}$
be pairwise disjoint, and observe that for $B:=\bigcup_{i\in\N}A_{i}\in\mathcal{C}$
and any $E\subseteq\mathcal{X}$
\[
\mu^{*}\left(E\right)\geq\mu^{*}\left(E\cap B^{C}\right)+\sum_{i=1}^{\infty}\mu^{*}\left(E\cap A_{i}\right)
\]
due to our previous work. Letting $E=B$, we have 
\begin{align*}
\mu^{*}\left(B\right) & \geq\sum_{i=1}^{\infty}\mu^{*}\left(B\cap A_{i}\right)\\
 & =\sum_{i=1}^{\infty}\mu^{*}\left(A_{i}\right).
\end{align*}
Since the reverse inequality follows by the subadditivity of the outer
measure, our proof is complete.
\end{proof}
\begin{rem}
\label{rem:noRingReqd}Note that the proof of the facts that $\mathcal{C}\left(\mu^{*}\right)$
is a $\sigma$-algebra and $\mu^{*}|_{\mathcal{C}}$ is countably
additive do not depend on the fact $\mathcal{A}$ is a ring; the proofs
would hold if $\mathcal{A}$ was any collection of sets and $\mu^{*}$
was any outer measure (as opposed to a \emph{canonical }outer measure).
\end{rem}

\begin{rem}
\label{rem:zeroOuterMeasureMeasurable}Observe that every set $A\subseteq2^{\X}$
such that $\mu^{*}\left(A\right)=0$ is in the $\sigma-$algebra $\mathcal{C}\left(\mu^{*}\right)$.
To see why, note that $\mu^{*}\left(E\right)\leq\mu^{*}\left(A\cap C\right)+\mu^{*}\left(A^{C}\cap E\right)$
for any $E\in2^{\X}$ by the subadditivity of outer measure. On the
other hand, $\mu^{*}\left(A\cap E\right)=0$ by monotonicity, and
so $\mu^{*}\left(E\right)\geq\mu^{*}\left(A^{C}\cap E\right)=\mu^{*}\left(A\cap E\right)+\mu^{*}\left(A^{C}\cap E\right)$
where the inequality is again monotonicity. This tells us that the
measure $\mu^{*}\mid_{\mathcal{C}}$ is \emph{complete }in that every
subset of a measure zero set is measurable and has measure zero. Complete
measures form an important part of measure theory; as we shall see,
the Lebesgue measure is complete, and $\mathcal{C}\left(\lambda^{*}\right)$
-- called the \emph{Lebesgue $\sigma-$algebra }is the \emph{completion
}of the Borel sets $\borel\left(\R\right)$. In other words, you get
the Lebesgue sets by adjoining all subsets of measure zero Borel sets
to $\borel\left(\R\right)$. This fact is not trivial and shall be
proved later in this chapter.
\end{rem}

Note that in general such an extension may not be unique and we provide
sufficient conditions for uniqueness in Theorem \ref{thm:uniquenessMeasures}

\subsection{An extension of the extension theorem}

We now have enough machinery to construct the Lebesgue measure on
$\borel\left(\R\right)$; in fact, the Caratheodory measurabality
critierion discussed in the proof of the extension theorem \ref{thm:caratheodoryExtn}
is strictly larger than the Borel sets, a fact that we hinted at in
Remark \ref{rem:zeroOuterMeasureMeasurable}. To show the Lebesgue
measure exists, we observe that $\lambda_{2}$ is a countably-additive
pre-measure on $\mathcal{J}$ which is a ring. The canonical outer
measure $\lambda^{*}$ generated from $\left(\lambda_{2},\mathcal{J}\right)$
then can be restricted to the $\sigma$-algebra of measurable sets
$\mathcal{C}\left(\lambda^{*}\right)$ as a measure via Caratheodory's
extension theorem. That the Borel sets $\borel\left(\R\right)\subseteq\mathcal{C}\left(\lambda^{*}\right)$
is clear from the fact that $\mathcal{L}\subseteq\mathcal{\mathcal{J}\subseteq C}\left(\lambda^{*}\right)$
and $\sigma\left(\mathcal{L}\right)=\borel\left(\R\right)$ (see Proposition\ref{prop:sigmaAlgebraGeneratedbyLisBorel}).
The fact that this inclusion is strict is, of course, not obvious;
we will return to this point later.

While this strategy to build the Lebesgue meaure works, we can in
fact do something more general, which will incidentally also help
us establish the properties of the Lebesgue measure. This involves
the notion of what is called a \emph{Steiljes }measure, a concept
which is particularly useful in probability theory. We will return
to this topic at the end of the chapter, after we finish our discussion
on general measures on any arbitrary measurable space $\measurablespace.$
The main tool wewill use is the following generalization of Caratheodory's
extension theorem.
\begin{thm}[Extension from semi-rings]
\label{thm:semiRingCaratheodoryExtn}Let $\X$ be a set and $\mathcal{A}\subseteq2^{\X}$
be a semi-ring. Suppose $\mu:\mathcal{A}\longrightarrow\left[0,\infty\right]$
is a set function such that

\begin{enumerate}[label=(\roman*),leftmargin=.1\linewidth,rightmargin=.4\linewidth]
	\item $\mu\left(\emptyset\right) = 0 $
	\item  For any disjoint $A,B \in \mathcal{A}$ such that $A\cup B \in \mathcal{A}$
	\[
			\mu\left(A\cup B\right) = \mu\left(A\right) + \mu\left(B\right)
	\]
	\item For any collection $A_i \in \mathcal{A}$ such that $\bigcup_{i\in \mathds{N}} A_i \in \mathcal{A}$
	\[
		\mu\left(\bigcup_{i\in\mathds{N}} A_i \right) \leq \sum_{i \in \mathds{N}} \mu \left(A_i\right)
	\]
\end{enumerate}then the restriction of the canonical outer measure $\mu^{*}$ generated
by $\left(\mu,\mathcal{A}\right)$ to $\mathcal{C}\left(\mu^{*}\right)$
is a measure.
\end{thm}

\begin{proof}
Note that Remark \ref{rem:noRingReqd} tells us that $\mathcal{C}\left(\mu^{*}\right)$
is a $\sigma-$algebra and $\mu^{*}|_{\mathcal{C}}$ is a measure.
Thus our two tasks are to show \emph{(i)} that $\mu^{*}$ and $\mu$
agree on\emph{ $\mathcal{A}$ }and\emph{ (ii)} that $\mathcal{A}\subseteq\mathcal{C}\left(\mu^{*}\right)$.
The first result is mostly straightforward; to see that $\mu^{*}\left(A\right)\leq\mu\left(A\right)$
for $A\in\mathcal{A}$ we can simply observe that $A$ is a cover
for itself. For the reverse inequality, first note that finite additivity
and the fact that $\mathcal{A}$ is a semi-ring implies that for any
$A,B\in\mathcal{A}$ such that $A\subseteq B,\mu\left(A\right)\leq\mu\left(B\right).$
Indeed, there exist disjoint $\left\{ C_{i}\right\} _{1\leq i\leq n}\in\mathcal{A}$
where $n\in\N$ such that $B\setminus A=\cup_{1\leq i\leq n}C_{i}$
and so $B=A\cup\cup_{1\leq i\leq n}C_{i}$ and $\mu\left(B\right)=\mu\left(A\right)+\sum_{1\leq i\leq n}\mu\left(C_{i}\right)\geq\mu\left(A\right).$
Then for any cover $\left\{ A_{i}\right\} _{i\in\N}\in\mathcal{A}$
of $A$, we have that 
\[
\mu\left(A\right)\leq\sum_{i=1}^{\infty}\mu\left(A\cap A_{i}\right)\leq\sum_{i=1}^{\infty}\mu\left(A_{i}\right)
\]
where the first ineqality follows from subadditivity and the fact
that $A\cap A_{i}\in\mathcal{A}$ since semi-rings are closed under
intersection along with the fact that $\bigcup_{i\in\N}\left(A\cap A_{i}\right)=A$
and the second inequality follows from monotonicty. Since the cover
was arbitrary, we have $\mu^{*}\left(A\right)\geq\mu\left(A\right).$

To prove \emph{(ii), }note that for any $A\in\mathcal{A}$, countable
subadditivity of the outer measure implies that 
\[
\mu^{*}\left(E\right)\leq\mu^{*}\left(A\cap E\right)+\mu^{*}\left(A^{C}\cap E\right)
\]
for any $E\in2^{\X}.$ To deduce the other inequality, we follow almost
exactly the same steps as we did in the proof of Theorem \ref{thm:caratheodoryExtn}.
First, we pick an $E\subseteq2^{\X}$ such that $\mu^{*}\left(E\right)<\infty$since
otherwise the claim follows trivially. Then we use this fact (since
only empty subsets of the reals have inifinite infima) to deduce that
for any $\epsilon>0$, there exists a cover $\left\{ A_{i}\right\} _{i\in\N}\in\mathcal{A}$of
$E$ such that
\[
\mu^{*}\left(E\right)\leq\sum_{i=1}^{\infty}\mu\left(A_{i}\right)<\mu^{*}\left(E\right)+\epsilon.
\]
Again, we observe that 
\begin{align*}
E\cap A & \subseteq\bigcup_{i=1}^{\infty}A_{i}\cap A\\
E\cap A^{C} & \subseteq\bigcup_{i=1}^{\infty}A_{i}\cap A^{C}
\end{align*}
Note that $A_{i}\cap A^{C}=A_{i}\setminus A$ and so by the properties
of semi-rings there exists, for each $i,$a disjoint collection of
sets $\left\{ C_{j}^{i}\right\} _{1\leq j\leq n_{i}}\in\mathcal{A}$
such that $A_{i}\cap A^{C}=\bigcup_{1\leq j\leq n_{i}}C_{j}^{i}.$
Using the monotonicty and countable-subadditivity of outer measures
as before, we have that
\begin{align*}
\mu^{*}\left(E\cap A\right)+\mu^{*}\left(E\cap A^{C}\right) & \leq\sum_{i=1}^{\infty}\left(\mu^{*}\left(A_{i}\cap A\right)+\mu^{*}\left(\bigcup_{1\leq j\leq n_{i}}C_{j}^{i}\right)\right)\\
 & \leq\sum_{i=1}^{\infty}\left(\mu^{*}\left(A_{i}\cap A\right)+\sum_{j=1}^{n_{i}}\mu^{*}\left(C_{j}^{i}\right)\right)\\
 & =\sum_{i=1}^{\infty}\left(\mu\left(A_{i}\cap A\right)+\sum_{j=1}^{n_{i}}\mu\left(C_{j}^{i}\right)\right)\\
 & =\sum_{i=1}^{\infty}\left(\mu\left(A_{i}\right)\right)\\
 & <\mu^{*}\left(E\right)+\epsilon
\end{align*}
where the first equality follows from part \emph{(i) }and the fact
that $A_{i}\cap A,\left\{ C_{j}^{i}\right\} _{1\leq j\leq n_{i}}\in\mathcal{A}$
whereas the second equality follows from finite additivity. Since
$\epsilon$can be as small as one wants, our result follows.
\end{proof}

\section{Abstract measure spaces}
\begin{defn}
\label{def:measurableSpace}A pair $\left(\mathcal{X},\mathcal{F}\right)$,
where $\mathcal{X}$ is an arbitrary set and $\mathcal{F}$ is a $\sigma-$algebra
on $\mathcal{X}$, is called a \emph{measurable space.}
\end{defn}

Although we had implicitly defined a measure in the previous section,
it's appopriate to write down a formal definition in this section.
\begin{defn}
\label{def:measureSpace}Let $\left(\mathcal{X},\mathcal{F}\right)$
be a measurable space. A function $\mu:\mathcal{F}\longrightarrow\left[0,\infty\right]$
is a \emph{measure }on $\mathcal{X}$ if

\begin{enumerate}[label=(\roman*),leftmargin=.1\linewidth,rightmargin=.4\linewidth]
	\item $\mu\left(\emptyset\right)= 0$
	\item For disjoint $\{A_i\}_{i\in \N} \in \mathcal{F}$ 
	\[
			\mu\left(\bigcup_{i=1}^{\infty}A_i\right) = \sum_{i=1}^{\infty}\mu\left(A_i\right).
	\]
\end{enumerate}The triple $\left(\mathcal{X},\mathcal{F},\mu\right)$ is called a
\emph{measure space. }If $\mu\left(\mathcal{X}\right)=1$ then $\mu$
is called a \emph{probability measure }and $\left(\mathcal{X},\mathcal{F},\mu\right)$
is called a \emph{probability space.}
\end{defn}

\begin{defn}
\label{def:measurableSet}Given a measurable space $\left(\mathcal{X},\mathcal{F}\right)$,
any set $A\in\mathcal{F}$ is called a \emph{measurable }set. Conversely,
any set $A\subset\mathcal{X}$ such that $A\notin\mathcal{F}$ is
referred to as a \emph{non-measurable }set.
\end{defn}

While the definition of a measure is simple, it turns out to have
some remarkable properties that are useful in the theory of integration
and probability that is built on top of measure theory (or, as we
shall later see, is equivalent to it).
\begin{prop}
\label{prop:measureProperties}Let $\left(\mathcal{X},\mathcal{F}\right)$
be a measurabe space and let 
\[
\mu:\mathcal{F}\longrightarrow\left[0,\infty\right]
\]
be a function. Then $\mu$ is a measure if and only if

\begin{enumerate}[label=(\roman*),leftmargin=.1\linewidth,rightmargin=.4\linewidth]
	\item $\mu\left(\emptyset\right)= 0$
	\item For disjoint $A,B \in \mathcal{F}$ 
	\[
			\mu\left(A \cup B\right) = \mu\left(A\right) + \mu\left(B\right) .
	\]
	\item For any increasing sequence of sets $ A_1 \subseteq A_2 \ldots $ in $\mathcal{F}$ such that $\bigcup_{i\in\N} A_i = A $
	\[
			\mu\left(A\right) = \lim_{i \to \infty}\mu\left(A_i\right)
	\]
\end{enumerate}
\end{prop}

\begin{proof}
First we shall establish that Definition \ref{def:measureSpace} implies
properties (i)-(iii) above. Property (i) is inherited straight from
the definition; to see (ii), we can let $A_{1}=A,A_{2}=B$ and $A_{j}=\emptyset$
for all $j\geq3$. Then
\[
\mu\left(A\cup B\right)=\mu\left(\bigcup_{j\in\N}A_{j}\right)=\sum_{j=1}^{\infty}\mu\left(A_{j}\right)=\mu\left(A\right)+\mu\left(B\right)
\]
where the second equality is due countably additiivity and the third
equality is due to property (i). To see property (iii), let $\left\{ A_{i}\right\} _{i\in\N}$
be an increasing sequence of sets such that $A_{i}\subseteq A_{i+1}$
for every $i\in\N$ and let $A:=\bigcup_{i\in N}A_{i}$. Define 
\[
B_{i}:=A_{i}\setminus\bigcup_{j=1}^{i-1}A_{j}
\]
which is the standard ``disjointification'' of $\left\{ A_{i}\right\} _{i\in\N}$
as we have seen earlier. By countable additivity
\begin{align*}
\mu\left(A\right) & =\sum_{i=1}^{\infty}\mu\left(B_{i}\right)\\
 & =\lim_{n\to\infty}\sum_{i=1}^{n}\mu\left(B_{i}\right)\\
 & =\lim_{n\to\infty}\mu\left(\bigcup_{i=1}^{n}B_{i}\right)\\
 & =\lim_{n\to\infty}\mu\left(\bigcup_{i=1}^{n}A_{i}\right)\\
 & =\lim_{n\to\infty}\mu\left(A_{n}\right)
\end{align*}
where the third equality is due to property (ii). The fourth equality
follows from the disjointification and the last equality is due to
the increasing nature of the sequence of sets.

Next, we shall establish countable additivity while assuming properties
(i)-(iii) in order to complete the equivalence. Let $\left\{ A_{i}\right\} _{i\in\N}$
be pairwise disjoint in $\mathcal{F}$. Then, letting $A:=\bigcup_{i\in\N}A_{i}$
we can define
\[
B_{n}:=\bigcup_{i=1}^{n}A_{i}
\]
and observe that $\bigcup_{n\in\N}B_{n}=A$ and $B_{n}\subseteq B_{n+1}.$
Then, by property (iii), 
\begin{align*}
\mu\left(A\right) & =\lim_{n\to\infty}\mu\left(B_{n}\right)\\
 & =\lim_{n\to\infty}\sum_{i=1}^{n}\mu\left(A_{i}\right)\\
 & =\sum_{i=1}^{\infty}\mu\left(A_{i}\right)
\end{align*}
where the second equality is due to finite additivity (property (ii)).
This completes the proof.
\end{proof}
\begin{rem*}
Property (iii) resembles a continuity condition, and is indeed called
\emph{continuity from below }of measures. There is an analagous definition
for \emph{continuity from above} which is implied by \emph{continuity
from above }for finitely additive measures and pre-measures. If the
measures are finite, these two notions of continuity are in fact equivalent.
\end{rem*}
\begin{cor}
\label{cor:countableSubadditivity}Every measure $\mu$ on an arbitrary
measurable space $\left(\mathcal{X},\mathcal{F}\right)$ is countably
subadditive i.e. for any collection $\left\{ A_{i}\right\} _{i\in\N}\in\mathcal{F}$
\[
\mu\left(\bigcup_{i=1}^{\infty}A_{i}\right)\leq\sum_{i=1}^{\infty}\mu\left(A_{i}\right).
\]
\end{cor}

\begin{proof}
We shall first establish \emph{finite }subadditivity and bootstrap
this result to countable subadditivity. To see finite subadditivity,
let $A,B\in\mathcal{F}$ be arbitrary, and observe that
\[
A\cup B=\left(A\setminus B\right)\cup B.
\]
The two sets on the right hand side are disjoint and so by finite
additivity
\begin{align*}
\mu\left(A\cup B\right) & =\mu\left(A\setminus B\right)+\mu\left(B\right).
\end{align*}
Adding $\mu\left(A\cap B\right)$ and applying finite additivity again,
we deduce that
\[
\mu\left(A\cup B\right)+\mu\left(A\cap B\right)=\mu\left(A\right)+\mu\left(B\right)
\]
which establishes finite subadditivity. To prove the countable analogue,
let
\[
B_{n}:=\bigcup_{i=1}^{n}A_{i}
\]
and observe that by finite subadditivity
\[
\mu\left(B_{n}\right)\leq\sum_{i=1}^{n}\mu\left(A_{i}\right)\leq\sum_{i=1}^{\infty}\mu\left(A_{i}\right)
\]
where the last inequality follows by the non-negativity of $\mu.$
Note that since $B_{n}$ is an increasing sequence, we can apply Proposition
\ref{prop:measureProperties} (iii) to infer that
\[
\mu\left(\bigcup_{i=1}^{\infty}A_{i}\right)=\lim_{n\to\infty}\mu\left(B_{n}\right)\leq\sum_{i=1}^{\infty}\mu\left(A_{i}\right).
\]
\end{proof}
\begin{prop}
\label{prop:equivalenceContinuityMeasures}For a finitely additive
measure $\mu:\mathcal{F}\longrightarrow\left[0,\infty\right),$ the
following statements are equivalent:

\begin{enumerate}[label=(\roman*),leftmargin=.1\linewidth,rightmargin=.4\linewidth]
	\item For any increasing sequence of sets $\left\{ A_{i}\right\} _{i\in\N}$ such that $A_i \subseteq A_{i+1}$ for all $i\in \N$ 
	\[
					\mu\left(\bigcup_{i=1}^{\infty} A_i\right) = \lim_{i\to\infty}\mu\left(A_i\right).
	\]
	\item For any decreasing sequence of sets $\left\{ A_{i}\right\} _{i\in\N}$ such that $ A_{i+1}\subseteq A_i$ for all $i\in \N$ 
	\[
					\mu\left(\bigcap_{i=1}^{\infty} A_i\right) = \lim_{i\to\infty}\mu\left(A_i\right).
	\]
\end{enumerate}
\end{prop}

\begin{proof}
Assuming (i), let $\left\{ A_{i}\right\} _{i\in\N}$ be a decreasing
sequence of sets and let $A:=\bigcap_{i\in\N}A_{i}$. Then define
$B_{i}=A_{1}\setminus A_{i}$ which is an increasing sequence of sets
such that $A_{1}\setminus A=\bigcup_{i\in\N}B_{i}$. By (i), 
\[
\mu\left(A_{1}\right)-\mu\left(A\right)=\mu\left(A_{1}\setminus A\right)=\lim_{i\to\infty}\mu\left(B_{i}\right)=\mu\left(A_{1}\right)-\lim_{i\to\infty}\mu\left(A_{i}\right)
\]
where the first and last equality are due to finite additivity, the
finiteness of $\mu.$ We can subtract $\mu\left(A_{1}\right)$ from
both sides to yield the result.

To establish the converse, assume (ii) and let $\left\{ A_{i}\right\} _{i\in\N}$
be an increasing sequence of sets and define $A:=\bigcup_{i\in\N}A_{i}$.
Let $B_{i}:=A\setminus A_{i}$ which is a decreasing sequence of sets
such that $\bigcap_{i\in\N}B_{i}=\emptyset$. By (ii), we have that
\[
0=\mu\left(\emptyset\right)=\lim_{i\to\infty}\mu\left(B_{i}\right)=\lim_{i\to\infty}\mu\left(A\setminus A_{i}\right)=\mu\left(A\right)-\lim_{i\to\infty}\mu\left(A_{i}\right)
\]
where the last equality is again due to finite additivity and the
finitenesss of $\mu.$ Rearrangement yields the proof.
\end{proof}
Observe how the two results apply without modification to pre-measures
as well and so we can establish the countable additivity of $\lambda_{2}$
(see the previous section) using a continuity argument instead of
the Heine-Borel argument we previously used (Exercise!).
\begin{prop}
\label{prop:sumOfCountableMeasures}Let $\left(\mathcal{X},\mathcal{F}\right)$
be a measurable space and let $\left\{ \mu_{i}\right\} _{i\in\mathcal{I}}$
be a collection of measures on $\mathcal{F}$ where $\mathcal{I}$
is at most countable. Then
\[
\mu:=\sum_{i\in\mathcal{I}}\mu_{i}
\]
is a measure on $\mathcal{F}$.
\end{prop}

\begin{proof}
First observe that 
\[
\mu\left(\emptyset\right)=\sum_{i\in\mathcal{I}}\mu_{i}\left(\emptyset\right)=0.
\]
Next, let $\left\{ A_{j}\right\} _{j\in\N}\in\mathcal{F}$ be disjoint.
Then
\begin{align*}
\mu\left(\bigcup_{j\in\N}A_{j}\right) & =\sum_{i\in\mathcal{I}}\mu_{i}\left(\bigcup_{j\in\N}A_{j}\right)\\
 & =\sum_{i\in\mathcal{I}}\sum_{j\in\N}\mu_{i}\left(A_{j}\right)\\
 & =\sum_{j\in\N}\sum_{i\in\mathcal{I}}\mu_{i}\left(A_{j}\right)\\
 & =\sum_{j\in\N}\mu\left(A_{j}\right)
\end{align*}
where the second equality follows from the countable additivity of
$\mu_{i}$ and the third equality follows from the non-negativity
of measures and Lemma \ref{lem:TonelliForSeries}. This completes
the proof.
\end{proof}

\subsection{$\sigma-$finite measure spaces}
\begin{defn}
\label{def:sigmaFinite}Let $\left(\X,\F,\mu\right)$ be a measure
space. The measure $\mu$ is said to be $\sigma-$\emph{finite }if
there exists some increasing sequence of sets $\left\{ E_{i}\right\} _{i\in\N}\in\F$
such that $\mu\left(E_{i}\right)<\infty$ and
\[
\bigcup_{i\in\N}E_{i}=\X.
\]
\end{defn}

\begin{prop}
\label{prop:equivSigmaFinite}Let $\left(\X,\F,\mu\right)$ be a measure
space. The measure $\mu$ being $\sigma-$finite is equivalent to
any of the following conditions

\begin{enumerate}[label=(\roman*),leftmargin=.1\linewidth,rightmargin=0.15\linewidth]
	\item There exists some \textbf{pairwise disjoint} countable collection of sets $\{A_i\}_{i\in\N} \in \F$ such that $ \mu\left(A_i\right) < \infty $ for all $ i \in \N $ and
	\[
					\bigcup_{i\in\N}A_i = \X
	\]
	\item There exists some countable collection of sets $\{B_i\}_{i\in\N} \in \F$ such that $ \mu\left(B_i\right) < \infty $ for all $ i \in \N $ and
	\[
					\bigcup_{i\in\N}B_i = \X
	\]
\end{enumerate}
\end{prop}

\begin{proof}
First assume that the measure $\mu$is $\sigma-$finite and so there
exists some increasing sequence $\left\{ E_{i}\right\} _{i\in\N}\in\F$
such that $E_{i}\subseteq E_{I+1}$, $\mu\left(E_{i}\right)<\infty$
and 
\[
\bigcup_{i\in\N}E_{i}=\X.
\]
Recall the disjointification
\[
A_{i}:=E_{i}\setminus\bigcup_{j=1}^{i-1}E_{j}
\]
and notice that
\begin{align*}
\mu\left(A_{i}\right) & =\mu\left(E_{i}\right)-\mu\left(\bigcup_{j=1}^{i-1}E_{j}\right)\\
 & =\mu\left(E_{i}\right)-\mu\left(E_{i-1}\right)\\
 & <\infty
\end{align*}
where the first equality follows from the fact that $\mu\left(E_{i}\right)<\infty$
and $\bigcup_{j=1}^{i-1}E_{j}=E_{i-1}\subseteq E_{i}$ along with
(finite) additivity. Further,
\[
\bigcup_{i\in\N}A_{i}=\X
\]
 and so Definition \ref{def:sigmaFinite} implies (i).

Next notice that (i) trivially implies (ii) and so all we just need
to verify (ii) $\implies$Definition \ref{def:sigmaFinite}. To this
end, observe that if $\left\{ B_{i}\right\} _{i\in\N}\in\F$ is an
arbitrary collection that satisfies (ii), then
\[
E_{n}:=\text{\ensuremath{\bigcup_{i=1}^{n}B_{i}}}
\]
is an increasing sequence of sets $E_{n}\subseteq E_{n+1}$ such that
\[
\mu\left(E_{n}\right)\leq\sum_{i=1}^{n}\mu\left(B_{i}\right)<\infty
\]
and
\[
\bigcup_{n\in\N}E_{n}=\X.
\]
\end{proof}
\begin{prop}
\label{prop:sumSigmaFiniteMeasures}Let $\left(\X,\F\right)$ be a
measurable space and let $\left\{ \mu_{i}\right\} _{i=1}^{N}$be a
finite collection of $\sigma-$finite measure on $\F.$ Then the total
measure
\[
\mu:\F\longrightarrow\R
\]
 given by
\[
\mu\left(A\right):=\sum_{i=1}^{N}\mu_{i}\left(A\right)
\]
is also $\sigma-$finite.
\end{prop}

\begin{proof}
A weaker variant of Proposition \ref{prop:sumOfCountableMeasures}
shows that $\mu$is at least a measure on $\F$. We show $\sigma-$finiteness
for $N=2$; the general case follows by induction. Note that if $\mu_{1}$
and $\mu_{2}$ are both $\sigma-$finite then by Proposition \ref{prop:equivSigmaFinite}
there exist $\left\{ E_{1,i}\right\} _{i\in\N},\left\{ E_{2,i}\right\} _{i\in\N}\in\F$
such that $\mu_{1}\left(E_{i,1}\right)<\infty$ and $\mu_{2}\left(E_{2,i}\right)<\infty$
for all $i\in\N$. Further,
\[
\bigcup_{i\in\N}E_{1,i}=\bigcup_{i\in\N}E_{2,i}=\X.
\]
Then, define
\[
C_{i,j}:=E_{1,i}\bigcap E_{2,j}
\]
and observe that $C_{i,j}\in\F$ and that 
\begin{align*}
\mu\left(C_{i,j}\right) & =\mu_{1}\left(E_{1,i}\cap E_{2,j}\right)+\mu_{2}\left(E_{1,i}\cap E_{2,j}\right)\\
 & \leq\mu_{1}\left(E_{1,i}\right)+\mu_{2}\left(E_{2,j}\right)\\
 & <\infty
\end{align*}
for all $\left(i,j\right)\in\N^{2}$. Finally,
\[
\bigcup_{i\in\N}\bigcup_{j\in\N}C_{i,j}=\X
\]
which by Proposition \ref{prop:equivSigmaFinite} establishes the
result.
\end{proof}

\section{The Stieljes measure on $\protect\R$}
\begin{defn}
\label{def:stieljesFunction}A function $F:\R\to\R$ is called a Stieljes
function if it is non-decreasing on $\R$ and if it is \emph{right-continuous
}i.e. for any $c\in\R$
\[
\lim_{x\to c^{+}}f\left(x\right)=f\left(c\right).
\]
\end{defn}

\begin{prop}
\label{prop:measureDefinesStieljes}Let $\mu$ be a measure on $\borel\left(\R\right)$
that is finite on any bounded interval $I\subseteq\R.$ Define the
function $F:\R\to\R$ such that for $b>a$
\[
F\left(x\right)=\mu\left(\left(-\infty,x\right]\right).
\]
Then $F$ is a Stieljes function.
\end{prop}

\begin{proof}
Note that $F$ is non-decreasing, since for $x\leq y$ $\left(-\infty,x\right]\subseteq\left(-\infty,y\right]$
and so $F\left(x\right)\leq F\left(y\right)$ by the monotonicity
of measures. Further, let $x_{n}\to x$ from the right. Then there
exists some decreasing subsequence $x_{n_{k}}\to x$ as well. Now
$\left(-\infty,x_{n_{k}}\right]\subseteq\left(-\infty,x_{n_{k+1}}\right]$
and $\bigcap_{k}\left(-\infty,x_{n_{k}}\right]=\left(-\infty,x\right]$
and so by Propositions \ref{prop:measureProperties} and \ref{prop:equivalenceContinuityMeasures}
(noting that $\mu$ is always finite for any such intervals)
\[
F\left(x_{n_{k}}\right)\to F\left(x\right).
\]
Next, suppose that there is some $\epsilon>0$ such that for some
subsequence $x_{m_{k}}$
\[
F\left(x_{m_{k}}\right)>F\left(x\right)+\epsilon.
\]
By convergence, there exists some $k_{0}$ such that for all $k\geq k_{0}$
\[
F\left(x_{n_{k}}\right)<F\left(x\right)+\epsilon.
\]
Since $x_{m_{k}}\to x$, there must be some $k$ such that $x_{m_{k}}\leq x_{n_{k_{0}}}$
and so, by the monotonicity of $F$
\[
F\left(x_{m_{k}}\right)\leq F\left(x_{n_{k_{0}}}\right)<F\left(x\right)+\epsilon,
\]
a contradiction. Therefore $F\left(x_{n}\right)\to F\left(x\right).$
\end{proof}
\begin{thm}
\label{thm:stieljesMeasure}Let $F:\R\to\R$ be a Stieljes function.
Then the set function on $\mu:\mathcal{L\to}\left[0,\infty\right)$
given by
\[
\mu\left(\left(a,b\right]\right):=F\left(b\right)-F\left(a\right)
\]
extends to a measure on $\borel\left(\R\right).$
\end{thm}

\begin{proof}
Note that $\mu\left(\emptyset\right)=\mu\left(\left(a,a\right]\right)=0$
for any $a\in\R$. Next, let $A,B\in\mathcal{L}$ such that $\text{A}\cap B=\emptyset$
and $A\cup B\in\mathcal{L}$. Then if $A=\left(a_{1},a_{2}\right]$
and $B=\left(b_{1},b_{2}\right]$ , it must be that $b_{1}=a_{2}$
or $b_{2}=a_{1}.$ Suppose without loss of generality that it is the
former. Then
\begin{align*}
\mu\left(A\cup B\right) & =\mu\left(\left(a_{1},b_{2}\right]\right)\\
 & =F\left(b_{2}\right)-F\left(a_{1}\right)\\
 & =F\left(b_{2}\right)-F\left(b_{1}\right)+\left(F\left(b_{1}\right)-F\left(a_{1}\right)\right)\\
 & =F\left(b_{2}\right)-F\left(b_{1}\right)+\left(F\left(a_{2}\right)-F\left(a_{1}\right)\right)\\
 & =\mu\left(\left(b_{1},b_{2}\right]\right)+\mu\left(\left(a_{1},a_{2}\right]\right)\\
 & =\mu\left(A\right)+\mu\left(B\right)
\end{align*}
where the fourth equality is due to the fact that $b_{1}=a_{2}.$

Next, let $A,B\in\mathcal{L}$ and such that $A\cup B\in\mathcal{L}.$
Since $\mathcal{L}$ is a semi ring, $A\setminus B=\bigcup_{i=1}^{m}C_{i}$
where $C_{i}\in\mathcal{L}$ are disjoint and $B\setminus A=\bigcup_{i=1}^{n}D_{i}$
where $D_{i}\in\mathcal{L}$ are disjoint. Then $A\cup B=\bigcup_{i=1}^{m}C_{i}\cup B=\bigcup_{i=1}^{n}D_{i}\cup A$
and so 
\[
\mu\left(A\cup B\right)=\mu\left(A\right)+\mu\left(B\right)+\frac{1}{2}\left(\sum_{i=1}^{m}\mu\left(C_{i}\right)+\sum_{j=1}^{n}\mu\left(D_{i}\right)\right)
\]
and so by non-negativity
\[
\mu\left(A\cup B\right)\leq\mu\left(A\right)+\mu\left(B\right)
\]
which is finite subadditivity.

Finally, let $\left\{ A_{i}\right\} _{i\in\N}\in\mathcal{L}$ be such
that $A:=\bigcup_{i\in\N}A_{i}\in\mathcal{L}.$ Let $A_{i}=\left(a_{i},b_{i}\right]$,
$A=\left(a,b\right]$ and notice that for any $0<\epsilon<b-a$ 
\[
\left[a+\epsilon,b\right]\subseteq\bigcup_{i\in\N}\left(a_{i},b_{i}+\epsilon\right).
\]
By the Heine Borel theorem, there exists a finite subcover $\left\{ \left(a_{i_{k}},b_{i_{k}}+\epsilon\right)\right\} _{k=1}^{n}$
such that
\[
\left(a+\epsilon,b\right]\subset\left[a+\epsilon,b\right]\subseteq\bigcup_{k=1}^{n}\left(a_{i_{k}},b_{i_{k}}+\epsilon\right)\subset\bigcup_{k=1}^{n}\left(a_{i_{k}},b_{i_{k}}+\epsilon\right].
\]
We can choose this subcover such that its union is in $\mathcal{L}$
without loss of generality\hl{prove this}and so, by the finite subadditivity
result and the monotonicity of $\mu$\footnote{If $A=\left(a_{1},a_{2}\right]\subseteq\left(b_{1},b_{2}\right]=B$
then clearly $F\left(b_{2}\right)-F\left(b_{1}\right)\geq F\left(a_{2}\right)-F\left(a_{1}\right)$
by the fact that $F$ is non-decreasing.}
\[
F\left(b\right)-F\left(a+\epsilon\right)\leq\sum_{k=1}^{n}F\left(b_{i_{k}}+\epsilon\right)-F\left(a_{i_{k}}\right).
\]
Letting $\epsilon\to0$ and applying the right continuity of $F$,
we have that 
\begin{align*}
\mu\left(A\right)=F\left(b\right)-F\left(a\right) & \leq\sum_{k=1}^{n}F\left(b_{i_{k}}\right)-F\left(a_{i_{k}}\right)\\
 & \leq\sum_{i=1}^{\infty}F\left(b_{i}\right)-F\left(a_{i}\right)\\
 & =\sum_{i=1}^{\infty}\mu\left(A_{i}\right).
\end{align*}
Applying Theorem \ref{thm:semiRingCaratheodoryExtn}, our function
$\mu$ extends to a measure on $\borel\left(\R\right)$.
\end{proof}
Thus Proposition \ref{prop:measureDefinesStieljes} and Theorem \ref{thm:stieljesMeasure}
together show that Stieljes functions and measures that are finite
on bounded intervals are in a one-to-one correspondence. For any Stieljes
function $F$ , we denote the corresponding measure by $\mu_{F}$.
Stieljes measures enjoy a few important properties.
\begin{prop}
\label{prop:stieljesSingleton}Let $F:\R\to\R$ be a Stieljes function
and let $\mu_{F}:\borel\left(\R\right)\to\left[0,\infty\right]$ be
the corresponding measure. Then for any $x\in\R$, $\mu\left(\left\{ x\right\} \right)=0$
if and only if $F$ is continuous at $x.$
\end{prop}

\begin{proof}
Observe that $\left\{ x\right\} \in\borel\left(\R\right)$ since it
is closed. Then,
\[
\mu_{F}\left(\left\{ x\right\} \right)=\lim_{n\to\infty}\mu_{F}\left(\left(x-\frac{1}{n},x\right]\right)=\lim_{n\to\infty}F\left(x\right)-F\left(x-\frac{1}{n}\right)
\]
where the first equality is Proposition \ref{prop:equivalenceContinuityMeasures}.
The result then follows.
\end{proof}
We need a couple a lemmas to show that the Lebesgue measure exists
with all our requisite properties. In the following, $\lambda^{*}\left(A\right):=\inf\left\{ \sum_{j=1}^{\infty}\lambda_{1}\left(I_{j}\right)\mid A\subseteq\bigcup_{j\in\N}I_{j},I_{j}\in\mathcal{L}\right\} $
is the canonical Lebesgue outer measure.
\begin{lem}
\label{lem:outerMeasureTranslationInvariant}Let $A\subseteq\R$ be
an arbitrary set. Then the translated set $A+t:=\left\{ a+t\mid a\in A\right\} $
has Lebesgue outer measure
\[
\lambda^{*}\left(A+t\right)=\lambda^{*}\left(A\right)
\]
for all $t\in\R$. Moreover, for any set $A\in\borel\left(\R\right),$$A+t\in\borel\left(\R\right).$
\end{lem}

\begin{proof}
First, observe that $\lambda_{1}:\mathcal{L}\to\left[0,\infty\right]$
is translation invariant. Then, if $\left\{ I_{j}\right\} _{j\in\N}\in\mathcal{L}$
is a cover for $A\subseteq\R$, then $\left\{ I_{j}+t\right\} _{j\in\N}$
is a cover for $A+t$ and so
\[
\lambda^{*}\left(A+t\right)\leq\sum_{j\in\N}\lambda_{1}\left(I_{j}+t\right)=\sum_{j\in\N}\lambda_{1}\left(I_{j}\right).
\]
Taking infimums on the right side yield
\[
\lambda^{*}\left(A+t\right)\leq\lambda^{*}\left(A\right).
\]
To get the other inequality, we can let $A-t$ play the role of $A$
in the above inequality to yield
\[
\lambda^{*}\left(A\right)\leq\lambda^{*}\left(A-t\right)
\]
for all $t\in\R$. In particular, the inequality holds if we replace
$t$ with $-t$ and the result follows.

Next, for some fixed $t\in\R$ let $\mathcal{B}_{t}:=\left\{ B\in\borel\left(\R\right)\mid B+t\in\borel\left(\R\right)\right\} \subseteq\borel\left(\R\right).$
Clearly, $\R\in\mathcal{B}_{t}$ since $\R+t=\R$. Moreover, let $f:\R\to\R$
be given by $f\left(x\right)=x+t$. Then, note that $\mathcal{B}_{t}=\left\{ f^{-1}\left[B\right]\in\borel\left(\R\right)\mid B\in\borel\left(\R\right)\right\} .$\footnote{This works because $f^{-1}\left[f\left[A\right]\right]=f\left[f^{-1}\left[A\right]\right]=A$
for any $A\in\borel\left(\R\right)$.} Therefore for any $A\in\mathcal{B}_{t}$ where $A=f^{-1}\left[B\right]$
for some $B\in\borel\left(\R\right)$, we have $A^{C}=\left(f^{-1}\left[B\right]\right)^{C}=f^{-1}\left[B^{C}\right]\in\borel\left(\R\right).$Note
that $B^{C}\in\borel\left(\R\right)$ as it is a $\sigma-$algebra
and so $A^{C}\in\mathcal{B}_{t}.$ Next, let $\left\{ A_{i}\right\} _{i\in\N}\in\mathcal{B}_{t}$.
Then there exist $B_{i}\in\borel\left(\R\right)$ such that $A_{i}=f^{-1}\left[B_{i}\right]$
and $\bigcup_{i\in\N}A_{i}=\bigcup_{i\in\N}f^{-1}\left[B_{i}\right]=f^{-1}\left[\bigcup_{i\in\N}B_{i}\right]$
where $\bigcup_{i\in\N}B_{i}\in\borel\left(\R\right)$. Therefore
$\bigcup_{i\in\N}A_{i}\in\mathcal{B}_{t}.$ Therefore $\mathcal{B}_{t}$
is a $\sigma-$algebra and moreover, since every interval in $\mathcal{L}$
is a translation of another interval in $\mathcal{L}$ so that $\mathcal{L}+t=\mathcal{L}$,
we have that $\mathcal{B}_{t}$ contains $\mathcal{L}$. Therefore,
$\borel\left(\R\right)=\sigma\left(\mathcal{L}\right)\subseteq\mathcal{B}_{t}$.
We have shown that $\borel\left(\R\right)=\mathcal{B}_{t}$ for any
$t\in\R$ and our claim follows.
\end{proof}
\begin{rem*}
The type of argument that we used to prove that $\borel\left(\R\right)=\mathcal{B}_{t}$
is a type of \emph{generating class }argument. The abstract version
of how such arguments go is described later in section \ref{sec:genClassArgs}.
\end{rem*}
A similar result holds for dilations. That is, the Lebesgue measure
is scales under dilations.
\begin{lem}
\label{lem:outerMeasureScales}Let $A\subseteq\R$ be an arbitrary
set. Then for any $\delta\neq0$, the dilated set $\delta A:=\left\{ \delta a\mid a\in A\right\} $
has Lebesgue outer measure 
\[
\lambda^{*}\left(\delta A\right)=\lvert\delta\rvert\lambda^{*}\left(A\right).
\]
Moreover, for any set $A\in\borel\left(\R\right),$ $\delta A\in\borel\left(\R\right)$.
\end{lem}

\begin{proof}
Let $\delta\neq0$ be fixed. First, notice that $\lambda_{1}:\mathcal{L}\to\R$
scales under dilations in that for any $I\in\mathcal{L}$, $\lambda_{1}\left(\delta I\right)=\lvert\delta\rvert\lambda\left(I\right)$.
Let $\left\{ I_{j}\right\} _{j\in\N}$ be a cover of $A$. Then clearly,
$\left\{ \delta I_{j}\right\} _{j\in\N}$ is a cover of $\delta A$
and so 
\[
\lambda^{*}\left(\delta A\right)\leq\sum_{j\in\N}\lambda_{1}\left(\delta I_{j}\right)=\lvert\delta\rvert\sum_{j\in\N}\lambda_{1}\left(I_{j}\right).
\]
Taking infimums,
\[
\lambda^{*}\left(\delta A\right)\leq\lvert\delta\rvert\lambda^{*}\left(A\right).
\]
 Conversely, if $\left\{ \delta I_{j}\right\} _{i\in\N}$ is a cover
of $\delta A$ then $\frac{1}{\delta}\left\{ \delta I_{j}\right\} _{j\in\N}$
is a cover for $A$ and so 
\[
\lambda^{*}\left(A\right)\leq\sum_{j\in\N}\lambda_{1}\left(\frac{1}{\delta}\delta I_{j}\right)=\lvert\frac{1}{\delta}\rvert\sum_{j\in\N}\lambda_{1}\left(\delta I_{j}\right).
\]
Taking infimums again yields 
\[
\lambda^{*}\left(A\right)\leq\lvert\frac{1}{\delta}\rvert\lambda^{*}\left(\delta A\right)
\]
which completes the proof.

Next, we apply the generating class argument as above, and let $\mathcal{B}_{\delta}=\left\{ B\in\borel\left(\R\right)\mid\delta B\in\borel\left(\R\right)\right\} $.
Letting $f\left(x\right)=\delta x$ (which is an invertible map since
$\delta neq0$), we can apply the same arguments as earlier to show
that $\mathcal{B}_{\delta}$ is a $\sigma-$algebra and then observe
that for any $I\in\mathcal{L}:\delta I\in\mathcal{L}$ so $\mathcal{L}\subseteq\mathcal{B}_{t}$
and so $\sigma\left(\mathcal{L}\right)=\borel\left(\R\right)\subseteq\mathcal{B}_{t}$
which completes the proof.
\end{proof}
\begin{thm}[Existence of the Lebesgue measure]
\label{thm:existenceLebesgueR}There exists a complete $\sigma-$algebra
$\F$ which contains all the open sets in $\R$ and a $\sigma-$finite
measure $\lambda:\F\longrightarrow\left[0,\infty\right]$ such that

\begin{enumerate}[label=(\roman*),leftmargin=.1\linewidth,rightmargin=.4\linewidth]
	\item For any set $A \in \borel\left(\R\right)$ and any $t\in\mathbb{R}$ we have that 
    \[
			\lambda\left(A+t\right) = \lambda\left(A\right).
    \]
	\item For any set $A \in \borel\left(\R\right)$ and any $\delta > 0$
	\[
		\lambda\left(\delta A\right) = \delta \lambda\left(A\right).
	\]

	\item $\lambda\left((a,b]\right) = b - a $ for any $a \leq b \in \mathds{R}$\footnote{The intervals could be open, closed or neither.}

\end{enumerate}
\end{thm}

\begin{proof}
Let $F\left(x\right)=x$ and notice that $F$ increasing and continuous
and so by Theorem \ref{thm:stieljesMeasure}, the function $\lambda:\mathcal{L}\to\left[0,\infty\right]$
given by
\[
\lambda\left(\left(a,b\right]\right)=b-a
\]
extends to a measure on $\borel\left(\R\right).$ To see the $\sigma-$finiteness
of $\lambda$, note that we can write $\R=\bigcup_{n\in\N}\left(n,n+1\right]\cup\left(-n-1,-n\right]$
where each component of the union has finite measure. As our discussion
in Theorem \ref{thm:semiRingCaratheodoryExtn} shows, the extension
is the restriction of the outer measure $\lambda^{*}$ to the Caratheodory
measurable sets $\mathcal{C}\left(\lambda^{*}\right)$ generated by
$\left(\mathcal{L},\lambda^{*}\right)$, which contain the borel sets
$\borel\left(\R\right)$. Let $\mathcal{F}=\mathcal{C}\left(\lambda^{*}\right)$
and notice that our discussion in Remark \ref{rem:zeroOuterMeasureMeasurable}
tells us that $\F$ is complete. Property $\left(i\right)$ follows
by Lemma \ref{lem:outerMeasureTranslationInvariant} and property
$\left(ii\right)$ by Lemma \ref{lem:outerMeasureScales}. To see
property $(iii)$, we need to look at each case separately. For any
compact interval $\left[a,b\right]=\left\{ a\right\} \cup\left(a,b\right]$,
the result follows by additivity and Proposition \ref{prop:stieljesSingleton}.
For an open set $\left(a,b\right)$ note that $\left(a,b\right]=\left(a,b\right)\cup\left\{ b\right\} $
and so again additivity and Proposition \ref{prop:stieljesSingleton}applies.
Finally, we show the same result for $\left[a,b\right)=\left\{ a\right\} \cup\left(a,b\right)$.
\end{proof}

\subsection{The Lebesgue $\sigma-$algebra}

Note that we had claimed earlier that $\borel\left(\R\right)\subsetneq\mathcal{C\left(\lambda^{*}\right)}$.
This actually takes a considerable amount of work to prove; we will
begin to lay the groundwork for the proof here and give the actual
proof in the next chapter. First, we begin with an equivalent definition
of the $\sigma-$algebra $\mathcal{C}\left(\lambda^{*}\right)$, which
we call the Lebesgue $\sigma-$algebra. This helps us characterize
the Lebesgue $\sigma-$algebra as the \emph{completion }of the Borel
$\sigma-$algebra.
\begin{prop}
\label{prop:borelApproximateLebesgue}Let $A\in\mathcal{C}\left(\lambda^{*}\right)$
if and only if for every $\epsilon>0$ there exists some open set
$O$ such that $A\subseteq O$ and
\[
\lambda\left(O\setminus A\right)<\epsilon
\]
\end{prop}

\begin{proof}
Let $A\in\mathcal{C}\left(\lambda^{*}\right)$. First, assume that
$\lambda\left(A\right)<\infty.$ By definition of the Lebesgue outer
measure, for every $\epsilon>0$, there exists some collection of
half-open intervals $\left\{ \left(a_{i},b_{i}\right]\right\} \in\mathcal{L}$
such that $A\subseteq\bigcup_{i\in\N}\left(a_{i},b_{i}\right]$ and
\[
\lambda\left(A\right)\leq\sum_{i=1}^{\infty}\lambda\left(\left(a_{i},b_{i}\right]\right)=\sum_{i=1}^{\infty}\lambda\left(\left(a_{i},b_{i}\right)\right)<\lambda\left(A\right)+\frac{\epsilon}{2}.
\]
where the second equality is property $\left(ii\right)$ in Theorem
\ref{thm:existenceLebesgueR}.Next, we consider the open sets $\left(a_{i},b_{i}+\frac{\epsilon}{4^{i}}\right)$.
Clearly, $A\subseteq\bigcup_{i\in\N}\left(a_{i},b_{i}+\frac{\epsilon}{4^{i}}\right)$.
Letting $O=\bigcup_{i\in\N}\left(a_{i},b_{i}+\frac{\epsilon}{4^{i}}\right)$,
we see that $O$ is open since the union of open sets is open. Further,
$\lambda\left(O\right)=\frac{\epsilon}{2}+\sum_{i=1}^{\infty}b_{i}-a_{i}<\lambda\left(A\right)+\epsilon$.
Therefore, since $O,E\in\mathcal{C}\left(\lambda^{*}\right)$ and
$\lambda\left(O\right)<\infty$
\[
\lambda\left(O\setminus A\right)=\lambda\left(O\right)-\lambda\left(A\right)<\epsilon.
\]
 Of course, $\lambda$ is $\sigma-$finite and so for any $A$ with
$\lambda\left(A\right)=\infty$ we can find a partition $\left\{ A_{i}\right\} _{i\in\N}$
of $A$ such that each $\lambda\left(A_{i}\right)<\infty$. For each
$A_{i}$, there exists some open $B_{i}$ such that $A_{i}\subseteq B_{i}$
and $\lambda\left(B_{i}\setminus A_{i}\right)<\frac{\epsilon}{2^{i}}$.
Let $B=\bigcup_{i\in\N}B_{i}$ be open. Then
\[
B\setminus A\subseteq\bigcup_{i\in\N}B_{i}\setminus A_{i}
\]
and so 
\[
\lambda\left(B\setminus A\right)\leq\lambda\left(\bigcup_{i\in\N}\left(B_{i}\setminus A_{i}\right)\right)\leq\sum_{i=1}^{\infty}\lambda\left(B_{i}\setminus A_{i}\right)<\epsilon
\]
which completes the proof in one direction.

Conversely, now suppose that $A\subseteq\R$ is a set such that for
any $\epsilon>0$ there exists some open set $O\subseteq\R$ such
that $A\subseteq O$ and $\lambda\left(O\setminus A\right)<\epsilon.$
Then, for any 
\begin{align*}
\lambda^{*}\left(E\cap A\right)+\lambda^{*}\left(E\cap A^{C}\right) & =\lambda^{*}\left(E\cap A\right)+\lambda^{*}\left(E\cap A^{C}\cap O\right)+\lambda^{*}\left(E\cap A^{C}\cap O^{C}\right)\\
 & \leq\lambda^{*}\left(E\cap O\right)+\lambda^{*}\left(E\cap A^{C}\cap O\right)+\lambda^{*}\left(E\cap O^{C}\right)\\
 & \leq\lambda^{*}\left(E\cap O\right)+\lambda^{*}\left(E\cap O^{C}\right)+\epsilon\\
 & =\lambda^{*}\left(E\right)+\epsilon
\end{align*}
where the first equality follows by the fact that $O\in\mathcal{C}\left(\lambda^{*}\right)$,
the second since $A\subseteq O$, the third by our hypothesis, the
fourth again since $O\in\mathcal{C}\left(\lambda^{*}\right)$. Since
$\epsilon$ was arbitrary, we have that
\[
\lambda^{*}\left(E\right)\geq\lambda^{*}\left(E\cap A\right)+\lambda^{*}\left(E\cap A^{C}\right)
\]
and since the other inequality follows by subadditivity, the result
follows.
\end{proof}
Recall from basic topology, that arbitrary unions and finite intersections
of open sets are open. Since closed sets are exactly those sets whose
complements are open, we have that arbitrary intersections and finite
unions of closed sets are closed. Given a topological space $\left(\X,\tau\right)$,
the collection $F_{\sigma}$ are the subsets of $\X$ that are countable
unions of closed sets. Similarly, the collection $G_{\delta}$ consists
of sets that are countable intersections of open sets. We can use
these definitions to formulate more equivalent characterizations of
the Lebesgue $\sigma-$algebra.
\begin{prop}
\label{prop:equivalentLebesgueMeasurability}The following are equivalent
for any $A\subseteq\R$

\begin{enumerate}[label=(\roman*),leftmargin=.1\linewidth,rightmargin=.4\linewidth]
\item $A$ is Lebesgue measurable
\item There exists a set  $G \in G_{\delta}$ such that $A \subseteq G$ and 
\[
	\lambda\left(G\setminus A\right) = 0.
\]
\item There exists a Borel set $B$ such that $A \subseteq B$ and 
\[
	\lambda\left(B\setminus A\right) = 0.
\]
\item For every $\epsilon > 0$ there exists some closed set $F \subseteq A$ such that 
\[
	\lambda\left(A \setminus F\right) < \epsilon
\]
\item There exists a set $F \in F_{\sigma}$ such that $F \subseteq A$ and 
\[
	\lambda\left(A\setminus F\right)= 0
\]
\item There exists a Borel set $B$ such that $B\subseteq A$ and 
\[
	\lambda\left(A\setminus B\right) = 0
\]
\end{enumerate}
\end{prop}

\begin{proof}
$\left[\left(i\right)\implies\left(ii\right)\right]$ First assume
$\left(i\right)$ and observe that by Proposition \ref{prop:equivalentLebesgueMeasurability},
for every $n\in\N$ there exists some open set $O_{n}$ such that
$A\subseteq O_{n}$ and \footnote{Note that $O_{n}\setminus A=O_{n}\cap A^{C}\in\mathcal{C}\left(\lambda^{*}\right)$
since $A,O_{n}\in\mathcal{C}\left(\lambda^{*}\right)$}
\[
\lambda\left(O_{n}\setminus A\right)<\frac{1}{n}.
\]
Then, notice that $G:=\bigcap_{n\in\N}O_{n}\in G_{\delta}\subset\borel\left(\R\right)$
and that $A\subseteq G\subseteq O_{n}$ and so the monotonicity of
measures implies that 
\[
\lambda\left(G\setminus A\right)<\frac{1}{n}
\]
for every $n\in\N$. Our result then follows.

$\left[\left(ii\right)\implies\left(iii\right)\right]$ This implication
is trivial since $G$ above is a Borel set.

$\left[\left(iii\right)\implies\left(i\right)\right]$ This follows
the same argument as in Proposition \ref{prop:equivalentLebesgueMeasurability}
to show that $\lambda\left(B\setminus A\right)=0\implies\lambda^{*}\left(E\right)\geq\lambda^{*}\left(E\cap A\right)+\lambda^{*}\left(E\cap A^{C}\right).$

$\left[\left(i\right)\implies\left(iv\right)\right]$ Fix $\epsilon>0$
and let $A\in\mathcal{C}\left(\lambda^{*}\right)$. Then $A^{C}\in\mathcal{C}\left(\lambda^{*}\right)$
and so by Proposition \ref{prop:equivalentLebesgueMeasurability},
there exists some open $O$ such that $A^{C}\subseteq O$ and 
\[
\lambda\left(O\setminus A^{C}\right)<\epsilon.
\]
Then notice that $O^{C}\subseteq A$ is closed and 
\begin{align*}
\lambda\left(A\setminus O^{C}\right) & =\lambda\left(A\cap O\right)\\
 & =\lambda\left(O\setminus A^{C}\right)\\
 & <\epsilon
\end{align*}
which completes the argument.

$\left[\left(iv\right)\implies\left(v\right)\right]$ This argument
is analagous to $\left[\left(i\right)\implies\left(ii\right)\right]$.
Notice that for every $n\in\N$, there exists some closed $F_{n}\subseteq A$
such that 
\[
\lambda\left(A\setminus F_{n}\right)=\lambda\left(A\cap F_{n}^{C}\right)<\frac{1}{n}.
\]
Let $F:=\bigcup_{n\in\N}F_{n}\subseteq A$ and so 
\[
\lambda\left(A\setminus F\right)=\lambda\left(A\cap\bigcap_{n\in\N}F_{n}\right)<\frac{1}{n}
\]
for every $n\in\N$ by monotonicity. The result then follows.

$\left[\left(v\right)\implies\left(vi\right)\right]$ This is trivial
since $F$ above is a Borel set.

$\left[\left(vi\right)\implies\left(i\right)\right]$ This is analgous
to $\left[\left(iii\right)\implies\left(i\right)\right].$ Let $B\subseteq A$
be a Borel set such that $\lambda\left(A\setminus B\right)=0$. Let
$E\subseteq\R$ be arbitrary
\begin{align*}
\lambda^{*}\left(E\cap A\right)+\lambda^{*}\left(E\cap A^{C}\right) & =\lambda^{*}\left(E\cap A\cap B^{C}\right)+\lambda^{*}\left(E\cap A\cap B\right)+\lambda^{*}\left(E\cap A^{C}\right)\\
 & =\lambda^{*}\left(E\cap A\cap B\right)+\lambda^{*}\left(E\cap A^{C}\right)\\
 & \geq\lambda^{*}\left(E\cap B\right)+\lambda\left(E\cap B^{C}\right)\\
 & =\lambda^{*}\left(E\right)
\end{align*}
where the first equality is because $B\in\mathcal{C}\left(\lambda^{*}\right)$,
second because $\lambda\left(A\cap B^{C}\right)=0$, the inequality
because of monotonicity, and the last equality again due to $B\in\mathcal{C}\left(\lambda^{*}\right)$.
This completes the proof.
\end{proof}
Note that Proposition \ref{prop:vitalitSetNotMeasurable} tells us
that the Vitali sets $\left\{ E_{q}\right\} _{q\in\mathbb{Q}}$ as
described in Example \ref{exa:vitaliSet}are not measurable since
the Lebesgue measure violates countable additivity on such sets. This
can be generalized to show that every set of positive Lebesgue measure
contains Vitali-like subsets that are not measurable.
\begin{thm}
\label{thm:positiveMeasureNonMeasurable}For any set $A\in\mathcal{C}\left(\lambda^{*}\right)$
with $\lambda\left(A\right)>0$ there exists a subset $B\subset A$
such that $B\notin\mathcal{C}\left(\lambda^{*}\right)$.
\end{thm}

\begin{proof}
First suppose that $\lambda\left(A\right)<\infty$ and notice that
$A\subset\left[-b,b\right]$ for some $b>0$ i.e $A$ is bounded.
Then $\left\{ bE_{q}\right\} _{q\in\mathbb{Q}}$and $\left\{ -bE_{q}\right\} _{q\in\mathbb{Q}}$are
a disjoint collection such that $\bigcup_{q\in\mathbb{Q}}bE_{q}\cup-bE_{q}=\left[-b,b\right]$
and so $\bigcup_{q\in\mathbb{Q}}\underbrace{\left(bE_{q}\cup-bE_{q}\right)\cap A}_{=:A_{q}}=A$.
Suppose $\left\{ A_{q}\right\} _{q\in\mathbb{Q}}$ are measurable,
then
\begin{align*}
\lambda\left(A\right) & =\lambda\left(\bigcup_{q\in\mathbb{Q}}A_{q}\right)\\
 & =\sum_{q\in\mathbb{Q}}\lambda\left(A_{q}\right)\\
 & =\sum_{q\in\mathbb{Q}}2b\lambda\left(E_{q}\right)\\
 & =\sum_{q\in\mathbb{Q}}2bc\\
 & =0\ \text{or \ensuremath{\infty}}
\end{align*}
whre the second equality is countable additivity, the third Lemma
\ref{lem:outerMeasureScales} (scaling) and the fourth Lemma \ref{lem:outerMeasureTranslationInvariant}(translation
invariance), which is a contradiction.

If $\lambda\left(A\right)=\infty$ then by $\sigma-$finiteness there
exist disjoint $\left\{ B_{i}\right\} _{i\in\N}\in\mathcal{C}\left(\lambda^{*}\right)$
such that $\lambda\left(B_{i}\right)<\infty$ and $\bigcup_{i\in\N}B_{i}=A$.
Let $B_{i}^{q}$ be akin to $A_{q}$ above and then our result follows.
\end{proof}

\subsection{Cantor sets and functions\label{subsec:cantorSets}}

Our results show that countable sets are measurable and have Lebesgue
measure zero. Indeed, every singleton $\left\{ x\right\} $ is closed
and hence measurable and has measure zero since $\left\{ x\right\} =\left[x,x\right].$
Countable additivity then tells us that every countable set has measure
zero. In particular, the rational numbers have measure zero. Do there
exist uncountable sets that also have measure zero? We can indeed
construct such sets. Perhaps more surprisingly, we can use a similar
construction to show that there exist sets of \emph{positive }Lebesgue
measure, that nevertheless contain no non-trivial intervals within
them.
\begin{defn}
\label{def:cantorSet}Let the set $C\subseteq\left[0,1\right]$ such
that for any $x\in C$
\[
x=\sum_{i=1}^{\infty}\frac{a_{i}}{3^{i}}
\]
where $a_{i}\in\left\{ 0,2\right\} $. We call $C$ is the standard
\emph{Cantor set.}
\end{defn}

That is the Cantor set consists of those elements in $\left[0,1\right]$
whose ternary (base 3) representation does not contain any $1$ in
its digits. A hurdle with this definition is that the representation
is not unique; as with any decimal or binary representation, we have
things like $0.02222222..._{3}=0.1_{3}$ where the subscripts highlight
the fact that we are using ternary representations. A discussion on
base-$b$ representations of real numbers can be found in Appendix
section \ref{subsec:baseBRepresentations}. In this case, we say that
an element $x\in\left[0,1\right]$ is in the Cantor set $C$ if any
one of its representations has coefficients only in $\left\{ 0,2\right\} $.
In other words, real numbers like $0.022222..._{3}\in C$.

An alternative characterization of the Cantor set is to take the set
$C_{1}:=\left[0,1\right]$ and remove the the middle third set $G_{1}=\left(\frac{1}{3},\frac{2}{3}\right)$.
Then from the remaining $C_{2}:=\left[0,1\right]\setminus G_{1}=\left[0,\frac{1}{3}\right]\cup\left[\frac{2}{3},1\right]$
remove the middle third of each interval component; that is, remove
$G_{2}=\left(\frac{1}{9},\frac{2}{9}\right)\cup\left(\frac{7}{9},\frac{8}{9}\right)$and
so on. The recursive description of the sets $\left\{ G_{n}\right\} _{n\in\N}$
can be given as follows
\begin{align*}
G_{1} & =\left(\frac{1}{3},\frac{2}{3}\right)\\
G_{n} & =\frac{G_{n-1}}{3}\cup\left(\frac{G_{n-1}}{3}+\frac{2}{3}\right).
\end{align*}
Symmetrically, we recursively define the sets approximating the Cantor
set $\left\{ C_{n}\right\} _{n\in\N\cup\left\{ 0\right\} }$
\begin{align*}
C_{0} & =\left[0,1\right]\\
C_{n} & =\frac{C_{n-1}}{3}\cup\left(\frac{C_{n-1}}{3}+\frac{2}{3}\right)
\end{align*}

\begin{lem}
\label{lem:cantorCnDecreasing}$C_{n+1}\subseteq C_{n}$ for all $n\in\N\cup\left\{ 0\right\} $.
\end{lem}

\begin{proof}
Note that $\left[1,\frac{1}{3}\right]\cup\left[\frac{2}{3},1\right]=C_{1}\subset C_{0}=\left[0,1\right]$.
Let $f\left(x\right)=\frac{x}{3}$ and let $g\left(x\right)=\frac{x}{3}+\frac{2}{3}$.
Next suppose that $C_{n+1}=f\left[C_{n}\right]\cup g\left[C_{n}\right]\subseteq C_{n}.$
Then $f\left[C_{n+1}\right]\subseteq f\left[C_{n}\right]$ and $g\left[C_{n+1}\right]\subseteq g\left[C_{n}\right]$.
Together, $C_{n+2}=f\left[C_{n+1}\right]\cup g\left[C_{n+1}\right]\subseteq f\left[C_{n}\right]\cup g\left[C_{n}\right]=C_{n+1}$.
This completes the proof.
\end{proof}
\begin{prop}
\label{prop:cantorSetIntersectionCn}$C=\bigcap_{n\in\N}C_{n}$.
\end{prop}

\begin{proof}
We will show that for any $x\in\left[0,1\right]$ where write $x=\sum_{i=1}^{\infty}\frac{a_{i}}{3^{i}}$in
its base-$3$ representation with $a_{i}\in\left\{ 0,1,2\right\} $,
the following claims are equivalent $\left(1\right)$ For all $1\leq i\leq n$
$a_{i}\neq1$ $\left(2\right)$ $x\in C_{n}$. We certainly know that
$a_{1}\neq1$ if and only if $x\in C_{1}=\left[0,\frac{1}{3}\right]\cup\left[\frac{1}{3},1\right].$
Now assume that that the claims are equivalent for some fixed $n$
and suppose that for $1\leq i\leq n+1:a_{i}\neq1.$ By the induction
hypothesis, $x\in C_{n}$ and so if $a_{1}=0$ then $3x\in\left[0,1\right]$
is such that its first $n$ digits after the delimter are 0 or $2$
and so $3x\in C_{n}$ (again by the hypothesis) and so $x\in\frac{C_{n}}{3}.$
If $a_{1}=2$ then $3\left(x-\frac{2}{3}\right)\in C_{n}$ by the
hypothesis and so $x\in\frac{C_{n}}{3}+\frac{2}{3}$. Therefore, $x\in\frac{C_{n}}{3}\cup\left(\frac{C_{n}}{3}+\frac{2}{3}\right)=C_{n+1}$
which completes one implication.

Conversely, suppose that $x\in C_{n+1}$. Since $C_{n+1}\subseteq C_{n}$,
our hypothesis tells us that $a_{i}\neq1$ for $1\leq i\leq n$. Now
$x\in\frac{C_{n}}{3}$ or $x\in\frac{C_{n}}{3}+\frac{2}{3}$. Clearly,
if $a_{n+1}=1$ then $3x\notin C_{n}$ and $3\left(x-\frac{2}{3}\right)\notin C_{n}$
and the proof is complete.
\end{proof}
\begin{prop}
\label{prop:CantorSetEquiv}For every $n\in\N$ $C_{n}=\left[0,1\right]\setminus\bigcup_{i=1}^{n}G_{i}$.
\end{prop}

\begin{proof}
For $n=1$ the claim is obvious. Suppose the claim holds for $n$
i.e. $C_{n}=\left[0,1\right]\setminus\bigcup_{i=1}^{n}G_{i}$ and
note that 
\begin{align*}
C_{n+1} & =\frac{C_{n}}{3}\cup\left(\frac{C_{n}}{3}+\frac{2}{3}\right)\\
 & =\frac{1}{3}\left(\left[0,1\right]\setminus\bigcup_{i=1}^{n}G_{i}\right)\cup\left(\frac{1}{3}\left(\left[0,1\right]\setminus\bigcup_{i=1}^{n}G_{i}\right)+\frac{2}{3}\right)\\
 & =\left(\left[0,\frac{1}{3}\right]\cup\left[\frac{2}{3},1\right]\right)\setminus\bigcup_{i=1}^{n}\left(\frac{G_{i}}{3}\cup\left(\frac{G_{i}}{3}\cup\frac{2}{3}\right)\right)\\
 & =C_{1}\setminus\bigcup_{i=1}^{n}\left(\frac{G_{i}}{3}\cup\left(\frac{G_{i}}{3}\cup\frac{2}{3}\right)\right)\\
 & =\left[0,1\right]\setminus G_{1}\setminus\bigcup_{i=2}^{n+1}G_{i}\\
 & =\left[0,1\right]\setminus\bigcup_{i=1}^{n+1}G_{i}
\end{align*}
which completes the proof.
\end{proof}
\begin{prop}
\label{prop:cantorSetClosed}$C$ is closed under the usual topology
of $\R$.
\end{prop}

\begin{proof}
Note that $C_{0}=\left[0,1\right]$ is closed and $C_{n}=\frac{C_{n-1}}{3}\cup\left(\frac{C_{n-1}}{3}+\frac{2}{3}\right)$
is closed if $C_{n-1}$ is closed. Therefore, $C_{n}$ is closed for
all $n\in\N$ and so the intersection is closed as well.
\end{proof}
\begin{prop}
\label{prop:cantorSetMeasureZero}$C\in\borel\left(\R\right)$ and
$\lambda\left(C\right)=0$.
\end{prop}

\begin{proof}
Since $C$ is closed its in $\borel\left(\R\right).$ We first prove
that $\lambda\left(C_{n}\right)\leq\left(\frac{2}{3}\right)^{n}$
for any $n\in\N$. Note that for $n=0$ this is obvious, for $n=1$
we have that 
\begin{align*}
\lambda\left(C_{1}\right) & =\lambda\left(\left[0,\frac{1}{3}\right]\cup\left[\frac{2}{3},1\right]\right)\\
 & =\lambda\left(\left[0,\frac{1}{3}\right]\right)+\lambda\left(\left[\frac{2}{3},1\right]\right)\\
 & =\frac{2}{3}.
\end{align*}
Now suppose that $\lambda\left(C_{n}\right)\leq\left(\frac{2}{3}\right)^{n}$
and observe that 
\begin{align*}
\lambda\left(C_{n+1}\right) & \leq\lambda\left(\frac{C_{n+1}}{3}\right)+\lambda\left(\frac{C_{n}}{3}+\frac{2}{3}\right)\\
 & =\frac{1}{3}\left(\frac{2}{3}\right)^{n}+\frac{1}{3}\left(\frac{2}{3}\right)^{n}\\
 & =\left(\frac{2}{3}\right)^{n+1}
\end{align*}
where the inequality is by subadditivity, and the first equality by
the properties (translation invariance and scaling) outlined in Theorem
\ref{thm:existenceLebesgueR}. Then, by Proposition\ref{prop:equivalenceContinuityMeasures}
\begin{align*}
\lambda\left(C\right) & =\lambda\left(\bigcap_{n\in\N}C_{n}\right)\\
 & =\lim_{n\to\infty}\lambda\left(C_{n}\right)\\
 & \leq\lim_{n\to\infty}\left(\frac{2}{3}\right)^{n}\\
 & =0.
\end{align*}
\end{proof}
\begin{prop}
\label{prop:cantorSetUncountable}The Cantor set $C$ is uncountable.
\end{prop}

\begin{proof}
Because the Cantor set can be described as the collection of all ternary
strings with only 0s and 2s, we know by Cantor's diagonal argument
that such a collection is not countable.
\end{proof}
\begin{prop}
\label{prop:cantorSetDisconnected}The Cantor set contains no non-trivial
(i.e. non-singleton) intervals.
\end{prop}

\begin{proof}
$I\subset C$ is a non-trivial interval then $\lambda\left(I\right)>0$
which would imply, by monotonicity, that $\lambda\left(C\right)>0$
which contradicts Proposition\ref{prop:cantorSetMeasureZero}.
\end{proof}
Another way to establish the uncountability of $C$ is to examine
the behavior of the following function.
\begin{defn}
\label{def:cantorFunction}Let $\psi_{C}:\left[0,1\right]\to\left[0,1\right]$
be given as
\[
\psi_{C}\left(x\right)=\begin{cases}
\sum_{i=1}^{\infty}\frac{a_{i}}{2^{i}}, & x=\sum_{i=1}^{\infty}\frac{2a_{i}}{3^{i}},a_{i}\in\left\{ 0,1\right\} \\
\sup_{y\leq x,y\in C}\psi_{C}\left(y\right) & x\in\left[0,1\right]\setminus C
\end{cases}.
\]
$\psi_{C}$ is called the \emph{Cantor function.}
\end{defn}

Note that the Cantor function $f$ is well defined since every member
of the Cantor set has a \emph{unique }base 3 representation where
coefficients are only $0$ and $2$. Here's a practical way to think
about the Cantor function. First, for any $x\in C$, $\psi_{C}\left(x\right)$
is computed from the unique base$-3$ representation (i.e. the one
with only 0s and 2s) by turning all the 2s into 1s and then interpreting
the resulting string as a binary number. For $x\notin C,$ $\psi_{C}\left(x\right)$
is computed by taking the base-$3$representation of $x$, truncating
after the first $1$ and then replacing each $2$ before the first
1 with 1s and then interpreting the result as a binary number.
\begin{prop}
\label{prop:imageCantorSetUnderCantorFunctionFull}$\psi_{C}\left[C\right]=\left[0,1\right].$
\end{prop}

\begin{proof}
Let $y\in\left[0,1\right]$ be arbitrary. Then $y$ has a binary representation
$y=\sum_{i=1}^{\infty}\frac{a_{i}}{2^{i}}$ where $a_{i}\in\left\{ 0,1\right\} $.
We take the terminating expansion to stave off questions about uniqueness
(i.e. the expansion that doesn't end in all 1s). Then, 
\[
x=\sum_{i=1}^{\infty}\frac{2a_{i}}{3^{i}}
\]
is the unique element in the Cantor set such that $\psi_{C}\left(x\right)=y$
which completes the proof.
\end{proof}
This yields yet another proof that $C$ is uncountable, since $\left[0,1\right]$
is uncountable. \hl{Expand}
\begin{prop}
\label{prop:cantorFunctionIncreasing}The Cantor function is non-decreasing
i.e for any $x,y\in\left[0,1\right]$ such that $x\geq y$
\[
\psi_{C}\left(x\right)\geq\psi_{C}\left(y\right).
\]
\end{prop}

\begin{proof}
First suppose both $x,y\in C$ and $x\geq y$. Write the unique expansions
$x=\sum_{i=1}^{\infty}\frac{a_{i}}{3^{i}}$ and $y=\sum_{i=1}^{\infty}\frac{b_{i}}{3^{i}}$
and notice that $a_{i}\geq b_{i}$ for all $i\in\N$ and so clearly
$\psi_{C}\left(x\right)\geq\psi_{C}\left(y\right)$. The case when
both $x,y\notin C$ is trivial by the first part and the definition
of the supremum. Now suppose that $x\in C$ and $y\notin C.$ If $x\geq y$
then then clearly $\psi_{C}\left(x\right)\geq\psi_{C}\left(y\right)$
by the definition and the first part of the proof. A similar argument
holds if $y\geq x$. This completes the proof.
\end{proof}
%
\begin{prop}
\label{prop:cantorFunctionContinuous}The Cantor function $\psi_{C}$
is continuous on $\left[0,1\right]$.
\end{prop}

\begin{proof}
Suppose that $\psi_{C}$ is discontinuous at $c\in\left[0,1\right]$.
Since $\psi_{C}$ is increasing, we have that the limits $f\left(c^{-}\right)$
and $f\left(c^{+}\right)$ exist and are in $\left[0,1\right]$ since
it is closed. Moreover, $f\left(c^{-}\right)<f\left(c^{+}\right)$
which implies that there is some real number $y\in\left[0,1\right]$such
that $f\left(c^{-}\right)<y<f\left(c^{+}\right).$ But then there's
no $x\in\left[0,1\right]$ such that $f\left(x\right)=y$ which is
a contradiction.
\end{proof}
Note that that since $\psi_{C}$ is a non-decreasing and continuous
function on $\left[0,1\right]$, it can be extended to a Stieljes
function in an obvious way, define 
\[
F_{\psi_{C}}\left(x\right)=\begin{cases}
1, & x>1\\
\psi_{c}\left(x\right), & x\in\left[0,1\right]\\
0, & x<0
\end{cases}
\]
The function $F_{\psi_{C}}$ is still non-decreasing and continuous
and so there it induces a Stieljes measure on $\borel\left(\R\right)$
by Theorem \ref{thm:stieljesMeasure}. This measure, is in some sense
\emph{orthogonal }to the Lebesgue measure, a notion that we will make
precise in Chapter \ref{chap:Differentiation}. This idea is intimately
connected to the fact that while $\psi_{C}$ is differentiable almost
everywhere (the meaning of this phrase will be made precise later),
it is not the integral of its derivative, a fact which we shall also
establish in that chapter.

The Cantor function is symmetric, a fact that is important in establishing
moments of the Cantor distribution in probability theory. For now,
we use this fact to compute the integral of this function.
\begin{prop}
\label{prop:cantorFunctionSymmetry}For any $x\in\left[0,1\right]$,
$\psi_{C}\left(1-x\right)=1-\psi_{C}\left(x\right)$.
\end{prop}

\begin{proof}
First suppose that $x\in C$ in which case $1-x\in C$ . To see this,
write $x=\sum_{i=1}^{\infty}\frac{2a_{i}}{3^{i}}$ where $a_{i}\in\left\{ 0,1\right\} $
and then $1-x=\sum_{i=1}^{\infty}\frac{\left(2-2a_{i}\right)}{3^{i}}$
where $2-2a_{i}$ is $0$ if $a_{i}=1$ and $2$ if $a_{i}=0$. Then,
$\psi_{C}\left(1-x\right)=\sum_{i=1}^{\infty}\frac{1-a_{i}}{2^{i}}=\sum_{i=1}^{\infty}\frac{1}{2^{i}}-\sum_{i=1}^{\infty}\frac{a_{i}}{2^{i}}=1-\psi_{C}\left(x\right).$
Similarly, let $x\in\left[0,1\right]\setminus C$ and observe that
we can write $x=\sum_{i=1}^{\infty}\frac{a_{i}}{3^{i}}$ where $a_{i}\in\left\{ 0,1,2\right\} $.
Moreover, $n=\inf\left\{ i\mid a_{i}=1\right\} <\infty$ and so 
\begin{align*}
\psi_{C}\left(x\right) & =\psi_{C}\left(\sum_{i=1}^{\infty}\frac{a_{i}}{3^{i}}\right)\\
 & =\psi_{C}\left(\sum_{i=1}^{n-1}\frac{a_{i}}{3^{i}}+\sum_{i=n+1}^{\infty}\frac{2}{3^{i}}\right)\\
 & =\sum_{i=1}^{n-1}\frac{b_{i}}{2^{i}}+\sum_{i=n+1}^{\infty}\frac{1}{2^{i}}\\
 & =\sum_{i=1}^{n-1}\frac{b_{i}}{2^{i}}+\frac{1}{2^{n}}
\end{align*}
where $b_{i}=\frac{a_{i}}{2}\in\left\{ 0,1\right\} .$ Note that the
third equality follows by \hl{fill}. Now similarly, $1-x\in\left[0,1\right]\setminus C$
with $1-x=\sum_{i=1}^{\infty}\frac{\left(2-a_{i}\right)}{3^{i}}$
and $2-a_{n}=1$ is the first occurrence of 1 in its ternary expansion.
Then
\begin{align*}
\psi_{C}\left(1-x\right) & =\psi_{C}\left(\sum_{i=1}^{\infty}\frac{\left(2-a_{i}\right)}{3^{i}}\right)\\
 & =\psi_{C}\left(\sum_{i=1}^{n-1}\frac{\left(2-a_{i}\right)}{3^{i}}+\sum_{i=n+1}^{\infty}\frac{2}{3^{i}}\right)\\
 & =\sum_{i=1}^{n-1}\frac{1-b_{i}}{2^{i}}+\sum_{i=n+1}^{\infty}\frac{1}{2^{i}}\\
 & =\sum_{i=1}^{n-1}\frac{1}{2^{i}}+2\sum_{i=n+1}^{\infty}\frac{1}{2^{i}}-\left(\sum_{i=1}^{n-1}\frac{b_{i}}{2^{i}}+\sum_{i=n+1}^{\infty}\frac{1}{2^{i}}\right)\\
 & =\sum_{i=1}^{\infty}\frac{1}{2^{i}}-\left(\sum_{i=1}^{n-1}\frac{b_{i}}{2^{i}}+\frac{1}{2^{n}}\right)\\
 & =1-\psi_{C}\left(x\right)
\end{align*}
which completes the proof.
\end{proof}
A corollary of this result is that the integral of the Cantor function
over the unit interval is $\frac{1}{2}.$
\begin{cor}
\label{cor:cantorIntegral}The Cantor function $\psi_{C}$ is Riemann
integrable on $\left[0,1\right]$ with integral
\[
\int_{0}^{1}\psi_{C}\left(x\right)dx=\frac{1}{2}.
\]
\end{cor}

\begin{proof}
The function is Riemann integrable since it is continuous (we provide
an exact characterization of Riemann integrable functions in Theorem
\ref{thm:riemannIntegrableAEContinuous}, when we review the Riemann
integral). Note that 
\begin{align*}
\int_{0}^{1}\psi_{C}\left(1-x\right)dx & =-\int_{1}^{0}\psi_{C}\left(u\right)du\\
 & =\int_{0}^{1}\psi_{C}\left(u\right)du
\end{align*}
where we have used the $u-$substitution method from elementary calculus.
\hl{add reference to later chapters}. But since $\psi_{C}\left(1-x\right)=1-\psi_{C}\left(x\right)$
and the integral is linear, we have
\[
1=\int_{0}^{1}1dx=2\int_{0}^{1}\psi_{C}\left(x\right)dx
\]
 and the result follows.
\end{proof}
The Cantor function also has the remarkable property that it is actually
constant on each interval component of the complement of the Cantor
set in $\left[0,1\right].$This is particularly striking since then
it's only strictly increasing inside the Cantor set, which is a set
of measure zero, and yet manages to hit every point in $\left[0,1\right].$
\begin{prop}
\label{prop:cantorFunctionConstantOutside}The Cantor function is
constant on every interval component of $\left[0,1\right]\setminus C.$
\end{prop}

\begin{proof}
First note that since $\bigcup_{i=1}^{\infty}G_{i}=\left[0,1\right]\setminus C$
is open since each set $G_{i}$ is open. Moreover, by Lemma \ref{lem:openSetDisjointUnionInterval},
there are countably many disjoint open intervals $\left\{ O_{j}\right\} _{j\in\N}\subseteq\left[0,1\right]$such
that $\bigcup_{i=1}^{\infty}G_{i}=\bigcup_{j=1}^{\infty}O_{j}$. Let
$x\in\bigcup G_{i}$ and notice that then $x\in O_{j}$ for some $j.$
Define $y_{0}=\sup\left\{ y\in C\mid y\leq x\right\} $. Note that
$y_{0}\in C$ since the sets $C$ and $\left\{ y\in\left[0,1\right]\mid y\leq x\right\} $
are both closed. For any $z\in O_{j}$, $y_{0}<z$ since otherwise
$\min\left\{ z,x\right\} \leq y_{0}\leq\max\left\{ z,x\right\} $
would mean that $O_{j}$ is not contained in $\left[0,1\right]\setminus C$.
Next, suppose that there's some $y_{1}\in C$ such that $y_{0}<y_{1}<z$.
Clearly $y_{1}>x$ (since otherwise $y_{1}\in\left\{ y\in C\mid y\leq x\right\} $
and so $y_{1}\leq y_{0}$) which again means that $z>y_{1}>x$ and
so $O_{j}\cap C\neq\emptyset$, which is a contradiction. Therefore,
it must be that $y_{0}=\sup\left\{ y\in C\mid y\leq z\right\} $ and
so for any $z\in O_{j}$, $\psi_{C}\left(z\right)=\psi_{C}\left(y_{0}\right)$,
yielding the result.
\end{proof}

\subsubsection{Fat cantor sets}
