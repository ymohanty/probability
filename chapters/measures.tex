
\chapter{Measures}

\section{Why is measurement hard?}

On the real line $\mathds{R}$, we may want our measure to satisfy
some properties that are consistent with our intuitive notion of ``length''.
Formally, we want a function
\[
\lambda:2^{\mathds{R}}\longrightarrow\left[0,\infty\right]
\]
that satisfies
\begin{enumerate}
\item $\lambda\left(\emptyset\right)=0$
\item $\lambda\left(\left[a,b\right]\right)=b-a$ for $a\leq b\in\mathds{R}$
\item Countable additivity: For a countable collection of pairwise-disjoint
sets $\left\{ A_{i}\right\} _{i\in\mathbb{N}}\subseteq\mathds{R}$
\begin{equation}
\lambda\left(\bigcup_{i\in\mathbb{N}}A_{i}\right)=\sum_{i\in\mathbb{N}}\lambda\left(A_{i}\right)\label{eq:countableAdditivity}
\end{equation}
\item Translation invariance: $\lambda\left(A+a\right)=\lambda\left(A\right)$
for any $a\in\mathds{R}$ where $A+a:=\left\{ \alpha+a\mid\alpha\in A\right\} $.
\end{enumerate}
Quite counterintuitively, it turns out that no such function exists!
To prove this assertion, we need to construct some special kinds of
sets that only exist if we assume the Axiom of Choice.
\begin{example}
\label{exa:vitaliSet} Define an equivalence relation $\sim$ on $\left[0,1\right]$
such that
\[
x\sim y\Leftrightarrow x-y\in\mathbb{Q}.
\]
Note that there are uncountably many classes in such a construction
as the equivalence class for any given irrational number can contain
at most countably many other irrational numbers. For example,
\[
\left[\frac{\pi}{4}\right]=\left\{ \frac{\pi}{4}+q\mod1\mid q\in\mathbb{Q}\right\} .
\]
Thus, using the Axiom of Choice, we can construct a set $E\subseteq\left[0,1\right]$
such that $E$ consists of exactly one ``representative'' from each
equivalence class. Next, we can define
\[
E_{q}:=\left\{ x+q\mod1\mid x\in E\right\} 
\]
so that $\left\{ E_{q}\right\} _{q\in\mathbb{Q}}$ is a partition
of $\left[0,1\right]$. To see that the sets are disjoint, suppose
for contradiction that for any distinct $q,\tilde{q}\in\mathbb{Q}\cap\left[0,1\right]$,
$E_{q}\cap E_{\tilde{q}}\neq\emptyset$. If $x\in E_{q}\cap E_{\tilde{q}}$,
then $x-q\in E$ and $x-\tilde{q}\in E$. But they clearly belong
to the same equivalence class and this is a contradiction given our
construction of $E.$ To see that the union of these sets is $\left[0,1\right]$,
consider an arbitrary $y\in\left[0,1\right]$ and observe that since
our equivalence relation $\sim$ partitions $\left[0,1\right]$, $y\in\left[x\right]$
for some $x\in E$. Then $q^{*}=y-x\in\mathbb{Q}$ and so $y=x+q^{*}\mod1\in E_{q^{*}}$.
Thus we have that
\[
\left[0,1\right]\subseteq\bigcup_{q\in\mathbb{Q}}E_{q}.
\]
Since the reverse inclusion follows by the definition of $E_{q}$,
we have that $\left\{ E_{q}\right\} _{q\in\mathbb{Q}}$ is a partition
of $\left[0,1\right]$.
\end{example}

\begin{prop}
\label{prop:vitalitSetNotMeasurable}There exists no function $\lambda:2^{\mathds{R}}\longrightarrow\left[0,\infty\right]$
that satisfies properties (1)-(4) described above
\end{prop}

\begin{proof}
Suppose, for contradiction, that such a function $\lambda$ exists.
We can define the collection of sets $\left\{ E\right\} _{q\in\mathbb{Q}}$
as in Example \ref{exa:vitaliSet} and observe that
\begin{align*}
1=\lambda\left(\left[0,1\right]\right) & =\lambda\left[\bigcup_{q\in\mathbb{Q}}E_{q}\right]\\
 & =\sum_{q\in\mathbb{Q}}\lambda\left[E_{q}\right]\\
 & =\sum_{q\in\mathbb{Q}}c
\end{align*}
where the first equality follows from property (2), the second equality
follows from the fact that $\left\{ E\right\} _{q\in\mathbb{Q}}$
is a partition of $\left[0,1\right]$, the third equality is due to
property (3). The last equality follows as a consequence of translation
invariance (property (4)). Since $c\in\left[0,1\right]$
\[
\sum_{q\in\mathbb{Q}}c=0\text{ or }\infty\neq1
\]
which is a contradiction. Thus no such function $\lambda$ exists.
\end{proof}
This particular example of a \emph{non-measurable }set is called a
\emph{Vitali set. }While we used the interval $\left[0,1\right]$
to construct such a set, it turns out that this contruction can be
extended to any set of positive length in the Lebesgue sense.

\section{Constructing measures on $\sigma-$algebras}

The key issue with our previous definition of a measure on $\mathds{R}$
is that one cannot have a set-valued function that both has our four
desired properties \emph{and }is defined on all subsets of the real
line. As a convention, the canonical construction of a measure retains
the desired properties in exchange for restricting the class of subsets
on which the measure is defined. These subsets are called $measurable$
and the standard construction of the Lebesgue measure leads to the
class of measurable subsets on the real line to have a special structure
of a \emph{$\sigma-$algebra. }Before we define this structure it
might be worthwhile looking at various types of structures a class
of sets could have

\subsection{Structures of sets}

In the rest of this chapter, we assume that $\left(\mathcal{X},\tau\right)$
is an abstract topological space.
\begin{defn}
\label{def:ring}Let $\mathcal{F}\subseteq2^{\mathcal{X}}$. We call
$\mathcal{F}$ a \emph{ring }if

\begin{enumerate}[label=(\roman*),leftmargin=.1\linewidth,rightmargin=.4\linewidth]
	\item $\emptyset \in \mathcal{F}$
	\item $A,B \in \mathcal{F} \Rightarrow A\cup B \in \mathcal{F}$
	\item $A,B \in \mathcal{F} \Rightarrow A\setminus B \in \mathcal{F}$.
\end{enumerate}
\end{defn}

\noindent Note that the above definition implies that $A\cap B=A\setminus\left(A\setminus B\right)\in\mathcal{F}$.
\begin{defn}
\label{def:algebra}Let $\mathcal{F}\subseteq2^{\mathcal{X}}$. We
call $\mathcal{F}$ an \emph{algebra }if

\begin{enumerate}[label=(\roman*),leftmargin=.1\linewidth,rightmargin=.4\linewidth]
	\item $\mathcal{F}$ is a ring
	\item $\mathcal{X} \in \mathcal{F}$.
\end{enumerate}
\end{defn}

\noindent For example, if we let $\mathcal{X}$ be an arbitrary infinite
set, the collection of all finite subsets of $\mathcal{X}$ forms
a ring but not an algebra.
\begin{defn}
\label{def:sigmaRing}Let $\mathcal{F}\subseteq2^{\mathcal{X}}$.
We call $\mathcal{F}$ a $\sigma$-ring if

\begin{enumerate}[label=(\roman*),leftmargin=.1\linewidth,rightmargin=.4\linewidth]
	\item $\mathcal{F}$ is a ring
	\item $\mathcal{F}$ is closed under countable unions.
\end{enumerate}
\end{defn}

\begin{defn}
\label{def:sigmaAlgebra}Let $\mathcal{F}\subseteq2^{\mathcal{X}}$.
We call $\mathcal{F}$ a $\sigma$-algebra if

\begin{enumerate}[label=(\roman*),leftmargin=.1\linewidth,rightmargin=.4\linewidth]
	\item $\mathcal{F}$ is an algebra.
	\item $\mathcal{F}$ is closed under countable unions.
\end{enumerate}
\end{defn}

\noindent Naturally, the power set $2^{\mathcal{X}}$ is a ring,
algebra, $\sigma$-ring, and $\sigma$-algebra all rolled into one.
\begin{rem*}
Algebras are sometimes referred to as \emph{fields} in the probability
literature.
\end{rem*}
As we said earlier, the notion of a $\sigma$-algebra is important
because the standard Lebesgue measurable sets form a $\sigma$-algebra
of subsets of $\mathds{R}$. However, the other structures we have
defined are also important; as we ``extend'' the notion of the length
of an interval on the real line to more complicated sets, we shall
first expand our class of measurable sets to a ring of sets.

\subsection{Lengths of intervals}

The mosts intuitive notion of a measure on $\mathds{R}$ arises from
the length of an interval. Thus, in our construction of the Lebesgue
measure, we start with the simplest class of sets which consists of
intervals in $\mathds{R}.$ Define $\mathcal{L}=\left\{ \left(a,b\right]\mid-\infty<a\leq b<\infty\right\} $
and let $\lambda_{1}:\mathcal{L}\longrightarrow\left[0,\infty\right]$
be given by $\lambda_{1}\left(\left(a,b\right]\right)=b-a$. It turns
out that our collection of half-open intervals in $\mathds{R}$ has
the structure of a \emph{semi-ring.}
\begin{defn}
\label{def:semiRing} Let $\mathcal{F}\subseteq2^{\mathcal{X}}$.
We call $\mathcal{F}$ a \emph{semi-ring }if

\begin{enumerate}[label=(\roman*),leftmargin=.1\linewidth,rightmargin=.4\linewidth]
	\item $\emptyset \in \mathcal{F}$.
	\item $A,B \in \mathcal{F} \Rightarrow A\cap B \in \mathcal{F}$.
	\item $A,B \in \mathcal{F} \Rightarrow \exists \left\{A_i\right\}_{i=1}^{n} \in \mathcal{F}$ such that $ A_i \cap A_j = \emptyset$ for $i \neq j$ and
	\[
		A \setminus B = \bigcup_{i=1}^{n}A_i 
	\]
\end{enumerate}
\end{defn}

\begin{prop}
\label{prop:intervalSigmaRing}$\mathcal{L}$ is a semi-ring.
\end{prop}

\begin{proof}
To see (i), note that $\emptyset=\left(a,a\right]\in\mathcal{L}$.
For (ii), note that for any intervals $A=\left(a_{1},b_{1}\right]$
and $B=\left(a_{2},b_{2}\right]$, \footnote{If $\max\left(a_{1},a_{2}\right)>\min\left(b_{1},b_{2}\right)$, then
$\left(\max\left(a_{1},a_{2}\right),\min\left(b_{1},b_{2}\right)\right]=\emptyset\in\mathcal{L}$}
\[
\left(a_{1},b_{1}\right]\cap\left(a_{2},b_{2}\right]=\left(\max\left(a_{1},a_{2}\right),\min\left(b_{1},b_{2}\right)\right]\in\mathcal{L}.
\]
To see (iii), we have to consider two possible cases. First, if $A,B$
are disjoint then $A\setminus B=A\in\mathcal{L}.$ If $A,B$ have
a non-trivial intersection, then
\begin{align*}
A\setminus B=A\cap B^{C} & =\left(a_{1},b_{1}\right]\cap\left\{ \left(-\infty,a_{2}\right]\cup\left(b_{2},\infty\right)\right\} \\
 & =\left(a_{1},b_{1}\right]\cap\left(-\infty,a_{2}\right]\bigcup\left(a_{1},b_{1}\right]\cap\left(b_{2},\infty\right)\\
 & =\left(a_{1},\min\left(b_{1},a_{2}\right)\right]\bigcup\left(\max\left(a_{1},b_{2}\right),b_{1}\right]
\end{align*}
where the components of the union expressed in the last equality are
in $\mathcal{L}$. This completes the proof.
\end{proof}
The fact that $\mathcal{L}$ is a semi-ring is important because there's
a relatively straightforward way to ``expand'' a semi-ring into
a ring.
\begin{thm}
\label{thm:expandSemiRing}Let $\mathcal{F}$ be a semi-ring and let
$\mathcal{B}$ be the set of all finite disjoint unions of sets in
$\mathcal{F}$. Then $\mathcal{B}$ is a ring.
\end{thm}

\begin{proof}
Property (i) in Definition \ref{def:ring} is trivially satisfied
thus we need to prove properties (ii) and (iii). Let $A,B\in\mathcal{B}$.
To prove property (iii), we first establish the weaker claim that
$A\cap B\in\mathcal{B}$. Observe that
\begin{align*}
A & =\bigcup_{i=1}^{n_{A}}A_{i},\ A_{i}\in\mathcal{F},A_{i}\cap A_{j}=\emptyset\mathrm{\ for\ }i\neq j,\\
B & =\bigcup_{i=1}^{n_{B}}B_{i},\ B_{i}\in\mathcal{F},B_{i}\cap B_{j}=\emptyset\mathrm{\ for\ }i\neq j,
\end{align*}
by the definition of $\mathcal{B}.$ Then
\begin{align*}
A\cap B & =\left(\bigcup_{i=1}^{n_{A}}A_{i}\right)\bigcap\left(\bigcup_{j=1}^{n_{B}}B_{j}\right)\\
 & =\bigcup_{i=1}^{n_{A}}\bigcup_{j=1}^{n_{B}}\left(A_{i}\cap B_{j}\right)
\end{align*}
where $\forall i,j:\ A_{i}\cap B_{j}\in\mathcal{F}$ as $\mathcal{F}$
is a semi-ring. Clearly, $A_{i}\cap B_{j}$ is disjoint from $A_{i^{\prime}}\cap B_{j^{\prime}}$,
thus proving the claim. Next, we establish property (iii) by noting
that
\begin{align*}
A\setminus B & =\left(\bigcup_{i=1}^{n_{A}}A_{i}\right)\setminus B\\
 & =\left(\bigcup_{i=1}^{n_{A}}A_{i}\right)\bigcap B^{C}\\
 & =\bigcup_{i=1}^{n_{A}}\left(A_{i}\cap B^{C}\right)\\
 & =\bigcup_{i=1}^{n_{A}}\left(A_{i}\cap\left(\bigcap_{j=1}^{n_{B}}B_{j}^{C}\right)\right)\\
 & =\bigcup_{i=1}^{n_{A}}\bigcap_{j=1}^{n_{B}}\left(A_{i}\cap B_{j}^{C}\right)\\
 & =\bigcup_{i=1}^{n_{A}}\bigcap_{j=1}^{n_{B}}A_{i}\setminus B_{j}
\end{align*}
where the $A_{i}\setminus B_{j}\in\mathcal{B}$ since $A_{i},B_{j}\in\mathcal{F}$.
By the closure under finite intersections property established earlier,
$E_{i}=\bigcap_{j=1}^{n_{B}}A_{i}\setminus B_{j}\in\mathcal{B}$ for
any $1\leq i\leq n_{A}$. Thus we can rewrite the chain of equalities
above as
\[
A\setminus B=\bigcup_{i=1}^{n_{A}}E_{i}
\]
where $E_{i}\cap E_{i^{\prime}}=\emptyset$ because $A_{i}\cap A_{i^{\prime}}=\emptyset.$
Since the finite disjoint union of elements of $\mathcal{B}$ is also
a finite disjoint union of elements of $\mathcal{F}$, our claim follows.
Finally, to establish property (ii), observe that
\[
A\cup B=\left(A\setminus B\right)\cup\left(A\cap B\right)\cup\left(B/A\right)
\]
which is a disjoint union of elements in $\mathcal{B}$ and so is
also in $\mathcal{B}$ by the same argument as earlier.
\end{proof}
\begin{cor}
Let $\mathcal{J}$ be the set of all finite disjoint unions of sets
in $\mathcal{L}$. Then $\mathcal{J}$ is a ring.
\end{cor}

\begin{proof}
By Proposition \propref{intervalSigmaRing}, $\mathcal{L}$ is a semi-ring.
The claim then follows by an application of Theorem \thmref{expandSemiRing}.
\end{proof}
Now we can extend our proto-measure $\lambda_{1}$ to a new proto-measure
$\lambda_{2}:\mathcal{J}\longrightarrow\left[0,\infty\right]$ as
follows:
\[
\lambda_{2}\left(A\right):=\begin{cases}
\lambda_{1}\left(A\right), & A\in\mathcal{L}\\
\sum_{i=1}^{n}\lambda_{1}\left(B_{i}\right), & A=\bigcup_{i=1}^{n}B_{i},\left\{ B_{i}\right\} _{i=1}^{n}\text{ are disjoint in }\mathcal{L}
\end{cases}
\]

\begin{prop}
\label{prop:ringMeasureFinitelyAdditive}$\lambda_{2}$ is fiinitely
additive on $\mathcal{J}$. That is, for any finite disjoint collection
of sets $\left\{ A_{i}\right\} _{i=1}^{n}\in\mathcal{J}$
\[
\lambda_{2}\left(\bigcup_{i=1}^{n}A_{i}\right)=\sum_{i=1}^{n}\lambda_{2}\left(A_{i}\right).
\]
\end{prop}

\begin{proof}
For clarity, we will prove finite additivity for two sets , since
the general case follows by induction. Let $A,B\in\mathcal{J}$ such
that $A\cap B=\emptyset$. By definition,
\begin{align*}
A & =\bigcup_{i=1}^{n_{A}}A_{i},\left\{ A_{i}\right\} _{i=1}^{n_{A}}\text{ are disjoint in }\mathcal{L}\\
B & =\bigcup_{i=1}^{n_{B}}B_{i},\left\{ B_{i}\right\} _{i=1}^{n_{B}}\text{ are disjoint in }\mathcal{L}
\end{align*}
and so we have that
\begin{align*}
\lambda_{2}\left(A\cup B\right) & =\lambda_{2}\left(\left(\bigcup_{i=1}^{n_{A}}A_{i}\right)\cup\left(\bigcup_{i=1}^{n_{B}}B_{i}\right)\right)\\
 & =\sum_{i=1}^{n_{A}}\lambda_{2}\left(A_{i}\right)+\sum_{i=1}^{n_{B}}\lambda_{2}\left(B_{i}\right)\\
 & =\lambda_{2}\left(A\right)+\lambda_{2}\left(B\right)
\end{align*}
where the second equality follows from associativity of addition along
with the fact that $\left(\bigcup_{i=1}^{n_{A}}A_{i}\right)\cup\left(\bigcup_{i=1}^{n_{B}}B_{i}\right)$
is a disjoint union of sets in $\mathcal{L}$.
\end{proof}

\subsection{Structures generated by a class of sets}

A key way to ``expand'' a particular class of sets into a larger
structure is to look at the structure \emph{generated }by the class
of sets. This idea can be formalized in the following definition,
which serves as particular example of this general concept of generation.
\begin{defn}
\label{def:ringGeneratedByClass}For any $\mathcal{A}\subseteq2^{\mathcal{X}},$
we refer to the intersection of all rings that contain $\mathcal{A}$
as the ring \emph{generated }by $\mathcal{A}$. Formally, we write
\[
\ring\left(\mathcal{A}\right)=\bigcap\left\{ \mathcal{R\subseteq}2^{\mathcal{X}}\text{ is a ring }\mid\mathcal{A\subseteq\mathcal{R}}\right\} .
\]
\end{defn}

\begin{prop}
\label{prop:ringGeneratedByClassIsRing}For any $\mathcal{A}\subseteq2^{\mathcal{X}},$
$\ring\left(\mathcal{A}\right)$ is a ring.
\end{prop}

\begin{proof}
First note that $\ring\left(\mathcal{A}\right)$ exists since $2^{\mathcal{X}}$
is a ring and so $\left\{ \mathcal{R\subseteq}2^{\mathcal{X}}\text{ is a ring }\mid\mathcal{A\subseteq\mathcal{R}}\right\} $
is non-empty. Next observe that $\emptyset\in\ring\left(\mathcal{A}\right)$
vacuously, so property (i) in Definition \defref{ring} is easily
satisfied. For property (ii), let $A,B\in\ring\left(\mathcal{A}\right)$
and observe that $A,B\in\mathcal{R}$ for every $\mathcal{R}\in\left\{ \mathcal{R\subseteq}2^{\mathcal{X}}\text{ is a ring }\mid\mathcal{A\subseteq\mathcal{R}}\right\} $.
Since $\mathcal{R}$ is a ring, $A\cup B\in\mathcal{R}$ for every
$\mathcal{R}$ and thus $A\cup B$ is in the intersection i.e. $\ring\left(\mathcal{A}\right)$.
A similar argument establishes property (iii) and thus we can conclude
that $\ring\left(\mathcal{A}\right)$ is a ring (as it should, given
its name).
\end{proof}
In our construction of the Lebesgue measure on $\mathds{R}$, we discovered
that $\mathcal{J}$, which is the set of all disjoint unions of half-open
intervals in $\mathds{R}$, is a ring. It turns out that we can make
a stronger statement using the language of generators developed here.
\begin{prop}
\label{prop:JisRingGeneratedByL}$\mathcal{J}=\ring\left(\mathcal{L}\right)$
\end{prop}

\begin{proof}
Let $A\in\mathcal{J}$ be arbitrary. Then we can write 
\[
A=\bigcup_{i=1}^{n_{A}}A_{i}
\]
where $A_{i}\in\mathcal{L}$ are pairwise disjoint. Let $\mathcal{R}$
be an arbitrary ring that contains $\mathcal{L}$ and observe that
since rings are closed under finite unions, $A\in\mathcal{R}.$ Since
$\mathcal{R}$ was arbitrary, $A$ is contained by every ring that
contains $\mathcal{L}$ and is thus contained in the intersection
of all such rings i.e. $\ring\left(\mathcal{L}\right).$ This proves
that $\mathcal{J}\subseteq\ring\left(\mathcal{L}\right).$

To see reverse inclusion, recall that $\mathcal{J}$ is a ring that
contains $\mathcal{L}$, and so the intersection of all rings that
contain $\mathcal{L}$ is certaintly contained in $\mathcal{J}$.
This completes the proof.
\end{proof}
In measure theory, the most important structure on sets is the $\sigma$-algebra,
and the $\sigma$-algebra generated by a class of sets $\mathcal{A}$,
defined analagously to Definition \ref{def:ringGeneratedByClass}
about rings and notated as $\sigma\left(\mathcal{A}\right)$, plays
in an important role in this theory. Using a similar argument as the
one shown earlier, one can conclude that $\sigma\left(\mathcal{A}\right)$
is indeed a $\sigma$-algebra. In analysis and probability theory,
mathematicians are interested in $\sigma$-algebras generated by a
special class of sets.
\begin{defn}
\label{def:borelSigma}The $\sigma$-algebra generated by the topology
$\tau$ on set $\mathcal{X}$ is called the \emph{Borel $\sigma$-algebra
}on $\mathcal{X}$ and is denoted $\mathscr{B}\left(\mathcal{X}\right)$.
\end{defn}

The Borel $\sigma$-algebra is interesting because it turns that it
is the $\sigma$-algebra generated by $\mathcal{L}$ is indeed $\mathscr{B}\left(\mathds{R}\right)$,
where $\mathds{R}$ has the usual topology. To prove this fact, we
need a little lemma from an introductory course on analysis and topology.
\begin{lem}
\label{lem:openSetDisjointUnionInterval} Any open set in the usual
topology of $\R$ can be written as a countable disjoint union of
open intervals in $\R$.
\end{lem}

\begin{proof}
Let $O$ be an open set in $\R$ and let $x\in O$ be arbitrary. Define
$I_{x}\subseteq O$ to be the largest open interval that contains
$x$ (that is, $I_{x}$ is the union of all open intervals in $O$
that contain $x$). Note that at least one such interval exists because
$O$ is open and so there exists some $\varepsilon>0$ such that $\left(x-\varepsilon,x+\varepsilon\right)\subseteq O.$
Now for any distinct $x,y\in O$, $I_{x}$ and $I_{y}$ are either
disjoint or equal since if they were neither, $I_{x}\cup I_{y}\subseteq O$
would be a larger interval that contains both $x$ and $y$. Let $\mathcal{I}$
denote the collection of all disjoint such intervals (that is, we
get $\mathcal{I}$ by discarding all the ``redundant'' intervals
in $\left\{ I_{x}\right\} _{x\in O}$). We can do this without invoking
the Axiom of Choice since there are only countably many intervals
in $\mathcal{I}$: every interval $I\in\mathcal{I}$ contains at least
one rational number because the rationals are a countably dense subset
of $\R$. Thus, since the intervals are disjoint, $\mathcal{I}$ can
have at most countably many intervals. Of course
\[
O=\bigcup_{I\in\mathcal{I}}I
\]
and so our claim follows.
\end{proof}
\begin{prop}
\label{prop:sigmaAlgebraGeneratedbyLisBorel}$\sigma\left(\mathcal{L}\right)=\borel\left(\R\right)$
\end{prop}

\begin{proof}
Let $O$ be an open set in $\R$. Then, by Lemma \ref{lem:openSetDisjointUnionInterval}
\begin{align*}
O & =\bigcup_{i=1}^{\infty}\left(a_{i},b_{i}\right)\\
 & =\bigcup_{i=1}^{\infty}\bigcup_{n=1}^{\infty}\left(a_{i},b_{i}-\frac{1}{n}\right]
\end{align*}
which is in $\sigma\left(\mathcal{L}\right)$ by closure under countable
unions (property (ii) in Definition \ref{def:sigmaAlgebra}). Therefore
the topology of $\R$ is in $\sigma\left(\mathcal{L}\right)$ which
implies that $\borel\left(\R\right)\subseteq\sigma\left(\mathcal{L}\right)$.
The

To see the reverse inclusion, observe that for any $\left(a,b\right]\in\mathcal{L}$,
we can write
\[
\left(a,b\right]=\left(a,b\right)\cup\left\{ b\right\} \in\borel\left(\R\right)
\]
since $\left\{ b\right\} $ is closed in $\R$ and closed sets are
the complements of open sets and thus contained in $\borel\left(\R\right)$.\footnote{$\sigma$-algebras on $\mathcal{X}$ are closed under complements
since they are closed under set-differences and contain $\mathcal{X}$.} Therefore $\mathcal{L\subseteq\borel\left(\R\right)}$ and so $\sigma\left(\mathcal{L}\right)\subseteq\borel\left(\R\right)$,
completing the proof.
\end{proof}
Now we are ready to prove that our proto-measure $\lambda_{2}$ is
actually a countably-additive pre-measure on $\ring\left(\mathcal{L}\right)$.
But first, we need a lemma about double sums!
\begin{lem}[Tonelli for series]
\label{lem:TonelliForSeries}Let $\left\{ x_{ij}\right\} _{i,j\in\N\times\N}$
be a sequence of non-negative (extended) real numbers. Then

\[
\sum_{i,j\in\N^{2}}x_{ij}=\sum_{i=1}^{\infty}\sum_{j=1}^{\infty}x_{ij}=\sum_{j=1}^{\infty}\sum_{i=1}^{\infty}x_{ij}.
\]
\end{lem}

\begin{proof}
We will prove the first equality since the second then follows by
symmetry. Let $F\subset\N^{2}$ be arbitrary and finite. Then, there
exists some $N\in\N$ such that $F\subseteq\left\{ 1,2\ldots,N\right\} ^{2}$
and so, by the non-negativity of $x_{ij}$
\[
\sum_{i,j\in F}x_{ij}\leq\sum_{i,j\in\left\{ 1,2\ldots,N\right\} ^{2}}x_{ij}=\sum_{i=1}^{N}\sum_{j=1}^{N}x_{ij}\leq\sum_{i=1}^{\infty}\sum_{j=1}^{\infty}x_{ij}.
\]
This inequality holds for any finite $F\subset\N^{2}$ and so it holds
for the supremum of all such finite sums. That is to say,

\[
\sup_{F\subset\N^{2}\mid F\text{ is finite}}\sum_{i,j\in F}x_{ij}\leq\sum_{i=1}^{\infty}\sum_{j=1}^{\infty}x_{ij}.
\]
But recall that for any $\left\{ a_{i}\right\} _{i\in\mathcal{I}}\in\left[0,\infty\right]$
where $\mathcal{I}$ is any index set
\[
\sum_{i\in\mathcal{I}}a_{i}:=\sup_{I\subset\mathcal{I}\mid I\text{ is finite}}\sum_{i\in I}a_{i},
\]
and so we have that
\[
\sum_{i,j\in\N^{2}}x_{ij}\leq\sum_{i=1}^{\infty}\sum_{j=1}^{\infty}x_{ij}.
\]

To derive the other inequality, observe that it is sufficient to prove
that
\[
\sum_{i,j\in\N^{2}}x_{ij}\geq\sum_{i=1}^{I}\sum_{j=1}^{\infty}x_{ij}
\]
for every $I\in\N$. Fix $I=I_{0}$ and note that
\[
\sum_{i=1}^{I_{0}}\sum_{j=1}^{\infty}x_{ij}=\sum_{i=1}^{I_{0}}\lim_{J\to\infty}\sum_{j=1}^{J}x_{ij}=\lim_{J\to\infty}\sum_{i=1}^{I_{0}}\sum_{j=1}^{J}x_{ij}.
\]
Thus to prove $\sum_{i,j\in\N^{2}}x_{ij}\geq\sum_{i=1}^{I_{0}}\sum_{j=1}^{\infty}x_{ij}$
we need to prove that 
\[
\sum_{i,j\in\N^{2}}x_{ij}\geq\sum_{i=1}^{I_{0}}\sum_{j=1}^{J}x_{ij}
\]
for every $J\in\N$. Fix $J=J_{0}$ and then observe that
\[
\sum_{i=1}^{I_{0}}\sum_{j=1}^{J_{0}}x_{ij}=\sum_{i,j\in\left\{ 1,2,\ldots,I_{0}\right\} \times\left\{ 1,2,\ldots,J_{0}\right\} }x_{ij}\leq\sum_{i,j\in\N^{2}}x_{ij}
\]
where the inequality follows due to non-negativity of $x_{ij}$. This
concludes the proof.
\end{proof}
\begin{rem*}
This lemma is a special case of Tonelli's theorem, a fundamental theorem
that allows us to construct measures on Cartesian products of measure
spaces from the measures on those spaces themselves. This theorem
will be motivated and proved in Chapter 5.
\end{rem*}
\begin{prop}
\label{prop:ringMeasureCountablyAdditive} $\lambda_{2}$ is a countably
additive pre-measure on $\ring\left(\mathcal{L}\right)$, that is
to say,

\begin{enumerate}[label=(\roman*),leftmargin=.1\linewidth,rightmargin=.4\linewidth]
	\item $\lambda_2\left(\emptyset\right) = 0$ 
	\item For disjoint $\left\{A_i\right\}_{i=1}^{\infty}\in \mathcal{J}$ such that $\bigcup_{i=1}^{\infty}A_i \in \mathcal{J}$
	\[
			\lambda_2\left(\bigcup_{i=1}^{\infty}A_i\right) = \sum_{i=1}^{\infty}\lambda_2\left(A_i\right).
	\]
\end{enumerate}
\end{prop}

\begin{proof}
Property (i) is inherited from $\lambda_{1}$. To see property (ii),
let $\left\{ A_{i}\right\} _{i=1}^{\infty}\in\mathcal{J}$ be disjoint
and write $A:=\bigcup_{i=1}^{\infty}A_{i}$ where $A\in\mathcal{J}$
by assumption. First, note that if $\lambda_{2}\left(A_{i}\right)=\infty$
for any $i\in\N$, then $\infty=\lambda_{2}\left(A_{i}\right)\leq\lambda_{2}\left(\bigcup_{i=1}^{\infty}A_{i}\right)=\infty$
where the inequality is due to the monotonicity\footnote{For any $A,B\in\ring\left(\mathcal{L}\right)\text{ such that }A\subseteq B,\lambda_{2}\left(B\right)=\lambda_{2}\left(A\right)+\lambda_{2}\left(B\setminus A\right)\geq\lambda_{2}\left(A\right)$}
of $\lambda_{2}$. Thus, in this case, the claim follows vacuously.
So, without loss of generality, we can assume that $\lambda_{2}\left(A_{i}\right)<\infty$
for every $i\in\N$. First, note that for any $n\in\N$,$\bigcup_{i=1}^{n}A_{i}\subseteq A$
and so, by the monotonictity and finite additivity of $\lambda_{2}$,
we have that
\[
\lambda_{2}\left(A\right)\geq\lambda_{2}\left(\bigcup_{i=1}^{n}A_{i}\right)=\sum_{i=1}^{n}\lambda_{2}\left(A_{i}\right)
\]
for every $n\in\N$. Taking limits, we have countable superadditivity:
\[
\lambda_{2}\left(A\right)\geq\sum_{i=1}^{\infty}\lambda_{2}\left(A_{i}\right).
\]
In order to deduce the reverse inequality, first suppose that both
$A$ and $\left\{ A_{i}\right\} $ are in $\mathcal{L}.$ Then, we
can write 
\[
A:=\left(a,b\right]
\]
and
\[
A_{i}=\left(a_{i},b_{i}\right]
\]
for each $i\in\N.$ Pick an arbitrary $0<\epsilon<b-a$ and observe
that 
\[
\left[a+\epsilon,b\right]\subseteq\bigcup_{i=1}^{\infty}\left(a_{i},b_{i}+\frac{\epsilon}{2^{i}}\right)
\]
and so by the Heine-Borel theorem, there exists some finite $K$ such
that 
\[
\left[a+\epsilon,b\right]\subseteq\bigcup_{k=1}^{K}\left(a_{i_{k}},b_{i_{k}}+\frac{\epsilon}{2^{i_{k}}}\right).
\]
By the finite additivity established in Proposition \ref{prop:ringMeasureFinitelyAdditive}
and monotonicity, we have that
\[
\underbrace{b-a}_{\lambda_{2}\left(A\right)}-\epsilon\leq\sum_{k=1}^{K}b_{i_{k}}+\frac{\epsilon}{2^{i_{k}}}-a_{i_{k}}\leq\underbrace{\sum_{i=1}^{\infty}\left(b_{i}-a_{i}\right)}_{\sum_{i\in\N}\lambda_{2}\left(A_{i}\right)}+\epsilon
\]
and since $\epsilon$ can be arbitrary small the claim follows.

Deducing the general case from the special one outlined above is straightforward.
If $A,\left\{ A_{i}\right\} \in\mathcal{J}$ then 
\[
A=\bigcup_{j=1}^{J}B_{j}
\]
where $\left\{ B_{j}\right\} \in\mathcal{L}$ are pairwise disjoint.
Similarly, 
\[
A_{i}=\bigcup_{k=1}^{n_{i}}C_{ik}
\]
where $\left\{ C_{ij}\right\} _{i\in\N,j\in\N}\in\mathcal{L}$ are
pairwise disjoint and $n_{i}\in\N$. Note that then
\[
\lambda_{2}\left(A\right)=\sum_{j=1}^{J}\lambda_{2}\left(B_{j}\right)\leq\sum_{i=1}^{\infty}\sum_{k=1}^{n_{i}}\lambda_{2}\left(C_{ik}\right)=\sum_{i=1}^{\infty}\lambda_{2}\left(A_{i}\right)
\]
where the first equality follow from the finite additivity of $\lambda_{2}$
on $\mathcal{J},$ the inequality by the fact that for any $j\in\left\{ 1,2,\ldots,J\right\} ,$
there exists a partition of the collection $\left\{ C_{ik}\right\} $
into subcollections $\left\{ C_{ik}^{j}\right\} _{1\leq j\leq J}$such
that 
\[
B_{j}=\bigcup_{i,k}C_{ik}^{j}
\]
and so the special case of our result on $\mathcal{L}$ applies (along
with an application of Lemma \ref{lem:TonelliForSeries}). The final
equality again follows by finite additivity. 
\end{proof}

\subsection{Outer measures}
\begin{defn}
\label{def:outerMeasure}A set valued function 
\[
\mu^{*}:2^{\mathcal{X}}\longrightarrow\left[0,\infty\right]
\]
is called an outer measure on $\mathcal{X}$ if

\begin{enumerate}[label=(\roman*),leftmargin=.1\linewidth,rightmargin=.4\linewidth]
	\item $ \mu^*\left(\emptyset\right) = 0$ 
	\item $A\subseteq B \in 2^\mathcal{X} \Longrightarrow \mu^*\left(A\right) \leq \mu^*\left(B\right) $
	\item For $\left\{A_i\right\}_{i=1}^{\infty}\in 2^\mathcal{X}$ 
	\[
			\mu^*\left(\bigcup_{i=1}^{\infty}A_i\right) \leq \sum_{i=1}^{\infty}\mu^*\left(A_i\right).
	\]
\end{enumerate}
\end{defn}

\begin{example}
\label{exa:canonicalOuterMeasure}Given a non-negative extended-real
valued function $\mu$ on a collection $\mathcal{A\subseteq}2^{\mathcal{X}}$
such that $\mu\left(\emptyset\right)=0$, define for any $E\subseteq\mathcal{X}$
\[
\mu^{*}\left(E\right):=\inf\left\{ \sum_{i=1}^{\infty}\mu\left(A_{i}\right)\mid A_{i}\in\mathcal{A},E\subseteq\bigcup_{i=1}^{\infty}A_{i}\right\} 
\]
\end{example}

Note that this function is defined on $2^{\mathcal{X}}$ since every
bounded below subset of the (extended) real numbers has an infimum.
Now we prove that the set-function descibed above is indeed an outer
measure.
\begin{prop}
\label{prop:canonicalOuterMeasureIsOuterMeasure}The function $\mu^{*}:2^{\mathcal{X}}\longrightarrow\left[0,\infty\right]$
defined in Example \ref{exa:canonicalOuterMeasure} is an outer measure
\end{prop}

\begin{proof}
For (i), observe that $\emptyset\in\mathcal{A}$ and so $\mu^{*}\left(\emptyset\right)=\mu\left(\emptyset\right)=0.$
Next, let $A\subseteq B\subseteq\mathcal{X}$ and observe that 
\[
\left\{ \sum_{i=1}^{\infty}\mu\left(A_{i}\right)\mid A_{i}\in\mathcal{A},B\subseteq\bigcup_{i=1}^{\infty}A_{i}\right\} \subseteq\left\{ \sum_{i=1}^{\infty}\mu\left(A_{i}\right)\mid A_{i}\in\mathcal{A},A\subseteq\bigcup_{i=1}^{\infty}A_{i}\right\} 
\]
and so 
\[
\mu^{*}\left(B\right)=\inf\left\{ \sum_{i=1}^{\infty}\mu\left(A_{i}\right)\mid A_{i}\in\mathcal{A},B\subseteq\bigcup_{i=1}^{\infty}A_{i}\right\} \geq\inf\left\{ \sum_{i=1}^{\infty}\mu\left(A_{i}\right)\mid A_{i}\in\mathcal{A},A\subseteq\bigcup_{i=1}^{\infty}A_{i}\right\} =\mu^{*}\left(A\right)
\]
which gives us (ii). For (iii), let $\left\{ E_{i}\right\} _{i=1}^{\infty}\in2^{\mathcal{X}}$
be disjoint and assume that $\sum_{i=1}^{\infty}\mu^{*}\left(E_{i}\right)<\infty$
since otherwise the claim is trivial. Fix $\epsilon>0$ and choose
$A_{ij}\in\mathcal{A}$ such that $E_{i}\subseteq\bigcup_{j=1}^{\infty}A_{ij}$
and

\[
\mu^{*}\left(E_{i}\right)\leq\sum_{j=1}^{\infty}\mu(A_{ij})<\mu^{*}\left(E_{i}\right)+\frac{\epsilon}{2^{i}}
\]
for every $i\in\N$\footnote{This is possible due to the assumption that $\mu^{*}\left(E_{i}\right)<\infty$,
which implies that the set $\left\{ \sum_{j=1}^{\infty}\mu\left(A_{ij}\right)\mid A_{ij}\in\mathcal{A},E_{i}\subseteq\bigcup_{j=1}^{\infty}A_{ij}\right\} $
is non-empty. The definition of an infimum then implies that such
a cover $\left\{ A_{ij}\right\} $ exists.}. Observe that
\begin{align*}
E & :=\bigcup_{i=1}^{\infty}E_{i}\subseteq\bigcup_{i=1}^{\infty}\bigcup_{j=1}^{\infty}A_{ij}
\end{align*}
and so 
\[
\mu^{*}\left(E\right)\leq\sum_{i,j\in\N^{2}}\mu\left(A_{ij}\right)=\sum_{i=1}^{\infty}\sum_{j=1}^{\infty}\mu\left(A_{ij}\right)\leq\sum_{i=1}^{\infty}\mu^{*}(E_{i})+\epsilon
\]
where the equality follows by Lemma \ref{lem:TonelliForSeries} and
the second inequality is due to properties of the geometric series.
Since $\epsilon$ was arbitrary, the claim folllows.
\end{proof}
\begin{rem}
The outer measure described above is called the \emph{canonical }outer-measure
as it as by far the most useful type of outer measure in measure theory.
Given a space $\X$, a collection of subsets $\mathcal{A}\subseteq2^{\X}$,
and a countably additive pre-measure $\mu$ on $\mathcal{A}$, we
can call
\[
\mu^{*}\left(E\right):=\inf\left\{ \sum_{i=1}^{\infty}\mu\left(A_{i}\right)\mid A_{i}\in\mathcal{A},E\subseteq\bigcup_{i=1}^{\infty}A_{i}\right\} 
\]
the canonical outer measure generated by $\left(\mu,\mathcal{A}\right)$.
\end{rem}

\begin{prop}
\label{prop:restrictionOfOuterMeasure}Let $\mathcal{A}$, $\mu,$
and $\mu^{*}$ be defined as in Example \ref{exa:canonicalOuterMeasure}.
Then, for any $A\in\mathcal{A}$
\[
\mu^{*}\left(A\right)=\mu\left(A\right).
\]
\end{prop}

\begin{proof}
First, observe that $A$ is a cover for itself and that $\emptyset\in\mathcal{A}$
and so 
\[
\mu^{*}\left(A\right)=\inf\left\{ \sum_{i=1}^{\infty}\mu(A_{i})\mid A_{i}\in\mathcal{A},A\subseteq\bigcup_{i=1}^{\infty}A_{i}\right\} \leq\sum_{i=1}^{\infty}\mu(A_{i})
\]
where $A_{1}=A$ and $A_{i}=\emptyset$ for $i\neq1.$ Therefore,
\[
\mu^{*}\left(A\right)\leq\mu\left(A\right).
\]

To see the reverse inequality, let $\left\{ A_{i}\right\} _{i\in\N}\in\mathcal{A}$
be an arbitrary cover of $A.$ Define,
\[
B_{i}:=A\cap\left(A_{i}\setminus\bigcup_{j=1}^{i-1}A_{j}\right)
\]
and notice that the $\left\{ B_{i}\right\} $ is a pairwise disjoint
collections whose union is $A$ such that $B_{i}\subseteq A_{i}$
for every $i\in\N$. By countable additivity and monotonicity,

\[
\mu\left(A\right)=\sum_{i=1}^{\infty}\mu\left(B_{i}\right)\leq\sum_{i=1}^{\infty}\mu\left(A_{i}\right).
\]
Since $\left\{ A_{i}\right\} \subseteq\mathcal{A}$ is an arbitrary
cover of $A$ , we have that
\[
\mu\left(A\right)\leq\inf\left\{ \sum_{i=1}^{\infty}\mu\left(A_{i}\right)\mid A_{i}\in\mathcal{A},A\subseteq\bigcup_{i=1}^{\infty}A_{i}\right\} =\mu^{*}\left(A\right)
\]
which completes the proof.
\end{proof}
Now we are (finally!!) ready to extend our pre-measure to a bona-fide
measure on a $\sigma$-algebra, using the following theorem.
\begin{thm}[Caratheodory's Extension Theorem]
\label{thm:caratheodoryExtn}Let $\X$ be a set. Given a countably-additive
pre-measure $\mu$ on ring $\mathcal{A\subseteq}2^{\mathcal{X}}$
with canonical outer measure $\mu^{*}$ generated by $\left(\mu,\mathcal{A}\right)$,
define the collection 
\[
\mathcal{C}\left(\mu^{*}\right):=\left\{ A\subseteq\mathcal{X}\mathrm{\ such\ that\ }\mu^{*}\left(E\right)=\mu^{*}\left(A\cap E\right)+\mu^{*}\left(A^{C}\cap E\right)\forall E\in2^{\mathcal{X}}\right\} .
\]
Then

\begin{enumerate}[label=(\roman*),leftmargin=.1\linewidth,rightmargin=.4\linewidth]
	\item $ \mathcal{A}\subseteq \mathcal{C}$.
	\item $ \mathcal{C}\left(\mu^*\right) $ is a $\sigma$-algebra.
	\item $\left.\mu^*\right|_{\mathcal{C}}$ is a countably additive measure on $\mathcal{C}$. 
\end{enumerate}
\end{thm}

\begin{proof}
First we will show (i). Let $A\in\mathcal{A}$ be arbitrary. By the
countable subadditivity of $\mu^{*}$, we know that 
\[
\mu^{*}\left(E\right)=\mu^{*}\left(\left(A\cap E\right)\bigcup\left(A^{C}\cap E\right)\right)\leq\mu^{*}\left(A\cap E\right)+\mu^{*}\left(A^{C}\cap E\right)
\]
for every $E\subseteq\mathcal{X}$. To deduce the reverse inequality,
fix $E$ such that $\mu^{*}\left(E\right)<\infty$ because otherwise
the claim follows trivially. Pick an $\epsilon>0$ and find a cover
$\left\{ A_{i}\right\} _{i=1}^{\infty}\in\mathcal{A}$ of $E$ such
that
\[
\mu^{*}\left(E\right)\leq\sum_{i=1}^{\infty}\mu\left(A_{i}\right)<\mu^{*}\left(E\right)+\epsilon
\]
As in the proof of Proposition \ref{prop:canonicalOuterMeasureIsOuterMeasure},
this is possible because $\mu^{*}\left(E\right)<\infty$ and the definition
of an infimum. Next, observe that
\begin{align*}
E\cap A & \subseteq\bigcup_{i=1}^{\infty}(A_{i}\cap A),\\
E\cap A^{C} & \subseteq\bigcup_{i=1}^{\infty}(A_{i}\cap A^{C})
\end{align*}
and so 
\begin{align*}
\mu^{*}\left(E\cap A\right) & \leq\mu^{*}\left(\bigcup_{i=1}^{\infty}(A_{i}\cap A)\right)\leq\sum_{i=1}^{\infty}\mu^{*}\left(A_{i}\cap A\right)\\
\mu^{*}\left(E\cap A^{C}\right) & \leq\mu^{*}\left(\bigcup_{i=1}^{\infty}(A_{i}\cap A^{C})\right)\leq\sum_{i=1}^{\infty}\mu^{*}\left(A_{i}\cap A^{C}\right)
\end{align*}
where the first inequality follows due to monotonicity and the second
due to subadditivity. Together, these inequalities imply that
\begin{align*}
\mu^{*}\left(A\cap E\right)+\mu^{*}\left(A^{C}\cap E\right) & \leq\sum_{i=1}^{\infty}\mu^{*}\left(A_{i}\cap A\right)+\mu^{*}\left(A_{i}\cap A^{C}\right)\\
 & =\sum_{i=1}^{\infty}\mu\left(A_{i}\cap A\right)+\mu\left(A_{i}\cap A^{C}\right)\\
 & =\sum_{i=1}^{\infty}\mu\left(A_{i}\right)\\
 & <\mu^{*}\left(E\right)+\epsilon
\end{align*}
where the first equality is due to the fact that rings are closed
under intersections and set-differences along with Proposition \ref{prop:restrictionOfOuterMeasure}
and the second equality is due to the countable additivity of $\mu.$
Since $\epsilon$ and $E$ are arbitrary, we have that 
\[
\mu^{*}\left(A\cap E\right)+\mu^{*}\left(A^{C}\cap E\right)\leq\mu^{*}\left(E\right)
\]
for every $E\subseteq\mathcal{X},$ establishing that $\mathcal{A}\subseteq\mathcal{C}$.

Next we show (ii); that is, we prove $\mathcal{C}$ is a $\sigma-$algebra.
Recall Definition \ref{def:sigmaAlgebra} and notice that it is sufficient
to prove that (1) $\emptyset,\mathcal{X}\in\mathcal{C}$; (2) if $A\in\mathcal{C}$
then $A^{C}\in\mathcal{C}$; (3) if $\left\{ A_{i}\right\} _{i=1}^{\infty}\in\mathcal{C}$
then $\bigcup_{i=1}^{\infty}A_{i}\in\mathcal{C}.$ Note that $\emptyset,\mathcal{X}\in\mathcal{C}$
because, trivially,
\[
\mu^{*}\left(E\cap\mathcal{X}\right)+\mu^{*}\left(E\cap\emptyset\right)=\mu^{*}\left(E\right).
\]
Symmetry between $A$ and $A^{C}$ in the definition of $\mathcal{C}$
establishes (2). For (3), we first establish closure under finite
unions and bootstrap this weaker result to yield the stronger claim.
Let $A,B\in\mathcal{C}$ and let $E\subseteq\mathcal{X}$ be arbitrary.
Then

\begin{align*}
\mu^{*}\left(E\right) & =\mu^{*}\left(E\cap A\right)+\mu^{*}\left(E\cap A^{C}\right)\\
 & =\mu^{*}\left(\left(E\cap A\right)\cap B\right)+\mu^{*}\left(\left(E\cap A\right)\cap B^{C}\right)+\mu^{*}\left(E\cap A^{C}\right)\\
 & =\mu^{*}\left(E\cap A\cap B\right)+\mu^{*}\left(E\cap\left(A\cap B\right)^{C}\cap A\right)+\mu^{*}\left(E\cap\left(A\cap B\right)^{C}\cap A^{C}\right)\\
 & =\mu^{*}\left(E\cap A\cap B\right)+\mu^{*}\left(E\cap\left(A\cap B\right)^{C}\right)
\end{align*}
where the second equality is due to the definition of $\mathcal{C}$
and the fact that $B\in\mathcal{\mathcal{C}}$, the third equality
is due to the identities
\begin{align*}
\left(A\cap B\right)^{C}\cap A & =\left(A^{C}\cup B^{C}\right)\cap A=A\cap B^{C}\\
\left(A\cap B\right)^{C}\cap A^{C} & =\left(A^{C}\cup B^{C}\right)\cap A^{C}=A^{C},
\end{align*}
and the fourth equality follows from the definition of $\mathcal{C}$
and that $A\in\mathcal{C}$. This proves that for any $A,B\in\mathcal{C}$,
$A\cap B\in\mathcal{C}.$ Property (2) then implies that $A\cup B\in\mathcal{C}.$

To establish closure under countable unions, fix $E\subseteq\mathcal{X}$
and let $\left\{ A_{i}\right\} _{i=1}^{\infty}\in\mathcal{C}$ be
arbitrary with $B=\bigcup_{i\in\N}A_{i}$ and define
\[
B_{n}:=\bigcup_{i=1}^{n}A_{i}
\]
where $B_{n}\in\mathcal{C}$ by our result on closure under finite
unions. Without loss of generality, we can assume that the $\left\{ A_{i}\right\} $
are pairwise disjoint (since we could otherwise replace $A_{i}$ with
$C_{i}:=A_{i}\setminus\bigcup_{j=1}^{i-1}A_{j}$ which are disjoint
such that $\bigcup_{i=1}^{\infty}A_{i}=\bigcup_{i=1}^{\infty}C_{i}$).
Then, we have that
\begin{align*}
\mu^{*}\left(E\right) & =\mu^{*}\left(E\cap B_{n}^{C}\right)+\mu^{*}\left(E\cap B_{n}\right)\\
 & =\mu^{*}\left(E\cap B_{n}^{C}\right)+\mu^{*}\left(E\cap B_{n}\cap A_{n}\right)+\mu^{*}\left(E\cap B_{n}\cap A_{n}^{C}\right)\\
 & =\mu^{*}\left(E\cap B_{n}^{C}\right)+\mu^{*}\left(E\cap A_{n}\right)+\mu^{*}\left(E\cap B_{n-1}\right)
\end{align*}
where we used the fact that $A_{n}\in\mathcal{C}$ for the second
equality and the disjointness of $A_{i}$ for the third equality.
Observe that the equality $\mu^{*}\left(E\cap B_{n}\right)=\mu^{*}\left(E\cap A_{n}\right)+\mu^{*}\left(E\cap B_{n-1}\right)$
is a recurrence relation that can be expanded as
\[
\mu^{*}\left(E\cap B_{n}\right)=\sum_{i=1}^{n}\mu^{*}\left(E\cap A_{i}\right)
\]
and so 
\begin{align*}
\mu^{*}\left(E\right) & =\mu^{*}\left(E\cap B_{n}^{C}\right)+\sum_{i=1}^{n}\mu^{*}\left(E\cap A_{i}\right)\\
 & \geq\mu^{*}\left(E\cap B^{C}\right)+\sum_{i=1}^{n}\mu^{*}\left(E\cap A_{i}\right)
\end{align*}
for every $n\in\N$ where the inequality is due to the the fact that
$B^{C}\subseteq B_{n}^{C}$ and the monotonicity of outer measures.
After taking limits, we have that
\begin{align*}
\mu^{*}\left(E\right) & \geq\mu^{*}\left(E\cap B^{C}\right)+\sum_{i=1}^{\infty}\mu^{*}\left(E\cap A_{i}\right)\\
 & \geq\mu^{*}\left(E\cap B^{C}\right)+\mu^{*}\left(\bigcup_{i\in\N}\left(E\cap A_{i}\right)\right)\\
 & =\mu^{*}\left(E\cap B^{C}\right)+\mu^{*}\left(E\cap B\right)
\end{align*}
where the second inequality follows by countable subadditivity. Another
application of countable subadditivity yields
\[
\mu^{*}\left(E\right)\leq\mu^{*}\left(E\cap B^{C}\right)+\mu^{*}\left(E\cap B\right)
\]
and together the two inequalities establish that $B\in\mathcal{C},$
finishing the proof of (ii).

Finally, in order to show that $\left.\mu^{*}\right|_{\mathcal{C}}$
is indeed a countably additive measure on $\mathcal{C}$, let $\left\{ A\right\} _{i=1}^{\infty}\in\mathcal{C}$
be pairwise disjoint, and observe that for $B:=\bigcup_{i\in\N}A_{i}\in\mathcal{C}$
and any $E\subseteq\mathcal{X}$
\[
\mu^{*}\left(E\right)\geq\mu^{*}\left(E\cap B^{C}\right)+\sum_{i=1}^{\infty}\mu^{*}\left(E\cap A_{i}\right)
\]
due to our previous work. Letting $E=B$, we have 
\begin{align*}
\mu^{*}\left(B\right) & \geq\sum_{i=1}^{\infty}\mu^{*}\left(B\cap A_{i}\right)\\
 & =\sum_{i=1}^{\infty}\mu^{*}\left(A_{i}\right).
\end{align*}
Since the reverse inequality follows by the subadditivity of the outer
measure, our proof is complete.
\end{proof}
\begin{rem}
\label{rem:noRingReqd}Note that the proof of the facts that $\mathcal{C}\left(\mu^{*}\right)$
is a $\sigma$-algebra and $\mu^{*}|_{\mathcal{C}}$ is countably
additive do not depend on the fact $\mathcal{A}$ is a ring; the proofs
would hold if $\mathcal{A}$ was any collection of sets and $\mu^{*}$
was any outer measure (as opposed to a \emph{canonical }outer measure).
\end{rem}

Note that in general such an extension may not be unique and we provide
sufficient conditions for uniqueness in Theorem \ref{thm:uniquenessMeasures}.

\section{The Steiltjes measure on $\protect\R$}

We now have enough machinery to construct the Lebesgue measure on
$\borel\left(\R\right)$; in fact, the Caratheodory measurabality
critierion discussed in the proof of the extension theorem \ref{thm:caratheodoryExtn}
is strictly larger than the Borel sets, a fact that we shall be able
to establish soon. To show the Lebesgue measure exists, we observe
that $\lambda_{2}$ is a countably-additive pre-measure on $\mathcal{J}$
which is a ring. The canonical outer measure $\lambda^{*}$ generated
from $\left(\lambda_{2},\mathcal{J}\right)$ then can be restricted
to the $\sigma$-algebra of measurable sets $\mathcal{C}\left(\lambda^{*}\right)$
as a measure via Caratheodory's extension theorem. That the Borel
sets $\borel\left(\R\right)\subseteq\mathcal{C}\left(\lambda^{*}\right)$
is clear from the fact that $\mathcal{L}\subseteq\mathcal{\mathcal{J}\subseteq C}\left(\lambda^{*}\right)$
and $\sigma\left(\mathcal{L}\right)=\borel\left(\R\right)$ (see Proposition\ref{prop:sigmaAlgebraGeneratedbyLisBorel}).
The fact that this inclusion is strict is, of course, not obvious;
we will return to this point later.

While this strategy to build the Lebesgue meaure works, we can in
fact do something more general, which will incidentally also help
us establish the properties of the Lebesgue measure. This involves
the notion of what is called a \emph{Steiljes }measure, a concept
which is particularly useful in probability theory. To start, we prove
a stronger version of Caratheodory's theorem.
\begin{thm}[Extension from semi-rings]
\label{thm:semiRingCaratheodoryExtn}Let $\X$ be a set and $\mathcal{A}\subseteq2^{\X}$
be a semi-ring. Suppose $\mu:\mathcal{A}\longrightarrow\left[0,\infty\right]$
is a set function such that

\begin{enumerate}[label=(\roman*),leftmargin=.1\linewidth,rightmargin=.4\linewidth]
	\item $\mu\left(\emptyset\right) = 0 $
	\item  For any disjoint $A,B \in \mathcal{A}$ such that $A\cup B \in \mathcal{A}$
	\[
			\mu\left(A\cup B\right) = \mu\left(A\right) + \mu\left(B\right)
	\]
	\item For any collection $A_i \in \mathcal{A}$ such that $\bigcup_{i\in \mathds{N}} A_i \in \mathcal{A}$
	\[
		\mu\left(\bigcup_{i\in\mathds{N}} A_i \right) \leq \sum_{i \in \mathds{N}} \mu \left(A_i\right)
	\]
\end{enumerate}then the restriction of the canonical outer measure $\mu^{*}$ generated
by $\left(\mu,\mathcal{A}\right)$ to $\mathcal{C}\left(\mu^{*}\right)$
is a measure.
\end{thm}

\begin{proof}
Note that Remark \ref{rem:noRingReqd} tells us that $\mathcal{C}\left(\mu^{*}\right)$
is a $\sigma-$algebra and $\mu^{*}|_{\mathcal{C}}$ is a measure.
Thus our two tasks are to show \emph{(i)} that $\mu^{*}$ and $\mu$
agree on\emph{ $\mathcal{A}$ }and\emph{ (ii)} that $\mathcal{A}\subseteq\mathcal{C}\left(\mu^{*}\right)$.
The first result is mostly straightforward; to see that $\mu^{*}\left(A\right)\leq\mu\left(A\right)$
for $A\in\mathcal{A}$ we can simply observe that $A$ is a cover
for itself. For the reverse inequality, first note that finite additivity
and the fact that $\mathcal{A}$ is a semi-ring implies that for any
$A,B\in\mathcal{A}$ such that $A\subseteq B,\mu\left(A\right)\leq\mu\left(B\right).$
Indeed, there exist disjoint $\left\{ C_{i}\right\} _{1\leq i\leq n}\in\mathcal{A}$
where $n\in\N$ such that $B\setminus A=\cup_{1\leq i\leq n}C_{i}$
and so $B=A\cup\cup_{1\leq i\leq n}C_{i}$ and $\mu\left(B\right)=\mu\left(A\right)+\sum_{1\leq i\leq n}\mu\left(C_{i}\right)\geq\mu\left(A\right).$
Then for any cover $\left\{ A_{i}\right\} _{i\in\N}\in\mathcal{A}$
of $A$, we have that 
\[
\mu\left(A\right)=\sum_{i=1}^{\infty}\mu\left(A\cap A_{i}\right)\leq\sum_{i=1}^{\infty}\mu\left(A_{i}\right)
\]
where the equality follows from the fact that $A\cap A_{i}\in\mathcal{A}$
since semi-rings are closed under intersection along with the fact
that $\left\{ A\cap A_{i}\right\} _{i\in\N}$ forms a partition of
$A$ and the inequality follows from monotonicty. Since the cover
was arbitrary, we have $\mu^{*}\left(A\right)\geq\mu\left(A\right).$

To prove \emph{(ii), }note that for any $A\in\mathcal{A}$, countable
subadditivity of the outer measure implies that 
\[
\mu^{*}\left(E\right)\leq\mu^{*}\left(A\cap E\right)+\mu^{*}\left(A^{C}\cap E\right)
\]
for any $E\in2^{\X}.$ To deduce the other inequality, we follow almost
exactly the same steps as we did in the proof of Theorem \ref{thm:caratheodoryExtn}.
First, we pick an $E\subseteq2^{\X}$ such that $\mu^{*}\left(E\right)<\infty$since
otherwise the claim follows trivially. Then we use this fact (since
only empty subsets of the reals have inifinite infima) to deduce that
for any $\epsilon>0$, there exists a cover $\left\{ A_{i}\right\} _{i\in\N}\in\mathcal{A}$of
$E$ such that
\[
\mu^{*}\left(E\right)\leq\sum_{i=1}^{\infty}\mu\left(A_{i}\right)<\mu^{*}\left(E\right)+\epsilon.
\]
Again, we observe that 
\begin{align*}
E\cap A & \subseteq\bigcup_{i=1}^{\infty}A_{i}\cap A\\
E\cap A^{C} & \subseteq\bigcup_{i=1}^{\infty}A_{i}\cap A^{C}
\end{align*}
Note that $A_{i}\cap A^{C}=A_{i}\setminus A$ and so by the properties
of semi-rings there exists, for each $i,$a disjoint collection of
sets $\left\{ C_{j}^{i}\right\} _{1\leq j\leq n_{i}}\in\mathcal{A}$
such that $A_{i}\cap A^{C}=\bigcup_{1\leq j\leq n_{i}}C_{j}^{i}.$
Using the monotonicty and countable-subadditivity of outer measures
as before, we have that
\begin{align*}
\mu^{*}\left(E\cap A\right)+\mu^{*}\left(E\cap A^{C}\right) & \leq\sum_{i=1}^{\infty}\left(\mu^{*}\left(A_{i}\cap A\right)+\mu^{*}\left(\bigcup_{1\leq j\leq n_{i}}C_{j}^{i}\right)\right)\\
 & \leq\sum_{i=1}^{\infty}\left(\mu^{*}\left(A_{i}\cap A\right)+\sum_{j=1}^{n_{i}}\mu^{*}\left(C_{j}^{i}\right)\right)\\
 & =\sum_{i=1}^{\infty}\left(\mu\left(A_{i}\cap A\right)+\sum_{j=1}^{n_{i}}\mu\left(C_{j}^{i}\right)\right)\\
 & =\sum_{i=1}^{\infty}\left(\mu\left(A_{i}\right)\right)\\
 & <\mu^{*}\left(E\right)+\epsilon
\end{align*}
where the first equality follows from part \emph{(i) }and the fact
that $A_{i}\cap A,\left\{ C_{j}^{i}\right\} _{1\leq j\leq n_{i}}\in\mathcal{A}$
whereas the second equality follows from finite additivity. Since
$\epsilon$can be as small as one wants, our result follows.
\end{proof}
\hl{ADD STIELJES MEASURE CONSTRUCTION FROM ASH PROBABILITY}
\begin{thm}[Existence of the Lebesgue measure]
\label{thm:existenceLebesgueR}There exists a $\sigma-$algebra $\F$
which contains all the open sets in $\R$ and a set function $\lambda:\F\longrightarrow\left[0,\infty\right]$
such that

\begin{enumerate}[label=(\roman*),leftmargin=.1\linewidth,rightmargin=.4\linewidth]
	\item $\lambda\left((a,b]\right) = b - a $ for any $a \leq b \in \mathds{R}$\footnote{The intervals could be open, closed or neither.}
	\item For any disjoint $A_i \in \mathcal{F}$
	\[
			\lambda\left(\bigcup_{i\in\mathds{N}}\right) = \sum_{i \in \mathds{N}} \lambda\left(A_i\right)
	\]
	\item For disjoint $\{A_i\}_{i\in \N} \in \mathcal{F}$ 
	\[
			\mu\left(\bigcup_{i=1}^{\infty}A_i\right) = \sum_{i=1}^{\infty}\mu\left(A_i\right).
	\]
	\item
\end{enumerate}
\end{thm}


\section{Abstract measure spaces}
\begin{defn}
\label{def:measurableSpace}A pair $\left(\mathcal{X},\mathcal{F}\right)$,
where $\mathcal{X}$ is an arbitrary set and $\mathcal{F}$ is a $\sigma-$algebra
on $\mathcal{X}$, is called a \emph{measurable space.}
\end{defn}

Although we had implicitly defined a measure in the previous section,
it's appopriate to write down a formal definition in this section.
\begin{defn}
\label{def:measureSpace}Let $\left(\mathcal{X},\mathcal{F}\right)$
be a measurable space. A function $\mu:\mathcal{F}\longrightarrow\left[0,\infty\right]$
is a \emph{measure }on $\mathcal{X}$ if

\begin{enumerate}[label=(\roman*),leftmargin=.1\linewidth,rightmargin=.4\linewidth]
	\item $\mu\left(\emptyset\right)= 0$
	\item For disjoint $\{A_i\}_{i\in \N} \in \mathcal{F}$ 
	\[
			\mu\left(\bigcup_{i=1}^{\infty}A_i\right) = \sum_{i=1}^{\infty}\mu\left(A_i\right).
	\]
\end{enumerate}The triple $\left(\mathcal{X},\mathcal{F},\mu\right)$ is called a
\emph{measure space. }If $\mu\left(\mathcal{X}\right)=1$ then $\mu$
is called a \emph{probability measure }and $\left(\mathcal{X},\mathcal{F},\mu\right)$
is called a \emph{probability space.}
\end{defn}

\begin{defn}
\label{def:measurableSet}Given a measurable space $\left(\mathcal{X},\mathcal{F}\right)$,
any set $A\in\mathcal{F}$ is called a \emph{measurable }set. Conversely,
any set $A\subset\mathcal{X}$ such that $A\notin\mathcal{F}$ is
referred to as a \emph{non-measurable }set.
\end{defn}

While the definition of a measure is simple, it turns out to have
some remarkable properties that are useful in the theory of integration
and probability that is built on top of measure theory (or, as we
shall later see, is equivalent to it).
\begin{prop}
\label{prop:measureProperties}Let $\left(\mathcal{X},\mathcal{F}\right)$
be a measurabe space and let 
\[
\mu:\mathcal{F}\longrightarrow\left[0,\infty\right]
\]
be a function. Then $\mu$ is a measure if and only if

\begin{enumerate}[label=(\roman*),leftmargin=.1\linewidth,rightmargin=.4\linewidth]
	\item $\mu\left(\emptyset\right)= 0$
	\item For disjoint $A,B \in \mathcal{F}$ 
	\[
			\mu\left(A \cup B\right) = \mu\left(A\right) + \mu\left(B\right) .
	\]
	\item For any increasing sequence of sets $ A_1 \subseteq A_2 \ldots $ in $\mathcal{F}$ such that $\bigcup_{i\in\N} A_i = A $
	\[
			\mu\left(A\right) = \lim_{i \to \infty}\mu\left(A_i\right)
	\]
\end{enumerate}
\end{prop}

\begin{proof}
First we shall establish that Definition \ref{def:measureSpace} implies
properties (i)-(iii) above. Property (i) is inherited straight from
the definition; to see (ii), we can let $A_{1}=A,A_{2}=B$ and $A_{j}=\emptyset$
for all $j\geq3$. Then
\[
\mu\left(A\cup B\right)=\mu\left(\bigcup_{j\in\N}A_{j}\right)=\sum_{j=1}^{\infty}\mu\left(A_{j}\right)=\mu\left(A\right)+\mu\left(B\right)
\]
where the second equality is due countably additiivity and the third
equality is due to property (i). To see property (iii), let $\left\{ A_{i}\right\} _{i\in\N}$
be an increasing sequence of sets such that $A_{i}\subseteq A_{i+1}$
for every $i\in\N$ and let $A:=\bigcup_{i\in N}A_{i}$. Define 
\[
B_{i}:=A_{i}\setminus\bigcup_{j=1}^{i-1}A_{j}
\]
which is the standard ``disjointification'' of $\left\{ A_{i}\right\} _{i\in\N}$
as we have seen earlier. By countable additivity
\begin{align*}
\mu\left(A\right) & =\sum_{i=1}^{\infty}\mu\left(B_{i}\right)\\
 & =\lim_{n\to\infty}\sum_{i=1}^{n}\mu\left(B_{i}\right)\\
 & =\lim_{n\to\infty}\mu\left(\bigcup_{i=1}^{n}B_{i}\right)\\
 & =\lim_{n\to\infty}\mu\left(\bigcup_{i=1}^{n}A_{i}\right)\\
 & =\lim_{n\to\infty}\mu\left(A_{n}\right)
\end{align*}
where the third equality is due to property (ii). The fourth equality
follows from the disjointification and the last equality is due to
the increasing nature of the sequence of sets.

Next, we shall establish countable additivity while assuming properties
(i)-(iii) in order to complete the equivalence. Let $\left\{ A_{i}\right\} _{i\in\N}$
be pairwise disjoint in $\mathcal{F}$. Then, letting $A:=\bigcup_{i\in\N}A_{i}$
we can define
\[
B_{n}:=\bigcup_{i=1}^{n}A_{i}
\]
and observe that $\bigcup_{n\in\N}B_{n}=A$ and $B_{n}\subseteq B_{n+1}.$
Then, by property (iii), 
\begin{align*}
\mu\left(A\right) & =\lim_{n\to\infty}\mu\left(B_{n}\right)\\
 & =\lim_{n\to\infty}\sum_{i=1}^{n}\mu\left(A_{i}\right)\\
 & =\sum_{i=1}^{\infty}\mu\left(A_{i}\right)
\end{align*}
where the second equality is due to finite additivity (property (ii)).
This completes the proof.
\end{proof}
\begin{rem*}
Property (iii) resembles a continuity condition, and is indeed called
\emph{continuity from below }of measures. There is an analagous definition
for \emph{continuity from above} which is implied by \emph{continuity
from above }for finitely additive measures and pre-measures. If the
measures are finite, these two notions of continuity are in fact equivalent.
\end{rem*}
\begin{cor}
\label{cor:countableSubadditivity}Every measure $\mu$ on an arbitrary
measurable space $\left(\mathcal{X},\mathcal{F}\right)$ is countably
subadditive i.e. for any collection $\left\{ A_{i}\right\} _{i\in\N}\in\mathcal{F}$
\[
\mu\left(\bigcup_{i=1}^{\infty}A_{i}\right)\leq\sum_{i=1}^{\infty}\mu\left(A_{i}\right).
\]
\end{cor}

\begin{proof}
We shall first establish \emph{finite }subadditivity and bootstrap
this result to countable subadditivity. To see finite subadditivity,
let $A,B\in\mathcal{F}$ be arbitrary, and observe that
\[
A\cup B=\left(A\setminus B\right)\cup B.
\]
The two sets on the right hand side are disjoint and so by finite
additivity
\begin{align*}
\mu\left(A\cup B\right) & =\mu\left(A\setminus B\right)+\mu\left(B\right).
\end{align*}
Adding $\mu\left(A\cap B\right)$ and applying finite additivity again,
we deduce that
\[
\mu\left(A\cup B\right)+\mu\left(A\cap B\right)=\mu\left(A\right)+\mu\left(B\right)
\]
which establishes finite subadditivity. To prove the countable analogue,
let
\[
B_{n}:=\bigcup_{i=1}^{n}A_{i}
\]
and observe that by finite subadditivity
\[
\mu\left(B_{n}\right)\leq\sum_{i=1}^{n}\mu\left(A_{i}\right)\leq\sum_{i=1}^{\infty}\mu\left(A_{i}\right)
\]
where the last inequality follows by the non-negativity of $\mu.$
Note that since $B_{n}$ is an increasing sequence, we can apply Proposition
\ref{prop:measureProperties} (iii) to infer that
\[
\mu\left(\bigcup_{i=1}^{\infty}A_{i}\right)=\lim_{n\to\infty}\mu\left(B_{n}\right)\leq\sum_{i=1}^{\infty}\mu\left(A_{i}\right).
\]
\end{proof}
\begin{prop}
\label{prop:equivalenceContinuityMeasures}For a finitely additive
measure $\mu:\mathcal{F}\longrightarrow\left[0,\infty\right),$ the
following statements are equivalent:

\begin{enumerate}[label=(\roman*),leftmargin=.1\linewidth,rightmargin=.4\linewidth]
	\item For any increasing sequence of sets $\left\{ A_{i}\right\} _{i\in\N}$ such that $A_i \subseteq A_{i+1}$ for all $i\in \N$
	\[
					\mu\left(\bigcup_{i=1}^{\infty} A_i\right) = \lim_{i\to\infty}\mu\left(A_i\right).
	\]
	\item For any decreasing sequence of sets $\left\{ A_{i}\right\} _{i\in\N}$ such that $ A_{i+1}\subseteq A_i$ for all $i\in \N$ 
	\[
					\mu\left(\bigcap_{i=1}^{\infty} A_i\right) = \lim_{i\to\infty}\mu\left(A_i\right).
	\]
\end{enumerate}
\end{prop}

\begin{proof}
Assuming (i), let $\left\{ A_{i}\right\} _{i\in\N}$ be a decreasing
sequence of sets and let $A:=\bigcap_{i\in\N}A_{i}$. Then define
$B_{i}=A_{1}\setminus A_{i}$ which is an increasing sequence of sets
such that $A_{1}\setminus A=\bigcup_{i\in\N}B_{i}$. By (i), 
\[
\mu\left(A_{1}\right)-\mu\left(A\right)=\mu\left(A_{1}\setminus A\right)=\lim_{i\to\infty}\mu\left(B_{i}\right)=\mu\left(A_{1}\right)-\lim_{i\to\infty}\mu\left(A_{i}\right)
\]
where the first and last equality are due to finite additivity, the
finiteness of $\mu.$ We can subtract $\mu\left(A_{1}\right)$ from
both sides to yield the result.

To establish the converse, assume (ii) and let $\left\{ A_{i}\right\} _{i\in\N}$
be an increasing sequence of sets and define $A:=\bigcup_{i\in\N}A_{i}$.
Let $B_{i}:=A\setminus A_{i}$ which is a decreasing sequence of sets
such that $\bigcap_{i\in\N}B_{i}=\emptyset$. By (ii), we have that
\[
0=\mu\left(\emptyset\right)=\lim_{i\to\infty}\mu\left(B_{i}\right)=\lim_{i\to\infty}\mu\left(A\setminus A_{i}\right)=\mu\left(A\right)-\lim_{i\to\infty}\mu\left(A_{i}\right)
\]
where the last equality is again due to finite additivity and the
finitenesss of $\mu.$ Rearrangement yields the proof.
\end{proof}
Observe how the two results apply without modification to pre-measures
as well and so we can establish the countable additivity of $\lambda_{2}$
(see the previous section) using a continuity argument instead of
the Heine-Borel argument we previously used (Exercise!).
\begin{prop}
\label{prop:sumOfCountableMeasures}Let $\left(\mathcal{X},\mathcal{F}\right)$
be a measurable space and let $\left\{ \mu_{i}\right\} _{i\in\mathcal{I}}$
be a collection of measures on $\mathcal{F}$ where $\mathcal{I}$
is at most countable. Then
\[
\mu:=\sum_{i\in\mathcal{I}}\mu_{i}
\]
is a measure on $\mathcal{F}$.
\end{prop}

\begin{proof}
First observe that 
\[
\mu\left(\emptyset\right)=\sum_{i\in\mathcal{I}}\mu_{i}\left(\emptyset\right)=0.
\]
Next, let $\left\{ A_{j}\right\} _{j\in\N}\in\mathcal{F}$ be disjoint.
Then
\begin{align*}
\mu\left(\bigcup_{j\in\N}A_{j}\right) & =\sum_{i\in\mathcal{I}}\mu_{i}\left(\bigcup_{j\in\N}A_{j}\right)\\
 & =\sum_{i\in\mathcal{I}}\sum_{j\in\N}\mu_{i}\left(A_{j}\right)\\
 & =\sum_{j\in\N}\sum_{i\in\mathcal{I}}\mu_{i}\left(A_{j}\right)\\
 & =\sum_{j\in\N}\mu\left(A_{j}\right)
\end{align*}
where the second equality follows from the countable additivity of
$\mu_{i}$ and the third equality follows from the non-negativity
of measures and Lemma \ref{lem:TonelliForSeries}. This completes
the proof.
\end{proof}

\subsection{$\sigma-$finite measure spaces}
\begin{defn}
\label{def:sigmaFinite}Let $\left(\X,\F,\mu\right)$ be a measure
space. The measure $\mu$ is said to be $\sigma-$\emph{finite }if
there exists some increasing sequence of sets $\left\{ E_{i}\right\} _{i\in\N}\in\F$
such that $\mu\left(E_{i}\right)<\infty$ and
\[
\bigcup_{i\in\N}E_{i}=\X.
\]

\hl{Describe the Lebesgue measure on real line as a canonical example of a sigma-finite measure}
\end{defn}

\begin{prop}
\label{prop:equivSigmaFinite}Let $\left(\X,\F,\mu\right)$ be a measure
space. The measure $\mu$ being $\sigma-$finite is equivalent to
any of the following conditions

\begin{enumerate}[label=(\roman*),leftmargin=.1\linewidth,rightmargin=0.15\linewidth]
	\item There exists some \textbf{pairwise disjoint} countable collection of sets $\{A_i\}_{i\in\N} \in \F$ such that $ \mu\left(A_i\right) < \infty $ for all $ i \in \N $ and
	\[
					\bigcup_{i\in\N}A_i = \X
	\]
	\item There exists some countable collection of sets $\{B_i\}_{i\in\N} \in \F$ such that $ \mu\left(B_i\right) < \infty $ for all $ i \in \N $ and
	\[
					\bigcup_{i\in\N}B_i = \X
	\]
\end{enumerate}
\end{prop}

\begin{proof}
First assume that the measure $\mu$is $\sigma-$finite and so there
exists some increasing sequence $\left\{ E_{i}\right\} _{i\in\N}\in\F$
such that $E_{i}\subseteq E_{I+1}$, $\mu\left(E_{i}\right)<\infty$
and 
\[
\bigcup_{i\in\N}E_{i}=\X.
\]
Recall the disjointification
\[
A_{i}:=E_{i}\setminus\bigcup_{j=1}^{i-1}E_{j}
\]
and notice that
\begin{align*}
\mu\left(A_{i}\right) & =\mu\left(E_{i}\right)-\mu\left(\bigcup_{j=1}^{i-1}E_{j}\right)\\
 & =\mu\left(E_{i}\right)-\mu\left(E_{i-1}\right)\\
 & <\infty
\end{align*}
where the first equality follows from the fact that $\mu\left(E_{i}\right)<\infty$
and $\bigcup_{j=1}^{i-1}E_{j}=E_{i-1}\subseteq E_{i}$ along with
(finite) additivity. Further,
\[
\bigcup_{i\in\N}A_{i}=\X
\]
 and so Definition \ref{def:sigmaFinite} implies (i).

Next notice that (i) trivially implies (ii) and so all we just need
to verify (ii) $\implies$Definition \ref{def:sigmaFinite}. To this
end, observe that if $\left\{ B_{i}\right\} _{i\in\N}\in\F$ is an
arbitrary collection that satisfies (ii), then
\[
E_{n}:=\text{\ensuremath{\bigcup_{i=1}^{n}B_{i}}}
\]
is an increasing sequence of sets $E_{n}\subseteq E_{n+1}$ such that
\[
\mu\left(E_{n}\right)\leq\sum_{i=1}^{n}\mu\left(B_{i}\right)<\infty
\]
and
\[
\bigcup_{n\in\N}E_{n}=\X.
\]
\end{proof}
\begin{prop}
\label{prop:sumSigmaFiniteMeasures}Let $\left(\X,\F\right)$ be a
measurable space and let $\left\{ \mu_{i}\right\} _{i=1}^{N}$be a
finite collection of $\sigma-$finite measure on $\F.$ Then the total
measure
\[
\mu:\F\longrightarrow\R
\]
 given by
\[
\mu\left(A\right):=\sum_{i=1}^{N}\mu_{i}\left(A\right)
\]
is also $\sigma-$finite.
\end{prop}

\begin{proof}
A weaker variant of Proposition \ref{prop:sumOfCountableMeasures}
shows that $\mu$is at least a measure on $\F$. We show $\sigma-$finiteness
for $N=2$; the general case follows by induction. Note that if $\mu_{1}$
and $\mu_{2}$ are both $\sigma-$finite then by Proposition \ref{prop:equivSigmaFinite}
there exist $\left\{ E_{1,i}\right\} _{i\in\N},\left\{ E_{2,i}\right\} _{i\in\N}\in\F$
such that $\mu_{1}\left(E_{i,1}\right)<\infty$ and $\mu_{2}\left(E_{2,i}\right)<\infty$
for all $i\in\N$. Further,
\[
\bigcup_{i\in\N}E_{1,i}=\bigcup_{i\in\N}E_{2,i}=\X.
\]
Then, define
\[
C_{i,j}:=E_{1,i}\bigcap E_{2,j}
\]
and observe that $C_{i,j}\in\F$ and that 
\begin{align*}
\mu\left(C_{i,j}\right) & =\mu_{1}\left(E_{1,i}\cap E_{2,j}\right)+\mu_{2}\left(E_{1,i}\cap E_{2,j}\right)\\
 & \leq\mu_{1}\left(E_{1,i}\right)+\mu_{2}\left(E_{2,j}\right)\\
 & <\infty
\end{align*}
for all $\left(i,j\right)\in\N^{2}$. Finally,
\[
\bigcup_{i\in\N}\bigcup_{j\in\N}C_{i,j}=\X
\]
which by Proposition \ref{prop:equivSigmaFinite} establishes the
result.
\end{proof}

