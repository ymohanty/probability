
\chapter{Design of experiments\label{chap:designOfExperiments}}
\begin{example}
\label{exa:isi2004samplepsb5}Here is a partial key-block of a $2^{4}$
factorial experiment (with factors $A,B,C,D)$ conducted in two blocks
of size 8 each: Partial key-block: $ad\quad bd\quad c\quad\ldots$
Search out the other five treatment combinations for the key-block
and also the confounded interaction. Also, give the treatment combination
of the second block.\hl{TODO}
\end{example}

\begin{example}
\label{exa:isi2006samplepsb10}For the data collected via a randomized
block design with $v$ treatments and $b$ blocks, the following model
is postulated: 
\[
\E\left(y_{ij}\right)=\mu+\tau_{i}+\beta_{j},\quad1\leq i\leq v,1\leq j\leq b,
\]
 where $t_{i}$ and $\beta_{j}$ are the effects of the ith treatment
and the jth block respectively, and $\mu$ is a general mean. For
$1\leq i\leq v$, define $Q_{i}=T_{i}-\frac{G}{v}$, where $T_{i}$
is the total of observations under the ith treatment and $G=\sum_{i=1}^{v}T_{i}$.
Show that 
\[
\begin{array}{r}
E\left(Q_{i}\right)=\left(b-\frac{b}{v}\right)\tau_{i},\quad\operatorname{Var}\left(Q_{i}\right)=\sigma^{2}\left(b-\frac{b}{v}\right),\\
\operatorname{Cov}\left(Q_{i},Q_{j}\right)=-\left(\frac{b}{v}\right)\sigma^{2}\text{ for }i\neq j,
\end{array}
\]
 where $\sigma^{2}$ is the per observation variance.\hl{TODO}
\end{example}

\begin{example}
\label{exa:isi2008samplepsb12}Consider a randomized block experiment
with 4 treatments and 3 replicates (blocks) and let $\tau_{i}$ be
the effect of the $i$ th treatment $(1\leq i\leq4)$. Find all possible
covariances between the least squares estimators of the following
treatment contrasts: (a) $\tau_{1}-\tau_{2}$ (b) $\tau_{1}+\tau_{2}-2\tau_{3}$
(c) $\tau_{1}+\tau_{2}+\tau_{3}-3\tau_{4}$.\hl{TODO}
\end{example}

\begin{example}
	\label{exa:isi2009samplepsb12}
	If $\tau_i$ denotes the effect due to the $i$ th treatment of a latin square design of order $m \times m$, then derive a test for testing the null hypothesis $H_0: \tau_i-2 \tau_j+\tau_k=0$. Also obtain a $100(1-\alpha) \%$ confidence interval for $\tau_i-2 \tau_j+\tau_k$.
	\hl{TODO}
\end{example}


