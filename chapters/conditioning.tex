
\chapter{Conditioning\label{chap:conditioning}}

\section{Elementary notions of conditional probability}

The basic notion of conditional probabilities is clear when we are
conditioning of events of positive probability. That is, on a probability
space $\probabilityspace$ with an event $B\in\F$ such that $\P\left(B\right)>0$,
we can easily define a conditional measure
\begin{equation}
\P_{B}\left(A\right):=\frac{\P\left(A\cap B\right)}{\P\left(B\right)}\label{eq:conditionalProbabilityPositive}
\end{equation}
\footnote{Sometimes we write $\P\left(A\mid B\right)$ instead of $\P_{B}\left(A\right)$
if the event $B$ is actually some long expression like ``$\left\{ X=1\right\} ".$}for any $A\in\F$. It's easy to verify that $\P_{B}$ is in fact a
probability measure on $\F$ and thus we can derive the corresponding
integral 
\[
\E_{B}\left[f\right]=\frac{\E\left[f\indicate_{B}\right]}{\P\left(B\right)}.
\]
This definition quickly yields some of the most foundational and basic
probabilistic ideas: for instance with two events $A,B$ where both
occur with positive probability, they are independent if and only
if $\P\left(A\right)=\P_{B}\left(A\right).$The following basic facts
\begin{lem}
\label{lem:totalProbability}Let $\probabilityspace$ be a probability
space and let $B\in\F$ be an event. Then for any countable partition
$\left\{ A_{i}\right\} _{i\in\N}$ of $\Omega$ such that $\P\left(A_{i}\right)>0$,
we have that 
\[
\P\left(B\right)=\sum_{i=1}^{\infty}\P\left(A_{i}\right)\P_{A_{i}}\left(B\right).
\]
\end{lem}

\begin{proof}
Note that by countable additivity
\begin{align*}
\P\left(B\right) & =\sum_{i=1}^{\infty}\P\left(B\cap A_{i}\right)\\
 & =\sum_{i=1}^{\infty}\P\left(A_{i}\right)\P_{A_{i}}\left(B\right).
\end{align*}
This immediately yields the famous Bayes Theorem.
\end{proof}
\begin{thm}
\label{thm:bayesTheorem}Let $\probabilityspace$ be a probability
space and let $\left\{ A_{i}\right\} _{i\in\N}$be a partition of
$\Omega$ such that $\P\left(A_{i}\right)>0$. Then, for any $A,B\in\F$
with positive probability,
\[
\P_{B}\left(A\right)=\frac{\P_{A}\left(B\right)\P\left(A\right)}{\sum_{i=1}^{\infty}\P\left(A_{i}\right)\P_{A_{i}}\left(B\right)}.
\]
\end{thm}

Of course, we are not satisfied with a theory of conditional probabilities
where we restrict the conditioning event to one with positive probability.
After all, we are often interested in conditioning on events like
``$\left\{ X=a\right\} "$ where $X$ is a continuous random variable
and $a$ is a real number. For instance, consider the problem of choosing
a point uniformly at random on the unit square. What is the probability
that the point has an $x-$coordinate greater than $\frac{1}{2}$
if it lies on the diagonal? We can formally model this problem by
taking two independent random variables $X,Y\sim U\left[0,1\right]$
and computing $\P\left(X>\frac{1}{2}\mid X=Y\right)$. Of course,
the event $X=Y$ has probability zero as can be verified by Tonelli's
theorem. Intuitively, we can guess that the answer should be $\frac{1}{2}$
since we should effectively have a uniform distribution on the diagonal
in this setting. It turns out that the answer is not as straightforward
as we would expect.
\begin{example}
\label{exa:borelKolmogorov}Let 
\end{example}

\begin{example}
\label{exa:isi2004samplepsb3}Suppose a random vector $(X,Y)$ has
joint probability density function 
\[
f(x,y)=3y
\]
 on the triangle bounded by the lines $y=0,y=1-x$, and $y=1+x$.
Compute $\E\left(Y\left\lvert \,X\leq\frac{1}{2}\right.\right)$.\hl{TODO}
\end{example}

\begin{example}
\label{exa:isi2007samplepsb7}Let $X$ and $Y$ be i.i.d. exponentially
distributed random variables with mean $\lambda>0$. Define $Z$ by:
\[
Z=\begin{cases}
1 & \text{ if }X<Y\\
0 & \text{ otherwise. }
\end{cases}
\]
Find the conditional mean $\E\left[X\mid Z=1\right]$. \hl{TODO}
\end{example}

\begin{example}
\label{exa:isi2008samplepsb7}Let $X$ and $Y$ be exponential random
variables with parameters 1 and 2 respectively. Another random variable
$Z$ is defined as follows.

A coin, with probability $p$ of Heads (and probability $1-p$ of
Tails) is tossed. Define $Z$ by 
\[
Z=\begin{cases}
X & \text{ if the coin turns Heads }\\
Y & \text{ if the coin turns Tails }
\end{cases}
\]

Find $P(1\leq Z\leq2)$.\hl{TODO}
\end{example}


\section{Kolmogorov conditional expectations}

The abstract formulation of the conditional expectation was quite
unintuitive when it was first put forward by Kolmogorov (1933) \hl{add ref}
as the right way to think about conditioning.
\begin{example}
\label{exa:isi2005samplepsb11}Let $X$ and $Y$ be two random variables
with joint probability density function 
\[
f(x,y)=\begin{cases}
1 & \text{ if }-y<x<y,0<y<1\\
0 & \text{ elsewhere }
\end{cases}
\]

Find the regression equation of $Y$ on $X$ and that of $X$ on $Y$.\hl{TODO}
\end{example}


\section{Regular conditional probabilities}
