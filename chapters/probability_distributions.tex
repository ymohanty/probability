
\chapter{Common probability distributions\label{chap:probabilityDistributions}}

\section{General families of distributions}

\subsection{Location-scale families}

\subsection{General exponential families}

\subsection{Stable distributions}

\subsection{Infinitely divisible distributions}

\subsection{Power series distributions}

\section{Special parametric families of distributions}

\subsection{Normal distributions and their associates}

\subsubsection{The univariate normal distribution}

\subsubsection{The multivariate normal distribution}
\begin{example}
\label{exa:isi2008samplepsb8}Let $\underline{Y}=\left(Y_{1},Y_{2}\right)^{\prime}$
have the bivariate normal distribution $N_{2}(\underline{0},\Sigma)$,
where 
\[
\Sigma=\left(\begin{array}{cc}
\sigma_{1}^{2} & \rho\sigma_{1}\sigma_{2}\\
\rho\sigma_{1}\sigma_{2} & \sigma_{2}^{2}
\end{array}\right).
\]

Obtain the mean and variance of $U=\underline{Y^{\prime}}\Sigma^{-1}\underline{Y}-Y_{1}^{2}/\sigma_{1}^{2}$.\hl{TODO}
\end{example}


\subsubsection{The lognormal distribution}

\subsubsection{The folded normal distribution}

\subsubsection{The Rayleigh distribution}

\subsubsection{The Maxwell distribution}

\subsubsection{The Levy distribution}

\subsection{Distributions useful for basic statistical inference}

\subsubsection{The Gamma distribution}

\begin{example}
\label{exa:isi2009samplepsb3}
Using an appropriate probability distribution or otherwise show that
$$
\lim _{n \rightarrow \infty} \int_0^n \frac{\exp (-x) x^{n-1}}{(n-1)!} d x=\frac{1}{2} .
$$
\hl{TODO}
\end{example}

\subsubsection{The Chi-squared distributtion}

\subsubsection{Student's $t$ distribution}

\subsubsection{The $F$ distribution}

\subsection{Continuous distributions with bounded support}

\subsubsection{The uniform distribution\label{subsec:uniformDistribution}}

The uniform distribution on the interval $\left[a,b\right]$ is the
simplest example of a distribution that is absolutely continuous with
respect to the Lebesgue measure. In fact, its density is $\frac{1}{b-a}\indicate_{\left[a,b\right]}$
which means the distribution is simply the restriction of the Lebesgue
measure to the interval $\left[a,b\right]$ , with an appropriate
normalization to ensure that it is a probability measure. A random
variable $X$ distributed uniformly on $\left[a,b\right]$ is often
denoted $X\sim U\left[a,b\right]$. The CDF of such an $X$ is $F_{X}\left(x\right)=\lebInt{\lambda}{\frac{1}{b-a}\indicate_{\left[a,x\right]}}=\frac{x-a}{b-a}\indicate\left\{ a\leq x\leq b\right\} +\indicate\left\{ x>b\right\} .$
The moments can be computed easily.
\begin{prop}
\label{prop:momentsUniformDistribution}Let $\probabilityspace$ be
a probability space and let $X\sim U\left[a,b\right]$. Then,
\begin{align*}
\E\left[X^{k}\right] & =\frac{b^{k+1}-a^{k+1}}{(k+1)\left(b-a\right)}.\\
\end{align*}
\end{prop}

\begin{proof}
Note that by \hyperref[cor:changeOfVariables]{change of variables}
and the fundamental theorem of calculus
\begin{align*}
\E\left[X^{k}\right] & =\lebInt{\lambda_{x}}{x^{k}\frac{1}{b-a}\indicate\left[a,b\right]}\\
 & =\frac{1}{b-a}\left.\frac{x^{k+1}}{k+1}\right|_{a}^{b}\\
 & =\frac{b^{k+1}-a^{k+1}}{(k+1)\left(b-a\right)}.
\end{align*}
\end{proof}
\begin{prop}
\label{prop:mgfUniformDistribution}Let $\probabilityspace$ be a
probability space and let $X\sim U\left[a,b\right]$. Then,
\begin{align*}
M_{X}\left(t\right) & =\frac{e^{bt}-e^{at}}{t\left(b-a\right)}\indicate\left\{ t\neq0\right\} +\indicate\left\{ t=0\right\} .
\end{align*}
\end{prop}

\begin{proof}
Suppose $t\neq0$, in which case
\begin{align*}
M_{X}\left(t\right) & =\E\left[e^{tX}\right]\\
 & =\frac{1}{b-a}\lebInt{\lambda_{x}}{e^{tx}\indicate\left[a,b\right]}\\
 & =\frac{1}{b-a}\frac{1}{t}\left.e^{tx}\right|_{a}^{b}\\
 & =\frac{e^{tb}-e^{ta}}{t\left(b-a\right)}.
\end{align*}
 The other case is trivial.
\end{proof}
Other interesting moments of the distribution are the variance, skewness
and kurtosis.The variance is found easily by the identity 
\begin{align*}
\Var\left[X\right] & =\E\left[X^{2}\right]-\left(\E\left[X\right]\right)^{2}\\
 & =\frac{b^{3}-a^{3}}{3\left(b-a\right)}-\left(\frac{a+b}{2}\right)^{2}\\
 & =\frac{\left(a+b\right)^{2}-ab}{3}-\frac{\left(a+b\right)^{2}}{4}\\
 & =\frac{\left(b-a\right)^{2}}{12}.
\end{align*}
Similarly, the skewness is given
\begin{align*}
\mathrm{skew}\left(X\right) & =\E\left[\left(\frac{X-\mu}{\sigma}\right)^{3}\right]\\
 & =
\end{align*}


\subsubsection{The Beta distribution}

\subsubsection{The Beta Prime distribution}

\subsubsection{The arcsine distribution}

\subsubsection{The semicircle distribution}

\subsubsection{The triangle distribution}

\subsubsection{The Irwin-Hall distribution}

\subsection{Continuous distributions with positve support}

\subsubsection{Exponential-logarithmic distribution}

\subsubsection{The Gompertz distribution}

\subsubsection{The Log-logistic distribution}

\subsubsection{The Pareto distribution}

\subsubsection{The Wald distribution}

\subsubsection{The Weibull distribution}

\subsection{Continuous distributions supported on the real line}

\subsubsection{The Laplace distribution}

\subsubsection{The logistic distribution}

\subsubsection{The exreme value distribution}

\subsubsection{The hyperbolic secant distribution}

\subsubsection{The Cauchy distribution}

\subsection{Distributions associated with modeling Bernoulli trials}

\subsubsection{The Bernoulli distribution\label{subsec:bernoulliDistribution}}

The Bernoulli distribution is the simplest of all discrete probability
distributions. It represents the mathematical abstraction of coin
tossing with not-necessarily fair coins.
\begin{defn}
\label{def:bernoulliDistribution}A random variable $X$ on probability
space $\probabilityspace$ is said to have a Bernoulli distribution
with parameter $p\in\left[0,1\right]$ if $\P\left(X=1\right)=p$
and $\P\left(X=0\right)=1-p$.
\end{defn}

Formally, any indicator variable of an event $A\in\F$ is a Bernoulli
random variable. However, we typically reserve this description for
random variables used to model the outcome of a binary experiment
trial. Usually we are interested in the \emph{sequence }of such trials.
\begin{prop}
\label{prop:bernoulliMoments}The raw moments of a Bernoulli random
variable $X$ with parameter $p$ is given 
\[
\E\left[X^{k}\right]=p
\]
for any $k\in\N$. The central moments are
\[
\E\left[\left(X-p\right)^{k}\right]=\left(1-p\right)^{k}p+\left(1-p\right)\left(-p\right)^{k}
\]
\end{prop}

\begin{proof}
Note that 
\begin{align*}
\E\left[\left(X-p\right)^{k}\right] & =\left(1-p\right)^{k}\P\left(X=1\right)+\left(0-p\right)^{k}\P\left(X=0\right)\\
 & =\left(1-p\right)^{k}p+\left(1-p\right)\left(-p\right)^{k}.
\end{align*}
The raw moment case is simpler and follows exactly in the same way.
\end{proof}
\begin{prop}
\label{prop:mgfBernoulli}The moment generating function of a Bernoulli
random variable $X$ with parameter $p$ is given
\[
M_{X}\left(t\right)=1-p+pe^{t}.
\]
\end{prop}

\begin{proof}
Again follow the same approach as in \ref{prop:bernoulliMoments}.
\end{proof}

\subsubsection{The Binomial distribution\label{subsec:binomialDistribution}}

The Binomial distribution is perhaps the most well known of all discrete
distributions. It models the number of successes in a fixed number
of trials (say coin tosses) and as such is a defined as a sum of Bernoulli
trials.
\begin{defn}
\label{def:binomialDistribution}A random variable $Y$ has a Binomial
distribution with parameters $n$ and $p$ if 
\[
Y=\sum_{i=1}^{n}X_{i}
\]
where $X_{i}$ are independent and identically distributed Bernoulli
random variables with parameter $p$. In this case we write $Y\sim\textrm{Bin}\left(n,p\right)$
\end{defn}

\begin{prop}
\label{prop:binomialCDF}The mass function of a random variable $Y$
with a Binomial distribution with parameters $n$ and $p$ is given
by
\[
\P\left(Y=k\right)=\left(\begin{array}{c}
n\\
k
\end{array}\right)p^{k}\left(1-p\right)^{n-k}\indicate\left\{ 0\leq k\leq n\right\} 
\]
\end{prop}

\begin{proof}
For $n=2$, we use the result about convolutions. More specifically,
by Corollary \ref{cor:discreteConvolution}
\begin{align*}
\P\left(Y=k\right) & =\sum_{x\in\left\{ 0,1\right\} }\P\left(X_{1}=x\right)\P\left(X_{2}=k-x\right)\\
 & =\left(1-p\right)\left[p\indicate\left\{ k=1\right\} +\left(1-p\right)\indicate\left\{ k=0\right\} \right]\\
 & \ \ \ \!\ +p\left[p\indicate\left\{ k=2\right\} +\left(1-p\right)\indicate\left\{ k=1\right\} \right]\\
 & =p^{2}\indicate\left\{ k=2\right\} +2p\left(1-p\right)\indicate\left\{ k=1\right\} +\left(1-p\right)^{2}\indicate\left\{ k=0\right\} \\
 & =\left(\begin{array}{c}
2\\
k
\end{array}\right)p^{k}\left(1-p\right)^{2-k}\indicate\left\{ 0\leq k\leq2\right\} .
\end{align*}
Now for the induction step, assume that the result holds for $n$
and then we write $Y=\sum_{i=1}^{n+1}X_{i}=Z+X_{n+1}$ where $Z\sim\textrm{Bin}\left(n-1,p\right)$.
Then, another convolution argument shows
\begin{align*}
\P\left(Y=k\right) & =\sum_{z=0}^{n}\P\left(Z=z\right)\P\left(X_{n+1}=k-z\right)\\
 & =\sum_{z=0}^{n}\left(\begin{array}{c}
n\\
z
\end{array}\right)p^{z}\left(1-p\right)^{n-z}\indicate\left\{ 0\leq z\leq n\right\} \left(p\indicate\left\{ z=k-1\right\} +\left(1-p\right)\indicate\left\{ z=k\right\} \right)\\
 & =\sum_{z=0}^{n}\left(\begin{array}{c}
n\\
z
\end{array}\right)p^{z+1}\left(1-p\right)^{n-z}\indicate\left\{ 0\leq z\leq n,z=k-1\right\} \\
 & \ \ \!\ +\sum_{z=0}^{n}\left(\begin{array}{c}
n\\
z
\end{array}\right)p^{z}\left(1-p\right)^{n-z+1}\indicate\left\{ 0\leq z\leq n,z=k\right\} \\
 & =\left(\begin{array}{c}
n\\
k-1
\end{array}\right)p^{k}\left(1-p\right)^{n-k+1}+\left(\begin{array}{c}
n\\
k
\end{array}\right)p^{k}\left(1-p\right)^{n-k+1}\\
 & =\left(\begin{array}{c}
n+1\\
k
\end{array}\right)p^{k}\left(1-p\right)^{n+1-k}
\end{align*}
 where we used \hyperref[prop:pascalRule]{Pascal's rule} in the last
equality.
\end{proof}
The moments of the Binomial distribution are easily characterized
using the Bernoulli moments and the multinomial theorem
\begin{prop}
\label{prop:momentsBinomial}Let $\probabilityspace$ be a probability
space and let $Y\sim\mathrm{Bin}\left(n,p\right)$ for some $n\in\N$
and $p\in\left[0,1\right]$. Then,
\[
\E\left[Y^{k}\right]=\sum_{i_{1}+i_{2}+\ldots+i_{n}=k,i_{j}\geq0}\left(\begin{array}{cccc}
 & k\\
i_{1}, & i_{2}, & \ldots & i_{n}
\end{array}\right)p^{\sum_{j=1}^{n}\indicate\left\{ i_{j}\geq1\right\} }.
\]
\end{prop}

\begin{proof}
Note that $Y=\sum_{i=1}^{n}X_{i}$ where $X_{i}$ are i.i.d Bernoulli
random variables with parameter $p$. Then
\begin{align*}
\E\left[Y^{k}\right] & =\E\left[\left(\sum_{i=1}^{n}X_{i}\right)^{k}\right]\\
 & =\E\left[\sum_{i_{1}+i_{2}+\ldots+i_{n}=k,i_{j}\geq0}\left(\begin{array}{cccc}
 & k\\
i_{1}, & i_{2}, & \ldots & i_{n}
\end{array}\right)\prod_{j=1}^{n}X_{j}^{i_{j}}\right]\\
 & =\sum_{i_{1}+i_{2}+\ldots+i_{n}=k,i_{j}\geq0}\left(\begin{array}{cccc}
 & k\\
i_{1}, & i_{2}, & \ldots & i_{n}
\end{array}\right)\E\left[\prod_{j=1}^{n}X_{j}^{i_{j}}\right]\\
 & =\sum_{i_{1}+i_{2}+\ldots+i_{n}=k,i_{j}\geq0}\left(\begin{array}{cccc}
 & k\\
i_{1}, & i_{2}, & \ldots & i_{n}
\end{array}\right)\prod_{j=1}^{n}\E\left[X_{j}^{i_{j}}\right]\\
 & =\sum_{i_{1}+i_{2}+\ldots+i_{n}=k,i_{j}\geq0}\left(\begin{array}{cccc}
 & k\\
i_{1}, & i_{2}, & \ldots & i_{n}
\end{array}\right)p^{\sum_{j=1}^{n}\indicate\left\{ i_{j}\geq1\right\} }
\end{align*}
where in the second equality we have used the \hyperref[thm:multinomialTheorem]{multinomial theorem},
the third is linearity of integration, the fourth is Proposition \ref{prop:indepExpectationFactors},
and the last equality uses Proposition \ref{prop:bernoulliMoments}.
\end{proof}
The first moment is $\E\left[Y\right]=np$ which can be seen just
by linearity and the definition. The second moment is
\begin{align*}
\E\left[Y^{2}\right] & =\E\left[\sum_{i=1}^{n}X_{i}^{2}+\sum_{i=1}^{n}\sum_{j=1,i\neq j}^{n}X_{i}X_{j}\right]\\
 & =np+n\left(n-1\right)p^{2}.
\end{align*}
We are more interested in higher \emph{central }moments like the variance.
Of course, $\Var\left[Y\right]=\E\left[Y^{2}\right]-\left(\E\left[Y\right]\right)^{2}=np+n\left(n-1\right)p^{2}-n^{2}p^{2}=np-np^{2}=np\left(1-p\right).$
The moment generating function is can be computed in a similar way.
\begin{prop}
\label{prop:mgfBinomial}Let $\probabilityspace$ be a probability
space and let $Y\sim\mathrm{Bin}\left(n,p\right)$ for some $n\in\N$
and $p\in\left[0,1\right]$. Then,
\[
M_{Y}\left(t\right)=\left(1-p+pe^{t}\right)^{n}.
\]
\end{prop}

\begin{proof}
Note that $Y=\sum_{i=1}^{n}X_{i}$ where as before $X_{i}$ are iid
Bernoulli with parameter $p$ and so
\begin{align*}
M_{Y}\left(t\right) & =\E\left[e^{tY}\right]\\
 & =\E\left[e^{t\sum X_{i}}\right]\\
 & =\E\left[\prod_{i=1}^{n}e^{tX_{i}}\right]\\
 & =\prod_{i=1}^{n}\E\left[e^{tX_{i}}\right]\\
 & =\left(1-p+pe^{t}\right)^{n}
\end{align*}
wheree the fourth equality is again due to Proposition \ref{prop:indepExpectationFactors}
and the last is Proposition \ref{prop:mgfBernoulli}.
\end{proof}

\subsubsection{The geometric distribution}

\subsubsection{The negative binomial distribution}

\subsubsection{The multinomial distribution}

\subsubsection{The discrete arcine distribution}

\subsubsection{The Beta-binomial distribution}

\subsubsection{The Beta-negative binomial distribution}

\subsubsection{The discrete uniform distribution\label{subsec:discreteUniformDistribution}}

The discrete uniform distribution is the generalization of the distribution
we encountered in Example \ref{exa:floorFuncCDF}. The idea is that
it is a distribution supported on any finite set, where each element
of the set has equal probability. Thus for a finite set $A$ and a
discrete uniform random variable $X$ supported on $A$ we have that
$\P\left(X=k\right)=\frac{1}{\lvert A\rvert}\indicate\left\{ k\in A\right\} .$
Typically, the set $A=\left\{ a,a+1,a+2,\ldots,b\right\} $ in which
case the $\lvert A\rvert=b-a+1$. In the discussion below we will
focus on the special case where $a=0$ and $b=n-1$ for some $n\in\N$.
\begin{prop}
\label{prop:discreteUniformCDF}Let $\probabilityspace$ be a probability
space and suppose $X$ is a discrete uniform distribution supported
on $A=\left\{ 0,1,\ldots,n-1\right\} $. Then, the CDF of $X$ is
given
\[
F_{X}\left(x\right)=\frac{\lfloor x\rfloor+1}{n}\indicate\left\{ x\in\left[0,n-1\right]\right\} +\indicate\left\{ x>n-1\right\} .
\]
\end{prop}

\begin{proof}
Observe
\begin{align*}
F_{X}\left(x\right) & =\E\left[\indicate\left\{ X\leq x\right\} \right]\\
 & =\frac{1}{n}\sum_{a=0}^{n-1}\indicate\left\{ a\leq x\right\} \\
 & =\frac{\lfloor x\rfloor+1}{n}\indicate\left\{ x\in\left[0,n-1\right]\right\} +\indicate\left\{ x>n-1\right\} .
\end{align*}
\end{proof}
\begin{prop}
\label{prop:momentsDiscreteUniform}Let $\probabilityspace$ be a
probability space and suppose $X$ is a discrete uniform distribution
supported on $A=\left\{ 0,1,\ldots,n-1\right\} $. Then for any $k\in\N$
\[
\E\left[X^{k}\right]=\frac{1}{n}\sum_{a=0}^{n-1}a^{k}.
\]
In particular 
\[
\E\left[X\right]=\frac{1}{2}\left(n-1\right)
\]
and 
\[
\E\left[X^{2}\right]=\frac{\left(n-1\right)\left(2n-1\right)}{6}
\]
and so
\[
\Var\left[X\right]=\frac{n^{2}-1}{12}.
\]
\end{prop}

\begin{proof}
This is a standard application of Corollary \ref{cor:changeOfVariables}.
The specific formulas for the first and second moments follow from
partial sum formulas for consecutive integers and squares, respectively.
The variance then is given by $\Var\left[X\right]=\E\left[X^{2}\right]-\left(\E\left[X\right]\right)^{2}$.
\end{proof}
\begin{prop}
\label{prop:mgfDiscreteUniform}Let $\probabilityspace$ be a probability
space and let $X$ be a discrete uniform random variable supported
on $A=\left\{ 0,1,\ldots,n-1\right\} $. Then, the MGF of $X$ is
given
\[
M_{X}\left(t\right)=\frac{1-e^{nt}}{n\left(1-e^{t}\right)}\indicate\left\{ t\neq0\right\} +\indicate\left\{ t=0\right\} .
\]
\end{prop}

\begin{proof}
Observe that
\begin{align*}
M_{X}\left(t\right) & =\E\left[e^{tX}\right]\\
 & =\frac{1}{n}\sum_{a=0}^{n-1}e^{ta}\\
 & =\frac{1}{n}\sum_{a=0}^{n-1}\left(e^{t}\right)^{a}\\
 & =\frac{1-e^{nt}}{n\left(1-e^{t}\right)}\indicate\left\{ t\neq0\right\} +\indicate\left\{ t=0\right\} 
\end{align*}
by the geometric partial sum formula.
\end{proof}

\subsection{Distributions associated with finite sampling models}

\subsubsection{The hypergeometric distribution}

\subsubsection{The multivariate hypergeometric distribution}

\subsubsection{The matching distribution}

\subsubsection{The birthday distribution}

\subsubsection{The coupon collector distribution}

\subsubsection{The Polya distribution}

\subsection{Distributions associated with the Poisson process}

\subsubsection{The exponential distribution}

\subsubsection{The Erlang distribution}

\subsubsection{The Poisson distribution}
