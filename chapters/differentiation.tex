
\chapter{\label{chap:Differentiation}Differentiation}

In elementary calculus, the theory of the Riemann integral was developed
with a corresponding theory of differentiation, and these two operations
were shown to be the inverse of each other for a suitable class of
functions, namely the differentiable functions; this is the fundamental
theorem of calculus. The Lebesgue theory also implies a fundamental
theorem of calculus for a broader class of functions. The goal of
this chapter is to lay the groundwork for this result -- the full
proof of which will have to be deferred to Chapter \ref{chap:productMeasures}--
and in order to do so, we will develop some theory that is very important
in its own right. Before delving in this material, you should review differentiation on the line in Appendix section \ref{subsec:reviewDifferentiation}. In the first section here, I will suitably generalize the notion of a derivative by defining
differentiability as a property of functions on arbitrary normed vector
spaces.

\section{Differentiation in normed vector spaces\label{sec:differentiation}}



\section{Decomposition of measures}

Recall that in Proposition \ref{prop:densities}, we showed that for
a measure space $\left(\X,\F,\mu\right)$ and a non-negative measurable
function $f\in\nonnegMeasurableFunctions$ , the function $\nu\left(A\right):=\lebInt{\mu}{f\indicate_{A}}$is
a measure. Moreover, in the remark following the proposition, we observed
that such a measure $\nu$ was \emph{absolutely continuous }with respect
to $\mu$, that is to say, for any $A\in\F,\mu(A)=0\implies\nu(A)=0.$
Further, we also claimed that every absolutely continuous measure
could be represented this way i.e as an integral with respect to the
dominating measure. In other words, we claimed that there was always
an $f\in\nonnegMeasurableFunctions$ such that this relation would
hold. It turns out we can establish a slightly stronger result, which
in turn is key to establishing the fundamental theorem of calculus
for the Lebesgue measure. What we can show, using the Hilbert space
machinery we developed earlier, is that we can decompose a measure
(relative to another measure) into two measures that are in some sense
orthogonal. This is in direct correspondence with the decomposition
of Hilbert spaces given to us by the projection theorem (Theorem \ref{thm:projectionThm}).

To motivate how the Hilbert space theory can be useful in this context,
we can begin with a change of perspective in the spirit of Theorem
\ref{thm:integralMeasureEquivalence}. Note that saying $\nu(A)=\lebInt{\mu}{f\indicate_{A}}$
for every $A\in\F$ is equivalent to saying $\lebInt{\nu}g=\lebInt{\mu}{fg}\forall g\in\Lp 1{\X,\F,\nu}.$
The right hand side is the inner product associated with the space
$\Lp 2{\mu}$ and so we wish to show that a linear functional $g\to\lebInt{\nu}g$
can be represented as an inner product on $\Lp 2{\mu}.$ At this juncture,
one can guess that the Riesz representation theorem for Hilbert spaces
is clearly right tool to finish off the proof; however, several hurdles
remain. For one, $g\to\lebInt{\nu}g$ a linear functional on $\Lp 1{\nu}$
and it is not $\Lp 2{\mu}.$ Further, it is unclear whether our functional
is bounded, which is a necessary condition for the theorem to apply.
The following lemma provides the conditions under which the theorem
finds purchase.
\begin{lem}
\label{lemma:LebesgueRadonNikodym}Let $\left(\X,\F\right)$ be a
measurable space. If $\mu,\nu$ are two $\sigma-$finite measures
on $\F$, then there exists some non-negative (almost everywhere with
respect to $\mu$) measurable function $k\in\nonnegMeasurableFunctions$
and a set $G\in\F$ such that $\mu(G)=0$ and
\[
\nu\left(A\right)=\lebInt{\mu}{k\indicate_{A}}+\nu(A\cap G)
\]
for every $A\in\F$. Moreover, the function $k$ is unique $\mu-$almost-everywhere
and the set $G$ is unique $\mu+\nu$-almost everywhere.
\end{lem}

\begin{proof}
Suppose first that the measures $\mu$and $\nu$are finite. Define
the measure $\psi:=\mu+\nu$ and observe that by Proposition \ref{prop:integralSumOfMeasures}
\[
\lebInt{\psi}f=\lebInt{\mu}f+\lebInt{\nu}f.
\]
Next define an operator 
\[
\Gamma:\Lp 2{\psi}\longrightarrow\R
\]
by
\[
\Gamma\left(f\right):=\lebInt{\nu}f
\]
and notice that $\Gamma$ is a continuous linear functional by a slight
variant of the argument presented in the proof of Corollary \ref{cor:integrationContinuousLinearFunctional},
using the fact that $\mu\left(\X\right),\nu\left(\X\right)<\infty$.
Since $\mathcal{L}^{2}\left(\psi\right)$ is a Hilbert space by Corollary
\ref{cor:L2Hilbert} and Theorem \ref{thm:completenessLp}, the \hyperref[thm:rieszRep]{Riesz representation theorem} tells
us there exists an (almost $\psi-$everywhere unique) function $h\in\Lp 2{\psi}$
such that
\begin{align*}
\Gamma\left(f\right) & =\innerproduct fh
\end{align*}
for every $f\in\Lp 2{\psi}.$ Of course, in the context of $\mathcal{L}^{2}$,
\begin{equation}
\innerproduct fh=\pnorm{fh}1=\lebInt{\psi}{fh}=\lebInt{\mu}{fh}+\lebInt{\nu}{fh}.\label{eq:rieszRepresenterL2}
\end{equation}
Next consider the following measurable partition of $\X$
\[
N:=\left\{ x\in\X\mid h\left(x\right)<0\right\} ,M:=\left\{ x\in\X\mid0\leq h\left(x\right)<1\right\} ,G:=\left\{ x\in\X\mid h\left(x\right)\geq1\right\} .
\]
Note that by the fact that $0\geq h\indicate_{N}$ and the monotonicity
of integration
\begin{align*}
0 & \geq\lebInt{\psi}{h\indicate_{N}}\\
 & =\lebInt{\mu}{h\indicate_{N}}+\lebInt{\nu}{h\indicate_{N}}\\
 & =\lebInt{\nu}{\indicate_{N}}
\end{align*}
where the last equality is due to (\ref{eq:rieszRepresenterL2}).
But since $\lebInt{\nu}{\indicate_{N}}=\nu\left(N\right),$we have
$\nu\left(N\right)=0$ by non-negativity of measures. Then
\[
\lebInt{\mu}{h\indicate_{N}}+\lebInt{\nu}{h\indicate_{N}}=0
\]
where the second term is automatically zero since $h\indicate_{N}\stackrel{\nu-\text{a.e}}{=}0$.
Therefore, 
\[
\lebInt{\mu}{h\indicate_{N}}=0
\]
which by Proposition \ref{prop:intZeroFuncZero} implies that
\[
h\indicate_{N}\stackrel{\mu-\text{a.e}}{=}0\implies\mu\left(N\right)=0
\]
and so both $\mu\left(N\right)=\nu\left(N\right)=0.$

On the other hand, observe that 
\begin{align*}
\Gamma\left(\indicate_{G}\right) & =\nu\left(G\right)\\
 & =\lebInt{\mu}{h\indicate_{G}}+\lebInt{\nu}{h\indicate_{G}}\\
 & \geq\mu\left(G\right)+\nu\left(G\right)
\end{align*}
where the inequality follows from the fact that $h\indicate_{G}\geq\indicate_{G}$
and the monotonicity of the integral. Since $\nu\left(G\right)<\infty$
as $\nu$ is a finite measure, we can subtract if from both sides
of the inequality to deduce
\[
\mu\left(G\right)\leq0
\]
 which reduces to equality by the non-negativity of measures.

To control the final piece of the partition, observe that we can use
the definition $\Gamma\left(f\right)=\lebInt{\nu}f$ and Eq. (\ref{eq:rieszRepresenterL2})
together to deduce (since all terms are finite) that
\[
\lebInt{\nu}f-\lebInt{\nu}{fh}=\lebInt{\mu}{fh}
\]
for $f\in\Lp 2{\psi}.$ By linearity this reduces to~
\begin{equation}
\lebInt{\nu}{\left(1-h\right)f}=\lebInt{\mu}{hf}.\label{eq:oneminusH}
\end{equation}
Next we define the increasing sequence of sets
\[
M_{n}:=\left\{ x\in\X\mid0\leq h\left(x\right)\leq1-\frac{1}{n}\right\} 
\]
and note that
\[
M=\bigcup_{n\in\N}M_{n}.
\]
Further, observe that for any $A\in\F$ the functions
\[
f_{n}:=\frac{\indicate_{M_{n}}\indicate_{A}}{1-h}\leq n\indicate_{M_{n}}\indicate_{A}
\]
and
\[
0\leq f_{n}\leq f_{n+1}
\]
pointwise for all $n\in\N$. This shows us that our functions $f_{n}\in\Lp 2{\psi}$\footnote{$n\indicate_{M_{n}}$is square integrable since $\psi$ is a finite
measure; $f_{n}$ then is integrable since both functions are non-negative
and so 
\[
\lebInt{\psi}{\lvert f_{n}\rvert^{2}}\leq\lebInt{\psi}{\lvert n\indicate_{M_{n}}\indicate_{A}\rvert^{2}}\leq n^{2}\psi\left(\left\{ 0\leq h\leq1-\frac{1}{n}\right\} \right)
\]
} and are monotonically increasing. Therefore,
\begin{align*}
\nu\left(M\cap A\right) & =\lebInt{\nu}{\indicate_{M}\indicate_{A}\frac{1-h}{1-h}}\\
 & =\lim_{n\to\infty}\lebInt{\nu}{\frac{\indicate_{M_{n}}\indicate_{A}}{1-h}1-h}\\
 & =\lim_{n\to\infty}\lebInt{\mu}{\frac{\indicate_{M_{n}}\indicate_{A}}{1-h}h}\\
 & =\lebInt{\mu}{\indicate_{M}\indicate_{A}\frac{h}{1-h}}+\lebInt{\mu}{\indicate_{N\cup G}\indicate_{A}\frac{h}{1-h}}\\
 & =\lebInt{\mu}{\frac{h}{1-h}\indicate_{A}}
\end{align*}
where the second equality follows by the \hyperref[thm:generalizedMonotoneConvergence]{monotone convergence theorem},
the third equality is due to (\ref{eq:oneminusH}), the fourth equality
is due to the fact that $\mu\left(G\right)=\mu\left(N\right)=0$ and
the last equality is due to the linearity of integration. 

Letting $k=\frac{h}{1-h}$ we have that 
\[
\mu\left(\left\{ x\in\X\mid k\left(x\right)<0\right\} \right)=0
\]
since $\mu\left(N\right)=\mu\left(G\right)=0.$ Therefore
\begin{align*}
\nu\left(A\right) & =\nu\left(\left(A\cap N\right)\cup\left(A\cap M\right)\cup\left(A\cap G\right)\right)\\
 & =\nu\left(A\cap N\right)+\nu\left(A\cap M\right)+\nu\left(A\cap G\right)\\
 & =\lebInt{\mu}{k\indicate_{A}}+\nu(A\cap G)
\end{align*}
where the first equality follows from the fact that $N,M,$ and $G$
form a partition of $\X$, the second equality by finite additivity,
and the final equality by the fact that $N$ is a $\nu-$null set.

The extension to $\sigma-$finite measures is relatively straightforward.
Let $\left\{ E_{i}\right\} _{i\in\N}\in\F$ be a partition of $\X$
such that $\mu$and $\nu$are finite on each $E_{i}$ (this is possible
by Propositions \ref{prop:equivSigmaFinite} and \ref{prop:sumSigmaFiniteMeasures}).
Then the measures given by
\begin{align*}
\nu_{i}\left(A\right) & :=\nu\left(A\cap E_{i}\right),\\
\mu_{i}\left(A\right) & :=\mu\left(A\cap E_{i}\right)
\end{align*}
are finite on $\F$ and so the finite version of theorem implies the
existence of functions $k_{i}\in\nonnegMeasurableFunctions$ and $\mu_{i}-$null
sets $B_{i}$ such that 
\[
\nu_{i}\left(A\right)=\lebInt{\mu_{i}}{k_{i}\indicate_{A}}+\nu_{i}\left(A\cap B_{i}\right)
\]
which is equivalent to saying that 
\[
\nu\left(A\cap E_{i}\right)=\lebInt{\mu}{k_{i}\indicate_{A}\indicate_{E_{i}}}+\nu\left(A\cap B_{i}\cap E_{i}\right)
\]
for each $i\in\N$. Summing over $i$ we have 
\begin{align*}
\nu\left(A\right) & =\sum_{i=1}^{\infty}\nu\left(A\cap E_{i}\right)\\
 & =\sum_{i=1}^{\infty}\lebInt{\mu}{k_{i}\indicate_{E_{i}}\indicate_{A}}+\sum_{i=1}^{\infty}\nu\left(A\cap B_{i}\cap E_{i}\right)\\
 & =\lebInt{\mu}{\sum_{i=1}^{\infty}k_{i}\indicate_{E_{i}}\indicate_{A}}+\nu\left(A\cap\left(\cup_{i\in\N}\left(B_{i}\cap E_{i}\right)\right)\right).
\end{align*}
where the last equality follows by monotone convergence and countable
additivity. Note that $\mu\left(\bigcup_{i\in\N}\left(B_{i}\cap E_{i}\right)\right)=\sum_{i=1}^{\infty}\mu\left(B_{i}\cap E_{i}\right)=\sum_{i=1}^{\infty}\mu_{i}\left(B_{i}\right)=0.$
This completes the proof of existence with $k=\sum_{i=1}^{\infty}k_{i}\indicate_{E_{i}}$
and $B=\bigcup_{i\in\N}\left(B_{i}\cap E_{i}\right)$.

Finally, for uniqueness, consider two pairs $\left(k_{1},G_{1}\right)$
and $\left(k_{2},G_{2}\right)$ such that $k_{1},k_{2}\in\nonnegMeasurableFunctions,$
and $G_{1},G_{2}\in N_{\mu},$ and
\[
\nu\left(A\right)=\lebInt{\mu}{k_{1}\indicate_{A}}+\nu\left(A\cap G_{1}\right)=\lebInt{\mu}{k_{2}\indicate_{A}}+\nu\left(A\cap G_{2}\right)
\]
for every $A\in\F.$ Then,
\begin{align*}
\nu\left(G_{2}\cap G_{1}^{C}\right) & =\lebInt{\mu}{k_{1}\indicate_{G_{2}}\indicate_{G_{1}^{C}}}+\nu\left(\underbrace{G_{2}\cap G_{1}^{C}\cap G_{1}}_{=\emptyset}\right)\\
 & =0
\end{align*}
where the first term is zero because $G_{1}\in N_{\mu}.$ Similarly,
we can show that 
\begin{align*}
\nu\left(G_{1}\cap G_{2}^{C}\right) & =\lebInt{\mu}{k_{2}\indicate_{G_{1}}\indicate_{G_{2}^{C}}}+\nu\left(\underbrace{G_{1}\cap G_{2}^{C}\cap G_{2}}_{=\emptyset}\right)\\
 & =0.
\end{align*}
In other words, $\nu\left(G_{1}\Delta G_{2}\right)=0$ which by Proposition
\ref{prop:almostEverywhereEqualSets} implies that $G_{1}\stackrel{\nu-\text{a.e}}{=}G_{2}.$
Of course, since $G_{1},G_{2}\in N_{\mu},$ we have that $\mu\left(A\Delta B\right)=0$
and so $G_{1}\stackrel{\mu+\nu-\text{a.e}}{=}G_{2}.$ Note that since
$G_{1}^{C}\Delta G_{2}^{C}=G_{1}\Delta G_{2}$, we have that $G_{1}^{C}\stackrel{\mu+\nu-\text{a.e}}{=}G_{2}^{C}$
and so for any $A\in\F$
\[
\nu\left(A\cap G_{1}^{C}\right)=\nu\left(A\cap G_{2}^{C}\right).
\]
But
\begin{align*}
\nu\left(A\cap G_{1}^{C}\right) & =\lebInt{\mu}{k_{1}\indicate_{A}\indicate_{G_{1}^{C}}}+\nu\left(\underbrace{A\cap G_{1}^{C}\cap G_{1}}_{=\emptyset}\right)\\
\nu\left(A\cap G_{2}^{C}\right) & =\lebInt{\mu}{k_{2}\indicate_{A}\indicate_{G_{2}^{C}}}+\nu\left(\underbrace{A\cap G_{2}^{C}\cap G_{2}}_{=\emptyset}\right)
\end{align*}
This combined with the fact that $G_{1}$ and $G_{2}$ are $\mu-$null
show that for every $A\in\F$ we have $\lebInt{\mu}{k_{1}\indicate_{A}}=\lebInt{\mu}{k_{2}\indicate_{A}}$
which by Proposition \ref{prop:intEqualFuncEqual} implies that $k_{1}\stackrel{\mu-\text{a.e}}{=}k_{2}.$
\end{proof}
This is a powerful lemma as it gives us two very important results
as simple corollaries.
\begin{defn}
\label{def:absoluteContinuityMeasures}Let $(\X,\F)$ be a measurable
space and let $\mu,\nu$ be two measures on this space. We say $\nu$
is absolutely continuous with respect to $\mu$ if for any $A\in F:\mu(F)=0\implies\nu(F)=0$.
We denote this relation by $\nu<<\mu$.

Why is this relation labeled continuity? There are two good reasons
for this: first, at least for finite measures, the above definition
is equivalent to an $\epsilon-\delta$ definition that resembles the
continuity definitions that we have seen before. More importantly,
we will see that measures that are absolutely continuous with respect
to the Lebesgue measure give rise to a particular class of real-valued
\emph{functions }with the property of absolute continuity. Such functions
are exactly the class of functions for which the Lebesgue fundamental
theorem of calculus applies.
\end{defn}

\begin{prop}
\label{prop:epsdeltaAbsContinuity}Let $(\X,\F)$ be a measurable
space and let $\mu,\nu$ be two measures on this space such that $\nu(X)<\infty$.
Then $\nu<<\mu$ if and only if for every $\epsilon>0$ there exists
a $\delta>0$ such that for any $A\in\F$, $\mu(A)<\delta\implies\nu(A)<\epsilon$
\end{prop}

\begin{proof}
Suppose that $\epsilon-\delta$ characterization holds. Take a sequence
$\epsilon_{n}>0$ such that $\epsilon_{n}\to0$ and find the corresponding
$\delta_{n}>0$ such that $\mu(A)<\delta_{n}\implies\nu(A)<\epsilon_{n}$.
If $\mu(A)=0$ then $\mu(A)<\delta_{n}$ for all $n\in\N$. Then $\nu(A)<\epsilon_{n}$
for every $n.$ Taking limits yields $\nu(A)=0.$

Conversely, assume that $\epsilon-\delta$ property does not hold.
Then, there exists some $\epsilon_{0}>0$ such that for any $\delta>0$,
there's some $A\in\F$ such that $\mu(A)<\delta$ but $\nu(A)\geq\epsilon$.
Let $\delta_{n}=\frac{1}{2^{n}}\to0$ and let $A_{n}\in\F$ be the
corresponding sequence of sets such that $\mu(A_{n})<\delta_{n}$
but $\nu(A_{n})\geq\epsilon.$ Note then $\sum_{n=1}^{\infty}\mu(A_{n})<1$
and so by the \hyperref[thm:borelCantelli]{Borel-Cantelli lemma},
\[
\mu(\limsup_{n\to\infty}A_{n})=0.
\]
where $\limsup_{n\to\infty}A_{n}=\cap_{n=1}^{\infty}\cup_{i=1}^{n}A_{i}\in\F$.
Since $\nu(X)<\infty,$ by the \hyperref[cor:reverseFatouLemma]{reverse Fatou lemma}
\[
\nu(\limsup_{n\to\infty}A_n)\geq\limsup_{n\to\infty}\nu(A_{n})\geq\epsilon
\]
which completes the proof.
\end{proof}
While this is one way to characterize absolutely continuous measures,
a different characterization provides the shortest route to the fundamental
theorem of calculus.
\begin{thm}[Radon-Nikodym]
\label{thm:radonNikodym}Let $\left(\X,\F\right)$ be a measurable
space. If $\mu,\nu$ are two $\sigma$-finite measures on $\F$ then
$\nu<<\mu$ if and only if there exists almost-everywhere unique (with
respect to both measures) non-negative function $f\in\nonnegMeasurableFunctions$
such that 
\[
\nu(A)=\lebInt{\mu}{f\indicate_{A}}
\]
for every $A\in\F$
\end{thm}

\begin{proof}
Note that the existence of $f$ implies absolute continuity by Proposition
\ref{prop:densities}. The converse is far more challenging; fortunately
for us, Lemma \ref{lemma:LebesgueRadonNikodym} does all of the work.
To see this, note that by our lemma

\[
\nu(A)=\lebInt{\mu}{\tilde{f}\indicate_{A}}+\nu(A\cap B)
\]
where $A\in\F$, $B\in N_{\mu}$ and $\tilde{f}$ is $\mu-$almost
everywhere non-negative and unique. If $\nu<<\mu$ then $\nu(A\cap B)=0$.
Letting $f=\tilde{f}\indicate_{{\tilde{f}\geq0}}$, our result follows
by Proposition \ref{prop:funcEqualityAlmostEverywhere}.
\end{proof}
\begin{cor}
\label{cor:radonNikodymIntegral}Let $\left(\X,\F\right)$ be a measurable
space. If $\mu,\nu$ are two $\sigma$-finite measures on $\F$ such
that $\mu<<\nu$ then there exists a unique $f\in\nonnegMeasurableFunctions$
such that for any $g\in\nonnegMeasurableFunctions$ 
\[
\lebInt{\mu}g=\lebInt{\nu}{gf}.
\]
\end{cor}

\begin{proof}
(Sketch) This is the standard approximation argument. For indicator
functions the result follows by the Radon-Nikodym theorem. For any
simple function, it follows by the linearity of the integral (and
the result for indicators). Finally for non-negative measurable function
it follows by monotone convergence (and the result for simple functions).
Uniqueness follows by Proposition \ref{prop:intEqualFuncEqual}.
\end{proof}
The Radon-Nikodym theorem builds on the Hilbert space theory we described
earlier in Chapter \ref{chap:spaces_of_functions} and is the proper
converse of the comparatively simple result on constructing new measures
using non-negative measurable functions that we saw in Proposition
\ref{prop:densities}. This theorem is also fundamental to developing
conditional expectations; indeed the existence and uniqueness of conditional
expectations is an almost trivial corollary of the Radon-Nikodym theorem.
In the context of probability theory, the functions described by the
Radon-Nikodym theorem are probability density functions. Recall from
undergraduate probability that probability density functions could
be recovered as derivatives of cumulative distribution functions,
provided those distribution functions were sufficiently well behaved.
It turns out such densities can be thought of as derivatives generally
in an admittedly contrived sense: if we write the Radon Nikodym theorem
in traditional notation, then we have can write for $\nu<<\mu$
\[
\nu\left(A\right)=\int_{A}fd\mu.
\]
Then, in Leibniz notation, we could write $f$ as $\frac{d\nu}{d\mu}$
and refer to $f$ as the \emph{Radon-Nikodym derivative }of $\nu$
with respect to $\mu$. Of course, for function to actually be some
sort of derivative of measures, we need to be able to represent it
as a limiting ratio of the two measures; in the special case of the
Lebesgue measure, we can indeed do this. However, the notation is
useful more generally, partly due to the following facts
\begin{prop}
\label{prop:RadonNikodymFacts}Let $\measurablespace$ be a measurable
space and let $\mu,\nu$, and $\gamma$be $\sigma-$finite measures
on the space. Then

\begin{enumerate}[label=(\roman*),leftmargin=.1\linewidth,rightmargin=.4\linewidth]
	\item If $\mu<<\nu$ and $\nu<<\gamma$  then $\mu << \gamma$  and
\begin{equation}
\tag{Chain rule}
	\frac{d\mu}{d\gamma}\stackrel{\text{a.e}}{=}\frac{d\mu}{d\nu}\frac{d\nu}{d\gamma}
\label{eq:chainRule}
\end{equation}
	\item  If $\mu << \gamma $ and $\nu << \gamma $ then the sum measure $\mu + \nu$ is $\sigma-$finite, $\mu + \nu << \gamma$, and 
\begin{equation}
\tag{Sum Rule}
	\frac{d\mu+\nu}{d\gamma}\stackrel{\gamma-\text{a.e}}{=}\frac{d\mu}{d\gamma}+\frac{d\nu}{d\gamma}
\label{eq:sumRule}
\end{equation}
	\item If $\mu << \nu$ then $\nu << \mu$ if and only if $\frac{d\mu}{d\nu}>0$ $\nu-$a.e and then
\begin{equation}
\tag{Inverse ``function" rule}
	\frac{d\nu}{d\mu}=\frac{1}{\frac{d\mu}{d\nu}}.
\label{eq:invFunctionRule}
\end{equation}
\end{enumerate}
\end{prop}

\begin{proof}
To show \emph{(i),} let $A\in\F$ be such that $\gamma\left(A\right)=0.$Then
$\nu\left(A\right)=0$ since $\nu<<\gamma.$ Then $\mu\left(A\right)=0$
since $\mu<<\nu$ which establishes that absolute continuity is transitive.
Note that since the measures are $\sigma-$finite, by the Radon Nikodym
theorem there exists some almost everywhere unique non-negative measurable
functions $f,g,h$ such that for any $A\in\F$
\begin{align*}
\mu\left(A\right) & =\lebInt{\gamma}{h\indicate_{A}}\\
\mu\left(A\right) & =\lebInt{\nu}{f\indicate_{A}}\\
\nu\left(A\right) & =\lebInt{\gamma}{g\indicate_{A}}
\end{align*}
and so by Corollary \ref{cor:densityIntegral}, we have that - for
any $A\in\F$
\[
\mu\left(A\right)=\lebInt{\nu}{f\indicate_{A}}=\lebInt{\gamma}{fg\indicate_{A}}.
\]
Since the function $fg$ is non-negative measurable, we have by the
uniqueness of Radon-Nikodym derivatives that 
\[
h\stackrel{\text{a.e}}{=}fg
\]
which completes the proof.

For \emph{(ii), }note that $\mu+\nu$ is $\sigma-$finite by Proposition
\ref{prop:sumSigmaFiniteMeasures}. Absolute continuity follows trivially.
To show the almost sure equivalence of the Radon-Nikodym derivatives,
let $f,g,h\in\nonnegMeasurableFunctions$ be Radon-Nikodym derivatives
such that for every $A\in\F$
\begin{align*}
\mu\left(A\right)+\nu\left(A\right) & =\lebInt{\gamma}{h\indicate_{A}}\\
\mu\left(A\right) & =\lebInt{\gamma}{f\indicate_{A}}\\
\nu\left(A\right) & =\lebInt{\gamma}{g\indicate_{A}}.
\end{align*}
By linearity of the Lebesgue integral, we have that for every $A\in\F$
\[
\lebInt{\gamma}{\left(f+g\right)\indicate_{A}}=\lebInt{\gamma}{h\indicate_{A}}.
\]
By Proposition \ref{prop:intEqualFuncEqual} we have that 
\[
h\stackrel{\gamma-\text{a.e}}{=}f+g.
\]
Finally, to show \emph{(iii), }let $f$ denote $\frac{d\mu}{d\nu}$
and suppose that that $f\stackrel{\nu-\text{a.e}}{>}0$ and that $\mu\left(A\right)=0$
for some $A\in\F.$ The Radon-Nikodym theorem implies then that $\lebInt{\nu}{f\text{}\indicate_{A}}=0$.
Since $f$ is positive almost everywhere, Proposition \ref{prop:intZeroFuncZero}
implies that $\indicate_{A}\stackrel{\nu-\text{a.e}}{=}0\Longleftrightarrow\nu\left(A\right)=0.$
Conversely, suppose $\nu<<\mu$ and there's a set of positive (with
respect to both measures) mass $B\in\F$ such that $f=0$ on $B$.
Then $\mu\left(B\right)=\lebInt{\nu}{f\indicate_{B}}=0$ which is
a contradiction. To conclude, observe that when $\frac{d\mu}{d\nu}\stackrel{\text{a.e}}{>}0$,
both measures are \emph{mutually }absolutely continuous (or equivalent)
and so $\frac{d\nu}{d\mu}$ exists and by part \emph{(i) }above
\[
1=\frac{d\mu}{d\mu}=\frac{d\mu}{d\nu}\frac{d\nu}{d\mu}.
\]
Taking reciprocals yields the result.
\end{proof}
Absolute continuity is a special property between pairs of measures
characterized by the notion that the null sets of a measure are a
subset of the other; that is to say $\mu<<\nu\Longleftrightarrow N_{\nu}\subseteq N_{\mu}$.
Of course, if the two measures $\mu,\nu$ are mututally absolutely
continuous (or equivalent), then $N_{\mu}=N_{\nu}.$ Dual to this
notion of absolute continuity, we can define the concept of mutual
singularity, which corresponds to the situation where $N_{\nu}$ and
$N_{\mu}$ can together cover $\X$
\begin{defn}
\label{def:concentrationOfMeasure}Let $\measurespace$ be a measure
space. The measure $\mu$ is said to \emph{concentrate }on a set $A\in\F$
if $\mu\left(E\right)=\mu\left(A\cap E\right).$ Equivalenty, one
can say that $\mu$ concentrates on $A$ if $\mu\left(E\right)=0$
if and only if $E\cap A=\emptyset.$
\end{defn}

\begin{defn}
\label{def:mutualSingularity}Let $\measurablespace$ be a measurable
space. Two measures $\mu,\nu$ on $\F$ are said to be \emph{mutually
singular }if there exists two disjoint sets $A,B\in\F$ such that
$\mu$ concentrates on $A$ and $\nu$ concentrates on $B$. We can
then write $\mu\perp\nu$.
\end{defn}

In the next result we summarize some basic properties of mutual singularity
and its relationship with absolute continuity.
\begin{prop}
\label{prop:propertiesSingularityAbsContinuity}Let $\measurablespace$
be a measurable space and let $\mu,\nu,$ and $\gamma$ be three measures
on $\F$. Then

\begin{enumerate}[label=(\roman*),leftmargin=.1\linewidth,rightmargin=.4\linewidth]
	\item $\mu \perp \nu$ if and only if there exist two disjoint sets $A,B \in \F$ such that $A \cup B= \X$ and $\mu(A) = 0 , \nu(A) = 0 $ .
	\item  If $\mu \perp \gamma $ and $\nu \perp \gamma $ then $\mu + \nu \perp \gamma$.
	\item If $\mu << \gamma $ and $\nu \perp \gamma$ then $\mu \perp \nu$.
	\item If $\mu << \nu $ and $\mu \perp \nu$ then $\mu = 0$.
\end{enumerate}
\end{prop}

\begin{proof}
For \emph{(i), }observe that if $\mu\perp\nu$ then there exist disjoint
sets $C,D\in\F$ such that $\mu$ concentrates on $C$ and $\nu$
concentrates on $D.$ By definition, we have that $\mu\left(C^{C}\right)=\mu\left(C\cap C^{C}\right)=0$
and $\nu\left(D^{C}\right)=\nu\left(D\cap D^{C}\right)=0.$ Note that
$\nu\left(C\right)=\nu\left(C\cap D\right)=0$ and so $A=C^{C}$and
$B=D^{C}$ does the trick. Conversely, let $A,B$ be as given in the
hypothesis. Then $\mu\left(F\right)=\mu\left(F\cap A\right)+\mu\left(F\cap B\right)=\mu\left(F\cap B\right)$
and similarly $\nu\left(F\right)=\nu\left(F\cap A\right)$ for every
$F\in\F$.

For \emph{(ii), }let $A_{1},B_{1}\in\F$ be a partition of $\X$ such
that $\mu\left(A_{1}\right)=0$ and $\gamma\left(B_{1}\right)=0.$
Similarly, let $A_{2},B_{2}\in\F$ be a partition of $\X$ such that
$\nu\left(A_{2}\right)=0$ and $\gamma\left(B_{2}\right)=0.$ Let
$C=A_{1}\cap A_{2}$ and $D=\left(A_{1}\cap A_{2}\right)^{C}=B_{1}^{C}\cup B_{2}^{C}.$
Clearly, $C$ and $D$ form a partition of $\X$ such that $\mu+\nu\left(A_{1}\cap A_{2}\right)=\mu\left(A_{1}\cap A_{2}\right)+\nu\left(A_{1}\cap A_{2}\right)=0$
by subadditivity. Similarly, $\gamma\left(B_{1}\cup B_{2}\right)\leq\gamma\left(B_{1}\right)+\gamma\left(B_{2}\right)=0.$

Next, for \emph{(iii) }let $A\in\F$ such that $\nu\left(A\right)=0$
and $\gamma\left(A^{C}\right)=0$. Absolute continuity implies that
$\mu\left(A^{C}\right)=0.$ This shows $\mu\perp\nu.$

Finally, suppose that $A\in\F$ such that $\mu\left(A\right)=0$ and
$\nu\left(A^{C}\right)=0.$ By absolute continuity, $\mu\left(A\right)=0.$
Then, $\mu\left(\X\right)=\mu\left(A\right)+\mu\left(A^{C}\right)=0.$
\end{proof}
It turns out that absolute continuity and singularity represent a
sort of \emph{orthogonality }notion for measures. Considering that
the spaces of measures is a subset of the dual space of $L^{1}$ functions,
this does not correspond to the canonical notion of orthogonality.
\hl{MAKE PRECISE USING TARCSAY 2014. }Again, this result falls trivially
out of Lemma \ref{lemma:LebesgueRadonNikodym}.
\begin{thm}[Lebesgue Decomposition]
\label{thm:lebesgueDecomposition}Let $\measurablespace$ be a measurable
space and let $\mu$ and $\nu$ be two $\sigma-$finite measures on
$\F$. Then there exist two unique measures $\mu_{1},\mu_{2}$ such
that 
\[
\mu=\mu_{1}+\mu_{2}
\]
and $\mu_{1}<<\nu$ and $\mu_{2}\perp\nu$. Such a pair $\left(\mu_{1},\mu_{2}\right)$
is called a Lebesgue decomposition of $\mu$ with respect to $\nu$.
\end{thm}

\begin{proof}
Note that existence follows from Lemma \ref{lemma:LebesgueRadonNikodym}.
To see this, write 
\[
\mu\left(A\right)=\lebInt{\nu}{k\indicate_{A}}+\mu\left(A\cap B\right)
\]
where $B\in N_{\nu}$ and $k$ is non-negative almost-everywhere with
respect to $\nu.$ Then setting $\text{\ensuremath{\mu_{1}\left(A\right):=\lebInt{\nu}{k\indicate_{A}}}}$we
have that $\mu_{1}<<\nu$. Similarly, setting $\mu_{2}\left(A\right):=\mu\left(A\cap B\right)$
which is a measure since it satisfies countable additivity and is
null on the empty set. Then note that $\nu\left(B\right)=0$ and $\mu_{2}\left(B^{C}\right)=0$
which implies that $\mu_{2}\perp\nu$.
\end{proof}
Note that we postpone the proof of the uniqueness of the decomposition
to the following subsection, since we shall need the concept of a
signed measure.

\subsection{Signed measures, Duality of $L^{p}$ spaces, and the Riesz representation
theorem.}

\hl{Tao (epsilon of room vol 1, section 1.3.2)?}

\section{Absolutely continuous functions}

The absolute continutiy of measures is in some sense a generalization
of the notion of absolute continuity of a real-valued function on
$\left[a,b\right]\subset\R$.
\begin{defn}
\label{def:absolutelyContinuousFunction}A function $f:\left[a,b\right]\to\R$
is said to be \emph{absolutely continuous }if for every $\epsilon>0$,
there exists some $\delta>0$ such that for any finite collection
of disjoint open intervals $\left\{ \left(a_{i},b_{i}\right)\right\} _{i=1}^{n}\subset\left[a,b\right]$,
\[
\sum_{i=1}^{n}\left(b_{i}-a_{i}\right)<\delta\implies\sum_{i=1}^{n}\lvert f\left(b_{i}\right)-f\left(a_{i}\right)\rvert<\epsilon.
\]
\end{defn}

It should be immediately clear that every absolutely continuous function
is uniformly continuous (and hence continuous). The converse is not
true, however: the \hyperref[def:cantorFunction]{Cantor function}
is a canonical example of a function that is uniformly continuous
but not absolutely continuous. The Cantor function is continuous on
$\left[0,1\right]$ by Proposition \ref{prop:cantorFunctionContinuous}
and hence uniformly continuous since $\left[0,1\right]$ is compact
(see Theorem\ref{thm:compactUniformContinuity}). One the other hand,
the Cantor function ``concentrates'' on $C$, in that the image
of the Cantor set under the Cantor function is all of $\left[0,1\right]$,
and so for any finite collection $\left\{ \left(a_{i}b_{i}\right)\right\} _{i=1}^{n}$
of disjoint intervals that covers the entire Cantor set, the corresponding
image sequence $\sum_{i=1}^{n}\lvert f\left(b_{i}\right)-f\left(a_{i}\right)\rvert\geq1.$
Of course, since the Cantor set has measure zero, it can be covered
by countably many open intervals whose total length is arbitrarily
small. Moreover, since the Cantor set is compact, we can extract a
finite subcollection of such intervals (disjointly, without loss of
generality) and so absolute continuity fails for any $0<\epsilon<1$.

The fact that the Cantor function concentrates on $C$ should be telling;
the measure that the Cantor function corresponds to is singular with
respect to the Lebesgue measure. As we shall see, the duality of singular
and absolutely continuous functions mimics the duality of singular
and absolutely continuous measures with respect to the Lebesgue measure.
We start with the following fact that links absolute continuity of
Stieljes functions to the absolute continuity of the measures induced
by such functions with respect to the Lebesgue measure.
\begin{prop}
\label{prop:absoluteContinuityStieljesFunctions}Let $F:\left[a,b\right]\to\R$
be a Stieljes function. Then the measure $\mu_{F}\left(\left(x,y\right]\right):=F\left(y\right)-F\left(x\right)$
for $a\leq x\leq y\leq b$ on $\borel\left(\left[a,b\right]\right)$
is absolutely continuous with respect to the Lebesgue measure if and
only if $F$ is absolutely continuous as a function.
\end{prop}

\begin{proof}
Note that both $\mu_{F}$ and $\lambda$ are finite measures on $\borel\left(\left[a,b\right]\right)$
and so if $\mu_{F}\ll\lambda$ then (by Proposition \ref{prop:epsdeltaAbsContinuity})
for any $\epsilon>0$ there exists some $\delta>0$ such that for
any $B\in\borel\left(\left[a,b\right]\right):\lambda\left(B\right)<\delta\implies\mu_{F}\left(B\right)<\epsilon.$
In particular, for $B=\bigcup_{i=1}^{n}\left(a_{i},b_{i}\right)$
where $\left(a_{i},b_{i}\right)\subset\left[a,b\right]$ are disjoint,
we have that 
\[
\lambda\left(\bigcup_{i=1}^{n}\left(a_{i},b_{i}\right)\right)>\delta\implies\mu_{F}\left(\bigcup_{i=1}^{n}\left(a_{i},b_{i}\right)\right)<\epsilon
\]
and applying finite additivity yields one side of the result. Conversely,
assume $F$ is absolutely continuous as a function and so for any
$\epsilon>0$ there exists some $\delta>0$ such that for any finite
collection of disjoint open intervals $\left\{ \left(a_{i},b_{i}\right)\right\} _{i=1}^{n}\subset\left[a,b\right]$
\[
\sum_{i=1}^{n}\left(b_{i}-a_{i}\right)<\delta\implies\sum_{i=1}^{n}f\left(b_{i}\right)-f\left(a_{i}\right)<\epsilon.
\]
Taking limits, we can extend this result to countable disjoint collections
so that for any $\left\{ \left(a_{i},b_{i}\right)\right\} _{i\in\N}$
disjoint
\[
\sum_{i=1}^{\infty}\left(b_{i}-a_{i}\right)<\delta\implies\sum_{i=1}^{\infty}f\left(b_{i}\right)-f\left(a_{i}\right)<\epsilon.
\]
 Note that by Lemma \ref{lem:openSetDisjointUnionInterval}, each
open set $O\subset\left[a,b\right]$ can be written as a countable
union of disjoint open intervals and by Proposition \ref{prop:borelApproximateLebesgue},
every Borel set $B\in\borel\left(\left[a,b\right]\right)$ (except
sets that contain the endpoints) is contained in some open set $O\subset\R$
such that 
\[
\lambda\left(O\setminus B\right)=\lambda\left(O\right)-\lambda\left(B\right)<\frac{\delta}{2}.
\]
Choosing $B$ such that $\lambda\left(B\right)<\frac{\delta}{2}$,
we can assume without loss of generality that $O\subset\left[a,b\right]$
and so 
\[
\mu_{F}\left(B\right)\leq\mu_{F}\left(O\right)<\epsilon
\]
where the inequality is due to monotonicity of measures. This completes
the proof by yet another application of Proposition \ref{prop:epsdeltaAbsContinuity}.
\end{proof}
\begin{cor}
\label{cor:stieljesAbsContRepresentation}A Stieljes function $F:\left[a,b\right]\to\R$
is absolutely continuous if and only if there exists some unique $f\in\mathcal{M}^{+}\left(\left[a,b\right],\borel\left(\left[a,b\right]\right)\right)$
such that for any $x\in\left[a,b\right]$
\[
F\left(x\right)=F\left(a\right)+\lebInt{\lambda}{f\indicate_{\left[a,x\right]}}.
\]
\end{cor}

\begin{proof}
First suppose that $F$ is absolutely continuous as a function. Then
by Proposition \ref{prop:absoluteContinuityStieljesFunctions}, the
measure $\mu_{F}$ extended from $\mu_{F}\left(\left(a,x\right]\right):=F\left(x\right)-F\left(a\right)$
is absolutely continuous with respect to the Lebesgue measure on $\left[a,b\right].$
By the Radon-Nikodym theorem, there exists some $f\in\mathcal{M}^{+}\left(\left[a,b\right],\borel\left(\left[a,b\right]\right)\right)$
such that
\[
\mu_{F}\left(A\right)=\lebInt{\lambda}{f\indicate_{A}}
\]
for any $A\in\borel\left(\left[a,b\right]\right).$ In particular
this works for $A=\left[a,x\right]$ where $x\in\left[a,b\right]$
is a arbitrary. Conversely, if $F$ has this integral representation
then $\mu_{F}$ and $\gamma\left(A\right):=\lebInt{\lambda}{f\indicate_{A}}$
agree on all sets of the form $\left[a,x\right]$ which means they
agree on intersections of such sets which is the collection of all
closed intervals in $\left[a,b\right].$ This collection is a $\pi-$system
that generates $\borel\left(\left[a,b\right]\right)$ and so by Theorem
\ref{thm:uniquenessMeasures} $\mu_{F}=\gamma$ on all Borel sets
which implies absolute continuity of the measure $\mu_{F}$ and thus
of the function $F$ by the previous Proposition.

Uniqueness of $f$ follows by the standard argument we used to prove
that Radon Nikodym derivatives are unique.
\end{proof}
\begin{rem*}
Note that the function $f$ here actually turns out to be the (almost
everywhere) derivative of $F$, a fact that requires considerable
effort to prove. We leave the proof of this result (and thus the fundamental
theorems of calculus) to the chapter on product spaces, where we would
be able to show some remarkable results related to the geometry of
the Euclidean space $\R^{n}$, and use those results to complete the
proof started here.
\end{rem*}
Next we extend this representation result to absolutely continuous
functions that are not necessarily non-decreasing. As usual, we do
this by decomposing an arbitrary absolutely continuous function as
a difference of absolutely continuous Stieljes functions and apply
the above result separately to each component.
\begin{lem}
\label{lem:decomposeAbsolutelyContinuous}Let $f:\left[a,b\right]\to\R$
be absolutely continuous. Then there exist non-negative, non-decreasing,
and absolutely continuous functions $f_{1},f_{2}:\left[a,b\right]\to\R$
such that 
\[
f\left(x\right)-f\left(a\right)=f_{1}\left(x\right)-f_{2}\left(x\right)
\]
for all $x\in\left[a,b\right]$.
\end{lem}

\begin{proof}
Define 
\begin{align*}
f_{1}\left(x\right) & :=\sup_{\pi\left[a,x\right]}\sum_{i=1}^{k\left(\pi\right)}\left(f\left(t_{i}\right)-f\left(t_{i-1}\right)\right)^{+}\\
f_{2}\left(x\right) & :=\sup_{\pi\left[a,x\right]}\sum_{i=1}^{k\left(\pi\right)}\left(f\left(t_{i}\right)-f\left(t_{i-1}\right)\right)^{-}
\end{align*}
where $\pi\left[a,x\right]:\left\{ t_{i}\mid t_{0}=a,t_{k\left(\pi\right)}=x,t_{i}<t_{i+1}\right\} $
is a \hyperref[def:partitionInterval]{partition} of $\left[a,x\right].$
Note that the non-negativity of these functions follows by definition.
Next, notice that for $x_{2}\geq x_{1}$ , $f_{j}\left(x_{2}\right)\geq f_{j}\left(x_{1}\right)$
for $j\in\left\{ 1,2\right\} .$To see this, notice that any partition
$\pi$ of $\left[a,x_{1}\right]$ can be extended to a partition $\pi^{\prime}$
of $\left[a,x_{2}\right]$ by adding $t_{k\left(\pi^{\prime}\right)}=t_{k\left(\pi\right)+1}=x_{2}.$
Then,
\[
\sum_{i=1}^{k\left(\pi\right)}\left(f\left(t_{i}\right)-f\left(t_{i-1}\right)\right)^{\pm}\leq\sum_{i=0}^{k\left(\pi\right)+1}\left(f\left(t_{i}\right)-f\left(t_{i-1}\right)\right)^{\pm}
\]
since the summands are non-negative. Next, observe that for any partition
$\pi$ of $\left[a,x\right]\subset\left[a,b\right]$, we have that
\begin{align*}
f\left(x\right)-f\left(a\right) & =\sum_{i=1}^{k\left(\pi\right)}\left(f\left(t_{i}\right)-f\left(t_{i-1}\right)\right)\\
 & =\sum_{i=1}^{k\left(\pi\right)}\left(f\left(t_{i}\right)-f\left(t_{i-1}\right)\right)^{+}-\sum_{i=1}^{k\left(\pi\right)}\left(f\left(t_{i}\right)-f\left(t_{i-1}\right)\right)^{-}
\end{align*}
and so taking supremums we get the equality $f\left(x\right)-f\left(a\right)=f_{1}\left(x\right)-f_{2}\left(x\right).$
Next, we shall show absolute continuity for $f_{1}$; the proof for
$f_{2}$ is identical. Fix $\epsilon>0$ and note that by the absolute
continuity of $f$ we have that there exists some $\delta>0$ such
that for any disjoint collection $\left\{ \left(a_{i},b_{i}\right)\right\} _{i=1}^{n}$$\subset\left[a,b\right]$,
$\sum_{i=1}^{n}b_{i}-a_{i}<\delta$ implies $\sum_{i=1}^{n}\lvert f\left(b_{i}\right)-f\left(a_{i}\right)\rvert<\epsilon$.
Note that 
\begin{align*}
\sum_{i=1}^{n}\lvert f_{1}\left(b_{i}\right)-f_{1}\left(a_{i}\right)\rvert & =\sum_{i=1}^{n}\lvert\sup_{\pi\left[a,b_{i}\right]}\sum_{j=1}^{k\left(\pi\right)}\left(f\left(t_{j}^{\pi}\right)-f\left(t_{j-1}^{\pi}\right)\right)^{+}-\sup_{\pi^{\prime}\left[a,a_{i}\right]}\sum_{j=1}^{k\left(\pi^{\prime}\right)}\left(f\left(t_{j}^{\pi^{\prime}}\right)-f\left(t_{j-1}^{\pi^{\prime}}\right)\right)^{+}\rvert\\
 & =\sum_{i=1}^{n}\sup_{\pi\left[a_{i},b_{i}\right]}\sum_{j=1}^{k\left(\pi\right)}\left(f\left(t_{j}^{\pi}\right)-f\left(t_{j-1}^{\pi}\right)\right)^{+}\\
 & \leq\sum_{i=1}^{n}\sup_{\pi\left[a_{i},b_{i}\right]}\sum_{j=1}^{k\left(\pi\right)}\left(f\left(t_{j}^{\pi}\right)-f\left(t_{j-1}^{\pi}\right)\right)^{+}+\left(f\left(t_{j}^{\pi}\right)-f\left(t_{j-1}^{\pi}\right)\right)^{-}\\
 & =\sum_{i=1}^{n}\sup_{\pi\left[a_{i},b_{i}\right]}\sum_{j=1}^{k\left(\pi\right)}\lvert f\left(t_{j}^{\pi}\right)-f\left(t_{j-1}^{\pi}\right)\rvert\\
 & \leq\epsilon
\end{align*}
where in the second equality we have used the fact that we can enlarge
any partition $\pi$on $\left[a,b_{i}\right]$ to contain any given
partition $\pi^{\prime}$ on $\left[a,a_{i}\right]$ since $\left[a,a_{i}\right]\subset\left[a,b_{i}\right].$
The final inequality then follows since for any partition $\pi_{i}$
of $\left[a_{i},b_{i}\right]$, entire collection $\left\{ \left(t_{j-1}^{\pi_{i}},t_{j}^{\pi_{i}}\right)\right\} _{1\leq j\leq k\left(\pi_{i}\right),1\leq i\leq n}$
is a disjoint collection with total length
\[
\sum_{i=1}^{n}\sum_{j=1}^{k\left(\pi_{i}\right)}t_{j}^{\pi_{i}}-t_{j-1}^{\pi_{i}}=\sum_{i=1}^{n}b_{i}-a_{i}<\delta
\]
and so by absolute continuity
\[
\sum_{i=1}^{n}\sum_{j=1}^{k\left(\pi_{i}\right)}\lvert f\left(t_{j}^{\pi}\right)-f\left(t_{j-1}^{\pi}\right)\rvert\leq\epsilon.
\]
Taking supremums over partitions preserves the inequality.
\end{proof}
\begin{thm}
\label{thm:integralRepresentationAbsoluteContinuity}Let $f:\left[a,b\right]\to\R$
be a measurable function. Then $f$ is absolutely continuous if and
only if there exists an (almost-everywhere) unique function $h\in\Lp 1{\left[a,b\right],\borel\left[a,b\right],\lambda}$
such that for each $x\in [a,b]$,
\[
f\left(x\right)=f\left(a\right)+\lambda\left(h\indicate_{\left[a,x\right]}\right).
\]
\end{thm}

\begin{proof}
First suppose that $f$ is absolutely continuous and so by Lemma \ref{lem:decomposeAbsolutelyContinuous}
we can write 
\[
f\left(x\right)-f\left(a\right)=f_{1}\left(x\right)-f_{2}\left(x\right)
\]
where $f_{i}$ are non-negative, non-decreasing, and absolutely continuous
(and thus Stieljes functions). By Corollary \ref{cor:stieljesAbsContRepresentation}
there exist functions $h_{1},h_{2}\in\mathcal{M}^{+}\left(\left[a,b\right],\borel\left(\left[a,b\right]\right)\right)$
such that
\[
f_{i}\left(x\right)=f_{i}\left(a\right)+\lambda\left(h_{i}\indicate_{\left[a,x\right]}\right)
\]
and so by the linearity of integration we have that 
\[
f\left(x\right)-f\left(a\right)=\underbrace{f_{1}\left(a\right)-f_{2}\left(a\right)}_{=0}+\lambda\left(\left(h_{1}-h_{2}\right)\indicate_{\left[a,x\right]}\right).
\]
Note that $h:=h_{1}-h_{2}$ is integrable since $f\left(b\right)-f\left(a\right)<\infty$.
Uniqueness is a generating class argument as usual. To spell it out, notice that for  $h,g\in\Lp{1}{\left[a,b\right],\borel\left(\left[a,b\right]\right),\lambda}$, where the representation result holds we have that 
\[
	\mathcal{D} : = \{ B \in \borel \left(\left[a,b\right] \right) \mid \lebInt{\lambda}{h\indicate_{B}} = \lebInt{\lambda}{g\indicate_{B}} \}
\]
is a $\lambda-$system: $\left[a,b\right]$ is clearly in $\mathcal{D}$, and if $B \in \mathcal{D}$ then $B^C \in \mathcal{D}$ because $\indicate_{B^C} = \indicate_{[a,b]} - \indicate_{B}$, the linearity of integration, and the fact that $[a,b] \in \mathcal{D}$. Finally, if $\{B_i\}_{i\in\N} \in \mathcal{D}$ are disjoint then 
\begin{align*}
	\lebInt{\lambda}{h\indicate_{\bigcup_{i\in\N}B_i}} &= \lebInt{\lambda}{\sum_{i=1}^\infty h \indicate_{B_i}} \\
	&= \sum_{i=1}^\infty \lebInt{\lambda}{h\indicate_{B_i}} \\
	&=  \sum_{i=1}^\infty \lebInt{\lambda}{g\indicate_{B_i}} \\
	&= \lebInt{\lambda}{\sum_{i=1}^{\infty}g\indicate_{B_i}} \\
	&= 	\lebInt{\lambda}{g\indicate_{\bigcup_{i\in\N}B_i}}
\end{align*}
where in the second and fourth equalities leverage dominated convergence. Note that $\mathcal{E}$ the collection of all closed interval subsets of $[a,b]$ generates $\borel\left([a,b]\right)$ and the equality of representations holds on $\mathcal{E}$ by definition. By the $\pi-\lambda$ theorem, the equality extends to the entirety of $\borel\left([a,b]\right)$. By Proposition \ref{prop:intEqualFuncEqual} we have that $h \stackrel{\text{a.e}}{=} g $.

Finally, suppose that $f$ has the integral representation
\[
f(x) - f(a) = \lebInt{\lambda}{h\indicate_{[a,x]}}
\] 
and fix $\epsilon > 0$ . Note that by the linearity of integration, for any $\left[a_i, b_i\right] \subseteq \left[a,b\right]$ 
\[
f(b_i) - f(a_i) = \lebInt{\lambda}{h\indicate_{[a_i,b_i]}}
\]
and so for any disjoint collection $\{[a_i,b_i]\}_{i=1}^n$
\begin{align*}
	\sum_{i=1}^{n} \lvert f(b_i)-f(a_i) \rvert &= \sum_{i=1}^n\lvert \lebInt{\lambda}{h\indicate_{[a_i,b_i]}}\rvert \\
	&\leq \sum_{i=1}^n\lebInt{\lambda}{\lvert h\rvert \indicate_{[a_i,b_i]}} 
\end{align*}
 by Corollary \ref{cor:triangleIneqLebIntL1}. Then, using Proposition \ref{prop:epsdeltaAbsContinuity}, we have that there exists some $\delta > 0$ such that for $\lambda\left(\bigcup_{i=1}^n[a_i,b_i]\right) < \delta $ we have $ \lebInt{\lambda}{\lvert h \rvert \indicate_{\bigcup_{i=1}^n[a_i,b_i]}} < \epsilon $ which completes the proof.
\end{proof}
%

\section{Optimization\label{sec:optimization}}
\begin{example}
\label{exa:isi2006samplepsb2}Maximize $x+y$ subject to the condition
that $2x^{2}+3y^{2}\leq1$.
\end{example}


