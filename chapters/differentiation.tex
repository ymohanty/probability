
\chapter{\label{chap:Differentiation}Differentiation}

In elementary calculus, the theory of the Riemann integral was developed
with a corresponding theory of differentiation, and these two operations
were shown to be the inverse of each other for a suitable class of
functions, namely the differentiable functions; this is the fundamental
theorem of calculus. The Lebesgue theory also implies a fundamental
theorem of calculus for a broader class of functions. The goal of
this chapter is to recover this theorem and in order to do so, we
will develop some theory that is very important in its own right.
Before delving into new material, it might be worthwhile to review
some basic facts about differentiation in the spirit of our discussion
on the Riemann integral. In addition, I will take this opportunity
to suitably generalize the notion of a derivative by defining differentiability
as a property of functions on arbitrary normed vector spaces.

\section{Differentiation in normed vector spaces}

\subsection{Review of derivatives on the line}

We first review some basic material from single variable calculus
for completeness; note that this section is technically presumed knowledge
and so results appearing previously in the text use some of the facts
established here. We will avoid circularity by keeping this section
self-contained without any reference to the material we have developed
so far in the body of the text (we will use results from the appendices).
\begin{defn}
\label{def:differentiable}A function $f:\left[a,b\right]\to\R$ is
said to be \emph{differentiable at a point }$c\in\left(a,b\right)$
if 
\[
\lim_{x\to a}\frac{f\left(x\right)-f\left(a\right)}{x-a}
\]
exists as a real number. The function is said to be \emph{differentiable
}if it is differentiable at every point in the interior of its domain.
If a function $f$ is differentiable everywhere, the derivative function
is denoted $f^{\prime}$as in 
\[
f^{\prime}\left(y\right)=\lim_{x\to y}\frac{f\left(x\right)-f\left(y\right)}{x-y}.
\]
\end{defn}

\begin{rem*}
The derivative is also denoted $\frac{df}{dy}=f^{\prime}\left(y\right)$
which captures the heuristic that the derivative can be thought as
the ratio of small quantities.
\end{rem*}
\begin{example}
\label{exa:derivativeConstant}It should be immediately clearly that
for any constant function $f\left(x\right)=a$, its derivative is
zero everywhere, and for the function $f\left(x\right)=x$, its derivative
is 1 is everywhere.
\end{example}

\begin{prop}
\label{prop:differentiableImpliesContinuous}Let $f:\left[a,b\right]\to\R$
be differentiable at a point $c\in\left(a,b\right).$Then it is continuous
at $c.$
\end{prop}

\begin{proof}
Recall that the limit of the product of two functions is the product
of the limit, provided those individual limits exist. 
\begin{align*}
\lim_{x\to c}f\left(x\right)-f\left(c\right) & =\lim_{x\to c}\frac{f\left(x\right)-f\left(c\right)}{x-c}x-c\\
 & =\lim_{x\to c}\frac{f\left(x\right)-f\left(c\right)}{x-c}\lim_{x\to c}x-c\\
 & =f^{\prime}\left(c\right)0\\
 & =0
\end{align*}
which completes the proof.
\end{proof}
\begin{rem*}
The converse of this theorem is not true; for instance, consider that
$\lvert x\rvert$ is continuous (this follows by the ``reverse''
triangle inequality $\lvert\lvert x\rvert-\lvert y\rvert\rvert\leq\lvert x-y\rvert)$.
That it is not differentiable can be seen by looking at 
\[
\lim_{x\to0^{-}}\frac{\lvert x\rvert}{x}=-1\neq1=\lim_{x\to0^{+}}\frac{\lvert x\rvert}{x}.
\]
\end{rem*}
\begin{prop}
\label{prop:algebraOfDifferentiableFunctions}Let $f,g$ be real-valued
functions on $\left[a,b\right]$ that are differentiable at some $x\in\left[a,b\right]$
. Then $f+g$ and $fg$ are differentiable at $x.$ If $g\left(x\right)\neq0$
then $\frac{f}{g}$is also differentaible. The derivatives are given
\begin{align}
\left(f+g\right)^{\prime}\left(x\right) & =f^{\prime}\left(x\right)+g^{\prime}\left(x\right)\\
\left(fg\right)^{\prime}\left(x\right) & =f^{\prime}\left(x\right)g\left(x\right)+g^{\prime}\left(x\right)f\left(x\right)\\
\left(\frac{f}{g}\right)^{\prime} & \left(x\right)=\frac{f^{\prime}\left(x\right)g\left(x\right)-g^{\prime}\left(x\right)f\left(x\right)}{\left(g\left(x\right)\right)^{2}}
\end{align}
\end{prop}

\begin{proof}
The first one, called the sum rule, follows by algebra of limits.
The second, known as the product rule, follows by the same trick that
is used to prove that the product of two sequences converges to the
product of their limits. To reiterate, note that 
\begin{align*}
\frac{f\left(y\right)g\left(y\right)-f\left(x\right)g\left(x\right)}{y-x} & =\frac{f\left(y\right)g\left(y\right)-f\left(y\right)g\left(x\right)+f\left(y\right)g\left(x\right)-f\left(x\right)g\left(x\right)}{x-c}\\
 & =\frac{f\left(y\right)\left(g\left(y\right)-g\left(x\right)\right)}{y-x}+\frac{g\left(x\right)\left(f\left(y\right)-f\left(x\right)\right)}{y-x}.
\end{align*}
Taking limits and applying Proposition \ref{prop:differentiableImpliesContinuous}
gets the result.

Finally, the third result, called the quotient rule, follows as 
\begin{align*}
\frac{\frac{f\left(y\right)}{g\left(y\right)}-\frac{f\left(x\right)}{g\left(x\right)}}{y-x} & =\frac{1}{g\left(x\right)g\left(y\right)}\left(\frac{f\left(y\right)g\left(x\right)-f\left(x\right)g\left(y\right)}{y-x}\right)\\
 & =\frac{1}{g\left(x\right)g\left(y\right)}\left(\frac{f\left(y\right)g\left(x\right)-f\left(y\right)g\left(y\right)+f\left(y\right)g\left(y\right)-f\left(x\right)g\left(y\right)}{y-x}\right)\\
 & =\frac{1}{g\left(x\right)g\left(y\right)}\left(\frac{f\left(y\right)\left[g\left(x\right)-g\left(y\right)\right]}{y-x}+\frac{g\left(y\right)\left[f\left(y\right)-f\left(x\right)\right]}{y-x}\right)
\end{align*}
and taking limits and again applying the continuity of differentiable
functions completes the proof.
\end{proof}
\begin{cor}[Power rule]
\label{cor:powerRule}The derivative of a real-valued function on
$\R$ defined $f\left(x\right)=x^{n}$ where $n\in\mathds{Z}$ is
given by $f^{\prime}\left(x\right)=nx^{n-1}.$
\end{cor}

\begin{proof}
First we examine the case when $n\in\N$. For the base cases of $n=0$
and $n=1$ look at the remark above. Now suppose that the rule holds
for $n\in\N$. Then, writing $x^{n+1}=xx^{n},$ we apply the product
rule to yield
\[
\frac{dx^{n+1}}{dx}=xnx^{n-1}+x^{n}=\left(n+1\right)x^{n}
\]
which completes the proof.

To extend the result to negative integers, we can apply the quotient
rule. Note that for a negative integer $m$, we have some positive
integern $n$ such that $m=-n$ and so $f\left(x\right)=x^{m}=\frac{1}{x^{n}}$
which has derivative (by applying the power rule for positive integers
and the quoteint rule)
\[
f^{\prime}\left(x\right)=\frac{-nx^{n-1}}{x^{2n}}=mx^{m-1}.
\]
\end{proof}
\begin{rem*}
The power rule also holds for rational exponents; we shall establish
this fact after we state the chain rule. For real exponents, the power
rule holds if $x\geq0$ since we define $x^{a}:=\exp\left(a\log\left(x\right)\right)$
when $a\in\R$. Since the logarithm is only defined on the positive
reals, the result only extends to that case (the case of $x=0$ is
trivial). We shall prove this after we construct the exponential and
natural logarithm functions in Appendix section \ref{sec:specialFunctions}.
\end{rem*}
\begin{prop}
\label{prop:differentiabilityOfPolynomials}Let $f\in P_{n}\left(x\right)$
the space of $n-$degree polynomials. Then $f$ is differentiable
and its derivative $f^{\prime}\in P_{n-1}\left(x\right).$
\end{prop}

\begin{proof}
Let $f\left(x\right)=\sum_{i=0}^{n}a_{i}x^{i}.$ By the ``sum rule'',''product
rule'' and the ``power rule'', 
\[
f^{\prime}\left(x\right)=\sum_{i=0}^{n}a_{i}ix^{i-1}\in P_{n-1}\left(x\right).
\]
\end{proof}
\begin{rem*}
An immediate corrolary is that every polynomial is infinitely differentiable.
Infinitely differentiable functions are called \emph{smooth.}
\end{rem*}
\begin{thm}[Chain rule]
\label{thm:chainRuleR}Let $g:C\subseteq\R\to\R$ be differentiable
at some $c\in D$ and let $f:D\subseteq\R\to\R$ be differentiable
at $g\left(c\right)\in D.$ Then the composition $f\circ g$ is differentiable
at $c$ with derivative 
\[
\left(f\circ g\right)^{\prime}\left(c\right)=f^{\prime}\left(g\left(c\right)\right)g^{\prime}\left(c\right).
\]
\end{thm}

We omit the proof of this theorem right now; we will prove it in the
more general setting of derivatves on functions between Banach spaces
in the next section.
\begin{prop}
\label{prop:rationalPowerRule}A function $f\left(x\right)=x^{r}$
where $r\in\mathds{Q}$ and $x\in\R$ is differentiable with derivative
$f^{\prime}\left(x\right)=rx^{r-1}.$
\end{prop}

\begin{proof}
We will establish the result when $x\neq0$ (when $x=0$, the derivative
is 0 as established in an earlier remark). Let $r=\frac{p}{q}$ where
$p,q\in\mathds{Z}$ and note that $g\left(x\right):=x^{q}$ has derivative
$g^{\prime}\left(x\right)=qx^{q-1}$ by the integral power rule\ref{cor:powerRule}.
Then, observe that $g\left(f\left(x\right)\right)=x^{p}$ and so applying
the chain rule on the left hand hand side and the integral power rule
on the right hand side, we have
\begin{align*}
qf\left(x\right)^{q-1}f^{\prime}\left(x\right) & =px^{p-1}
\end{align*}
which can be re-arranged to yield
\begin{align*}
f^{\prime}\left(x\right) & =\frac{px^{p-1}}{qf\left(x\right)^{q-1}}\\
 & =\left(\frac{p}{q}\right)x^{p-1-\frac{p}{q}\left(q-1\right)}\\
 & =\frac{p}{q}x^{\frac{p}{q}-1}\\
 & =rx^{r-1}
\end{align*}
which completes the proof.
\end{proof}

\subsubsection{Mean value theorems}
\begin{defn}
\label{def:localMax}A function $f:\left[a,b\right]\to\R$ is set
to have a local maximum at $c\in\left[a,b\right]$ if there exists
some $\delta>0$ such that for every $x\in\left(c-\delta,c+\delta\right)\cap\left[a,b\right]$
\[
f\left(c\right)\geq f\left(x\right).
\]
Local minima are defined analagously.
\end{defn}

\begin{prop}
\label{prop:firstOrderConditionR}Let $f:\left[a,b\right]\to\R$ have
a local maximum or minimum at $c\in\left(a,b\right).$ If $f^{\prime}\left(c\right)$
exists then $f^{\prime}\left(c\right)=0.$
\end{prop}

\begin{proof}
We consider the case of the local maximum; the case of the local minimum
follows by applying the maximum result to $-f.$ Suppose $c\in\left(a,b\right)$
is a local maximum for $f$, in which case we can choose some $\delta>0$
small enough that 
\[
\left(c-\delta,c+\delta\right)\subseteq\left(a,b\right)
\]
and $f\left(c\right)\geq f\left(x\right)$ on $\left(c-\delta,c+\delta\right).$
Note that for $c-\delta<x<c$ we have that 
\[
\frac{f\left(x\right)-f\left(c\right)}{x-c}\geq0
\]
and so 
\[
f^{\prime}\left(c^{-}\right):=\lim_{x\to c^{-}}\frac{f\left(x\right)-f\left(c\right)}{x-c}\geq0.
\]
Similarly, for $c<x<c+\delta$
\[
\frac{f\left(x\right)-f\left(c\right)}{x-c}\leq0
\]
and so 
\[
f^{\prime}\left(c^{+}\right):=\lim_{x\to c^{+}}\frac{f\left(x\right)-f\left(c\right)}{x-c}\leq0.
\]
But of course, $f^{\prime}\left(c\right)=f^{\prime}\left(c^{-}\right)=f^{\prime}\left(c^{+}\right)$
since left and right hand limits are always equal to the limit when
it exists and thus 
\[
f^{\prime}\left(c\right)=0.
\]
\end{proof}
\begin{thm}[Cauchy's mean value theorem]
\label{thm:cauchyMeanValue}If $f,g$ are real-valued continuous
functions on $\left[a,b\right]$ such that the derivative functions
$f^{\prime},g^{\prime}$ exist on $\left(a,b\right)$ , there exists
some $x\in\left(a,b\right)$ such that 
\[
\left(f\left(b\right)-f\left(a\right)\right)g^{\prime}\left(x\right)=\left(g\left(b\right)-g\left(a\right)\right)f^{\prime}\left(x\right)
\]
\end{thm}

\begin{proof}
Write 
\[
h\left(t\right):=\left(f\left(b\right)-f\left(a\right)\right)g\left(t\right)-\left(g\left(b\right)-g\left(a\right)\right)f\left(t\right)
\]
and observe that since it's built out of differeniable functions by
adding and multiplying them, $h$ is continuous and differentiable.
Since $\left[a,b\right]$ is compact, $h$ has a minimum and maximum
by Weierstrass'. If the minimum and maximum are both at the end points
$a$ and $b$, then the function is constant and the derivative is
zero everywhere, which yields the result. If at least one of the minimum
or maximum is interior, say at $x\in\left(a,b\right)$, then by Proposition
\ref{prop:firstOrderConditionR}, $f^{\prime}\left(x\right)=0$ which
again yields the result.
\end{proof}
\begin{cor}[Mean~ value theorem]
\label{cor:meanValueThm}Let $f:\left[a,b\right]\to\R$ be continuous
on $\left[a,b\right]$ and differentiable on $\left(a,b\right).$
Then there exists some $x\in\left(a,b\right)$ such that
\[
f^{\prime}\left(c\right)=\frac{f\left(b\right)-f\left(a\right)}{b-a}.
\]
\end{cor}

\begin{proof}
Apply Cauchy's mean value theorem with $g\left(x\right)=x$.
\end{proof}
\begin{prop}
\label{prop:derivativeAndMonotonicity}If $f:\left[a,b\right]\to\R$
is differentiable on $\left(a,b\right)$ then
\begin{align*}
\forall x\in\left(a,b\right):f^{\prime}\left(x\right) & \geq0\implies\left(x_{1}\geq x_{2}\implies f\left(x_{1}\right)\geq f\left(x_{2}\right)\right)\\
\forall x\in\left(a,b\right):f^{\prime}\left(x\right) & \leq0\implies\left(x_{1}\geq x_{2}\geq f\left(x_{1}\right)\leq f\left(x_{2}\right)\right)\\
\end{align*}
\end{prop}

\begin{proof}
Applying the mean value theorem, we have 
\[
f\left(x_{1}\right)-f\left(x_{2}\right)=f^{\prime}\left(c\right)\left(x_{1}-x_{2}\right)
\]
for $c\in\left(a,b\right)$ and if $f^{\prime}\left(c\right)\geq0$
then $x_{1}\geq x_{2}$ implies that $f\left(x_{!}\right)\geq f\left(x_{2}\right)$
and the other implication follows in the same way.
\end{proof}
\begin{rem*}
An obvious corollary of the above result is that if $\forall x\in\left(a,b\right)$$:f^{\prime}\left(x\right)=0\implies f$
is constant on $\left(a,b\right)$.
\end{rem*}

\subsubsection{Continuity of derivatives}

For a function $f$ on $\left[a,b\right]$ that is differentiable
everywhere on $\left(a,b\right)$, it need not be the case that the
derivative function $f^{\prime}$ is continuous everywhere. In fact,
the points of discontinuity of a derivative can be dense in the domain
and have positive Lebesgue measure. We postpone the discussion of
these more complicated examples until we prove the fundamental theorem
of calculus. The simplest canonical example of a discontinuous derivative
is the following
\begin{example}
\label{exa:discontinuousDerivative}Let $f\left(x\right)=x^{2}\sin\left(\frac{1}{x}\right)\indicate\left\{ x\neq0\right\} .$We
take on faith the fact that the function $\sin\left(x\right)$ is
bounded between $-1$ and 1, is infinitely differentiable and that
its derivative is $\cos\left(x\right)$. We construct these functions
from first principles in Appendix section \ref{sec:specialFunctions}.
Note that the derivative of this function away from zero is given
by the chain and product rules as 
\[
f^{\prime}\left(x\right)=2x\sin\left(\frac{1}{x}\right)-\cos\left(\frac{1}{x}\right).
\]
The derivative at zero can be computed from first principles by looking
at the limit
\[
\lim_{x\to0}\frac{x^{2}\sin\left(\frac{1}{x}\right)\indicate\left\{ x\neq0\right\} }{x}=\lim_{x\to0}x\sin\left(\frac{1}{x}\right)\indicate\left\{ x\neq0\right\} =0
\]
since $\lvert\sin\left(\frac{1}{x}\right)\rvert\leq1$. Then we have
that 
\[
f^{\prime}\left(x\right)=\begin{cases}
2x\sin\left(\frac{1}{x}\right)-\cos\left(\frac{1}{x}\right), & x\neq0\\
0, & x=0
\end{cases}
\]
but $\lim_{x\to0}f^{\prime}\left(x\right)$ does not exist since $\cos\left(\frac{1}{x}\right)$
oscillates rapidly near zero.
\end{example}

While derivatives can be discontinuous, they share the intermediate
value property with continuous functions.
\begin{prop}[Darboux's theorem]
\label{prop:darbouxTheorem}Let $f:\left[a,b\right]\to\R$ be differentiable
on $\left(a,b\right).$ Then the derivative function $f^{\prime}:\left(a,b\right)\to\R$
has the intermediate value property that the image $f\left[\left(a,b\right)\right]$
is an interval.
\end{prop}

\begin{proof}
Let $c,d\in\left(a,b\right)$ be such that, without loss generality,
$f^{\prime}\left(c\right)>f^{\prime}\left(d\right).$ Take any $y\in\left(f^{\prime}\left(c\right),f^{\prime}\left(d\right)\right)$.
The function $g\left(x\right)=f\left(x\right)-yx$ is continuous on
$\left[\min\left\{ c,d\right\} ,\max\left\{ c,d\right\} \right]$
and thus achieves a maximum and a minimum by Weierstrass' theorem.
If both the minimum and maximum occur at the end points then the function
$g$ is constant on $\left[\min\left\{ c,d\right\} ,\max\left\{ c,d\right\} \right]$
and so its derivative is zero everywhere, completing the proof. If
there's at least one interior extrema, say at $t\in\left(\min\left\{ c,d\right\} ,\max\left\{ c,d\right\} \right),$
then
\[
f^{\prime}\left(t\right)=y
\]
by Proposition \ref{prop:firstOrderConditionR}.
\end{proof}
\begin{lem}
\label{lem:IVTJump}Any function that satisifies the intermediate
value property cannot have removable or jump discontinuities.
\end{lem}

\begin{proof}
Let $f:\left[a,b\right]\to\R$ be a function that satisfies the intermediate
value property and let $c\in\left(a,b\right)$ such that $\lim_{x\to c^{-}}f\left(x\right)=L$
and $\lim_{x\to c^{+}}f\left(x\right)=M$. Suppose, without loss of
generality, that $L\neq f\left(c\right)$ and let $\epsilon=\frac{\lvert L-f\left(c\right)\rvert}{2}.$
Note that by definition, there exists some $\delta>0$ such that for
any $x$ with $0<c-x<\delta\implies\lvert f\left(x\right)-L\rvert<\epsilon$
which implies that $\lvert f\left(x\right)-f\left(c\right)\rvert\geq\epsilon.$
By the intermediate value property, for any $y$ in between $f\left(x\right)$
and $f\left(c\right)$, there exists some $z\in\left(x,c\right)$
such that $f\left(z\right)=y.$ In particular, this holds true for
any $\text{y}.$such that $\lvert y-f\left(c\right)\rvert<\epsilon.$
But then $0<c-z<\delta$ which implies that $\lvert y-f\left(c\right)\rvert\geq\epsilon$
which is a contradiciton.
\end{proof}

\subsubsection{Differentiability classes and local approximation by polynomials}

\subsection{Derivatives of functions between Banach spaces}

\section{Decomposition of measures}

Recall that in Proposition \ref{prop:densities}, we showed that for
a measure space $\left(\X,\F,\mu\right)$ and a non-negative measurable
function $f\in\nonnegMeasurableFunctions$ , the function $\nu\left(A\right):=\lebInt{\mu}{f\indicate_{A}}$is
a measure. Moreover, in the remark following the proposition, we observed
that such a measure $\nu$ was \emph{absolutely continuous }with respect
to $\mu$, that is to say, for any $A\in\F,\mu(A)=0\implies\nu(A)=0.$
Further, we also claimed that every absolutely continuous measure
could be represented this way i.e as an integral with respect to the
dominating measure. In other words, we claimed that there was always
an $f\in\nonnegMeasurableFunctions$ such that this relation would
hold. It turns out we can establish a slightly stronger result, which
in turn is key to establishing the fundamental theorem of calculus
for the Lebesgue measure. What we can show, using the Hilbert space
machinery we developed earlier, is that we can decompose a measure
(relative to another measure) into two measures that are in some sense
orthogonal. This is in direct correspondence with the decomposition
of Hilbert spaces given to us by the projection theorem (Theorem \ref{thm:projectionThm}).

To motivate how the Hilbert space theory can be useful in this context,
we can begin with a change of perspective in the spirit of Theorem
\ref{thm:integralMeasureEquivalence}. Note that saying $\nu(A)=\lebInt{\mu}{f\indicate_{A}}$
for every $A\in\F$ is equivalent to saying $\lebInt{\nu}g=\lebInt{\mu}{fg}\forall g\in\Lp 1{\X,\F,\nu}.$
The right hand side is the inner product associated with the space
$\Lp 2{\mu}$ and so we wish to show that a linear functional $g\to\lebInt{\nu}g$
can be represented as an inner product on $\Lp 2{\mu}.$ At this juncture,
one can guess that the Riesz representation theorem for Hilbert spaces
is clearly right tool to finish off the proof; however, several hurdles
remain. For one, $g\to\lebInt{\nu}g$ a linear functional on $\Lp 1{\nu}$
and it is not $\Lp 2{\mu}.$ Further, it is unclear whether our functional
is bounded, which is a necessary condition for the theorem to apply.
The following lemma provides the conditions under which the theorem
finds purchase.
\begin{lem}
\label{lemma:LebesgueRadonNikodym}Let $\left(\X,\F\right)$ be a
measurable space. If $\mu,\nu$ are two $\sigma-$finite measures
on $\F$, then there exists some non-negative (almost everywhere with
respect to $\mu$) measurable function $k\in\nonnegMeasurableFunctions$
and a set $G\in\F$ such that $\mu(G)=0$ and
\[
\nu\left(A\right)=\lebInt{\mu}{k\indicate_{A}}+\nu(A\cap G)
\]
for every $A\in\F$. Moreover, the function $k$ is unique $\mu-$almost-everywhere
and the set $G$ is unique $\mu+\nu$-almost everywhere.
\end{lem}

\begin{proof}
Suppose first that the measures $\mu$and $\nu$are finite. Define
the measure $\psi:=\mu+\nu$ and observe that by Proposition \ref{prop:integralSumOfMeasures}
\[
\lebInt{\psi}f=\lebInt{\mu}f+\lebInt{\nu}f.
\]
Next define an operator 
\[
\Gamma:\Lp 2{\psi}\longrightarrow\R
\]
by
\[
\Gamma\left(f\right):=\lebInt{\nu}f
\]
and notice that $\Gamma$ is a continuous linear functional by a slight
variant of the argument presented in the proof of Corollary \ref{cor:integrationContinuousLinearFunctional},
using the fact that $\mu\left(\X\right),\nu\left(\X\right)<\infty$.
Since $\mathcal{L}^{2}\left(\psi\right)$ is a Hilbert space by Corollary
\ref{cor:L2Hilbert} and Theorem \ref{thm:completenessLp}, the \hyperref[thm:rieszRep]{Riesz representation theorem} tells
us there exists an (almost $\psi-$everywhere unique) function $h\in\Lp 2{\psi}$
such that
\begin{align*}
\Gamma\left(f\right) & =\innerproduct fh
\end{align*}
for every $f\in\Lp 2{\psi}.$ Of course, in the context of $\mathcal{L}^{2}$,
\begin{equation}
\innerproduct fh=\pnorm{fh}1=\lebInt{\psi}{fh}=\lebInt{\mu}{fh}+\lebInt{\nu}{fh}.\label{eq:rieszRepresenterL2}
\end{equation}
Next consider the following measurable partition of $\X$
\[
N:=\left\{ x\in\X\mid h\left(x\right)<0\right\} ,M:=\left\{ x\in\X\mid0\leq h\left(x\right)<1\right\} ,G:=\left\{ x\in\X\mid h\left(x\right)\geq1\right\} .
\]
Note that by the fact that $0\geq h\indicate_{N}$ and the monotonicity
of integration
\begin{align*}
0 & \geq\lebInt{\psi}{h\indicate_{N}}\\
 & =\lebInt{\mu}{h\indicate_{N}}+\lebInt{\nu}{h\indicate_{N}}\\
 & =\lebInt{\nu}{\indicate_{N}}
\end{align*}
where the last equality is due to (\ref{eq:rieszRepresenterL2}).
But since $\lebInt{\nu}{\indicate_{N}}=\nu\left(N\right),$we have
$\nu\left(N\right)=0$ by non-negativity of measures. Then
\[
\lebInt{\mu}{h\indicate_{N}}+\lebInt{\nu}{h\indicate_{N}}=0
\]
where the second term is automatically zero since $h\indicate_{N}\stackrel{\nu-\text{a.e}}{=}0$.
Therefore, 
\[
\lebInt{\mu}{h\indicate_{N}}=0
\]
which by Proposition \ref{prop:intZeroFuncZero} implies that
\[
h\indicate_{N}\stackrel{\mu-\text{a.e}}{=}0\implies\mu\left(N\right)=0
\]
and so both $\mu\left(N\right)=\nu\left(N\right)=0.$

On the other hand, observe that 
\begin{align*}
\Gamma\left(\indicate_{G}\right) & =\nu\left(G\right)\\
 & =\lebInt{\mu}{h\indicate_{G}}+\lebInt{\nu}{h\indicate_{G}}\\
 & \geq\mu\left(G\right)+\nu\left(G\right)
\end{align*}
where the inequality follows from the fact that $h\indicate_{G}\geq\indicate_{G}$
and the monotonicity of the integral. Since $\nu\left(G\right)<\infty$
as $\nu$ is a finite measure, we can subtract if from both sides
of the inequality to deduce
\[
\mu\left(G\right)\leq0
\]
 which reduces to equality by the non-negativity of measures.

To control the final piece of the partition, observe that we can use
the definition $\Gamma\left(f\right)=\lebInt{\nu}f$ and Eq. (\ref{eq:rieszRepresenterL2})
together to deduce (since all terms are finite) that
\[
\lebInt{\nu}f-\lebInt{\nu}{fh}=\lebInt{\mu}{fh}
\]
for $f\in\Lp 2{\psi}.$ By linearity this reduces to~
\begin{equation}
\lebInt{\nu}{\left(1-h\right)f}=\lebInt{\mu}{hf}.\label{eq:oneminusH}
\end{equation}
Next we define the increasing sequence of sets
\[
M_{n}:=\left\{ x\in\X\mid0\leq h\left(x\right)\leq1-\frac{1}{n}\right\} 
\]
and note that
\[
M=\bigcup_{n\in\N}M_{n}.
\]
Further, observe that for any $A\in\F$ the functions
\[
f_{n}:=\frac{\indicate_{M_{n}}\indicate_{A}}{1-h}\leq n\indicate_{M_{n}}\indicate_{A}
\]
and
\[
0\leq f_{n}\leq f_{n+1}
\]
pointwise for all $n\in\N$. This shows us that our functions $f_{n}\in\Lp 2{\psi}$\footnote{$n\indicate_{M_{n}}$is square integrable since $\psi$ is a finite
measure; $f_{n}$ then is integrable since both functions are non-negative
and so 
\[
\lebInt{\psi}{\lvert f_{n}\rvert^{2}}\leq\lebInt{\psi}{\lvert n\indicate_{M_{n}}\indicate_{A}\rvert^{2}}\leq n^{2}\psi\left(\left\{ 0\leq h\leq1-\frac{1}{n}\right\} \right)
\]
} and are monotonically increasing. Therefore,
\begin{align*}
\nu\left(M\cap A\right) & =\lebInt{\nu}{\indicate_{M}\indicate_{A}\frac{1-h}{1-h}}\\
 & =\lim_{n\to\infty}\lebInt{\nu}{\frac{\indicate_{M_{n}}\indicate_{A}}{1-h}1-h}\\
 & =\lim_{n\to\infty}\lebInt{\mu}{\frac{\indicate_{M_{n}}\indicate_{A}}{1-h}h}\\
 & =\lebInt{\mu}{\indicate_{M}\indicate_{A}\frac{h}{1-h}}+\lebInt{\mu}{\indicate_{N\cup G}\indicate_{A}\frac{h}{1-h}}\\
 & =\lebInt{\mu}{\frac{h}{1-h}\indicate_{A}}
\end{align*}
where the second equality follows by the \hyperref[thm:generalizedMonotoneConvergence]{monotone convergence theorem},
the third equality is due to (\ref{eq:oneminusH}), the fourth equality
is due to the fact that $\mu\left(G\right)=\mu\left(N\right)=0$ and
the last equality is due to the linearity of integration. 

Letting $k=\frac{h}{1-h}$ we have that 
\[
\mu\left(\left\{ x\in\X\mid k\left(x\right)<0\right\} \right)=0
\]
since $\mu\left(N\right)=\mu\left(G\right)=0.$ Therefore
\begin{align*}
\nu\left(A\right) & =\nu\left(\left(A\cap N\right)\cup\left(A\cap M\right)\cup\left(A\cap G\right)\right)\\
 & =\nu\left(A\cap N\right)+\nu\left(A\cap M\right)+\nu\left(A\cap G\right)\\
 & =\lebInt{\mu}{k\indicate_{A}}+\nu(A\cap G)
\end{align*}
where the first equality follows from the fact that $N,M,$ and $G$
form a partition of $\X$, the second equality by finite additivity,
and the final equality by the fact that $N$ is a $\nu-$null set.

The extension to $\sigma-$finite measures is relatively straightforward.
Let $\left\{ E_{i}\right\} _{i\in\N}\in\F$ be a partition of $\X$
such that $\mu$and $\nu$are finite on each $E_{i}$ (this is possible
by Propositions \ref{prop:equivSigmaFinite} and \ref{prop:sumSigmaFiniteMeasures}).
Then the measures given by
\begin{align*}
\nu_{i}\left(A\right) & :=\nu\left(A\cap E_{i}\right),\\
\mu_{i}\left(A\right) & :=\mu\left(A\cap E_{i}\right)
\end{align*}
are finite on $\F$ and so the finite version of theorem implies the
existence of functions $k_{i}\in\nonnegMeasurableFunctions$ and $\mu_{i}-$null
sets $B_{i}$ such that 
\[
\nu_{i}\left(A\right)=\lebInt{\mu_{i}}{k_{i}\indicate_{A}}+\nu_{i}\left(A\cap B_{i}\right)
\]
which is equivalent to saying that 
\[
\nu\left(A\cap E_{i}\right)=\lebInt{\mu}{k_{i}\indicate_{A}\indicate_{E_{i}}}+\nu\left(A\cap B_{i}\cap E_{i}\right)
\]
for each $i\in\N$. Summing over $i$ we have 
\begin{align*}
\nu\left(A\right) & =\sum_{i=1}^{\infty}\nu\left(A\cap E_{i}\right)\\
 & =\sum_{i=1}^{\infty}\lebInt{\mu}{k_{i}\indicate_{E_{i}}\indicate_{A}}+\sum_{i=1}^{\infty}\nu\left(A\cap B_{i}\cap E_{i}\right)\\
 & =\lebInt{\mu}{\sum_{i=1}^{\infty}k_{i}\indicate_{E_{i}}\indicate_{A}}+\nu\left(A\cap\left(\cup_{i\in\N}\left(B_{i}\cap E_{i}\right)\right)\right).
\end{align*}
where the last equality follows by monotone convergence and countable
additivity. Note that $\mu\left(\bigcup_{i\in\N}\left(B_{i}\cap E_{i}\right)\right)=\sum_{i=1}^{\infty}\mu\left(B_{i}\cap E_{i}\right)=\sum_{i=1}^{\infty}\mu_{i}\left(B_{i}\right)=0.$
This completes the proof of existence with $k=\sum_{i=1}^{\infty}k_{i}\indicate_{E_{i}}$
and $B=\bigcup_{i\in\N}\left(B_{i}\cap E_{i}\right)$.

Finally, for uniqueness, consider two pairs $\left(k_{1},G_{1}\right)$
and $\left(k_{2},G_{2}\right)$ such that $k_{1},k_{2}\in\nonnegMeasurableFunctions,$
and $G_{1},G_{2}\in N_{\mu},$ and
\[
\nu\left(A\right)=\lebInt{\mu}{k_{1}\indicate_{A}}+\nu\left(A\cap G_{1}\right)=\lebInt{\mu}{k_{2}\indicate_{A}}+\nu\left(A\cap G_{2}\right)
\]
for every $A\in\F.$ Then,
\begin{align*}
\nu\left(G_{2}\cap G_{1}^{C}\right) & =\lebInt{\mu}{k_{1}\indicate_{G_{2}}\indicate_{G_{1}}}+\nu\left(\underbrace{G_{2}\cap G_{1}^{C}\cap G_{1}}_{=\emptyset}\right)\\
 & =0
\end{align*}
where the first term is zero because $G_{1}\in N_{\mu}.$ Similarly,
we can show that 
\begin{align*}
\nu\left(G_{1}\cap G_{2}^{C}\right) & =\lebInt{\mu}{k_{2}\indicate_{G_{1}}\indicate_{G_{2}^{C}}}+\nu\left(\underbrace{G_{1}\cap G_{2}^{C}\cap G_{2}}_{=\emptyset}\right)\\
 & =0.
\end{align*}
In other words, $\nu\left(G_{1}\Delta G_{2}\right)=0$ which by Proposition
\ref{prop:almostEverywhereEqualSets} implies that $G_{1}\stackrel{\nu-\text{a.e}}{=}G_{2}.$
Of course, since $G_{1},G_{2}\in N_{\mu},$ we have that $\mu\left(A\Delta B\right)=0$
and so $G_{1}\stackrel{\mu+\nu-\text{a.e}}{=}G_{2}.$ Note that since
$G_{1}^{C}\Delta G_{2}^{C}=G_{1}\Delta G_{2}$, we have that $G_{1}^{C}\stackrel{\mu+\nu-\text{a.e}}{=}G_{2}^{C}$
and so for any $A\in\F$
\[
\nu\left(A\cap G_{1}^{C}\right)=\nu\left(A\cap G_{2}^{C}\right).
\]
But
\begin{align*}
\nu\left(A\cap G_{1}^{C}\right) & =\lebInt{\mu}{k_{1}\indicate_{A}\indicate_{G_{!}^{C}}}+\nu\left(\underbrace{A\cap G_{1}^{C}\cap G_{1}}_{=\emptyset}\right)\\
\nu\left(A\cap G_{2}^{C}\right) & =\lebInt{\mu}{k_{2}\indicate_{A}\indicate_{G_{2}^{C}}}+\nu\left(\underbrace{A\cap G_{2}^{C}\cap G_{2}}_{=\emptyset}\right)
\end{align*}
This combined with the fact that $G_{1}$ and $G_{2}$ are $\mu-$null
show that for every $A\in\F$ we have $\lebInt{\mu}{k_{1}\indicate_{A}}=\lebInt{\mu}{k_{2}\indicate_{A}}$
which by Proposition \ref{prop:intEqualFuncEqual} implies that $k_{1}\stackrel{\mu-\text{a.e}}{=}k_{2}.$
\end{proof}
This is a powerful lemma as it gives us two very important results
as simple corollaries.
\begin{defn}
\label{def:absoluteContinuityMeasures}Let $(\X,\F)$ be a measurable
space and let $\mu,\nu$ be two measures on this space. We say $\nu$
is absolutely continuous with respect to $\mu$ if for any $A\in F:\mu(F)=0\implies\nu(F)=0$.
We denote this relation by $\nu<<\mu$.

Why is this relation labeled continuity? There are two good reasons
for this: first, at least for finite measures, the above definition
is equivalent to an $\epsilon-\delta$ definition that resembles the
continuity definitions that we have seen before. More importantly,
we will see that measures that are absolutely continuous with respect
to the Lebesgue measure give rise to a particular class of real-valued
\emph{functions }with the property of absolute continuity. Such functions
are exactly the class of functions for which the Lebesgue fundamental
theorem of calculus applies.
\end{defn}

\begin{prop}
\label{prop:epsdeltaAbsContinuity}Let $(\X,\F)$ be a measurable
space and let $\mu,\nu$ be two measures on this space such that $\nu(X)<\infty$.
Then $\nu<<\mu$ if and only if for every $\epsilon>0$ there exists
a $\delta>0$ such that for any $A\in\F$, $\mu(A)<\delta\implies\nu(A)<\epsilon$
\end{prop}

\begin{proof}
Suppose that $\epsilon-\delta$ characterization holds. Take a sequence
$\epsilon_{n}>0$ such that $\epsilon_{n}\to0$ and find the corresponding
$\delta_{n}>0$ such that $\mu(A)<\delta_{n}\implies\nu(A)<\epsilon_{n}$.
If $\mu(A)=0$ then $\mu(A)<\delta_{n}$ for all $n\in\N$. Then $\nu(A)<\epsilon_{n}$
for every $n.$ Taking limits yields $\nu(A)=0.$

Conversely, assume that $\epsilon-\delta$ property does not hold.
Then, there exists some $\epsilon_{0}>0$ such that for any $\delta>0$,
there's some $A\in\F$ such that $\mu(A)<\delta$ but $\nu(A)\geq\epsilon$.
Let $\delta_{n}=\frac{1}{2^{n}}\to0$ and let $A_{n}\in\F$ be the
corresponding sequence of sets such that $\mu(A_{n})<\delta_{n}$
but $\nu(A_{n})\geq\epsilon.$ Note then $\sum_{n=1}^{\infty}\mu(A_{n})<1$
and so by the \hyperref[thm:borelCantelli]{Borel-Cantelli lemma},
\[
\mu(\limsup_{n\to\infty}A_{n})=0.
\]
where $\limsup_{n\to\infty}A_{n}=\cap_{n=1}^{\infty}\cup_{i=1}^{n}A_{i}\in\F$.
Since $\nu(X)<\infty,$ by the \hyperref[cor:reverseFatouLemma]{reverse Fatou lemma}
\[
\nu(\limsup_{n\to\infty}A)\geq\limsup_{n\to\infty}\nu(A_{n})\geq\epsilon
\]
which completes the proof.
\end{proof}
While this is one way to characterize absolutely continuous measures,
a different characterization provides the shortest route to the fundamental
theorem of calculus.
\begin{thm}[Radon-Nikodym]
\label{thm:radonNikodym}Let $\left(\X,\F\right)$ be a measurable
space. If $\mu,\nu$ are two $\sigma$-finite measures on $\F$ then
$\nu<<\mu$ if and only if there exists almost-everywhere unique (with
respect to both measures) non-negative function $f\in\nonnegMeasurableFunctions$
such that 
\[
\nu(A)=\lebInt{\mu}{f\indicate_{A}}
\]
for every $A\in\F$
\end{thm}

\begin{proof}
Note that the existence of $f$ implies absolute continuity by Proposition
\ref{prop:densities}. The converse is far more challenging; fortunately
for us, Lemma \ref{lemma:LebesgueRadonNikodym} does all of the work.
To see this, note that by our lemma

\[
\nu(A)=\lebInt{\mu}{\tilde{f}\indicate_{A}}+\nu(A\cap B)
\]
where $A\in\F$, $B\in N_{\mu}$ and $\tilde{f}$ is $\mu-$almost
everywhere non-negative and unique. If $\nu<<\mu$ then $\nu(A\cap B)=0$.
Letting $f=\tilde{f}\indicate_{{\tilde{f}\geq0}}$, our result follows
by Proposition \ref{prop:funcEqualityAlmostEverywhere}.
\end{proof}
The Radon-Nikodym theorem builds on the Hilbert space theory we described
earlier in Chapter \ref{chap:spaces_of_functions} and is the proper
converse of the comparatively simple result on constructing new measures
using non-negative measurable functions that we saw in Proposition
\ref{prop:densities}. This theorem is also fundamental to developing
conditional expectations; indeed the existence and uniqueness of conditional
expectations is an almost trivial corollary of the Radon-Nikodym theorem.
In the context of probability theory, the functions described by the
Radon-Nikodym theorem are probability density functions. Recall from
undergraduate probability that probability density functions could
be recovered as derivatives of cumulative distribution functions,
provided those distribution functions were sufficiently well behaved.
It turns out such densities can be thought of as derivatives generally
in an admittedly contrived sense: if we write the Radon Nikodym theorem
in traditional notation, then we have can write for $\nu<<\mu$
\[
\nu\left(A\right)=\int_{A}fd\mu.
\]
Then, in Leibniz notation, we could write $f$ as $\frac{d\nu}{d\mu}$
and refer to $f$ as the \emph{Radon-Nikodym derivative }of $\nu$
with respect to $\mu$. Of course, for function to actually be some
sort of derivative of measures, we need to be able to represent it
as a limiting ratio of the two measures; in the special case of the
Lebesgue measure, we can indeed do this. However, the notation is
useful more generally, partly due to the following facts
\begin{prop}
\label{prop:RadonNikodymFacts}Let $\measurablespace$ be a measurable
space and let $\mu,\nu$, and $\gamma$be $\sigma-$finite measures
on the space. Then

\begin{enumerate}[label=(\roman*),leftmargin=.1\linewidth,rightmargin=.4\linewidth]
	\item If $\mu<<\nu$ and $\nu<<\gamma$  then $\mu << \gamma$  and
\begin{equation}
\tag{Chain rule}
	\frac{d\mu}{d\gamma}\stackrel{\text{a.e}}{=}\frac{d\mu}{d\nu}\frac{d\nu}{d\gamma}
\label{eq:chainRule}
\end{equation}
	\item  If $\mu << \gamma $ and $\nu << \gamma $ then the sum measure $\mu + \nu$ is $\sigma-$finite, $\mu + \nu << \gamma$, and 
\begin{equation}
\tag{Sum Rule}
	\frac{d\mu+\nu}{d\gamma}\stackrel{\gamma-\text{a.e}}{=}\frac{d\mu}{d\gamma}+\frac{d\nu}{d\gamma}
\label{eq:sumRule}
\end{equation}
	\item If $\mu << \nu$ then $\nu << \mu$ if and only if $\frac{d\mu}{d\nu}>0$ $\nu-$a.e and then
\begin{equation}
\tag{Inverse ``function" rule}
	\frac{d\nu}{d\mu}=\frac{1}{\frac{d\mu}{d\nu}}.
\label{eq:invFunctionRule}
\end{equation}
\end{enumerate}
\end{prop}

\begin{proof}
To show \emph{(i),} let $A\in\F$ be such that $\gamma\left(A\right)=0.$Then
$\nu\left(A\right)=0$ since $\nu<<\gamma.$ Then $\mu\left(A\right)=0$
since $\mu<<\nu$ which establishes that absolute continuity is transitive.
Note that since the measures are $\sigma-$finite, by the Radon Nikodym
theorem there exists some almost everywhere unique non-negative measurable
functions $f,g,h$ such that for any $A\in\F$
\begin{align*}
\mu\left(A\right) & =\lebInt{\gamma}{h\indicate_{A}}\\
\mu\left(A\right) & =\lebInt{\nu}{f\indicate_{A}}\\
\nu\left(A\right) & =\lebInt{\gamma}{g\indicate_{A}}
\end{align*}
and so by Corollary \ref{cor:densityIntegral}, we have that - for
any $A\in\F$
\[
\mu\left(A\right)=\lebInt{\nu}{f\indicate_{A}}=\lebInt{\gamma}{fg\indicate_{A}}.
\]
Since the function $fg$ is non-negative measurable, we have by the
uniqueness of Radon-Nikodym derivatives that 
\[
h\stackrel{\text{a.e}}{=}fg
\]
which completes the proof.

For \emph{(ii), }note that $\mu+\nu$ is $\sigma-$finite by Proposition
\ref{prop:sumSigmaFiniteMeasures}. Absolute continuity follows trivially.
To show the almost sure equivalence of the Radon-Nikodym derivatives,
let $f,g,h\in\nonnegMeasurableFunctions$ be Radon-Nikodym derivatives
such that for every $A\in\F$
\begin{align*}
\mu\left(A\right)+\nu\left(A\right) & =\lebInt{\gamma}{h\indicate_{A}}\\
\mu\left(A\right) & =\lebInt{\gamma}{f\indicate_{A}}\\
\nu\left(A\right) & =\lebInt{\gamma}{g\indicate_{A}}.
\end{align*}
By linearity of the Lebesgue integral, we have that for every $A\in\F$
\[
\lebInt{\gamma}{\left(f+g\right)\indicate_{A}}=\lebInt{\gamma}{h\indicate_{A}}.
\]
By Proposition \ref{prop:intEqualFuncEqual} we have that 
\[
h\stackrel{\gamma-\text{a.e}}{=}f+g.
\]
Finally, to show \emph{(iii), }let $f$ denote $\frac{d\mu}{d\nu}$
and suppose that that $f\stackrel{\nu-\text{a.e}}{>}0$ and that $\mu\left(A\right)=0$
for some $A\in\F.$ The Radon-Nikodym theorem implies then that $\lebInt{\nu}{f\text{}\indicate_{A}}=0$.
Since $f$ is positive almost everywhere, Proposition \ref{prop:intZeroFuncZero}
implies that $\indicate_{A}\stackrel{\nu-\text{a.e}}{=}0\Longleftrightarrow\nu\left(A\right)=0.$
Conversely, suppose $\nu<<\mu$ and there's a set of positive (with
respect to both measures) mass $B\in\F$ such that $f=0$ on $B$.
Then $\mu\left(B\right)=\lebInt{\nu}{f\indicate_{B}}=0$ which is
a contradiction. To conclude, observe that when $\frac{d\mu}{d\nu}\stackrel{\text{a.e}}{>}0$,
both measures are \emph{mutually }absolutely continuous (or equivalent)
and so $\frac{d\nu}{d\mu}$ exists and by part \emph{(i) }above
\[
1=\frac{d\mu}{d\mu}=\frac{d\mu}{d\nu}\frac{d\nu}{d\mu}.
\]
Taking reciprocals yields the result.
\end{proof}
Absolute continuity is a special property between pairs of measures
characterized by the notion that the null sets of a measure are a
subset of the other; that is to say $\mu<<\nu\Longleftrightarrow N_{\nu}\subseteq N_{\mu}$.
Of course, if the two measures $\mu,\nu$ are mututally absolutely
continuous (or equivalent), then $N_{\mu}=N_{\nu}.$ Dual to this
notion of absolute continuity, we can define the concept of mutual
singularity, which corresponds to the situation where $N_{\nu}$ and
$N_{\mu}$ can together cover $\X$
\begin{defn}
\label{def:concentrationOfMeasure}Let $\measurespace$ be a measure
space. The measure $\mu$ is said to \emph{concentrate }on a set $A\in\F$
if $\mu\left(E\right)=\mu\left(A\cap E\right).$ Equivalenty, one
can say that $\mu$ concentrates on $A$ if $\mu\left(E\right)=0$
if and only if $E\cap A=\emptyset.$
\end{defn}

\begin{defn}
\label{def:mutualSingularity}Let $\measurablespace$ be a measurable
space. Two measures $\mu,\nu$ on $\F$ are said to be \emph{mutually
singular }if there exists two disjoint sets $A,B\in\F$ such that
$\mu$ concentrates on $A$ and $\nu$ concentrates on $B$. We can
then write $\mu\perp\nu$.
\end{defn}

In the next result we summarize some basic properties of mutual singularity
and its relationship with absolute continuity.
\begin{prop}
\label{prop:propertiesSingularityAbsContinuity}Let $\measurablespace$
be a measurable space and let $\mu,\nu,$ and $\gamma$ be three measures
on $\F$. Then

\begin{enumerate}[label=(\roman*),leftmargin=.1\linewidth,rightmargin=.4\linewidth]
	\item $\mu \perp \nu$ if and only if there exist two disjoint sets $A,B \in \F$ such that $A \cup B= \X$ and $\mu(A) = 0 , \nu(A) = 0 $ .
	\item  If $\mu \perp \gamma $ and $\nu \perp \gamma $ then $\mu + \nu \perp \gamma$.
	\item If $\mu << \gamma $ and $\nu \perp \gamma$ then $\mu \perp \nu$.
	\item If $\mu << \nu $ and $\mu \perp \nu$ then $\mu = 0$.
\end{enumerate}
\end{prop}

\begin{proof}
For \emph{(i), }observe that if $\mu\perp\nu$ then there exist disjoint
sets $C,D\in\F$ such that $\mu$ concentrates on $C$ and $\nu$
concentrates on $D.$ By definition, we have that $\mu\left(C^{C}\right)=\mu\left(C\cap C^{C}\right)=0$
and $\nu\left(D^{C}\right)=\nu\left(D\cap D^{C}\right)=0.$ Note that
$\nu\left(C\right)=\nu\left(C\cap D\right)=0$ and so $A=C^{C}$and
$B=D^{C}$ does the trick. Conversely, let $A,B$ be as given in the
hypothesis. Then $\mu\left(F\right)=\mu\left(F\cap A\right)+\mu\left(F\cap B\right)=\mu\left(F\cap B\right)$
and similarly $\nu\left(F\right)=\nu\left(F\cap A\right)$ for every
$F\in\F$.

For \emph{(ii), }let $A_{1},B_{1}\in\F$ be a partition of $\X$ such
that $\mu\left(A_{1}\right)=0$ and $\gamma\left(B_{1}\right)=0.$
Similarly, let $A_{2},B_{2}\in\F$ be a partition of $\X$ such that
$\nu\left(A_{2}\right)=0$ and $\gamma\left(B_{2}\right)=0.$ Let
$C=A_{1}\cap A_{2}$ and $D=\left(A_{1}\cap A_{2}\right)^{C}=B_{1}^{C}\cup B_{2}^{C}.$
Clearly, $C$ and $D$ form a partition of $\X$ such that $\mu+\nu\left(A_{1}\cap A_{2}\right)=\mu\left(A_{1}\cap A_{2}\right)+\nu\left(A_{1}\cap A_{2}\right)=0$
by subadditivity. Similarly, $\gamma\left(B_{1}\cup B_{2}\right)\leq\gamma\left(B_{1}\right)+\gamma\left(B_{2}\right)=0.$

Next, for \emph{(iii) }let $A\in\F$ such that $\nu\left(A\right)=0$
and $\gamma\left(A^{C}\right)=0$. Absolute continuity implies that
$\mu\left(A^{C}\right)=0.$ This shows $\mu\perp\nu.$

Finally, suppose that $A\in\F$ such that $\mu\left(A\right)=0$ and
$\nu\left(A^{C}\right)=0.$ By absolute continuity, $\mu\left(A\right)=0.$
Then, $\mu\left(\X\right)=\mu\left(A\right)+\mu\left(A^{C}\right)=0.$
\end{proof}
It turns out that absolute continuity and singularity represent a
sort of \emph{orthogonality }notion for measures. Considering that
the spaces of measures is a subset of the dual space of $L^{1}$ functions,
this does not correspond to the canonical notion of orthogonality.
\hl{MAKE PRECISE USING TARCSAY 2014. }Again, this result falls trivially
out of Lemma \ref{lemma:LebesgueRadonNikodym}.
\begin{thm}[Lebesgue Decomposition]
\label{thm:lebesgueDecomposition}Let $\measurablespace$ be a measurable
space and let $\mu$ and $\nu$ be two $\sigma-$finite measures on
$\F$. Then there exist two unique measures $\mu_{1},\mu_{2}$ such
that 
\[
\mu=\mu_{1}+\mu_{2}
\]
and $\mu_{1}<<\nu$ and $\mu_{2}\perp\nu$. Such a pair $\left(\mu_{1},\mu_{2}\right)$
is called a Lebesgue decomposition of $\mu$ with respect to $\nu$.
\end{thm}

\begin{proof}
Note that existence follows from Lemma \ref{lemma:LebesgueRadonNikodym}.
To see this, write 
\[
\mu\left(A\right)=\lebInt{\nu}{k\indicate_{A}}+\mu\left(A\cap B\right)
\]
where $B\in N_{\nu}$ and $k$ is non-negative almost-everywhere with
respect to $\nu.$ Then setting $\text{\ensuremath{\mu_{1}\left(A\right):=\lebInt{\nu}{k\indicate_{A}}}}$we
have that $\mu_{1}<<\nu$. Similarly, setting $\mu_{2}\left(A\right):=\mu\left(A\cap B\right)$
which is a measure since it satisfies countable additivity and is
null on the empty set. Then note that $\nu\left(B\right)=0$ and $\mu_{2}\left(B^{C}\right)=0$
which implies that $\mu_{2}\perp\nu$.
\end{proof}

\subsection{{*}Signed measures, Duality of $L^{p}$ spaces, and the Riesz representation
theorem.}

\hl{Tao (epsilon of room vol 1, section 1.3.2)?}

\section{Derivatives of measures}

\hl{Bass chapter 14 117-123}

\section{Integrals of almost sure derivatives\label{sec:ftc}}

\hl{Bass Chapter 14 pg 123 onwards} 


