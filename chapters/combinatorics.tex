
\chapter{Basic combinatorics\label{chap:Combinatorics}}

\section{Binomial and multinomial coefficients}

The following is an \emph{algebraic }proof of Pascal's rule, a basic
combinatorial identity. In the remark that follows, we provide a more
combinatorial interpretation of the identity. Throughout this appendix,
we will try to provide both algebraic and combinatorial arguments
for results, unless the algebraic arguments are too cumbersome.
\begin{prop}[Pascal's rule]
\label{prop:pascalRule}For $n\geq k\geq1$
\[
\left(\begin{array}{c}
n\\
k
\end{array}\right)=\left(\begin{array}{c}
n-1\\
k-1
\end{array}\right)+\left(\begin{array}{c}
n-1\\
k
\end{array}\right).
\]
\end{prop}

\begin{proof}
Observe that 
\begin{align*}
\left(\begin{array}{c}
n-1\\
k-1
\end{array}\right)+\left(\begin{array}{c}
n-1\\
k
\end{array}\right) & =\frac{\left(n-1\right)!}{\left(k-1\right)!\left(n-k\right)!}+\frac{\left(n-1\right)!}{k!\left(n-k-1\right)!}\\
 & =\left(n-1\right)!\left[\frac{1}{\left(k-1\right)!\left(n-k\right)!}+\frac{1}{k!\left(n-k-1\right)!}\right]\\
 & =\left(n-1\right)!\left[\frac{k}{k!\left(n-k\right)!}+\frac{n-k}{k!\left(n-k\right)!}\right]\\
 & =\left(n-1\right)!\frac{n}{k!\left(n-k\right)!}\\
 & =\frac{n!}{k!\left(n-k\right)!}\\
 & =\left(\begin{array}{c}
n\\
k
\end{array}\right).
\end{align*}
\end{proof}
\begin{rem*}
The combinatorial idea for the above result is simple. Suppose you
want to select $k$ items from a collection of $n$ items, without
any consideration for order. How does one do so? . First we arbitrarily
pick some item and label it $x$. How many ways are ways are there
to pick items excluding $x$? Well, there are $\left(\begin{array}{c}
n-1\\
k
\end{array}\right)$ ways to pick $k$ items if we explicitly exclude $x$. What if we
insist on including $x?$ Then we have to pick $k-1$ items from $n-1$
remaining total items, that is $\left(\begin{array}{c}
n-1\\
k-1
\end{array}\right).$
\end{rem*}
\begin{example}[ISI 2021 PSA-11]
\label{exa:isi2021psa11}We can use Pascal's formula to recover a
more succinct formula for the expression
\[
\prod_{i=1}^{n}\left(\left(\begin{array}{c}
n\\
i
\end{array}\right)+\left(\begin{array}{c}
n\\
i-1
\end{array}\right)\right).
\]
We can apply Pascal's rule to each term in the product so that 
\begin{align*}
\prod_{i=1}^{n}\left(\left(\begin{array}{c}
n\\
i
\end{array}\right)+\left(\begin{array}{c}
n\\
i-1
\end{array}\right)\right) & =\prod_{i=1}^{n}\left(\begin{array}{c}
n+1\\
i
\end{array}\right)\\
 & =k\prod_{i=1}^{n}\left(\begin{array}{c}
n\\
i
\end{array}\right)
\end{align*}
where $k=\frac{\left(n+1\right)^{n}}{n!}.$
\end{example}

Pascal's rule helps us establish the binomial theorem.
\begin{thm}[Binomial theorem]
\label{thm:binomialTheorem}Let $n\geq1$ be an integer and let $x,y\in\R$.
Then,
\[
\left(x+y\right)^{n}=\sum_{k=0}^{n}\left(\begin{array}{c}
n\\
k
\end{array}\right)x^{k}y^{n-k}.
\]
\end{thm}

\begin{proof}
The identity is easily verified for $n=1$ and $n=2$. Supopose that
it holds for $n-1$ and observe
\begin{align*}
\left(x+y\right)^{n} & =x\left(x+y\right)^{n-1}+y\left(x+y\right)^{n-1}\\
 & =x\sum_{k=0}^{n-1}\left(\begin{array}{c}
n-1\\
k
\end{array}\right)x^{k}y^{n-1-k}+y\sum_{k=0}^{n-1}\left(\begin{array}{c}
n-1\\
k
\end{array}\right)x^{k}y^{n-1-k}\\
 & =\sum_{k=0}^{n-1}\left(\begin{array}{c}
n-1\\
k
\end{array}\right)x^{k+1}y^{n-1-k}+\sum_{k=0}^{n-1}\left(\begin{array}{c}
n-1\\
k
\end{array}\right)x^{k}y^{n-k}\\
 & =\sum_{k=1}^{n}\left(\begin{array}{c}
n-1\\
k-1
\end{array}\right)x^{k}y^{n-k}+\sum_{k=0}^{n-1}\left(\begin{array}{c}
n-1\\
k
\end{array}\right)x^{k}y^{n-k}\\
 & =\left(\begin{array}{c}
n-1\\
n-1
\end{array}\right)x^{n}+\left(\begin{array}{c}
n-1\\
0
\end{array}\right)y^{n}+\sum_{k=1}^{n-1}\left(\begin{array}{c}
n\\
k
\end{array}\right)x^{k}y^{n-k}\\
 & =\left(\begin{array}{c}
n\\
n
\end{array}\right)x^{n}+\left(\begin{array}{c}
n\\
0
\end{array}\right)y^{n}+\sum_{k=1}^{n-1}\left(\begin{array}{c}
n\\
k
\end{array}\right)x^{k}y^{n-k}\\
 & =\sum_{k=0}^{n}\left(\begin{array}{c}
n\\
k
\end{array}\right)x^{k}y^{n-k}
\end{align*}
where in the third line we used a change of variables, and in the
fourth we used Pascal's rule.
\end{proof}
\begin{rem*}
The combinatorial argument is again simple. We know that the coefficient
for $x^{k}y^{n-k}$ in the expansion of $\left(x+y\right)^{n}$ involves
choosing $k$ out of $n$ available $x'$s and the number of ways
to do that is $\left(\begin{array}{c}
n\\
k
\end{array}\right)$.
\end{rem*}
\begin{example}[ISI 2016 PSA 1]
\label{exa:isi2016psa1}How many terms in the binomial expansion
of $\left(3x^{2}+\frac{1}{x}\right)^{5}$ are independent of $x$?
Note that the binomial expansion of this expression is of the form
\begin{align*}
\left(3x^{2}+\frac{1}{x}\right)^{5} & =\sum_{i=0}^{5}\left(\begin{array}{c}
5\\
i
\end{array}\right)3x^{2i}x^{i-5}\\
 & =\sum_{i=0}^{5}\left(\begin{array}{c}
5\\
i
\end{array}\right)3x^{3i-5}.
\end{align*}
Note that $3i-5=0\implies i=\frac{5}{3}$ which is not an integer.
Thus no terms are independent of $x.$
\end{example}

The binomial theorem gives us a simple formula for the sum of binomial
coefficients.
\begin{cor}
\label{cor:sumOfBinomialCoefficients}Let $n\geq1$ be a fixed integer.
Then,
\[
\sum_{k=0}^{n}\left(\begin{array}{c}
n\\
k
\end{array}\right)=2^{n}.
\]
\end{cor}

\begin{proof}
Let $x=y=1$ in the binomial theorem.
\end{proof}
\begin{rem*}
Think about a set with $n$ elements. You want to construct all possible
subsets. You know the number of all possible subsets is $2^{n}$.
How do you count all possible subsets of a finite set? Well, order
doesn't matter in a set so all you have to do is count the empty set,
the sets with one element (singletons), the set with two elements,
those with three and so on.
\end{rem*}
\begin{example}[ISI 2013 PSB 3]
\label{exa:isi2013psb3}Let $S=\left\{ 1,2,\ldots,n\right\} .$ How
many ways can we choose two subsets $B\subseteq A\subseteq S$ such
that $B\neq\emptyset$? How many subsets $B\subsetneq A\subseteq S$
can we choose? For the first question, we know that for each subset
$A$ of size $k$ we can select $2^{k}-1$ nonempty subsets. Further,
there are $\left(\begin{array}{c}
n\\
k
\end{array}\right)$ subsets of size $k$ and so the number of such sets is $\sum_{k=1}^{n}\left(\begin{array}{c}
n\\
k
\end{array}\right)\left(2^{k}-1\right)$.
\end{example}

There are a number of interesting identities regarding binomial coefficients
which have simple combinatorial interpretations. We list a few here.
\begin{prop}
\label{prop:mdmActivity94}Let $n$ be a non-negative integer. Then,
\[
\left(\begin{array}{c}
2n\\
2
\end{array}\right)=2\left(\begin{array}{c}
n\\
2
\end{array}\right)+n^{2}.
\]
\end{prop}

\begin{proof}
Note that 
\begin{align*}
\left(\begin{array}{c}
2n\\
2
\end{array}\right) & =\frac{\left(2n\right)!}{2!\left(2n-2\right)!}\\
 & =\frac{2n\left(2n-1\right)}{2!}\\
 & =2n^{2}-n\\
 & =n(n-1)+n^{2}\\
 & =2\left(\begin{array}{c}
n\\
2
\end{array}\right)+n^{2}.
\end{align*}
\end{proof}
\begin{rem*}
Suppose you have $n$distinguihsable red balls and $n$ distinguishble
green balls. How many ways are there to choose two balls from this
collection? Well, you can get one red and one green and there are
$n^{2}$ ways of getting such pairs. Alternatively, you can get two
of red or two of green and there are $\left(\begin{array}{c}
n\\
2
\end{array}\right)$ possible ways to select two reds (or two greens).
\end{rem*}
\begin{prop}
\label{prop:mdmActivity95}Let $m,n\geq1$ be positive integers. Then,
\[
\left(\begin{array}{c}
m+n\\
2
\end{array}\right)-\left(\begin{array}{c}
m\\
2
\end{array}\right)-\left(\begin{array}{c}
n\\
2
\end{array}\right)=mn.
\]
\end{prop}

\begin{proof}
Note that 
\begin{align*}
\left(\begin{array}{c}
m+n\\
2
\end{array}\right)-\left(\begin{array}{c}
m\\
2
\end{array}\right)-\left(\begin{array}{c}
n\\
2
\end{array}\right) & =\frac{\left(m+n\right)!}{2!\left(m+n-2\right)!}-\frac{m!}{2!\left(m-2\right)!}-\frac{n!}{2!\left(n-2\right)!}\\
 & =\frac{\left(m+n\right)\left(m+n-1\right)-m(m-1)-n(n-1)}{2}\\
 & =\frac{m^{2}+n^{2}+2mn-m^{2}+m-n^{2}+n}{2}\\
 & =mn.
\end{align*}
\end{proof}
\begin{rem*}
Suppose you want to count the number of ways in which you can choose
2 items from $m$ distinguishable green balls and $n$ distinguishable
red balls and you want one of each. Then you can count the number
of ways you can choose two of those in aggregate and subtract out
the number of ways in which you could choose both of one color. On
the other hand there are clearly $mn$ such ways (by the product rule
if you will).
\end{rem*}
\begin{prop}
\label{prop:prodSumBinomialCoefficients}Let $n\geq1$ be an integer,.
Then,
\[
\sum_{i=1}^{n}i\left(\begin{array}{c}
n\\
i
\end{array}\right)=n2^{n-1}.
\]
\end{prop}

\begin{proof}
Note that by the binomial theorem,
\[
\left(1+x\right)^{n}=\sum_{k=0}^{n}\left(\begin{array}{c}
n\\
k
\end{array}\right)x^{k}.
\]
Taking derivatives on both sides, we have that 
\[
n\left(1+x\right)^{n-1}=\sum_{k=0}^{n}\left(\begin{array}{c}
n\\
k
\end{array}\right)kx^{k-1}.
\]
Letting $x=1$ yields the result.
\end{proof}
\begin{rem*}
\hl{Fill later}
\end{rem*}
\begin{prop}
\label{prop:squaredSumBinomialCoefficients}Let $n\geq1$ be a fixed
integer. Then,
\[
\sum_{k=0}^{n}\left(\begin{array}{c}
n\\
k
\end{array}\right)^{2}=\left(\begin{array}{c}
2n\\
n
\end{array}\right).
\]
\end{prop}

\begin{proof}
Consider the algebraic identity
\[
\left[\left(1+x\right)^{n}\right]^{2}=\left(1+x\right)^{2n}.
\]
We can expand the left hand side as 
\[
\left(\sum_{k=0}^{n}\left(\begin{array}{c}
n\\
k
\end{array}\right)x^{k}\right)^{2}=\sum_{k=0}^{n}\sum_{j=0}^{n}\left(\begin{array}{c}
n\\
k
\end{array}\right)\left(\begin{array}{c}
n\\
j
\end{array}\right)x^{j+k}.
\]
The coefficient to the term $x^{n}$ is $\sum_{k=0}^{n}\left(\begin{array}{c}
n\\
k
\end{array}\right)\left(\begin{array}{c}
n\\
n-k
\end{array}\right)=\sum_{k=0}^{n}\left(\begin{array}{c}
n\\
k
\end{array}\right)^{2}.$On the other hand, a simple application of binomial theorem tells
us that the coefficient of $x^{n}$ on the right hand side is $\left(\begin{array}{c}
2n\\
n
\end{array}\right)$ which completes the proof.
\end{proof}
\begin{rem*}
Suppose you have $n$ distinguishable red balls and $n$distinguishable
green balls. How many ways can you collect $n$ items from this group?
Well the answer is clearly $\left(\begin{array}{c}
2n\\
n
\end{array}\right)$ , but we can break this up by considering that we can select 1 red
ball and $n-1$ green balls, or $2$ red balls and $n-2$ green balls...
\end{rem*}
The above result is a special case of Vandermonde's identity, which
was known to mathematicians as early as the 14th century.
\begin{prop}
\label{prop:vandermondeIdentity}Let $m,n,$ and $r$ be non-negative
integers. Then,
\[
\left(\begin{array}{c}
m+n\\
r
\end{array}\right)=\sum_{k=0}^{r}\left(\begin{array}{c}
m\\
k
\end{array}\right)\left(\begin{array}{c}
n\\
r-k
\end{array}\right).
\]
\end{prop}

\begin{proof}
The algebraic proof is similar to the one for the special case in
Proposition \ref{prop:squaredSumBinomialCoefficients}. The combinatorial
argument is more illuminating and significantly less cumbersome. Consider
a committee consisting of $m$ men and $n$ women. We want to form
a subcommittee of $r$ members. Clearly the number of such subcommittes
is $\left(\begin{array}{c}
m+n\\
r
\end{array}\right)$. One way to form such a committee is to consider a committee with
one man and $r-1$ women, or two men and $r-2$ women, or ...
\end{proof}
\begin{prop}
\label{prop:mdmActivity103}Let $n\geq1$ be an integer. Then,
\[
\sum_{i=1}^{n}i\left(n+1-i\right)=\left(\begin{array}{c}
n+2\\
3
\end{array}\right).
\]
\end{prop}

\begin{proof}
For $n=1$ both the left and right hand sides of the identity above
are 1 and so the base case holds. Suppose the identity holds for $n$.
Then,
\begin{align*}
\sum_{i=1}^{n+1}i\left(n+1+1-i\right) & =\sum_{i=1}^{n+1}i\left(n+1-i\right)+\sum_{i=1}^{n+1}i\\
 & =\sum_{i=1}^{n}i\left(n+1-i\right)+\left(\begin{array}{c}
n+2\\
2
\end{array}\right)\\
 & =\left(\begin{array}{c}
n+2\\
3
\end{array}\right)+\left(\begin{array}{c}
n+2\\
2
\end{array}\right)\\
 & =\left(\begin{array}{c}
n+2\\
3
\end{array}\right)
\end{align*}
where the second equality is the Gaussian formula for the sum of consecutive
natural numbers, the third is the induction hypothesis, and the last
is Pascal's rule.
\end{proof}
\begin{prop}
\label{prop:sumOfBinomialCoefficients2}Let $n\geq2$ be an integer.
Then
\[
\sum_{k=2}^{n}\left(\begin{array}{c}
k\\
2
\end{array}\right)=\left(\begin{array}{c}
n+1\\
3
\end{array}\right).
\]
\end{prop}

\begin{proof}
$n=2$ the result follows easily since $\left(\begin{array}{c}
2\\
2
\end{array}\right)=\left(\begin{array}{c}
3\\
3
\end{array}\right)=1$. For $n=3$ we have $\left(\begin{array}{c}
2\\
2
\end{array}\right)+\left(\begin{array}{c}
3\\
2
\end{array}\right)=4=\left(\begin{array}{c}
4\\
3
\end{array}\right).$ Now suppose the result holds for $n-1$ and note that 
\begin{align*}
\sum_{k=2}^{n}\left(\begin{array}{c}
k\\
2
\end{array}\right) & =\sum_{k=2}^{n-1}\left(\begin{array}{c}
k\\
2
\end{array}\right)+\left(\begin{array}{c}
n\\
2
\end{array}\right)\\
 & =\left(\begin{array}{c}
n\\
3
\end{array}\right)+\left(\begin{array}{c}
n\\
2
\end{array}\right)\\
 & =\left(\begin{array}{c}
n+1\\
3
\end{array}\right)\\
\end{align*}
where the second equality uses the induction hypothesis and the last
uses Pascal's rule.
\end{proof}
We can generalize the notion of binomial coefficients to that of multinomial
coefficients with the following result.
\begin{prop}
\label{prop:multinomialCoefficients}Let $r_{1},r_{2},\ldots,r_{k}$
be positive integers and let $n=\sum_{i=1}^{k}r_{i}$. Then the number
of ways to split a set of size $n$ into $k$ ordered subsets where
the $i$th subset contains $r_{i}$ elements is given by
\[
\left(\begin{array}{cccc}
 & n\\
r_{1}, & r_{2}, & \ldots & r_{k}
\end{array}\right):=\frac{n!}{r_{1}!r_{2}!\ldots r_{k}!}.
\]
\end{prop}

\begin{proof}
Note that first we choose $r_{1}$ items from $n$ items, then $r_{2}$
items from $n-r_{1}$ remaining items, then $r_{3}$ from the remaining
$n-r_{1}-r_{2}$ and so on, yielding
\begin{align*}
\left(\begin{array}{cccc}
 & n\\
r_{1}, & r_{2}, & \ldots & r_{k}
\end{array}\right) & =\prod_{i=1}^{k}\left(\begin{array}{c}
n-\sum_{j=0}^{i-1}r_{j}\\
r_{i}
\end{array}\right)\\
 & =\prod_{i=1}^{k}\frac{\left(n-\sum_{j=0}^{i-1}r_{j}\right)!}{r_{i}!\left(n-\sum_{j=0}^{i}r_{j}\right)!}\\
 & =\frac{n!}{r_{1}!r_{2}!\ldots r_{k}!}.
\end{align*}
The last equality follows because there's a ``telescoping'' product
and so the all $\left(n-\sum_{j=0}^{i-1}r_{j}\right)!$ terms except
the first (which is is $n!$) cancel.
\end{proof}
This gives us the multinomial theorem.
\begin{thm}[Multinomial theorem]
\label{thm:multinomialTheorem}Let $m,n\geq1$ be fixed integers.
Then, for $x_{1},x_{2},\ldots,x_{m}\in\R$
\[
\left(\sum_{i=1}^{m}x_{i}\right)^{n}=\sum_{k_{1}+\ldots+k_{m}=n,k_{i}\geq0}\left(\begin{array}{cccc}
 & n\\
k_{1}, & k_{2}, & \ldots & k_{m}
\end{array}\right)\prod_{i=1}^{m}x_{i}^{k_{i}}.
\]
\end{thm}

\begin{proof}
For $m=1$ the result is trivial. For $m=2$, we shall verify that
it is in fact the binomial theorem. To see this, note that for $m=2$,
the binomial theorem implies that 
\begin{align*}
\left(x_{1}+x_{2}\right)^{n} & =\sum_{k=0}^{n}\left(\begin{array}{c}
n\\
k
\end{array}\right)x_{1}^{k}x_{2}^{n-k}\\
 & =\sum_{k=0}^{n}\left(\begin{array}{c}
n\\
k,n-k
\end{array}\right)x_{1}^{k}x_{2}^{n-k}\\
 & =\sum_{k_{1}+k_{2}=n,k_{i}\geq0}\left(\begin{array}{c}
n\\
k_{1},k_{2}
\end{array}\right)x_{1}^{k_{1}}x_{2}^{k_{2}}
\end{align*}
where the second equality uses Proposition \ref{prop:multinomialCoefficients}.
To extend this result to arbitrary $m$, we assume that it holds for
$m-1$ and then note that 
\begin{align*}
\left(\sum_{i=1}^{m-1}x_{i}+x_{m}\right)^{n} & =\sum_{K+k_{m}=n,K,k_{m}\geq0}\left(\begin{array}{c}
n\\
K,k_{m}
\end{array}\right)\left(\sum_{i=1}^{m-1}x_{i}\right)^{K}x_{m}^{k_{m}}\\
 & =\sum_{K+k_{m}=n,K,k_{m}\geq0}\left(\begin{array}{c}
n\\
K,k_{m}
\end{array}\right)\sum_{k_{1}+\ldots+k_{m-1}=K,k_{i}\geq0}\left(\begin{array}{c}
K\\
k_{1},\ldots,k_{m-1}
\end{array}\right)\prod_{i=1}^{m}x_{i}^{k_{i}}\\
 & =\sum_{K+k_{m}=n,K,k_{m}\geq0}\sum_{k_{1}+\ldots+k_{m-1}=K,k_{i}\geq0}\left(\begin{array}{c}
n\\
K,k_{m}
\end{array}\right)\left(\begin{array}{c}
K\\
k_{1},\ldots,k_{m-1}
\end{array}\right)\prod_{i=1}^{m}x_{i}^{k_{i}}\\
 & =\sum_{K+k_{m}=n,K,k_{m}\geq0}\sum_{k_{1}+\ldots+k_{m-1}=K,k_{i}\geq0}\left(\begin{array}{cccc}
 & n\\
k_{1}, & k_{2}, & \ldots & k_{m}
\end{array}\right)\prod_{i=1}^{m}x_{i}^{k_{i}}\\
 & =\sum_{k_{1}+\ldots+k_{m}=n,k_{i}\geq0}\left(\begin{array}{cccc}
 & n\\
k_{1}, & k_{2}, & \ldots & k_{m}
\end{array}\right)\prod_{i=1}^{m}x_{i}^{k_{i}}
\end{align*}
which completes the proof.

There is an obvious generalization of Vandermonde's identity that
should occur to you at this point.
\end{proof}
\begin{prop}[Generalized Vandermonde identity]
\label{prop:generalizedVandermondeIdentity}Let $\left\{ n_{i}\right\} _{i=1}^{k},r$
be non-negative integers. Then,
\[
\left(\begin{array}{c}
\sum_{i=1}^{k}n_{i}\\
r
\end{array}\right)=\sum_{i_{1}+i_{2}+\ldots+i_{k}=r}\prod_{j=1}^{k}\left(\begin{array}{c}
n_{j}\\
i_{j}
\end{array}\right).
\]
\end{prop}


\subsubsection{Stars and bars}

Sums like $\sum_{i_{1}+i_{2}+\ldots+i_{k}=r}$ should trouble you,
since it's not obvious how many terms are in such a sum. Fortunately,
we have the tools we need to be able to answer such questions. To
do so, we make an analogy with a different type of question: suppose
you have $k$ urns and $r$ indistinguishable balls. How many ways
could you allocate balls to urns? William Feller's famous textbook
on probability provided an ingenious framework to think about this
problem: suppose you have $k+1$ bars and $r$ stars, you could then
fix two bars at two ends and put all the stars and remaining bars
in the middle. The gap between bars would act as urns and the stars
would be balls. Suppose you have four bars and ten stars, one arrangment
could be like $|**|*****|**|$. This corresponds to the solution where
there are $3$ urns and the first urn has $2$ balls, the second has
$5$ balls, and the last again has 2 balls. How many other possibilities
are there? Well we can rearrange the bars and stars in the interior
of by looking at the number of slots in the interior (in this case
12) and choosing the slots for either the bars or the stars (in this
case 2 or 10). So the answer is $\left(\begin{array}{c}
12\\
2
\end{array}\right)=\left(\begin{array}{c}
12\\
10
\end{array}\right)$. More generally, there are $k-1$ bars in the interior along with
$r$ stars, leading to $\left(\begin{array}{c}
r+k-1\\
k-1
\end{array}\right)=\left(\begin{array}{c}
r+k-1\\
r
\end{array}\right)$ ways to arrange the interior bars and stars. Equivalently, there
are $\left(\begin{array}{c}
r+k-1\\
k
\end{array}\right)$ ways to put $r$ balls in $k$ urns. This simple idea has remarkable
power in its ability to resolve combinatorial problems. For instance,
if we apply the restriction that each urn needs to have at least $1$
ball, then our result becomes $\left(\begin{array}{c}
k-1+(r-k)\\
r-k
\end{array}\right)=\left(\begin{array}{c}
r-1\\
k-1
\end{array}\right).$
\begin{example}[ISI 2018 PSA 13 (variant)]
\label{exa:countingFunctions}How many functions $f:\left\{ 1,2,\ldots,n\right\} \to\left\{ 1,2,\ldots,m\right\} $
are strictly increasing? How many are non-decreasing? To answer questions
like this, we can use some of the ideas from the discussion above.
First, you should notice that there are $\left(\begin{array}{c}
m\\
n
\end{array}\right)$ ways to choose the range of the function (it has to be injective
after all), and exactly one way for each of those ways to arrange
them in increasing order. This gives us the total number of strictly
increasing functions. To count non-decreasing functions, we
\end{example}

\begin{example}[ISI 2016 PSA 21]
\label{exa:isi2016psa21}Let $A=\{1,2,\ldots n,\ldots,m\}$. How
many functions $f:A\rightarrow A$ are there such that $f(1)<f(2)<\ldots,<f(n)$
? Well there are $\left(\begin{array}{c}
m\\
n
\end{array}\right)$ ways of selecting the strictly increasing elements and $\left(m-n\right)^{\left(m-n\right)}$
ways of choosing how the remaining elements are arranged. Thus the
total number of of such functions is $\left(\begin{array}{c}
m\\
n
\end{array}\right)\left(m-n\right)^{\left(m-n\right)}.$
\end{example}

\begin{example}[ISI 2016 PSA 11]
\label{exa:isi2016psa11}The number of ordered pairs $\left(a,b\right)\in\N^{2}$
such that $a+b\leq n$ where $n\in\N$ is $\sum_{i=2}^{n}\left(\begin{array}{c}
i-1\\
2-1
\end{array}\right)=1+2+3+\ldots+n-1=\frac{n\left(n-1\right)}{2}$
\end{example}


\section{Inclusion-exclusion and its consequences}
\begin{lem}
\label{lem:inclusionExclusion}Let $A$ denote the union of sets $A_{1},A_{2},\ldots,A_{n}$,
all of which are subsets of some ambient set $\X$. Then,
\begin{align}
\indicate_{A} & =\sum_{i=1}^{n}\left(-1\right)^{i-1}\sum_{J\subset\left\{ 1,2,\ldots n\right\} ,\lvert J\rvert=i}\indicate_{\bigcap_{j\in J}A_{j}}.\label{eq:inclusionExclusionIndicator}
\end{align}
\end{lem}

\begin{proof}
Consider the function 
\[
g\left(x\right)=\prod_{i=1}^{n}\left(\indicate_{A}\left(x\right)-\indicate_{A_{i}}\left(x\right)\right).
\]
We claim that $g\left(x\right)=0$ for all $x\in\X$. To see this,
notice that if $x\in A_{i}$ for any $1\leq i\leq n$ then that particular
factor is zero. Conversely if $x\notin A$ then all the factors are
zero. Rearranging the equation $g\left(x\right)=0$ and using Fact
(\ref{fact:indicatorFunctionsFiniteOperations})yields the result.
\end{proof}
\begin{thm}[Inclusion-Exclusion]
\label{thm:inclusionExclusionCardinality}Let $A_{1},A_{2},\ldots,A_{n}$
be finite sets and let $A=\bigcup A_{i}$ . Then,
\[
\lvert A\rvert=\sum_{i=1}^{n}\left(-1\right)^{i-1}\sum_{J\subset\left\{ 1,2,\ldots n\right\} ,\lvert J\rvert=i}\lvert\bigcap_{j\in J}A_{j}\rvert.
\]
\end{thm}

\begin{proof}
We are going to cheat here and use measure theory. We can take the
integral with respect to the counting measure on (\ref{eq:inclusionExclusionIndicator})and
recover the result by linearity.
\end{proof}
The inclusion-exclusion principle is very useful in counting the number
of \emph{derangements} of a given set. A derangement of a finite set
$A$ is a permutation on that set with no fixed points.
\begin{prop}
\label{prop:numDerangements}Let $A$ be a finite set such that $\lvert A\rvert=n.$
The number of derangements $\sigma:A\to A$ is given by 
\[
!n:=n!\sum_{i=0}^{n}\frac{\left(-1\right)^{i}}{i!}.
\]
\end{prop}

\begin{proof}
Without loss of generality, we can assume that $A=\left\{ 1,2,\ldots,n\right\} .$
Let $S_{k}:=\left\{ \sigma\in\mathrm{Perm}\left(A\right)\mid\sigma\left(k\right)=k\right\} $
for $1\leq k\leq n.$ That is, each $S_{k}$ fixes $k$ and may or
may not fix any other elements. Then, for any $J\subset A$, we have
that 
\[
\sum_{J\subset\left\{ 1,2,\ldots n\right\} ,\lvert J\rvert=i}\lvert\bigcap_{j\in J}S_{j}\rvert=\left(\begin{array}{c}
n\\
i
\end{array}\right)\left(n-i\right)!
\]
because the intersection of any $i$ elements of $\left\{ S_{k}\right\} _{1\leq k\leq n}$
consists of permutations which fix at least $i$ points and there
are $\left(\begin{array}{c}
n\\
i
\end{array}\right)$ ways to pick $i$ fixed points and $\left(n-i\right)!$ ways to permute
all the other elements. Then, the number of ways in which you can
have at least one fixed point is given by the inclusion-exclusion
formula
\begin{align*}
\lvert\bigcup_{i=1}^{n}S_{i}\rvert & =\sum_{i=1}^{n}\left(-1\right)^{i-1}\sum_{J\subset\left\{ 1,2,\ldots n\right\} ,\lvert J\rvert=i}\lvert\bigcap_{j\in J}A_{j}\rvert\\
 & =\sum_{i=1}^{n}\left(-1\right)^{i-1}\left(\begin{array}{c}
n\\
i
\end{array}\right)\left(n-i\right)!\\
 & =\sum_{i=1}^{n}\left(-1\right)^{i-1}\frac{n!}{i!\left(n-i\right)!}\left(n-i\right)!\\
 & =n!\sum_{i=1}^{n}\frac{\left(-1\right)^{i-1}}{i!}.
\end{align*}
The number of derangements is then simply the difference between the
total number of permutations and the number of permutations that has
at least one fixed point i.e.
\begin{align*}
!n & =n!-n!\sum_{i=1}^{n}\frac{\left(-1\right)^{i-1}}{i!}\\
 & =n!\left(1-\sum_{i=1}^{n}\frac{\left(-1\right)^{i-1}}{i!}\right)\\
 & =n!\left(1+\sum_{i=1}^{n}\frac{\left(-1\right)^{i}}{i!}\right)\\
 & =n!\sum_{i=0}^{n}\frac{\left(-1\right)^{i}}{i!}.
\end{align*}
\end{proof}
\begin{rem*}
There's a recursive formulation for counting derangements as well.
To see this, first think about \hl{Finish later}
\end{rem*}
\begin{cor}
\label{cor:recontresNumbers}Let $A$ be a finite set with cardinality
$n.$ The number of permutations $\sigma:A\to A$ with exactly $k$
fixed points, where $1\leq k\leq n$. is 
\[
D_{n,k}:=\left(\begin{array}{c}
n\\
k
\end{array}\right)!\left(n-k\right).
\]
\end{cor}

It's worth noting down the values of small derangements so that one
isn't forced to compute these when solving problems (much in the way
we often memorize small factorials).

\begin{table}

\caption{Values of derangements}

\begin{centering}
\begin{tabular}{ccc}
\hline 
$n$ & $n!$ & $!n$\tabularnewline
\hline 
\hline 
1 & 1 & 0\tabularnewline
2 & 2 & 1\tabularnewline
3 & 6 & 2\tabularnewline
4 & 24 & 9\tabularnewline
5 & 120 & 44\tabularnewline
6 & 720 & 265\tabularnewline
7 & 5040 & 1854\tabularnewline
8 & 40320 & 14833\tabularnewline
9 & 362880 & 133496\tabularnewline
10 & 3628800 & 1334961\tabularnewline
\end{tabular}
\par\end{centering}
\end{table}

\begin{example}[JAM 2022 P-54]
\label{exa:jam2022p54}Suppose that five men go to a restaurant together
and each of them orders a dish that is different from the dishes ordered
by the other members of the group. However, the waiter serves the
dishes randomly. Then what is the number of ways in which exactly
one of them gets the dish he ordered? The answer is $D_{5,1}=\left(\begin{array}{c}
5\\
1
\end{array}\right)!4=5\times9=45$.
\end{example}

We can generalize the idea of derangements by considering permutations
that don't have \emph{cycles. }A cycle is itself a sort of generalization
of a fixed point. A permutation $f$ on a finite set $A$ of size
$n$ is said to have a $k-$cycle if there exists some $x\in A$ such
that $f^{k}\left(x\right)=x$ where the exponent is denoting repeated
composition rather than multiplication. Of course, here $k$ represents
the \emph{smallest }positive integer such that the equality holds
true. It should be clear that it also holds true for any \emph{multiple
}of $k$. To understand cycles in permutations, it's useful to adopt
a notation for describing permutations that can help clarify their
cyclic structure. This is the so called \emph{cyclic notation}. For
example, we can write a permutation on $\left\{ 1,2,3,4,5\right\} $
\[
f=\left(351\right)\left(24\right)
\]
which basically tells us that the permutation consists of two cycles
$\left(351\right)$ and $\left(24\right)$. The cycles here are \emph{ordered},
in that the first cycle represents the fact that $f\left(3\right)=5,f\left(5\right)=1$
and $f\left(1\right)=3$ and the second cycle tells us that $f\left(2\right)=4$
and $f\left(4\right)=2$. Every permutation can be decomposed into
cycles, and \emph{non-cyclic }permutations are those that consist
of only the trivial cycle: for instance, a non-cyclic permutation
on $\left\{ 1,2,3,4,5\right\} $ would be the permutation $\left(14235\right).$
Note that two cycles of length $k$ $\left(x_{1}x_{2}\ldots x_{k}\right)$
and $\left(y_{1}y_{2}\ldots y_{k}\right)$ are equivalent if there
exists some $p\in\N$ such that $y_{i}=f^{p}\left(x_{i}\right)$.
This defines an equivalence relation (as should be clear), so we can
talk about \emph{equivalence classes} of cycles, denoted $\left[\left(x_{1}x_{2}\ldots x_{k}\right)\right]$.
Such an equivalence class consists of exactly $k$ distinct members
since $f^{k}\left(x_{i}\right)=x_{i}.$ Thus the total number of cycles
of length $k$ that can be formed out of $k$ fixed elements is $\frac{k!}{k}$.
We can use this to count the total number of permutations without
any cycles using an inclusion-exclusion argument analagous to the
one used for counting derangements in Proposition \ref{prop:numDerangements}.
\begin{example}[ISI 2021 PSA 14]
\label{exa:isi2021psa14}What is the total number of permutations
$\sigma:\left\{ 1,2,\ldots,6\right\} \to\left\{ 1,2,\ldots,6\right\} $
such that $\sigma\left(\sigma\left(i\right)\right)\neq i$ for any
$i\in\left\{ 1,2,\ldots,6\right\} $?\hl{TODO}
\end{example}


\section{Miscellaneous problems}
\begin{example}[ISI 2019 PSA 11]
\label{exa:isi2019psa11}What are the total number of divisors of
$2^{5}5^{3}11^{4}$ that are perfect squares? The prime square divisors
are $2^{2},5^{2},$ and $11^{2}$, where the first and last divisors
appear twice. Thus the first product can appear at most twice, the
second at most once, and the third at most twice in any square divisor
and so the number is $(2+1)(1+1)(2+1)=18$. More generally, for any
positive integer $n$ with prime factorization $n=\prod_{i=1}^{k}p_{i}^{r_{i}}$,
we can decompose the product into the largest square divisor 
\[
n=\prod_{i=1}^{k}\left(p_{i}^{2}\right)^{\lfloor\frac{r_{i}}{2}\rfloor}\prod_{i=1}^{k}p_{i}^{\indicate\left\{ \lfloor\frac{r_{i}}{2}\rfloor\neq\frac{r_{i}}{2}\right\} }
\]
where the second product is square-free and so the number of square
divisors is given $\prod_{i=1}^{k}\left(\lfloor\frac{r_{i}}{2}\rfloor+1\right)$.
\end{example}

\begin{example}[ISI 2019 PSA 18]
\label{exa:isi2019psa18}Draw one observation $N$ at random from
the set $\{1,2,\ldots,100\}$. What is the probability that the last
digit of $N^{2}$ is 1 ? Well note that only if the units digit of
$N$ is 1 or 9 does the units digit of $N^{2}$ equal 1, which tells
us that the probability is $\frac{1}{5}.$
\end{example}

\begin{example}[ISI 2019 PSA 6]
\label{exa:isi2019psa6} How many times does the digit '2' appear
in the set of integers $\{1,2,..,1000\}$ ? In the units digit, '2'
appears $10\times10=100$ times; in the tens digit, it appears 10
times and in the 100s digit it appears once. Thus in total it appears
111 times.
\end{example}

\begin{example}[ISI 2019 PSA 5]
\label{exa:isi2019psa5}Let the sum $3+33+333+\cdots+\underbrace{33\ldots3}_{200\text{ times }}$
be $...zyx$ in the decimal system, i.e., $x$ is the unit's digit,
$y$ the ten's digit, and so on. What is $z$ ?
\end{example}

\begin{example}[ISI 2021 PSA 12]
\label{exa:isi2021psa12} Let $\pi=\left(a_{1},a_{2},\cdots,a_{2021}\right)$
be a permutation of $(1,2,\cdots,2021)$. For every such permutation
$\pi$, 
\[
P(\pi)=\prod_{j=1}^{2021}\left(a_{j}-j\right).
\]

Is $P\left(\pi\right)$ always even for any permutation $\pi$? Yes.
To see this, note that $\sum_{j=1}^{2021}\left(a_{j}-j\right)=0$
and so 
\[
\sum_{j\neq i}\left(a_{j}-j\right)=-\left(a_{i}-i\right)
\]
where for any $i\in\left\{ 1,2,\ldots,2021\right\} $ the sum on the
LHS has an even number of terms. If the terms are all odd, the sum
(and therefore the RHS) is even and thus the product is even.
\end{example}

\begin{example}[ISI 2020 PSA 14]
\label{exa:isi2020psa14}Let $S$ be the set of all $3\times3$ matrices$A$
such that among the 9 entries of $A$, there are exactly three 0 's,
exactly three 1 's and exactly three 2 's. What is the number of matrices
in $S$ that have trace divisible by 3? \hl{TODO}
\end{example}

\begin{example}[ISI 2023 PSB 2]
\label{exa:isi2023psb2}How many permutations of the numbers $1,2,\ldots,n$
where $n$is even exist such that no two adjacent numbers have an
odd product? Let's first count the number of such permutations where
the first slot is taken up by an even number. The number of positive
even integers less or equal to $n$ is $\frac{n}{2}$ and so answer
is for these is $\frac{n}{2}!^{2}$. The odd first answer is $\frac{n}{2}!^{2}$
so together the answer is $2\frac{n}{2}!^{2}$.
\end{example}

\begin{example}[ISI 2015 PSB 4]
\label{exa:isi2015psb4}Suppose 15 identical balls are placed in
3 boxes labeled A, B and C. What is the number of ways in which Box
A can have more balls than Box C?\hl{TODO}
\end{example}


