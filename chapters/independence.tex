
\chapter{Independence and random variables}

Formal probability theory is often described as measure theory (or
more generally, analysis) on measure spaces $\measurespace$ where
the $\mu\left(\X\right)=1$ and to a certain extent the foundational
theory does resemble this characterization. However probability itself
is distinct from foundational probability \emph{theory}, in that its
goals and aims are to solve problems that are of a fundamentally different
character than problems seen in analysis.

Probability as as subject has a strong combinatorial flavor, since
it owes its origins to gambling and games of chance considered by
amateur mathematicians in the 17th and 18th centuries. These classical
ideas still permeate the modern \emph{probabilistic way of thinking.}
As such, it would be useful to revise the basic combinatorial tools
that are indispensable when tackling problems of this nature. These
are discussed in Appendix \ref{chap:Combinatorics}.

In general, we can say that probability theory ``has a right hand
and a left hand''. The right hand is the rigorous measure-theoretic
axiomatization that one encounters in a course on probability theory.
The left hand is the probablistic intuition that one develops through
trying to model stochastic phenomena with the aforementioned probabilistic
way of thinking. Probabilists aim to be ambidextrous in this regard,
using the axiomatic framework of probability theory to deduce facts
about processes whose parts and subparts can be reduced to \textquotedblleft na�ve\textquotedblright{}
probabilistic concepts.

In order to cover both parts of probability adequately, we will use
the background in analysis developed in Part I of these notes to prove
general theorems, while using numerous examples to develop probabilistic
intuition. I hope that the combination of theory and examples prove
sufficient in painting a vivid picture of probability theory, with
a view towards applications in statistics and economics.

\section{Probability spaces and probability measures}

While formal probability theory is based on measure theory, the language
of probability theory is different. To start our exploration into
probability theory, we shall first have to translate a lot of the
basic terminology of measure theory into the langauge of probability.
To begin with, we specialize the notion of a measure space $\measurespace$
to a probability space $\probabilityspace$ where $\mathbb{P}\left(\Omega\right)=1$.
Measurable sets $A\in\F$ are called \emph{events }in the language
of probability. The basic properties of probability measures carry
over from Chapter \ref{chap:measures}; we list a few more for completeness.
\begin{prop}[Inclusion-Exclusion]
\label{prop:inclusionExclusionProbability}Let $\probabilityspace$
be a probability space and let $A_{1},A_{2},\ldots,A_{n}\in\F$. Then
\[
\mathbb{P}\left(\bigcup_{i=1}^{n}A_{i}\right)=\sum_{i=1}^{n}\left(-1\right)^{i-1}\sum_{J\subset\left\{ 1,2,\ldots n\right\} ,\lvert J\rvert=i}\mathbb{P}\left(\bigcap_{j\in J}A_{j}\right).
\]
\end{prop}

\begin{proof}
Integrate the equality (\ref{eq:inclusionExclusionIndicator}) in
Lemma (\ref{lem:inclusionExclusion}).
\end{proof}
\begin{example}[ISI 2017 PSA 23]
\label{exa:isi2017psa23}Three numbers are chosen at random from
$\{1,2,\ldots,10\}$ without replacement. What is the probability
that the minimum of the chosen numbers is $3$ or their maximum is
$7$ ? Let $A$ be the event that we select three numbers with minimum
$3$ and $B$ be the event that we select three numbers with maximum
$7$. We want $\mathbb{P}\left(A\cup B\right)=\mathbb{P}\left(A\right)+\mathbb{P}\left(B\right)-\mathbb{P}\left(A\cap B\right).$
We need to compute each component. To compute the first term, note
that there are $\left(\begin{array}{c}
7\\
2
\end{array}\right)$ ways to pick 2 numbers without replacement after fixing the first
one at 3 since we are picking only from the remaining numbers which
are greater than three. Therefore $\mathbb{P}\left(A\right)=\left(\begin{array}{c}
7\\
2
\end{array}\right)/\left(\begin{array}{c}
10\\
3
\end{array}\right).$ Similarly, we have that $\mathbb{P}\left(B\right)=\left(\begin{array}{c}
6\\
2
\end{array}\right)/\left(\begin{array}{c}
10\\
3
\end{array}\right).$ To find the intersection, note that after fixing $3$ and $7$ there
are only 4 choices of the middle number, leaving us with $\mathbb{P}\left(A\cap B\right)=$$4/\left(\begin{array}{c}
10\\
3
\end{array}\right)$. In sum,
\begin{align*}
\mathbb{P}\left(A\cup B\right) & =\frac{\left(\begin{array}{c}
7\\
2
\end{array}\right)+\left(\begin{array}{c}
6\\
2
\end{array}\right)-4}{\left(\begin{array}{c}
10\\
3
\end{array}\right)}\\
 & =\frac{21+15-4}{120}\\
 & =\frac{4}{15}.
\end{align*}
\end{example}

\begin{prop}
\label{prop:bonferroniInequality}Let $\probabilityspace$ be a probability
space and let $A_{1},A_{2},\ldots,A_{n}\in\F$. Then
\[
\mathbb{P}\left(\bigcap_{i=1}^{n}A_{i}\right)\geq\sum_{i=1}^{n}\mathbb{P}\left(A_{i}\right)-(n-1).
\]
\end{prop}

\begin{proof}
First note that for $n=2$, the result follows due to the fact that
for any $A,B\in\F$
\begin{align*}
\mathbb{P}\left(A\cup B\right) & =\mathbb{P}\left(A\right)+\mathbb{P}\left(B\right)-\mathbb{P}\left(A\cap B\right)\\
 & \leq\mathbb{P}\left(\Omega\right)\\
 & =1.
\end{align*}
Now assume the induction hypothesis and note that 
\begin{align*}
\mathbb{P}\left(\bigcap_{i=1}^{n}A_{i}\right) & =\mathbb{P}\left(\bigcap_{i=1}^{n-1}A_{i}\cap A_{n}\right)\\
 & \geq\mathbb{P}\left(\bigcap_{i=1}^{n-1}A_{i}\right)+\mathbb{P}\left(A_{n}\right)-1\\
 & \geq\sum_{i=1}^{n-1}\mathbb{P}\left(A_{i}\right)-(n-2)+\mathbb{P}\left(A_{n}\right)-1\\
 & =\sum_{i=1}^{n}\mathbb{P}\left(A_{i}\right)-\left(n-1\right).
\end{align*}
\end{proof}

\section{Independent events}

In Chapter \ref{chap:productMeasures} we discussed the notion of
product measures; the analagous concept in probability is that of
independence. Informally, we think of independent events as those
where the occurence or non-occurenece of one event does not impact
the occurence or non-occurence of another. We can formalize this idea
with the following definition-
\begin{defn}
\label{def:independence}Let $I$ be an arbitrary index set. Then
a collection of events\emph{ }$\left\{ A_{i}\right\} _{i\in I}\subset\F$
are independent if for every finite subset $J\subset I$
\[
\mathbb{P}\left[\bigcap_{j\in J}A_{j}\right]=\prod_{j\in J}\mathbb{P}\left[A_{j}\right].
\]
Two independent events $A,B\in\F$ are often denoted $A\indep B$.
\end{defn}

Notice how the index set $I$ was arbitrary and so we can have countable
or uncountable collections of independent events. Of course, this
notion isn't particualrly interesting unless we have some rich examples
of such events and the probability spaces they live in. More generally,
we have a question of existence: do independent events always exist,
no matter the probability space? The answer to this question is trivially
yes since $\mathbb{P}\left(\Omega\cap\emptyset\right)=\mathbb{P}\left(\Omega\right)\mathbb{P}\left(\emptyset\right)=0.$
So then the question reduces to asking whether \emph{non-trivial }independent
events always exist. Here of course, the answer is ``no'', since
for $\F=\left\{ \emptyset,A,A^{C},\Omega\right\} $ where $0<\mathbb{P}\left(A\right)<1$,
we have that $\mathbb{P}\left(A\cap A^{C}\right)=0$ but $\mathbb{P}\left(A\right)\mathbb{P}\left(A^{C}\right)=\mathbb{P}\left(A\right)\left(1-\mathbb{P}\left(A\right)\right)\neq0.$
Nevertheless, we have a rich collection of examples of independent
events: think about rolling a dice twice and observing a six in each
roll; these event of seeing a six in the first roll is independent
of the event of seeing one in the next roll.
\begin{example}[ISI 2017 PSA 15]
\label{exa:isi2017psa15}Suppose that the events $A,B$, and $C$
are pairwise independent such that each of them occurs with probability
$p$. Assume that all three of them cannot occur simultaneously. What
is $P(A\cup B\cup C)$ ? Well, we apply independence and inclusion
exclusion to note that 
\begin{align*}
\mathbb{P}\left(A\cup B\cup C\right) & =\mathbb{P}\left(A\right)+\mathbb{P}\left(B\right)+\mathbb{P}\left(C\right)-\mathbb{P}\left(A\cap B\right)-\mathbb{P}\left(A\cap C\right)-\mathbb{P}\left(B\cap C\right)\\
 & =\mathbb{P}\left(A\right)+\mathbb{P}\left(B\right)+\mathbb{P}\left(C\right)-\mathbb{P}\left(A\right)\mathbb{P}\left(B\right)-\mathbb{P}\left(A\right)\mathbb{P}\left(C\right)-\mathbb{P}\left(B\right)\mathbb{P}\left(C\right)\\
 & =3p(1-p).
\end{align*}
\end{example}

The canonical example of the an infinite collection of independent
events is given by the following description of an infinitely repeated
experiment.
\begin{example}
\label{exa:infinitelyRepeatedExperiment}Let $E$ consist of a finite
set of outcomes and let $\Omega=E^{\N}$ be the collection of $E$-valued
sequences. Thus for any $\omega\in\Omega$, we can write
\[
\omega=\left(\omega_{1},\omega_{2},\ldots\right)
\]
where $\omega_{i}\in E$. We can then construct the collection
\[
\left[\omega_{1}^{*},\ldots,\omega_{n}^{*}\right]:=\left\{ \omega\in\Omega\mid\omega_{i}=\omega_{i}^{*},1\leq i\leq n\right\} 
\]
\hl{TODO}
\end{example}

\begin{lem}
\label{lem:independenceComplements}Let $\probabilityspace$ be a
probability space and let $A,B\in\F$ be events. Then the claims that
$A,B$ are independent, $A,B^{C}$ are independent, $A^{C},B$ are
independent, and $A^{C},B^{C}$are independent are equivalent.
\end{lem}

\begin{proof}
Note that 
\begin{align*}
\mathbb{P}\left(A\cap B^{C}\right) & =\mathbb{P}\left(A\setminus B\right)\\
 & =\mathbb{P}\left(A\setminus A\cap B\right)\\
 & =\mathbb{P}\left(A\right)-\mathbb{P}\left(A\cap B\right)\\
 & =\mathbb{P}\left(A\right)-\mathbb{P}\left(A\right)\mathbb{P}\left(B\right)\\
 & =\mathbb{P}\left(A\right)\left(1-\mathbb{P}\left(B\right)\right)\\
 & =\mathbb{P}\left(A\right)\mathbb{P}\left(B^{C}\right)
\end{align*}
where the fourth equality uses independence. Then it should be clear
that $A^{C}$ and $B$ are independent exactly for the same reason.
Finally, we can apply the logic above using $A^{C}$and $B$ instead
of $A$ and $B$ to get that $A^{C}$ and $B^{C}$ are independent.
\end{proof}
\begin{prop}
\label{prop:independenceFacts}Let $\probabilityspace$ be a probability
space and let $I$ be an arbitrary index set. Let $\left\{ A_{i}\right\} _{i\in I}\in\F$
be a collection of events and define $B_{i}^{1}=A_{i}^{C}$ and $B_{i}^{0}=A_{i}.$
Then, the following statements are equivalent

\begin{enumerate}[label=(\roman*),leftmargin=.1\linewidth,rightmargin=.4\linewidth]
\item $\left\{ A_{i}\right\} _{i\in I}\in\F$ are independent events.
\item There exists some $\alpha \in \{0,1\}^I$ such that $\{B_i^{a(i)}\}$ are independent.
\item For every $\alpha \in \{0,1\}^I$, $\{B_i^{a(i)}\}$ are independent. 
\end{enumerate}
\end{prop}

\begin{proof}
(Sketch) Note that if \emph{(i) }holds, then \emph{(ii)} holds automatically
with $a\left(i\right)=0$ for all $i\in I$. Similarly, if $(iii)$
holds then $(i)$ holds trivially. Thus we need to prove that $(ii)\implies(iii)$.
First, fix $\alpha\in\{0,1\}^{I}$ such that our claim holds. Notice
that by Lemma \ref{lem:independenceComplements}, for any $J\subset I$
such that $\lvert J\rvert=2$,
\[
\mathbb{P}\left(\prod_{j\in J}B_{j}^{\gamma(j)}\right)=\prod_{j\in J}\mathbb{P}\left(B_{j}^{\gamma\left(j\right)}\right)
\]
for any $\gamma\in\left\{ 0,1\right\} ^{I}$. For induction, suppose
that the claim holds for any $J\subset I$ such that $\lvert J\rvert=n$
and the consider a subset $J^{\prime}\subset I$ with $\lvert J^{\prime}\rvert=n+1$.
Note that for any $i\in J^{\prime}$, we can define $B_{-i}:=\bigcap_{j\in J^{\prime}\setminus\left\{ i\right\} }B_{j}^{\alpha\left(j\right)}$
and observe that 
\[
\mathbb{P}\left(B_{-i}\cap B_{i}^{\alpha\left(i\right)}\right)=\mathbb{P}\left(B_{-i}\right)\mathbb{P}\left(B_{i}^{\alpha\left(i\right)}\right)
\]
and so $B_{-i}\indep B_{i}^{\alpha\left(i\right)}$ and so by our
Lemma, $B_{-i}\indep B_{i}^{\gamma\left(i\right)}.$ The induction
hypothesis then implies that $\left\{ B_{j}^{\alpha\left(j\right)}\right\} ,B_{i}^{\gamma\left(i\right)}$
are all mutually independent. We can yet again repeat this process
with some $i^{\prime}\in J^{\prime}$ where now $B_{-i^{\prime}}:=\bigcap_{j\in J^{\prime}\setminus\left\{ i,i^{\prime}\right\} }B_{j}^{\alpha\left(j\right)}\cap B_{i}^{\gamma\left(i\right)}$
and so on until we have replaced all $\alpha$s with $\gamma$s.
\end{proof}
This idea of mutual independence finds purchase in some unexpected
contexts as well. For instance, we can use it to prove Euler's prime
number formula.
\begin{thm}
\label{thm:eulerPrime}Let $\mathcal{P}$ denote the set of primes.
The Riemann zeta function 
\[
\zeta\left(s\right):=\sum_{n=1}^{\infty}n^{-s}
\]
has the representation
\[
\zeta\left(s\right)=\prod_{p\in\mathcal{P}}\left(1-p^{-s}\right)^{-1}.
\]
\end{thm}

\begin{proof}
Let $\probabilityspace=\left(\N,2^{\N},\mathbb{P}_{s}\right)$ where
$\mathbb{P}_{s}\left(\left\{ n\right\} \right)=\frac{n^{-s}}{\zeta\left(s\right)}$
for $s>1.$ Define $p\N:=\left\{ pn\mid n\in\N\right\} $ and let
$ $for any $p\in\mathcal{P}$ and notice that $\mathbb{P}_{s}\left(p\N\right)=\sum_{n=1}^{\infty}\mathbb{P}\left(pn\right)=p^{-s}$
by countable additivity. Then, for any distinct collection $p_{1},\ldots,p_{k}\in\mathcal{P}$
we have that 
\begin{align*}
\mathbb{P}_{s}\left(\bigcap_{i=1}^{k}p_{i}\N\right) & =\sum_{n=1}^{\infty}\mathbb{P}_{s}\left(\prod_{i=1}^{k}pn\right)\\
 & =\prod_{i=1}^{k}p^{-s}\\
 & =\prod_{i=1}^{k}\mathbb{P}_{s}\left(p_{i}\N\right).
\end{align*}
In other words, the events $\left\{ p\N\right\} _{p\in\mathcal{P}}$are
mutually independent and so by Proposition \ref{prop:independenceFacts}
\begin{align*}
\frac{1}{\zeta\left(s\right)} & =\mathbb{P}_{s}\left(\left\{ 1\right\} \right)\\
 & =\mathbb{P}_{s}\left(\bigcap_{p\in\mathcal{P}}\left(p\N\right)^{C}\right)\\
 & =\lim_{n\to\infty}\mathbb{P}_{s}\left(\bigcap_{p\in\mathcal{P},p\leq n}\left(p\N\right)^{C}\right)\\
 & =\lim_{n\to\infty}\prod_{p\in\mathcal{P},p\leq n}\left(1-\mathbb{P}_{s}\left(p\N\right)\right)\\
 & =\lim_{n\to\infty}\prod_{p\in\mathcal{P},p\leq n}\left(1-p^{-s}\right)\\
 & =\prod_{p\in\mathcal{P}}\left(1-p^{-s}\right)
\end{align*}
where the second equality is due to the fact that $1$ is neither
prime nor a product of primes, the third due to the continuity of
measures (see Propositions \ref{prop:measureProperties} and \ref{prop:equivalenceContinuityMeasures}),
and the fourth due the fact that $\mathbb{P}_{s}\left(\N\right)=1.$
\end{proof}
We are finally ready to present the second Borel-Cantelli lemma, which
we promised back in Chapter \ref{chap:measurableFunctions}. The idea
here is that if you an infinite sequence of independent events, like
coin-tosses where the outcomes can either be head or tails, the probability
that you will only see a finite number of heds is intuitively zero.
\begin{thm}[Second Borel-Cantelli lemma]
\label{thm:secondBorelCantelli}Let $\probabilityspace$ be a probability
space and let $\left\{ A_{i}\right\} _{i\in\N}\in\mathcal{F}$ be
mutually independent events. If
\[
\sum_{i=1}^{\infty}\mathbb{P}\left(A_{i}\right)=\infty
\]
then 
\[
\mathbb{P}\left(\limsup_{i\to\infty}A_{i}\right)=1.
\]
\end{thm}

\begin{proof}
Note that 
\begin{align*}
\mathbb{P}\left(\left(\limsup A_{i}\right)^{C}\right) & =\mathbb{P}\left(\left(\bigcap_{n\in\N}\bigcup_{i\geq n}A_{i}\right)^{C}\right)\\
 & =\mathbb{P}\left(\bigcup_{n\in\N}\bigcap_{i\geq n}A_{i}^{C}\right)\\
 & =\lim_{n\to\infty}\mathbb{P}\left(\bigcap_{i\geq n}A_{i}^{C}\right)\\
 & =\lim_{n\to\infty}\lim_{m\to\infty}\mathbb{P}\left(\bigcap_{m\geq i\geq n}A_{i}^{C}\right)\\
 & =\lim_{n\to\infty}\lim_{m\to\infty}\prod_{m\geq i\geq n}\left(1-\mathbb{P}\left(A_{i}\right)\right)\\
 & =\lim_{n\to\infty}\lim_{m\to\infty}\exp\left(\sum_{m\geq i\geq n}\log\left(1-\mathbb{P}\left(A_{i}\right)\right)\right)\\
 & \leq\lim_{n\to\infty}\lim_{m\to\infty}\exp\left(-\sum_{m\geq i\geq n}\mathbb{P}\left(A_{i}\right)\right)\\
 & =\lim_{n\to\infty}\exp\left(-\sum_{i\geq n}\mathbb{P}\left(A_{i}\right)\right)\\
 & =0
\end{align*}
where in the second equality we have used DeMorgan's laws, the third
and fourth follow from the continuity of measures, the fifth by independence
and the fact that $\mathbb{P}\left(\Omega\right)=1$, and the inequality
by the fact that $\log\left(1-x\right)\leq-x$ for $x\in\left[0,1\right).$\footnote{This can be verified by noting that the function $g\left(x\right)=\log\left(1-x\right)+x$
is $0$ at $x=0$ and has derivative $g^{\prime}\left(x\right)=1-\frac{1}{1-x}=\frac{-x}{1-x}<0$
for $x\in\left(0,1\right).$}
\end{proof}
%
Note that we can't dispense with the independence assumption on this
version of the Borel Cantelli lemma, since for identical events $A_{i}$,
$\limsup A_{i}=A_{1}$ and so if $\mathbb{P}\left(A_{1}\right)<1$
then $\mathbb{P}\left(\limsup A_{i}\right)<1$ even though $\sum\mathbb{P}\left(A_{i}\right)=\infty$.

\hl{Differences between mutual and pariwise independence}

\section{Independent $\sigma-$algebras}

The idea of independence of events can be extended to the idea of
independence of structures of events. For a given probability space
$\probabilityspace$, we say that a collection of events $\left\{ \F_{i}\right\} _{i\in I}$
-- where $I$ is an arbitrary index set and each $\F_{i}\subset\F$
-- is mutually independent if for any finite $J\subset I$ and $F_{j}\in\F_{j}$
\[
\mathbb{P}\left(\bigcap_{j\in J}F_{j}\right)=\prod_{j\in J}\mathbb{P}\left(F_{j}\right).
\]
Usually, the structures we consider are $\sigma-$algebras, since
those are the most important types of collections of events in probability
theory. Immediately, our \hyperref[cor:piLambdaGeneratingClassArg]{$\pi-\lambda$ theorem}
gives us some useful tools that allows us to use independence on generating
classes to prove independence on the $\sigma-$algebras generated
by them.
\begin{prop}
\label{prop:indepSigmaAlgebras}Let $\probabilityspace$ be a probability
space and let $I$ be an arbitrary index set. If a collection of $\pi-$systems
$\left\{ \mathcal{E}_{i}\right\} _{i\in I}\subset\F$ are independent,
then $\left\{ \sigma\left(\mathcal{E}_{i}\right)\right\} _{i\in I}$
are independent.
\end{prop}

\begin{proof}
Pick some $J\subset I$ such that $\lvert J\rvert=n$. Without loss
of generality, we can assume that $J=\left\{ 1,2,\ldots,n\right\} .$
Define
\[
\mathcal{D}_{1}:=\left\{ F_{1}\in\sigma\left(\mathcal{E}_{1}\right)\mid\mathbb{P}\left(F_{1}\cap E_{2}\cap\ldots\cap E_{n}\right)=\mathbb{P}\left(F_{1}\right)\mathbb{P}\left(E_{2}\right)\ldots\mathbb{P}\left(E_{n}\right)\forall E_{i}\in\mathcal{E}_{i},2\leq i\leq n\right\} 
\]
and note that we can show that $\mathcal{D}_{1}$ is a $\lambda-$system.
To see this, first note that $\Omega\in\mathcal{D}_{1}$ since 
\begin{align*}
\mathbb{P}\left(\Omega\cap E_{2}\cap\ldots\cap E_{n}\right) & =\mathbb{P}\left(E_{2}\cap\ldots\cap E_{n}\right)\\
 & =\mathbb{P}\left(E_{2}\right)\ldots\mathbb{P}\left(E_{n}\right)\\
 & =\mathbb{P}\left(\Omega\right)\mathbb{P}\left(E_{2}\right)\ldots\mathbb{P}\left(E_{n}\right)
\end{align*}
where in the second equality we use the independence of the generators.
Next, suppose that $F_{1},F_{2}\in\mathcal{D}_{1}$ such that $F_{2}\subseteq F_{1}$
and observe that 
\begin{align*}
\mathbb{P}\left(\left(F_{1}\setminus F_{2}\right)\cap E_{2}\cap\ldots\cap E_{n}\right) & =\mathbb{P}\left(F_{1}\cap E_{2}\cap\ldots\cap E_{n}\right)-\mathbb{P}\left(F_{2}\cap E_{2}\cap\ldots\cap E_{n}\right)\\
 & =\mathbb{P}\left(F_{1}\right)\mathbb{P}\left(E_{2}\right)\ldots\mathbb{P}\left(E_{n}\right)-\mathbb{P}\left(F_{2}\right)\mathbb{P}\left(E_{2}\right)\ldots\mathbb{P}\left(E_{n}\right)\\
 & =\left(\mathbb{P}\left(F_{1}\right)-\mathbb{P}\left(F_{2}\right)\right)\mathbb{P}\left(E_{2}\right)\ldots\mathbb{P}\left(E_{n}\right)\\
 & =\mathbb{P}\left(F_{1}\setminus F_{2}\right)\mathbb{P}\left(E_{2}\right)\ldots\mathbb{P}\left(E_{n}\right)
\end{align*}
which shows that $F_{1}\setminus F_{2}\in\mathcal{D}_{1}$. Finally,
suppose that $\left\{ F_{i}\right\} _{i\in\N}\in\mathcal{D}_{1}$
such that $F_{i}\subset F_{i+1}$ and note that 
\begin{align*}
\mathbb{P}\left(\left(\bigcup_{i\in\N}F_{i}\right)\cap E_{2}\cap\ldots\cap E_{n}\right) & =\lim_{n\to\infty}\mathbb{P}\left(F_{n}\cap E_{2}\cap\ldots\cap E_{n}\right)\\
 & =\lim_{n\to\infty}\mathbb{P}\left(F_{n}\right)\mathbb{P}\left(E_{2}\right)\ldots\mathbb{P}\left(E_{n}\right)\\
 & =\mathbb{P}\left(\bigcup_{i\in\N}F_{i}\right)\mathbb{P}\left(E_{2}\right)\ldots\mathbb{P}\left(E_{n}\right)
\end{align*}
which shows that $\bigcup_{i\in\N}F_{i}\in\mathcal{D}_{1}$. Thus
$\mathcal{D}_{1}$ is a $\lambda-$system containing $\mathcal{E}_{1}$
and so $\sigma\left(\mathcal{E}_{1}\right)\subseteq\mathcal{D}_{1}$.
Next, we can define 
\[
\mathcal{D}_{2}:=\left\{ F_{2}\in\sigma\left(\mathcal{E}_{2}\right)\mid\mathbb{P}\left(F_{1}\cap F_{2}\cap\ldots\cap E_{n}\right)=\mathbb{P}\left(F_{1}\right)\mathbb{P}\left(F_{2}\right)\ldots\mathbb{P}\left(E_{n}\right)F_{1}\in\sigma\left(\mathcal{E}_{1}\right),E_{i}\in\mathcal{E}_{i},3\leq i\leq n\right\} ,
\]
observe that $\mathcal{E}_{2}\subseteq\mathcal{D}_{2}$ by the result
on $\mathcal{D}_{1}$ , and show that $\mathcal{D}_{2}$ is a $\lambda-$system
and so $\sigma\left(\mathcal{E}_{2}\right)\subseteq\mathcal{D}_{2}$
and so on. This completes the argument.
\end{proof}
\begin{cor}
\label{cor:clumping}Let $\left\{ \mathcal{F}_{i}\right\} _{i\in I}$
be a collection of independent $\sigma-$algebras. If $J,K\subset I$
are disjoint, then 
\[
\sigma\left(\bigcup_{j\in J}\F_{j}\right)\indep\sigma\left(\bigcup_{k\in K}\F_{k}\right).
\]
\end{cor}

\begin{proof}
Let $\mathcal{E}_{J}$ be the collection of all finite intersections
of sets in $\bigcup_{j\in J}\F_{j}$ and note that $\bigcup_{j\in J}\F_{j}\subset\mathcal{E}_{J}$
for any $j\in J$ since we can take all sets one at a time. Thus $\sigma\left(\bigcup_{j\in J}\F_{j}\right)\subseteq\sigma\left(\mathcal{E}_{J}\right)$.
Conversely, $\mathcal{E}_{J}\subseteq\sigma\left(\bigcup_{j\in J}\F_{j}\right)$
and so $\sigma\left(\bigcup_{j\in J}\F_{j}\right)=\sigma\left(\mathcal{E}_{J}\right).$
Note that $\mathcal{E}_{J}$ and $\mathcal{E}_{K}$ (where $\mathcal{E}_{K}$
is defined analogously) are $\pi-$systems and so if $\mathcal{E}_{J}\indep\mathcal{E}_{K}$
then $\sigma\left(\mathcal{E}_{J}\right)\indep\sigma\left(\mathcal{E}_{K}\right)$
by Proposition \ref{prop:indepSigmaAlgebras}. Now for any $E_{J}\in\mathcal{E}_{J}$
, we can write $E_{J}=\bigcap_{j\in J}F_{j}$ where $F_{j}\in\F_{j}$
(where for some $j$s $F_{j}=\Omega$). Similarly $E_{K}\in\mathcal{E}_{K}$
could be written $E_{K}=\bigcap_{k\in K}F_{k}$. Therefore,
\begin{align*}
\mathbb{P}\left(E_{J}\cap E_{K}\right) & =\mathbb{P}\left(\bigcap_{j\in J}F_{j}\cap\bigcap_{k\in K}F_{k}\right)\\
 & =\mathbb{P}\left(\bigcap_{j\in J}F_{j}\right)\mathbb{P}\left(\bigcap_{k\in K}F_{k}\right)
\end{align*}
where we have used the independence of $\left\{ \F_{i}\right\} _{i\in I}.$
\end{proof}
These results yield as our first \emph{zero-one law }which is a result
that characterizes some type of $\sigma-$algebra that only contain
events iwth probability zero or one.
\begin{defn}
\label{def:tailSigmaAlgebra}Let $\probabilityspace$ be a probability
space and suppose $\left\{ \F_{i}\right\} _{i\in\N}\subset\F$ is
a collection of sub $\sigma-$algebras on this space. Suppose further
that $\F=\sigma\left(\bigcup_{i\in I}\F_{i}\right)$ and $\mathcal{H}_{n}:=\sigma\left(\bigcup_{i\geq n}\F_{i}\right)$
and so $\mathcal{H}_{n+1}\subseteq\mathcal{H}_{n}$. The \emph{tail
$\sigma-$algebra }with respect to $\left\{ \F_{i}\right\} _{i\in\N}$
is given
\[
\mathcal{H}_{\infty}=\bigcap_{n\in\N}\mathcal{H}_{n}.
\]
\end{defn}

\begin{thm}[Kolmogorov's Zero-One Law]
\label{thm:kolmogorovZeroOne}Let $\probabilityspace$ be a probability
space and suppose $\left\{ \F_{i}\right\} _{i\in\N}\subset\F$ is
a collection of mutually independent sub $\sigma-$algebras on this
space. Then the tail $\sigma-$algebra $\mathcal{H}_{\infty}$ is
trivial in that for any $A\in\mathcal{H}_{\infty}$
\[
\mathbb{P}\left(A\right)\in\left\{ 0,1\right\} .
\]
\end{thm}

\begin{proof}
Note that $\F_{1},\ldots,\F_{n-1}$ and $\mathcal{H}_{n}$ are mutually
independent by an inductive extension of Corollary \ref{cor:clumping}.
Of course, since $\mathcal{H}_{\infty}\subseteq\mathcal{H}_{n}$ so
$\F_{1},\ldots,\F_{n}$ and $\mathcal{H}_{\infty}$ are all mutually
independent. Of course, since $n$ is arbitrary, we have that $\mathcal{H}_{\infty}$
and $\left\{ \F_{i}\right\} _{i\in\N}$are all mutually independent.
Again, applying Corollary \ref{cor:clumping}, we have that $\mathcal{H}_{\infty}\indep\sigma\left(\bigcup_{i\in\N}\F_{i}\right).$Note
that since $\mathcal{H}_{\infty}\subset\F=\sigma\left(\bigcup_{i\in\N}\F_{i}\right)$
we have that $\mathcal{H}_{\infty}\indep\mathcal{H}_{\infty}$and
so for any $A\in\mathcal{H}_{\infty}$
\[
\mathbb{P}\left(A\right)=\mathbb{P}\left(A\right)^{2}\implies\mathbb{P}\left(A\right)\in\left\{ 0,1\right\} .
\]
\end{proof}

\section{Random variables}

\subsection{General description and terminology}
\begin{defn}
\label{def:randomVariable}Let $\probabilityspace$ be a probability
space. A \emph{random variable }is a real-valued Borel-measurable
map $X:\left(\Omega,\F\right)\to\left(\R,\borel\left(\R\right)\right)$.
A \emph{random vector }is a measurable map $X:\left(\Omega,\F\right)\to\left(\R^{n},\borel\left(\R^{n}\right)\right)$.
A \emph{random element }is a measurable map $X:\left(\Omega,\F\right)\to\left(S,\mathcal{B}\right)$
where $\left(S,\mathcal{B}\right)$ is a Polish space with its Borel
$\sigma-$algebra.
\end{defn}

Our definitions are increasingly general in that random variables,
are random vectors, which in turn are random elements. Much of our
language about measurable functions has analogues in probability.
For instance, the image measure $X\mathbb{P}$ (which we shall denote
as $\mathbb{P}_{X}$ from now on) of $X$ under $\mathbb{P}$ is called
the \emph{distribution }of $X$. The Radon-Nikodym derivative of this
distribution with respect to the Lebesgue measure (if it exists) is
called the \emph{density }of $X$, which is always non-negative (like
all RN-derivatives) and integrates to $1$. The Stieljes function
$F_{X}\left(x\right):=\mathbb{P}_{X}\left(\left(-\infty,x\right]\right)$
is called the \emph{cumulative distribution function (CDF) }of $X$.
Theorem \ref{thm:stieljesMeasure} tells us that CDFs completely characterize
the distribution of a random variable $X$ and so random variables
with the same CDF are called \emph{identically distributted. }Moreover,
Proposition \ref{prop:absoluteContinuityStieljesFunctions} tells
us that the absolute continuity of CDFs (as functions on the real
line) is equivalent to the absolute continuity of the distribution
with respect to the Lebesgue measure. Of course, not all random variables
have absolutely continuous CDFs. The canonical pathological example
of a random variable which has a continuous but not absolutely continuous
CDF is the Cantor random variable, whose CDF is the \hyperref[def:cantorFunction]{Cantor function}.
More importantly, many random variables have countable supports\footnote{This is again a shorthand! What we mean here is that the CDF has a
countable support as a real valued function on $\R$.} and can be thought of as \emph{discrete. }A reasonably comprehensive
accounting of various distributions that we can find in the wild can
be found in Appendix \ref{chap:probabilityDistributions}.

The CDF has two additional properties that we did not see for Stieljes
functions in general.
\begin{prop}
\label{prop:cdfLimits}Let $\probabilityspace$ be a probability space
and let $X:\Omega\to\R$ be a random variable. Then its cumulative
distribution function $F_{X}$ has the property that 
\[
\lim_{x\to\infty}F_{X}\left(x\right)=1
\]
and 
\[
\lim_{x\to-\infty}F_{X}\left(x\right)=0.
\]
\end{prop}

\begin{proof}
Note that the first limit is equivalent to
\begin{align*}
\lim_{n\to\infty}\mathbb{P}_{X}\left(\left(-\infty,n\right]\right) & =\mathbb{P}_{X}\left(\bigcup_{n\in\N}\left(-\infty,n\right]\right)\\
 & =\mathbb{P}_{X}\left(\left(-\infty,\infty\right)\right)\\
 & =1.
\end{align*}
where we used the upper continuity of measures. Similarly, the other
limit can be written as 
\begin{align*}
\lim_{n\to\infty}\mathbb{P}_{X}\left(\left(-\infty,-n\right]\right) & =\mathbb{P}_{X}\left(\bigcap_{n\in\N}\left(-\infty,-n\right]\right)\\
 & =\mathbb{P}_{X}\left(\emptyset\right)\\
 & =0.
\end{align*}
\end{proof}
\begin{example}
\label{exa:floorFuncCDF}Let $n$ be a positive integer and suppose
that 
\[
F\left(x\right)=\frac{\lfloor x\rfloor}{n}\indicate\left\{ 0\leq x\leq n\right\} +\indicate\left\{ x>n\right\} 
\]
 and observe that $F$ is clearly non-decreasing, it is right continuous
as the floor function is right continuous. The limiting behavior is
obvious. To construct the probability distribution that generates
this CDF, one can define a random variable $X$ such that 
\begin{align*}
\mathbb{P}\left(X=x\right) & =F\left(x\right)-F\left(x-1\right)\\
 & =\frac{1}{n}
\end{align*}
which is the \emph{discrete uniform distribution} over $\left\{ 1,2,\ldots,n\right\} $.
\end{example}

\begin{example}
\label{exa:geometricDistOneHalf}Let $\probabilityspace$ be a probability
space and let $X$ be a random variable with distribution characterized
by
\[
\mathbb{P}_{X}\left(\left\{ x\right\} \right)=\left(\frac{1}{2}\right)^{x}\indicate\left\{ x\in\N\right\} .
\]
Since 
\[
\mathbb{P}_{X}\left(\N\right)=\sum_{i=1}^{\infty}\frac{1}{2^{i}}=1,
\]
 for this to be a valid distribution, it must be that $\mathbb{P}_{X}\left(\R\setminus\N\right)=0.$
What is the CDF of such a distribution? Well, we know that 
\begin{align*}
F_{X}\left(x\right) & =\mathbb{P}_{X}\left(\left(-\infty,x\right]\right)\\
 & =\mathbb{P}_{X}\left(\left(-\infty,x\right]\cap\N\right)\\
 & =\sum_{i=1}^{\lfloor x\rfloor}\frac{1}{2^{i}}\\
 & =1-\frac{1}{2^{\lfloor x\rfloor}}.
\end{align*}
\end{example}

The convex combination of CDFs remains a CDF , which gives rise to
mixture distributions.
\begin{prop}
\label{prop:convexCombinationDistribution}Let $\probabilityspace$
be a probability space and let $X,Y$ be random variables with CDFs
$F_{X}$ and $F_{Y}$. For any $\lambda\in\left[0,1\right]$ , the
function $F\left(x\right):=\lambda F_{X}\left(x\right)+\left(1-\lambda\right)F_{Y}\left(x\right)$
satisfies all the properties of CDF.
\end{prop}

\begin{proof}
Note that convex combinations of increasing functions is increasing
and the convex combinations of right continuous functions is right
continuous. The limiting properties also follow easily since limits
behave linearly.
\end{proof}
%
\begin{example}[ISI 2023 PSB 3]
\label{exa:isi2023psb3}Do there exist CDFs on $\R$ such that $F\left(x\right)=F\left(x^{n}\right)$
where $n>1$? First, note that if $n$is even, then for any $x<0$
$F\left(x\right)=F\left(x^{n}\right)$ which implies that 
\begin{align*}
0 & =\lim_{x\to-\infty}F\left(x\right)\\
 & =\lim_{x\to-\infty}F\left(x^{n}\right)\\
 & =\lim_{x\to\infty}F\left(x^{n}\right)\\
 & =1
\end{align*}
which is a contradiction and so no such distribution exists. If $n$
is odd then, we have that for any $x,y>1$, we $F\left(x\right)=F\left(y\right)$
since if $x<y$ there exists some $k\in\N$ such $x^{n^{k}}>y$ and
so the non-decreasing nature of $F$ yields the result. Of course,
since $\lim_{x\to\infty}F\left(x\right)=1$, we must have that $F\left(x\right)=1$
for every $x>1$. A similar result shows that $F\left(x\right)=F\left(y\right)=0$
for $x<-1$. For the intermediate values of $x$, we know that for
any $x_{1},x_{2}\in\left[-1,0\right)$ and $y_{1},y_{2}\in\left(0,1\right]$
that $0\leq F\left(x_{1}\right)=F\left(x_{2}\right)\leq F\left(0\right)\leq F\left(y_{1}\right)=F\left(y_{2}\right)\leq1$.
\end{example}

\begin{example}
\label{exa:isi2008samplepsb6}Let $F$ and $G$ be (one dimensional)
distribution functions. Decide which of the following are distribution
functions. (a) $F^{2}$, (b) $H$ where $H(t)=\max\{F(t),G(t)\}$.

Justify your answer.
\end{example}

\begin{example}
	\label{exa:isi2009samplepsb5}
	Suppose $F$ and $G$ are continuous and strictly increasing distribution functions. Let $X$ have distribution function $F$ and $Y=G^{-1} F(X)$.
	(a) Find the distribution function of $Y$.
	(b) Hence, or otherwise, show that the joint distribution function of $(X, Y)$, denoted by $H(x, y)$, is given by $H(x, y)=\min (F(x), G(y))$.\hl{TODO}
\end{example}

The integral of a random variable is called an \emph{expectation }(provided
it exists). That is, $\lebInt{\mathbb{P}}X$ is the expectation of
$X$. In the probability literature, this is often denoted $\mathbb{E}\left[X\right]$.
Since random variables are defined on measures spaces with unit measure,
the integral captures the average value of the function $X$. If $X\in\mathcal{L}^{p}$
for $p>1$ then the random variable is said to have finite $p$th
moment. Of course, by Proposition \ref{prop:nestingLpSpace} we know
that if a random variable has a finite $p$th moment, it has a finite
$q$th moment for $1\leq q\leq p$. Some moments have special names.
So, for instance, for the centered random variable $Y:=X-\mathbb{E}\left[X\right]$,
the second moment $\text{\ensuremath{\Var\left[X\right]}:=}\mathbb{E}\left[Y^{2}\right]=\E\left[X^{2}\right]-\left(\E\left[X\right]\right)^{2}$
is called the \emph{variance} of $X$. It is often denoted as $\sigma^{2}$
and it captures the degree to which a random variable deviates from
its mean value. The third \emph{standardized }moment $\mathbb{E}\left[\left(\frac{Y}{\sigma}\right)^{3}\right]$
is called the \emph{skewness; }it captures the degree to which a random
variables distribution is asymmetric. Here $\sigma$ is the square
root of the variance, and is referred to as the \emph{standard deviation
}of $X$. The fourth standardized moment $\E\left[\left(\frac{Y}{\sigma}\right)^{4}\right]$
is called the kurtosis. Of course, random variables need not even
by $\mathcal{L}^{1}$, as the following two examples illustrate. A
more detailed discussion on moments can be found in the Appendix on
common probability distributions 
\begin{example}
\label{exa:cauchyDistribution}Let $\probabilityspace$ be a probabiltiy
space and let $X:\Omega\to\R$ be a random variable whose distribution
is absolutely continuous with respect to the Lebesgue measure with
density $f_{X}\left(x\right)=\frac{1}{\pi\left(1+x^{2}\right)}$.
Note that this is a valid density since 
\begin{align*}
\lebInt{\lambda}{\frac{1}{\pi\left(1+x^{2}\right)}} & =\int_{-\infty}^{\infty}\frac{1}{\pi\left(1+x^{2}\right)}\\
 & =\int_{-\infty}^{0}\frac{1}{\pi\left(1+x^{2}\right)}+\int_{0}^{\infty}\frac{1}{\pi\left(1+x^{2}\right)}\\
 & =\frac{2}{\pi}\int_{0}^{\infty}\frac{1}{\left(1+x^{2}\right)}\\
 & =\frac{2}{\pi}\lim_{b\to\infty}\tan^{-1}\left(b\right)-\tan^{-1}\left(0\right)\\
 & =\frac{2}{\pi}\frac{\pi}{2}\\
 & =1
\end{align*}
where in the third equality we used the the symmetry of the density
(and hence distribution)\footnote{The symmetry of the distribution of a random variable is an important
property that characterizes many of the common probability distributions
we hear about in probability and statistics. For instance, the normal
distribution, and the uniform distribution, are both symmetric distributions.
A probability distribution of a random variable $X$ is said to be
symmetric about $0$ if for any $x\in\R$$:F_{X}(x)=1-F_{X}\left(-x\right).$
One can show that the skewness of such a random variable is zero by } and in the fourth the fact that the derivative of $\tan^{-1}\left(x\right)$
is $\frac{1}{1+x^{2}}$ and the fundamental theorem of calculus. To
see that the first moment doesn't exist, observe that 
\begin{align*}
\mathbb{E}\left[X\right] & =\E\left[X^{+}\right]-\E\left[X^{-}\right]\\
 & =\int_{0}^{\infty}\frac{x}{\pi\left(1+x^{2}\right)}dx-\int_{0}^{\infty}\frac{x}{\pi\left(1+x^{2}\right)}dx\\
\end{align*}
where in the second equality we have used Corollaries \ref{cor:changeOfVariables}
and \ref{cor:densityIntegral} along with the Radon-Nikodym theorem.
Thus the expectation is zero if the integral of the parts is finite;
otherwise it is not defined. To this end, note that 
\begin{align*}
\int_{0}^{\infty}\frac{x}{\pi\left(1+x^{2}\right)}dx & =\frac{1}{2\pi}\int_{1}^{\infty}\frac{1}{u}du\\
 & =\frac{1}{2\pi}\left[\lim_{b\to\infty}\log\left(b\right)-\log\left(1\right)\right]\\
 & =\infty
\end{align*}
which shows that $\E\left[X\right]$ is not defined.
\end{example}

\begin{example}
\label{exa:isi2008samplepsb4}Consider the unit interval $(0,1)$
that is divided into two sub-intervals by picking a \hyperref[subsec:uniformDistribution]{point at random}
from inside the interval. Denoting by $Y$ and $Z$ the lengths of
the longer and the shorter sub-intervals respectively, we can show
that $Y/Z$ does not have finite expectation. Indeed, letting $X$
be the uniformly distributed random variable in $\left(0,1\right)$,
we have that 
\begin{align*}
\E\left[\frac{Y}{Z}\right] & =\E\left[\frac{1-X}{X}\indicate\left\{ 0\leq X\leq0.5\right\} +\frac{X}{1-X}\indicate\left\{ 0.5\leq X\leq1\right\} \right]\\
 & =\E\left[\frac{1-X}{X}\indicate\left\{ 0\leq X\leq0.5\right\} \right]+\E\left[\frac{X}{1-X}\indicate\left\{ 0.5\leq X\leq1\right\} \right]\\
 & =\int_{0}^{0.5}\frac{1-x}{x}dx+\int_{0.5}^{1}\frac{x}{1-x}dx\\
\end{align*}
where neither integral is finite.
\end{example}

The following is an example of a discrete probability distribution
where the expectation does exist.
\begin{example}
\label{exa:isi2007samplepsb6}18 boys and 2 girls are made to stand
in a line in a random order. Let $X$ be the number of boys standing
in between the girls. Whati s $P(X=k)$ and $\E(X)$? Appendix \ref{chap:Combinatorics}
will be useful here. Note that if there are $b$ boys and 2 girls,
there are $\left(b+2\right)!$ ways to arrange them and $b+2-k-1$
ways to sandwich $k$ boys between two girls, with $b!$ ways to arrange
the boys and $g!$ ways to arrange the girls. Therefore
\begin{align*}
\P\left(X=k\right) & =\frac{\left(b+2-k-1\right)b!2!}{\left(b+2\right)!}\\
 & =\frac{b+2-k-1}{\left(b+2\right)!}
\end{align*}
 To find the expectation, note that 
\begin{align*}
\E\left[X\right] & =\frac{2}{\left(b+2\right)\left(b+1\right)}\sum_{k=0}^{b}k\left(b+2-k-1\right)\\
 & =\frac{2}{\left(b+2\right)\left(b+1\right)}\sum_{k=1}^{b}k\left(b+1-k\right)\\
 & =\frac{2}{\left(b+2\right)\left(b+1\right)}\left(\begin{array}{c}
b+2\\
3
\end{array}\right)\\
 & =\frac{2}{\left(b+2\right)\left(b+1\right)}\frac{b\left(b+1\right)\left(b+2\right)}{6}\\
 & =\frac{b}{3}
\end{align*}
where in the third equality we used Lemma \ref{prop:mdmActivity103}.
\end{example}

We can characterize \emph{joint }moments of multiple random variables
just as easily. For instance, the \emph{covariance }of two random
variables $X$ and $Y$ on the same probability space is simply their
centered inner product $\Cov\left[X,Y\right]:=\mathbb{E}\left[\left(X-\mathbb{E}\left[X\right]\right)\left(Y-\mathbb{E}\left[Y\right]\right)\right]=\E\left[XY\right]-\E\left[X\right]\E\left[Y\right]$.
Notice that this is a generalization of the variance which is essentially
the covariance of a random variable with itself. The bilinearity of
inner products lends itself very nicely to covariances.
\begin{example}
\label{exa:isi2015psb5ptA}Let $\probabilityspace$ be a probability
space and let $X$ and $Y$ be $\mathcal{L}^{2}$ random variables
such that $\Var\left[X+Y\right]=3,$ and $\Var\left[X-Y\right]=1$.
What is $\Cov\left[X,Y\right]$? We use the bilinearity of covariances
here. Note that $\Var\left[X+Y\right]=\Var\left[X\right]+\Var\left[Y\right]+2\Cov\left[X,Y\right]$
and $\Var\left[X-Y\right]=\Var\left[X\right]+\Var\left[Y\right]-2\Cov\left[X,Y\right]$.
Subtracting the two equations yields $\Cov\left[X,Y\right]=\frac{1}{2}$.
\end{example}

\begin{example}
\label{exa:isi2006samplepsb6}Let $Y_{1},Y_{2},Y_{3}$ be i.i.d. continuous
random variables. For $i=1,2$, define $U_{i}$ as 
\[
U_{i}=\begin{cases}
1 & \text{ if }Y_{i+1}>Y_{i},\\
0 & \text{ otherwise }
\end{cases}
\]

What is the mean and variance of $U_{1}+U_{2}$? The key here is that
the random variables are continuous and so $\P\left(Y_{i}=Y_{i+1}\right)=0$.
Since then $\P\left(Y_{i}>Y_{i+1}\right)=\P\left(Y_{i}<Y_{i+1}\right)=\frac{1}{2}$,
we have that $\E\left[U_{1}+U_{2}\right]=\P\left(Y_{2}>Y_{1}\right)+\P\left(Y_{3}>Y_{2}\right)=1$.
For the variance, note that 
\begin{align*}
\Var\left[U_{1}+U_{2}\right] & =\Var\left[U_{1}\right]+\Var\left[U_{2}\right]+2\Cov\left[U_{1},U_{2}\right]\\
 & =2\left(\E\left[U_{1}^{2}\right]-\E\left[U_{1}\right]^{2}\right)+2\left(\E\left[U_{1}U_{2}\right]-\E\left[U_{1}\right]\E\left[U_{2}\right]\right)\\
 & =2\left(\frac{1}{2}-\frac{1}{4}\right)+2\left(\frac{1}{6}-\frac{1}{4}\right)\\
 & =\frac{1}{2}-\frac{1}{6}\\
 & =\frac{1}{3}.
\end{align*}
where in the second equality we have used the fact that $U_{1}$ and
$U_{2}$ are identically distributed and in the third the fact that
$\P\left(Y_{3}>Y_{2}>Y_{1}\right)=\frac{1}{6}$ since there are $3!$
ways to arrange those three random variables.
\end{example}

The \emph{Pearson correlation coefficient }between $X$ and $Y$,
is defined 
\[
r_{X,Y}:=\frac{\Cov\left[X,Y\right]}{\sqrt{\Var\left[X\right]\Var\left[Y\right]}}
\]
which captures the degree of linear association between $X$ and $Y$.
This plays a large role in the context of linear regression which
we shall see in Chapter \ref{chap:linearModels}. 

\begin{example}
\label{exa:isi2004samplepsb1}Suppose two teams play a series of games,
each producing a winner and a loser, until one team has won two more
games than the other. Let $G$ be the total number of games played.
Assume each team has a chance of 0.5 to win each game, independent
of the results of the previous games. (a) Find the probability distribution
of $G$. (b) Find the expected value of $G$.
\end{example}

\begin{sol*}
Note that for this series to end the last two games have to be won
by the same time, with the previous games being won alternatively
by one team and the other. For a series of $n$ games, there are $2^{n}$
total possible outcomes and only two lead to termination, one where
team 1 wins the last two games and the previous ones are won alternatingly
and the other where team 2 wins the last two games etc. Therefore
$\P\left(G=n\right)=\frac{n}{2^{n-1}}1\indicate\left\{ n\geq2\right\} .$
Then the expectation is 
\begin{align*}
\E\left[G\right] & =\sum_{n=2}^{\infty}\frac{n}{2^{n-1}}
\end{align*}
which is arithmo-geometric series that can be shown to converge using
a ratio test. We can note that for $\lvert x\rvert\leq1$
\[
\frac{1}{1-x}=\sum_{n=0}^{\infty}x^{n}
\]
and so by the fact that the power series converges uniformly, 
\[
\frac{1}{\left(1-x\right)^{2}}=\sum_{n=1}^{\infty}nx^{n-1}
\]
where we have interchanged derivatives and limits\hl{Add hyperrefs once differentiation section is complete}.
Finally, we have 
\[
\frac{1}{\left(1-x\right)^{2}}-1=\sum_{n=2}^{\infty}nx^{n-1}.
\]
Letting $x=\frac{1}{2}$, we have that 
\[
\E\left[G\right]=4-1=3.
\]
\end{sol*}

\subsubsection{Quantile functions and coupling}

The expectation of a random variable $X$, if it exists, provides
one measure of central tendency for a random variable. Other useful
measures include the \emph{median, }which is a real number $c$ such
that the probability that $X$ less than or equal $c$ is $\frac{1}{2}$.
This is a useful measure of central tendency when the distribution
of $X$ is highly skewed. The median may not be unique, as is the
case when the random variable $X$ has a discrete distribution. More
formally, we can define a median\emph{ }as any real number $c\in\left[\sup\left\{ x\in\R\mid F\left(x\right)<\frac{1}{2}\right\} ,\sup\left\{ x\in\R\mid F\left(x\right)\leq\frac{1}{2}\right\} \right]$
where $F$ is the CDF of $X$. In general, the $p$th quantile is
any element of the set $Q_{F}\left(p\right):=\left[\sup\left\{ x\in\R\mid F\left(x\right)<p\right\} ,\sup\left\{ x\in\R\mid F\left(x\right)\leq p\right\} \right]$.
To prevent ambiguity, we often use the lowest value of this interval
to construct a quantile function.
\begin{defn}
\label{def:quantileFunction}A quantile function $q_{F}$ for a given
CDF $F$ is defined
\[
q_{F}\left(p\right):=\inf\left\{ x\in\R\mid F\left(x\right)\geq p\right\} .
\]
\end{defn}

\begin{prop}
\label{prop:quantileProperty}Let $q_{F}$ be the quantile function
associated with a CDF $F$. Then $q_{F}$ is the unique function such
that $F(x)\geq p\Longleftrightarrow q_{F}(p)\leq x$.
\end{prop}

\begin{proof}
First we prove that $q_{F}$ satisfies these properties. Assume that
$F\left(x\right)\geq p$ for some $x\in\R$. Then clearly $q_{F}\left(p\right)\leq x$
by the definition of an infimum. Conversely, assume that $q_{F}\left(p\right)\leq x$
and observe that since $F$ is non-decreasing, $F\left(x\right)\geq p$.
Next, suppose that some function $\tilde{q}$ satisfies this equivalence
i.e. 
\begin{equation}
F(x)\geq p\Longleftrightarrow\tilde{q}(p)\leq x.\label{eq:assumptionEquivalence}
\end{equation}
Observe that by the right continuity of $F$, $F\left(q_{F}\left(p\right)\right)\geq p$
and so $\tilde{q}\left(p\right)\leq q_{F}\left(p\right)$. On the
other hand, if there exists some $p^{*}\in\left[0,1\right]$ such
that $\tilde{q}\left(p^{*}\right)<q_{F}\left(p^{*}\right)$ then $F\left(\tilde{q}\left(p^{*}\right)\right)<p^{*}$
which by (\ref{eq:assumptionEquivalence}) implies that $\tilde{q}\left(p^{*}\right)>\tilde{q}\left(p^{*}\right),$
a contradiction. Therefore,
\[
q_{F}\left(p\right)=\tilde{q}\left(p\right)
\]
for all $p\in\left[0,1\right]$.
\end{proof}
Note that the quantile function can also be written as $q_{F}\left(p\right)=\sup\left\{ x\in\R\mid F\left(x\right)<p\right\} $,
a fact that can be established by noting that the non-decreasing nature
of $F$ implies that $\sup\left\{ x\in\R\mid F\left(x\right)<p\right\} \leq\inf\left\{ x\in\R\mid F\left(x\right)\geq p\right\} $.
If the inequality were strict, then there would be some $\sup\left\{ x\in\R\mid F\left(x\right)<p\right\} <c<\inf\left\{ x\in\R\mid F\left(x\right)\geq p\right\} $
and so $F\left(c\right)\not<p$ and $F\left(c\right)\not\geq p$ which
is a contradiction.

We can use the quantile function to construct a standard probability
space for random variables to live in. To begin, let $\probabilityspace$
be arbitrary and let $X$ be a random variable on the space with CDF
$F_{X}$. Define an auxiliary random variable on $\tilde{X}:\left(\left[0,1\right],\borel\left[0,1\right],\lambda\right)\to\left(\R,\borel\left(\R\right)\right)$
by 
\[
\tilde{X}\left(p\right):=q_{F_{X}}\left(p\right)
\]
 and notice that its CDF 
\begin{align*}
F_{\tilde{X}}\left(x\right) & :=\lambda\left(\tilde{X}^{-1}\left[\left(-\infty,x\right]\right]\right)\\
 & =\lambda\left(\left\{ p:q_{F_{X}}\left(p\right)\leq x\right\} \right)\\
 & =\lambda\left(\left\{ p:F_{X}\left(x\right)\geq p\right\} \right)\\
 & =\lambda\left(\left[0,F_{X}\left(x\right)\right]\right)\\
 & =F_{X}\left(x\right).
\end{align*}
Thus we have taken an arbitrary random variable on $X$ on an unspecified
probability space $\probabilityspace$ and constructed an auxiliary
random variable $\tilde{X}$ on $\left(\left[0,1\right],\borel\left[0,1\right],\lambda\right)$
such that $X$ and $\tilde{X}$ are identically distributed. Since
probability is fundamentally concerned with \emph{distributions }rather
than random variables as functions themselves, we have found a canonical
representation for all random variables on the same space $\left(\left[0,1\right],\borel\left[0,1\right],\lambda\right)$.
The next result provides some evidence that this representation is
indeed special.
\begin{defn}
\label{def:coupling} Let $X$ and $Y$ be random variables, not necessarily
on the same probability space. A \emph{coupling }of $X$ and $Y$are
two new random variables $\tilde{X}$ and $\tilde{Y}$ on a common
probability space $\probabilityspace$ such that $\tilde{X}$ has
the same distribution as $X$ and $\tilde{Y}$ has the same distribution
as $Y$. When we define $\tilde{X}\left(p\right)=q_{F_{X}}\left(p\right)$
and $\tilde{Y}\left(p\right)=q_{F_{Y}}\left(p\right)$ on $\left(\left[0,1\right],\borel\left[0,1\right],\lambda\right)$,
then we call $\left(\tilde{X},\tilde{Y},\left(\left[0,1\right],\borel\left[0,1\right],\lambda\right)\right)$
a \emph{quantile coupling}.
\end{defn}

\begin{prop}
\label{prop:quantileCouplingBest}Let $X$ and $Y$ be random variables,
not necessarily on the same probability space. Then for any coupling
$\left(\hat{X},\hat{Y},\probabilityspace\right)$, we have that 
\[
\lambda\left(\lvert\hat{X}-\hat{Y}\rvert\right)\leq\P\left(\lvert\tilde{X}-\tilde{Y}\rvert\right).
\]
\end{prop}

\begin{proof}
Note that for any $s\in\R$
\begin{align*}
\P\left(\hat{X}>s,\hat{Y}>s\right) & \leq\min\left\{ \P\left(\hat{X}>s\right),\P\left(\hat{Y}>s\right)\right\} \\
 & =\min\left\{ 1-F_{X}\left(s\right),1-F_{Y}\left(s\right)\right\} \\
 & =1-\max\left\{ F_{X}\left(s\right),F_{Y}\left(s\right)\right\} \\
 & =1-\lambda\left(p:0\leq p\leq\max\left\{ F_{X}\left(s\right),F_{Y}\left(s\right)\right\} \right)\\
 & =\lambda\left(p:\max\left\{ F_{X}\left(s\right),F_{Y}\left(s\right)\right\} <p\leq1\right)\\
 & =\lambda\left(p:F_{X}\left(s\right)<p,F_{Y}\left(s\right)<p\right)\\
 & =\lambda\left(p:q_{F_{X}}\left(p\right)>s,q_{F_{Y}}\left(p\right)>s\right)\\
 & =\lambda\left(p:\tilde{X}\left(p\right)>s,\tilde{Y}\left(p\right)>s\right)
\end{align*}
where in the second to last equality wee have used Proposition \ref{prop:quantileProperty}.
Further, by the definition of a coupling
\begin{align*}
\P\left(\hat{X}>s\right) & =\lambda\left(\tilde{X}>s\right)\\
\P\left(\hat{Y}>s\right) & =\lambda\left(\tilde{Y}>s\right)
\end{align*}
and so
\[
\lebInt{\P}{\indicate\left\{ \hat{X}>s\right\} +\indicate\left\{ \hat{Y}>x\right\} -2\indicate\left\{ \hat{X}>s,\hat{Y}>s\right\} }\geq\lambda\left(\indicate\left\{ \text{\ensuremath{\tilde{X}}}>s\right\} +\indicate\left\{ \tilde{Y}>x\right\} -2\indicate\left\{ \tilde{X}>s,\tilde{Y}>s\right\} \right).
\]
Note that the expressions inside the integrals are indicators of the
symmetric differences $\left\{ \hat{X}>s\right\} \Delta\left\{ \hat{Y}>s\right\} $
and $\left\{ \tilde{X}>s\right\} \Delta\left\{ \tilde{Y}>s\right\} $
and so we can write the above as 
\begin{align}
\P\left(\omega\in\Omega:\hat{X}\left(\omega\right)>s>\hat{Y}\left(\omega\right)\right)+\P\left(\omega\in\Omega:\hat{Y}\left(\omega\right)>s>\hat{X}\left(\omega\right)\right)\nonumber \\
\geq\lambda\left(p\in\left[0,1\right]:\tilde{X}\left(p\right)>s>\tilde{Y}\left(p\right)\right)+\lambda\left(p\in\left[0,1\right]:\tilde{Y}\left(p\right)>s>\tilde{X}\left(p\right)\right)\label{eq:quantileInequality}
\end{align}
Taking an integral with respect to the Lebesgue measure on $\R$ on
the left hand side, we hav
\begin{align*}
\lambda^{s}\left(\P^{\omega}\left(\omega\in\Omega:\hat{X}\left(\omega\right)>s>\hat{Y}\left(\omega\right)\right)+\P^{\omega}\left(\omega\in\Omega:\hat{Y}\left(\omega\right)>s>\hat{X}\left(\omega\right)\right)\right)\\
=\lambda^{s}\left(\P^{\omega}\left(\indicate\left\{ \hat{X}\left(\omega\right)>s>\hat{Y}\left(\omega\right)\right\} \right)\right) & +\lambda^{s}\left(\P^{\omega}\left(\indicate\left\{ \hat{Y}\left(\omega\right)>s>\hat{X}\left(\omega\right)\right\} \right)\right)\\
=\P^{\omega}\left(\lambda^{s}\left(\indicate\left\{ \hat{X}\left(\omega\right)>s>\hat{Y}\left(\omega\right)\right\} \right)\right) & +\P^{\omega}\left(\lambda^{s}\left(\indicate\left\{ \hat{Y}\left(\omega\right)>s>\hat{X}\left(\omega\right)\right\} \right)\right)\\
=\P^{\omega}\left(\hat{X}-\hat{Y}\right)^{+}+\ \P^{\omega}\left(\hat{Y}-\hat{X}\right)^{+}\quad\\
=\P^{\omega}\left(\lvert\hat{X}-\hat{Y}\rvert\right)\qquad\qquad\qquad\qquad\quad\  & .
\end{align*}
where we have used \hyperref[thm:tonelli]{Tonelli's theorem} in the
second equality and the fact that $f^{-}+f^{+}=\lvert f\rvert$ in
the last. A similar argument where we apply $\lambda^{s}$ to the
right side of (\ref{eq:quantileInequality}) shows (by monotonicity
of integration) that
\[
\P^{\omega}\left(\lvert\hat{X}-\hat{Y}\rvert\right)\geq\lambda\left(\lvert\tilde{X}-\tilde{Y}\rvert\right).
\]
\end{proof}

\subsection{Independence of random variables}

\begin{defn}
\label{def:indepRandomVariables}Let $\probabilityspace$ be a probability
space and let $I$ be an arbitrary index set. Random variables $X_{i}:\Omega\to\R$
where $i\in I$ are independent if $\left\{ \sigma\left(X_{i}\right)\right\} _{i\in I}$
are mutually independent $\sigma-$algebras.

Note that when $I$ is finite, the existence of finite product measures
tells us that we can always construct independent random variables
with distributions identical to $\left\{ X_{i}\right\} _{i\in I}$
. For infinite products we use the Kolmogorov extension theorem. \hl{TODO: Flesh out}A
collection of random variables which are independent and have the
same distribution are called \emph{independent and identically distributed
}or \emph{i.i.d }for short.
\end{defn}

\begin{prop}
\label{prop:indepFunctionsRandomVariables}Let $\probabilityspace$
be a probability space and let $I$ be an index set. If $X_{i}:\Omega\to\R$
-- where $i\in I$ -- are independent random variables and $f_{i}:\R\to\R$
are Borel-measuarable maps then the random variables $f_{i}\left(X_{i}\right)$
are mutually independent.
\end{prop}

\begin{proof}
Note that by Proposition \ref{prop:sigmaAlgebraGeneratedByFunction},
$\sigma\left(f_{i}\left(X_{i}\right)\right)\subseteq\sigma\left(X_{i}\right)$
and so $\sigma\left(f_{i}\left(X_{i}\right)\right)$ are all mutually
independent by definition.
\end{proof}
%
\begin{prop}
\label{prop:indepCDFFactors}Let $\probabilityspace$ be a probability
space and let $X$ and $Y$ be random variables. Then $X$ and $Y$
are independent if and only if 
\[
F_{X,Y}\left(x,y\right):=\P\left(X\leq x\cap Y\leq y\right)=\P\left(X\leq x\right)\P\left(Y\leq y\right)=F_{X}\left(x\right)F_{Y}\left(y\right).
\]
\end{prop}

\begin{proof}
One direction of this is trivial. For the non-trivial direction, notice
that -- by Lemma \ref{lem:collectionIntervalsMeasurable} -- sets
like $\left(-\infty,x\right]$ can be used to generate $\borel\left(\R\right).$
Further, observe that the collection is a $\pi-$system, and so by
Corollary \ref{cor:generatorPreimage}, $\left\{ X^{-1}\left(\left(-\infty,x\right]\right)\right\} _{x\in\R}$
and $\left\{ Y^{-1}\left(\left(-\infty,y\right]\right)\right\} _{y\in\R}$
are $\pi-$systems (since intersections and preimages commute) that
generate $\sigma\left(X\right)$ and $\sigma\left(Y\right)$ , respectively.
The result then follows by Proposition \ref{prop:indepSigmaAlgebras}.
\end{proof}
The generalization of this result to the arbitrary collection of independent
random variables proceeds in the obvious way.
\begin{prop}
\label{prop:indepExpectationFactors}Let $\probabilityspace$ be a
probabiltiy space and let $X$ and $Y$ be non-negative (or integrable)
independent random variables. Then,
\[
\E\left[XY\right]=\E\left[X\right]\E\left[Y\right].
\]
\end{prop}

\begin{proof}
First suppose $X,Y\geq0$ almost surely. Note that
\begin{align*}
\E\left[XY\right] & =\E_{X,Y}\left[xy\right]\\
 & =\E_{X}\left[\E_{Y}\left[xy\right]\right]\\
 & =\E_{X}\left[x\right]\E_{Y}\left[y\right]\\
 & =\E\left[X\right]\E\left[Y\right]
\end{align*}
where in the first equality we have moved to the image measure under
the random vector $\left(X,Y\right)$ through the standard change
of variables (Corollary \ref{cor:changeOfVariables}). Then, since
the image measure under the pair of independent random vectors is
a product measure, we can use Tonelli's theorem to turn it into an
iterated integral. The final equality is yet another application of
the change of variables formula. Now if $X,Y\in\Lp 1{\P},$ then we
can apply the result for non-negative random variables to $\lvert X\rvert,\lvert Y\rvert$
to show that 
\[
\E\left[\lvert XY\rvert\right]=\E\left[\lvert X\rvert\right]\E\left[\lvert Y\rvert\right]<\infty
\]
and then applying the same argument (except replacing the use of Tonelli
with the use of Fubini) yields the result.
\end{proof}
\begin{example}
\label{exa:isi2008samplepsb5} Consider the i.i.d. sequence 
\[
X_{1},X_{2},X_{3},X_{4},X_{5},X_{6}
\]
 where each $X_{i}$ is one of the four symbols $\{a,t,g,c\}$. Further
suppose that 
\[
\begin{array}{ll}
P\left(X_{1}=a\right)=0.1 & P\left(X_{1}=t\right)=0.2\\
P\left(X_{1}=g\right)=0.3 & P\left(X_{1}=c\right)=0.4.
\end{array}
\]

Let $Z$ denote the random variable that counts the number of times
that the subsequence cat occurs (i.e. the letters $c,a$ and $t$
occur consecutively and in the correct order) in the above sequence.
Find $\E\left[Z\right]$.
\end{example}

\begin{sol*}
Let $Y_{i}=\indicate\left\{ X_{i}=c\right\} \indicate\left\{ X_{i+1}=a\right\} \indicate\left\{ X_{i+2}=t\right\} $
for $1\leq i\leq4$ and notice that $Z=\sum_{i=1}^{4}Y_{i}$ and so
\begin{align*}
\E\left[Z\right] & =\sum_{i=1}^{4}\E\left[Y_{i}\right]\\
 & =\sum_{i=1}^{4}\P\left(X_{i}=c\right)\P\left(X_{i+1}=a\right)\P\left(X_{i+1}=t\right)\\
 & =4\times0.4\times0.1\times0.2\\
 & =0.032
\end{align*}
where we used independence in the second equality.
\end{sol*}
\begin{example}
\label{exa:isi2005samplepsb2}Let $X$ and $Y$ be independent random
variables with $X$ having a \hyperref[subsec:binomialDistribution]{binomial distribution}
with parameters $n_{1}$ and $p_{1}$ and $Y$ having a binomial distribution
with parameters $n_{2}$ and $p_{2}$. What is the probability that
$|X-Y|$ is even? Well, we don't need to really worry about the absolute
value sign since if $X-Y$ is even then $Y-X$ is even. Moreover,
we know that $X-Y$ is even if and only if either both $X$ and $Y$
are even or both $X$ and $Y$ are odd. Therefore, formally letting
$E:=\left\{ 0,2,4,\ldots\right\} $ and $O:=\left\{ 1,3,5,\ldots\right\} $
\begin{align*}
\P\left(\lvert X-Y\rvert\in E\right) & =\P\left(X\in E,Y\in E\right)+\P\left(X\in O,Y\in O\right)\\
 & =\P\left(X\in E\right)\P\left(Y\in E\right)+\P\left(X\in O\right)\P\left(Y\in O\right)\\
 & =\sum_{l\in E}\left(\begin{array}{c}
n_{1}\\
l
\end{array}\right)p_{1}^{l}\left(1-p_{1}\right)^{n_{1}-l}\sum_{k\in E}\left(\begin{array}{c}
n_{2}\\
k
\end{array}\right)p_{2}^{k}\left(1-p_{2}\right)^{n_{1}-k}\\
 & \ \ +\sum_{l\in O}\left(\begin{array}{c}
n_{1}\\
l
\end{array}\right)p_{1}^{l}\left(1-p_{1}\right)^{n_{1}-l}\sum_{k\in O}\left(\begin{array}{c}
n_{2}\\
k
\end{array}\right)p_{2}^{k}\left(1-p_{2}\right)^{n_{1}-k}
\end{align*}
\end{example}


\subsection{Multiple random variables and random vectors}

Most of the results that we have deduced for random variables hold
\emph{mutis mutandis }for random vectors, although some ideas need
clarification in this more general context. For instance, the notion
of a CDF is not immediately obvious since there is no canonical ordering
on $\R^{n}$ for $n>1.$

\subsection{Transformations of random variables}

A typical question in basic probability theory is trying to find the
distribution of some transformation of a collection of random variables.
More formally, given random variables $X_{1},X_{2},\ldots,X_{n}$,
on some probability space $\probabilityspace$, we wish to find the
distribution of $T\left(X_{1},X_{2},\ldots,X_{n}\right)$ where $T:\R^{n}\to\R^{m}$
is a Borel-measurable map.

\subsubsection{Transformations on scalar random variables}

\subsubsection{Adding independent random variables}

The simplest transformation we make on a collection of random variables
is simply adding them. The theory of \hyperref[subsec:convolutions]{convolutions}
we developed in Chapter \ref{chap:productMeasures} gives us the tools
to find the distribution of the sum of random variables.
\begin{prop}
\label{prop:distSumIndep}Let $\probabilityspace$ be a probability
space and let $X$ and $Y$ be $\Lp 1{\P}$ (or non-negative) independent
random variables on the space. Then, the distribution of $Z:=X+Y$
is given by 
\[
\P_{Z}\left(B\right)=\E_{Y}\left[\P_{X}\left(B-y\right)\right].
\]
for any $B\in\borel\left(\R\right).$ In particular, the CDF
\[
F_{Z}\left(z\right)=\E_{Y}\left[F_{X}\left(z-y\right)\right]
\]
\end{prop}

\begin{proof}
Note that 
\begin{align*}
\P_{Z}\left(B\right) & =\E_{Y}\left[\E_{X}\left[\indicate_{B}\left(x+y\right)\right]\right]\\
 & =\E_{Y}\left[\E_{X}\left[\indicate_{B-y}\left(x\right)\right]\right]\\
 & =\E_{Y}\left[\P_{X}\left(B-y\right)\right]
\end{align*}
 since $x+y\in B\Longleftrightarrow x\in B-y$. Of course, when $B=\left(-\infty,z\right]$
the result about CDFs follows.
\end{proof}
\begin{cor}
\label{cor:discreteConvolution}Let $\probabilityspace$ be a probability
space and let $X$ and $Y$ be $\Lp 1{\P}$ (or non-negative) independent
and discrete random variables supported on some countable set $S$.
Then, the distribution of $Z:=X+Y$ is given 
\[
\P\left(Z=z\right)=\sum_{y\in S}\P\left(Y=y\right)\mathbb{P}\left(X=z-y\right).
\]
\end{cor}

\begin{proof}
Let $B=\left\{ z\right\} $ in Proposition \ref{prop:distSumIndep}.
\end{proof}
\begin{example}
\label{exa:isi2007samplepsb5}Let $X$ and $Y$ be i.i.d. random variables,
with $P(X=k)=2^{-k}$ for $k=1,2,3,\ldots$ Find $P(X>Y)$ and $P(X>2Y)$.
The trick here is to define $Z:=X-\alpha Y$ for $\alpha>0$ and note
that
\begin{align*}
\mathbb{P}\left(Z=z\right) & =\sum_{y\in\N}\P\left(Y=y\right)\mathbb{P}\left(X=z+\alpha y\right)\\
 & =\sum_{y\in\N}2^{-y}2^{-\left(z+\alpha y\right)}\indicate\left\{ z+\alpha y\geq1\right\} \\
 & =2^{-z}\sum_{y\in\N}2^{-\left(1+\alpha\right)y}\indicate\left\{ y\geq\frac{1-z}{\alpha}\right\} \\
 & =2^{-z}\sum_{y=\max\left\{ 1,\lceil\frac{1-z}{\alpha}\rceil\right\} }^{\infty}2^{-\left(1+\alpha\right)y}.
\end{align*}
For $\alpha=1$, the above reduces to $\P\left(Z=z\right)=\frac{2^{-z}}{3}$
and so $\P\left(Z\geq0\right)=\sum_{z=0}^{\infty}\frac{2^{-z}}{3}=\frac{2}{3}.$
If $\alpha=2$ then $\P\left(Z=z\right)=\frac{2^{-z}}{7}$ and so
$\P\left(Z\geq0\right)=\sum_{z=0}^{\infty}\frac{2^{-z}}{7}=\frac{2}{7}.$
\end{example}

More often than not, we are dealing with a situation where we are
adding random variables that are not of the same ``type'', as is
the case in the following example.
\begin{example}
\label{exa:isi2005samplepsb5}Suppose $X$ and $U$ are independent
random variables with 
\[
P(X=k)=\frac{1}{N+1},\quad k=0,1,2,\ldots,N,
\]
 and $U$ having a uniform distribution on $[0,1]$. Let $Y=X+U$.
What is the distribution of $Y$? What is the Pearson correlation
coefficient $r_{Y,X}$? First note that by Proposition \ref{prop:distSumIndep}
\begin{align*}
F_{Y}\left(y\right) & =\E_{X}\left[F_{U}\left(y-x\right)\right]\\
 & =\E_{X}\left[\left(y-x\right)\indicate\left\{ y-1\leq x\leq y\right\} +\indicate\left\{ x<y-1\right\} \right]\\
 & =\frac{1}{N+1}\sum_{x=0}^{N}\left(y-x\right)\indicate\left\{ y-1\leq x\leq y\right\} +\frac{1}{N+1}\sum_{x=0}^{N}\indicate\left\{ x<y-1\right\} \\
 & =\frac{y-\lfloor y\rfloor}{N+1}\indicate\left\{ 0\leq y\leq N+1\right\} +\frac{\lfloor y\rfloor}{N+1}\indicate\left\{ 0\leq y\leq N+1\right\} +\indicate\left\{ y>N+1\right\} \\
 & =\frac{y}{N+1}\indicate\left\{ 0\leq y\leq N+1\right\} +\indicate\left\{ y>N+1\right\} .
\end{align*}
For the correlation between $X$ and $Y$, note that 
\begin{align*}
\Cov\left[X+U,X\right] & =\Var\left[X\right]\\
 & =
\end{align*}
and $\Var\left[Y\right]=\Var\left[X\right]+\Var\left[U\right]$ and
so 
\[
r_{X,Y}=\frac{\Var\left[X\right]}{\Var\left[X\right]+\Var\left[U\right]}
\]

\hl{TODO}
\end{example}


\subsubsection{Products of random variables}

\subsubsection{Absolute values of random variables}

\subsubsection{Other non-linear transformations}

\begin{example}
	\label{exa:isi2009samplepsb4}
	Let $R$ and $\theta$ be independent non-negative random var?ables such that $R^2 \sim \chi_2^2$ and $\theta \sim \mathrm{U}(0,2 \pi)$. Fix $\theta_0 \in(0,2 \pi)$. Find the distribution of $R \sin \left(\theta+\theta_0\right)$
	\hl{TODO}
\end{example}

\subsection{Order statistics}
\begin{example}
\label{exa:isi2007samplepsb8}Let $U_{1},U_{2},\ldots,U_{n}$ be i.i.d.
uniform $(0,1)$ random variables and suppose 
\[
X=\max\left(U_{1},U_{2},\ldots,U_{n}\right)\text{ and }Y=\min\left(U_{1},U_{2},\ldots,U_{n}\right).
\]

Find the distribution of $Z=X-Y$. \hl{TODO}
\end{example}


\subsection{Concentration inequalities}
\begin{example}
\label{exa:isi2015psb5ptB}Consider the setting in Example \ref{exa:isi2015psb5ptA}
again. Can we provide a bound for $\E\left[\lvert X+Y\rvert\right]$
with the additional information that $\mathbb{E}\left[X+Y\right]=\mathbb{E}\left[X-Y\right]=0$?
Note that in this case $\E\left[X\right]=\E\left[Y\right]=0$ and
so $\Var\left[X+Y\right]=\E\left[\left(X+Y\right)^{2}\right]=3$ and
so by \hyperref[thm:jensenInequality]{Jensen's inequality} and the
fact that $x\to\sqrt{x}$ is concave, we have that
\[
\E\left[\lvert X+Y\rvert\right]\leq\sqrt{\E\left[\left(X+Y\right)^{2}\right]}=\sqrt{3}.
\]
\end{example}

\begin{example}[ISI 2016 PSA 26]
\label{exa:isi2016psa26}Two integers $m$ and $n$ are chosen at
random with replacement from $\{1,2,\ldots,9\}$. What is the probabiliy
that that $m^{2}-n^{2}$ is even? First note that $m^{2}-n^{2}=\left(m+n\right)\left(m-n\right)$
so we need to either $m+n$ or $m-n$ to be even. For this to be true,
both $m$ and $n$ need to be even, or both odd. The probability that
they are both even is $\frac{4}{9}\times\frac{4}{9}=\frac{16}{81}$
since sampling with replacement leads to indepenent events. The probability
that they are both odd is similarly $\frac{5}{9}\times\frac{5}{9}=\frac{25}{81}.$
Thus the probability that they are either both odd or even is $\frac{16+25}{81}=\frac{41}{81}.$
\end{example}

\begin{example}
\label{exa:isi2004samplepsb4}Two policemen are sent to watch a road
that is $1\mathrm{~km}$ long. Each of the two policemen is assigned
a position on the road which is chosen according to a uniform distribution
along the length of the road and independent of the other's position.
Find the probability that the policemen will be less than $1/4$ kilometer
apart when they reach their assigned posts.\hl{TODO}
\end{example}

\begin{example}
\label{exa:isi2004samplepsb7}Suppose $X$ has a normal distribution
with mean 0 and variance 25 . Let $Y$ be an independent random variable
taking values -1 and 1 with equal probability. Define$S=XY+\frac{X}{Y}$
and$T=XY-\frac{X}{Y}$. (a) Find the probability distribution of $S$.
(b) Find the probability distribution of $\left(\frac{S+T}{10}\right)^{2}$.\hl{TODO}
\end{example}

\begin{example}
\label{exa:isi2006samplepsb3}Let $X_{1},X_{2}\ldots$ be i.i.d. Bernoulli
random variables with parameter $\frac{1}{4}$, let $Y_{1},Y_{2},\ldots$
be another sequence of i.i.d. Bernoulli random variables with parameter
$\frac{3}{4}$ and Let $N$ be a geometric random variable with parameter
$\frac{1}{2}$ (i. e. $P(N=k)=\frac{1}{2^{k}}$for $k=1,2\ldots)$.
Assume the $X_{i}$ 's, $Y_{j}$'s and $N$ are all independent. Compute
$\Cov\left(\sum_{i=1}^{N}X_{i},\sum_{i=1}^{N}Y_{i}\right)$.\hl{TODO}
%
\end{example}


