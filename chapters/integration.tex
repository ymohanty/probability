
\chapter{Integration}

\section{Constructing the Lebesgue integral}
\begin{thm}
\label{thm:existenceUniquenessLebesgueIntegral}Let $\left(\X,\mathcal{F}\right)$
be measurable space. For any measure $\mu$ on $\mathcal{F}$, there
exists a unique linear functional
\[
\bar{\mu}:\mathcal{\mathcal{M}^{+}}\left(\X,\mathcal{F}\right)\longrightarrow\left[0,\infty\right]
\]
that satisfies the following properties

\begin{enumerate}[label=(\roman*),leftmargin=.1\linewidth,rightmargin=.4\linewidth]
	\item $ \bar{\mu}\left(\indicate_A\right) = \mu\left(A\right) $ for any $ A \in \mathcal{F} $
	\item \textnormal{(Linearity)}  For any $f,g\in \mathcal{\mathcal{M}^{+}}\left(\X,\mathcal{F}\right)$ and any $\alpha,\beta \geq 0 $ 
	\[
				\bar{\mu}\left(\alpha f + \beta g\right) = \alpha \bar{\mu}\left(f\right) + \beta \bar{\mu}\left(g\right)                
	\]
	\item \textnormal{(Monotone convergence)} For a sequence of increasing functions $ \left\{f_n\right\}_{n\in\N} \in \mathcal{\mathcal{M}^{+}}\left(\X,\mathcal{F}\right)$ 
	\[
				\bar{\mu}\left(\lim_{n\to\infty}f_n\right) = \lim_{n\to\infty}\bar{\mu}\left(f_n\right)
	\]
\end{enumerate}
\end{thm}

We can prove Theorem \ref{thm:existenceUniquenessLebesgueIntegral}
constructively, in a similar fashion to how we proved the existence
of the Lebesgue measure in Chapter 1. In this spirit, we shall define
a functional on the non-negative measurable simple functions and extend
the domain of this functional to more complicated function spaces.
\begin{defn}
\label{def:simpleFuncIntegral}For any measurable simple function
$s\in\mathcal{M}_{\textnormal{sim}}\left(\X,\F\right)$ with the standard
representation
\[
s=\sum_{i=1}^{I}\alpha_{i}\indicate_{A_{i}}
\]
define, for any measure $\mu$ on $\F$, the functional $\bar{\mu}_{0}$
as
\[
\bar{\mu}_{0}\left(s\right):=\sum_{i=1}^{I}\alpha_{i}\mu\left(A_{i}\right).
\]
\end{defn}

Immediately, we can see that our proto-integral $\bar{\mu}_{0}$ behaves
quite nicely: it is always non-negative and since indicator functions
are special cases of simple functions, we have that 
\[
\bar{\mu}_{0}\left(\indicate_{A}\right)=\mu\left(A\right)
\]
for any $A\in\F$. Moreover, our functional satisfies the linearity
property on the space of simple measurable functions. To see this
we shall need the following lemma which solves a minor technical issue
that arises due to our definition of $\bar{\mu}$ relying on the standard
representation of simple functions.
\begin{lem}
\label{lem:lebIntDiffRep}For any non-negative measurable simple function
$s\in\mathcal{M}_{\textnormal{sim}}\left(\X,\F\right)$ with a standard
representation
\[
s=\sum_{i=1}^{I}\alpha_{i}\indicate_{A_{i}}
\]
and another representation
\[
s=\sum_{j=1}^{J}\beta_{i}\indicate_{B_{j}}
\]
where $\left\{ B_{j}\right\} _{j=1}^{J}$ is a partition of $\X$
we have 
\[
\bar{\mu}_{0}\left(s\right):=\sum_{i=1}^{I}\alpha_{i}\mu\left(A_{i}\right)=\sum_{j=1}^{J}\beta_{j}\mu\left(B_{j}\right).
\]
where $\bar{\mu}_{0}$ is the functional derived from a measure $\mu$
as in Definition \ref{def:simpleFuncIntegral}.
\end{lem}

\begin{proof}
Note that both $A_{i}$ and $B_{j}$ partition $\X$ and so observe
that
\begin{align*}
\sum_{i=1}^{I}\alpha_{i}\mu\left(A_{i}\right) & =\sum_{i=1}^{I}\alpha_{i}\mu\left(\bigcup_{j=1}^{J}\left(A_{i}\cap B_{j}\right)\right)\\
 & =\sum_{i=1}^{I}\sum_{j=1}^{J}\alpha_{i}\mu\left(A_{i}\cap B_{j}\right)
\end{align*}
where the last equality follows by finite addivity. Similarly,

\[
\sum_{j=1}^{J}\beta_{j}\mu\left(B_{j}\right)=\sum_{i=1}^{I}\sum_{j=1}^{J}\beta_{j}\mu\left(A_{i}\cap B_{j}\right).
\]
Now observe that since $\sum_{i=1}^{I}\alpha_{i}\indicate_{A_{i}}=\sum_{j=1}^{J}\beta_{i}\indicate_{B_{j}}$
we know that $\alpha_{i}=\beta_{j}$ if $A_{i}\cap B_{j}\neq\emptyset$;
conversely, if $A_{i}\cap B_{j}=\emptyset$ then $\mu\left(A_{i}\cap B_{j}\right)=0$
and so we have that 
\[
\sum_{i=1}^{I}\alpha_{i}\mu\left(A_{i}\right)=\sum_{i=1}^{I}\sum_{j=1}^{J}\alpha_{i}\mu\left(A_{i}\cap B_{j}\right)=\sum_{i=1}^{I}\sum_{j=1}^{J}\beta_{j}\mu\left(A_{i}\cap B_{j}\right)=\sum_{j=1}^{J}\beta_{j}\mu\left(B_{j}\right).
\]
\end{proof}
\begin{prop}
\label{prop:lebIntLinearitySimpleFunc}Let $s,t\in\mathcal{M}_{\textnormal{sim}}\left(\X,\F\right)$
and let $\mu$be a measure on $\mathcal{F}$. Then, for any $\alpha,\beta\geq0$,
we have that
\[
\bar{\mu}_{0}\left(\alpha s+\beta t\right)=\alpha\bar{\mu}_{0}\left(s\right)+\beta\bar{\mu}_{0}\left(t\right).
\]
\end{prop}

\begin{proof}
It is sufficient to prove that $\bar{\mu}_{0}\left(\alpha s\right)=\alpha\bar{\mu}_{0}\left(s\right)$
and $\bar{\mu}_{0}\left(s+t\right)=\bar{\mu}_{0}\left(s\right)+\bar{\mu}_{0}\left(t\right).$
To show the first equality, simply notice that if $s$ is given by
the standard representation 
\[
s=\sum_{i=1}^{I}a_{i}\indicate_{A_{i}}
\]
then the standard representation of $\alpha s$ when $\alpha>0$ is
simply
\[
\alpha s=\sum_{i=1}^{I}\alpha a_{i}\indicate_{A_{i}}
\]
and so 
\begin{align*}
\bar{\mu}_{0}\left(\alpha s\right) & =\sum_{i=1}^{I}\alpha a_{i}\mu\left(A_{i}\right)\\
 & =\alpha\sum_{i=1}^{I}a_{i}\mu\left(A_{i}\right)\\
 & =\alpha\bar{\mu}_{0}\left(s\right).
\end{align*}
When $\alpha=0$, $\alpha s=\indicate_{\emptyset}$ and so $\bar{\mu}_{0}\left(\alpha s\right)=\mu\left(\emptyset\right)=0=\alpha\bar{\mu}_{0}\left(s\right).$

In order to prove the second equality, observe that if the standard
representations of $s$ and $t$ are given by
\begin{align*}
s & =\sum_{i=1}^{I}a_{i}\indicate_{A_{i}}\\
t & =\sum_{j=1}^{J}b_{j}\indicate_{B_{j}}
\end{align*}
then
\begin{align*}
\bar{\mu}_{0}\left(s+t\right) & =\bar{\mu}_{0}\left(\sum_{i=1}^{I}\sum_{j=1}^{J}\left(a_{i}+b_{j}\right)\indicate_{A_{i}\cap B_{j}}\right)\\
 & =\sum_{i=1}^{I}\sum_{j=1}^{J}\left(a_{i}+b_{j}\right)\mu\left(A_{i}\cap B_{j}\right)\\
 & =\sum_{i=1}^{I}a_{i}\sum_{j=1}^{J}\mu\left(A_{i}\cap B_{j}\right)+\sum_{j=1}^{J}b_{j}\sum_{i=1}^{I}\mu\left(A_{i}\cap B_{j}\right)\\
 & =\sum_{i=1}^{I}a_{i}\mu\left(A_{i}\right)+\sum_{j=1}^{J}b_{j}\mu\left(B_{j}\right)\\
 & =\bar{\mu}_{0}\left(s\right)+\bar{\mu}_{0}\left(t\right)
\end{align*}
where the first equality follows from Proposition \ref{prop:simpleFunctionsAddMultiply},
the second equality due to Lemma \ref{lem:lebIntDiffRep}, and the
fourth equality due to finite additivity of $\mu$. This completes
the proof.
\end{proof}
Linearity and non-negativity of $\bar{\mu}_{0}$ tells us that for
any simple measurable functions $f\leq g$, we have that
\[
\bar{\mu}_{0}\left(f\right)\leq\bar{\mu}_{0}\left(g\right).
\]
Indeed, we can decompose $g=\left(g-f\right)+f$ where $g-f$ and
$f$ are both non-negative simple functions and so
\[
\bar{\mu}_{0}\left(g\right)=\bar{\mu}_{0}\left(g-f\right)+\bar{\mu}_{0}\left(f\right)
\]
which by non-negativity proves our claim. In other words, our functional
$\bar{\mu}_{0}$ is an \emph{increasing }or \emph{monotone }functional.
With this final fact, we are now ready to extend $\bar{\mu}_{0}$
to the space of non-negative measurable functions.
\begin{proof}[Proof of Theorem \ref{thm:existenceUniquenessLebesgueIntegral}]
For any $f\in\nonnegMeasurableFunctions$, define 
\[
\lebInt{\mu}f:=\sup\left\{ \bar{\mu}_{0}\left(s\right)\mid s\in\mathcal{M}_{\textnormal{sim}}\left(\X,\F\right)\textnormal{ such that }s\leq f\right\} .
\]
We claim that $\bar{\mu}$ is the unique functional (for a given measure
$\mu$) that satisfies the properties described in the statement of
the theorem. To see this, we first have to show that the functional
defined above indeed satisfes the three requisite properties and then
show that it is the only such functional. Before we do this, note
that the set $\left\{ \bar{\mu}_{0}\left(s\right)\mid s\in\mathcal{M}_{\textnormal{sim}}\left(\X,\F\right)\text{s.t\ }s\leq f\right\} $
is always non-empty, thanks to Proposition \ref{prop:simpleFunctionMonotoneConvergence}.
Therefore the supremum is at least 0 since $\bar{\mu}_{0}\left(s\right)\geq0.$
Moreover, since for any non-negative measurable functions $f\leq g$,
$\left\{ \bar{\mu}_{0}\left(s\right)\mid s\in\mathcal{M}_{\textnormal{sim}}\left(\X,\F\right)\textnormal{ such that }s\leq f\right\} \subseteq\left\{ \bar{\mu}_{0}\left(s\right)\mid s\in\mathcal{M}_{\textnormal{sim}}\left(\X,\F\right)\textnormal{ such that }s\leq g\right\} ,$we
can conclude that $\lebInt{\mu}f\leq\lebInt{\mu}g.$ Thus we already
know that our functional $\bar{\mu}$ is both non-negative and monotone.
Next, to show property (i), observe that for any simple function $t\in\mathcal{M}_{\textnormal{sim}}\left(\X,\F\right)$
\begin{align*}
\lebInt{\mu}t & =\sup\left\{ \bar{\mu}_{0}\left(s\right)\mid s\in\mathcal{M}_{\textnormal{sim}}\left(\X,\F\right)\textnormal{ such that }s\leq t\right\} \\
 & =\bar{\mu}_{0}\left(t\right)
\end{align*}
where the second equality follows from the monotonicty of $\bar{\mu}_{0}$.
Property (i) then follows from letting $t=\indicate_{A}$ for any
set $A\in\F.$ Next, we prove property (iii) which is monotone convergence.

Let $\left\{ f_{n}\right\} _{n\in\N}\in\nonnegMeasurableFunctions$
an increasing sequence of measurable functions. Since we are working
with the extended non-negative real numbers, we know that 
\[
f:=\lim_{n\to\infty}f_{n}=\sup_{n\in\N}f_{n}\in\nonnegMeasurableFunctions
\]
by Proposition \ref{prop:supInfMeasurable}. Note that by the monotonicity
of $\bar{\mu},$ $\bar{\mu}\left(f_{n}\right)\leq\bar{\mu}\left(f\right)$
for all $n\in\N$ and so 
\[
\lim_{n\to\infty}\lebInt{\mu}{f_{n}}\leq\lebInt{\mu}f.
\]
To deduce the reverse inequality, let $s\leq f$ be a non-negative
measurable simple function, fix some $t\in\left(0,1\right)$ and define
$A_{n}=\left\{ x\in\X\mid f_{n}\left(x\right)\geq ts\left(x\right)\right\} $.
Since $f_{n}\leq f_{n+1}$, we have that $A_{n}\subseteq A_{n+1}$.
Moreover, for any $x\in\X$, if $f\left(x\right)>0$ then there is
some $\epsilon_{x}>0$ such that $f\left(x\right)-ts\left(x\right)=2\epsilon_{x}$
and so, by pointwise convergence, there exists some $n_{\epsilon_{x}}\in\N$
such that 
\begin{align*}
f_{n}\left(x\right)-ts\left(x\right) & =\underbrace{\left(f\left(x\right)-ts\left(x\right)\right)}_{2\epsilon_{x}}-\underbrace{\left(f\left(x\right)-f_{n}\left(x\right)\right)}_{\leq\epsilon_{x}}\geq\epsilon_{x}>0
\end{align*}
for all $n\geq n_{\epsilon_{x}},$ proving that $x\in A_{n}$ for
such $n.$ Conversely, if $f\left(x\right)=0$ then $f\left(x\right)=f_{n}\left(x\right)=ts\left(x\right)=0$
and so $x\in A_{n}$ for every $n\in\N$. Together, these two cases
show that $\bigcup_{n\in\N}A_{n}=\X$. Now, by the montonicity of
$\bar{\mu}$, observe that
\begin{align}
\lebInt{\mu}{f_{n}} & \geq\lebInt{\mu}{f_{n}\indicate_{A_{n}}}\nonumber \\
 & \geq\lebInt{\mu}{ts\indicate_{A_{n}}}\nonumber \\
 & =\bar{\mu}_{0}\left(ts\indicate_{A_{n}}\right)\nonumber \\
 & =t\bar{\mu}_{0}\left(s\indicate_{A_{n}}\right)\nonumber \\
 & =t\lebInt{\mu}{s\indicate_{A_{n}}}\label{eq:monotoneConvPointwiseEq}
\end{align}
where the third and last equalities follow from the result above that
$\bar{\mu}$ extends $\bar{\mu}_{0}$ and the fourth equality is due
to the \hyperref[prop:lebIntLinearitySimpleFunc]{linearity of $\bar{\mu}_0$}.
Let $s=\sum_{i=1}^{J}b_{i}\indicate_{B_{i}}$then by Lemma \ref{lem:lebIntDiffRep}
\[
\bar{\mu}\left(s\indicate_{A_{n}}\right)=\sum_{i=1}^{I}b_{i}\mu\left(B_{i}\cap A_{n}\right).
\]
Since $A_{n}\subseteq A_{n+1}\Longrightarrow B_{i}\cap A_{n}\subseteq B_{i}\cap A_{n+1}$,
by \hyperref[prop:measureProperties]{continuity from below} of measures,
we have that 
\begin{align}
\lim_{n\to\infty}\lebInt{\mu}{s\indicate_{A_{n}}} & =\lim_{n\to\infty}\sum_{i=1}^{I}b_{i}\mu\left(B_{i}\cap A_{n}\right)\nonumber \\
 & =\sum_{i=1}^{I}b_{i}\mu\left(\bigcup_{n\in\N}\left(B_{i}\cap A_{n}\right)\right)\nonumber \\
 & =\sum_{i=1}^{I}b_{i}\mu\left(B_{i}\right)\nonumber \\
 & =\mu\left(s\right).\label{eq:monotoneConvSimpleLimit}
\end{align}
where the third equality follows from the fact that $\bigcup_{n\in\N}A_{n}=\X$.
Together, (\ref{eq:monotoneConvPointwiseEq}) and (\ref{eq:monotoneConvSimpleLimit})
imply that 
\[
\lim_{n\to\infty}\lebInt{\mu}{f_{n}}\geq t\lebInt{\mu}s.
\]
Since this is true for any $t\in\left(0,1\right)$, we have that 
\[
\lim_{n\to\infty}\lebInt{\mu}{f_{n}}\geq\lebInt{\mu}s.
\]
Finally, since $s\leq f$ was an arbitrary simple function, we have
that 
\[
\lim_{n\to\infty}\lebInt{\mu}{f_{n}}\geq\sup\left\{ \bar{\mu}_{0}\left(s\right)\mid s\in\mathcal{M}_{\textnormal{sim}}\left(\X,\F\right)\textnormal{ such that }s\leq f\right\} =\lebInt{\mu}f.
\]

Next, we show that $\bar{\mu}$ is a linear functional. As before,
it is sufficient to show that for any $f,g\in\nonnegMeasurableFunctions$
and any $\alpha\geq0$,
\begin{align*}
\lebInt{\mu}{\alpha f} & =\alpha\lebInt{\mu}f\\
\lebInt{\mu}{f+g} & =\lebInt{\mu}f+\lebInt{\mu}g.
\end{align*}
First we shall show the homogenous scaling property. If $\alpha=0$
then the proof is trivial; if $\alpha>0$ then by the monotone convergence
property 
\begin{align*}
\lebInt{\mu}{\alpha f} & =\lim_{n\to\infty}\lebInt{\mu}{s_{n}}
\end{align*}
where $\left\{ s_{n}\right\} _{n\in\N}$ is an increasing sequence
of simple functions such that $s_{n}\nearrow\alpha f$. Then, $h_{n}=\frac{s_{n}}{\alpha}$
is an increasing sequence of simple functions such that $h_{n}\nearrow f$
and so 
\begin{align*}
\lebInt{\mu}f & =\lim_{n\to\infty}\lebInt{\mu}{h_{n}}\\
 & =\lim_{n\to\infty}\frac{1}{\alpha}\bar{\mu}_{0}\left(s_{n}\right)\\
 & =\frac{1}{\alpha}\lebInt{\mu}{\alpha f}.
\end{align*}
Rearranging yields the proof. Now let $\left\{ s_{n}\right\} _{n\in\N},\left\{ t_{n}\right\} _{n\in\N}$
be increasing sequences of measurable simple functions such that $s_{n}\nearrow f,t_{n}\nearrow g$.
By the linearity of limits, we have that $s_{n}+t_{n}\nearrow f+g$
and so
\begin{align*}
\lebInt{\mu}{f+g} & =\lim_{n\to\infty}\lebInt{\mu}{s_{n}+t_{n}}\\
 & =\lim_{n\to\infty}\bar{\mu}_{0}\left(s_{n}+t_{n}\right)\\
 & =\lim_{n\to\infty}\bar{\mu}_{0}\left(s_{n}\right)+\lim_{n\to\infty}\bar{\mu}_{0}\left(t_{n}\right)\\
 & =\lebInt{\mu}f+\lebInt{\mu}g
\end{align*}
where the first equality follows from the monotone convergence property,
the second equality follows from the fact that measurable simple functions
are \hyperref[prop:simpleFunctionsAddMultiply]{closed under addition}
and the fact that $\bar{\mu}$ extends $\bar{\mu}_{0}$, the third
equality due to the linearity of $\bar{\mu}_{0}$, and finally the
fourth equality due to a second application of monotone convergence.

To show that our functional $\bar{\mu}$ is unique, suppose that there
were two functionals $\bar{\mu}_{1},\bar{\mu}_{2}$ that satisfied
properties (i)-(iii) with respect to some measure $\mu.$ Then, by
property (i), $\bar{\mu}_{1}\left(\indicate_{A}\right)=\mu\left(A\right)=\bar{\mu}_{2}\left(\indicate_{A}\right)$
for any measurable set $A\in\F$. Next, by the linearity property
along with equality on indicator functions, $\bar{\mu}_{1}\left(s\right)=\bar{\mu}_{2}\left(s\right)$
for any $s\in\mathcal{M}_{\textnormal{sim}}\left(\X,\F\right).$ Finally,
for any arbitrary $f\in\nonnegMeasurableFunctions$ and $s_{n}\in\mathcal{M}_{\textnormal{sim}}\left(\X,\F\right)$
such that $s_{n}\nearrow f$, observe that
\begin{align*}
\bar{\mu}_{1}\left(f\right) & =\lim_{n\to\infty}\bar{\mu}_{1}\left(s_{n}\right)\\
 & =\lim_{n\to\infty}\bar{\mu}_{2}\left(s_{n}\right)\\
 & =\bar{\mu}_{2}\left(f\right)
\end{align*}
where the first equality uses the monotone convergence propert of
$\bar{\mu}_{1}$, the second equality uses the fact that our two functionals
are equal on simple functions, and the last equality uses the monotone
convergence property of $\bar{\mu}_{2}$. This completes the proof.
\end{proof}
\begin{defn}
\label{def:integrable}A function $f\in\mathcal{M}\left(\X,\mathcal{F}\right)$
is called \emph{integrable }with respect to a measure $\mu$(or $\mu-$\emph{integrable
}in short) if 
\[
\bar{\mu}\left(\left|f\right|\right)<\infty.
\]
The collection of all such functions is denoted $\Lp 1{\X,\F,\mu}$.
\end{defn}

When the underlying space and $\sigma-$algebra are clear, we shall
simply write $\Lp 1{\mu}.$ The significance of the exponent $1$
will become clear in the next chapter.
\begin{prop}
\label{prop:L1conditions}A function $f\in\measurableFunctions$ is
in $\Lp 1{\mu}$ if and only if 
\[
\lebInt{\mu}{f^{+}},\lebInt{\mu}{f^{-}}<\infty.
\]
\end{prop}

\begin{proof}
Note that if $\lebInt{\mu}{\left|f\right|}<\infty$ then
\begin{align*}
\lebInt{\mu}{\left|f\right|} & =\lebInt{\mu}{f^{+}+f^{-}}\\
 & =\lebInt{\mu}{f^{+}}+\lebInt{\mu}{f^{-}}\\
 & <\infty
\end{align*}
which would imply that $\lebInt{\mu}{f^{+}},\lebInt{\mu}{f^{-}}<\infty.$
The converse follows similarly.
\end{proof}
\begin{prop}
\label{prop:L1VectorSpace}For any measure space $\left(\X,\F,\mu\right)$,
the space of $\mu-$integrable functions $\Lp 1{\mu}$ is a vector
space over $\R$.
\end{prop}

\begin{proof}
First observe that the zero function $\indicate_{\emptyset}\in\Lp 1{\mu}$since
$\lebInt{\mu}{\left|\indicate_{\emptyset}\right|}=\mu\left(\emptyset\right)=0<\infty$
and so we have an additive identity. Next, notice that $f\in\Lp 1{\mu}\Longleftrightarrow-f\in\Lp 1{\mu}$
and so we have additive inverses for each function. Commutativity
and associativity of addition follow from the definition of addition
on spaces of functions, as does distributivity of scalar multiplication
over addition. Finally, notice that for any $\alpha\in\R$ and any
functions $f,g\in\Lp 1{\mu}$
\begin{align*}
\lebInt{\mu}{\left|\alpha f+g\right|} & \leq\lebInt{\mu}{\left|\alpha f\right|+\left|g\right|}\\
 & =\left|\alpha\right|\lebInt{\mu}{\left|f\right|}+\lebInt{\mu}{\left|g\right|}\\
 & <\infty
\end{align*}
where the first inequality follows from the triangle inequality of
$|\cdot|$ and monotonicty of $\bar{\mu}$, and the equality follows
from the linearity of $\bar{\mu}$. This proves that $\Lp 1{\mu}$
is closed under finite linear combinations and so is a vector space.
\end{proof}
\begin{defn}
\label{def:integral}For any function $f\in\measurableFunctions$
and a measure $\mu$ on $\F$, we can define the \emph{Lebesgue integral}
\[
\tilde{\mu}\left(f\right):=\lebInt{\mu}{f^{+}}-\lebInt{\mu}{f^{-}}
\]
wherever the difference is defined (i.e. at least one of $\lebInt{\mu}{f^{+}}$
and $\lebInt{\mu}{f^{-}}$ is finite).
\end{defn}

\begin{rem*}
Note that for any non-negative measurable function $f$, $\tilde{\mu}\left(f\right)=\lebInt{\mu}f$
and so $\tilde{\mu}$ extends $\bar{\mu}$ just like $\bar{\mu}$
extended $\bar{\mu}_{0}$
\end{rem*}

\section{Properties of the integral}
\begin{prop}
\label{prop:linearityLebIntL1}For any functions $f,g\in\Lp 1{\X,\F,\mu}$
and scalar $\alpha\in\R$, we have that 
\[
\tilde{\mu}\left(\alpha f+g\right)=\alpha\tilde{\mu}\left(f\right)+\tilde{\mu}\left(g\right).
\]
\end{prop}

\begin{proof}
First we shall prove that $\tilde{\mu}\left(\alpha f\right)=\alpha\tilde{\mu}\left(f\right).$
Note that if $\alpha=0$ then the equality follows trivially. If $\alpha>0$
then 
\begin{align*}
\tilde{\mu}\left(\alpha f\right) & =\lebInt{\mu}{\left(\alpha f\right)^{+}}-\lebInt{\mu}{\left(\alpha f\right)^{-}}\\
 & =\lebInt{\mu}{\alpha f\indicate_{\left\{ \alpha f>0\right\} }}-\lebInt{\mu}{-\alpha f\indicate_{\left\{ \alpha f<0\right\} }}\\
 & =\lebInt{\mu}{\alpha f\indicate_{\left\{ f>0\right\} }}-\lebInt{\mu}{-\alpha f\indicate_{\left\{ f<0\right\} }}\\
 & =\alpha\lebInt{\mu}{f^{+}}-\alpha\lebInt{\mu}{f^{-}}\\
 & =\alpha\tilde{\mu}\left(f\right)
\end{align*}
where the third equality follows from the fact that $\alpha>0$ and
the fourth equality due to the linearity of $\bar{\mu}.$ Finally,
if $\alpha<0$ then
\begin{align*}
\tilde{\mu}\left(\alpha f\right) & =\lebInt{\mu}{\left(\alpha f\right)^{+}}-\lebInt{\mu}{\left(\alpha f\right)^{-}}\\
 & =\lebInt{\mu}{\alpha f\indicate_{\left\{ \alpha f>0\right\} }}-\lebInt{\mu}{-\alpha f\indicate_{\left\{ \alpha f<0\right\} }}\\
 & =\lebInt{\mu}{\alpha f\indicate_{\left\{ f<0\right\} }}-\lebInt{\mu}{-\alpha f\indicate_{\left\{ f>0\right\} }}\\
 & =\lebInt{\mu}{-\alpha\times-f\indicate_{\left\{ f<0\right\} }}-\lebInt{\mu}{-\alpha f\indicate_{\left\{ f>0\right\} }}\\
 & =\alpha\bar{\mu}\left(f^{+}\right)-\alpha\lebInt{\mu}{f^{-}}\\
 & =\alpha\tilde{\mu}\left(f\right).
\end{align*}

Next, let $h=f+g$ and observe that
\begin{align*}
h & =h^{+}-h^{-}\\
 & =f+g\\
 & =\left(f^{+}-f^{-}\right)+\left(g^{+}-g^{-}\right).
\end{align*}
Rearranging, we have that 
\[
h^{+}+f^{-}+g^{-}=h^{-}+f^{+}+g^{-}
\]
where the functions on each side are non-negative measurable functions
and so
\[
\lebInt{\mu}{h^{+}+f^{-}+g^{-}}=\lebInt{\mu}{h^{+}}+\lebInt{\mu}{f^{-}}+\lebInt{\mu}{g^{-}}=\lebInt{\mu}{h^{-}}+\lebInt{\mu}{f^{+}}+\lebInt{\mu}{g^{+}}=\lebInt{\mu}{h^{-}+f^{+}+g^{-}}
\]
by linearity of $\bar{\mu}.$ As $h,g,f\in\Lp 1{\mu}$, the integrals
of the individual components are not infinite and so we can rearrange
the second equality above as
\[
\underbrace{\lebInt{\mu}{h^{+}}-\lebInt{\mu}{h^{-}}}_{\tilde{\mu}\left(h\right)}=\underbrace{\lebInt{\mu}{f^{+}}-\lebInt{\mu}{f^{-}}}_{\tilde{\mu}\left(f\right)}+\underbrace{\lebInt{\mu}{g^{+}}-\lebInt{\mu}{g^{-}}}_{\tilde{\mu}\left(g\right)}
\]
which completes the proof.
\end{proof}
\begin{cor}
\label{cor:monotonicityLebIntL1}For any functions $f,g\in\Lp 1{\X,\F,\mu}$
such that $f\leq g$ pointwise,
\[
\tilde{\mu}\left(f\right)\leq\tilde{\mu}\left(g\right).
\]
\end{cor}

\begin{proof}
Note that $h=g-f\geq0$ and so 
\[
\tilde{\mu}\left(g-f\right)=\lebInt{\mu}{g-f}\geq0.
\]
Then, by Proposition \ref{prop:linearityLebIntL1}, 
\[
\tilde{\mu}\left(g-f\right)=\tilde{\mu}\left(g\right)-\tilde{\mu}\left(f\right)\geq0
\]
which completes the proof.
\end{proof}
\begin{cor}
\label{cor:triangleIneqLebIntL1}For any function $f\in\Lp 1{\X,\F,\mu}$
\[
\left|\tilde{\mu}\left(f\right)\right|\leq\tilde{\mu}\left(\left|f\right|\right).
\]
\end{cor}

\begin{proof}
By Corollary \ref{cor:monotonicityLebIntL1} and the triangle inequality
for $\left|\cdot\right|$
\[
\tilde{\mu}\left(f\right)\leq\tilde{\mu}\left(\left|f\right|\right).
\]
Similarly, we have that
\[
-f\leq\left|f\right|\Longrightarrow\tilde{\mu}\left(-f\right)\leq\tilde{\mu}\left(\left|f\right|\right)\Longrightarrow\tilde{\mu}\left(f\right)\geq-\tilde{\mu}\left(\left|f\right|\right)
\]
where the second implication follows due to linearity. Together, the
two inequalities imply that
\[
\left|\tilde{\mu}\left(f\right)\right|\leq\tilde{\mu}\left(\left|f\right|\right)
\]
which is the result.
\end{proof}

\subsection{Interchanging limits and integrals}

At this point we have defined the functionals $\bar{\mu}_{0},\bar{\mu,}$and
$\tilde{\mu}$ to operate on simple functions, non-negative measurable
functions, and all measurable functions respectively. Since $\bar{\mu}$
extends $\bar{\mu}_{0}$ and $\tilde{\mu}$ extends $\bar{\mu}$,
we can dispense with the unnecessary amounts of new notation and simply
denote the integral as $\bar{\mu}\left(f\right)$ for any function
$f\in\measurableFunctions$, provided that the integral is defined.
As discussed earlier, this is analagous to our construction of the
Lebesgue measure in Chapter 1, where we ``extended'' the measure
from simple sets to more complicated sets over the course of the chapter.

Next we show the power of this integration theory by establishing
two results which allow us to interchange pointwise limits of functions
and with their integrals. We have already proved the following result
in Theorem \ref{thm:existenceUniquenessLebesgueIntegral}.
\begin{thm}[Monotone convergence theorem]
\label{thm:monotoneConvergenceLebInt}Let $\left\{ f_{n}\right\} _{n\in\N}\in\nonnegMeasurableFunctions$
be such that $f_{n}\leq f_{n+1}$ pointwise. Then, for any meaasure
$\mu$on $\F$,
\[
\lim_{n\to\infty}\lebInt{\mu}{f_{n}}=\lebInt{\mu}{\lim_{n\to\infty}f_{n}}.
\]
\end{thm}

The other theorem drops the requirement that the functions $f_{n}$
be non-negative or that the sequence be monotone, in exchange for
asking the sequence $f_{n}$ to be uniformly bounded by an integrable
function.
\begin{thm}[Dominated convergence theorem]
\label{thm:dominatedConvergenceLebInt}Let $\left\{ f_{n}\right\} _{n\in\N}$
be a sequence of functions in $\Lp 1{\X,\F,\mu}$ and suppose there
exists some $g\in\Lp 1{\mu}$ such that
\[
\left|f_{n}\right|\leq g
\]
pointwise for every $n\in\N$. Then, if $f:=\lim_{n\to\infty}f_{n}$
is defined, we have that
\[
\lim_{n\to\infty}\lebInt{\mu}{\left|f_{n}-f\right|}=0
\]
which also shows that
\[
\lim_{n\to\infty}\bar{\mu}\left(f_{n}\right)=\lebInt{\mu}f.
\]
\end{thm}

\begin{proof}
Note that by the continuity of the absolute value function, $\left|f_{n}\right|\leq g\Longrightarrow\left|f\right|\leq g$
pointwise which shows $f\in\Lp 1{\mu}$. Then, by the triangle inequality
\[
\left|f_{n}\left(x\right)-f\left(x\right)\right|\leq\left|f_{n}\left(x\right)\right|+\left|f\left(x\right)\right|\leq2g\left(x\right)
\]
for every $n\in\N$ and every $x\in\X.$ Note that
\[
\phi_{i}\left(x\right):=\sup_{n\geq i}\left|f_{n}\left(x\right)-f\left(x\right)\right|\leq2g\left(x\right)
\]
for every $i\in\N$ and every $x\in\X,$ which proves that $\phi_{i}\in\Lp 1{\mu}$
for every $i\in\N$. Next, observe that 
\begin{align*}
\lim_{i\to\infty}\phi_{i} & =\limsup_{n\to\infty}\left|f_{n}-f\right|\\
 & =\lim_{n\to\infty}\left|f_{n}-f\right|\\
 & =0
\end{align*}
where the second and third equalities follows from the assumption
that $f_{n}\longrightarrow f$ pointwise. Further, define
\[
\psi_{i}\left(x\right):=2g\left(x\right)-\phi_{i}\left(x\right)\geq0
\]
and observe that since $\phi_{i}\geq\phi_{i+1}$ pointwise for every
$i\in\N$, $\psi_{i}\leq\psi_{i+1}$ and $\lim_{i\to\infty}\psi_{i}=2g$.
Then, by the monotone convergence theorem
\begin{align*}
\lim_{i\to\infty}\lebInt{\mu}{\psi_{i}} & =\lebInt{\mu}{2g}.
\end{align*}
However, note that
\begin{align*}
\lebInt{\mu}{\psi_{i}} & =\lebInt{\mu}{2g-\phi_{i}}\\
 & =\lebInt{\mu}{2g}-\lebInt{\mu}{\phi_{i}}
\end{align*}
and so, since $\lebInt{\mu}{\psi_{i}},\lebInt{\mu}{2g}<\infty$ ,
\[
\lim_{i\to\infty}\lebInt{\mu}{\phi_{i}}=\lebInt{\mu}{2g}-\lim_{i\to\infty}\lebInt{\mu}{\psi_{i}}=0.
\]
Finally, note that $0\leq\lebInt{\mu}{\left|f_{n}-f\right|}\leq\lebInt{\mu}{\phi_{n}}$
for every $n\in\N$ by the monotonicty of $\bar{\mu}$ and so 
\[
\lim_{n\to\infty}\lebInt{\mu}{\left|f_{n}-f\right|}=0.
\]

By Corollary \ref{cor:triangleIneqLebIntL1},
\[
0\leq\lim_{n\to\infty}\left|\lebInt{\mu}{f_{n}-f}\right|\leq\lim_{n\to\infty}\lebInt{\mu}{\left|f_{n}-f\right|}=0
\]
which shows that 
\[
\lim_{n\to\infty}\lebInt{\mu}{f_{n}}=\lebInt{\mu}f.
\]
\end{proof}
We have shown results that establish interchanging limits and integrals
when such limits exist. The next result makes a more general statement
about interchanging limit inferiors of non-negative measurable functions
(which always exist in the extended real numbers) with their integrals,
although we cannot get strict equality.
\begin{thm}[Fatou's lemma]
\label{thm:fatouLemmaLebInt}For any functions $f_{n}\in\nonnegMeasurableFunctions$
\[
\lebInt{\mu}{\liminf_{n\to\infty}f_{n}}\leq\liminf_{n\to\infty}\lebInt{\mu}{f_{n}}.
\]
\end{thm}

\begin{proof}
Define $g_{n}\left(x\right)=\inf_{i\geq n}f_{i}\left(x\right).$ Clearly,
$g_{n}\leq f_{n}$ pointwise and
\[
\liminf_{n\to\infty}f_{n}=\lim_{n\to\infty}g_{n}\in\nonnegMeasurableFunctions
\]
by Corollary \ref{cor:limSupLimInfMeasurable}. Further, $g_{n}\leq g_{n+1}$
pointwise and so by the monotone convergence theorem
\[
\lebInt{\mu}{\liminf_{n\to\infty}f_{n}}=\lebInt{\mu}{\lim_{n\to\infty}g_{n}}=\lim_{n\to\infty}\lebInt{\mu}{g_{n}}=\liminf_{n\to\infty}\lebInt{\mu}{g_{n}}\leq\liminf_{n\to\infty}\lebInt{\mu}{f_{n}}
\]
where the fourth equality follows from the fact the when the limit
exists, limit superiors and inferiors are both equal to the limit.
\end{proof}
\begin{cor}
\label{cor:reverseFatouLemma}For any functions $f_{n}\in\nonnegMeasurableFunctions$
such that $f_{n}\leq g$ for some $g\in\nonnegMeasurableFunctions$
such that $\lebInt{\mu}g<\infty$
\[
\lebInt{\mu}{\limsup_{n\to\infty}f_{n}}\geq\limsup_{n\to\infty}\lebInt{\mu}{f_{n}}.
\]
\end{cor}

\begin{proof}
Define $h_{n}:=g-f_{n}\in\nonnegMeasurableFunctions$ and observe
that by Fatou's lemma
\[
\lebInt{\mu}{\liminf_{n\to\infty}h_{n}}\leq\liminf_{n\to\infty}\lebInt{\mu}{h_{n}}.
\]
Notice that we can rewrite the left hand side as 
\begin{align*}
\lebInt{\mu}{\liminf_{n\to\infty}h_{n}} & =\lebInt{\mu}{g+\liminf_{n\to\infty}-f_{n}}\\
 & =\lebInt{\mu}{g-\limsup_{n\to\infty}f_{n}}\\
 & =\lebInt{\mu}g-\lebInt{\mu}{\limsup_{n\to\infty}f_{n}}.
\end{align*}
Similarly, we can rewrite the right hand side as 
\[
\liminf_{n\to\infty}\lebInt{\mu}{h_{n}}=\lebInt{\mu}g-\limsup_{n\to\infty}\lebInt{\mu}{f_{n}}.
\]
Since $\lebInt{\mu}g<\infty$, we can subtract it from both sides
to yield the result.
\end{proof}
Note that Fatou's lemma is essentially a simple corollarly of the
monotone convergence theorem; it turns out that we can deduce the
monotone convergence theorem from Fatou's lemma as well, meaning that
the two theorems are in fact equivalent. Thus we could characterisze
Lebesgue integrals as linear functionals that satisfy Fatou's lemma
instead of montone convergence.
\begin{prop}
\label{prop:monotoneFromFatou}Let $\measurespace$ be a measure space
and let $\mu^{\prime}:\nonnegMeasurableFunctions\to\left[0,\infty\right]$
be a linear functional satisfying $\mu\left(A\right)=\mu^{\prime}\left(\indicate_{A}\right)$,
linearity and Fatou's lemma. Then for any increasing sequence$f_{n}\in\nonnegMeasurableFunctions$
\[
\mu^{\prime}\left(\lim_{n\to\infty}f_{n}\right)=\lim_{n\to\infty}\mu^{\prime}\left(f_{n}\right).
\]
\end{prop}

One important application of the dominated convergence theorem is
differentiating under the integral sign.
\begin{thm}[Differentiating under the integral sign]
\label{thm:diffIntSign}For any $\theta\in\left[-\delta,\delta\right],$
define $g\left(x,\theta\right)\in\Lp 1{\X,\F,\mu}$ to be differentiable
with respect to $\theta$, with derivative
\[
g_{\theta}\left(x,\theta\right)=\lim_{\epsilon\to0}\frac{g\left(x,\theta+\epsilon\right)-g\left(x,\theta\right)}{\epsilon}.
\]
Define the function
\[
m\left(\theta\right):=\lebInt{{\mu}_{x}}{g\left(x,\theta\right)}
\]
where the subscript $x$ clarifies the variable of integration. If
there exists a function $G\in\Lp 1{\mu}$ such that $\left|g_{\theta}\right|\leq G$
for every $\theta\in\left(-\delta,\delta\right)$ then
\[
\frac{dm\left(\theta\right)}{d\theta}=\frac{d\lebInt{{\mu}_{x}}{g\left(x,\theta\right)}}{d\theta}=\lebInt{{\mu}_{x}}{g_{\theta}\left(x,\theta\right)}.
\]
\end{thm}

\begin{proof}
Define the function 
\[
f_{n}\left(x,\theta\right):=\frac{g\left(x,\theta+\frac{1}{n}\right)-g\left(x,\theta\right)}{\frac{1}{n}}
\]
and observe by the linearity of the Lebesgue integral that 
\[
\frac{m\left(\theta+\frac{1}{n}\right)-m\left(\theta\right)}{\frac{1}{n}}=\lebInt{\mu_{x}}{f_{n}\left(x,\theta\right)}
\]
and that 
\[
\lim_{n\to\infty}f_{n}\left(x,\theta\right)=g_{\theta}\left(x,\theta\right).
\]
Next, observe that by the mean value theorem
\[
f_{n}\left(x,\theta\right)=g_{\theta}\left(x,\tilde{\theta}_{n}\right)
\]
for some $\theta<\tilde{\theta}_{n}<\theta+\frac{1}{n}$ for every
$n\in\N$, which implies that for large enough $n$
\[
\left|f_{n}\left(x,\theta\right)\right|=\left|g_{\theta}\left(x,\tilde{\theta}_{n}\right)\right|\leq G\left(x\right).
\]
Applying the dominated convergence theorem, we have that
\begin{align*}
\lebInt{{\mu}_{x}}{g_{\theta}\left(x,\theta\right)} & =\lebInt{{\mu}_{x}}{\lim_{n\to\infty}f_{n}\left(x,\theta\right)}\\
 & =\lim_{n\to\infty}\lebInt{{\mu}_{x}}{f_{n}\left(x,\theta\right)}\\
 & =\lim_{n\to\infty}\frac{m\left(\theta+\frac{1}{n}\right)-m\left(\theta\right)}{\frac{1}{n}}\\
 & =\frac{dm\left(\theta\right)}{d\theta}
\end{align*}
which completes the proof.
\end{proof}

\subsection{New measures from old and their integrals}
\begin{prop}
\label{prop:densities}Let $\left(\X,\F,\mu\right)$ be a measure
space. For any function $f\in\nonnegMeasurableFunctions$, the set
valued function $\nu:\F\longrightarrow\left[0,\infty\right]$ given
by
\[
\nu\left(A\right):=\lebInt{\mu}{f\indicate_{A}}
\]
is a measure on $\F$. Moreover, for any $A\in\F$, if $\mu\left(A\right)=0$
then $\nu\left(A\right)=0$.
\end{prop}

\begin{proof}
Note that $\nu\left(\emptyset\right)=$$\lebInt{\mu}{f\indicate_{\emptyset}}=\lebInt{\mu}{\indicate_{\emptyset}}=\mu\left(\emptyset\right)=0.$
Next, let $\left\{ A_{i}\right\} _{i\in\N}\in\F$ be disjoint and
define
\[
B_{n}=\bigcup_{i=1}^{n}A_{i}
\]
which is an increasing sequence of sets such that $\bigcup_{n\in\N}B_{n}=\bigcup_{i\in\N}A_{i}$.
Then
\begin{align*}
\nu\left(\bigcup_{i\in\N}A_{i}\right) & =\lebInt{\mu}{f\indicate_{\bigcup_{i\in\N}A_{i}}}\\
 & =\lebInt{\mu}{f\indicate_{\lim_{n\to\infty}B_{n}}}\\
 & =\lebInt{\mu}{f\lim_{n\to\infty}\indicate_{B_{n}}}\\
 & =\lebInt{\mu}{f\lim_{n\to\infty}\sum_{i=1}^{n}\indicate_{A_{i}}}\\
 & =\lim_{n\to\infty}\lebInt{\mu}{\sum_{i=1}^{n}f\indicate_{A_{i}}}\\
 & =\lim_{n\to\infty}\sum_{i=1}^{n}\lebInt{\mu}{f\indicate_{A_{i}}}\\
 & =\sum_{i=1}^{\infty}\lebInt{\mu}{f\indicate_{A_{i}}}\\
 & =\sum_{i=1}^{\infty}\nu\left(A_{i}\right)
\end{align*}
where the second equality follows from the discussion on convergence
of sets in Section 2.1, the third equality due to Proposition \ref{prop:limSupInfIndicator},
the fourth equality from induction on Fact \ref{fact:indicatorFunctionsFiniteOperations},
the fifth equality from the monotone convergence theorem, and the
sixth equality due to the linearity of $\bar{\mu}$. Finally, let
$A\in\F$ be a $\mu-$measure zero set and define $\left\{ s_{n}\right\} _{n\in\N}\in\mathcal{M}_{\textnormal{sim}}\left(\X,\F\right)$
to be an \hyperref[prop:simpleFunctionMonotoneConvergence]{increasing sequence of simple functions}
which converges to $f$ with standard representation 
\[
s_{n}=\sum_{i=1}^{I_{n}}a_{i.n}\indicate_{A_{i,n}}.
\]
Then the $s_{n}\indicate_{A}\nearrow f\indicate_{A}$and 
\begin{align*}
\lebInt{\mu}{s_{n}\indicate_{A}} & =\lebInt{\mu}{\sum_{i=1}^{I_{n}}a_{i,n}\indicate_{A_{i,n}\cap A}}\\
 & =\sum_{i=1}^{I_{n}}a_{i,n}\mu\left(A_{i,n}\cap A\right)\\
 & =0
\end{align*}
where the last equality follows from the monotonicty of measures.
Applying the monotone convergence theorem, we have that 
\begin{align*}
\nu\left(A\right) & =\lebInt{\mu}{f\indicate_{A}}\\
 & =\lim_{n\to\infty}\lebInt{\mu}{s_{n}\indicate_{A}}\\
 & =0
\end{align*}
which completes the proof.
\end{proof}
\begin{rem}
\label{rem:absoluteContinuity}A measure $\nu$ on $\F$ with the
relation $\mu\left(A\right)=0\Longrightarrow\nu\left(A\right)=0$
for every $A\in\F$ and a refence measure $\mu$ is called \emph{absolutely
continuous }with respect to $\mu$. This relation is denoted as $\nu<<\mu$
symbolically. We discuss absolute continuity in Chapter 6, where we
prove the converse of this theorem under a minor restriction. The
non-negative measurable function $f$ which generates the new measure
$\nu$ is called the \emph{density }of $\nu$ with respect to $\mu.$
If $\nu$ is a probability measure then $f$ is called the\emph{ probability
density function} of $\nu$ with respect to $\mu.$
\end{rem}

\begin{cor}
\label{cor:densityIntegral}Let $\mu,\nu$ be measures on $\measurablespace$
and let $f$ be a density of $\nu$ with respect to $\mu.$ The unique
Lebesgue integral associated with the measure $\nu$ is given by 
\[
\lebInt{\nu}g=\lebInt{\mu}{fg},
\]
for any function $g\in\measurableFunctions$, provided the right-hand
side is defined.
\end{cor}

\begin{proof}
Note that for any set $A\in\F$, $\lebInt{\nu}{\indicate_{A}}=\lebInt{\mu}{f\indicate_{A}}=\nu\left(A\right)$.
Further, $\bar{\nu}$ inherits linearity and monotone convergence
on $\nonnegMeasurableFunctions$ from $\bar{\mu}$ and so by Theorem
\ref{thm:existenceUniquenessLebesgueIntegral}, $\bar{\nu}$ is the
unique integral (with respect to measure $\nu$) on $\nonnegMeasurableFunctions,$
which of course extends uniquely to $\measurableFunctions$ through
Definition \ref{def:integral}.
\end{proof}
\begin{prop}
\label{prop:imageMeasures}Let $\left(\X,\F,\mu\right)$ be a measure
space. For any function $f\in\measurableFunctions$, the function
$f\mu:\borel\left(\R\right)\longrightarrow\left[0,\infty\right]$
given by 
\[
f\mu\left(B\right):=\mu\left(f^{-1}\left[B\right]\right)
\]
is a measure on $\borel\left(\R\right).$
\end{prop}

\begin{proof}
Note that $f\mu\left(\emptyset\right)=\mu\left(f^{-1}\left[\emptyset\right]\right)=\mu\left(\emptyset\right).$
Next, let $\left\{ B_{i}\right\} _{i\in\N}\in\borel\left(\R\right)$
be disjoint and observe that
\begin{align*}
f\mu\left(\bigcup_{i\in\N}B_{i}\right) & =\mu\left(f^{-1}\left[\bigcup_{i\in\N}B_{i}\right]\right)\\
 & =\mu\left(\bigcup_{i\in\N}f^{-1}\left[B_{i}\right]\right)\\
 & =\sum_{i=1}^{\infty}\mu\left(f^{-1}\left[B_{i}\right]\right)\\
 & =\sum_{i=1}^{\infty}f\mu\left(B_{i}\right)
\end{align*}
where the second equality is a property of inverse maps and the third
equality is due to the countable additivity of $\mu$.
\end{proof}
\begin{rem*}
Here the measure $f\mu$ is called the \emph{image measure }generated
by $f$ via $\mu.$ If $\mu$ is a probability measure, then $f\mu$
is called the \emph{probability distribution }of $f$.
\end{rem*}
\begin{cor}
\label{cor:changeOfVariables}Let $\left(\X,\F,\mu\right)$ be a measure
space and define $f\mu$ to be the image measure of a function $f\in\nonnegMeasurableFunctions$
with respect to measure $\mu.$ Then, for any function $g\in\mathcal{M}\left(\R,\borel\left(\R\right)\right)$,
the Lebesgue integral associated with measaure $f\mu$ is given by
\[
\lebInt{f\mu}g:=\lebInt{\mu}{g\circ f}
\]
provided the right hand side is defined.
\end{cor}

\begin{proof}
First note that for any $B\in\borel\left(\R\right)$
\begin{align*}
\lebInt{f\mu}{\indicate_{B}} & =\lebInt{\mu}{\indicate_{B}\circ f}\\
 & =\lebInt{\mu}{\indicate_{f^{-1}\left[B\right]}}\\
 & =\mu\left(f^{-1}\left[B\right]\right)\\
 & =f\mu\left(B\right)
\end{align*}
which satisfies the first required property of Theorem \ref{thm:existenceUniquenessLebesgueIntegral}.
Linearity and monotone convergence follow from $\bar{\mu}$ and so
the uniqueness criterion of the theorem tells us that $\bar{f\mu}$
is indeed the unique integral associated with the image measure $f\mu$.
\end{proof}
Recall from Proposition(\ref{prop:sumOfCountableMeasures}) that the
countable sum of measures on $\left(\X,\F\right)$ is a measure on
$\left(\X,\F\right)$. The integral resulting from this compound measure
can be decomposed into the integrals from constituent summand measures
in certain situations.
\begin{prop}
\label{prop:integralSumOfMeasures}Let $\left(\X,\F\right)$ be a
measurable space and let $\left\{ \mu_{i}\right\} _{i\in\N}$ be a
countable collection of measures on $\F$ with their respective integrals
$\left\{ \lebInt{\mu_{i}}{\cdot}\right\} _{i\in\N}$. Then the integral
associated with the sum measure
\[
\mu:=\sum_{i=1}^{\infty}\mu_{i}
\]
is given by
\[
\lebInt{\mu}f:=\sum_{i=1}^{\infty}\lebInt{\mu}f
\]
for any $f\in\nonnegMeasurableFunctions.$
\end{prop}

\begin{proof}
We first show this holds for $f=\indicate_{A}$for any $A\in\F$;
indeed, an application of Proposition \ref{prop:sumOfCountableMeasures}
is sufficient for this purpose. Next, for any $s\in M_{\textnormal{sim}}\left(\X,\mathcal{F}\right)$,
we have that
\[
s=\sum_{i=1}^{I}\alpha_{i}\indicate_{A_{i}}
\]
where $\alpha_{i}>0$ , $A_{i}:=\left\{ x\in\X:s\left(x\right)=\alpha_{i}\right\} $,
and $I\in\N$. Thus,
\begin{align*}
\lebInt{\mu}s & =\sum_{i=1}^{I}\alpha_{i}\mu\left(A_{i}\right)\\
 & =\sum_{i=1}^{I}\alpha_{i}\sum_{j=1}^{\infty}\mu_{j}\left(A_{i}\right)\\
 & =\sum_{j=1}^{\infty}\sum_{i=1}^{I}\mu_{j}\left(A_{i}\right)\\
 & =\sum_{j=1}^{\infty}\lebInt{\mu_{j}}s
\end{align*}
where the first equality is by the definition of integrals on simple
functions and the third is by the linearity of limits of sequences.
This establishes the result for simple functions. Finally, let $f\in\nonnegMeasurableFunctions$
be arbitrary and observe by Proposition \ref{prop:simpleFunctionMonotoneConvergence}
that there exists some increasing sequence $\left\{ s_{n}\right\} \in M_{\textnormal{sim}}\left(\X,\mathcal{F}\right)$
such that $s_{n}\nearrow f$ and 
\begin{align*}
\lebInt{\mu}f & =\lebInt{\mu}{\lim_{n\to\infty}s_{n}}\\
 & =\lim_{n\to\infty}\lebInt{\mu}{s_{n}}\\
 & =\lim_{n\to\infty}\sum_{i=1}^{\infty}\lebInt{\mu_{i}}{s_{n}}\\
 & =\sum_{i=1}^{\infty}\lebInt{\mu_{i}}{\lim_{n\to\infty}s_{n}}\\
 & =\sum_{i=1}^{\infty}\lebInt{\mu_{i}}f
\end{align*}
where the second equality follows by monotone convergence and the
fourth equality follows by two applications of monotone convergence
(you can think of the countable sum as integration with respect to
the counting measure). This completes the proof.
\end{proof}


\subsection{Equivalence of integrals and measures}

Note that while in Theorem \ref{thm:existenceUniquenessLebesgueIntegral}
we constructed the integral from a seemingly more primitive concept
of a measure, it turns out that a measure can be constructed out a
linear functional on the space of measurable functions such that the
functional is the integral with respect to the measure we have constructed.
In this sense, measures and integrals are really equivalent. At this
point, this may seem to be a trivial observation (and the proof of
this result is indeed trivial); this shift of perspective, however,
offers powerful simplifications to questions concerning the existence
and uniqueness of measures. Indeed, our approach to product measures
in Chapter 6 would rely on this insight, allowing us to prove both
the existence of product measures, and the representation of integrals
with respect to product measures as iterated integrals, in a single
stroke.
\begin{thm}
\label{thm:integralMeasureEquivalence}Let $\left(\X,\F\right)$ be
a measurable space and let 
\[
\Lambda:\nonnegMeasurableFunctions\longrightarrow\left[0,\infty\right]
\]
be a linear functional that satisfies monotone convergence i.e for
any sequence $f_{n}\in\measurableFunctions$ such that $f_{n}\leq f_{n+1}$
and $f:=\lim_{n\to\infty}f_{n}\in\F$ we have 
\[
\Lambda\left(f\right)=\lim_{n\to\infty}\Lambda\left(f_{n}\right).
\]
Further, suppose there exists a function $g\in\nonnegMeasurableFunctions$
such that $\Lambda\left(g\right)<\infty$. Then the function 
\[
\lambda:\F\longrightarrow\left[0,\infty\right]
\]
given by 
\[
\lambda\left(A\right):=\Lambda\left(\indicate_{A}\right)
\]
for any $A\in\F$ is a measure and $\Lambda$ is the integral with
respect to $\lambda$.
\end{thm}

\begin{proof}
First, note that 
\begin{align*}
\Lambda\left(g\right) & =\Lambda\left(g+\indicate_{\emptyset}\right)\\
 & =\Lambda\left(g\right)+\Lambda\left(\indicate_{\emptyset}\right)\\
 & =\Lambda\left(g\right)+\lambda\left(\emptyset\right)
\end{align*}
by linearity and since $\Lambda\left(g\right)<\infty$, we can subtract
it from both sides to deduce $\lambda\left(\emptyset\right)=0.$ To
establish countable additivity, note that for a disjoint collection
$A_{n}\in\F$, 
\begin{align*}
\lambda\left(\bigcup_{n\in\N}A_{n}\right) & =\Lambda\left(\sum_{n=1}^{\infty}\indicate_{A_{n}}\right)\\
 & =\sum_{n=1}^{\infty}\Lambda\left(\indicate_{A_{n}}\right)\\
 & =\sum_{n=1}^{\infty}\lambda\left(A_{n}\right)
\end{align*}
where the first equality is by (an extension of) Proposition \ref{prop:indicatorFunctionsArbitraryOperations}
and the second by linearity and monotone convergence. Of course, since
$\Lambda$ satisfies the properties of Theorem \ref{thm:existenceUniquenessLebesgueIntegral},
it is the unique integral induced by the measure $\lambda$; this
completes the proof.
\end{proof}

\section{Null sets}
\begin{defn}
\label{def:nullSet}Let $\left(\X,\F,\mu\right)$ be a measure space.
A set $A\in\F$ is called a $\mu-$null set if it's a measure zero
set with respect to $\mu$ i.e.
\[
\mu\left(A\right)=0.
\]
The collection of all $\mu-$null sets in $\F$ is denoted $N_{\mu}$.
\end{defn}

Note we will often omit the ``$\mu$'' when describing a $\mu-$null
set and simply say ``null set'' if the measure is clear from context.
\begin{prop}
\label{prop:nullClosureCountableUnion}Let $\left(\X,\F,\mu\right)$
be a measure space. The set of all null sets $N_{\mu}$ is closed
under countable unions.
\end{prop}

\begin{proof}
Let $\left\{ A_{i}\right\} _{i\in\N}\in N_{\mu}$ be arbitrary. Then,
by \hyperref[cor:countableSubadditivity]{countable subadditivity}
\begin{align*}
\mu\left(\bigcup_{i\in\N}A_{i}\right) & \leq\sum_{i=1}^{\infty}\mu\left(A_{i}\right)\\
 & =\lim_{n\to\infty}\sum_{i=1}^{n}\mu\left(A_{i}\right)\\
 & =0.
\end{align*}
\end{proof}
\begin{prop}
\label{prop:intZeroFuncZero}For any function $f\in\nonnegMeasurableFunctions$
and a measure $\mu$ on $\F$, if the integral $\lebInt{\mu}f=0$
then the set $\left\{ x\in\X\mid f\left(x\right)>0\right\} \in N_{\mu}.$
\end{prop}

\begin{proof}
Observe that for any $n\in\N$ we have the pointwise inequality
\[
h_{n}:=\indicate_{\left\{ f\geq\frac{1}{n}\right\} }\leq nf
\]
where both sides are measurable since $f$ is Borel-measurable. By
the monotonicity and linearity of the integral, we have 
\[
\mu\left(\left\{ x\in\X\mid f\left(x\right)\geq\frac{1}{n}\right\} \right)=\lebInt{\mu}{h_{n}}\leq n\lebInt{\mu}f=0.
\]
Since $f\left(x\right)\geq\frac{1}{n}\Longrightarrow f\left(x\right)\geq\frac{1}{n+1}$
and $\bigcup_{n\in\N}\left\{ x\in\X\mid f\left(x\right)\geq\frac{1}{n}\right\} =\left\{ x\in\X\mid f\left(x\right)>0\right\} ,$
by the \hyperref[prop:measureProperties]{continuity from below of measures}
(or, equivalently, the monotone convergence theorem)
\[
\mu\left(\left\{ x\in\X\mid f\left(x\right)>0\right\} \right)=\lim_{n\to\infty}\mu\left(\left\{ x\in\X\mid f\left(x\right)\geq\frac{1}{n}\right\} \right)=0
\]
which completes the proof.
\end{proof}
\begin{rem*}
When two function $f,g\in\measurableFunctions$ are such that $f=g$
on $\X\setminus A$ for some $A\in N_{\mu}$ , we say that the functions
are equal $\mu-$almost everywhere. In the literature, this is often
shortened to writing $\mu-$a.e or simply a.e if the measure is clear
from context. If $\mu$ is a probability measure, then we say $f=g$
almost surely, which is often shortened to a.s in the literature.\footnote{Sometimes, this is denoted even more compactly as $f\stackrel{\textnormal{a.e}}{=}g.$}
From this it's clear that the set $\left\{ x\in\X\mid f\left(x\right)\neq g\left(x\right)\right\} \subseteq A.$
Of course, if we can show that $\left\{ x\in\X\mid f\left(x\right)\neq g\left(x\right)\right\} \in\F$
then $\left\{ x\in\X\mid f\left(x\right)\neq g\left(x\right)\right\} \in N_{\mu}$
by the monotonicity of measures.

Corresponding to the notion of almost-everywhere equality of measurable
functions, there's a notion of almost-everywhere equality of sets.
Recall from set theory that the symmetric difference of two sets $A,B$
is given
\[
A\Delta B:=\left(A\setminus B\right)\bigcup\left(B\setminus A\right)
\]
Given a measure $\mu$on some measurable space $\measurablespace$
and sets $A,B\in\F$, we say $A\stackrel{\mu-\text{a.e}}{=}B$ if
$\mu\left(A\Delta B\right)=0.$ Of course, in the spirit of Theorem
\ref{thm:integralMeasureEquivalence}, this definition must coincide
with the one for measurable functions in that we must have that $\indicate_{A}\stackrel{\mu-\text{a.e}}{=}\indicate_{B}$.
This is easily shown in the following result.
\end{rem*}
\begin{prop}
\label{prop:almostEverywhereEqualSets}Let $\measurespace$ be a measure
space. For any $A,B\in\F$, $A\stackrel{\mu-\text{a.e}}{=}B$ if and
only if $\indicate_{A}\stackrel{\mu-\text{a.e}}{=}\indicate_{B}$
\end{prop}

\begin{proof}
Note that $\mu\left(A\Delta B\right)=0\Longleftrightarrow\mu\left(A\cap B^{C}\right)=\mu\left(B\cap A^{C}\right)=0.$
Further, observe that
\[
\left\{ x\in\X\mid\indicate_{A}\left(x\right)\neq\indicate_{B}\left(x\right)\right\} =\underbrace{\left\{ x\in\X\mid\indicate_{A}\left(x\right)>\indicate_{B}\left(x\right)\right\} }_{=A\cap B^{C}}\bigcup\underbrace{\left\{ x\in\X\mid\indicate_{A}\left(x\right)<\indicate_{B}\left(x\right)\right\} }_{=B\cap A^{C}}
\]
which completes the proof.
\end{proof}
\begin{lem}
\label{lem:setFuncNotEqualMeasurable}Let $f,g\in\measurableFunctions$
be arbitrary. Then, the set
\[
A:=\left\{ x\in\X\mid f\left(x\right)\neq g\left(x\right)\right\} 
\]
is measurable i.e. $A\in\F.$
\end{lem}

\begin{proof}
Let $h:=\left(f-g\right)\indicate_{\left\{ f=g=\infty\textnormal{or}f=g=-\infty\right\} ^{C}}$and
observe that $h$ is measurable by Corollary \ref{cor:examplesBinaryOpsMeasFunc}.
Further, $A\subseteq\left\{ x\in\X\mid f\left(x\right)=g\left(x\right)=\infty\textnormal{ or }f\left(x\right)=g\left(x\right)=-\infty\right\} ^{C}$
and so
\[
A=\left\{ x\in\X\mid h\left(x\right)\neq0\right\} =\left\{ x\in\X\mid h\left(x\right)>0\right\} \bigcup\left\{ x\in\X\mid h\left(x\right)<0\right\} 
\]
is in $\F$ by the measurability of $h$.
\end{proof}
\begin{prop}
\label{prop:intFiniteFuncFinite}For any function $f\in\nonnegMeasurableFunctions$
and a measure $\mu$ on $\F$, if the integral $\lebInt{\mu}f<\infty$
then $f<\infty$ $\mu-$almost everywhere.
\end{prop}

\begin{proof}
Observe that the pointwise equality
\[
\indicate_{\left\{ f=\infty\right\} }\leq\frac{f}{n}
\]
where both sides are measurable since $f$ is Borel-measurable. By
the monotonicity and linearity of the integral and our assumptions,
\[
0\leq\mu\left(\left\{ x\in\X\mid f\left(x\right)=\infty\right\} \right)\leq\frac{1}{n}\lebInt{\mu}f<\infty.
\]
Since weak inequalities are preserved under limits, we have that 
\[
0\leq\mu\left(\left\{ x\in\X\mid f\left(x\right)=\infty\right\} \right)\leq\lim_{n\to\infty}\frac{1}{n}\lebInt{\mu}f=0
\]
which completes the proof.
\end{proof}
\begin{prop}
\label{prop:funcEqualityAlmostEverywhere}Let $f,g\in\nonnegMeasurableFunctions$
be arbitrary and let $\mu$be a measure on $\F$. If $f=g$ $\mu-$almost
everywhere, then 
\[
\lebInt{\mu}f=\lebInt{\mu}g.
\]
\end{prop}

\begin{proof}
Define the set 
\[
A:=\left\{ x\in\X\mid f\left(x\right)\neq g\left(x\right)\right\} 
\]
which has measure zero by Lemma \ref{lem:setFuncNotEqualMeasurable}
and our assumption. Then, I claim that the pointwise inequality
\[
\min\left\{ g,n\right\} \leq n\indicate_{A}+f
\]
holds for every $n\in\N.$ To see this, we can look at the following
cases:

\begin{enumerate}

\item$g\left(x\right)>n$ and $g\left(x\right)\neq f\left(x\right)$:
In this case, we see that the inequality resolves to 
\[
n\leq n+f\left(x\right)
\]
which is true by the non-negativity of $f.$

\item$g\left(x\right)\leq n$ and $g\left(x\right)\neq f\left(x\right):$
In this case, the inequality resolves to 
\[
g\left(x\right)\leq n+f\left(x\right)
\]
which is again true by the non-negativity of $f.$

\item$g\left(x\right)>n$ and $g\left(x\right)=f\left(x\right):$
In this case, the inequality resolves to 
\[
n\leq f\left(x\right)
\]
which is true since $f\left(x\right)=g\left(x\right).$

\item$g\left(x\right)\leq n$ and $g\left(x\right)=f\left(x\right)$:
In this case, the inequality resolves to 
\[
g\left(x\right)\leq f\left(x\right)
\]
which is true by assumption.

\end{enumerate}

Note that both the left and right hand side are measurable by Proposition
\ref{prop:binaryOperationsMeasurableFunctions}and so, integrating
both sides, we have
\begin{align*}
\lebInt{\mu}{\min\left\{ g,n\right\} } & \leq n\mu\left(A\right)+\lebInt{\mu}f\\
 & =\lebInt{\mu}f
\end{align*}
since $\mu\left(A\right)=0.$ Finally, observe that $\min\left\{ g,n\right\} \leq\min\left\{ g,n+1\right\} $
and that $\lim_{n\to\infty}\min\left\{ g,n\right\} =g.$ Then, applying
the monotone convergence theorem, we have 
\[
\lebInt{\mu}g=\lim_{n\to\infty}\lebInt{\mu}{\min\left\{ g,n\right\} }\leq\lebInt{\mu}f.
\]
We can deduce the reverse inequality with the analagous pointwise
inequality
\[
\min\left\{ f,n\right\} \leq n\indicate_{A}+g
\]
which proves the result.
\end{proof}
\begin{cor}
\label{cor:LPfuncEqualityAlmostEverywhere}Let $f,g\in L^{1}\left(\X,\F,\mu\right)$
be such that they are equal almost everywhere. Then
\[
\lebInt{\mu}f=\lebInt{\mu}g.
\]
\end{cor}

\begin{proof}
If $f=g$ a.e then
\[
f^{+}-f^{-}\stackrel{\textnormal{a.e}}{=}g^{+}-g^{-}\Longleftrightarrow f^{+}+g^{-}\stackrel{\textnormal{a.e}}{=}g^{+}+f^{-}
\]
and so, integrating both sides and applying Proposition \ref{prop:funcEqualityAlmostEverywhere}
we have
\[
\lebInt{\mu}{f^{+}+g^{-}}=\lebInt{\mu}{g^{+}+f^{-}}.
\]
Applying linearity and observing that the integral of each component
is finite by \ref{prop:L1conditions}, we get 
\[
\lebInt{\mu}f=\lebInt{\mu}{f^{+}}-\lebInt{\mu}{f^{-}}=\lebInt{\mu}{g^{+}}-\lebInt{\mu}{g^{-}}=\lebInt{\mu}g
\]
as desired.
\end{proof}
\begin{prop}
\label{prop:intEqualFuncEqual} Let $f,g\in\nonnegMeasurableFunctions$
such and let $\mu$ be a measure on $\F$ such that 
\[
\lebInt{\mu}{f\indicate_{F}}=\lebInt{\mu}{g\indicate_{F}}
\]
for every $F\in\F$. Then
\[
f=g
\]
$\mu-$almost everywhere.
\end{prop}

\begin{proof}
Assume the hypothesis is true and define
\[
A:=\left\{ x\in\X\mid f\left(x\right)\neq g\left(x\right)\right\} =\underbrace{\left\{ x\in\X\mid f\left(x\right)>g\left(x\right)\right\} }_{A_{1}}\bigcup\underbrace{\left\{ x\in\X\mid f\left(x\right)<g\left(x\right)\right\} }_{A_{2}}
\]
and suppose for contradiction that $\mu\left(A\right)>0.$ Then one
of $A_{1}$ or $A_{2}$ has positive measure. Assume, without loss
of generality, that $\mu\left(A_{1}\right)>0$ and further define
\[
A_{1,n}:=\left\{ x\in\X\mid f\left(x\right)-\frac{1}{n}\geq g\left(x\right)\right\} =\left\{ x\in\X\mid z\left(x\right)\geq\frac{1}{n}\right\} 
\]
where $z=f-g$ and so is measurable, which in turn imples that $A_{1,n}\in\F$
for all $i\in\N.$ Further, $A_{1,n}\subseteq A_{1,n+1}$ and $\bigcup_{n\in\N}A_{1,n}=A$,
and so by \hyperref[prop:measureProperties]{continuity from below}
\[
\lim_{n\to\infty}\mu\left(A_{1,n}\right)=\mu\left(A_{1}\right)>0.
\]
By the definition of limts, there exists some $n_{0}\in\N$ such that
\[
\mu\left(A_{1,n}\right)>0\ \forall n\geq n_{0}
\]
and for such $n$ we also have, by the monotonicty of integration
\[
\lebInt{\mu}{z\indicate_{A_{1,n}}}\geq\lebInt{\mu}{\frac{1}{n}\indicate_{A_{1,n}}}=\frac{1}{n}\mu\left(A_{1,n}\right)>0
\]
which implies, by the linearity of integration, that
\[
\lebInt{\mu}{f\indicate_{A_{1,n}}}>\lebInt{\mu}{g\indicate_{A_{1,n}}}
\]
which contradicts our hypotehsis. This completes the proof.
\end{proof}
\begin{defn}[Convergence almost everywhere]
\label{def:convAlmostEverywhere}Let $\left(\X,\F,\mu\right)$ be
a measure space and define $\left\{ f_{n}\right\} _{n\in\N}\in\measurableFunctions$
to be a sequence of functions. The sequence is $f_{n}$ is said to
\emph{converge almost everywhere} to a function $f\in\measurableFunctions$
if 
\[
\mu\left(\left\{ x\in\X\mid\lim_{n\to\infty}f_{n}\left(x\right)\neq f\left(x\right)\right\} \right)=0.
\]
In this case, we write
\[
\lim_{n\to\infty}f_{n}\stackrel{\textnormal{a.e}}{=}f
\]
or say $f_{n}\longrightarrow f$ $\mu-$a.e (or $f_{n}\stackrel{\textnormal{a.e}}{\longrightarrow}f$).
\end{defn}

\begin{thm}[Generalized monotone convergence theorem]
\label{thm:generalizedMonotoneConvergence} Let $\left\{ f_{n}\right\} _{n\in\N}\in\nonnegMeasurableFunctions$
be such that $f_{n}\leq f_{n+1}$ $\mu-$almost everyhwere; that is,
there is some $A\in N_{\mu}$ such that $f_{n}\left(x\right)\leq f_{n+1}\left(x\right)$
for all $x\in\X\setminus A.$ Then if 
\[
\lim_{n\to\infty}f_{n}=f
\]
 on $\X\setminus A$, we have that
\[
\lim_{n\to\infty}\lebInt{\mu}{f_{n}}=\lebInt{\mu}f.
\]
\end{thm}

\begin{proof}
Define $g_{n}=f_{n}\indicate_{\X\setminus A}$ and observe that 
\[
g_{n}\leq g_{n+1}
\]
pointwise for all $n\in\N$and 
\[
\lim_{n\to\infty}g_{n}=f\indicate_{\X\setminus A}.
\]
By the standard \hyperref[thm:monotoneConvergenceLebInt]{monotone convergence theorem},
\[
\lim_{n\to\infty}\lebInt{\mu}{g_{n}}=\lebInt{\mu}{f\indicate_{\X\setminus A}}.
\]
But note that since $g_{n}\stackrel{\textnormal{a.e}}{=}f_{n}$ and
$f\stackrel{\textnormal{a.e}}{=}f\indicate_{\X\setminus A}$, by Proposition
\ref{prop:funcEqualityAlmostEverywhere},
\[
\lim_{n\to\infty}\lebInt{\mu}{f_{n}}=\lim_{n\to\infty}\lebInt{\mu}{g_{n}}=\lebInt{\mu}{f\indicate_{\X\setminus A}}=\lebInt{\mu}f
\]
which completes the proof.
\end{proof}
We can similarly strengthen the dominated convergence theorem.
\begin{thm}[Generalized dominated convergence theorem]
\label{thm:generalizedDominatedConvergence}Let $\left\{ f_{n}\right\} _{n}\in\Lp 1{\X,\F,\mu}$
and suppose there exists some $g\in\Lp 1{\mu}$ be such that
\[
\lvert f_{n}\rvert\leq g
\]
$\mu-$almost everywhere. Then, if there exists some $f\in\mathcal{M}\left(\X,\F\right)$
such that 
\[
\lim_{n\to\infty}f_{n}\stackrel{\textnormal{a.e}}{=}f
\]
we have that $f\in\Lp 1{\mu}$ and
\[
\lim_{n\to\infty}\lebInt{\mu}{\lvert f_{n}-f\rvert}=0
\]
and 
\[
\lim_{n\to\infty}\lebInt{\mu}{f_{n}}=\lebInt{\mu}f.
\]
\end{thm}

\begin{proof}
Without loss of generality\footnote{This can be justfied by Proposition \ref{prop:nullClosureCountableUnion}},
assume that there exists some set $A\in N_{\mu}$ such that 
\[
\lvert f_{n}\left(x\right)\rvert\leq g\left(x\right)
\]
for all $x\in\X\setminus A$ and every $n\in\N$, and that
\[
\lim_{n\to\infty}f_{n}\left(x\right)=f\left(x\right)
\]
for all $x\in\X\setminus A.$ Then, consider the functions $h_{n}:=f_{n}\indicate_{\X\setminus A}$and
observe that $g^{*}:=g\indicate_{\X\setminus A}\in\Lp 1{\mu}$ since
$g^{*}\stackrel{\text{a.e}}{=}g\Longrightarrow\lvert g^{*}\rvert\stackrel{\text{a.e}}{=}\lvert g\rvert\Longrightarrow\lebInt{\mu}{\lvert g^{*}\rvert}=\lebInt{\mu}{\lvert g\rvert}<\infty.$
Next, note that 
\[
\lvert h_{n}\rvert\leq g^{*}
\]
everywhere for each $n\in\N$ and 
\[
\lim_{n\to\infty}h_{n}=f\indicate_{\X\setminus A}
\]
pointwise and so, applying the usual \hyperref[thm:dominatedConvergenceLebInt]{dominated convergence theorem}
\[
\lim_{n\to\infty}\lebInt{\mu}{h_{n}}=\lebInt{\mu}{f\indicate_{\X\setminus A}}.
\]
As, before, by Corollary \ref{cor:LPfuncEqualityAlmostEverywhere}
\[
\lim_{n\to\infty}\lebInt{\mu}{f_{n}}=\lim_{n\to\infty}\lebInt{\mu}{h_{n}}=\lebInt{\mu}{f\indicate_{\X\setminus A}}=\lebInt{\mu}f.
\]
\end{proof}
In the context of our discussion on the equivalence between integrals
and measures, we had foreshadowed how the monotone convergence theorem
and the continuity from below of measures were basically the same
concept. We had also said that the \hyperref[thm:borelCantelli]{Borel-Cantelli lemma}
could be understood through from both an integration and measure-theoretic
perspective. We can make this precise with the following result.
\begin{thm}[Generalized Borel-Cantelli lemma]
\label{thm:generalizedBorelCantelli}Let $\left\{ f_{n}\right\} _{n\in\N}\in\nonnegMeasurableFunctions$
and let $\mu$ be measure on $\F$. If 
\[
\sum_{n=1}^{\infty}\lebInt{\mu}{f_{n}}<\infty
\]
then
\[
\sum_{n=1}^{\infty}f_{n}<\infty
\]
$\mu-$almost everywhere.
\end{thm}

\begin{proof}
Note that 
\begin{align*}
\lebInt{\mu}{\sum_{n=1}^{\infty}f_{n}} & =\lebInt{\mu}{\lim_{N\to\infty}\sum_{n=1}^{N}f_{n}}\\
 & =\lim_{N\to\infty}\lebInt{\mu}{\sum_{n=1}^{N}f_{n}}\\
 & =\sum_{n=1}^{\infty}\lebInt{\mu}{f_{n}}<\infty
\end{align*}
where the second equality follows from the monotone convergence theorem
and the third equality due to the linearity of the integral. By Proposition
\ref{prop:intFiniteFuncFinite}
\[
\sum_{n=1}^{\infty}f_{n}<\infty
\]
$\mu-$almost everywhere.
\end{proof}
It should be clear that we can recover our original Borel-Cantelli
lemma by letting $f_{n}=\indicate_{A_{n}}$for sets $\left\{ A_{n}\right\} _{n\in\N}\in\F.$

\section{Convergence of measurable functions}

So far, we have discussed two modes of convergence for measurable
functions explicitly: pointwise convergence and \hyperref[def:convAlmostEverywhere]{almost-everywhere convergence},
the latter of which is implied by the former. We have also implicitly
defined another type of convergence through the \hyperref[thm:dominatedConvergenceLebInt]{dominated convergence theorem}.
We can make this explicit with the following definition
\begin{defn}[Convergence in $L^{1}$]
\label{def:L1Convergence}A sequence of functions $\left\{ f_{n}\right\} _{n\in\N}\in\Lp 1{\X,\F,\mu}$
is said to converge to a function $f\in\Lp 1{\X,\F,\mu}$ in $\mathcal{L}^{1}$
if 
\[
\lim_{n\to\infty}\lebInt{\mu}{\lvert f_{n}-f\rvert}=0.
\]
In this case, we write 
\[
f_{n}\stackrel{\mathcal{L}^{1}}{\longrightarrow}f.
\]
\end{defn}

\begin{defn}[Convergence in measure]
\label{def:convergenceInMeasure}A sequence of functions $\left\{ f_{n}\right\} _{n\in\N}\in\measurableFunctions$
is said to \emph{converge in measure }with respect to measure $\mu$
on $\F$ if for every $\epsilon>0$
\[
\lim_{n\to\infty}\mu\left(\left\{ x\in\X\mid\lvert f_{n}\left(x\right)-f\left(x\right)\rvert>\epsilon\right\} \right)=0.
\]
In this case, we write
\[
f_{n}\stackrel{\mu}{\longrightarrow}f.
\]

Immediately, we would like to know if these ``limits'' are well
behaved in some sense. That is, we woud like to ensure that they satisfy
some basic properties that we expect limits to satisfy.
\end{defn}

\begin{prop}
\label{prop:convFconvAbsFnMinusF}Let $(\X,\F,\mu)$ be a measure
space and let $\left\{ f_{n}\right\} _{n\in\N}\in\measurableFunctions$
converge to $f\in\measurableFunctions$ in any of the modes described
earlier. Then $\lvert f_{n}-f\rvert\to0$ in the same mode.
\end{prop}

\begin{proof}
We first show this for almost everywhere convergence. Suppose $f_{n}\stackrel{\text{a.e}}{\longrightarrow}f$
and consider the function $g_{n}:=\lvert f_{n}-f\rvert$. Note that
if for any $x\in\X$
\[
\lim_{n\to\infty}g_{n}(x)=0\Longleftrightarrow\lim_{n\to\infty}f_{n}(x)=f(x)
\]
by the definition of convergence of sequences. The result then follows
by observing that the points where this does not occur are identical
(and thus so are their measures). The result for the other two modes
are trivial.
\end{proof}
\begin{prop}[Linearity of convergence]
\label{prop:linearityConvergence}Let $(\X,\F,\mu)$ be a measure
space and let $\left\{ f_{n}\right\} _{n\in\N},\left\{ g_{n}\right\} \in\measurableFunctions$
converge to $f,g\in\measurableFunctions$ respectively in any of the
modes described earlier. For any $c\in\R$
\[
cf_{n}+g_{n}\to cf+g
\]
in the same mode of convergence.
\end{prop}

\begin{proof}
In the case of almost sure convergence, let the null sets where $\lim_{n\to\infty}f_{n}(x)\neq f(x)$
and $\lim_{n\to\infty}g_{n}(x)\neq g(x)$ be $N_{f}$ and $N_{g}$
respectively. Then $N=N_{f}\cup N_{g}$ is a null set and so
\[
\lim_{n\to\infty}cf_{n}+g_{n}=cf+g
\]
on $N^{C}$ which establishes linearity.

In the case of convergence in measure, fix $\epsilon>0$ and observe
that 
\begin{align*}
\left\{ x\in\X\mid\lvert cf_{n}(x)+g(x)-cf(x)-g(x)\rvert>\epsilon\right\}  & \subseteq\left\{ x\in\X\mid\lvert c\rvert\lvert f_{n}(x)-f(x)\rvert+\lvert g_{n}(x)-g(x)\rvert>\epsilon\right\} \\
 & \subseteq\left\{ x\in\X\mid\lvert f_{n}(x)-f(x)\rvert>\frac{\epsilon}{2c}\right\} \cup\left\{ x\in\X\mid\lvert g_{n}(x)-g(x)\rvert>\frac{\epsilon}{2}\right\} .
\end{align*}
Subadditvity and monotonicty implies
\begin{align*}
\mu\left(\left\{ x\in\X\mid\lvert cf_{n}(x)+g(x)-cf(x)-g(x)\rvert>\epsilon\right\} \right) & \leq\mu\left(\left\{ x\in\X\mid\lvert f_{n}(x)-f(x)\rvert>\frac{\epsilon}{2c}\right\} \right)\\
 & +\mu\left(\left\{ x\in\X\mid\lvert g_{n}(x)-g(x)\rvert>\frac{\epsilon}{2}\right\} \right).
\end{align*}
Taking limits yields the result.

The case for $L^{1}$ convergence will be established in chapter 4
where we show that $L^{1}$ convergence corresponds to convergence
in a (semi) norm, which automatically implies the result; for now
we take it as given.
\end{proof}
\begin{prop}[Squeeze theorem]
\label{prop:squeezeThm}Let $\left(\X,\F,\mu\right)$ be a measure
space and let $f_{n}\leq g_{n}\leq h_{n}$ be in $\measurableFunctions$
such that $f_{n},h_{n}\to\psi$ in one of the three modes of convergence.
Then $g_{n}\to\psi$ in the same mode of convergence.
\end{prop}

\begin{proof}
Suppose that $f_{n},h_{n}\stackrel{\text{a.s}}{\longrightarrow}\psi$.
Let $N_{f}$ and $N_{h}$ be the null sets where pointwise convergence
fails for the sequences $f_{n}$ and $h_{n}$ respectively. On the
complement of their union, we have pointwise convergence. Take any
point $x\in N_{f}^{C}\cap N_{h}^{C};$for any $\epsilon>0$ there's
some $n_{x,\epsilon}\in\N$ such that for all $n\geq n_{x,\epsilon}$
\[
\lvert f_{n}(x)-\psi(x)\rvert<\epsilon
\]
and
\[
\lvert h_{n}(x)-\psi(x)\rvert<\epsilon
\]
and so
\[
-\epsilon<f_{n}(x)-\psi(x)\leq g_{n}(x)-g(x)\leq h_{n}(x)<\epsilon\Longleftrightarrow\lvert g_{n}(x)-\psi(x)\rvert<\epsilon
\]
which completes the proof.

Now for the convergence in measure, observe that for any $\epsilon>0$,
\[
\left\{ x\in\X\mid\lvert h_{n}(x)-\psi(x)\rvert\leq\epsilon\right\} \bigcap\left\{ x\in\X\mid\lvert f_{n}(x)-\psi(x)\rvert\leq\epsilon\right\} \subseteq\left\{ x\in\X\mid\lvert g_{n}(x)-\psi(x)\rvert\leq\epsilon\right\} 
\]
and so
\[
\left\{ x\in\X\mid\lvert g_{n}(x)-\psi(x)\rvert>\epsilon\right\} \subseteq\left\{ x\in\X\mid\lvert h_{n}(x)-\psi(x)\rvert>\epsilon\right\} \bigcup\left\{ x\in\X\mid\lvert f_{n}(x)-\psi(x)\rvert>\epsilon\right\} 
\]
and so
\[
\mu\left(\left\{ x\in\X\mid\lvert g_{n}(x)-\psi(x)\rvert>\epsilon\right\} \right)\leq\mu\left(\left\{ x\in\X\mid\lvert h_{n}(x)-\psi(x)\rvert>\epsilon\right\} \right)+\mu\left(\left\{ x\in\X\mid\lvert f_{n}(x)-\psi(x)\rvert>\epsilon\right\} \right)
\]
by monotonicty and sub-additivity. Taking limits yields the result.

For $L^{1}$ convergence, note that we have $f_{n}-\psi\leq g_{n}-\psi\leq h_{n}-\psi$
and so\footnote{If $a\leq b\leq c$ then,$\lvert b\rvert\leq\max\left\{ b,-b\right\} \leq\max\left\{ c,-a\right\} \leq\max\left\{ \lvert c\rvert,\lvert a\rvert\right\} .$}
\[
\lvert g_{n}-\psi\rvert\leq\max\left\{ \lvert f_{n}-\psi\rvert,\lvert h_{n}-\psi\rvert\right\} \leq\lvert f_{n}-\psi\rvert+\lvert h_{n}-\psi\rvert.
\]
The linearity of integrals and the squeeze theorem for real sequences
then implies the result.
\end{proof}
Finally, for our limiting operations to make sense, limits should
be unique. Unfortunately, this is not strictly true; however all limits
are almost everywhere unique. For the case of convergence in measure
and $L^{1}$ convergence, we shall show that this follows from the
fact that the topologies for these modes of convergence are given
by (pseudo) metrics (in the special case of $L^{1}$, by a semi-norm);
the case of almost-everywhere convergence, the result follows by the
fact that the union of null sets is null. Therefore we have the following.
\begin{prop}[Uniqueness of limits]
\label{prop:limitUnique}Let $\left(\X,\F,\mu\right)$ be a measure
space and let $\left\{ f_{n}\right\} _{n\in\N},f,g\in\measurableFunctions$
be such that $f_{n}\longrightarrow f$ and $f_{n}\longrightarrow g$
in any of the modes of convergence. Then $f\stackrel{\text{a.s}}{=}g.$
\end{prop}

It would be good to know the conditions under which one type of convergence
implies another, one of which we have already explored in the form
of the (generalized) dominated convergence theorem, which gives us
sufficient conditions for almost everywhere convergence implies $\mathcal{L}^{1}$
convergence. The next <few> results link the other types of convergence.
\begin{prop}
\label{prop:finiteMeasuresAEimpliesConvInMeasures}Let $\left(\X,\F,\mu\right)$
be a measure space such that $\mu\left(\X\right)<\infty$ and let
$\left\{ f_{n}\right\} _{n\in\N}\in\measurableFunctions$ such that
\[
f_{n}\stackrel{\text{a.e}}{\longrightarrow}f.
\]
Then
\[
f_{n}\stackrel{\mu}{\longrightarrow}f.
\]
\end{prop}

\begin{proof}
Define 
\[
A:=\left\{ x\in\X\mid\lim_{n\to\infty}f_{n}\left(x\right)\neq f\left(x\right)\right\} 
\]
and observe that by our hypothesis $\mu\left(A\right)=0.$ Let's unpack
what this means carefully. Fix any $x_{0}\in A$ and observe that
there exists some $\epsilon>0$ such that $\lvert f_{n}\left(x_{0}\right)-f\left(x_{0}\right)\rvert>\epsilon$
for infinitely many $n\in\N.$ Recall the discussion on convergence
of sets in Section 2.1 and notice that we can formalize our intuition
with the equalities
\begin{align*}
A & =\bigcup_{\epsilon>0}\bigcap_{n_{0}\in\N}\bigcup_{n\geq n_{0}}\left\{ x\in\X\mid\lvert f_{n}\left(x\right)-f\left(x\right)\rvert>\epsilon\right\} \\
 & =\bigcup_{\epsilon>0}\limsup_{n\to\infty}\left\{ x\in\X\mid\lvert f_{n}\left(x\right)-f\left(x\right)\rvert>\epsilon\right\} 
\end{align*}
where the set $\left\{ x\in\X\mid\lvert f_{n}\left(x\right)-f\left(x\right)\rvert>\epsilon\right\} \in\F$
by Lemma \ref{lem:compositionMeasurableFunctions} and so $\limsup_{n\to\infty}\left\{ x\in\X\mid\lvert f_{n}\left(x\right)-f\left(x\right)\rvert>\epsilon\right\} \in\F$
by closure under countable unions and intersections. Then,
\[
0\leq\mu\left(\limsup_{n\to\infty}\left\{ x\in\X\mid\lvert f_{n}\left(x\right)-f\left(x\right)\rvert>\epsilon\right\} \right)\leq\mu\left(A\right)=0
\]
for any $\epsilon>0$ by the monotonicity of measures. Finally, recall
that by Proposition \ref{prop:limSupInfIndicator}
\[
\indicate_{\limsup_{n\to\infty}\left\{ x\in\X\mid\lvert f_{n}\left(x\right)-f\left(x\right)\rvert>\epsilon\right\} }=\limsup_{n\to\infty}\indicate_{\left\{ x\in\X\mid\lvert f_{n}\left(x\right)-f\left(x\right)\rvert>\epsilon\right\} }\leq1
\]
where $\lebInt{\mu}1=\mu\left(\X\right)<\infty$. Then, by the \hyperref[cor:reverseFatouLemma]{reverse Fatou Lemma}
\[
0\leq\limsup_{n\to\infty}\mu\left(\left\{ x\in\X\mid\lvert f_{n}\left(x\right)-f\left(x\right)\rvert>\epsilon\right\} \right)\leq\mu\left(\limsup_{n\to\infty}\left\{ x\in\X\mid\lvert f_{n}\left(x\right)-f\left(x\right)\rvert>\epsilon\right\} \right)=0.
\]
By the non-negativity of measures
\begin{align*}
0 & \leq\liminf_{n\to\infty}\mu\left(\left\{ x\in\X\mid\lvert f_{n}\left(x\right)-f\left(x\right)\rvert>\epsilon\right\} \right)\\
 & \leq\lim_{n\to\infty}\mu\left(\left\{ x\in\X\mid\lvert f_{n}\left(x\right)-f\left(x\right)\rvert>\epsilon\right\} \right)\\
 & \leq\limsup_{n\to\infty}\mu\left(\left\{ x\in\X\mid\lvert f_{n}\left(x\right)-f\left(x\right)\rvert>\epsilon\right\} \right)\\
 & =0
\end{align*}
which completes the proof.
\end{proof}
Note that if we drop the finiteness assumption the result does not
hold: take $\indicate_{[n,n+1]}\longrightarrow0$ pointwise and the
Lebesgue measure $\lambda$ on $\R$ and note that for any $\epsilon\in\left(0,1\right)$
\[
\lim_{n\to\infty}\lambda\left(\left\{ x\in\R\mid\indicate_{[n,n+1]}>\epsilon\right\} \right)=\lim_{n\to\infty}\lambda\left(\left[n,n+1\right]\right)=1.
\]
Note that crucial failure here is that we cannot dominate $\indicate_{[n,n+1]}$
uniformly (i.e. for all $n$) with any integrable function $g$ and
so a violation of dominated convergence leads to a failure of convergence
in measure.

The converse to this result is not true either, that is, convergence
in measure does not imply almost everywhere convergence, even when
the measure in question in is finite.
\begin{example}
\label{exa:convMeasureDoesntImplyAEConv} Observe that for any $n\in\N$,
there exists some $k\in\N\cup\{0\}$ such that $2^{k}\leq n<2^{k+1}.$
By construction, there's only one such $k$ and so we can denote it
$k(n).$ Consider the following collection of sets in $\borel(\R)$
\[
E_{n}:=\left[\frac{n-2^{k(n)}}{2^{k(n)}},\frac{n+1-2^{k(n)}}{2^{k(n)}}\right]
\]
and observe that $E_{n}\subseteq[0,1]$ for all $n\in\N$. To see
this, note that the lower bound is smallest when $n=2^{k(n)}$ and
the upper bound is largest when $n=2^{k(n)+1}-1$. which correspond
to 0 and 1 respectively. Now for $\epsilon\in(0,1)$
\[
\lim_{n\to\infty}\lambda\left(\left\{ x\in\R\mid\indicate_{E_{n}}>\epsilon\right\} \right)=\lim_{n\to\infty}\lambda\left(E_{n}\right)=\lim_{n\to\infty}\frac{1}{2^{k(n)}}=0
\]
as $k(n)\to\infty.$ Further, for $\epsilon\geq1$
\[
\lambda\left(\left\{ x\in\R\mid\indicate_{E_{n}}>\epsilon\right\} \right)=0
\]
for every $n\in\N$ and so 
\[
\indicate_{E_{n}}\stackrel{\mu}{\longrightarrow}0.
\]
Now for $x\in[0,1]$, we have that $x\in E_{n}$ for infinitely many
$n\in\N$. We can establish this by noting that 
\[
\bigcup_{2^{k(n)}\leq n\leq2^{k(n)}-1}E_{n}=[0,1]
\]
and so for each $k\in\N\cup\{0\}$ there's some $n_{k}$ such that
$x\in E_{n_{k}}$. Then the subsequence $\indicate_{E_{n_{k}}}=1$
and so the our sequence $\indicate_{E_{n}}$doesnt converge pointwise
to zero anywhere in $[0,1].$ In particular, it doesn't converge almost
everywhere.
\end{example}

However, convergence in measure implies almost everywhere subsequential
convergence. To show this, we will first establish a useful lemma.
\begin{lem}
\label{lem:borelCantelliAEConvArgument}Let $\left(\X,\F,\mu\right)$
be a measure space and let $\left\{ f_{n}\right\} _{n\in\N}\in\measurableFunctions$
be such that 
\[
\sum_{n=1}^{\infty}\mu\left(\left\{ x\in\X\mid\lvert f_{n}\left(x\right)-f\left(x\right)\rvert>\epsilon\right\} \right)<\infty
\]
for every $\epsilon>0$ and some $f\in\measurableFunctions.$ Then
\[
f_{n}\stackrel{\text{a.e}}{\longrightarrow}f.
\]
\end{lem}

\begin{proof}
Applying the Borel-Cantelli lemma, we know that for each $\epsilon>0$
\[
\mu\left(\limsup_{n\to\infty}\left\{ x\in\X\mid\lvert f_{n}\left(x\right)-f\left(x\right)\rvert>\epsilon\right\} \right)=0.
\]
In particular, this is true for $\epsilon=\frac{1}{p}$ for every
$p\in\N.$ By the Archimedean property of natural numbers and our
discussion in the proof of Proposition \ref{prop:finiteMeasuresAEimpliesConvInMeasures},
the set
\[
\left\{ x\in\X\mid\lim_{n\to\infty}f_{n}\left(x\right)\neq f\left(x\right)\right\} =\bigcup_{p\in\N}\limsup_{n\to\infty}\left\{ x\in\X\mid\lvert f_{n}\left(x\right)-f\left(x\right)\rvert>\frac{1}{p}\right\} 
\]
and so by Proposition \ref{prop:nullClosureCountableUnion}
\[
\mu\left(\left\{ x\in\X\mid\lim_{n\to\infty}f_{n}\left(x\right)\neq f\left(x\right)\right\} \right)=0
\]
which completes the proof.
\end{proof}
\begin{prop}
\label{prop:convInMeasureImpliesSubsequenceAE}Let $\left(\X,\F,\mu\right)$
be a measure space and let $\left\{ f_{n}\right\} _{n\in\N}\in\measurableFunctions$
be a sequence of functions such that 
\[
f_{n}\stackrel{\mu}{\longrightarrow}f
\]
for some $f\in\measurableFunctions.$ Then there exists a sequence
of natural numbers $\left\{ n_{k}\right\} _{k\in\N}$such that
\[
f_{n_{k}}\stackrel{\text{a.e}}{\longrightarrow}f.
\]
\end{prop}

\begin{proof}
Note that by convergence in measure, for any $\epsilon,\delta>0$
there exists some $n_{\epsilon,\delta}^{*}\in\N$ such that for all
$n\geq n_{\epsilon,\delta}^{*}$
\[
\mu\left(\left\{ x\in\X\mid\lvert f_{n}\left(x\right)-f\left(x\right)\rvert>\epsilon\right\} \right)<\delta.
\]
In particular, this is true for $\epsilon=\delta=2^{-k}$ for all
$k\in\N.$ Define the sequence $\left\{ n_{k}\right\} _{k\in\N}$by
\[
n_{k}:=\begin{cases}
n_{1,1}^{*}, & k=1\\
n_{2^{-k},2^{-k}}^{*}, & n_{k-1}<n_{2^{-k},2^{-k}}^{*}\\
n_{2^{-k},2^{-k}}^{*}+1, & n_{k-1}=n_{2^{-k},2^{-k}}^{*}
\end{cases}
\]
and observe that $n_{k}$ is a strictly increasing sequence of natural
numbers and that 
\[
\sum_{k=1}^{\infty}\mu\left(\left\{ x\in\X\mid\lvert f_{n_{k}}\left(x\right)-f\left(x\right)\rvert>2^{-k}\right\} \right)<\sum_{k=1}^{\infty}2^{-k}=1<\infty.
\]
Applying Lemma \ref{lem:borelCantelliAEConvArgument} furnishes the
result.
\end{proof}
\begin{prop}[Markov's inequality]
\label{prop:markovInequality} Let $\left(\X,\F,\mu\right)$ be a
measure space. For any function $f\in\Lp 1{\mu}$ and any $a>0$
\[
\mu\left(\left\{ x\in\X\mid\lvert f\left(x\right)\rvert>a\right\} \right)\leq\frac{\lebInt{\mu}{\lvert f\rvert}}{a}.
\]
\end{prop}

\begin{proof}
Note the pointwise inequality
\[
a\indicate_{\left\{ x\in\X\mid\lvert f\left(x\right)\rvert>a\right\} }\leq\lvert f\left(x\right)\rvert
\]
and observe that by the monotonicity of the integral
\[
a\lebInt{\mu}{\indicate_{\{x\in\X\mid\lvert f\left(x\right)\rvert>a\}}}\leq\lebInt{\mu}{\lvert f\rvert}
\]
which finishes the proof.
\end{proof}
\begin{prop}
\label{prop:L1impliesConvergenceInMeasure}For a sequence of functions
$\left\{ f_{n}\right\} _{n\in\N}\in\Lp 1{\X,\F,\mu}$such that 
\[
f_{n}\stackrel{\mathcal{L}^{1}}{\longrightarrow}f
\]
for some $f\in\Lp 1{\mu},$ we have 
\[
f_{n}\stackrel{\mu}{\longrightarrow}f.
\]
\end{prop}

\begin{proof}
Note that for any fixed $\epsilon>0$
\[
0\leq\mu\left(\left\{ x\in\X\mid\lvert f_{n}\left(x\right)-f\left(x\right)\rvert>\epsilon\right\} \right)\leq\frac{1}{\epsilon}\lebInt{\mu}{\lvert f_{n}-f\rvert}
\]
by the Markov inequality. Taking the limit proves the result.
\end{proof}
\begin{thm}[Egorov's theorem]
\label{thm:egorovTheorem}Let $\left(\X,\F,\mu\right)$ be a measure
space such that $\mu\left(\X\right)<\infty$ and fix some $\epsilon>0$.
If $\left\{ f_{n}\right\} _{n\in\N}\in\measurableFunctions$ is a
sequence functions such that
\[
f_{n}\stackrel{\text{a.e}}{\longrightarrow}f
\]
where $f\in\measurableFunctions$ then there exists some $A\in\F$
such that $\mu\left(A\right)<\epsilon$ and 
\[
\lim_{n\to\infty}\sup_{x\in\X\setminus A}\lvert f_{n}\left(x\right)-f\left(x\right)\rvert=0
\]
\end{thm}

\begin{proof}
Define the set 
\[
A_{n,k}:=\bigcup_{i=n}^{\infty}\left\{ x\in\X\mid\lvert f_{i}\left(x\right)-f\left(x\right)\rvert>\frac{1}{k}\right\} 
\]
and recall from our earlier discussion that 
\[
\mu\left(\bigcap_{n\in\N}A_{n,k}\right)=0
\]
for every $k\in\N$ by almost everywhere convergence. Then, since
$\mu\left(\X\right)<\infty,$ we can apply the \hyperref[prop:equivalenceContinuityMeasures]{continuity from above of measures}
to deduce that 
\[
\lim_{n\to\infty}\mu\left(A_{n,k}\right)=0.
\]
This means there exists some $n_{k}\in\N$ such that for all $n\geq n_{k}$
\[
\mu\left(A_{n,k}\right)<\frac{\epsilon}{2^{k}}
\]
and so
\[
\mu\left(\bigcup_{k\in\N}A_{n_{k},k}\right)\leq\sum_{k=1}^{\infty}\mu\left(A_{n_{k},k}\right)\leq\epsilon.
\]
If we let $A:=\bigcup_{k\in\N}A_{n_{k},k}$ then
\[
\X\setminus A=\bigcap_{k\in\N}\bigcap_{i=n_{k}}^{\infty}\left\{ x\in\X\mid\left\lvert f_{i}\left(x\right)-f\left(x\right)\right\rvert \leq\frac{1}{k}\right\} .
\]
Fix any $k\in\N$, and observe that for any $n\geq n_{k}$ and any
$x\in\X\setminus A$
\[
\lvert f_{n}\left(x\right)-f\left(x\right)\rvert\leq\frac{1}{k}
\]
which means 
\[
\sup_{x\in\X\setminus A}\lvert f_{n}\left(x\right)-f\left(x\right)\rvert\leq\frac{1}{k}.
\]
Since $k$ was unspecified, we have uniform convergence.
\end{proof}

\subsection{Topologies of various convergence notions}
\begin{defn}
\label{def:pseudoMetric}Let $X$ be a set and let 
\[
d:X\times X\to\R
\]
be a function that satisfies\begin{enumerate}[label=(\roman*),leftmargin=.1\linewidth,rightmargin=.4\linewidth]
	\item $ d(x,y) = d(y,x) \forall x,y \in X$ (Symmetry)
	\item $ d(x,z) \leq d(x,y) + d(y,z) \forall x,y,z \in X $ (Triangle inequality)
\end{enumerate}

Then $(X,d)$ is called a \emph{pseudometric space }and $d$ is called
a \emph{pseudo-metric.}
\end{defn}

\begin{prop}
\label{prop:pseudoMetricConvMeasure}Let $\left(\X,\F,\mu\right)$
be a finite measure space i.e. $\mu(\X)<\infty$. Then the function
\[
d(f,g):=\lebInt{\mu}{\frac{\lvert f-g\rvert}{1+\lvert f-g\rvert}}
\]
defines a pseudo-metric. Moreover, for any $\left\{ f_{n}\right\} _{n\in\N},f\in\measurableFunctions$
\[
f_{n}\stackrel{\mu}{\longrightarrow}f
\]
if and only if
\[
d(f_{n},f)\longrightarrow0.
\]
\end{prop}

\begin{proof}
Note that 
\[
0\leq\frac{\lvert f-g\rvert}{1+\lvert f-g\rvert}\leq1
\]
and so the function $d$ is well defined as an integral on a finite
measure space of bounded a function. Note that the symmetry of $d$
is obvious given the symmetry of the absolute value function. To see
the triangle inequality, first observe that the function $g(x)=\frac{x}{1+x}$
is increasing on $x>0$ (it has a derivative $\frac{1}{(1+x)^{2}}$)
and so, by the triangle inequality for absolute value
\begin{align*}
\frac{\lvert f-h\rvert}{1+\lvert f-h\rvert} & \leq\frac{\lvert f-g\rvert+\lvert g-h\rvert}{1+\lvert f-g\rvert+\lvert g-h\rvert}\\
 & =\frac{\lvert f-g\rvert}{1+\lvert f-g\rvert+\lvert g-h\rvert}+\frac{\lvert g-h\rvert}{1+\lvert f-g\rvert+\lvert g-h\rvert}\\
 & \leq\frac{\lvert f-g\rvert}{1+\rvert f-g\rvert}+\frac{\lvert g-h\rvert}{1+\rvert g-h\rvert}.
\end{align*}
Linearity and monotonicity of integration yields the triangle inequality
for $d.$ Now suppose $f_{n}\stackrel{\mu}{\to}f$ , let $\epsilon>0$
and observe that 
\begin{align*}
d(f_{n},f) & =\lebInt{\mu}{\frac{\lvert f_{n}-f\rvert}{1+\lvert f_{n}-f\rvert}}\\
 & =\lebInt{\mu}{\frac{\lvert f_{n}-f\rvert}{1+\lvert f_{n}-f\rvert}\indicate_{\{\lvert f_{n}-f\rvert>\epsilon\}}}+\lebInt{\mu}{\frac{\lvert f_{n}-f\rvert}{1+\lvert f_{n}-f\rvert}\indicate_{\{\lvert f_{n}-f\rvert\leq\epsilon\}}}\\
 & \leq\mu\left(\lvert f_{n}-f\rvert>\epsilon\right)+\frac{\epsilon}{1+\epsilon}\mu\left(\lvert f_{n}-f\rvert\leq\epsilon\right)\\
 & \leq\mu\left(\lvert f_{n}-f\rvert>\epsilon\right)+\frac{\epsilon}{1+\epsilon}\mu\left(\X\right).
\end{align*}
Therefore by convergence in measure 
\[
\lim_{n\to\infty}d(f_{n},f)\leq\frac{\epsilon}{1+\epsilon}\mu\left(\X\right).
\]
Of course, $\epsilon$ can be arbitrarily small and so we have convergence
in the pseudo-metric.

Conversely, suppose $\lim_{n\to\infty}d(f_{n},f)=0$ and fix some
$\epsilon>0$. Then
\[
0\leq\frac{\epsilon}{1+\epsilon}\mu\left(\lvert f_{n}-f\rvert>\epsilon\right)\leq\lebInt{\mu}{\frac{\lvert f_{n}-f\rvert}{1+\lvert f_{n}-f\rvert}\indicate_{\{\lvert f_{n}-f\rvert>\epsilon\}}}\leq\lebInt{\mu}{\frac{\lvert f_{n}-f\rvert}{1+\lvert f_{n}-f\rvert}}
\]
and in the limit 
\[
\frac{\epsilon}{1+\epsilon}\mu\left(\lvert f_{n}-f\rvert>\epsilon\right)\longrightarrow0\implies\mu\left(\lvert f_{n}-f\rvert>\epsilon\right)\longrightarrow0
\]
which completes the proof.
\end{proof}
Note that $d(f,g)=0\Longleftrightarrow f\stackrel{\text{a.e}}{=}g$.
This is a simple corrolary of Proposition \ref{prop:intZeroFuncZero}.
Propositions \ref{prop:convInMeasureImpliesSubsequenceAE} and \ref{prop:finiteMeasuresAEimpliesConvInMeasures}
gives us a sort of equivalence between almost everywhere equivalence
as a corollary of a basic fact about sequences in topological spaces.
\begin{lem}
\label{lem:subsubSequence}Let $\left(\X,\tau\right)$ be a topological
space. The sequence$\left\{ x_{n}\right\} _{n\in\N}\in\X$ converges
to $x\in\X$ if and only if every subsequence of $x_{n}$ has a further
subsequence that converges to $x.$
\end{lem}

\begin{proof}
Convergence implies subsequential convergence for every subseqeuence
(and in particulary, subsubsequences). Conversely, suppose that $x_{n}\nrightarrow x$.
Then there's an open set $\mathcal{U\in\tau}$ that contains $x$
such that there are infinitely many elements of $x_{n}$ outside $\mathcal{U}$.
In other words, there exists some subsequence $x_{n_{k}}\notin\mathcal{U}$
for all $k\in\N$. Then no subsequence of $x_{n_{k}}$ converges to
$x$.
\end{proof}
\begin{prop}
\label{prop:subsubsequenceProbAEConv}Let $\left(\X,\F,\mu\right)$
be a finite measure space i.e $\mu(\X)<\infty.$ Then for any sequence
$\left\{ f_{n}\right\} _{n\in\N}\in\measurableFunctions$ and $f\in\measurableFunctions$
\[
f_{n}\stackrel{\mu}{\longrightarrow}f
\]
if and only if every subsequence $f_{n_{k}}$ has a further subsequence
$f_{n_{k_{j}}}$ such that 
\[
f_{n_{k_{j}}}\stackrel{\text{a.e}}{\longrightarrow}f.
\]
\end{prop}

\begin{proof}
Suppose $f_{n}\stackrel{\mu}{\longrightarrow}f$ and let $f_{n_{k}}$
be an arbitrary subsequence. This subsequence also converges to $f$
in measure and so by Proposition \ref{prop:convInMeasureImpliesSubsequenceAE}
there's some sub-subsequence $f_{n_{k_{j}}}\stackrel{\text{a.e}}{\longrightarrow}f$.
Conversely, suppose every subsequence $f_{n_{k}}$ has a further subsequence
$f_{n_{k_{j}}}\stackrel{\text{a.e}}{\longrightarrow}f$. Then by Proposition
\ref{prop:finiteMeasuresAEimpliesConvInMeasures} $f_{n_{k_{j}}}\stackrel{\mu}{\longrightarrow}f$.
By Proposition \ref{prop:subsubsequenceProbAEConv}, convergence in
measure is (pseudo)-metrizable and thus topological and so by Lemma
\ref{lem:subsubSequence} $f_{n}\stackrel{\mu}{\longrightarrow}f.$
\end{proof}
\begin{cor}
\label{cor:noTopologyForAEConv}There is no topology for almost everywhere
convergence.
\end{cor}

\begin{proof}
Note that if almost everywhere convergence was topological, then by
Lemma \ref{lem:subsubSequence} and Proposition \ref{prop:subsubsequenceProbAEConv},
$f_{n}\stackrel{\mu}{\longrightarrow}f\implies f_{n}\stackrel{\text{a.e}}{\longrightarrow}f$.
But this has been ruled out by Example \ref{exa:convMeasureDoesntImplyAEConv}.
\end{proof}
Thus we have established that the topology of convergence in measure
comes from a (pseudo)-metric, whereas almost everywhere convergence
is not topologizable at all! Convergence in $L^{1}$ , on the other
hand, corresponds to a (semi)-normed convergence as shown in Chapter
4. Proposition \ref{prop:subsubsequenceProbAEConv} implies a form
of dominated convergence theorem for convergence in measure.
\begin{prop}
\label{prop:convMeasureDomConv}Let $\left(\X,\F,\mu\right)$ be a
finite measure space i.e $\mu(\X)<\infty.$ Suppose $\left\{ f_{n}\right\} _{n\in\N}\in\measurableFunctions$
such that $\lvert f_{n}\rvert\leq g$ where $g\in\Lp 1{\X,\F,\mu}.$
Then
\[
f_{n}\stackrel{\mu}{\longrightarrow}f\implies f_{n}\stackrel{\mathcal{L}^{1}}{\longrightarrow}f.
\]
\end{prop}

\begin{proof}
Note that if $f_{n}\stackrel{\mu}{\longrightarrow}f$, then for any
subsequence $f_{n_{j}}$ of $f_{n}$ there exists a further subseqeunce
$f_{n_{j_{k}}}$ such that 
\[
f_{n_{j_{k}}}\stackrel{\text{a.e}}{\longrightarrow}f.
\]
Since $\lvert f_{n_{j_{k}}}\rvert\leq g\in\Lp 1{\mu}$, the \hyperref[thm:generalizedDominatedConvergence]{dominated convergence theorem}
implies that 
\[
f_{n_{j_{k}}}\stackrel{\mathcal{L}^{1}}{\longrightarrow}f.
\]
But $L^{1}$ convergence is topological and so by Lemma \ref{lem:subsubSequence},
\[
f_{n}\stackrel{\mathcal{L}^{1}}{\longrightarrow}f.
\]
\end{proof}

\subsection{Uniform integrability and uniform absolute continuity}

The dominated convergence theorem says that for sequence of random
variables that are dominated by an integrable function, pointwise
convergence implies convergence in $L^{1}$. We would like to extend
this notion more broadly.
\begin{defn}
\label{def:uniformIntegrability}Let $\left(\X,\F,\mu\right)$ be
a measure space and let $\mathcal{C}\subseteq\measurableFunctions$
be a collection of measurable functions. We say $\mathcal{C}$ is
\emph{uniformly integrable }if for every $\epsilon>0$, there exists
some $M\in\N$ such that 
\[
\lebInt{\mu}{\lvert f\rvert\indicate_{\{\lvert f\rvert>M\}}}<\epsilon
\]
for every $f\in\mathcal{C}$.
\end{defn}

\begin{defn}
\label{def:uniformAbsoluteContinuity}Let $\left(\X,\F,\mu\right)$
be a measure space and let $\mathcal{C}\subseteq\measurableFunctions$
be a collection of measurable functions. We say $\mathcal{C}$ is
\emph{uniformly absolutely continuous }if for every $\epsilon>0$
there exists a $\delta>0$ such that for any $f\in\mathcal{C}$
\[
F\in\F,\mu(F)<\delta\implies\lvert\lebInt{\mu}{f\indicate_{F}}\rvert<\epsilon.
\]
\end{defn}

\begin{rem}
\label{rem:uniformAbsoluteContinuity}Note that if $\mathcal{C}$
is a trivial class i.e. $\mathcal{C}=\{f\}$, then the absolute continuity
property always holds as observed in Proposition \ref{prop:densities}
and Remark \ref{rem:absoluteContinuity}. Of course, it is not clear
that this $\epsilon-\delta$ characterization of absolute continuity
is the same as the one discussed in that remark; later on in Chapter
6 (see Proposition \ref{prop:epsdeltaAbsContinuity}) we show they
are indeed equiavelent for finite measures.
\end{rem}

\begin{prop}
\label{prop:finiteMeasureUniformIntegrability}Let $\left(\X,\F,\mu\right)$
be a finite measure space i.e. $\mu(\X)<\infty.$ Then a class of
functions $\mathcal{C}\subseteq\measurableFunctions$ is uniformly
integrable if and only if it is uniformly absolutely continuous and
\[
\sup_{f\in\mathcal{C}}\lebInt{\mu}{\lvert f\rvert}<\infty.
\]
\end{prop}

\begin{proof}
Suppose $\mathcal{C}\subseteq\measurableFunctions$ is uniformly integrable.
Fix some $\epsilon>0$ and observe that by uniform integrability there
is some $M\in\N$ such that 
\[
\lebInt{\mu}{\lvert f\rvert\indicate_{\{\lvert f\rvert>M\}}}<\frac{\epsilon}{2}.
\]
Then for any $F\in\F$,
\begin{align*}
\lvert\lebInt{\mu}{f\indicate_{F}}\rvert & \leq\lebInt{\mu}{\lvert f\rvert\indicate_{F}}\\
 & =\lebInt{\mu}{\lvert f\rvert\left(\indicate_{\{\lvert f\rvert>M\}}+\indicate_{\{\lvert f\rvert\leq M\}}\right)\indicate_{F}}\\
 & =\lebInt{\mu}{\lvert f\rvert\indicate_{\{\lvert f\rvert>M\}}\indicate_{F}}+\lebInt{\mu}{\lvert f\rvert\indicate_{\{\lvert f\rvert\leq M\}}\indicate_{F}}\\
 & \leq\lebInt{\mu}{\lvert f\rvert\indicate_{\{\lvert f\rvert>M\}}}+\lebInt{\mu}{\lvert f\rvert\indicate_{\{\lvert f\rvert\leq M\}}\indicate_{F}}\\
 & \leq\lebInt{\mu}{\lvert f\rvert\indicate_{\{\lvert f\rvert>M\}}}+\lebInt{\mu}{M\indicate_{\{\lvert f\rvert\leq M\}}\indicate_{F}}\\
 & \leq\frac{\epsilon}{2}+M\mu(\{\lvert f\rvert\leq M\}\cap F)\\
 & \leq\frac{\epsilon}{2}+M\mu(F)
\end{align*}
where all the inequalities follow by the monotonicity of the integral.
Letting $\delta=\frac{\epsilon}{2M}$ then shows uniform absolute
continuity. Letting $F=\X$ then yields 
\[
\lebInt{\mu}{\lvert f\rvert}\leq\frac{\epsilon}{2}+M\mu(\X)
\]
for any $f\in\mathcal{C}$ which gives us the boundedness in $L^{1}$.

Conversely, assume that $\mathcal{C}$ is uniformly absolutely continuous
and $\sup_{f\in\mathcal{C}}\lebInt{\mu}{\lvert f\rvert}<\infty.$
Let $\epsilon>0$ and note that there exists some $\delta>0$ such
that $F\in\F,\mu(F)<\delta\implies\lvert\lebInt{\mu}{f\indicate_{F}}\rvert<\epsilon$.
Note that by Markov's inequality
\[
\mu\left(\lvert f\rvert>M\right)\leq\frac{\lebInt{\mu}{\lvert f\rvert}}{M}.
\]
for any $M\in\N$. Since $\lebInt{\mu}{\lvert f\rvert}<\infty,$we
can take $M$ large enough to so that $\mu\left(\lvert f\rvert>M\right)<\delta$
which by uniform absolute continuity implies $\lvert\lebInt{\mu}{\lvert f\lvert\indicate_{\{\lvert f\rvert>M\}}}\rvert=\lebInt{\mu}{\lvert f\lvert\indicate_{\{\lvert f\rvert>M\}}}<\epsilon.$
This shows uniform integrability, completing the proof.
\end{proof}
The above implies that integrable functions on finite measure spaces
are uniformly integrable on their own. Classes of uniformly integrable
functions are preserved under linear combinations.
\begin{prop}
\label{prop:uniformlyIntegrableLinearCombination}Let $\left(\X,\F,\mu\right)$
be a measure space and let $c\in\R$ be a scalar. Further, let $\mathcal{C},\mathcal{D}\subseteq\measurableFunctions$
be uniformly integrable collections of functions. Then 
\[
\mathcal{C}+c\mathcal{D}:=\left\{ f+cg\mid f\in\mathcal{C},g\in\mathcal{D}\right\} 
\]
is a uniformly integrable collection.
\end{prop}

\begin{proof}
Note that when $c=0$ then the result is triviall true so let $c\neq0.$
Fix $\epsilon>0$ and note that there exists some $M$ such that for
any $g\in\mathcal{D}$
\[
\lebInt{\mu}{\lvert g\rvert\indicate_{\{\lvert g\rvert>M\}}}<\frac{\epsilon}{\lvert c\rvert}.
\]
Then, for $N=M\lvert c\rvert$
\[
\lebInt{\mu}{\lvert cg\rvert\indicate_{\{\lvert cg\rvert>N\}}}=\lvert c\rvert\lebInt{\mu}{\lvert g\rvert\indicate_{\{\lvert g\rvert>M\}}}<\epsilon.
\]
Next, observe that \hl{COMPLETE LATER}
\end{proof}
\begin{thm}[Vitali's Convergence Theorem]
\label{thm:vitaliConvergence}Let $\left(\X,\F,\mu\right)$ be a
finite measure space i.e. $\mu(\X)<\infty.$ If $\left\{ f_{n}\right\} _{n\in\N}\in\measurableFunctions$
are uniformly integrable and $f_{n}\stackrel{\mu}{\longrightarrow}f$
then $f\in\Lp 1{\mu}$ and 
\[
f_{n}\stackrel{\mathcal{L}^{1}}{\longrightarrow}f.
\]
\end{thm}

\begin{proof}
Note that if $f_{n}\stackrel{\mu}{\longrightarrow}f$ then by Proposition
\ref{prop:convInMeasureImpliesSubsequenceAE} there exists some subsequence
$f_{n_{j}}\stackrel{\text{a.e}}{\longrightarrow}f$. By Fatou's lemma
and the fact that absolute value is continuous, we have
\[
\lebInt{\mu}{\lvert f\rvert}=\lebInt{\mu}{\liminf_{j\to\infty}\lvert f_{n_{j}}\rvert}\leq\liminf_{j\to\infty}\lebInt{\mu}{\lvert f_{n_{j}}\rvert}\leq\sup_{n\in\N}\lebInt{\mu}{\lvert f_{n}\rvert}<\infty
\]
where the last inequality follows by Proposition \ref{prop:finiteMeasureUniformIntegrability},
which shows $f\in\Lp 1{\mu}.$ Since $\left\{ f_{n}\right\} _{n\in\N},\left\{ f\right\} $
are uniformly integrable, $g_{n}:=f_{n}-f$ is uniormly integrable
by Proposition \ref{prop:uniformlyIntegrableLinearCombination}. Note
that by uniform absolute continuity, for any $\epsilon>0$, there
exists some $\delta>0$ such that for any $A\in\F:P(A)<\delta\implies\lebInt{\mu}{\lvert g_{n}\rvert\indicate_{A}}<\frac{\epsilon}{2}.$
Note that by convergence in probability, there exists some $n_{0}\in\N$
$\mu\left(\lvert g_{n}\rvert>\frac{\epsilon}{2}\right)<\delta$ and
so
\[
\lebInt{\mu}{\lvert g_{n}\rvert}=\lebInt{\mu}{\lvert g_{n}\rvert\indicate_{\{\lvert g_{n}\rvert\leq\frac{\epsilon}{2}\}}}+\lebInt{\mu}{\lvert g_{n}\rvert\indicate_{\{\lvert g_{n}\rvert>\frac{\epsilon}{2}\}}}\leq\frac{\epsilon}{2}+\frac{\epsilon}{2}=\epsilon
\]
which shows $L^{1}$ convergence.
\end{proof}

\section{The Riemann integral\label{sec:riemannIntegral}}

The theory of integration that we have developed so far is only useful
if (1) it is, in some sense, a suitable generalization of Riemann's
theory of integration, and (2) if it obeys the fundamental theorem
of calculus. We leave (2) to section \ref{sec:ftc} and in this section
review and characterize the Riemann integral. In particular, we give
a precise characterization of Riemann integrable functions using the
Lebesgue theory we have developed so far, and show that for such functions,
their Lebesgue integrals (with respect to the Lebesgue measure) and
Riemann integrals coincide.

We first consider the classical construction of the Riemann integral,
as articulated by Darboux and Riemann. After we have developed this
baisc material carefully, we can present the Riemann integral under
the more modern Lebesgue formulation. We do this by, a more primitive
version of the Lebesgue measure, called the \emph{Jordan content }or
\emph{Jordan measure}, which coincides with the Lebesgue measure for
sufficiently well behaved sets.

\subsection{The classical construction of the Riemann integral}
\begin{defn}
\label{def:partitionInterval}Let $\left[a,b\right]\subset\R$ be
a closed and bounded interval. A \emph{partition $\pi$ }of $\left[a,b\right]$
is a finite set
\[
\pi=\left\{ t_{i}\mid t_{0}=a,t_{k}=b,t_{i}<t_{i+1}\right\} 
\]
where $k\in\N$ is the \emph{size }of the partition. The size of a
partition $\pi$is often denoted $k\left(\pi\right).$
\end{defn}

A second notion of size is given by th concept of a mesh (which is
closely related to, and a special case of, the Lebesgue measure).
\begin{defn}
\label{def:meshPartition}Given a partition $\pi$of $\left[a,b\right]$,
the\emph{ mesh }of $\pi$is given by 
\[
\mesh\left(\pi\right):=\max_{1\leq i\leq k}t_{i}-t_{i-1}.
\]
\end{defn}

As one can see, the mesh is basically given by $\max_{i}\lambda\left(\left[t_{i-1},t_{i}\right]\right)$
where $\lambda$ is the Lebesgue measure. We now introduce the notion
of a \emph{Darboux }sum, which are the building blocks of Riemann
integrals much in the way that integrals of simple functions are in
the Lebesgue theory. In fact, Darboux sums are the Lebesgue integrals
of a special type of signed simple function called a step function.
\begin{defn}
\label{def:darbouxSums}Let $f:\left[a,b\right]\to\R$ be a bounded
function and let $\pi$ be a partition of $\left[a,b\right]$ with
size $k.$ Then the \emph{upper Darboux sum }is given by 
\[
U\left(f,\pi\right):=\sum_{i=1}^{k}\sup_{x\in\left[t_{i-1},t_{i}\right)}f\left(x\right)\left(t_{i}-t_{i-1}\right)
\]
whereas the \emph{lower Darboux sum }is given
\[
L\left(f,\pi\right):=\sum_{i=1}^{k}\inf_{x\in\left[t_{i-1},t_{i}\right)}f\left(x\right)\left(t_{i}-t_{i-1}\right).
\]
\end{defn}

\begin{prop}
\label{prop:darbouxSumInequality}Let $\pi^{\prime}\supseteq\pi$
be a refinement of $\pi$; that is a partition of $\left[a,b\right]$
that contains the coarser partition $\pi$ as a subset. Then
\[
L\left(f,\pi\right)\leq L\left(f,\pi^{\prime}\right)\leq U\left(f,\pi^{\prime}\right)\leq U\left(f,\pi\right).
\]
\end{prop}

\begin{proof}
Note that the second inequality is obvious and so we focus on proving
the first (the third inequality is analagous). First suppose $\pi^{\prime}\setminus\pi=\left\{ t^{*}\right\} $;
that is, the refinement contains only one additional element $t^{*}.$
Then, there is some $1\leq i\leq k$ such that $t_{i-1}<t^{*}<t_{i}.$
Then
\begin{align*}
\inf_{x\in\left[t_{i-1},t_{i}\right)}f\left(x\right)\left(t_{i}-t_{i-1}\right) & \leq\min\left\{ \inf_{x\in\left[t_{i-1},t^{*}\right)}f\left(x\right),\inf_{x\in\left[t^{*},t_{i}\right)}f\left(x\right)\right\} \left(t_{i}-t_{i-1}\right)\\
 & \leq\inf_{x\in\left[t_{i-1},t^{*}\right)}f\left(x\right)\left(t^{*}-t_{i-1}\right)+\inf_{x\in\left[t^{*},t_{i}\right)}f\left(x\right)\left(t_{i}-t^{*}\right)
\end{align*}
and so the first inequality follows since
\begin{align*}
L\left(f,\pi\right) & =\sum_{j\neq i,1\leq j\leq k}\inf_{x\in\left[t_{j-1},t_{j}\right)}f\left(x\right)\left(t_{j}-t_{j-1}\right)+\inf_{x\in\left[t_{i-1},t_{i}\right)}f\left(x\right)\left(t_{i}-t_{i-1}\right)=L\left(f,\pi^{\prime}\right).\\
 & \leq\sum_{j\neq i,1\leq j\leq k}\inf_{x\in\left[t_{j-1},t_{j}\right)}f\left(x\right)\left(t_{j}-t_{j-1}\right)+\inf_{x\in\left[t_{i-1},t^{*}\right)}f\left(x\right)\left(t^{*}-t_{i-1}\right)+\inf_{x\in\left[t^{*},t_{i}\right)}f\left(x\right)\left(t_{i}-t^{*}\right)\\
 & =L\left(f,\pi^{\prime}\right)
\end{align*}
The general case then follows by forming nested sequence $\pi\subset\pi_{1}\subset\ldots\subset\pi_{m}\subset\pi^{\prime}$,
each of which contain one additional element, and then applying the
above result in sequence.
\end{proof}
\begin{defn}
\label{def:riemannIntegrable}A bounded real-valued function $f:\left[a,b\right]\to\R$is
said to be \emph{Riemann integrable }(or just \emph{integrable, }in
context) if
\[
L^{*}\left(f\right):=\sup_{\pi}L\left(f,\pi\right)=\inf_{\pi}U\left(f,\pi\right)=:U^{*}\left(f\right)
\]
where the supremum and infimum are taken over the collection of all
finite partitions $\pi$ of $\left[a,b\right]$. The \emph{Riemann
integral }of an integrable $f$ on $\left[a,b\right]$ is then denoted
\[
\int_{a}^{b}f\left(x\right)dx:=L^{*}\left(f\right)=U^{*}\left(f\right).
\]
The collection of Riemann integrable functions on $\left[a,b\right]$
is denoted $\mathcal{R}\left[a,b\right].$
\end{defn}

\begin{prop}
\label{prop:upperAndLowerSumRelation}Let $f:\left[a,b\right]\to\R$
be a bounded function. Then 
\[
L^{*}\left(f\right)\leq U^{*}\left(f\right)
\]
and
\[
L^{*}\left(f\right)=-U^{*}\left(-f\right).
\]
\end{prop}

\begin{proof}
For the first claim, note that for any partitions $\pi$and $\pi^{\prime}$
, Proposition \ref{prop:darbouxSumInequality} implies that
\[
L\left(f,\pi\right)\leq L\left(f,\pi^{\prime}\cup\pi\right)\leq U\left(f,\pi^{\prime}\cup\pi\right)\leq U\left(f,\pi^{\prime}\right)
\]
and so 
\[
L\left(f,\pi\right)\leq\inf_{\pi^{\prime}}U\left(f,\pi^{\prime}\right)
\]
and similarly
\[
\sup_{\pi}L\left(f,\pi\right)\leq\inf_{\pi^{\prime}}U\left(f,\pi^{\prime}\right).
\]
For the second claim, notice that $\inf_{x\in\left[t_{i-1},t_{i}\right)}f\left(x\right)=-\sup_{x\in\left[t_{i-1},t_{i}\right)}-f\left(x\right)$
and so for any partition $\pi$
\[
L\left(f,\pi\right)=-U\left(-f,\pi\right)
\]
and the result follows by noting that 
\[
L^{*}\left(f\right)=\sup_{\pi}L\left(f,\pi\right)=\sup_{\pi}-U\left(-f,\pi\right)=-\inf_{\pi}U\left(-f,\pi\right)=-U^{*}\left(-f\right).
\]
\end{proof}
\begin{thm}
\label{thm:riemannIntegrability}The following are equivalent for
any bounded function $f:\left[a,b\right]\to\R$

\begin{enumerate}[label=(\roman*),leftmargin=.1\linewidth,rightmargin=.4\linewidth]
	\item $ f \in \mathcal{R}\left[a,b\right]$
	\item For every $\epsilon > 0$ there exists some partition $\pi$ such that 
		\[
			U\left(f,\pi\right) - L\left(f,\pi\right) < \epsilon.
		\]
	\item For every $\epsilon >0$ there exists some $\delta > 0$ such that for any partition $\pi$ with $\mesh \left(\pi\right) < \delta$
		\[
			U\left(f,\pi\right) - L\left(f,\pi\right) < \epsilon
		\]
	\item The limit 
	\[
		I := \lim_{\mesh\left(\pi\right) \to 0}\sum_{i: t_i \in \pi } f\left( c_i \right)\left(t_i - t_{i-1}\right)
	\]
	exists for any choice of intermediate values $t_{i-1} \leq c_i \leq t_i $. Further, this limit is equal to $I = L^*(f)=U^*(f)$.
\end{enumerate}
\end{thm}

\begin{proof}
We first prove that (i)$\implies$(ii). Suppose that $f\in\mathcal{R}\left[a,b\right]$,
and so $L^{*}\left(f\right)=U^{*}\left(f\right).$ Fix $\epsilon>0$
and observe that by the definition of supremum and infimum, there
exist partitions $\pi$ and $\pi^{\prime}$ such that $L^{*}\left(f\right)-L\left(f,\pi\right)<\frac{\epsilon}{2}$
and $U\left(f,\pi^{\prime}\right)-L^{*}\left(f\right)<\frac{\epsilon}{2}$.
Then clearly, for the common refinement $\pi\cup\pi^{\prime}$
\[
\lvert L\left(f,\pi\cup\pi^{\prime}\right)-U\left(f,\pi\cup\pi^{\prime}\right)\rvert\leq\lvert L\left(f,\pi\cup\pi^{\prime}\right)-L^{*}\left(f\right)\rvert+\lvert L^{*}\left(f\right)-U\left(f,\pi\cup\pi^{\prime}\right)\rvert<\epsilon.
\]

Next, we prove (ii) $\implies$(iii). Fix $\epsilon>0$ and note that
by (ii) there exists some partition $\pi:=\left\{ a=t_{0}<t_{1}<\ldots<t_{k}=b\right\} $such
that 
\begin{equation}
U\left(f,\pi\right)-L\left(f,\pi\right)=\sum_{i=1}^{k}\left(\sup_{x\in\left[t_{i-1},t_{i}\right)}f\left(x\right)-\inf_{x\in\left[t_{i-1},t_{i}\right)}f\left(x\right)\right)\left(t_{i}-t_{i-1}\right)<\frac{\epsilon}{2}.\label{eq:assumption(ii)}
\end{equation}
Now $\pi^{\prime}:=\left\{ a=t_{0}^{\prime}<t_{1}^{\prime}<\ldots<t_{l}^{\prime}=b\right\} $
be a partition such that $\mesh\left(\pi\right)<\delta:=\frac{\epsilon}{4k\lVert f\rVert_{\infty}}$
where $\lVert f\rVert_{\infty}=\sup_{x\in\left[a,b\right]}\lvert f\left(x\right)\rvert$
. Then for each $t_{i}\in\pi$, there can be at most one interval
$\left[t_{j-1}^{\prime},t_{j}^{\prime}\right)$ with $1\leq j\leq l$
that contains $t_{i}$ since these intervals form a partition of $\left[a,b\right].$
Thus the set $S:=\left\{ j\in\left\{ 1,\ldots,l\right\} \mid\text{\ensuremath{\exists i\in\left\{ 0,1,\ldots,k\right\} } }\text{s.t }t_{i}\in\left[t_{j-1}^{\prime},t_{j}^{\prime}\right)\right\} $
has size at most $k.$ Now consider the difference of Darboux sums
\begin{align*}
U\left(f,\pi^{\prime}\right)-L\left(f,\pi^{\prime}\right) & =\sum_{j=1}^{l}\left(\sup_{x\in\left[t_{j-1}^{\prime},t_{j}^{\prime}\right)}f\left(x\right)-\inf_{x\in\left[t_{j-1}^{\prime},t_{i}^{\prime}\right)}f\left(x\right)\right)\left(t_{j}^{\prime}-t_{j-1}^{\prime}\right)\\
 & =\sum_{j\in S}\left(\sup_{x\in\left[t_{j-1}^{\prime},t_{j}^{\prime}\right)}f\left(x\right)-\inf_{x\in\left[t_{j-1}^{\prime},t_{i}^{\prime}\right)}f\left(x\right)\right)\left(t_{j}^{\prime}-t_{j-1}^{\prime}\right)\\
 & \ \ \ +\sum_{j\in S^{C}}\left(\sup_{x\in\left[t_{j-1}^{\prime},t_{j}^{\prime}\right)}f\left(x\right)-\inf_{x\in\left[t_{j-1}^{\prime},t_{i}^{\prime}\right)}f\left(x\right)\right)\left(t_{j}^{\prime}-t_{j-1}^{\prime}\right).
\end{align*}
We control each of these sums separately. For the first sum, note
that there at most $k$ terms in the sum, where $\left(\sup_{x\in\left[t_{j-1}^{\prime},t_{j}^{\prime}\right)}f\left(x\right)-\inf_{x\in\left[t_{j-1}^{\prime},t_{i}^{\prime}\right)}f\left(x\right)\right)\leq2\lVert f\rVert_{\infty}$
and $\left(t_{j}^{\prime}-t_{j-1}^{\prime}\right)<\delta$ for every
$j\in S.$ Thus the first term is bounded above by $\frac{\epsilon}{2}.$
Next, note that for the second sum, by definition for every $j\in S^{C},$
there exists some $i_{j}\in\left\{ 1,\ldots,k\right\} $ such that
$\left[t_{j-1}^{\prime},t_{j}^{\prime}\right)\subset\left[t_{i_{j}-1},t_{i_{j}}\right)$
and that
\[
\sum_{j\in S^{C}}\left(\sup_{x\in\left[t_{j-1}^{\prime},t_{j}^{\prime}\right)}f\left(x\right)-\inf_{x\in\left[t_{j-1}^{\prime},t_{i}^{\prime}\right)}f\left(x\right)\right)\left(t_{j}^{\prime}-t_{j-1}^{\prime}\right)\leq\sum_{j\in S^{C}}\left(\sup_{x\in\left[t_{i_{j}-1},t_{i_{j}}\right)}f\left(x\right)-\inf_{x\in\left[t_{i_{j}-1},t_{i_{j}}\right)}f\left(x\right)\right)\left(t_{i_{j}}-t_{i_{j-1}}\right)<\frac{\epsilon}{2}
\]
where the final inequality is due to (\ref{eq:assumption(ii)}). This
means 
\[
\lvert U\left(p,\pi^{\prime}\right)-L\left(f,\pi^{\prime}\right)\rvert<\epsilon
\]
 which completes this part of the proof.

Next, assume that (iii) holds and let $\epsilon>0$ . Observe that
there exists some $\delta>0$ such that for any partition $\pi:=\left\{ a=t_{0}<t_{1}<\ldots t_{k\left(\pi\right)}=b\right\} $
with mesh less than $\delta$, 
\[
0\leq\sum_{i=1}^{k\left(\pi\right)}f\left(c_{i}\right)\left(t_{i}-t_{i-1}\right)-L\left(f,\pi\right)\leq U\left(f,\pi\right)-L\left(f,\pi\right)<\frac{\epsilon}{2}
\]
where $c_{i}\in\left[t_{i-1},t_{i}\right).$ Now note that since $L\left(f,\pi_{n}\right)\leq L^{*}\left(f\right)\leq U^{*}\left(f\right)\leq U\left(f,\pi_{n}\right)$,
we have that $L^{*}\left(f\right)-L\left(f,\pi_{n}\right)<\frac{\epsilon}{2}$
(which shows that $L\left(f,\pi_{n}\right)\to L^{*}\left(f\right)$).
Then
\[
\left\lvert \sum_{i=1}^{k_{n}}f\left(c_{i}^{n}\right)\left(t_{i}^{n}-t_{i-1}^{n}\right)-L^{*}\left(f\right)\right\rvert \leq\left\lvert \sum_{i=1}^{k_{n}}f\left(c_{i}^{n}\right)\left(t_{i}^{n}-t_{i-1}^{n}\right)-L\left(f,\pi_{n}\right)\right\rvert +\left\lvert L\left(f,\pi\right)-L^{*}\left(f\right)\right\rvert <\epsilon.
\]
Similarly, we know tht 
\[
U\left(f,\pi\right)-\sum_{i=1}^{k\left(\pi\right)}f\left(c_{i}\right)\left(t_{i}-t_{i-1}\right)\leq U\left(f,\pi\right)-L\left(f,\pi\right)<\frac{\epsilon}{2}
\]
and so $U\left(f,\pi\right)-U^{*}\left(f\right)<\frac{\epsilon}{2}$
and
\[
\left\lvert \sum_{i=1}^{k_{n}}f\left(c_{i}^{n}\right)\left(t_{i}^{n}-t_{i-1}^{n}\right)-U^{*}\left(f\right)\right\rvert \leq\left\lvert \sum_{i=1}^{k_{n}}f\left(c_{i}^{n}\right)\left(t_{i}^{n}-t_{i-1}^{n}\right)-U\left(f,\pi_{n}\right)\right\rvert +\left\lvert U\left(f,\pi\right)-U^{*}\left(f\right)\right\rvert <\epsilon
\]
and so the uniqueness of limits shows (iv).

Finally, Suppose that (iv) hods i.e. that $I=\lim_{\mesh\left(\pi\right)\to0}\sum_{i=1}^{k\left(\pi\right)}f\left(c_{i}^{\pi}\right)\left(t_{i}^{\pi}-t_{i-1}^{\pi}\right)$
exists and is independent of any choice $t_{i-1}^{\pi}\leq c_{i}^{\pi}<t_{i}^{\pi}$.
Fix $\epsilon>0$ and notice that by assumption there exists some
$\delta>0$ such that for any partition $\pi$ with $\mesh\left(\pi\right)<\delta$
\[
\left\lvert I-\sum_{i=1}^{k\left(\pi\right)}f\left(c_{i}^{\pi}\right)\left(t_{i}^{\pi}-t_{i-1}^{\pi}\right)\right\rvert <\frac{\epsilon}{3}.
\]
Crucially, the choice of intermediate values $c_{i}^{\pi}$\emph{
does not }influence our choice of $\delta.$ Then, by Proposition
\ref{prop:darbouxSumInequality} and the definition of a supremum,
we can actually choose a partition $\pi_{0}$ fine enough, such that
its mesh is less than $\delta$ and
\[
\lvert L\left(f,\pi_{0}\right)-L^{*}\left(f\right)\rvert<\frac{\epsilon}{3}.
\]
Finally, notice that by the definition of infimum, we can choose the
intermediates $c_{i}^{\pi_{0}}$ such that that $f\left(c_{i}^{\pi_{0}}\right)-\inf_{x\in\left[t_{i-1}^{\pi_{0}},t_{i}^{\pi_{0}}\right)}f\left(x\right)<\frac{\epsilon}{3k\left(\pi_{0}\right)\mesh\left(\pi_{0}\right)}$
and so 
\begin{align*}
\left\lvert I-L^{*}\left(f\right)\right\rvert  & \leq\left\lvert I-\sum_{i=1}^{k\left(\pi_{0}\right)}f\left(c_{i}^{\pi_{0}}\right)\left(t_{i}^{\pi_{0}}-t_{i-1}^{\pi_{0}}\right)\right\rvert +\left\lvert \sum_{i=1}^{k\left(\pi_{0}\right)}f\left(c_{i}^{\pi_{0}}\right)\left(t_{i}^{\pi_{0}}-t_{i-1}^{\pi_{0}}\right)-L\left(f,\pi_{0}\right)\right\rvert +\left\lvert L\left(f,\pi_{0}\right)-L^{*}\left(f\right)\right\rvert \\
< & \epsilon.
\end{align*}
Since $\epsilon$is arbitrary, we have $I=L^{*}\left(f\right).$ A
similar argument shows that $I=U^{*}\left(f\right)$ which completes
the proof.
\end{proof}
Note that the existence of \emph{particular }sums of the form $\sum f\left(c_{i}^{\pi}\right)\left(t_{i}^{\pi}-t_{i-1}^{\pi}\right)$
that converge to a limit when the mesh goes to zero does not imply
the integrability of $f.$ We need the convergence to hold for \emph{any
}choice of intermediate values and \emph{any }set of partitions with
mesh going to zero. This is highlighted in the next example, which
is also the canonical example of a function that is not Riemann integrable.
\begin{example}
\label{exa:nonRiemannIntegrableFunction}Let $f:\left[0,1\right]\to\R$
be given by $f\left(x\right)=\indicate\left\{ x\in\left[0,1\right]\cap\mathds{Q}\right\} .$Then
$L\left(f,\pi\right)=0$ for any partition $\pi$given the density
of the rationals and the irrationals in $\left[0,1\right].$ Therefore
$L^{*}\left(f\right)=0.$ A similar argument shows that $U^{*}\left(f\right)=1$
and so $f$ is not integrable. On the other hand, the Riemann sums
given by 
\[
\sum_{i=1}^{k}f\left(c_{i}\right)\left(\frac{i}{k}-\frac{i-1}{k}\right)=\frac{1}{k}\sum_{i=1}^{k}f\left(c_{i}\right)
\]
can be made to converge to 0, for instance, by choosing all intermediate
points as irrational (this is possible to do again by the density
of irrationals). Nevertheless the integral does not exist.
\end{example}

\begin{thm}
\label{thm:lebesgueRiemannEqual}Suppose $f:\left[a,b\right]\to\R$
is Riemann integrable then $f\in L^{1}\left(\left[a,b\right],\borel\left(\left[a,b\right]\right),\lambda\right)$
and 
\[
\lebInt{\lambda}f=L^{*}\left(f\right)=U^{*}\left(f\right).
\]
\end{thm}

\begin{proof}
Let $\left\{ \pi_{i}\right\} _{i=1}^{\infty}$ be a sequences of partitions
such that $\pi_{i}\subset\pi_{i+1}$ and $\mesh\left(\pi_{i}\right)\to0$.
Notice we can define the Darboux functions -- which are step functions
-- as
\[
l\left(f,\pi_{i}\right)\left(x\right):=\sum_{j=1}^{k\left(\pi_{i}\right)}\inf_{c\in\left[t_{j-1}^{\pi_{i}},t_{j}^{\pi_{i}}\right)}f\left(c\right)\indicate\left\{ x\in\left[t_{j-1}^{\pi_{i}},t_{j}^{\pi_{i}}\right)\right\} 
\]
and
\[
u\left(f,\pi_{i}\right)\left(x\right):=\sum_{j=1}^{k\left(\pi_{i}\right)}\sup_{c\in\left[t_{j-1}^{\pi_{i}},t_{j}^{\pi_{i}}\right)}f\left(c\right)\indicate\left\{ x\in\left[t_{j-1}^{\pi_{i}},t_{j}^{\pi_{i}}\right)\right\} .
\]
For brevity we will write $l_{i}:=l\left(f,\pi_{i}\right)$ and $u_{i}:=u\left(f,\pi_{i}\right)$.
Note then that for all $i\in\N$
\[
l_{i}\leq l_{i+1}\leq\sup_{i}l_{i}\leq f\leq\inf_{i}u_{i}\leq u_{i+1}\leq u_{i}
\]
pointwise by our construction and the nested nature of the partitions
$\pi_{i}$ (see Proposition \ref{prop:darbouxSumInequality} for the
argument). Notice further that step functions are simple functions
on a finite measure space and hence integrable. Moreover, since $f$
is bounded (Riemann integrable functions are bounded by definition),
$\max\left\{ \lvert u_{i}\rvert,\lvert l_{i}\rvert\right\} \leq\lVert f\rVert_{\infty}$
and so by the dominated convergence theorem
\[
\lebInt{\lambda}{l_{i}}\to\lebInt{\lambda}{\sup l_{i}}
\]
and 
\[
\lebInt{\lambda}{u_{i}}\to\lebInt{\lambda}{\inf u_{i}}.
\]
But notice that $\lebInt{\lambda}{l_{i}}=L\left(f,\pi_{i}\right)$
and $\lebInt{\lambda}{u_{i}}=U\left(f,\pi_{i}\right)$ and so by Riemann
integrability
\[
\lebInt{\lambda}{\sup l_{i}}=L^{*}\left(f\right)=U^{*}\left(f\right)=\lebInt{\lambda}{\inf u_{i}}.
\]
Note further that $\inf_{i}u_{i}-\sup_{i}l_{i}\geq0$ and so $\lebInt{\lambda}{\inf_{i}u_{i}-\sup_{i}l_{i}}=0$
implies that $\inf_{i}u_{i}\stackrel{\text{a.e}}{=}\sup_{i}l_{i}$
by Proposition \ref{prop:intEqualFuncEqual}. Of couse, then $f\stackrel{\text{a.e}}{=}\inf_{i}u_{i}$
and so $f$ is measurable (and Lebesgue integrable since its bounded
in a finite measure space). Moreover, Propostion \ref{prop:funcEqualityAlmostEverywhere}
implies
\[
\lebInt{\lambda}f=L^{*}\left(f\right)=U^{*}\left(f\right)
\]
which completes the proof.
\end{proof}
\begin{cor}
\label{cor:charRiemannIntegrable}Let $f:\left[a,b\right]\to\R$ be
a bounded functon. Then $f$ is Riemann integrable if and only if
for any sequence of nested partitions $\left\{ \pi_{i}\right\} _{i=1}^{\infty}$
with $\pi_{i}\subseteq\pi_{i+1}$ and $\mesh\left(\pi_{i}\right)\to0$
\[
\sup_{i}l\left(f,\pi_{i}\right)\stackrel{\text{a.e}}{=}\inf_{i}u\left(f,\pi_{i}\right).
\]
\end{cor}

\begin{proof}
Note that the ``only if'' direction follows from Proposition \ref{thm:lebesgueRiemannEqual}.
To see the converse, fix $\epsilon>0$ and let $N:=\left\{ x\in\left[a,b\right]\mid\sup_{i}l_{i}\left(x\right)\neq\inf_{i}u_{i}\left(x\right)\right\} $.
Note that for any $x\in[a,b]\setminus N$ we have pointwise convergence
\[
\lim_{i\to\infty}l_{i}\left(x\right)=\sup_{i}l_{i}\left(x\right)
\]
and
\[
\lim_{i\to\infty}u_{i}\left(x\right)=\inf_{i}u_{i}\left(x\right).
\]
Then 
\[
\lim_{i\to\infty}\underbrace{u_{i}\left(x\right)-l_{i}\left(x\right)}_{=:g_{i}\left(x\right)}=0
\]
for $x\in\left[a,b\right]\setminus N$. But notice that $\lvert g_{i}\rvert\leq2\lVert f\rVert_{\infty}$
and so by (generalized) dominated convergence
\[
U^{*}\left(f\right)=L^{*}\left(f\right)
\]
which completes the proof.
\end{proof}
Now we are finally ready to characterize Riemann integrability in
terms of continuity.
\begin{thm}
\label{thm:riemannIntegrableAEContinuous}Let $f:\left[a,b\right]\to\R$
be a bounded function. Then $f$ is Riemann integrable if and only
if $f$ is continuous almost everywhere.
\end{thm}

\begin{proof}
Let $\left\{ \pi_{i}\right\} _{i=1}^{\infty},$ $l_{i},$ and $u_{i}$
be defined as before and let $c\in\left[a,b\right]$ be arbitrary.
Fix $\epsilon>0$ and notice that by the definition of a partition,
for every $i\in\N$ there exists some $1\leq j_{i}\leq k\left(\pi_{i}\right)$
such that $c\in\left[t_{j_{i}-1}^{\pi_{i}},t_{j_{i}}^{\pi_{i}}\right)$.
Further, note that there is some $i_{c}\in\N$ such that for all $i\geq i_{c}$
\[
u_{i}\left(c\right)-\inf_{i}u_{i}\left(c\right)<\frac{\epsilon}{2}
\]
and
\[
\sup_{i}l_{i}\left(c\right)-l_{i}\left(c\right)<\frac{\epsilon}{2}.
\]
Adding the two inequalities and re-arranging, we have that 
\[
u_{i}\left(c\right)-l_{i}\left(c\right)<\epsilon+\inf_{i}u_{i}\left(c\right)-\sup_{i}l_{i}\left(c\right).
\]
For any $x$ such that $\lvert x-c\rvert<\delta:=\frac{\min\left\{ t_{j_{i_{c}}}^{\pi_{i_{c}}}-c,c-t_{j_{i_{c}}-1}^{\pi_{i_{c}}}\right\} }{2}$we
have that $x,c\in\left[t_{j_{i_{c}}-1}^{\pi_{i_{c}}},t_{j_{i_{c}}}^{\pi_{i_{c}}}\right)$
and so
\[
\lvert f\left(x\right)-f\left(c\right)\rvert\leq u_{i_{c}}\left(c\right)-l_{i_{c}}\left(c\right)<\epsilon+\inf_{i}u_{i}\left(c\right)-\sup_{i}l_{i}\left(c\right).
\]
Thus $f$ is continuous at $c$ if $c\notin N$ which is a null set
if and only if $f$ is Riemann integrable. So we have proved that
a Riemann integrable function is continuous almost everywhere.

\end{proof}

\subsection{The improper Riemann integral}

\subsection{The Jordan content}
